\clearpage
\subsection{Simplest possible 4-byte XOR encryption}

If a longer pattern was used for XOR-encryption, for example a 4 byte pattern, it's easy to spot as well.

For example, here is the beginning of the kernel32.dll file (32-bit version from Windows Server 2008):

\begin{figure}[H]
\centering
\myincludegraphics{ff/XOR/4byte/original1.png}
\caption{Original file}
\end{figure}

\clearpage

Here it is \q{encrypted} with a 4-byte key:

\begin{figure}[H]
\centering
\myincludegraphics{ff/XOR/4byte/encrypted1.png}
\caption{\q{Encrypted} file}
\end{figure}

It's very easy to spot the recurring 4 symbols.

Indeed, the header of a PE-file has a lot of long zero areas, which are the reason for the key to become visible.

\clearpage

Here is the beginning of a PE-header in hexadecimal form:

\begin{figure}[H]
\centering
\myincludegraphics{ff/XOR/4byte/original2.png}
\caption{PE-header}
\end{figure}

\clearpage

Here it is \q{encrypted}:

\begin{figure}[H]
\centering
\myincludegraphics{ff/XOR/4byte/encrypted2.png}
\caption{\q{Encrypted} PE-header}
\end{figure}

It's easy to spot that the key is the following 4 bytes: \TT{8C 61 D2 63}.

With this information, it's easy to decrypt the whole file.

So it is important to keep in mind these properties of PE-files:
1) PE-header has many zero-filled areas;
2) all PE-sections are padded with zeros at a page boundary (4096 bytes),
so long zero areas are usually present after each section.

Some other file formats may contain long zero areas.

It's typical for files used by scientific and engineering software.

For those who want to inspect these files on their own, they are downloadable here: \url{http://go.yurichev.com/17352}.

\subsubsection{\Exercise}

\begin{itemize}
	\item \url{http://challenges.re/50}
\end{itemize}

