\section{\oracle: .SYM-files}
\myindex{\oracle}
\label{Oracle_SYM_files_example}

When an \oracle process experiences some kind of crash, it writes a lot of information into log files,
including stack trace, like this:

\begin{lstlisting}
----- Call Stack Trace -----
calling              call     entry                argument values in hex      
location             type     point                (? means dubious value)     
-------------------- -------- -------------------- ----------------------------
_kqvrow()                     00000000             
_opifch2()+2729      CALLptr  00000000             23D4B914 E47F264 1F19AE2
                                                   EB1C8A8 1
_kpoal8()+2832       CALLrel  _opifch2()           89 5 EB1CC74
_opiodr()+1248       CALLreg  00000000             5E 1C EB1F0A0
_ttcpip()+1051       CALLreg  00000000             5E 1C EB1F0A0 0
_opitsk()+1404       CALL???  00000000             C96C040 5E EB1F0A0 0 EB1ED30
                                                   EB1F1CC 53E52E 0 EB1F1F8
_opiino()+980        CALLrel  _opitsk()            0 0
_opiodr()+1248       CALLreg  00000000             3C 4 EB1FBF4
_opidrv()+1201       CALLrel  _opiodr()            3C 4 EB1FBF4 0
_sou2o()+55          CALLrel  _opidrv()            3C 4 EB1FBF4
_opimai_real()+124   CALLrel  _sou2o()             EB1FC04 3C 4 EB1FBF4
_opimai()+125        CALLrel  _opimai_real()       2 EB1FC2C
_OracleThreadStart@  CALLrel  _opimai()            2 EB1FF6C 7C88A7F4 EB1FC34 0
4()+830                                            EB1FD04
77E6481C             CALLreg  00000000             E41FF9C 0 0 E41FF9C 0 EB1FFC4
00000000             CALL???  00000000             
\end{lstlisting}

But of course, \oracle's executables must have some kind of debug information or map files with symbol
information included or something like that.

Windows NT \oracle has symbol information in files with .SYM extension, but the format is proprietary.
(Plain text files are good, but needs additional parsing, hence offer slower access.)

Let's see if we can understand its format.

We will pick the shortest \TT{orawtc8.sym} file that comes with the \TT{orawtc8.dll} file in Oracle 8.1.7
\footnote{We can chose an ancient \oracle version intentionally due to the smaller size of its modules}.

\clearpage
Here is the file opened in Hiew:

\begin{figure}[H]
\centering
\myincludegraphics{ff/Oracle_SYM/whole1.png}
\caption{The whole file in Hiew}
\label{fig:oracle_SYM_whole1}
\end{figure}

By comparing the file with other .SYM files, we can quickly see that \TT{OSYM} is always header (and footer),
so this is maybe the file's signature.

We also see that basically, the file format is: OSYM + some binary data + zero delimited text strings + OSYM.
The strings are, obviously, function and global variable names.

\clearpage
We will mark the OSYM signatures and strings here: 

\begin{figure}[H]
\centering
\myincludegraphics{ff/Oracle_SYM/whole2.png}
\caption{OSYM signature and text strings}
\label{fig:oracle_SYM_whole2}
\end{figure}

Well, let's see. 
In Hiew, we will mark the whole strings block (except the trailing OSYM signatures) and put it into a separate file.
Then we run UNIX \IT{strings} and \IT{wc} utilities to count the text strings:

\begin{lstlisting}
strings strings_block | wc -l
66
\end{lstlisting}

So there are 66 text strings.
Please note that number.

We can say, in general, as a rule, the number of \IT{anything} is often stored separately in binary files.

It's indeed so, we can find the 66 value (0x42) at the file's start, right after the OSYM signature:

\lstinputlisting{ff/Oracle_SYM/dump1.txt}

Of course, 0x42 here is not a byte, but most likely a 32-bit value packed as little-endian, hence we see
0x42 and then at least 3 zero bytes.

Why do we believe it's 32-bit?
Because, \oracle's symbol 
files may be pretty big.

The oracle.sym file for the main oracle.exe (version 10.2.0.4) executable contains \TT{0x3A38E} (238478) symbols.
A 16-bit value isn't enough here.

We can check other .SYM files like this and it proves our guess: the value after the 32-bit OSYM signature always
reflects the number of text strings in the file.

It's a general feature of almost all binary files: a header with a signature plus some other information 
about the file.

Now let's investigate closer what this binary block is.

Using Hiew again, we put the block starting at address 8 (i.e., after the 32-bit \IT{count} value) 
ending at the strings block, into a separate binary file.

\clearpage
Let's see the binary block in Hiew:

\begin{figure}[H]
\centering
\myincludegraphics{ff/Oracle_SYM/binary1.png}
\caption{Binary block}
\label{fig:oracle_SYM_binary1}
\end{figure}

There is a clear pattern in it. 

\clearpage
We will add red lines to divide the block: 

\begin{figure}[H]
\centering
\myincludegraphics{ff/Oracle_SYM/binary2.png}
\caption{Binary block patterns}
\label{fig:oracle_SYM_binary2}
\end{figure}

Hiew, like almost any other hexadecimal editor, shows 16 bytes per line.
So the pattern is clearly visible: 
there are 4 32-bit values per line.

The pattern is visually visible because some values here (till address \TT{0x104}) 
are always in \TT{0x1000xxxx} form, 
started with 0x10 and zero bytes.

Other values (starting at \TT{0x108}) are in \TT{0x0000xxxx} form, so always started with two zero bytes.

Let's dump the block as an array of 32-bit values:

\lstinputlisting[caption=first column is address]{ff/Oracle_SYM/dump2.txt}

There are 132 values, that's 66*2.
Probably, there are two 32-bit values for each symbol, but maybe there are two arrays? 
Let's see.

Values starting with \TT{0x1000} may be addresses.

This is a .SYM file for a DLL after all, and the default base address of
win32 DLLs is \TT{0x10000000}, and the code usually starts at \TT{0x10001000}.

When we open the orawtc8.dll file in \IDA, the base address is different, but nevertheless, the first function is:

\begin{lstlisting}[style=customasmx86]
.text:60351000 sub_60351000    proc near
.text:60351000
.text:60351000 arg_0    = dword ptr  8
.text:60351000 arg_4    = dword ptr  0Ch
.text:60351000 arg_8    = dword ptr  10h
.text:60351000
.text:60351000          push    ebp
.text:60351001          mov     ebp, esp
.text:60351003          mov     eax, dword_60353014
.text:60351008          cmp     eax, 0FFFFFFFFh
.text:6035100B          jnz     short loc_6035104F
.text:6035100D          mov     ecx, hModule
.text:60351013          xor     eax, eax
.text:60351015          cmp     ecx, 0FFFFFFFFh
.text:60351018          mov     dword_60353014, eax
.text:6035101D          jnz     short loc_60351031
.text:6035101F          call    sub_603510F0
.text:60351024          mov     ecx, eax
.text:60351026          mov     eax, dword_60353014
.text:6035102B          mov     hModule, ecx
.text:60351031
.text:60351031 loc_60351031:    ; CODE XREF: sub_60351000+1D
.text:60351031          test    ecx, ecx
.text:60351033          jbe     short loc_6035104F
.text:60351035          push    offset ProcName ; "ax_reg"
.text:6035103A          push    ecx             ; hModule
.text:6035103B          call    ds:GetProcAddress
...
\end{lstlisting}

Wow, \q{ax\_reg} string sounds familiar. 

It's indeed the first string in the strings block!
So the name of this function seems to be \q{ax\_reg}.

The second function is:

\begin{lstlisting}[style=customasmx86]
.text:60351080 sub_60351080    proc near
.text:60351080
.text:60351080 arg_0    = dword ptr  8
.text:60351080 arg_4    = dword ptr  0Ch
.text:60351080
.text:60351080          push    ebp
.text:60351081          mov     ebp, esp
.text:60351083          mov     eax, dword_60353018
.text:60351088          cmp     eax, 0FFFFFFFFh
.text:6035108B          jnz     short loc_603510CF
.text:6035108D          mov     ecx, hModule
.text:60351093          xor     eax, eax
.text:60351095          cmp     ecx, 0FFFFFFFFh
.text:60351098          mov     dword_60353018, eax
.text:6035109D          jnz     short loc_603510B1
.text:6035109F          call    sub_603510F0
.text:603510A4          mov     ecx, eax
.text:603510A6          mov     eax, dword_60353018
.text:603510AB          mov     hModule, ecx
.text:603510B1
.text:603510B1 loc_603510B1:    ; CODE XREF: sub_60351080+1D
.text:603510B1          test    ecx, ecx
.text:603510B3          jbe     short loc_603510CF
.text:603510B5          push    offset aAx_unreg ; "ax_unreg"
.text:603510BA          push    ecx             ; hModule
.text:603510BB          call    ds:GetProcAddress
...
\end{lstlisting}

The \q{ax\_unreg} string is also the second string in the strings block!

The starting address of the second function is \TT{0x60351080}, and the second value in the binary 
block is \TT{10001080}.
So this is the address, 
but for a DLL with the default base address.

We can quickly check and be sure that the first 66 values in the array (i.e., the first half of the array) 
are just function addresses in the DLL, including some labels, \etc{}.
Well, what's the other part of array then? 
The other 66 values that start with \TT{0x0000}? 
These seem to be in range \TT{[0...0x3F8]}. 
And they do not look like bitfields: 
the series of numbers is increasing.

The last hexadecimal digit seems to be random, so, it's unlikely the address of something 
(it would be divisible by 4 or maybe 8 or 0x10 otherwise).

Let's ask ourselves: what else \oracle's developers would save here, in this file?

Quick wild guess: it could be the address of the text string (function name).

It can be quickly checked, and yes, each number is just the position of the first character in the strings block.

This is it! All done.

\myindex{IDA}
We will write an utility to convert these .SYM files into \IDA script, 
so we can load the .idc script and it sets the function names:

\lstinputlisting[style=customc]{ff/Oracle_SYM/unpacker.c}

Here is an example of its work:

\begin{lstlisting}[style=customc]
#include <idc.idc>

static main() {
	MakeName(0x60351000, "_ax_reg");
	MakeName(0x60351080, "_ax_unreg");
	MakeName(0x603510F0, "_loaddll");
	MakeName(0x60351150, "_wtcsrin0");
	MakeName(0x60351160, "_wtcsrin");
	MakeName(0x603511C0, "_wtcsrfre");
	MakeName(0x603511D0, "_wtclkm");
	MakeName(0x60351370, "_wtcstu");
...
}
\end{lstlisting}

The example files were used in this example are here: 
\href{http://go.yurichev.com/17216}{beginners.re}.

\clearpage
Oh, let's also try \oracle for win64.
There has to be 64-bit addresses instead, right?

The 8-byte pattern is visible even easier here:

\begin{figure}[H]
\centering
\myincludegraphics{ff/Oracle_SYM/whole64.png}
\caption{.SYM-file example from \oracle for win64}
\label{fig:oracle_SYM_whole64}
\end{figure}

So yes, all tables now have 64-bit elements, even string offsets!

The signature is now \TT{OSYMAM64}, to distinguish the target platform, apparently.

This is it!

Here is also library which has functions to access \oracle .SYM-files:
\href{http://go.yurichev.com/17007}{GitHub}.
