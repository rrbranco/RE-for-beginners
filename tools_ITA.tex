\chapter{Strumenti}

\epigraph{Ora che Dennis Yurichev ha reso questo libro free (libre), è un
contributo a tutto il mondo della libera informazione ed educazione.
Comunque, per il bene della nostra libertà, abbiamo bisogno di strumenti free (libre) per il reverse
engineering in modo da rimpiazzare quelli proprietari descritti in questo libro.}{Richard M. Stallman}

\section{Analisi di Binari}

Strumenti da utilizzare senza eseguire nessun processo:

\myindex{Hiew}
\myindex{UNIX!strings}
\myindex{UNIX!xxd}
\myindex{UNIX!od}

\begin{itemize}
\item
(Free, open-source) \IT{ent}\footnote{\url{http://www.fourmilab.ch/random/}}: strumento per analizzare l'entropia.
Leggi di più riguardo l'entropia: \myref{entropy}.

\item
\label{Hiew}
\IT{Hiew}\footnote{\href{http://go.yurichev.com/17035}{hiew.ru}}:
per piccole modifiche del codice dei file binari.

\item (Free, open-source) \IT{xxd} and \IT{od}: standard UNIX utility per effettuare il dump nel formato desiderato. 

\item (Free, open-source) \IT{strings}: strumento *NIX per cercare stringhe ASCII all'interno di file binari, eseguibili compresi.
Sysinternals ha un'alternativa\footnote{\url{https://technet.microsoft.com/en-us/sysinternals/strings}}
che supporta la stringhe con caratteri di tipo "wide" (UTF-16, ampiamente utilizzati in Windows).

\item (Free, open-source) \IT{Binwalk}\footnote{\url{http://binwalk.org/}}: analisi di immagini firmware.

\item
\myindex{binary grep}
(Free, open-source) \IT{binary grep}:
piccola utility per cercare una sequenza di byte in molti file,
incluso file non eseguibili: \BGREPURL.
% TBT
\end{itemize}

\subsection{Disassemblers}

\myindex{IDA}
\myindex{Binary Ninja}
\myindex{BinNavi}
\myindex{objdump}

\begin{itemize}
\item \IT{IDA}. Una versione più vecchia è liberamente disponibile per lo scaricamento
\footnote{\href{http://go.yurichev.com/17031}{hex-rays.com/products/ida/support/download\_freeware.shtml}}.
\ShortHotKeyCheatsheet: \myref{sec:IDA_cheatsheet}

\item \IT{Binary Ninja}\footnote{\url{http://binary.ninja/}}

\item (Free, open-source) \IT{zynamics BinNavi}\footnote{\url{https://www.zynamics.com/binnavi.html}}

\item (Free, open-source) \IT{objdump}: semplice utility command-line per effettuare il dump e il disassembling.

\item (Free, open-ssource) \IT{readelf}\footnote{\url{https://sourceware.org/binutils/docs/binutils/readelf.html}}:
dump delle informazioni dei file ELF.
\end{itemize}

\subsection{Decompilers}

C'è solo un decompiler conosciuto, pubblicamente disponibile e di elevata qualità per decompilare in C: \IT{Hex-Rays}:
\href{http://go.yurichev.com/17033}{hex-rays.com/products/decompiler/}

\subsection{Comparazione Patch/diffing}

Potresti voler utilizzare questi strumenti quando devi comparare la versione originale di un eseguibile con quella patchata,
in modo da trovare cos'è stato patchato e perchè.

\begin{itemize}
\item (Free) \IT{zynamics BinDiff}\footnote{\url{https://www.zynamics.com/software.html}}

\item (Free, open-source) \IT{Diaphora}\footnote{\url{https://github.com/joxeankoret/diaphora}}
\end{itemize}

\section{Live analysis}

Strumenti da utilizzare per effettuare un'analisi live del sistema o di un processo in esecuzione.

\subsection{Debuggers}

\myindex{\olly}
\myindex{Radare}
\myindex{GDB}
\myindex{tracer}
\myindex{LLDB}
\myindex{WinDbg}

\begin{itemize}
\item (Free) \IT{OllyDbg}.
Popolare win32 debugger\footnote{\href{http://go.yurichev.com/17032}{ollydbg.de}}.
\ShortHotKeyCheatsheet: \myref{sec:Olly_cheatsheet}

\item (Free, open-source) \IT{GDB}.
Strumento non molto popolare tra reverse engineers perchè è per lo più inteso per programmatori.
Alcuni comandi: \myref{sec:GDB_cheatsheet}.
C'è anche un'interfaccia per GDB, ``GDB dashboard''\footnote{\url{https://github.com/cyrus-and/gdb-dashboard}}.

\item (Free, open-source) \IT{LLDB}\footnote{\url{http://lldb.llvm.org/}}.

\item \IT{WinDbg}\footnote{\url{https://developer.microsoft.com/en-us/windows/hardware/windows-driver-kit}}:
kernel debugger per Windows.

\item (Free, open-source) \IT{Radare} \ac{AKA} rada.re \ac{AKA} r2\footnote{\url{http://rada.re/r/}}.
Esiste anche una GUI: \IT{ragui}\footnote{\url{http://radare.org/ragui/}}.

\item (Free, open-source) \IT{tracer}.
\label{tracer}
L'autore usa spesso \IT{tracer}
\footnote{\href{http://go.yurichev.com/17338}{yurichev.com}}
invece di un debugger.

L'autore di queste righe ha smesso di utilizzare un debugger dato che l'unica cosa di cui aveva bisogno era di trovare gli
argomenti delle funzioni durante l'esecuzione o lo stato dei registri ad un determinato punto.
Caricare un debugger ogni volta risultava essere non ottimale, perciò è nacque una nuova utility chiamata \IT{tracer}.
Funziona dalla linea di comando e permette di intercettare l'esecuzione di funzioni,
impostare breakpoint in posizioni arbitrarie, leggere e modificare lo stato dei registri, ecc.

N.B.: \IT{tracer} non sta evolvendo perchè è nato principalmente come strumento di dimostrazione per questo libro, non come strumento di ogni giorno.
\end{itemize}

\subsection{Tracciare chiamate alle librerie}

\IT{ltrace}\footnote{\url{http://www.ltrace.org/}}.

\subsection{Tracciare chiamate di sistema}

\label{strace}
\myindex{strace}
\myindex{dtruss}
\subsubsection{strace / dtruss}

\myindex{syscall}
Mostra quali chiamate di sistema sono chiamate da un processo. (syscalls(\myref{syscalls}))

Per esempio:

\begin{lstlisting}
# strace df -h

...

access("/etc/ld.so.nohwcap", F_OK)      = -1 ENOENT (No such file or directory)
open("/lib/i386-linux-gnu/libc.so.6", O_RDONLY|O_CLOEXEC) = 3
read(3, "\177ELF\1\1\1\0\0\0\0\0\0\0\0\0\3\0\3\0\1\0\0\0\220\232\1\0004\0\0\0"..., 512) = 512
fstat64(3, {st_mode=S_IFREG|0755, st_size=1770984, ...}) = 0
mmap2(NULL, 1780508, PROT_READ|PROT_EXEC, MAP_PRIVATE|MAP_DENYWRITE, 3, 0) = 0xb75b3000
\end{lstlisting}

\myindex{\MacOSX}
\MacOSX ha dtruss per lo stesso compito.

\myindex{Cygwin}
Cygwin ha strace ma, per quanto ne so, funziona solo per file .exe compilati all'interno dell'ambiente
cygwin.

\subsection{Network sniffing}

\IT{Sniffing} significa intercettare informazioni di cui si potrebbe essere interessati.

(Free, open-source) \IT{Wireshark}\footnote{\url{https://www.wireshark.org/}} per lo sniffing di rete.
Puà sniffare anche USB\footnote{\url{https://wiki.wireshark.org/CaptureSetup/USB}}.

Wireshark ha un fratello chiamato \IT{tcpdump}\footnote{\url{http://www.tcpdump.org/}}, un semplice strumento a linea di comando.

\subsection{Sysinternals}

\myindex{Sysinternals}
(Free) Sysinternals (developed by Mark Russinovich)
\footnote{\url{https://technet.microsoft.com/en-us/sysinternals/bb842062}}.
Questi strumenti sono importanti e vale la pena studiarli: Process Explorer, Handle, VMMap, TCPView, Process Monitor.

\subsection{Valgrind}

(Free, open-source) strumento per rilevare memory leak: \url{http://valgrind.org/}.
A causa del suo potente meccanismo \ac{JIT}, Valgrind è utilizzato come framework per altri strumenti.

% TODO network fuzzing

\subsection{Emulatori}

\begin{itemize}
\item (Free, open-source) \IT{QEMU}\footnote{\url{http://qemu.org}}: emulatore per differenti tipi di CPU e architetture.

\item (Free, open-source) \IT{DosBox}\footnote{\url{https://www.dosbox.com/}}: emulatore MS-DOS, usato soprattutto per il retro-gaming.

\item (Free, open-source) \IT{SimH}\footnote{\url{http://simh.trailing-edge.com/}}: emulatore di antichi computer, mainframe, ecc.
\end{itemize}

\section{Altri strumenti}

\IT{Microsoft Visual Studio Express}
\footnote{\href{http://go.yurichev.com/17034}{visualstudio.com/en-US/products/visual-studio-express-vs}}:
Versione gratuita di Visual Studio, conveniente per semplici esperimenti.

Alcune opzioni utili: \myref{sec:MSVC_options}.

\section{Manca qualcosa qui?}

Se conosci qualche strumento non elencato qui, per favore segnalamelo tramite e-mail al seguente indirizzo:\\
\TT{\EMAIL}.

