\section{Calls to assert()}
\myindex{\CStandardLibrary!assert()}

Sometimes the presence of the \TT{assert()} macro is useful too: 
commonly this macro leaves source file name, line number and condition in the code.

The most useful information is contained in the assert's condition, we can deduce variable names or structure field
names from it. Another useful piece of information are the file names---we can try to deduce what type of
code is there.
Also it is possible to recognize well-known open-source libraries by the file names.

\lstinputlisting[caption=Example of informative assert() calls,style=customasmx86]{digging_into_code/assert_examples.lst}

It is advisable to \q{google} both the conditions and file names, which can lead us to an open-source library.
For example, if we \q{google} \q{sp->lzw\_nbits <= BITS\_MAX}, this predictably 
gives us some open-source code that's related to the LZW compression.
