\section{Using magic numbers while tracing}

Oft ist unser Hauptziel zu verstehen wie ein Programm einen Wert behandelt der entweder \"uber eine Datei oder \"uber das Netzwerk erhalten wurde.
Das manuelle tracen eines Wertes ist meistens ein ziemlich arbeits-intensiver Task. Eine der einfachsten Techniken um Werte zu Tracen (auch wenn nicht 100\% verl\"asslich)
ist eigene \IT{magic number}'s zu benutzen. 

Das \"ahnelt ein wenig dem Vorgang beim R\"ontgen auf gewisser weise: ein radioaktives Kontrastmittel wird dem Patienten injeziert,
welches dann benutzt wird um die Gef\"asse des Patienten besser zu erkennen duch die R\"onthgenstahlung. Wie das blut bei 
gesunden Menschen in den Nieren gereinigt wird wenn das Kontrastmittel im Blut ist, man kann dann sehr einfach auf dem
Bild der Tomografie erkennen ob sich Nierensteine oder Tumore in den Nierenbefinden. 

Wir k\"onnen einfach eine 32-Bit Zahl nehmen z.B \TT{0xbadf00d}, oder ein Geburtsdatum wie \TT{0x11101979}
und diese 4-Byte Zahl wird an einem bestimmten Punkt in eine Datei geschrieben welche von dem Programm 
das wir untersuchen genutzt wird. 

\myindex{\GrepUsage}
\myindex{tracer}

Dann w\"ahrend das programm getraced wird mit \tracer im \IT{code coverage} modus, mit der Hilfe von \IT{grep}
oder durch einfaches durchsuchen der Textdatei (der trace Ergebnisse), k\"onnen wir ganz einfach sehen wo der 
Wert benutzt wurde und wie er benutzt wurde. 

Beispiel der \IT{grepable} \tracer Ergebnissen im \IT{cc} mode:

\begin{lstlisting}[style=customasmx86]
0x150bf66 (_kziaia+0x14), e=       1 [MOV EBX, [EBP+8]] [EBP+8]=0xf59c934 
0x150bf69 (_kziaia+0x17), e=       1 [MOV EDX, [69AEB08h]] [69AEB08h]=0 
0x150bf6f (_kziaia+0x1d), e=       1 [FS: MOV EAX, [2Ch]] 
0x150bf75 (_kziaia+0x23), e=       1 [MOV ECX, [EAX+EDX*4]] [EAX+EDX*4]=0xf1ac360 
0x150bf78 (_kziaia+0x26), e=       1 [MOV [EBP-4], ECX] ECX=0xf1ac360 
\end{lstlisting}
% TODO: good example!

Das gleiche verfahren kann man auch auf Netzwerkpakete anwenden.
F\"ur die \IT{magic number} ist es wichtig das diese einzigartig ist und nicht im Programm code vorkommt.

\newcommand{\DOSBOXURL}{\href{http://go.yurichev.com/17222}{blog.yurichev.com}}

\myindex{DosBox}
\myindex{MS-DOS}
Neben dem \tracer Befehl, gibt es noch den DosBox (MS-DOS emulator) im heavydebug Modus,
welcher in der Lage ist alle Informationen \"uber alle Register zust\"ande f\"ur jede ausgef\"uhrte Instruktion des Programmes in
eine einfache Textdatei\footnote{See also my blog post about this DosBox feature: \DOSBOXURL{}} zu schreiben, so kann
diese Technik f\"ur DOS Programme n\"utzlich sein. 

