\section{assert() Aufrufe}
\myindex{\CStandardLibrary!assert()}

Manchmal ist die Pr\"asenz des \TT{assert()} macro's ebenfalls n\"utzlich:
allgemein erlaubt dieses Makro R\"uckschl\"usse auf source code Dateinamen,
Zeilen nummern und die Bedienung f\"ur das Makro im Code.

Die n\"utzlichste Informationen ist enthalten in der Bedingung von assert, wir k\"onnen Variablennamen oder Namen
von Struct Feldern ableiten. Ein weiteres n\"utzliches St\"uck Information sind die Datei Namen---Wir k\"onnen versuchen
abzuleiten von welcher Art der Code ist. 
Es ist ebenfalls m\"oglich bekannte open-source library-Namen von den Datei Namen abzuleiten.

\lstinputlisting[caption=Example of informative assert() calls,style=customasmx86]{digging_into_code/assert_examples.lst}

Es ist Empfehlenswert beides die Konditionen und die Datei Namen in \q{google} zu suchen, was zu einer open-source library f\"uhren kann. 
Zum Beispiel, wenn wir \q{sp->lzw\_nbits <= BITS\_MAX} in \q{google} suchen, ist es absehbar das wir als Ergebnis Code aus der 
Open-Source library f\"ur die LZW Kompression bekommen. 
