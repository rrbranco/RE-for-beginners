% TODO proof-reading
\subsection{Aufrufen von beep()}

Dies ist ein einfacher Aufruf zweier Funktionen ohne Argumente:

\begin{lstlisting}[style=customjava]
	public static void main(String[] args)
	{
		java.awt.Toolkit.getDefaultToolkit().beep();
	};
\end{lstlisting}

\begin{lstlisting}
  public static void main(java.lang.String[]);
    flags: ACC_PUBLIC, ACC_STATIC
    Code:
      stack=1, locals=1, args_size=1
         0: invokestatic  #2      // Method java/awt/Toolkit.getDefaultToolkit:()Ljava/awt/Toolkit;
         3: invokevirtual #3      // Method java/awt/Toolkit.beep:()V
         6: return        
\end{lstlisting}

Zuerst ruft \TT{invokestatic} bei Offset 0 \TT{java.awt.Toolkit.getDefaultToolkit()}
auf, was eine Referenz auf ein Objekt der Klasse \TT{Toolkit} zurück gibt.
Die \TT{invokevirtual}-Anweisung bei Offset 3 ruft die \TT{beep()}-Methode dieser
Klasse auf.
