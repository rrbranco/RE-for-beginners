% TODO proof-reading
\subsubsection{Единственный аргумент \main это также массив}


Будем использовать единственный аргумент \main, который массив строк:

\begin{lstlisting}[style=customjava]
public class UseArgument
{
	public static void main(String[] args)
	{
		System.out.print("Hi, ");
		System.out.print(args[1]);
		System.out.println(". How are you?");
	}
}
\end{lstlisting}


Нулевой аргумент это имя программы (как в \CCpp, итд),
так что первый аргумент это тот, что пользователь добавил первым.

\begin{lstlisting}
  public static void main(java.lang.String[]);
    flags: ACC_PUBLIC, ACC_STATIC
    Code:
      stack=3, locals=1, args_size=1
         0: getstatic     #2      // Field java/lang/System.out:Ljava/io/PrintStream;
         3: ldc           #3      // String Hi, 
         5: invokevirtual #4      // Method java/io/PrintStream.print:(Ljava/lang/String;)V
         8: getstatic     #2      // Field java/lang/System.out:Ljava/io/PrintStream;
        11: aload_0       
        12: iconst_1      
        13: aaload        
        14: invokevirtual #4      // Method java/io/PrintStream.print:(Ljava/lang/String;)V
        17: getstatic     #2      // Field java/lang/System.out:Ljava/io/PrintStream;
        20: ldc           #5      // String . How are you?
        22: invokevirtual #6      // Method java/io/PrintStream.println:(Ljava/lang/String;)V
        25: return        
\end{lstlisting}


\TT{aload\_0} на 11 загружают \IT{reference} на нулевой слот \ac{LVA} 
(первый и единственный аргумент \main).

\TT{iconst\_1} и \TT{aaload} на 12 и 13 берут \IT{reference} на первый (считая с 0) 
элемент массива.

\IT{Reference} на строковый объект на \ac{TOS} по смещению 14, и оттуда он 
берется методом \TT{println}.
