% TODO proof-reading
\subsection{\ac{JVM} memory model}

x86 and other low-level environments use the stack for argument passing and 
as a local variables storage.

\ac{JVM} is slightly different.

It has:

\begin{itemize}
\item Local variable array (\ac{LVA}).
Used as storage for incoming function arguments and local variables.

Instructions like \INS{iload\_0} load values from it.

\INS{istore} stores values in it.
At the beginning the function arguments are stored: starting at 0 or at 1 
(if the zeroth argument is occupied by \IT{this} pointer).

Then the local variables are allocated.


Each slot has size of 32-bit.

Hence, values of \IT{long} and \IT{double} data types occupy two slots.


\item Operand stack (or just \q{stack}).
It's used for computations and passing arguments while calling other functions.

Unlike low-level environments like x86, it's not possible to access the stack without using
instructions which explicitly pushes or pops values to/from it.


\item Heap. It is used as storage for objects and arrays.

\end{itemize}

These 3 areas are isolated from each other.

