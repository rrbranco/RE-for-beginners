% TODO proof-reading
\subsection{Циклы}

\begin{lstlisting}[style=customjava]
public class Loop
{
	public static void main(String[] args)
	{ 
		for (int i = 1; i <= 10; i++)
		{
			System.out.println(i); 
		}               
	}
}
\end{lstlisting}

\begin{lstlisting}
  public static void main(java.lang.String[]);
    flags: ACC_PUBLIC, ACC_STATIC
    Code:
      stack=2, locals=2, args_size=1
         0: iconst_1      
         1: istore_1      
         2: iload_1       
         3: bipush        10
         5: if_icmpgt     21
         8: getstatic     #2         // Field java/lang/System.out:Ljava/io/PrintStream;
        11: iload_1       
        12: invokevirtual #3         // Method java/io/PrintStream.println:(I)V
        15: iinc          1, 1
        18: goto          2
        21: return        
\end{lstlisting}


\TT{iconst\_1} загружает 1 в \ac{TOS}, \TT{istore\_1} сохраняет её в первом слоте \ac{LVA}.
Почему не нулевой слот?

Потому что функция \main имеет один аргумент (массив \TT{String}),
и указатель на него (или \IT{reference}) сейчас в нулевом слоте.


Так что локальная переменная \IT{i} всегда будет в первом слоте.


Инструкции по смещениями 3 и 5 сравнивают \IT{i} с 10.

Если \IT{i} больше, управление передается на смещение 21, где функция заканчивает работу.

Если нет, вызывается \TT{println}.

\IT{i} перезагружается по смещению 11, для \TT{println}.

Кстати, мы вызываем метод \TT{println} для типа данных \IT{integer}, 
и мы видим это в комментариях: \q{(I)V}
(\IT{I} означает \IT{integer} и \IT{V} означает, что возвращаемое значение имеет тип \IT{void}).


Когда \TT{println} заканчивается, \IT{i} увеличивается на 1 по смещению 15.

Первый операнд инструкции это номер слота (1), 
второй это число (1) для прибавления.


\TT{goto} это просто GOTO, она переходит на начало цикла по смещению 2.


Перейдем к более сложному примеру:

\begin{lstlisting}[style=customjava]
public class Fibonacci
{
	public static void main(String[] args)
	{ 
		int limit = 20, f = 0, g = 1;

		for (int i = 1; i <= limit; i++)
		{
			f = f + g;
			g = f - g;
			System.out.println(f); 
		}
	}
}
\end{lstlisting}

\begin{lstlisting}
  public static void main(java.lang.String[]);
    flags: ACC_PUBLIC, ACC_STATIC
    Code:
      stack=2, locals=5, args_size=1
         0: bipush        20
         2: istore_1      
         3: iconst_0      
         4: istore_2      
         5: iconst_1      
         6: istore_3      
         7: iconst_1      
         8: istore        4
        10: iload         4
        12: iload_1       
        13: if_icmpgt     37
        16: iload_2       
        17: iload_3       
        18: iadd          
        19: istore_2      
        20: iload_2       
        21: iload_3       
        22: isub          
        23: istore_3      
        24: getstatic     #2         // Field java/lang/System.out:Ljava/io/PrintStream;
        27: iload_2       
        28: invokevirtual #3         // Method java/io/PrintStream.println:(I)V
        31: iinc          4, 1
        34: goto          10
        37: return        
\end{lstlisting}
        

Вот карта слотов в \ac{LVA}:

\begin{itemize}
\item 0 --- единственный аргумент функции \main
\item 1 --- \IT{limit}, всегда содержит 20
\item 2 --- \IT{f}
\item 3 --- \IT{g}
\item 4 --- \IT{i}
\end{itemize}


Мы видим, что компилятор Java расположил переменные в слотах \ac{LVA} в точно таком же порядке,
в котором переменные были определены в исходном коде.


Существуют отдельные инструкции \TT{istore} для слотов 0, 1, 2, 3, но не 4 и более, 
так что здесь есть \TT{istore} с дополнительным операндом по смещению 8, 
которая имеет номер слота в операнде.

Та же история с \TT{iload} по смещению 10.


Но не слишком ли это сомнительно, выделить целый слот для переменной \IT{limit},
которая всегда содержит 20 (так что это по сути константа), и перезагружать её так часто?

\ac{JIT}-компилятор в \ac{JVM} обычно достаточно хорош, чтобы всё это оптимизировать.

Самостоятельное вмешательство в код, наверное, того не стоит.

