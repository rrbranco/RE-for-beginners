% TODO proof-reading
\subsection{Introduction}

\newcommand{\JADURL}{\url{http://varaneckas.com/jad/}}

There are some well-known decompilers for Java (or \ac{JVM} bytecode in general)
\footnote{For example, JAD: \JADURL}.

The reason is the decompilation of \ac{JVM}-bytecode is somewhat easier 
than for lower level x86 code:

\begin{itemize}
\item There is much more information about the data types.
\item The \ac{JVM} memory model is much more rigorous and outlined.
\item The Java compiler don't do any optimizations (the \ac{JVM} \ac{JIT} does them at runtime),
      so the bytecode in the class files is usually pretty readable.
      
\end{itemize}

When can the knowledge of \ac{JVM} be useful?

\newcommand{\URLListOfJVMLangs}{\url{http://en.wikipedia.org/wiki/List_of_JVM_languages}}

\begin{itemize}
\item Quick-and-dirty patching tasks of class files without the need to recompile the decompiler's results.
\item Analyzing obfuscated code.
\item Building your own obfuscator.
\item Building a compiler codegenerator (back-end) targeting \ac{JVM} (like Scala, Clojure, etc.
      \footnote{Full list: \URLListOfJVMLangs}).
      
\end{itemize}

Let's start with some simple pieces of code.
JDK 1.7 is used everywhere, unless mentioned otherwise.

This is the command used to decompile class files everywhere:\\
\GTT{javap -c -verbose}.

This is the book I used while preparing all examples: \JavaBook.

