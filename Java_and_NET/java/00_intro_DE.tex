% TODO proof-reading
\subsection{Einführung}

\newcommand{\JADURL}{\url{http://varaneckas.com/jad/}}

Es gibt einige bekannte Decompiler für Java (oder \ac{JVM}-Bytecode
allgemein)\footnote{Beispielsweise JAD: \JADURL}.

Der Grund ist, dass das Dekompilieren von \ac{JVM}-Bytecode einfacher ist als
von Low-Level x86-Code:

\begin{itemize}
\item Es gibt sehr viel mehr Informationen über die Datentypen.
\item Das \ac{JVM}-Speichermodell ist sehr viel strenger und genauer beschrieben.
\item Der Java-Compiler führt keine Optimierungen durch (dies macht der \ac{JVM}
\ac{JIT} während der Laufzeit, do dass der Bytecode in der Klassendatei normalerweise
gut lesbar ist).
\end{itemize}

Wann kann das Wissen über \ac{JVM} nützlich sein?

\newcommand{\URLListOfJVMLangs}{\url{http://en.wikipedia.org/wiki/List_of_JVM_languages}}

\begin{itemize}
\item Quick-and-dirty-Patchen von Klassendateien ohne das neukompilieren der Decompiler-Ergebnisse.
\item Analysieren von obfuskierten Code.
\item Erstellen eines eigenen Obfuscators.
\item Erstellen eines Compiler Code-Generators (back-end) mit dem Ziek \ac{JVM}
(wie Scala, Clojure, usw.) \footnote{vollständige Liste: \URLListOfJVMLangs}).
\end{itemize}

Starten wir mit einigen einfachen Code-Beispielen.
Wenn nicht anders erwähnt wird JDK 1.7 verwendet.

Das folgende Kommando wird verwendet um Klassen zu decompilieren:


\GTT{javap -c -verbose}.

Das folgende Buch habe ich während der Vorbereitung der Beispiele genutzt
\JavaBook.
