\subsection{Uma breve introdução a CPU}

A \ac{CPU} é o dispositivo que executa o código de máquina do qual consiste um programa.

\textbf{Um glossário resumido:}

\begin{description}
\item[Instrução]: Um comando primário da \ac{CPU}. Os exemplos mais simples incluem: mover informação entre os registradores, trabalhar com memória, operações primárias de aritimética.
Como regra, cada \ac{CPU} tem sua própria arquitetura do conjunto de instruções (\ac{ISA}).

\item[Código de máquina]: Código que a \ac{CPU} processa diretamente, cada instrução geralmente é codificada por vários bytes.

\item[Linguagem assembly]: Códigos mnemônicos e algumas extensões como macros que tem por intenção facilitar a vida do programador.

\item[Registrador da CPU]: Cada \ac{CPU} tem um conjunto fixo de registradores de propósito geral (\ac{GPR}).
Aproximadamente 8 na arquitetura x86, 16 na x86-64, 16 na ARM.
A maneira mais fácil de entender um registrador é imaginá-lo como uma variável temporário sem tipo.
Imagine que você está trabalhando em uma linguagem de alto nível e pudesse usar somente oito variáveis de 32-bit (ou 64).
Ainda assim uma gama de coisas podem ser feitas usando somente estes!

\end{description}

Você pode pensar por quê há a necessidade dessa diferença entre código de máquina e linguagens de programação de alto nível.
A resposta está no fato de humanos e CPUs não se parecerem nada --- é muito mais fácil para um humano usar uma linguagem de alto nível como \CCpp, Java, Python, etc., 
mas para a CPU é mais fácil usar um nível muito mais baixo de abstração.
Talvez seja possível inventar uma CPU que pode executar códigos em linguagem de alto nível, mas ela seria muitas vezes mais complexa que as CPUs que conhecemos hoje.
De uma maneira similar, é muito mais inconveniente para humanos escreverem em linguagem assemly,
devido ao fato dela ser tão baixo nível e difícil de se escrever sem cometer um alto número de erros irritantes. O programa que converte linguagem de alto nível em código assembly é chamado de compilador.

\iffalse
% another translation
\subsection{Uma breve introdução à CPU}

A \ac{CPU} é o dispositivo que executa o código de máquina que consiste num programa.
\textbf{Um pequeno glossário:}
\begin{description}
\item[Instrução]: Um primitivo \ac{CPU} comando.
Os exemplos mais simples incluem: mover dados entre registradores, trabalhar com a memória, operações aritiméticas primitivas.
Como regra geral, cada \ac{CPU} tem seu próprio conjunto de instruções (\ac{ISA}).

\item[\PTBRph{}]: Código que a \ac{CPU} processa diretamente. 
Cada instrução é normalmente codificada em vários bytes.
\item[Linguagem assembly]: Código mnemônico e algumas extensões como macros que têm a finalidade de facilitar a vida do programamdor.
\item[Registradores da CPU]: Cada \ac{CPU} tem um conjunto fixo de registradores de propósito geral (\ac{GPR}).
$\approx 8$ \PTBRph{} x86, $\approx 16$ \PTBRph{} x86-64, $\approx 16$ \PTBRph{} ARM.
A forma mais fácil de entender um registrador é pensar nele como uma variável temporária não tipada.
Imagine que você estivesse trabalhando com uma \ac{PL} de alto nível e pudesse usar apenas oito variáveis de 32-bit (ou de 64-bit).
No entanto, muito ainda pode ser feito usando apenas eles.
\end{description}

Alguém poderia perguntar por que é preciso haver diferença entre código de máquina e uma \ac{PL} de alto nível.  A resposta reside no fato de que humanos e \ac{CPU}s não são iguais---%
É muito mais fácil para os humanos usar uma \ac{PL} de alto nível como \CCpp, Java, Python, etc., mas é muito mais fácil para a \ac{CPU} usar um nível de abstração muito menor.
talvez fosse possível inventar uma \ac{CPU} que pudesse executar código feito em \ac{PL} alto nível, mas seria inúmeras vezes mais complexa do que as \ac{CPU}s que conhecemos hoje.
De forma semelhante, é muito inconveninente para os seres humanos escrever em linguagem assembly, devido ao fato dela ser tão baixo nível e difícil de escrever sem comenter uma enorme quantidade de erros irritantes.
O programa que converte o código de \ac{PL} de alto nível em assembly é chamado \IT{compiler}.
\fi

\myindex{ARM!\ARMMode}%
\myindex{ARM!\ThumbMode}%
\myindex{ARM!\ThumbTwoMode}%

\subsubsection{Algumas palavras a respeito de diferentes \ac{ISA}s}
O \ac{ISA} x86 sempre possuiu opcodes de tamanho variável, então com a chegada da era do 64-bit, as extensões x64 não impactaram a \ac{ISA} de forma muito significante. De fato, o \ac{ISA} x86 ainda contém uma série de instrucões que surgiram inicialmente na CPU 8086 16-bit, mas ainda são encontradas nas CPUs de hoje em dia.
ARM é uma \ac{CPU} \ac{RISC} desenvolvido com a idéia de opcodes com tamanho constante, o que trouxe algumas vantagens no passado.
Bem no início, todas as instruções ARM foram codificadas em 4 bytes%
\footnote{A propósito, instruções de tamanho fixo são úteis porque se pode calcular o endereço da próxima instrução (ou da anterior) sem esforço. Esta característica será discutida na seção do operador switch() ~(\myref{sec:SwitchARMLot}).}.
Este é atualmente referenciado como \q{ARM mode}.
Então concluiu-se que não era tão econômico quanto se imaginou a princípio.
Na verdade, as instruções de \ac{CPU} mais utilizadas \footnote{São estas MOV/PUSH/CALL/Jcc} em aplicações do mundo real podem ser codificadas usando menos informação.
Foi adicionado então outro \ac{ISA}, chamado Thumb, onde cada instrução era codificada em apenas 2 bytes.
Este é conhecido como \q{Thumb mode}.
No entanto, nem \IT{all} instruções ARM podem ser codificadas em apenas 2 bytes, então o conjunto de instruções Thumb é de certa forma limitado.
É interessante notar que códigos compilados para os modos ARM e Thumb podem, conforme esperado, coexistir num mesmo programa.
Os criadores do ARM concluíram que o Thumb poderia ser extendido, dando origem ao Thumb-2, que apareceu no ARMv7.
Thumb-2 ainda usa instruções de 2 bytes, mas possui algumas novas instruções com 4 bytes de tamanho.
Há um equívoco comum que Thumb-2 é uma mistura de ARM e Thumb. Isso é incorreto.
Em vez disso, Thumb-2 foi extendido para suportar completamente todos os recursos de processador de forma que ele pudesse competir com o modo ARM---um objetivo que foi claramente alcançado, uma vez que a maioria das aplicações para \idevices são compiladas para o conjunto de instruções do Thumb-2 (admitidamente, principalmente devido ao fato que o Xcode faz isso por padrão).
Posteriormente o ARM 64-bit foi lançado. Este \ac{ISA} tem opcodes de 4 bytes, e descarta a necessidade de qualquer modo Thumb adicional.
No entanto, os requisitos de 64-bit afetaram o \ac{ISA}, resultando em termos atualmente três conjuntos de instruções ARM: ARM mode, Thumb mode (incluindo Thumb-2) e ARM64.
Estes \ac{ISA}s se intersecionam parcialmente, porém podemos dizer que são \ac{ISA}s diferentes, ao invés de variações do mesmo.
Portanto, gostaríamos de tentar adicionar pedaços de código dos três \ac{ISA}s do ARM neste livro.
\myindex{PowerPC}%
\myindex{MIPS}%
\myindex{Alpha AXP}%
Existem, a propósito, muitos outros \ac{RISC} \ac{ISA}s com opcodes de tamanho fixo de 32-bit, como MIPS, PowerPC \PTBRph{} Alpha AXP.
