% TODO to be resynced with EN version
\section{La funzione più semplice}

La funzione più semplice possibile è probabilmente quella che restituisce un valore costante:

Eccola:

\lstinputlisting[caption=\CCpp Codice,style=customc]{patterns/011_ret/1.c}

Compiliamola!

\subsection{x86}

Questo è il risultato prodotto dai compilatori ottimizzanti GCC e MSVC sulla piattaforma x86:

\lstinputlisting[caption=\Optimizing GCC/MSVC (\assemblyOutput),style=customasmx86]{patterns/011_ret/1.s}

\myindex{x86!\Instructions!RET}
Ci sono solo due istruzioni: la prima mette il valore 123 nel registro \EAX, per convenzione usato per la memorizzazione del valore di ritorno, la seconda, \RET, restituisce l'esecuzione alla funzione \glslink{caller}{chiamante}.

La funzione chiamante prenderà il risultato dal registro \EAX.

\subsection{ARM}

Sulla piattaforma ARM ci sono alcune differenze:

\lstinputlisting[caption=\OptimizingKeilVI (\ARMMode) ASM Output,style=customasmARM]{patterns/011_ret/1_Keil_ARM_O3.s}

ARM usa il registro \Reg{0} per restituire i risultati delle funzioni, quindi 123 viene copiato in \Reg{0}.

Il return address sulla piattaforma ARM \ac{ISA} non viene salvato nello stack locale ma nel link register, 
così che l'istruzione \INS{BX LR} faccia in modo che l'esecuzione "salti" a quell'indirizzo --- restituendo effettivamente l'esecuzione alla funzione \glslink{caller}{chiamante}.

\myindex{ARM!\Instructions!MOV}
\myindex{x86!\Instructions!MOV}
Vale la pena notare che il nome dell'istruzione \MOV è fuorviante in entrambe le \ac{ISA} x86 e ARM. 

Il dato non viene infatti \IT{spostato}, bensì \IT{copiato}.

\subsection{MIPS}

\label{MIPS_leaf_function_ex1}
Esistono due naming conventions (convenzioni di denominazione) usate nel mondo MIPS per fare riferimento ai registri: per numero (da \$0 a \$31) or per pseudonome (\$V0, \$A0, etc.).

Il seguente output assembly di GCC elenca i registri per numero:

\lstinputlisting[caption=\Optimizing GCC 4.4.5 (\assemblyOutput),style=customasmMIPS]{patterns/011_ret/MIPS.s}

\ITAph{}

\lstinputlisting[caption=\Optimizing GCC 4.4.5 (IDA),style=customasmMIPS]{patterns/011_ret/MIPS_IDA.lst}

Il registro \$2 (o \$V0) è usato per memorizzare il valore di ritorno di una funzione.
\myindex{MIPS!\Pseudoinstructions!LI}
\INS{LI} sta per ``Load Immediate'' ed è l'equivalente MIPS di \MOV.

\myindex{MIPS!\Instructions!J}
L'altra istruzione è quella di jump (salto) (J or JR) che riporta il flusso di esecuzione alla funzione \glslink{caller}{chiamante}, \IT{saltando} all'indirizzo specificato nel registro \$31 (o \$RA).

Questo registro è l'analogo di \ac{LR} in ARM.

\myindex{MIPS!Branch delay slot}
Ci si potrebbe domandare perchè le posizioni delle istruzioni load (LI) e jump (J or JR) siano invertite. Cio' e' dovuto ad una feature di \ac{RISC} detta ``branch delay slot''.

La ragione per cui ciò avviene è una peculiarità dell'architettura di alcune \ac{ISA}s RISC e non è di importanza rilevante per il nostro scopo - sarà sufficiente semplicemente ricordare che in MIPS l'istruzione che segue un jump o branch viene eseguita \IT{prima} del jump/branch stesso.

Di conseguenza, le istruzioni branch scambiano sempre il posto con l'istruzione che deve essere eseguita prima.

\subsubsection{Una nota sui nomi delle istruzioni e dei registri MIPS}

I nomi dei registri e delle istruzioni nel mondo MIPS sono solitamente scritti in minuscolo. Tuttavia, per consistenza, useremo nomi in maiuscolo secondo la convenzione seguita da tutte le altre \ac{ISA}s trattate in questo libro.

