\subsection{Let's allocate space for a structure using malloc()}
\label{struct_malloc_example}


Sometimes it is simpler to place structures not the in local stack, but in the \gls{heap}:

\lstinputlisting[style=customc]{patterns/15_structs/2_using_malloc/systemtime_malloc.c}


Let's compile it now with optimization (\Ox) so it would be easy to see what we need.

\lstinputlisting[caption=\Optimizing MSVC,style=customasmx86]{patterns/15_structs/2_using_malloc/systemtime_malloc.asm}

\myindex{\CStandardLibrary!malloc()}

So, \TT{sizeof(SYSTEMTIME) = 16} and that is exact number of bytes to be allocated by \TT{malloc()}.
It returns a pointer to a freshly allocated memory block in the \EAX register,
which is then moved into the \ESI register.
\TT{GetSystemTime()} win32 function takes care of saving value in \ESI,
and that is why it is not saved here and continues to be used after the \TT{GetSystemTime()} call.

\myindex{x86!\Instructions!MOVZX}

New instruction~---\MOVZX (\IT{Move with Zero eXtend}).
It may be used in most cases as \MOVSX, but it sets the remaining bits to 0.
That's because \printf requires a 32-bit \Tint, but we got a WORD in the structure~---that
is 16-bit unsigned type.
That's why by copying the value from a WORD into \Tint{}, bits from 16 to 31 must be cleared, 
because a random noise may be there, which is left from the previous operations on the register(s).


In this example, it's possible to represent the structure as an array of 8 WORDs:

\lstinputlisting[style=customc]{patterns/15_structs/2_using_malloc/systemtime_malloc2.c}

We get:

\lstinputlisting[caption=\Optimizing MSVC,style=customasmx86]{patterns/15_structs/2_using_malloc/systemtime_malloc2.asm}


Again, we got the code that cannot be distinguished from the previous one.

And again it has to be noted, you haven't to do this in practice, unless you really know what you are doing.

