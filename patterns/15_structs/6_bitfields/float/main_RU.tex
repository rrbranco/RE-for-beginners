\subsubsection{\WorkingWithFloatAsWithStructSubSubSectionName}
\label{sec:floatasstruct}

Как уже ранее указывалось в секции о FPU~(\myref{sec:FPU}), 
и \Tfloat и \Tdouble содержат в себе \IT{знак}, \IT{мантиссу} и \IT{экспоненту}. 
Однако, можем ли мы работать с этими полями напрямую? Попробуем с \Tfloat.

\bigskip
% a hack used here! http://tex.stackexchange.com/questions/73524/bytefield-package
\begin{center}
\begin{bytefield}{32}
	\bitheader[endianness=big]{0,22,23,30,31} \\
	\bitbox{1}{S} & 
	\bitbox{8}{%
		\RU{экспонента}%
		\EN{exponent}%
		\ES{exponente}%
		\PTBRph{}%
		\DEph{}\PLph{}%
		\ITAph{}%
		\FR{exposant}
	} & 
	\bitbox{23}{%
		\RU{мантисса}%
		\EN{mantissa or fraction}%
		\ES{mantisa o fracci\'on}%
		\PTBRph{}%
		\DEph{}\PLph{}%
		\ITAph{}%
		\FR{mantisse ou fraction}
	}
\end{bytefield}
\end{center}

\begin{center}
( S\EMDASH{}%
	\RU{знак}%
	\EN{sign}%
	\ES{signo}%
	\PTBRph{}%
	\DEph{}\PLph{}%
	\ITAph{}%
	\FR{signe}
)
\end{center}


\lstinputlisting[style=customc]{patterns/15_structs/6_bitfields/float/float_RU.c}

Структура \TT{float\_as\_struct} занимает в памяти столько же места сколько и \Tfloat, 
то есть 4 байта или 32 бита.

Далее мы выставляем во входящем значении отрицательный знак, 
а также прибавляя двойку к экспоненте, мы тем 
самым умножаем всё значение на \TT{$2^2$}, то есть на 4.

Компилируем в MSVC 2008 без включенной оптимизации:

\lstinputlisting[caption=\NonOptimizing MSVC 2008,style=customasmx86]{patterns/15_structs/6_bitfields/float/float_msvc_RU.asm}

Слегка избыточно. В версии скомпилированной с флагом \Ox нет вызовов \TT{memcpy()}, 
там работа происходит сразу с переменной \TT{f}. Но по неоптимизированной версии будет проще понять.

А что сделает GCC 4.4.1 с опцией \Othree?

\lstinputlisting[caption=\Optimizing GCC 4.4.1,style=customasmx86]{patterns/15_structs/6_bitfields/float/float_gcc_O3_RU.asm}

Да, функция \ttf в целом понятна. Однако, что интересно, еще при компиляции, 
не взирая на мешанину с полями структуры, GCC умудрился вычислить результат функции \TT{f(1.234)} еще
во время компиляции и сразу подставить его в аргумент для \printf{}!

