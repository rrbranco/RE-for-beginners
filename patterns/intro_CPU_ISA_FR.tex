\subsection{Une courte introduction sur le CPU}

Le \ac{CPU} est le système qui exécute le code machine que constitue le programme.

\textbf{Un court glossaire:}

\begin{description}
\item[Instruction]: Une commande \ac{CPU} primitive.
Les exemples les plus simples incluent: déplacement de données entre les registres, travail avec la mémoire et les opérations arithmétiques primitives.
Généralement, chaque \ac{CPU} a son propre jeu d'instructions (\ac{ISA}).

\item[Code machine]: Code que le \ac{CPU} exécute directement. 
Chaque instruction est codée sur plusieurs octets.
\item[Langage d'assemblage]: Code mnémotechnique et quelques extensions comme les macros qui facilitent la vie du programmeur.
\item[Registre CPU]: Chaque \ac{CPU} a un ensemble de registres d'intérêt général (\ac{GPR}).
$\approx 8$ pour x86, $\approx 16$ pour x86-64, $\approx 16$ pour ARM.
Le moyen le plus simple de comprendre un registre est de le voir comme une variable temporaire non-typée.
Imaginez que vous travaillez avec un \ac{PL} de haut niveau et vous pouvez utiliser uniquement huit variables de 32-bit (ou de 64-bit).
C'est malgré tout possible de faire beaucoup de choses en les utilisant!
\end{description}

% TODO1 add about linker: "компоновщик" и "редактор связей" в русскоязычной лит-ре

% A note on the experiments in this area (like the LISP machines http://en.wikipedia.org/wiki/Lisp_machine
% might be useful
On pourrait se demander pourquoi il y a besoin d'une différence entre le code machine et un \ac{PL}. Le réponse est que les humains et les \ac{CPU}s ne sont pas semblables---%
c'est beaucoup plus simple pour les humains d'utiliser un \ac{PL} de haut niveau comme \CCpp, Java, Python, etc., mais c'est plus simple pour un \ac{CPU} d'utiliser un niveau d'abstraction de beaucoup plus bas niveau.
Peut-être qu'il serait possible d'inventer un \ac{CPU} qui puisse exécuter du code d'un \ac{PL} de haut niveau, mais il serait beaucoup plus complexe que les \ac{CPU}s que nous connaissons aujourd'hui.
D'une manière similaire, c'est moins facile pour les humains d'écrire en langage d'assemblage à cause de son bas niveau et de la difficulté d'écrire sans faire un nombre énorme de fautes agaçantes.
Le programme qui convertit d'un \ac{PL} haut niveau vers l'assemblage est appelé un \IT{compilateur}.

\myindex{ARM!\ARMMode}%
\myindex{ARM!\ThumbMode}%
\myindex{ARM!\ThumbTwoMode}%

\subsubsection{Quelques mots sur les différents \ac{ISA}s}
Le jeu d'instructions \ac{ISA} x86 a toujours été avec des instructions de taille variable. Donc quand l'époque du 64-bit arriva, les extensions x64 n'ont pas impacté le \ac{ISA} très signficativement. En fait, le \ac{ISA} x86 contient toujours beaucoup d'instructions apparus pour la première fois dans un CPU 8086 16-bit, et qui sont encore trouvables dans beaucoup de CPUs aujourd'hui.
ARM est un \ac{CPU} \ac{RISC} conçu avec l'idée d'instructions de taille fixe, ce qui présentait quelques avantages dans le passé.
Au tout début, toutes les instructions ARM étaient codés sur 4 octets%
\footnote{D'ailleurs, les instructions de taille fixe sont pratiques parce qu'il est possible de calculer l'instruction suivante (ou précédente) sans effort. Cette caracteristique sera discutée dans la section de l'opérateur switch()~(\myref{sec:SwitchARMLot}).}.
C'est maintenant connu comme le \q{ARM mode}.
Ensuite ils sont arrivés à la conclusion que ce n'était pas aussi économique qu'ils l'avaient imaginé sur le principe.
En realité, la majorité des instructions \ac{CPU} utilisées \footnote{Ce sont MOV/PUSH/CALL/Jcc} dans le monde réel peuvent être encodées en utilisant moins d'informations.
Ils ont par conséquent ajouté un autre \ac{ISA}, appelé Thumb, où chaque instruction était encodée sur seulement 2 octets.
C'est maintenant connu comme le \q{Thumb mode}.
Cependant, toutes les instructions ne peuvent être encodées sur seulement 2 octets, donc les instructions Thumb sont un peu limitées.
On peut noter que le code compilé pour le mode ARM et pour le mode Thumb peut, évidemment, coexister dans un seul programme.
Les créateurs de ARM pensèrent que Thumb pourrait être étendu, donnant naissance à Thumb-2, qui apparut dans ARMv7.
Thumb-2 utilise toujours des instructions de 2 octets, mais a de nouvelles instructions dont la taille est de 4 octets.
Une erreur couramment répandue est que Thumb-2 est un mélange de ARM et Thumb. C'est incorrect.
Plutôt, Thumb-2 fut étendu pour supporter totalement toutes les caractéristiques du processeur pour qu'il puisse
rivaliser avec le mode ARM---un objectif qui a été clairement réussi, comme la majorité des applications pour \idevices est compilée pour le jeu d'instructions de Thumb-2 (il est vrai que c'est largement dû au fait que Xcode le faisait par défaut).
Plus tard, le ARM 64-bit sortit. Ce \ac{ISA} a des instructions de 4 octets, et enlèvait le besoin d'un mode Thumb supplémentaire.
Cependant, les prérequis de 64-bit affectèrent le \ac{ISA}, résultant maintenant au fait que nous ayons trois jeux d'instructions ARM: ARM mode, Thumb mode (incluant Thumb-2) et ARM64.
Ces \ac{ISA}s s'intersectent partiellement, mais on peut dire que ce sont des \ac{ISA}s différents, plutôt que des variantes du même.
Par conséquent, nous voudrions essayer d'ajouter des fragments de code dans les trois \ac{ISA}s de ARM dans ce livre.
\myindex{PowerPC}%
\myindex{MIPS}%
\myindex{Alpha AXP}%
Il y a, d'ailleurs, bien d'autres \ac{ISA}s \ac{RISC} avec des instructions de taille fixe de 32-bit, comme MIPS, PowerPC et Alpha AXP.
