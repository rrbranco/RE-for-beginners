\subsubsection{Memorizzazione di variabili locali}

Una funzione puo' allocare spazio nello stack per le sue variabili locali, semplicemente decrementando
lo \gls{stack pointer} verso il basso (indirizzi piu' bassi) dello stack.

% I think here, "stack bottom" means the lowest address in the stack space,
% but the reader might also think it means towards the top of the stack space,
% like in a pop, so you might change "towards the stack bottom" to
% "towards the lowest address of the stack", or just take it out,
% since "decreasing" also suggests that.

Pertanto l'operazione risulta molto veloce, a prescinedere dal numero di variabili locali definite.
Anche in questo caso utilizzare lo stack per memorizzare variabili locali non e' un requisito necessario.
Si possono memorizzare dove si vuole, ma tradizionalmente si fa cosi'.
