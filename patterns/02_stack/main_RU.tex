\section{\Stack}
\label{sec:stack}
\myindex{\Stack}

Стек в информатике~--- это одна из наиболее фундаментальных структур данных
\footnote{\href{http://go.yurichev.com/17119}{wikipedia.org/wiki/Call\_stack}}.
\ac{AKA} \ac{LIFO}.

Технически это просто блок памяти в памяти процесса + регистр \ESP в x86 или \RSP в x64, либо \ac{SP} в ARM, который указывает где-то в пределах этого блока.

\myindex{ARM!\Instructions!PUSH}
\myindex{ARM!\Instructions!POP}
\myindex{x86!\Instructions!PUSH}
\myindex{x86!\Instructions!POP}
Часто используемые инструкции для работы со стеком~--- это \PUSH и \POP (в x86 и Thumb-режиме ARM). 
\PUSH уменьшает \ESP/\RSP/\ac{SP} на 4 в 32-битном режиме (или на 8 в 64-битном),
затем записывает по адресу, на который указывает \ESP/\RSP/\ac{SP}, содержимое своего единственного операнда.

\POP это обратная операция~--- сначала достает из \glslink{stack pointer}{указателя стека} значение и помещает его в операнд 
(который очень часто является регистром) и затем увеличивает указатель стека на 4 (или 8).

В самом начале \glslink{stack pointer}{регистр-указатель} указывает на конец стека.
Конец стека находится в начале блока памяти, выделенного под стек. Это странно, но это так.
\PUSH уменьшает \glslink{stack pointer}{регистр-указатель}, а \POP~--- увеличивает.

В процессоре ARM, тем не менее, есть поддержка стеков, растущих как в сторону уменьшения, так и в сторону увеличения.

\myindex{ARM!\Instructions!STMFD}
\myindex{ARM!\Instructions!LDMFD}
\myindex{ARM!\Instructions!STMED}
\myindex{ARM!\Instructions!LDMED}
\myindex{ARM!\Instructions!STMFA}
\myindex{ARM!\Instructions!LDMFA}
\myindex{ARM!\Instructions!STMEA}
\myindex{ARM!\Instructions!LDMEA}

Например, инструкции \ac{STMFD}/\ac{LDMFD}, \ac{STMED}/\ac{LDMED} предназначены для descending-стека (растет назад, начиная с высоких адресов в сторону низких).\\
Инструкции \ac{STMFA}/\ac{LDMFA}, \ac{STMEA}/\ac{LDMEA} предназначены для ascending-стека (растет вперед, начиная с низких адресов в сторону высоких).

% It might be worth mentioning that STMED and STMEA write first,
% and then move the pointer,
% and that LDMED and LDMEA move the pointer first, and then read.
% In other words, ARM not only lets the stack grow in a non-standard direction,
% but also in a non-standard order.
% Maybe this can be in the glossary, which would explain why E stands for "empty".

\subsection{Почему стек растет в обратную сторону?}
\label{stack_grow_backwards}

Интуитивно мы можем подумать, что, как и любая другая структура данных, стек мог бы расти вперед, т.е. в сторону увеличения адресов.

Причина, почему стек растет назад, видимо, историческая.
Когда компьютеры были большие и занимали целую комнату, было очень легко разделить сегмент на две части: для \glslink{heap}{кучи} и для стека.
Заранее было неизвестно, насколько большой может быть \glslink{heap}{куча} или стек, так что это решение было самым простым.

\begin{center}
	\begin{tikzpicture}
	\tikzstyle{every path}=[thick]

	\node [rectangle,draw,minimum width=6cm, minimum height=2cm] (memory) {};
	\node [] [right=0.2cm of memory.west] (heap) {\MLHeap};
	\node [] [left=0.2cm of memory.east] (stack) {\MLStack};

	\node [] (center1) [right=2cm of memory.west] {};
	\node [] (center2) [left=2cm of memory.east] {};

	\draw [->] (heap) -- (center1);
	\draw [->] (stack) -- (center2);

	\node [] [above left=1.1cm and 0.2cm of heap] (t1) {\MLStartOfHeap};
	\node [] [above right=1.1cm and 0.2cm of stack] (t2) {\MLStartOfStack};

	\draw [->] (t1) -- (memory.west);
	\draw [->] (t2) -- (memory.east);

	\end{tikzpicture}
\end{center}


В \RitchieThompsonUNIX можно прочитать:

\begin{framed}
\begin{quotation}
The user-core part of an image is divided into three logical segments. The program text segment begins at location 0 in the virtual address space. During execution, this segment is write-protected and a single copy of it is shared among all processes executing the same program. At the first 8K byte boundary above the program text segment in the virtual address space begins a nonshared, writable data segment, the size of which may be extended by a system call. Starting at the highest address in the virtual address space is a stack segment, which automatically grows downward as the hardware's stack pointer fluctuates.
\end{quotation}
\end{framed}

Это немного напоминает как некоторые студенты
пишут два конспекта в одной тетрадке:
первый конспект начинается обычным образом, второй пишется с конца, перевернув тетрадку.
Конспекты могут встретиться где-то посредине, в случае недостатка свободного места.

% I think if we want to expand on this analogy,
% one might remember that the line number increases as as you go down a page.
% So when you decrease the address when pushing to the stack, visually,
% the stack does grow upwards.
% Of course, the problem is that in most human languages,
% just as with computers,
% we write downwards, so this direction is what makes buffer overflows so messy.

\subsection{Для чего используется стек?}

% subsections
\EN{\subsubsection{Save the function's return address}

\myparagraph{x86}

\myindex{x86!\Instructions!CALL}
When calling another function with a \CALL instruction, the address of the point exactly after the \CALL instruction is saved 
to the stack and then an unconditional jump to the address in the \CALL operand is executed.

\myindex{x86!\Instructions!PUSH}
\myindex{x86!\Instructions!JMP}
The \CALL instruction is equivalent to a\\
\INS{PUSH address\_after\_call / JMP operand} instruction pair.

\myindex{x86!\Instructions!RET}
\myindex{x86!\Instructions!POP}
\RET fetches a value from the stack and jumps to it~---that is equivalent to a \TT{POP tmp / JMP tmp} instruction pair.

\myindex{\Stack!\MLStackOverflow}
\myindex{\Recursion}
Overflowing the stack is straightforward. Just run eternal recursion:

\begin{lstlisting}[style=customc]
void f()
{
	f();
};
\end{lstlisting}

MSVC 2008 reports the problem:

\begin{lstlisting}
c:\tmp6>cl ss.cpp /Fass.asm
Microsoft (R) 32-bit C/C++ Optimizing Compiler Version 15.00.21022.08 for 80x86
Copyright (C) Microsoft Corporation.  All rights reserved.

ss.cpp
c:\tmp6\ss.cpp(4) : warning C4717: 'f' : recursive on all control paths, function will cause runtime stack overflow
\end{lstlisting}

\dots but generates the right code anyway:

\begin{lstlisting}[style=customasmx86]
?f@@YAXXZ PROC			; f
; File c:\tmp6\ss.cpp
; Line 2
	push	ebp
	mov	ebp, esp
; Line 3
	call	?f@@YAXXZ	; f
; Line 4
	pop	ebp
	ret	0
?f@@YAXXZ ENDP			; f
\end{lstlisting}

\dots Also if we turn on the compiler optimization (\TT{\Ox} option) the optimized code will not overflow the stack 
and will work \IT{correctly}\footnote{irony here} instead:

\begin{lstlisting}[style=customasmx86]
?f@@YAXXZ PROC			; f
; File c:\tmp6\ss.cpp
; Line 2
$LL3@f:
; Line 3
	jmp	SHORT $LL3@f
?f@@YAXXZ ENDP			; f
\end{lstlisting}

GCC 4.4.1 generates similar code in both cases without, however,  issuing any warning about the problem.

\myparagraph{ARM}

\myindex{ARM!\Registers!Link Register}
ARM programs also use the stack for saving return addresses, but differently.
As mentioned in \q{\HelloWorldSectionName}~(\myref{sec:hw_ARM}),
the \ac{RA} is saved to the \ac{LR} (\gls{link register}).
If one needs, however, to call another function and use the \ac{LR} register
one more time, its value has to be saved.
\myindex{Function prologue}
Usually it is saved in the function prologue.

\myindex{ARM!\Instructions!PUSH}
\myindex{ARM!\Instructions!POP}
Often, we see instructions like \INS{PUSH {R4-R7,LR}} along with this instruction in epilogue
\INS{POP {R4-R7,PC}}---thus register values to be used in the function are saved in the stack, including \ac{LR}.

\myindex{ARM!Leaf function}
Nevertheless, if a function never calls any other function, in \ac{RISC} terminology it is called a
\IT{\gls{leaf function}}\footnote{\href{http://go.yurichev.com/17064}{infocenter.arm.com/help/index.jsp?topic=/com.arm.doc.faqs/ka13785.html}}. 
As a consequence, leaf functions do not save the \ac{LR} register (because they don't modify it).
If such function is small and uses a small number of registers, it may not use the stack at all.
Thus, it is possible to call leaf functions without using the stack,
which can be faster than on older x86 machines because external RAM is not used for the stack
\footnote{Some time ago, on PDP-11 and VAX, the CALL instruction (calling other functions) was expensive; up to 50\%
of execution time might be spent on it, so it was considered that having a big number of small functions is an \gls{anti-pattern} \InSqBrackets{\TAOUP Chapter 4, Part II}.}.
This can be also useful for situations when memory for the stack is not yet allocated or not available.

Some examples of leaf functions:
\myref{ARM_leaf_example1}, \myref{ARM_leaf_example2}, 
\myref{ARM_leaf_example3}, \myref{ARM_leaf_example4}, \myref{ARM_leaf_example5},
\myref{ARM_leaf_example6}, \myref{ARM_leaf_example7}, \myref{ARM_leaf_example10}.

}
\RU{\subsubsection{Сохранение адреса возврата управления}

\myparagraph{x86}

\myindex{x86!\Instructions!CALL}
При вызове другой функции через \CALL сначала в стек записывается адрес, указывающий на место после 
инструкции \CALL, затем делается безусловный переход (почти как \TT{JMP}) на адрес, указанный в операнде.

\myindex{x86!\Instructions!PUSH}
\myindex{x86!\Instructions!JMP}
\CALL~--- это аналог пары инструкций \INS{PUSH address\_after\_call / JMP}.

\myindex{x86!\Instructions!RET}
\myindex{x86!\Instructions!POP}
\RET вытаскивает из стека значение и передает управление по этому адресу~--- 
это аналог пары инструкций \TT{POP tmp / JMP tmp}.

\myindex{\Stack!\MLStackOverflow}
\myindex{\Recursion}
Крайне легко устроить переполнение стека, запустив бесконечную рекурсию:

\begin{lstlisting}[style=customc]
void f()
{
	f();
};
\end{lstlisting}

MSVC 2008 предупреждает о проблеме:

\begin{lstlisting}
c:\tmp6>cl ss.cpp /Fass.asm
Microsoft (R) 32-bit C/C++ Optimizing Compiler Version 15.00.21022.08 for 80x86
Copyright (C) Microsoft Corporation.  All rights reserved.

ss.cpp
c:\tmp6\ss.cpp(4) : warning C4717: 'f' : recursive on all control paths, function will cause runtime stack overflow
\end{lstlisting}

\dots но, тем не менее, создает нужный код:

\begin{lstlisting}[style=customasmx86]
?f@@YAXXZ PROC			; f
; File c:\tmp6\ss.cpp
; Line 2
	push	ebp
	mov	ebp, esp
; Line 3
	call	?f@@YAXXZ	; f
; Line 4
	pop	ebp
	ret	0
?f@@YAXXZ ENDP			; f
\end{lstlisting}

\dots причем, если включить оптимизацию (\TT{\Ox}), то будет даже интереснее, без переполнения стека, 
но работать будет \IT{корректно}\footnote{здесь ирония}:

\begin{lstlisting}[style=customasmx86]
?f@@YAXXZ PROC			; f
; File c:\tmp6\ss.cpp
; Line 2
$LL3@f:
; Line 3
	jmp	SHORT $LL3@f
?f@@YAXXZ ENDP			; f
\end{lstlisting}

GCC 4.4.1 генерирует точно такой же код в обоих случаях, хотя и не предупреждает о проблеме.

\myparagraph{ARM}

\myindex{ARM!\Registers!Link Register}
Программы для ARM также используют стек для сохранения \ac{RA}, куда нужно вернуться, но несколько иначе.
Как уже упоминалось в секции \q{\HelloWorldSectionName}~(\myref{sec:hw_ARM}),
\ac{RA} записывается в регистр \ac{LR} (\gls{link register}).
Но если есть необходимость вызывать какую-то другую функцию и использовать регистр \ac{LR} ещё
раз, его значение желательно сохранить.
\myindex{Function prologue}
\myindex{ARM!\Instructions!PUSH}
\myindex{ARM!\Instructions!POP}

Обычно это происходит в прологе функции, часто мы видим там инструкцию вроде \INS{PUSH \{R4-R7,LR\}}, а в эпилоге
\INS{POP \{R4-R7,PC\}}~--- так сохраняются регистры, которые будут использоваться в текущей функции, в том числе \ac{LR}.

\myindex{ARM!Leaf function}
Тем не менее, если некая функция не вызывает никаких более функций, в терминологии \ac{RISC} она называется
\IT{\gls{leaf function}}\footnote{\href{http://go.yurichev.com/17064}{infocenter.arm.com/help/index.jsp?topic=/com.arm.doc.faqs/ka13785.html}}. 
Как следствие, \q{leaf}-функция не сохраняет регистр \ac{LR} (потому что не изменяет его).
А если эта функция небольшая, использует мало регистров, она может не использовать стек вообще.
Таким образом, в ARM возможен вызов небольших leaf-функций не используя стек.
Это может быть быстрее чем в старых x86, ведь внешняя память для стека не используется
\footnote{Когда-то, очень давно, на PDP-11 и VAX на инструкцию CALL (вызов других функций) могло тратиться
вплоть до 50\% времени (возможно из-за работы с памятью),
поэтому считалось, что много небольших функций это \glslink{anti-pattern}{анти-паттерн}
\InSqBrackets{\TAOUP Chapter 4, Part II}.}.
Либо это может быть полезным для тех ситуаций, когда память для стека ещё не выделена, либо недоступна,

Некоторые примеры таких функций:
\myref{ARM_leaf_example1}, \myref{ARM_leaf_example2}, 
\myref{ARM_leaf_example3}, \myref{ARM_leaf_example4}, \myref{ARM_leaf_example5},
\myref{ARM_leaf_example6}, \myref{ARM_leaf_example7}, \myref{ARM_leaf_example10}.

}
\DE{\subsection{Rückgabe Adresse der Funktion speichern}

\myparagraph{x86}

\myindex{x86!\Instructions!CALL}
Wenn man eine Funktion mit der \CALL Instruktion aufruft, wird die Adresse direkt nach der
\CALL Instruktion auf dem Stack gespeichert und der unbedingte jump wird ausgeführt.

\myindex{x86!\Instructions!PUSH}
\myindex{x86!\Instructions!JMP}
Die \CALL Instruktion ist äquivalent zu dem \INS{PUSH address\_after\_call / JMP operand} Instruktions paar.

\myindex{x86!\Instructions!RET}
\myindex{x86!\Instructions!POP}
\RET ruft die Rückkehr Adresse vom Stack ab und springt zu dieser~---was äquivalent zu einem \TT{POP tmp / JMP tmp} Instruktions
paar ist.

\myindex{\Stack!\MLStackOverflow}
\myindex{\Recursion}

Den Stack zum überlaufen zu bringen ist recht einfach, einfach eine 
endlos rekursive Funktion Aufrufen:


\begin{lstlisting}[style=customc]
void f()
{
	f();
};
\end{lstlisting}


MSVC 2008 hat eine Erkennung für das Problem:


\begin{lstlisting}
c:\tmp6>cl ss.cpp /Fass.asm
Microsoft (R) 32-bit C/C++ Optimizing Compiler Version 15.00.21022.08 for 80x86
Copyright (C) Microsoft Corporation.  All rights reserved.

ss.cpp
c:\tmp6\ss.cpp(4) : warning C4717: 'f' : recursive on all control paths, function will cause runtime stack overflow
\end{lstlisting}

\dots aber der Compiler erzeugt den Code trotzdem:

\begin{lstlisting}[style=customasmx86]
?f@@YAXXZ PROC			; f
; File c:\tmp6\ss.cpp
; Line 2
	push	ebp
	mov	ebp, esp
; Line 3
	call	?f@@YAXXZ	; f
; Line 4
	pop	ebp
	ret	0
?f@@YAXXZ ENDP			; f
\end{lstlisting}

\dots Auch wenn wir die Compiler Optimierungen einschalten (\TT{/0x} Option) wird der optimierte Code nicht
den Stack zum überlaufen bringen. Stattdessen wird der Code \IT{korrekt}\footnote{Ironie hier} ausgeführt: 

\begin{lstlisting}[style=customasmx86]
?f@@YAXXZ PROC			; f
; File c:\tmp6\ss.cpp
; Line 2
$LL3@f:
; Line 3
	jmp	SHORT $LL3@f
?f@@YAXXZ ENDP			; f
\end{lstlisting}


GCC 4.4.1 generiert vergleichbaren Code in beiden Fällen, jedoch ohne über das Overflow Problem zu warnen.


\myparagraph{ARM}

\myindex{ARM!\Registers!Link Register}

ARM Programme benutzen den Stack um Rücksprung Adressen zu speichern, aber anders.
Wie bereits erwähnt in \q{\HelloWorldSectionName}~(\myref{sec:hw_ARM}),
wird der \ac{RA} Wert im \ac{LR} (\gls{link register}) gespeichert.
Wenn nun eine andere Funktion aufgerufen werden muss und auf das \ac{LR} Register 
zu greift, muss der aktuelle Wert im Register irgendwo gespeichert werden.

\myindex{Funktion Prologe}
Normal wird der Wert im Funktion Prolog gespeichert.

\myindex{ARM!\Instructions!PUSH}
\myindex{ARM!\Instructions!POP}

Oft sieht man Instruktionen wie z.B \INS{PUSH {R4-R7,LR}} zusammen mit dieser Instruktion im 
Epilog \INS{POP {R4-R7,PC}}---Somit werden Werte die in den Funktionen benötigt werden auf dem 
Stack gespeichert, inklusive \ac{LR}.

\myindex{ARM!Leaf Funktion}
Wenn eine Funktion nie eine andere Funktion aufruft, nennt man das in der \ac{RISC} Terminologie eine
\IT{\glslink{leaf function}{leaf Funktion}}\footnote{\href{http://go.yurichev.com/17064}{infocenter.arm.com/help/index.jsp?topic=/com.arm.doc.faqs/ka13785.html}}.  % <-- attention could be a compilier bug
Als Konsequenz ergibt sich, das leaf Funktionen nicht das \ac{LR} Register speichern (da sie es nicht modifizieren).
Wenn solche Funktionen klein sind und nur eine geringe Anzahl an Registern benutzt, ist es möglich das der Stack
gar nicht benutzt wird. Es ist also möglich leaf Funktionen zu benutzen ohne den Stack zurück zu greifen, die Ausführung
ist hier schneller als auf älteren x86 Maschinen weil kein externer RAM für den Stack benutzt wird 
\footnote{Bis vor einer weile war es sehr teuer auf PDP-11 und VAX Maschinen die CALL Instruktion zu benutzen; bis zu 50\%
der Rechenzeit wurde allein für diese Instruktion verschwendet, man hat dabei festgestellt das eine große Anzahl an kleinen
Funktionen zu haben ein \gls{anti-pattern} \InSqBrackets{\TAOUP Chapter 4, Part II}.} ist.
Diese Eigenschaft kann nützlich sein wenn der Speicher für den Stack noch nicht alloziert oder verfügbar ist.

Ein paar Beispiele für leaf Funktionen:

\myref{ARM_leaf_example1}, \myref{ARM_leaf_example2}, 
\myref{ARM_leaf_example3}, \myref{ARM_leaf_example4}, \myref{ARM_leaf_example5},
\myref{ARM_leaf_example6}, \myref{ARM_leaf_example7}, \myref{ARM_leaf_example10}.

}
\FR{\subsubsection{Sauvegarder l'adresse de retour de la fonction}

\myparagraph{x86}

\myindex{x86!\Instructions!CALL}
Lorsque l'on appelle une fonction avec une instruction \CALL, l'adresse du point
exactement après cette dernière est sauvegardée sur la pile et un saut inconditionnel
à l'adresse de l'opérande \CALL est exécuté.

\myindex{x86!\Instructions!PUSH}
\myindex{x86!\Instructions!JMP}
L'instruction \CALL est équivalente à la\\
paire d'instructions \INS{PUSH address\_after\_call / JMP operand}.

\myindex{x86!\Instructions!RET}
\myindex{x86!\Instructions!POP}
\RET va chercher une valeur sur la pile et y saute~---ce qui est équivalent à
la paire d'instructions \TT{POP tmp / JMP tmp}.

\myindex{\Stack!\MLStackOverflow}
\myindex{\Recursion}
Déborder de la pile est très facile. Il suffit de lancer une récursion éternelle:

\begin{lstlisting}[style=customc]
void f()
{
	f();
};
\end{lstlisting}

MSVC 2008 signale le problème:

\begin{lstlisting}
c:\tmp6>cl ss.cpp /Fass.asm
Microsoft (R) 32-bit C/C++ Optimizing Compiler Version 15.00.21022.08 for 80x86
Copyright (C) Microsoft Corporation.  All rights reserved.

ss.cpp
c:\tmp6\ss.cpp(4) : warning C4717: 'f' : recursive on all control paths, function will cause runtime stack overflow
\end{lstlisting}

\dots mais génère tout de même le code correspondant:

\begin{lstlisting}[style=customasmx86]
?f@@YAXXZ PROC			; f
; File c:\tmp6\ss.cpp
; Line 2
	push	ebp
	mov	ebp, esp
; Line 3
	call	?f@@YAXXZ	; f
; Line 4
	pop	ebp
	ret	0
?f@@YAXXZ ENDP			; f
\end{lstlisting}

\dots Si nous utilisons l'option d'optimisation du compilateur (option \TT{\Ox})
le code optimisé ne va pas déborder de la pile et au lieu de cela va fonctionner
\IT{correctemment}\footnote{ironique ici}:

\begin{lstlisting}[style=customasmx86]
?f@@YAXXZ PROC			; f
; File c:\tmp6\ss.cpp
; Line 2
$LL3@f:
; Line 3
	jmp	SHORT $LL3@f
?f@@YAXXZ ENDP			; f
\end{lstlisting}

GCC 4.4.1 génère un code similaire dans les deux cas, sans, toutefois émettre
d'avertissement à propos de ce problème.

\myparagraph{ARM}

\myindex{ARM!\Registers!Link Register}
Les programmes ARM utilisent également la pile pour sauver les adresses de retour,
mais différemment.
Comme mentionné dans \q{\HelloWorldSectionName}~(\myref{sec:hw_ARM}),
\ac{RA} est sauvegardé dans \ac{LR} (\gls{link register}).
Si l'on a toutefois besoin d'appeler une autre fonction et d'utiliser le registre
\ac{LR} une fois de plus, sa valeur doit être sauvegardée.
\myindex{Function prologue}
Usuellement, cela se fait dans le prologue de la fonction.

\myindex{ARM!\Instructions!PUSH}
\myindex{ARM!\Instructions!POP}
Souvent, nous voyons des instructions comme \INS{PUSH {R4-R7,LR}} en même temps
que cette instruction dans le prologue \INS{POP {R4-R7,PC}}---ces registres qui
sont utilisés dans la fonction sont sauvegardés sur la pile, \ac{LR} inclus.

\myindex{ARM!Fonction leaf} % FIXME traduire avec feuille ?
Néanmoins, si une fonction n'appelle jamais d'autre fonction, dans la terminologie
\ac{RISC} elle est appelée \IT{\glslink{leaf function}{fonction leaf}}\footnote{\href{http://go.yurichev.com/17064}{infocenter.arm.com/help/index.jsp?topic=/com.arm.doc.faqs/ka13785.html}}.
Ceci a comme conséquence que les fonctions leaf ne sauvegardent pas le registre
\ac{LR} (car elles ne le modifient pas).
Si une telle fonction est petite et utilise un petit nombre de registres, elle
peut ne pas utiliser du tout la pile.
Ainsi, il est possible d'appeler des fonctions leaf sans utiliser la pile.
Ce qui peut être plus rapide sur des vieilles machines x86 car la mémoire externe
n'est pas utilisée pour la pile
\footnote{Il y a quelques temps, sur PDP-11 et VAX, l'instruction CALL (appel d'autres fonctions) était coûteux; jusqu'à 50\%
du temps d'exécution pouvait être passé à ça, il était donc considèré qu'avoir un grand nombre de petite fonction était un \gls{anti-pattern} \InSqBrackets{\TAOUP Chapter 4, Part II}.}.
Cela peut être utile pour des situations où la mémoire pour la pile n'est pas
encore allouée ou disponible.

Quelques exemples de fonctions leaf:
\myref{ARM_leaf_example1}, \myref{ARM_leaf_example2},
\myref{ARM_leaf_example3}, \myref{ARM_leaf_example4}, \myref{ARM_leaf_example5},
\myref{ARM_leaf_example6}, \myref{ARM_leaf_example7}, \myref{ARM_leaf_example10}.

}
\PTBR{\subsubsection{Salvar o endereço de retorno de uma função}

\myparagraph{x86}

\myindex{x86!\Instructions!CALL}
Quando você chama outra função utilizando a instrução CALL, o endereço do ponto exato onde a 
instrução \CALL se encontra é salvo na pilha e então um jump incondicional para o endereço no operando de \CALL é executado.

\myindex{x86!\Instructions!PUSH}
\myindex{x86!\Instructions!JMP}
A instrução \CALL é equivalente a usar o par de instruções \TT{PUSH endereço\_depois\_chamada / JMP}.

\myindex{x86!\Instructions!RET}
\myindex{x86!\Instructions!POP}
\RET pega um valor da pilha e usa um jump para ele --- isso é equivalente a usar \INS{POP tmp / JMP tmp}.

\myindex{\Stack!\MLStackOverflow}
\myindex{\Recursion}
Estourar uma stack é fácil. Só execute alguma recursão externa:

\begin{lstlisting}[style=customc]
void f()
{
	f();
};
\end{lstlisting}

O compilador MSVC 2008 informa o problema:

\begin{lstlisting}
c:\tmp6>cl ss.cpp /Fass.asm
Microsoft (R) 32-bit C/C++ Optimizing Compiler Version 15.00.21022.08 for 80x86
Copyright (C) Microsoft Corporation.  All rights reserved.

ss.cpp
c:\tmp6\ss.cpp(4) : warning C4717: 'f' : recursive on all control paths, function will cause runtime stack overflow
\end{lstlisting}

\dots mas gera o código de qualquer maneira:

\begin{lstlisting}[style=customasmx86]
?f@@YAXXZ PROC			; f
; File c:\tmp6\ss.cpp
; Line 2
	push	ebp
	mov	ebp, esp
; Line 3
	call	?f@@YAXXZ	; f
; Line 4
	pop	ebp
	ret	0
?f@@YAXXZ ENDP			; f
\end{lstlisting}

\dots também, se ativarmos a otimização do compilador (opção \TT{/Ox}) 
o código otimizado não vai estourar a pilha e funcionará \IT{corretamente} \footnote{ironia aqui}:

\begin{lstlisting}[style=customasmx86]
?f@@YAXXZ PROC			; f
; File c:\tmp6\ss.cpp
; Line 2
$LL3@f:
; Line 3
	jmp	SHORT $LL3@f
?f@@YAXXZ ENDP			; f
\end{lstlisting}

\PTBRph{}

}
\ITA{\subsubsection{Salvare l'indirizzo di ritorno della funzione}

\myparagraph{x86}

\myindex{x86!\Instructions!CALL}
Quando si chiama una funzione con l'istruzione \CALL, l'indirizzo del punto esattamente dopo la \CALL viene salvato nello stack, e successivamente
viene eseguito un jump non condizionale all'indirizzo dell'operando di \CALL.

\myindex{x86!\Instructions!PUSH}
\myindex{x86!\Instructions!JMP}
L'istruzione \CALL e' equivalente alla coppia di istruzioni \INS{PUSH indirizzo\_dopo\_call / JMP operando}.

\myindex{x86!\Instructions!RET}
\myindex{x86!\Instructions!POP}
\RET preleva un valore dallo stack e effettua un jump ad esso~--- cio' equivale alla coppia di istruzioni \TT{POP tmp / JMP tmp}.

\myindex{\Stack!\MLStackOverflow}
\myindex{\Recursion}

Riempire lo stack fino allo straripamento e' semplicissimo. Basta ricorrere alla ricorsione eterna:

\begin{lstlisting}[style=customc]
void f()
{
	f();
};
\end{lstlisting}

MSVC 2008 riporta il problema:

\begin{lstlisting}
c:\tmp6>cl ss.cpp /Fass.asm
Microsoft (R) 32-bit C/C++ Optimizing Compiler Version 15.00.21022.08 for 80x86
Copyright (C) Microsoft Corporation.  All rights reserved.

ss.cpp
c:\tmp6\ss.cpp(4) : warning C4717: 'f' : recursive on all control paths, function will cause runtime stack overflow
\end{lstlisting}

\dots ma genera in ogni caso il codice correttamente:

\begin{lstlisting}[style=customasmx86]
?f@@YAXXZ PROC			; f
; File c:\tmp6\ss.cpp
; Line 2
	push	ebp
	mov	ebp, esp
; Line 3
	call	?f@@YAXXZ	; f
; Line 4
	pop	ebp
	ret	0
?f@@YAXXZ ENDP			; f
\end{lstlisting}

\dots Se attiviamo le ottimizzazioni del compilatore (\TT{\Ox} option) il codice ottimizzato non causera' overflow dello stack 
e funzionera' invece \IT{correttamente}\footnote{sarcasmo, si fa per dire}:

\begin{lstlisting}[style=customasmx86]
?f@@YAXXZ PROC			; f
; File c:\tmp6\ss.cpp
; Line 2
$LL3@f:
; Line 3
	jmp	SHORT $LL3@f
?f@@YAXXZ ENDP			; f
\end{lstlisting}

GCC 4.4.1 genera codice simile in antrambi i casi, senza avvertire del problema.

\myparagraph{ARM}

\myindex{ARM!\Registers!Link Register}
Anche i programmi ARM usano lo stack per salvare gli indirizzi di ritorno, ma lo fanno in maniera diversa.
Come detto in \q{\HelloWorldSectionName}~(\myref{sec:hw_ARM}),
As mentioned in 
il \ac{RA} viene salvato nel \ac{LR} (\gls{link register}).
Se si presenta comunque la necessita' di chiamare un'altra funzione ed usare il registro \ac{LR} ancora una volta, 
il suo valore deve essere salvato.
\myindex{Function prologue}
Solitamente questo valore e' slvato nel preambolo della funzione.

\myindex{ARM!\Instructions!PUSH}
\myindex{ARM!\Instructions!POP}
Spesso vediamo istruzioni come \INS{PUSH {R4-R7,LR}} insieme ad isrtuzioni nell'epilogo come 
\INS{POP {R4-R7,PC}}---percio' i valori dei registri che saranno usati nella funzione vengono salvati nello stack, incluso \ac{LR}.

\myindex{ARM!Leaf function}
Ciononostante, se una funzione non chiama al suo interno nessun'altra funzione, in terminologia \ac{RISC} e' detta 
\IT{\gls{leaf function}}, o funzione foglia.\footnote{\href{http://go.yurichev.com/17064}{infocenter.arm.com/help/index.jsp?topic=/com.arm.doc.faqs/ka13785.html}}. 
Di conseguenza, le leaf functions non salvano il registro \ac{LR} register (perche' difatti non lo modificano).
Se una simile funzione e' molto breve e usa un piccolo numero di registri, potrebbe non usare del tutto lo stack. 
E' quindi possible chiamare le leaf functions senza usare lo stack, cosa che puo' essere piu' veloce che sulle macchine x86 perche' ;a RA< esterna non viene usata per lo stack
\footnote{Tempo fa, su PDP-11 and VAX, l'istruzione CALL instruction (chiamare altre funzioni) era costosa; poteva richiedere fino al 50\%
del tempo di esecuzione, ed era quindi consuetudine pensare che avere un grande numero di piccole funzioni fosee un \gls{anti-pattern} \InSqBrackets{\TAOUP Chapter 4, Part II}.}.
Lo stesso principio puo' tornare utile quando la memoria per lo stack non e' stata ancora allocata o non e' disponibile.

Alcuni esempi di funzioni foglia:
\myref{ARM_leaf_example1}, \myref{ARM_leaf_example2}, 
\myref{ARM_leaf_example3}, \myref{ARM_leaf_example4}, \myref{ARM_leaf_example5},
\myref{ARM_leaf_example6}, \myref{ARM_leaf_example7}, \myref{ARM_leaf_example10}.
}

\EN{\subsubsection{Passing function arguments}

The most popular way to pass parameters in x86 is called \q{cdecl}:

\begin{lstlisting}[style=customasmx86]
push arg3
push arg2
push arg1
call f
add esp, 12 ; 4*3=12
\end{lstlisting}

\Gls{callee} functions get their arguments via the stack pointer.

Therefore, this is how the argument values are located in the stack before the execution of the \ttf{} function's very first instruction:

\begin{center}
\begin{tabular}{ | l | l | }
\hline
ESP & return address \\
\hline
ESP+4 & \argument \#1, \MarkedInIDAAs{} \TT{arg\_0} \\
\hline
ESP+8 & \argument \#2, \MarkedInIDAAs{} \TT{arg\_4} \\
\hline
ESP+0xC & \argument \#3, \MarkedInIDAAs{} \TT{arg\_8} \\
\hline
\dots & \dots \\
\hline
\end{tabular}
\end{center}

For more information on other calling conventions see also section~(\myref{sec:callingconventions}).

\par
By the way, the \gls{callee} function does not have any information about how many arguments were passed.
C functions with a variable number of arguments (like \printf) determine their number using format string specifiers (which begin with the \% symbol).

If we write something like:

\begin{lstlisting}
printf("%d %d %d", 1234);
\end{lstlisting}

\printf will print 1234, and then two random numbers\footnote{Not random in strict sense, but rather unpredictable: \myref{noise_in_stack}}, which were lying next to it in the stack.

\par
That's why it is not very important how we declare the \main function: as \main, \\
\TT{main(int argc, char *argv[])} or \TT{main(int argc, char *argv[], char *envp[])}.

In fact, the \ac{CRT}-code is calling \main roughly as:
	
\begin{lstlisting}[style=customasmx86]
push envp
push argv
push argc
call main
...
\end{lstlisting}

If you declare \main as \main without arguments, they are, nevertheless, still present in the stack, but are not used.
If you declare \main as  \TT{main(int argc, char *argv[])},
you will be able to use first two arguments, and the third will remain \q{invisible} for your function.
Even more, it is possible to declare \TT{main(int argc)}, and it will work.

\myparagraph{Alternative ways of passing arguments}

It is worth noting that nothing obliges programmers to pass arguments through the stack. It is not a requirement.
One could implement any other method without using the stack at all.

A somewhat popular way among assembly language newbies is to pass arguments via global variables, like:

\lstinputlisting[caption=Assembly code,style=customasmx86]{patterns/02_stack/global_args.asm}

But this method has obvious drawback: \IT{do\_something()} function cannot call itself recursively (or via another function),
because it has to zap its own arguments.
The same story with local variables: if you hold them in global variables, the function couldn't call itself.
And this is also not thread-safe
\footnote{Correctly implemented, each thread would have its own stack with its own arguments/variables.}.
A method to store such information in stack makes this easier---it can hold as many function arguments and/or values,
as much space it has.

\InSqBrackets{\TAOCPvolI{}, 189} mentions even weirder schemes particularly convenient on IBM System/360.

\myindex{MS-DOS}
\myindex{x86!\Instructions!INT}

MS-DOS had a way of passing all function arguments via registers, for example, this is piece of
code for ancient 16-bit MS-DOS prints ``Hello, world!'':

\begin{lstlisting}[style=customasmx86]
mov  dx, msg      ; address of message
mov  ah, 9        ; 9 means "print string" function
int  21h          ; DOS "syscall"

mov  ah, 4ch      ; "terminate program" function
int  21h          ; DOS "syscall"

msg  db 'Hello, World!\$'
\end{lstlisting}

\myindex{fastcall}
This is quite similar to \myref{fastcall} method.
And also it's very similar to calling syscalls in Linux (\myref{linux_syscall}) and Windows.

\myindex{x86!\Flags!CF}
If a MS-DOS function is going to return a boolean value (i.e., single bit, usually indicating error state),
\TT{CF} flag was often used.

For example:

\begin{lstlisting}[style=customasmx86]
mov ah, 3ch       ; create file
lea dx, filename
mov cl, 1
int 21h
jc  error
mov file_handle, ax
...
error:
...
\end{lstlisting}

In case of error, \TT{CF} flag is raised. Otherwise, handle of newly created file is returned via \TT{AX}.

This method is still used by assembly language programmers.
In Windows Research Kernel source code (which is quite similar to Windows 2003) we can find something like this
(file \IT{base/ntos/ke/i386/cpu.asm}):

\begin{lstlisting}[style=customasmx86]
        public  Get386Stepping
Get386Stepping  proc

        call    MultiplyTest            ; Perform multiplication test
        jnc     short G3s00             ; if nc, muttest is ok
        mov     ax, 0
        ret
G3s00:
        call    Check386B0              ; Check for B0 stepping
        jnc     short G3s05             ; if nc, it's B1/later
        mov     ax, 100h                ; It is B0/earlier stepping
        ret

G3s05:
        call    Check386D1              ; Check for D1 stepping
        jc      short G3s10             ; if c, it is NOT D1
        mov     ax, 301h                ; It is D1/later stepping
        ret

G3s10:
        mov     ax, 101h                ; assume it is B1 stepping
        ret

	...

MultiplyTest    proc

        xor     cx,cx                   ; 64K times is a nice round number
mlt00:  push    cx
        call    Multiply                ; does this chip's multiply work?
        pop     cx
        jc      short mltx              ; if c, No, exit
        loop    mlt00                   ; if nc, YEs, loop to try again
        clc
mltx:
        ret

MultiplyTest    endp
\end{lstlisting}

}
\RU{\subsubsection{Передача параметров функции}

Самый распространенный способ передачи параметров в x86 называется \q{cdecl}:

\begin{lstlisting}[style=customasmx86]
push arg3
push arg2
push arg1
call f
add esp, 12 ; 4*3=12
\end{lstlisting}

Вызываемая функция получает свои параметры также через указатель стека.

Следовательно, так расположены значения в стеке перед исполнением самой первой инструкции функции \ttf{}:

\begin{center}
\begin{tabular}{ | l | l | }
\hline
ESP & адрес возврата \\
\hline
ESP+4 & \argument \#1, \MarkedInIDAAs{} \TT{arg\_0} \\
\hline
ESP+8 & \argument \#2, \MarkedInIDAAs{} \TT{arg\_4} \\
\hline
ESP+0xC & \argument \#3, \MarkedInIDAAs{} \TT{arg\_8} \\
\hline
\dots & \dots \\
\hline
\end{tabular}
\end{center}

См. также в соответствующем разделе о других способах передачи аргументов через стек~(\myref{sec:callingconventions}).

\par Кстати, вызываемая функция не имеет информации о количестве переданных ей аргументов.
Функции Си с переменным количеством аргументов (как \printf) определяют их количество по спецификаторам строки формата (начинающиеся со знака \%).

Если написать что-то вроде:

\begin{lstlisting}
printf("%d %d %d", 1234);
\end{lstlisting}

\printf выведет 1234, затем ещё два случайных числа\footnote{В строгом смысле, они не случайны, скорее, непредсказуемы: \myref{noise_in_stack}}, которые волею случая оказались в стеке рядом.

\par
Вот почему не так уж и важно, как объявлять функцию \main{}:\\
как \main{}, \TT{main(int argc, char *argv[])}\\
либо \TT{main(int argc, char *argv[], char *envp[])}.

В реальности, \ac{CRT}-код вызывает \main примерно так:
	
\begin{lstlisting}[style=customasmx86]
push envp
push argv
push argc
call main
...
\end{lstlisting}

Если вы объявляете \main без аргументов, они, тем не менее, присутствуют в стеке, но не используются.
Если вы объявите \main как \TT{main(int argc, char *argv[])}, 
вы можете использовать два первых аргумента, а третий останется для вашей функции \q{невидимым}.
Более того, можно даже объявить \TT{main(int argc)}, и это будет работать.

\myparagraph{Альтернативные способы передачи аргументов}

Важно отметить, что, в общем, никто не заставляет программистов передавать параметры именно через стек, это не является требованием к исполняемому коду.
Вы можете делать это совершенно иначе, не используя стек вообще.

В каком-то смысле, популярный метод среди начинающих использовать язык ассемблера,
это передавать аргументы в глобальных переменных, например:

\lstinputlisting[caption=Код на ассемблере,style=customasmx86]{patterns/02_stack/global_args.asm}

Но у этого метода есть очевидный недостаток: ф-ция \IT{do\_something()} не сможет вызвать саму себя рекурсивно (либо, через
какую-то стороннюю ф-цию),
потому что тогда придется затереть свои собственные аргументы.
Та же история с локальными переменными: если хранить их в глобальных переменных, ф-ция не сможет вызывать сама себя.
К тому же, этот метод не безопасный для мультитредовой среды\footnote{При корректной реализации,
каждый тред будет иметь свой собственный стек со своими аргументами/переменными.}.
Способ хранения подобной информации в стеке заметно всё упрощает ---
он может хранить столько аргументов ф-ций и/или значений вообще,
сколько в нем есть места.

В \InSqBrackets{\TAOCPvolI{}, 189} можно прочитать про еще более странные схемы передачи аргументов,
которые были очень удобны на IBM System/360.

\myindex{MS-DOS}
\myindex{x86!\Instructions!INT}

В MS-DOS был метод передачи аргументов через регистры, например, этот фрагмент кода для древней 16-битной MS-DOS
выводит ``Hello, world!'':

\begin{lstlisting}[style=customasmx86]
mov  dx, msg      ; адрес сообщения
mov  ah, 9        ; §9 означает ф-цию "вывод строки"§
int  21h          ; DOS "syscall"

mov  ah, 4ch      ; §ф-ция "закончить программу"§
int  21h          ; DOS "syscall"

msg  db 'Hello, World!\$'
\end{lstlisting}

\myindex{fastcall}
Это очень похоже на метод \myref{fastcall}.
И еще на метод вызовов сисколлов в Linux (\myref{linux_syscall}) и Windows.

\myindex{x86!\Flags!CF}
Если ф-ция в MS-DOS возвращает булево значение (т.е., один бит, обычно сигнализирующий об ошибке),
часто использовался флаг \TT{CF}.

Например:

\begin{lstlisting}[style=customasmx86]
mov ah, 3ch       ; создать файл
lea dx, filename
mov cl, 1
int 21h
jc  error
mov file_handle, ax
...
error:
...
\end{lstlisting}

В случае ошибки, флаг \TT{CF} будет выставлен.
Иначе, хэндл только что созданного файла возвращается в \TT{AX}.

Этот метод до сих пор используется программистами на ассемблере.
В исходных кодах Windows Research Kernel (который очень похож на Windows 2003) мы можем найти такое\\
(файл \IT{base/ntos/ke/i386/cpu.asm}):

\begin{lstlisting}[style=customasmx86]
        public  Get386Stepping
Get386Stepping  proc

        call    MultiplyTest            ; Perform multiplication test
        jnc     short G3s00             ; if nc, muttest is ok
        mov     ax, 0
        ret
G3s00:
        call    Check386B0              ; Check for B0 stepping
        jnc     short G3s05             ; if nc, it's B1/later
        mov     ax, 100h                ; It is B0/earlier stepping
        ret

G3s05:
        call    Check386D1              ; Check for D1 stepping
        jc      short G3s10             ; if c, it is NOT D1
        mov     ax, 301h                ; It is D1/later stepping
        ret

G3s10:
        mov     ax, 101h                ; assume it is B1 stepping
        ret

	...

MultiplyTest    proc

        xor     cx,cx                   ; 64K times is a nice round number
mlt00:  push    cx
        call    Multiply                ; does this chip's multiply work?
        pop     cx
        jc      short mltx              ; if c, No, exit
        loop    mlt00                   ; if nc, YEs, loop to try again
        clc
mltx:
        ret

MultiplyTest    endp
\end{lstlisting}


}
\PTBR{\subsubsection{Passando argumento de funções}

A maneira mais comum de se passar parâmetros no x86 é chamada \q{cdecl}:

\begin{lstlisting}[style=customasmx86]
push arg3
push arg2
push arg1
call f
add esp, 12 ; 4*3=12
\end{lstlisting}

Uma função chamada recebe seus argumentos pelo ponteiro da pilha.

Portanto, é assim que os valores dos argumentos são alocados na pilha antes da execução das primeiras intruções da função \ttf{}:

\begin{center}
\begin{tabular}{ | l | l | }
\hline
ESP & endereço de retorno \\
\hline
ESP+4 & \argument \#1, \MarkedInIDAAs{} \TT{arg\_0} \\
\hline
ESP+8 & \argument \#2, \MarkedInIDAAs{} \TT{arg\_4} \\
\hline
ESP+0xC & \argument \#3, \MarkedInIDAAs{} \TT{arg\_8} \\
\hline
\dots & \dots \\
\hline
\end{tabular}
\end{center}

\PTBRph{}

Vale ressaltar que nada obriga o programador a passar os argumentos pela pilha. Não é um requerimento.
Você pode implementar qualquer outro método usando a pilha da maneira que desejar.

Por exemplo, é possível alocar um espaço para argumentos na \gls{heap}, preencher e passar para a função via um ponteiro para esse bloco no registrador \EAX{}.
% TBT: \InSqBrackets{\TAOCPvolI{}, 189} mentions even weirder schemes particularly convenient on IBM System/360.

Isso vai funcionar, entretando, é de senso comum no x86 e ARM a usar a pilha para esse fim.

\par
A propósito, a função chamada não tem nenhuma informação sobre quantos argumentos foram passados.
Funções em C com um número variável de argumentos (como \printf) determina seu número usando formatações específicas de string (que começam com o símbolo \%).

Se nós escrevermos algo como:

\begin{lstlisting}
printf("%d %d %d", 1234);
\end{lstlisting}

\printf vai mostrar 1234, e então dois números aleatórios, que estariam próximos a stack\footnote{\ac{TBT}}.

\par
É por isso que não é muito importante como declaramos a função \main{}: como \main{}, \TT{main(int argc, char *argv[])} ou \TT{main(int argc, char *argv[], char *envp[])}.

Na verdade, o código da C Runtime Library está chamando grosseiramente \main{} dessa maneira:
	
\begin{lstlisting}[style=customasmx86]
push envp
push argv
push argc
call main
...
\end{lstlisting}

Se você declarar \main como \main sem argumentos, mesmo assim eles ainda estarão presentes na pilha, mas não são usados.
Se você declarar \main como \TT{main(int argc, char *argv[])}, você será capaz de utilizar os primeiros dois argumentos e o terceiro vai continuar \q{invisível} para a sua função.
Da mesma maneira, é possível declarar a \main como \TT{main(int argc)} e ainda assim vai funcionar.

}
\DE{\subsubsection{Funktion Argumente übergeben}

Der übliche weg Argumente in x86 zu übergeben ist die \q{cdecl} Methode:

\begin{lstlisting}[style=customasmx86]
push arg3
push arg2
push arg1
call f
add esp, 12 ; 4*3=12
\end{lstlisting}

Die \Gls{callee} Funktionen bekommen ihre Argumente über den Stackpointer. 

So werden die Argumente auf dem Stack gefunden, noch vor der Ausführung der ersten Instruktion der \ttf{} Funktion:

\begin{center}
\begin{tabular}{ | l | l | }
\hline
ESP & return address \\
\hline
ESP+4 & \argument \#1, \MarkedInIDAAs{} \TT{arg\_0} \\
\hline
ESP+8 & \argument \#2, \MarkedInIDAAs{} \TT{arg\_4} \\
\hline
ESP+0xC & \argument \#3, \MarkedInIDAAs{} \TT{arg\_8} \\
\hline
\dots & \dots \\
\hline
\end{tabular}
\end{center}


Für mehr Informationen über andere Aufrufs Konventionen siehe Sektion:~(\myref{sec:callingconventions}).

\par
Übrigens, die \gls{callee} Funktion hat keine Informationen wie viele Argumente übergeben wurden.
C Funktionen mit einer variablen Anzahl an Argumenten (wie z.B \printf) errechnen die zahl der Argumente anhand der 
Formatstring spezifizier-er (alle spezifizier-er die mit dem \% beginnen).

Wenn wir etwas schreiben wie z.B:

\begin{lstlisting}
printf("%d %d %d", 1234);
\end{lstlisting}

\printf wird die Zahlen 1234 ausgeben und dann zwei Zufalls zahlen, welche direkt neben dem Zahlen
Werten auf dem Stack lagen\footnote{\ac{TBT}}.

\par
Das ist auch der Grund warum es nicht wichtig ist wie die \main Funktion definiert ist: Als \main, \\
\TT{main(int argc, char *argv[])} oder \TT{main(int argc, char *argv[], char *envp[])}.

Tatsächlich ruf der \ac{CRT}-Code die \main Funktion um Grunde so auf:
	
\begin{lstlisting}[style=customasmx86]
push envp
push argv
push argc
call main
...
\end{lstlisting}

Wenn man \main als \main Funktion ohne Argumente definiert, dann liegen sie trotzdem auf dem Stack auch wenn sie 
nicht benutzt werden. Wenn man \main als \TT{main(int argc, char *argv[])}, definiert kann man auf die ersten beiden
Argumente der Funktion zugreifen, das dritte bleibt aber weiterhin ``Unsichtbar'' für andere Funktionen.
Es ist aber auch u.a möglich die Main Funktion als \TT{main(int argc)} schreiben und sie wird noch immer funktionieren.

\myparagraph{Alternative Wege Argumente zu übergeben}

Es sollte bemerkt werden das nichts einen Programmierer dazu zwingt Argumente über den Stack zu übergeben. Das ist
keine generelle Anforderung. Jemand könnte auch einfach eine andere Methode implementieren ohne den Stack überhaupt zu benutzen.

Ein ziemlich beliebter Weg Argumente zu übergeben unter Assembler Neulingen ist über globale Variablen wie z.B:

\lstinputlisting[caption=Assembly code]{patterns/02_stack/global_args.asm}

Aber diese Methode hat Nachteile: Die \IT{do\_something()} Funktion kann sich selbst nicht rekursiv aufrufen (aber auch keine andere Funktion),
weil sie ihre eigenen Argumente löschen muss.
Die gleiche Geschichte mit lokalen Variablen: Wenn die Werte in globalen Variablen gespeichert sind, kann die Funktion sich nicht selbst aufrufen.
Und das bedeutet wiederum das die Funktion nicht thread-Safe ist.
\footnote{Korrekt implementiert, hat jeder Thread seinen eigenen Stack und seine eigenen Argumente/Variablen}.
Eine Methode solche Informationen auf dem Stack zu speichern macht die Dinge einfacher--- Der Stack kann so viele Funktion Arguemente und/oder Werte speichern,
so viel Speicher wie der Computer hat.

\InSqBrackets{\TAOCPvolI{}, 189} nennt sogar noch verrückter Methoden die speziell auf IBM System/360 benutzt werden.

\myindex{MS-DOS}
\myindex{x86!\Instructions!INT}

Auf MS-DOS gab es einen Weg Funktion Argumente über Register zu übergeben, zum Beispiel dies 
ist ein Stück Code einer veralteten 16-Bit MS-DOS ``Hallo, Welt!'' Funktion:

\begin{lstlisting}[style=customasmx86]
mov  dx, msg      ; Adresse der Naricht
mov  ah, 9        ; 9 bedeutet ``print string''
int  21h          ; DOS "syscall"

mov  ah, 4ch      ; ``Terminiere Programm'' Funktion
int  21h          ; DOS "syscall"

msg  db 'Hello, World!\$' 
\end{lstlisting}

\myindex{fastcall}
Diese Methode ist der \myref{fastcall} Methode sehr ähnlich. Sie ähnelt aber auch der Methode
wie man auf Linux (\myref{linux_syscall}) und Windows syscalls ausführt.

\myindex{x86!\Flags!CF}
Wenn eine MS-DOS Funktion einen Bool'schen Wert zurück gibt (z.B., Single Bit bedeutet ein Fehler ist aufgetreten), wird dafür das \TT{CF} Flag benutzt.

Zum Beispiel:

\begin{lstlisting}[style=customasmx86]
mov ah, 3ch       ; create file
lea dx, filename
mov cl, 1
int 21h
jc  error
mov file_handle, ax
...
error:
...
\end{lstlisting}

Im Falle eines Fehlers, wird das \TT{CF} Flag gesetzt. Anderenfalls wird ein handle für die neu erstellte Datei über \TT{AX} zurück gegeben. 


Diese Methode wird heute immer noch von Assembler Programmierern benutzt.
Im Windows Reseearch Kernel source Code (der sehr ähnlich zum Windows 2003 Kernel ist) können wir folgenden Code
finden (file \IT{base/ntos/ke/i386/cpu.asm}):

% muss noch die kommentare geändert werden
\begin{lstlisting}[style=customasmx86]
        public  Get386Stepping
Get386Stepping  proc

        call    MultiplyTest            ; Muliplikations Test durchführen
        jnc     short G3s00             ; wenn nc, ist muttest ok
        mov     ax, 0
        ret
G3s00:
        call    Check386B0              ; Prüfe das B0 stepping
        jnc     short G3s05             ; wenn nc, ist es B1/later
        mov     ax, 100h                ; It is B0/earlier stepping
        ret

G3s05:
        call    Check386D1              ; Prüfe das D1 stepping
        jc      short G3s10             ; wenn c, iust es NICHT NOT D1
        mov     ax, 301h                ; Es ist das D1/later stepping
        ret

G3s10:
        mov     ax, 101h                ; annahme das es das it is B1 stepping ist
        ret

	...

MultiplyTest    proc

        xor     cx,cx                   ; 64K durchläufe ist eine nette runde Nummer
mlt00:  push    cx
        call    Multiply                ; Funktioniert dis multiplikation auf diesem Chip?
        pop     cx
        jc      short mltx              ; wenn c c, Nein, exit
        loop    mlt00                   ; Wenn nc, Ja, weitere iteration für nächsten versuch
        clc
mltx:
        ret

MultiplyTest    endp
\end{lstlisting}

}
\ITA{\subsubsection{Passaggio di argomenti alle funzioni}

Il modo piu' diffuso di passare parametri in x86 e' detto \q{cdecl}:

\begin{lstlisting}[style=customasmx86]
push arg3
push arg2
push arg1
call f
add esp, 12 ; 4*3=12
\end{lstlisting}

La funzioni chiamate, \Gls{callee}, ricevono i propri argomenti tramite lo stack pointer.
Quindi e' cosi' che i valori degli arcomenti sono posizionati nello stack prima dell'esecuzione della prima istruzione della funzione \ttf{}:

\begin{center}
\begin{tabular}{ | l | l | }
\hline
ESP & return address \\
\hline
ESP+4 & \argument \#1, \MarkedInIDAAs{} \TT{arg\_0} \\
\hline
ESP+8 & \argument \#2, \MarkedInIDAAs{} \TT{arg\_4} \\
\hline
ESP+0xC & \argument \#3, \MarkedInIDAAs{} \TT{arg\_8} \\
\hline
\dots & \dots \\
\hline
\end{tabular}
\end{center}

Per ulteriori informazioni su altri tipi di convenzioni di chiamata (calling conventions), fare riferimento alla sezione~(\myref{sec:callingconventions}).
Vale la pena notare che non c'e' nulla che obbliga il programmatore a passare gli argomenti attraverso lo stack. Non e' un requisito necessario.
Si potrebbe implementare un qualunque altro metodo anche senza usare per niente lo stack.

Ad esempio e' possibile allocare spazio per gli argomenti nello \gls{heap}, riempirlo e passarlo ad una funzione tramite un puntatore a questo blocco di memoria, nel registro \EAX.
E cio' funzionerebbe.

% TBT: \InSqBrackets{\TAOCPvolI{}, 189} mentions even weirder schemes particularly convenient on IBM System/360.

Ad ogni modo, in x86 e ARM e' pratica diffusa e conveniente quella di usare lo stack per questo scopo.

\par
A proposito, la funzione chiamata (\gls{callee}) non ha alcuna informazione su quanti argomenti sono stati passati.
Le funzioni C functions con un numero variabile di argomenti (come \printf) determinano questo numero utilizzando i format string specifiers (che iniziano con il simbolo \% ).

Se scriviamo una cosa come:

\begin{lstlisting}
printf("%d %d %d", 1234);
\end{lstlisting}

\printf stampera' 1234, e due numeri randomici, che si trovano vicini a 1234 nello stack\footnote{\ac{TBT}}.

\par
Questo spiega perche' non e' molto importante come si dichiara la funzione \main : \main, \TT{main(int argc, char *argv[])} oppure \TT{main(int argc, char *argv[], char *envp[])}.

Infatti, il codice \ac{CRT} chiama \main grossomodo cosi':

\begin{lstlisting}[style=customasmx86]
push envp
push argv
push argc
call main
...
\end{lstlisting}

Se si dichiara \main come \main senza arogmenti, saranno comunque presenti nello stack anche se non usati.
Se si dichiara \main come \TT{main(int argc, char *argv[])},
si potranno usare i primi due argomenti, ed il terzo restera' \q{invisibile} alla funzione.
E' inoltre possibile dichiarare anche \TT{main(int argc)}, e funzionera'.
}
\FR{\subsubsection{Passage des arguments de fonction}

Le moyen le plus utilisé pour passer des arguments en x86 est appelé \q{cdecl}:

\begin{lstlisting}[style=customasmx86]
push arg3
push arg2
push arg1
call f
add esp, 12 ; 4*3=12
\end{lstlisting}

La fonction \glslink{callee}{appelée} reçoit ses arguments par la pile.

Voici donc comment sont stockés les arguments sur la pile avant l'exécution
de la première instruction de la fonction \ttf{}:

\begin{center}
\begin{tabular}{ | l | l | }
\hline
ESP & return address \\
\hline
ESP+4 & \argument \#1, \MarkedInIDAAs{} \TT{arg\_0} \\
\hline
ESP+8 & \argument \#2, \MarkedInIDAAs{} \TT{arg\_4} \\
\hline
ESP+0xC & \argument \#3, \MarkedInIDAAs{} \TT{arg\_8} \\
\hline
\dots & \dots \\
\hline
\end{tabular}
\end{center}

Pour plus d'information sur les conventions d'appel, voir cette section~(\myref{sec:callingconventions}).

\par
A propos, la fonction \glslink{callee}{appelée} n'a aucune d'information sur le
nombre d'arguments qui ont été passés.
Les fonctions C avec un nombre variable d'arguments (comme \printf) déterminent
leur nombre en utilisant les spécificateurs de la chaîne de format (qui commencent
pas le symbole \%).

Si nous écrivons quelque comme:

\begin{lstlisting}
printf("%d %d %d", 1234);
\end{lstlisting}

\printf va afficher 1234, et deux autres nombres aléatoires\footnote{Pas aléatoire
dans le sens strict du terme, mais plutôt imprévisibles: \myref{noise_in_stack}},
qui sont situés à côté dans la pile.

\par
C'est pourquoi la façon dont la fonction \main est déclarée n'est pas très importante:
comme \main, \\\TT{main(int argc, char *argv[])} ou \TT{main(int argc, char *argv[], char *envp[])}.

En fait, le code-\ac{CRT} appelle \main, schématiquement, de cette façon:
	
\begin{lstlisting}[style=customasmx86]
push envp
push argv
push argc
call main
...
\end{lstlisting}

Si vous déclarez \main comme \main sans argument, ils sont néanmoins toujours présents
sur la pile, mais ne sont pas utilisés.
Si vous déclarez \main as comme \TT{main(int argc, char *argv[])},
vous pourrez utiliser les deux premiers arguments, et le troisième restera \q{invisible}
pour votre fonction.
Il est même possible de déclarer \main comme \TT{main(int argc)}, cela fonctionnera.

\myparagraph{Autres façons de passer les arguments}

Il est à noter que rien n'oblige les programmeurs à passer les arguments à travers
la pile. Ce n'est pas une exigence.
On peut implémenter n'importe quelle autre méthode sans utiliser du tout la pile.

Une méthode répandue chez les débutants en assembleur est de passer les arguments
par des variables globales, comme:

\lstinputlisting[caption=Code assembleur,style=customasmx86]{patterns/02_stack/global_args.asm}

Mais cette méthode a un inconvénient évident: la fonction \IT{do\_something()}
ne peut pas s'appeler elle-même récursivement (ou par une autre fonction),
car il faudrait écraser ses propres arguments.
La même histoire avec les variables locales: si vous les stocker dans des variables
globales, la fonction ne peut pas s'appeler elle-même.
Et ce n'est pas thread-safe.
\footnote{Correctemment implémenter, chaque thread aurait sa propre pile avec ses propres arguments/variables.}.
Une méthode qui stocke ces informations sur la pile rend cela plus facile---elle
peut contenir autant d'arguments de fonctions et/ou de valeurs, que la pile a d'espace.

\InSqBrackets{\TAOCPvolI{}, 189} mentionne un schéma encore plus étrange, particulièrement
pratique sur les IBM System/360.

\myindex{MS-DOS}
\myindex{x86!\Instructions!INT}

MS-DOS a une manière de passer tout les arguments de fonctions via des registres,
par exemple, c'est un morceau de code pour un ancien MS-DOS 16-bit qui affiche
``Hello, world!'':

\begin{lstlisting}[style=customasmx86]
mov  dx, msg      ; address of message
mov  ah, 9        ; 9 means "print string" function
int  21h          ; DOS "syscall"

mov  ah, 4ch      ; "terminate program" function
int  21h          ; DOS "syscall"

msg  db 'Hello, World!\$'
\end{lstlisting}

\myindex{fastcall}
C'est presque similaire à la méthode \myref{fastcall}.
Et c'est aussi très similaire aux appels systèmes sous Linux (\myref{linux_syscall}) et Windows.

\myindex{x86!\Flags!CF}
Si une fonction MS-DOS devait renvoyer une valeur booléenne (i.e., un simple bit,
souvent pour indiquer un état d'erreur), le flag \TT{CF} était souvent utilisé.

Par exemple:

\begin{lstlisting}[style=customasmx86]
mov ah, 3ch       ; create file
lea dx, filename
mov cl, 1
int 21h
jc  error
mov file_handle, ax
...
error:
...
\end{lstlisting}

En cas d'erreur, le flag \TT{CF} est mis. Sinon, le handle du fichier nouvellement
créer est retourné via \TT{AX}.

Cette méthode est encore utilisée par les programmeurs en langage d'assemblage.
Dans le code source de Windows Research Kernel (qui est très similaire à Windows
2003) nous pouvons trouver quelque chose comme ça (file \IT{base/ntos/ke/i386/cpu.asm}):

\begin{lstlisting}[style=customasmx86]
        public  Get386Stepping
Get386Stepping  proc

        call    MultiplyTest            ; Perform multiplication test
        jnc     short G3s00             ; if nc, muttest is ok
        mov     ax, 0
        ret
G3s00:
        call    Check386B0              ; Check for B0 stepping
        jnc     short G3s05             ; if nc, it's B1/later
        mov     ax, 100h                ; It is B0/earlier stepping
        ret

G3s05:
        call    Check386D1              ; Check for D1 stepping
        jc      short G3s10             ; if c, it is NOT D1
        mov     ax, 301h                ; It is D1/later stepping
        ret

G3s10:
        mov     ax, 101h                ; assume it is B1 stepping
        ret

	...

MultiplyTest    proc

        xor     cx,cx                   ; 64K times is a nice round number
mlt00:  push    cx
        call    Multiply                ; does this chip's multiply work?
        pop     cx
        jc      short mltx              ; if c, No, exit
        loop    mlt00                   ; if nc, YEs, loop to try again
        clc
mltx:
        ret

MultiplyTest    endp
\end{lstlisting}

}


\EN{\subsubsection{Local variable storage}

A function could allocate space in the stack for its local variables just by decreasing 
the \gls{stack pointer} towards the stack bottom.

% I think here, "stack bottom" means the lowest address in the stack space,
% but the reader might also think it means towards the top of the stack space,
% like in a pop, so you might change "towards the stack bottom" to
% "towards the lowest address of the stack", or just take it out,
% since "decreasing" also suggests that.

Hence, it's very fast, no matter how many local variables are defined.
It is also not a requirement to store local variables in the stack.
You could store local variables wherever you like, 
but traditionally this is how it's done.

}
\RU{\subsubsection{Хранение локальных переменных}

Функция может выделить для себя некоторое место в стеке для локальных переменных, просто отодвинув 
\glslink{stack pointer}{указатель стека} глубже к концу стека.

% I think here, "stack bottom" means the lowest address in the stack space,
% but the reader might also think it means towards the top of the stack space,
% like in a pop, so you might change "towards the stack bottom" to
% "towards the lowest address of the stack", or just take it out,
% since "decreasing" also suggests that.

Это очень быстро вне зависимости от количества локальных переменных.
Хранить локальные переменные в стеке не является необходимым требованием. 
Вы можете хранить локальные переменные где угодно. 
Но по традиции всё сложилось так.

}
\PTBR{\subsubsection{Armazenamento de variáveis locais}

Uma função poderia alocar espaço na pilha para suas variáveis locais simplesmente decrementando o ponteiro da pilha.

% I think here, "stack bottom" means the lowest address in the stack space,
% but the reader might also think it means towards the top of the stack space,
% like in a pop, so you might change "towards the stack bottom" to
% "towards the lowest address of the stack", or just take it out,
% since "decreasing" also suggests that.

Consequentemente, é muito rápido, não importando quantas variáveis locais serão definidas.
Também não é um requisito armazenar variáveis locais na pilha.
Você pode armazenar variáveis locais onde você quiser, mas, tradicionalmente, é assim que é feito.

}
\EN{\subsubsection{x86: alloca() function}
\label{alloca}
\myindex{\CStandardLibrary!alloca()}

\newcommand{\AllocaSrcPath}{C:\textbackslash{}Program Files (x86)\textbackslash{}Microsoft Visual Studio 10.0\textbackslash{}VC\textbackslash{}crt\textbackslash{}src\textbackslash{}intel}

It is worth noting the \TT{alloca()} function
\footnote{In MSVC, the function implementation can be found in \TT{alloca16.asm} and \TT{chkstk.asm} in \\
\TT{\AllocaSrcPath{}}}.
This function works like \TT{malloc()}, but allocates memory directly on the stack.
% page break added to prevent "\vref on page boundary" error. it may be dropped in future.
The allocated memory chunk does not have to be freed via a \TT{free()} function call, \\
since the function epilogue (\myref{sec:prologepilog}) returns \ESP back to its initial state and 
the allocated memory is just \IT{dropped}.
It is worth noting how \TT{alloca()} is implemented.
In simple terms, this function just shifts \ESP downwards toward the stack bottom by the number of bytes you need and sets \ESP as a pointer to the \IT{allocated} block.

Let's try:

\lstinputlisting[style=customc]{patterns/02_stack/04_alloca/2_1.c}

\TT{\_snprintf()} function works just like \printf, but instead of dumping the result into \gls{stdout} (e.g., to terminal or 
console), it writes it to the \TT{buf} buffer. Function \puts copies the contents of \TT{buf} to \gls{stdout}. Of course, these two
function calls might be replaced by one \printf call, but we have to illustrate small buffer usage.

\myparagraph{MSVC}

Let's compile (MSVC 2010):

\lstinputlisting[caption=MSVC 2010,style=customasmx86]{patterns/02_stack/04_alloca/2_2_msvc.asm}

\myindex{Compiler intrinsic}
The sole \TT{alloca()} argument is passed via \EAX (instead of pushing it into the stack)
\footnote{It is because alloca() is rather a compiler intrinsic (\myref{sec:compiler_intrinsic}) than a normal function.
One of the reasons we need a separate function instead of just a couple of instructions in the code,
is because the \ac{MSVC} alloca() implementation also has code which reads from the memory just allocated, in order to let the \ac{OS} map
physical memory to this \ac{VM} region.
After the \TT{alloca()} call, \ESP points to the block of 600 bytes and we can use it as memory for the \TT{buf} array.}.

\myparagraph{GCC + \IntelSyntax}

GCC 4.4.1 does the same without calling external functions:

\lstinputlisting[caption=GCC 4.7.3,style=customasmx86]{patterns/02_stack/04_alloca/2_1_gcc_intel_O3_EN.asm}

\myparagraph{GCC + \ATTSyntax}

Let's see the same code, but in AT\&T syntax:

\lstinputlisting[caption=GCC 4.7.3,style=customasmx86]{patterns/02_stack/04_alloca/2_1_gcc_ATT_O3.s}

\myindex{\ATTSyntax}
The code is the same as in the previous listing.

By the way, \INS{movl \$3, 20(\%esp)} corresponds to
\INS{mov DWORD PTR [esp+20], 3} in Intel-syntax.
In the AT\&T syntax, the register+offset format of addressing memory looks like
\TT{offset(\%{register})}.

}
\FR{\subsubsection{x86: alloca() function}
\label{alloca}
\myindex{\CStandardLibrary!alloca()}

\newcommand{\AllocaSrcPath}{C:\textbackslash{}Program Files (x86)\textbackslash{}Microsoft Visual Studio 10.0\textbackslash{}VC\textbackslash{}crt\textbackslash{}src\textbackslash{}intel}

Intéressons-nous à la fonction \TT{alloca()}
\footnote{Avec MSVC, l'implémentation de cette fonction peut être trouvée dans les fichiers \TT{alloca16.asm} et \TT{chkstk.asm} dans \\
\TT{\AllocaSrcPath{}}}

Cette fonction fonctionne comme \TT{malloc()}, mais alloue de la mémoire directement sur la pile.
% page break added to prevent "\vref on page boundary" error. it may be dropped in future.
Le bout de mémoire ne doit pas être libéré via un appel à la fonction \TT{free()}, \\
puisque l'épilogue de fonction (\myref{sec:prologepilog}) retourne \ESP à son état initial précédant ce qui va automatiquement désallouer ce bout de mémoire.

Intéressons-nous à l'implémentation d'\TT{alloca()}.
Cette fonction décale simplement \ESP du nombre d'octets demandé vers le bas et vers le fond de la pile et définit \ESP en tant que pointeur vers la mémoire \IT{allouée}.

Essayons :

\lstinputlisting[style=customc]{patterns/02_stack/04_alloca/2_1.c}

La fonction \TT{\_snprintf()} fonctionne comme \printf, mais au lieu d'afficher le résultat sur \gls{stdout} (ex., dans un terminal ou une console), il l'écrit dans le buffer \TT{buf}. La fonction \puts copie le contenu de \TT{buf} dans \gls{stdout}. Évidemment, ces deux appels de fonctions peuvent être remplacés par un seul appel à la fonction \printf, mais nous devons illustrer l'utilisation de petit buffer.

\myparagraph{MSVC}

Compilons (MSVC 2010) :

\lstinputlisting[caption=MSVC 2010,style=customasmx86]{patterns/02_stack/04_alloca/2_2_msvc.asm}

\myindex{Compiler intrinsic}
Le seul argument d'\TT{alloca()} est passé via \EAX (au lieu de le mettre sur la pile )
\footnote{C'est parce que alloca() est plutôt une fonctionnalité intrinsèque du compilateur (\myref{sec:compiler_intrinsic}) qu'une fonction normale. Une des raisons pour lequelle nous avons besoin d'une fonction séparée au lieu de quelques instructions dans le code, est parce que l'implementation d'alloca() par \ac{MSVC} à également du code qui lit depuis la mémoire récemment allouée pour laisser l'\ac{OS} mapper la memoire physique vers la \ac{mémoire virtuelle}. Aprés l'appel à la fonction \TT{alloca()}, \ESP pointe sur un bloc de 600 octets que nous pouvons utiliser pour le tableau \TT{buf}.}

\myparagraph{GCC + \IntelSyntax}

GCC 4.4.1 fait la même chose sans effectuer d'appel à des fonctions externes :

\lstinputlisting[caption=GCC 4.7.3,style=customasmx86]{patterns/02_stack/04_alloca/2_1_gcc_intel_O3_FR.asm}

\myparagraph{GCC + \ATTSyntax}

Voyons le même code mais avec la syntaxe AT\&T :

\lstinputlisting[caption=GCC 4.7.3,style=customasmx86]{patterns/02_stack/04_alloca/2_1_gcc_ATT_O3.s}

\myindex{\ATTSyntax}
Le code est le même que le précédent.

Au fait, \INS{movl \$3, 20(\%esp)} correspond à
\INS{mov DWORD PTR [esp+20], 3} avec la syntaxe intel.
Dans la syntaxe AT\&T, le format registre+offset pour l'adressage mémoire
ressemble à \TT{offset(\%{register})}.
}
\RU{\subsubsection{x86: Функция alloca()}
\label{alloca}
\myindex{\CStandardLibrary!alloca()}

\newcommand{\AllocaSrcPath}{C:\textbackslash{}Program Files (x86)\textbackslash{}Microsoft Visual Studio 10.0\textbackslash{}VC\textbackslash{}crt\textbackslash{}src\textbackslash{}intel}

Интересен случай с функцией \TT{alloca()}
\footnote{В MSVC, реализацию функции можно посмотреть в файлах \TT{alloca16.asm} и \TT{chkstk.asm} в \\
\TT{\AllocaSrcPath{}}}. 
Эта функция работает как \TT{malloc()}, но выделяет память прямо в стеке.
Память освобождать через \TT{free()} не нужно, так как эпилог функции~(\myref{sec:prologepilog})
вернет \ESP в изначальное состояние и выделенная память просто \IT{выкидывается}.
Интересна реализация функции \TT{alloca()}.
Эта функция, если упрощенно, просто сдвигает \ESP вглубь стека на столько байт, сколько вам нужно и возвращает \ESP в качестве указателя на выделенный блок.

Попробуем:

\lstinputlisting[style=customc]{patterns/02_stack/04_alloca/2_1.c}

Функция \TT{\_snprintf()} работает так же, как и \printf, только вместо выдачи результата в \gls{stdout} (т.е. на терминал или в консоль),
записывает его в буфер \TT{buf}. Функция \puts выдает содержимое буфера \TT{buf} в \gls{stdout}. Конечно, можно было бы
заменить оба этих вызова на один \printf, но здесь нужно проиллюстрировать использование небольшого буфера.

\myparagraph{MSVC}

Компилируем (MSVC 2010):

\lstinputlisting[caption=MSVC 2010,style=customasmx86]{patterns/02_stack/04_alloca/2_2_msvc.asm}

\myindex{Compiler intrinsic}
Единственный параметр в \TT{alloca()} передается через \EAX, а не как обычно через стек
\footnote{Это потому, что alloca()~--- это не сколько функция, сколько т.н. \IT{compiler intrinsic} (\myref{sec:compiler_intrinsic})
Одна из причин, почему здесь нужна именно функция, а не несколько инструкций прямо в коде в том, что в реализации 
функции alloca() от \ac{MSVC}
есть также код, читающий из только что выделенной памяти, чтобы \ac{OS} подключила физическую память к этому региону \ac{VM}.
После вызова \TT{alloca()} \ESP указывает на блок в 600 байт, который мы можем использовать под \TT{buf}.}.

\myparagraph{GCC + \IntelSyntax}

А GCC 4.4.1 обходится без вызова других функций:

\lstinputlisting[caption=GCC 4.7.3,style=customasmx86]{patterns/02_stack/04_alloca/2_1_gcc_intel_O3_RU.asm}

\myparagraph{GCC + \ATTSyntax}

Посмотрим на тот же код, только в синтаксисе AT\&T:

\lstinputlisting[caption=GCC 4.7.3,style=customasmx86]{patterns/02_stack/04_alloca/2_1_gcc_ATT_O3.s}

\myindex{\ATTSyntax}
Всё то же самое, что и в прошлом листинге.

Кстати, \INS{movl \$3, 20(\%esp)}~--- это аналог \INS{mov DWORD PTR [esp+20], 3} в синтаксисе Intel.
Адресация памяти в виде \IT{регистр+смещение} записывается в синтаксисе AT\&T как \TT{смещение(\%{регистр})}.

}
\PTBR{\subsubsection{x86: a função alloca()}
\label{alloca}
\myindex{\CStandardLibrary!alloca()}

\newcommand{\AllocaSrcPath}{C:\textbackslash{}Program Files (x86)\textbackslash{}Microsoft Visual Studio 10.0\textbackslash{}VC\textbackslash{}crt\textbackslash{}src\textbackslash{}intel}

A função \TT{alloca()}
\footnote{No MSVC, a implementação da função pode ser encontrada nos arquivos \TT{alloca16.asm} e \TT{chkstk.asm} em \\
\TT{\AllocaSrcPath{}}}
funciona da mesma maneira que \TT{malloc()}, mas aloca memória diretamente na pilha.
O bloco de memória alocado não precisa ser limpo através da chamada da função free(),
desde que o rodapé da função (\myref{sec:prologepilog}) retorna \ESP de volta para seu estado inicial e a memória alocada é simplesmente desassociada.
Sobre como a função \TT{alloca()} é implementada, em termos simples, essa função só desloca \ESP para baixo 
(em direção ao fundo da pilha) pelo número de bytes que você precisa e define o ESP como um ponteiro para o bloco alocado.

\RU{Попробуем:}\EN{Let's try:}\PTBR{Vamos tentar:}

\lstinputlisting[style=customc]{patterns/02_stack/04_alloca/2_1.c}

A função \TT{\_snprintf()} funciona exatamente como \printf, mas ao invés de jogar o resultado em stdout
(terminal ou console, por exemplo), ela escreve no buffer buf.
A função \puts copia o conteúdo para um buf do stdout.
Lógico, essas duas chamadas de funções podem ser substituídas por um \printf, mas nós temos que ilustrar o uso pequeno do buffer.

\myparagraph{MSVC}

Vamos compilar (MSVC 2010):

\lstinputlisting[caption=MSVC 2010,style=customasmx86]{patterns/02_stack/04_alloca/2_2_msvc.asm}

\myindex{Compiler intrinsic}
O único argumento da função alloca() é passado via EAX (ao invés de ser empurrado na pilha)
\footnote{Isso é devido ao fato de que alloca() é mais nativa do compilador do que uma função normal (\myref{sec:compiler_intrinsic}).
Um dos motivos que se faz necessário o separamento da função ao invés de um pouco de linhas de código no código,
é porque a implementação da alloca() no MSVC também tem código que é lido da memória que acabou de ser alocada,
para deixar o sistema operacional mapear a memória física para essa região da memória virtual.}.

Depois da chamada de \TT{alloca()}, \ESP aponta para o bloco de 600 bytes que nós podemos usar como memória para o array.

\myparagraph{GCC + \IntelSyntax}

\PTBRph{}

}
\ITA{\subsubsection{x86: la funzione alloca() }
\label{alloca}
\myindex{\CStandardLibrary!alloca()}

\newcommand{\AllocaSrcPath}{C:\textbackslash{}Program Files (x86)\textbackslash{}Microsoft Visual Studio 10.0\textbackslash{}VC\textbackslash{}crt\textbackslash{}src\textbackslash{}intel}

Vale la pena esaminare la funzione \TT{alloca()}
\footnote{In MSVC, l'implementazione della funzione si trova in \TT{alloca16.asm} e \TT{chkstk.asm} in \\
\TT{\AllocaSrcPath{}}}.
Questa funzione opera come \TT{malloc()}, ma alloca memoria direttamente nello stack.
% page break added to prevent "\vref on page boundary" error. it may be dropped in future.
Il pezzo di memoria allocato non necessita di essere liberato tramite una chiamata alla funzione \TT{free()} function call, \\
poiche' l'epilogo della funzione (\myref{sec:prologepilog}) ripristina \ESP al suo valore iniziale e la memoria allocata viene semplicemente \IT{abbandonata}.
Vale anche la pena notare come e' implementata la funzione \TT{alloca()}.
In termini semplici, questa funzione shifta \ESP in basso, verso la base dello stack, per il numero di byte necessari e setta \ESP  
per puntare al blocco \IT{allocato}.

Proviamo:

\lstinputlisting[style=customc]{patterns/02_stack/04_alloca/2_1.c}

La funzione \TT{\_snprintf()} opera come \printf, ma invece di inviare il risultato a \gls{stdout} (es. al terminale o console),
lo scrive nel buffer \TT{buf}. La funzione \puts copia il contenuto di \TT{buf} in \gls{stdout}.
Ovviamente questo due chiamate potrebbero essere rimpiazzate da una sola chiamata a \printf, ma in questo caso era necessario per illustrare
l'uso di un piccolo buffer.

\myparagraph{MSVC}

Compiliamo (MSVC 2010):

\lstinputlisting[caption=MSVC 2010,style=customasmx86]{patterns/02_stack/04_alloca/2_2_msvc.asm}

\myindex{Compiler intrinsic}
L'unico argomento di \TT{alloca()} e' passato tramite il registro \EAX (anziche' metterlo nello stack)
\footnote{Questo perche' alloca() e' una "compiler intrinsic" (\myref{sec:compiler_intrinsic}) piuttosto che una funzione normale.
Una delle ragioni per cui abbiamo bisogno di una funzione separata, invece di un paio di istruzioni nel codice, e' che
l'implementazione di alloca() di \ac{MSVC} ha anche del codice che legge dalla memoria appena llocata, per far si che l'\ac{OS} effettui il mapping
della memoria fisica in questa regione della \ac{VM}.
Dopo la chiamata a \TT{alloca()} , \ESP punta al blocco di 600 byte, ed e' possibile utilizzarlo come memoria per l'array \TT{buf}.}.

\myparagraph{GCC + \IntelSyntax}

GCC 4.4.1 fa lo stesso senza chiamare funzioni esterne:

\lstinputlisting[caption=GCC 4.7.3,style=customasmx86]{patterns/02_stack/04_alloca/2_1_gcc_intel_O3_EN.asm}

\myparagraph{GCC + \ATTSyntax}

Esaminiamo lo stesso codice, ma in sintassi AT\&T:

\lstinputlisting[caption=GCC 4.7.3,style=customasmx86]{patterns/02_stack/04_alloca/2_1_gcc_ATT_O3.s}

\myindex{\ATTSyntax}
The code e' uguale a quello del listato precedente.

A proposito, \INS{movl \$3, 20(\%esp)} corrisponde a \INS{mov DWORD PTR [esp+20], 3} in sintassi Intel.
In sintassi AT\&T, il formato registro+offset per indirizzare memoria appare come \TT{offset(\%{register})}.
}
\DE{\subsubsection{x86: alloca() Funktion}
\label{alloca}
\myindex{\CStandardLibrary!alloca()}

\newcommand{\AllocaSrcPath}{C:\textbackslash{}Program Files (x86)\textbackslash{}Microsoft Visual Studio 10.0\textbackslash{}VC\textbackslash{}crt\textbackslash{}src\textbackslash{}intel}

Es macht Sinn einen Blick auf die \TT{alloca()} Funktion zu werfen
\footnote{In MSVC, kann die Funktions Implementierung in \TT{alloca16.asm} und \TT{chkstk.asm} in \\
\TT{\AllocaSrcPath{}}} gefunden werden.
Diese Funktion arbeitet wie \TT{malloc()}, nur das sie Speicher direkt auf dem Stack bereit stellt.

Der allozierte Speicher Chunk muss nicht wieder mit \TT{free()} freigegeben werden, weil
der Funktions Epilog (\myref{sec:prologepilog}) das \ESP Register wieder in seinen ursprünglichen 
Zustand versetzt und der allozierte Speicher wird einfach \IT{verworfen}. 
Es macht Sinn sich anzuschauen wie \TT{alloca()} implementiert ist.
Mit einfachen Begriffen erklärt, diese Funktion verschiebt \ESP in Richtung des Stack ende mit der 
Anzahl der Bytes die alloziert werden müssen und setzt \ESP als einen Zeiger auf den \IT{allozierten} block.

Beispiel:

\lstinputlisting{patterns/02_stack/04_alloca/2_1.c}


Die \TT{\_snprintf()} Funktion arbeitetet genau wie \printf, nur statt die Ergebnisse nach \gls{stdout} aus zu geben ( bsp. auf dem Terminal oder Konsole), schreibt sie in den \TT{buf} buffer. Die Funktion \puts kopiert den Inhalt aus \TT{buf} nach \gls{stdout}. Sicher könnte man die beiden Funktions Aufrufe könnten durch einen \printf Aufruf ersetzt werden, aber wir sollten einen genaueren Blick auf die Benutzung kleiner Buffer anschauen.

\myparagraph{MSVC}

Compilierung mit MSVC 2010: 

\lstinputlisting[caption=MSVC 2010]{patterns/02_stack/04_alloca/2_2_msvc.asm}

\myindex{Compiler intrinsisch}
Das einzige \\TT{alloca()} Argument wird über \EAX übergeben (anstatt es erst auf den Stack zu pushen)
\footnote{Das liegt daran, das alloca() Verhalten Compiler intrinsisch bestimmt (\myref{sec:compiler_intrinsic}) im Gegensatz zu einer normalen Funktion. Einer der Gründe dafür das man braucht eine separate Funktion braucht, statt ein paar Code Instruktionen im Code,  ist weil die \ac{MSCV} alloca() Implementierung ebenfalls Code hat welcher aus dem gerade allozierten Speicher gelesen wird. Damit in Folge das \ac{Betriebssystem} physikalischen Speicher in dieser \ac{VM} Region zu allozieren. Nach dem \TT{alloca()} Aufruf, zeigt \ESP auf den Block von 600 Bytes der nun als Speicher für das \TT{buf} Array dienen kann.}.

\myparagraph{GCC + \IntelSyntax}

GCC 4.4.1 macht das selbe, aber ohne externe Funktions aufrufe.

\lstinputlisting[caption=GCC 4.7.3]{patterns/02_stack/04_alloca/2_1_gcc_intel_O3_EN.asm}

\myparagraph{GCC + \ATTSyntax}

Nun der gleiche Code, aber in AT\&T Syntax:

\lstinputlisting[caption=GCC 4.7.3]{patterns/02_stack/04_alloca/2_1_gcc_ATT_O3.s}

\myindex{\ATTSyntax}
Der Code ist der gleiche wie im vorherigen listig.

Übrigens, \INS{movl \$3, 20(\%esp)} in AT\&T Syntax wird zu \
\INS{mov DWORD PTR [esp+20], 3} in Intel-syntax.
In der AT\&T Syntax, sehen Register+Offset Formatierungen einer Adresse so aus:
\TT{offset(\%{register})}.
}

\subsubsection{(Windows) SEH}
\myindex{Windows!Structured Exception Handling}

\ifdefined\RUSSIAN
В стеке хранятся записи \ac{SEH} для функции (если они присутствуют).
Читайте больше о нем здесь: (\myref{sec:SEH}).
\fi % RUSSIAN

\ifdefined\ENGLISH
\ac{SEH} records are also stored on the stack (if they are present).
Read more about it: (\myref{sec:SEH}).
\fi % ENGLISH

\ifdefined\BRAZILIAN
\ac{SEH} também são guardados na pilha (se estiverem presentes).
\PTBRph{}: (\myref{sec:SEH}).
\fi % BRAZILIAN

\ifdefined\ITALIAN
I record \ac{SEH}, se presenti, sono anch'essi memorizzati nello stack.
Maggiori informazioni qui: (\myref{sec:SEH}).
\fi % ITALIAN

\ifdefined\FRENCH
Les enregistrements \ac{SEH} sont aussi stockés dans la pile (s'ils sont présents).
Lire à ce propos: (\myref{sec:SEH}).
\fi % FRENCH

\ifdefined\ENGLISH
\subsubsection{Buffer overflow protection}

More about it here~(\myref{subsec:bufferoverflow}).
\fi

\ifdefined\RUSSIAN
\subsubsection{Защита от переполнений буфера}

Здесь больше об этом~(\myref{subsec:bufferoverflow}).
\fi

\ifdefined\BRAZILIAN
\subsubsection{Proteção contra estouro de buffer}

Mais sobre aqui~(\myref{subsec:bufferoverflow}).
\fi

\ifdefined\ITALIAN
\subsubsection{Protezione da buffer overflow}

Maggiori informazioni qui~(\myref{subsec:bufferoverflow}).
\fi

\ifdefined\FRENCH
\subsubsection{Protection contre les débordements de tampon}

Lire à ce propos~(\myref{subsec:bufferoverflow}).
\fi


\subsubsection{Автоматическое освобождение данных в стеке}

Возможно, причина хранения локальных переменных и SEH-записей в стеке в том, что после выхода из функции, всё эти данные освобождаются автоматически,
используя только одну инструкцию корректирования указателя стека (часто это \ADD).
Аргументы функций, можно сказать, тоже освобождаются автоматически в конце функции.
А всё что хранится в куче (\IT{heap}) нужно освобождать явно.

% sections
\EN{\subsection{A typical stack layout}

A typical stack layout in a 32-bit environment at the start of a function, 
before the first instruction execution looks like this:

\begin{center}
\begin{tabular}{ | l | l | }
\hline
\dots & \dots \\
\hline
ESP-0xC & \localVariable \#2, \MarkedInIDAAs{} \TT{var\_8} \\
\hline
ESP-8 & \localVariable \#1, \MarkedInIDAAs{} \TT{var\_4} \\
\hline
ESP-4 & \savedValueOf \EBP \\
\hline
ESP & \ReturnAddress \\
\hline
ESP+4 & \argument \#1, \MarkedInIDAAs{} \TT{arg\_0} \\
\hline
ESP+8 & \argument \#2, \MarkedInIDAAs{} \TT{arg\_4} \\
\hline
ESP+0xC & \argument \#3, \MarkedInIDAAs{} \TT{arg\_8} \\
\hline
\dots & \dots \\
\hline
\end{tabular}
\end{center}



% I think this only applies to RISC architectures
% that don't have a POP instruction that only lets you read one value
% (ie. ARM and MIPS).
% In x86, the return address is saved before entering the function,
% and the function does not have the chance to save the frame pointer.
% Also, you should mention that this is how the stack looks like
% right after the function prologue,
% which some readers might think is the first instruction,
% but is needed to save the frame pointer.
}
\RU{\subsection{Разметка типичного стека}

Разметка типичного стека в 32-битной среде
перед исполнением самой первой инструкции функции выглядит так:

\begin{center}
\begin{tabular}{ | l | l | }
\hline
\dots & \dots \\
\hline
ESP-0xC & \localVariable \#2, \MarkedInIDAAs{} \TT{var\_8} \\
\hline
ESP-8 & \localVariable \#1, \MarkedInIDAAs{} \TT{var\_4} \\
\hline
ESP-4 & \savedValueOf \EBP \\
\hline
ESP & \ReturnAddress \\
\hline
ESP+4 & \argument \#1, \MarkedInIDAAs{} \TT{arg\_0} \\
\hline
ESP+8 & \argument \#2, \MarkedInIDAAs{} \TT{arg\_4} \\
\hline
ESP+0xC & \argument \#3, \MarkedInIDAAs{} \TT{arg\_8} \\
\hline
\dots & \dots \\
\hline
\end{tabular}
\end{center}



% I think this only applies to RISC architectures
% that don't have a POP instruction that only lets you read one value
% (ie. ARM and MIPS).
% In x86, the return address is saved before entering the function,
% and the function does not have the chance to save the frame pointer.
% Also, you should mention that this is how the stack looks like
% right after the function prologue,
% which some readers might think is the first instruction,
% but is needed to save the frame pointer.

}
\PTBR{\subsection{Um modelo típico de pilha}

Um modelo típico de pilha em um ambiente 32-bits no início de uma função,
antes da execução da primeira instrução, se parece com isso:

\begin{center}
\begin{tabular}{ | l | l | }
\hline
\dots & \dots \\
\hline
ESP-0xC & \localVariable \#2, \MarkedInIDAAs{} \TT{var\_8} \\
\hline
ESP-8 & \localVariable \#1, \MarkedInIDAAs{} \TT{var\_4} \\
\hline
ESP-4 & \savedValueOf \EBP \\
\hline
ESP & \ReturnAddress \\
\hline
ESP+4 & \argument \#1, \MarkedInIDAAs{} \TT{arg\_0} \\
\hline
ESP+8 & \argument \#2, \MarkedInIDAAs{} \TT{arg\_4} \\
\hline
ESP+0xC & \argument \#3, \MarkedInIDAAs{} \TT{arg\_8} \\
\hline
\dots & \dots \\
\hline
\end{tabular}
\end{center}



% I think this only applies to RISC architectures
% that don't have a POP instruction that only lets you read one value
% (ie. ARM and MIPS).
% In x86, the return address is saved before entering the function,
% and the function does not have the chance to save the frame pointer.
% Also, you should mention that this is how the stack looks like
% right after the function prologue,
% which some readers might think is the first instruction,
% but is needed to save the frame pointer.
}
\EN{\subsection{Noise in stack}
\label{noise_in_stack}

\epigraph{When one says that something seems random, what one usually
means in practice is that one cannot see any regularities in it.}{Stephen Wolfram, A New Kind of Science.}

Often in this book \q{noise} or \q{garbage} values in the stack or memory are mentioned.
Where do they come from?
These are what has been left there after other functions' executions.
Short example:

\lstinputlisting[style=customc]{patterns/02_stack/08_noise/st.c}

Compiling \dots

\lstinputlisting[caption=\NonOptimizing MSVC 2010,style=customasmx86]{patterns/02_stack/08_noise/st.asm}

The compiler will grumble a little bit\dots

\begin{lstlisting}
c:\Polygon\c>cl st.c /Fast.asm /MD
Microsoft (R) 32-bit C/C++ Optimizing Compiler Version 16.00.40219.01 for 80x86
Copyright (C) Microsoft Corporation.  All rights reserved.

st.c
c:\polygon\c\st.c(11) : warning C4700: uninitialized local variable 'c' used
c:\polygon\c\st.c(11) : warning C4700: uninitialized local variable 'b' used
c:\polygon\c\st.c(11) : warning C4700: uninitialized local variable 'a' used
Microsoft (R) Incremental Linker Version 10.00.40219.01
Copyright (C) Microsoft Corporation.  All rights reserved.

/out:st.exe
st.obj
\end{lstlisting}

But when we run the compiled program \dots

\begin{lstlisting}
c:\Polygon\c>st
1, 2, 3
\end{lstlisting}

Oh, what a weird thing! We did not set any variables in \TT{f2()}. 
These are \q{ghosts} values, which are still in the stack.

\clearpage
Let's load the example into \olly:

\begin{figure}[H]
\centering
\myincludegraphics{patterns/02_stack/08_noise/olly1.png}
\caption{\olly: \TT{f1()}}
\label{fig:stack_noise_olly1}
\end{figure}

When \TT{f1()} assigns the variables $a$, $b$ and $c$, their values are stored at the address \TT{0x1FF860} and so on.

\clearpage
And when \TT{f2()} executes:

\begin{figure}[H]
\centering
\myincludegraphics{patterns/02_stack/08_noise/olly2.png}
\caption{\olly: \TT{f2()}}
\label{fig:stack_noise_olly2}
\end{figure}

... $a$, $b$ and $c$ of \TT{f2()} are located at the same addresses!
No one has overwritten the values yet, so at that point they are still untouched.
So, for this weird situation to occur, several functions have to be called one after another and
\ac{SP} has to be the same at each function entry (i.e., they have the same number
of arguments). Then the local variables will be located at the same positions in the stack.
Summarizing, all values in the stack (and memory cells in general) have values left there from previous function executions.
They are not random in the strict sense, but rather have unpredictable values.
Is there another option?
It would probably be possible to clear portions of the stack before each function execution,
but that's too much extra (and unnecessary) work.

\subsubsection{MSVC 2013}

The example was compiled by MSVC 2010.
But the reader of this book made attempt to compile this example in MSVC 2013, ran it, and got all 3 numbers reversed:%

\begin{lstlisting}
c:\Polygon\c>st
3, 2, 1
\end{lstlisting}

Why?
I also compiled this example in MSVC 2013 and saw this:


\begin{lstlisting}[caption=MSVC 2013,style=customasmx86]
_a$ = -12	; size = 4
_b$ = -8	; size = 4
_c$ = -4	; size = 4
_f2	PROC

...

_f2	ENDP

_c$ = -12	; size = 4
_b$ = -8	; size = 4
_a$ = -4	; size = 4
_f1	PROC

...

_f1	ENDP
\end{lstlisting}

Unlike MSVC 2010, MSVC 2013 allocated a/b/c variables in function \TT{f2()} in reverse order.%
And this is completely correct, because \CCpp standards has no rule, in which order local variables must be allocated in the local stack, if at all.
The reason of difference is because MSVC 2010 has one way to do it, and MSVC 2013 has supposedly something changed inside of compiler guts, so it behaves slightly different.

}
\FR{\subsection{Bruit dans la pile}
\label{bruit_dans_la_pile}

\epigraph{Quand quelqu'un dit que quelques chose est aléatoire,
  ce que cela signifie en pratique c'est qu'il n'est pas capable de
  voir les régularités de cette chose}{Stephen Wolfram, A New Kind of Science.}

Dans ce livre les valeurs dites \q{bruité} ou \q{poubelle} présente dans la pile ou dans la mémoire sont souvent mentionner.

D'où viennent-elles ?
Ces valeurs ont été laissées sur la pile après l'exécution de fonction précédente.
Par exemple: 

\lstinputlisting[style=customc]{patterns/02_stack/08_noise/st.c}

Compilons \dots

\lstinputlisting[caption=\NonOptimizing MSVC 2010,style=customasmx86]{patterns/02_stack/08_noise/st.asm}

Le compilateur va rouspéter un peu\dots 

\begin{lstlisting}
c:\Polygon\c>cl st.c /Fast.asm /MD
Microsoft (R) 32-bit C/C++ Optimizing Compiler Version 16.00.40219.01 for 80x86
Copyright (C) Microsoft Corporation.  All rights reserved.

st.c
c:\polygon\c\st.c(11) : warning C4700: uninitialized local variable 'c' used
c:\polygon\c\st.c(11) : warning C4700: uninitialized local variable 'b' used
c:\polygon\c\st.c(11) : warning C4700: uninitialized local variable 'a' used
Microsoft (R) Incremental Linker Version 10.00.40219.01
Copyright (C) Microsoft Corporation.  All rights reserved.

/out:st.exe
st.obj
\end{lstlisting}

Mais quand nous lançons le programme compilé \dots

\begin{lstlisting}
c:\Polygon\c>st
1, 2, 3
\end{lstlisting}

Quel résultat étrange ! Aucune variables n'a été initialisées dans \TT{f2()}.
Ce sont des valeurs \q{fantôme} qui sont toujours dans la pile.

\clearpage
Chargeons cet exemple dans \olly:

\begin{figure}[H]
\centering
\myincludegraphics{patterns/02_stack/08_noise/olly1.png}
\caption{\olly: \TT{f1()}}
\label{fig:stack_noise_olly1}
\end{figure}

Quand \TT{f1()} assigne les variable $a$, $b$ et $c$, leurs valeur sont stockées à l'adresse \TT{0x1FF860} et ainsi de suite.

\clearpage
Et quand \TT{f2()} s'exécute:

\begin{figure}[H]
\centering
\myincludegraphics{patterns/02_stack/08_noise/olly2.png}
\caption{\olly: \TT{f2()}}
\label{fig:stack_noise_olly2}
\end{figure}

... $a$, $b$ et $c$ de la fonction \TT{f2()} sont situées à la même adresse !
Aucunes autre fonction n'a encore écrasées ces valeurs, elles sont donc encore inchangées. Pour que cette situation arrive, il faut que plusieurs fonctions soit appelées les unes après les autres et \ac{SP} doit être le même sur chaque début de fonctions (i.e., les fonctions doivent avoir le même nombre d'arguments). Les variables locales seront donc positionnées au même endroit dans la pile. Pour résumer, toutes les valeurs sur la pile sont des valeurs laissé par des appels de fonction précédent. Ces valeurs laissées sur la pile ne sont pas réellement aléatoire dans le sens strict du terme, mais elles sont imprévisible. 
Y a t'il une autre option ?
Il serait probablement possible de nettoyer des parties de la pile avant chaque nouvelle exécution de fonction, mais cela engendrerai du travail et du temps d'exécution (non nécessaire) en plus.

\subsubsection{MSVC 2013}

Cette exemple a été compilé avec MSVC 2010.
Si vous essayez de compiler cette exemple avec MSVC 2013 et de l'exécuter, ces 3 nombres seront inversé:%

\begin{lstlisting}
c:\Polygon\c>st
3, 2, 1
\end{lstlisting}

Pourquoi ?
J'ai aussi compilé cette exemple avec MSVC 2013 et constaté ceci: 


\begin{lstlisting}[caption=MSVC 2013,style=customasmx86]
_a$ = -12	; size = 4
_b$ = -8	; size = 4
_c$ = -4	; size = 4
_f2	PROC

...

_f2	ENDP

_c$ = -12	; size = 4
_b$ = -8	; size = 4
_a$ = -4	; size = 4
_f1	PROC

...

_f1	ENDP
\end{lstlisting}

Contrairement à MSVC 2010, MSVC 2013 alloue les variables a/b/c dans la fonction \TT{f2()} dans l'ordre inverse puisqu'il se comporte différemment en raison d'un changement supposé dans son fonctionnement interne.%
Ceci est correct, car le standard du \CCpp n'impose aucune règle sur l'ordre d'allocation des variables locales sur la pile.
}
\RU{\subsection{Мусор в стеке}
\label{noise_in_stack}

\epigraph{When one says that something seems random, what one usually
means in practice is that one cannot see any regularities in it.}{Stephen Wolfram, A New Kind of Science.}

Часто в этой книге говорится о \q{шуме} или \q{мусоре} в стеке или памяти.
Откуда он берется?
Это то, что осталось там после исполнения предыдущих функций.

Короткий пример:

\lstinputlisting[style=customc]{patterns/02_stack/08_noise/st.c}

Компилируем\dots

\lstinputlisting[caption=\NonOptimizing MSVC 2010,style=customasmx86]{patterns/02_stack/08_noise/st.asm}

Компилятор поворчит немного\dots

\begin{lstlisting}
c:\Polygon\c>cl st.c /Fast.asm /MD
Microsoft (R) 32-bit C/C++ Optimizing Compiler Version 16.00.40219.01 for 80x86
Copyright (C) Microsoft Corporation.  All rights reserved.

st.c
c:\polygon\c\st.c(11) : warning C4700: uninitialized local variable 'c' used
c:\polygon\c\st.c(11) : warning C4700: uninitialized local variable 'b' used
c:\polygon\c\st.c(11) : warning C4700: uninitialized local variable 'a' used
Microsoft (R) Incremental Linker Version 10.00.40219.01
Copyright (C) Microsoft Corporation.  All rights reserved.

/out:st.exe
st.obj
\end{lstlisting}

Но когда мы запускаем\dots

\begin{lstlisting}
c:\Polygon\c>st
1, 2, 3
\end{lstlisting}

Ох. Вот это странно. Мы ведь не устанавливали значения никаких переменных в \TT{f2()}. 
Эти значения --- это \q{привидения}, которые всё ещё в стеке.

\clearpage
Загрузим пример в \olly:

\begin{figure}[H]
\centering
\myincludegraphics{patterns/02_stack/08_noise/olly1.png}
\caption{\olly: \TT{f1()}}
\label{fig:stack_noise_olly1}
\end{figure}

Когда \TT{f1()} заполняет переменные $a$, $b$ и $c$ они сохраняются по адресу \TT{0x1FF860}, итд.

\clearpage
А когда исполняется \TT{f2()}:

\begin{figure}[H]
\centering
\myincludegraphics{patterns/02_stack/08_noise/olly2.png}
\caption{\olly: \TT{f2()}}
\label{fig:stack_noise_olly2}
\end{figure}

... $a$, $b$ и $c$ в функции \TT{f2()} находятся по тем же адресам!
Пока никто не перезаписал их, так что они здесь в нетронутом виде.
Для создания такой странной ситуации несколько функций должны исполняться друг за другом
и \ac{SP} должен быть одинаковым при входе в функции, т.е. у функций должно быть равное количество
аргументов). Тогда локальные переменные будут расположены в том же месте стека.
Подводя итоги, все значения в стеке (да и памяти вообще) это значения оставшиеся от 
исполнения предыдущих функций.
Строго говоря, они не случайны, они скорее непредсказуемы.
А как иначе?
Можно было бы очищать части стека перед исполнением каждой функции,
но это слишком много лишней (и ненужной) работы.

\subsubsection{MSVC 2013}

Этот пример был скомпилирован в MSVC 2010.
Но один читатель этой книги сделал попытку скомпилировать пример в MSVC 2013, запустил и увидел 3 числа в обратном порядке:

\begin{lstlisting}
c:\Polygon\c>st
3, 2, 1
\end{lstlisting}

Почему?
Я также попробовал скомпилировать этот пример в MSVC 2013 и увидел это:

\begin{lstlisting}[caption=MSVC 2013,style=customasmx86]
_a$ = -12	; size = 4
_b$ = -8	; size = 4
_c$ = -4	; size = 4
_f2	PROC

...

_f2	ENDP

_c$ = -12	; size = 4
_b$ = -8	; size = 4
_a$ = -4	; size = 4
_f1	PROC

...

_f1	ENDP
\end{lstlisting}

В отличии от MSVC 2010, MSVC 2013 разместил переменные a/b/c в функции \TT{f2()} в обратном порядке.
И это полностью корректно, потому что в стандартах \CCpp нет правила, в каком порядке локальные переменные должны быть размещены в локальном стеке, если вообще.
Разница есть из-за того что MSVC 2010 делает это одним способом, а в MSVC 2013, вероятно, что-то немного изменили во внутренностях компилятора, так что он ведет себя слегка иначе.

}
\ITA{\subsection{Rumore nello stack}
\label{noise_in_stack}

\epigraph{When one says that something seems random, what one usually
means in practice is that one cannot see any regularities in it.}{Stephen Wolfram, A New Kind of Science.}

In questo libro si fa spesso riferimento a \q{rumore} o \q{spazzatura} (garbage) nello stack o in memoria.

Da dove arrivano?
Sono cio' che resta dopo l'esecuzione di altre funzioni.
Un piccolo esempio:

\lstinputlisting[style=customc]{patterns/02_stack/08_noise/st.c}

Compilando si ottiene:

\lstinputlisting[caption=\NonOptimizing MSVC 2010,style=customasmx86]{patterns/02_stack/08_noise/st.asm}

Il compilatore si lamentera' un pochino\dots

\begin{lstlisting}
c:\Polygon\c>cl st.c /Fast.asm /MD
Microsoft (R) 32-bit C/C++ Optimizing Compiler Version 16.00.40219.01 for 80x86
Copyright (C) Microsoft Corporation.  All rights reserved.

st.c
c:\polygon\c\st.c(11) : warning C4700: uninitialized local variable 'c' used
c:\polygon\c\st.c(11) : warning C4700: uninitialized local variable 'b' used
c:\polygon\c\st.c(11) : warning C4700: uninitialized local variable 'a' used
Microsoft (R) Incremental Linker Version 10.00.40219.01
Copyright (C) Microsoft Corporation.  All rights reserved.

/out:st.exe
st.obj
\end{lstlisting}

Ma quando avvieremo il programma \dots

\begin{lstlisting}
c:\Polygon\c>st
1, 2, 3
\end{lstlisting}

Oh, che cosa strana! Non abbiamo impostato il valore di alcuna variabile in \TT{f2()}. 
Si tratta di valori \q{fantasma} , che si trovano ancora nello stack.

\clearpage
Carichiamo l'esempio in \olly:

\begin{figure}[H]
\centering
\myincludegraphics{patterns/02_stack/08_noise/olly1.png}
\caption{\olly: \TT{f1()}}
\label{fig:stack_noise_olly1}
\end{figure}

Quando \TT{f1()} assegna le variabili $a$, $b$ e $c$, i loro valori sono memorizzati all'indirizzo \TT{0x1FF860} e seguenti.

\clearpage
E quando viene eseguita \TT{f2()}:

\begin{figure}[H]
\centering
\myincludegraphics{patterns/02_stack/08_noise/olly2.png}
\caption{\olly: \TT{f2()}}
\label{fig:stack_noise_olly2}
\end{figure}

... $a$, $b$ e $c$ di \TT{f2()} si trovano agli stessi indirizzi!
Nessuno ha ancora sovrascritto quei valori, e a quel punto restano intatti.
Quindi, affinche' questa strana situazione si verifichi, piu' funzioni devono essere chiamate una dopo l'altra e
\ac{SP} deve essere uguale ad ogni ingresso nella funzione (ovvero le funzioni devono avere lo stesso numero di argomenti).
A quel punto le variabili locali si troveranno nelle stesse posizioni nello stack.
Per riassumere, tutti i valori nello stack (e nelle celle di memoria in generale) hanno valori lasciati li' dall'esecuzione di funzioni precedenti.
Non sono letteralmente randomici, piuttosto hanno valori non predicibili.
C'e' un'altra opzione?
Sarebbe possibile ripulire porzioni dello stack prima di ogni esecuzione di una funzione, ma sarebbe un lavoro extra inutile.

\subsubsection{MSVC 2013}

L'esempio e' stato compilato con MSVC 2010.
Un lettore di questo libro ha provato a compilare l'esempio con MSVC 2013, lo ha eseguito, ed ha ottenuto i 3 numeri in ordine inverso:%

\begin{lstlisting}
c:\Polygon\c>st
3, 2, 1
\end{lstlisting}

Perche'?
Ho compilato anche io l'esempio in MSVC 2013 ed ho visto questo:


\begin{lstlisting}[caption=MSVC 2013,style=customasmx86]
_a$ = -12	; size = 4
_b$ = -8	; size = 4
_c$ = -4	; size = 4
_f2	PROC

...

_f2	ENDP

_c$ = -12	; size = 4
_b$ = -8	; size = 4
_a$ = -4	; size = 4
_f1	PROC

...

_f1	ENDP
\end{lstlisting}

Contrariamente a MSVC 2010, MSVC 2013 ha allocato le variabili a/b/c nella funzione \TT{f2()} in ordine inverso.%
E cio' e' del tutto corretto, perche' lo standard \CCpp non ha una regola che definisce in quale ordine le variabili locali devono essere allocate nello stack.
La ragione per cui si presenta questa differenza e' che MSVC 2010 ha un solo modo per farlo, mentre MSVC 2013 ha probabilmente subito modifiche all'interno del compilatore, e si comporta quindi in modo leggermente diverso. 
}
\DE{\subsection{Rauschen auf dem Stack}
\label{noise_in_stack}

\epigraph{When one says that something seems random, what one usually
means in practice is that one cannot see any regularities in it.}{Stephen Wolfram, A New Kind of Science.}

Oft wird in diesem Buch von \q{rauschen} oder \q{garbage} Werten im Bezug
auf den Stack gesprochen. Woher kommen diese Werte? Das sind Überbleibsel der
Ausführung von anderen Funktionen. Zum Beispiel: 

\lstinputlisting{patterns/02_stack/08_noise/st.c}

Compilieren \dots

\lstinputlisting[caption=\NonOptimizing MSVC 2010]{patterns/02_stack/08_noise/st.asm}

Hier wird sich der Compiler ein bisschen beschweren\dots

\begin{lstlisting}
c:\Polygon\c>cl st.c /Fast.asm /MD
Microsoft (R) 32-bit C/C++ Optimizing Compiler Version 16.00.40219.01 for 80x86
Copyright (C) Microsoft Corporation.  All rights reserved.

st.c
c:\polygon\c\st.c(11) : warning C4700: uninitialized local variable 'c' used
c:\polygon\c\st.c(11) : warning C4700: uninitialized local variable 'b' used
c:\polygon\c\st.c(11) : warning C4700: uninitialized local variable 'a' used
Microsoft (R) Incremental Linker Version 10.00.40219.01
Copyright (C) Microsoft Corporation.  All rights reserved.

/out:st.exe
st.obj
\end{lstlisting}

Aber wenn wir das compilierte Programm laufen lassen\dots

\begin{lstlisting}
c:\Polygon\c>st
1, 2, 3
\end{lstlisting}

sieh an! Wir haben keine Variablen gesetzt in \TT{f2()}.
Das sind \q{Geister} Werte, welche noch immer auf dem Stack rumliegen.

\clearpage
Lasst uns das Beispiel in \olly laden:

\begin{figure}[H]
\centering
\myincludegraphics{patterns/02_stack/08_noise/olly1.png}
\caption{\olly: \TT{f1()}}
\label{fig:stack_noise_olly1}
\end{figure}

Wenn \TT{f1()} den Variablen $a$, $b$ und $c$ ihre Werte zuordnet, wird ihre Adresse bei \TT{0x1FF860} gespeichert und so weiter. 

\clearpage
Und wenn \TT{f2()} ausgeführt wird:

\begin{figure}[H]
\centering
\myincludegraphics{patterns/02_stack/08_noise/olly2.png}
\caption{\olly: \TT{f2()}}
\label{fig:stack_noise_olly2}
\end{figure}

... liegen $a$, $b$ und $c$ von \TT{f2()}  an den gleichen Adressen!
Nichts hat bis jetzt Ihre Werte überschrieben und sie sind bisher
unberührt geblieben. Also, damit diese seltsame Situation eintritt, müssen
mehrere Funktionen nacheinander aufgerufen werden und \ac{SP} muss gleich 
sein für jede Funktions Instruktion ( z.B. die Funktionen haben die 
gleiche Anzahl an Argumenten). Dann werden die lokalen Variablen an den
gleichen Positionen im Stack liegen. Zusammen fassend kann man sagen, alle
Werte auf dem Stack (und Speicherzellen im allgemeinen) beinhalten Werte
von vorhergehenden Funktions aufrufen. Diese Werte sind nicht zufällig 
im klassischem Sinn, eher unvorhersehbar. Es wäre wahrscheinlich möglich 
Teile des Stacks auf zu räumen vor jedem Funktions Aufruf, aber das wäre
zu viel zusätzliche (und unnötige) Arbeit. 

\subsubsection{MSVC 2013}

Das Beispiel wurde compiliert mit dem MSVC 2010 Compiler.
Allerdings haben die Leser dieses Buch auch schon geschafft das Beispiel mit MSVC 2013 zu compilieren, sie haben es geschafft es zum laufen zu bringen und alle drei Nummern zu reversen.

\begin{lstlisting}
c:\Polygon\c>st
3, 2, 1
\end{lstlisting}

Warum? 
Ich habe das Beispiel auch mit MSCV 2013 compiliert und habe folgendes beobachtet:

\begin{lstlisting}[caption=MSVC 2013,style=customasmx86]
_a$ = -12	; size = 4
_b$ = -8	; size = 4
_c$ = -4	; size = 4
_f2	PROC

...

_f2	ENDP

_c$ = -12	; size = 4
_b$ = -8	; size = 4
_a$ = -4	; size = 4
_f1	PROC

...

_f1	ENDP
\end{lstlisting}

Im Gegensatz zu MSVC 2010, alloziert MSCV 2013 die Variablen in der Funktion \TT{f2()} in umgekehrter Reihenfolge.%
Was auch vollkommen Korrekt ist, weil es im \CCpp Standard keine Vorschriften gibt, in welcher Reihenfolge Variablen auf dem Stack alloziert werden müssen. Der Grund für den Unterschied liegt daran das MSCV 2010 eine Methode genutzt hat um  die allozierung durch zu führen und in MSCV 2013 wurde scheinbar eine anpassung im Compiler inneren gemacht, so das sich MSCV 2013 leicht anders verhält.

}

\subsection{\Exercises}

\begin{itemize}
	\item \url{http://challenges.re/51}
	\item \url{http://challenges.re/52}
\end{itemize}



