\subsubsection{x86: la funzione alloca() }
\label{alloca}
\myindex{\CStandardLibrary!alloca()}

\newcommand{\AllocaSrcPath}{C:\textbackslash{}Program Files (x86)\textbackslash{}Microsoft Visual Studio 10.0\textbackslash{}VC\textbackslash{}crt\textbackslash{}src\textbackslash{}intel}

Vale la pena esaminare la funzione \TT{alloca()}
\footnote{In MSVC, l'implementazione della funzione si trova in \TT{alloca16.asm} e \TT{chkstk.asm} in \\
\TT{\AllocaSrcPath{}}}.
Questa funzione opera come \TT{malloc()}, ma alloca memoria direttamente nello stack.
% page break added to prevent "\vref on page boundary" error. it may be dropped in future.
Il pezzo di memoria allocato non necessita di essere liberato tramite una chiamata alla funzione \TT{free()} function call, \\
poiche' l'epilogo della funzione (\myref{sec:prologepilog}) ripristina \ESP al suo valore iniziale e la memoria allocata viene semplicemente \IT{abbandonata}.
Vale anche la pena notare come e' implementata la funzione \TT{alloca()}.
In termini semplici, questa funzione shifta \ESP in basso, verso la base dello stack, per il numero di byte necessari e setta \ESP  
per puntare al blocco \IT{allocato}.

Proviamo:

\lstinputlisting[style=customc]{patterns/02_stack/04_alloca/2_1.c}

La funzione \TT{\_snprintf()} opera come \printf, ma invece di inviare il risultato a \gls{stdout} (es. al terminale o console),
lo scrive nel buffer \TT{buf}. La funzione \puts copia il contenuto di \TT{buf} in \gls{stdout}.
Ovviamente questo due chiamate potrebbero essere rimpiazzate da una sola chiamata a \printf, ma in questo caso era necessario per illustrare
l'uso di un piccolo buffer.

\myparagraph{MSVC}

Compiliamo (MSVC 2010):

\lstinputlisting[caption=MSVC 2010,style=customasmx86]{patterns/02_stack/04_alloca/2_2_msvc.asm}

\myindex{Compiler intrinsic}
L'unico argomento di \TT{alloca()} e' passato tramite il registro \EAX (anziche' metterlo nello stack)
\footnote{Questo perche' alloca() e' una "compiler intrinsic" (\myref{sec:compiler_intrinsic}) piuttosto che una funzione normale.
Una delle ragioni per cui abbiamo bisogno di una funzione separata, invece di un paio di istruzioni nel codice, e' che
l'implementazione di alloca() di \ac{MSVC} ha anche del codice che legge dalla memoria appena llocata, per far si che l'\ac{OS} effettui il mapping
della memoria fisica in questa regione della \ac{VM}.
Dopo la chiamata a \TT{alloca()} , \ESP punta al blocco di 600 byte, ed e' possibile utilizzarlo come memoria per l'array \TT{buf}.}.

\myparagraph{GCC + \IntelSyntax}

GCC 4.4.1 fa lo stesso senza chiamare funzioni esterne:

\lstinputlisting[caption=GCC 4.7.3,style=customasmx86]{patterns/02_stack/04_alloca/2_1_gcc_intel_O3_EN.asm}

\myparagraph{GCC + \ATTSyntax}

Esaminiamo lo stesso codice, ma in sintassi AT\&T:

\lstinputlisting[caption=GCC 4.7.3,style=customasmx86]{patterns/02_stack/04_alloca/2_1_gcc_ATT_O3.s}

\myindex{\ATTSyntax}
The code e' uguale a quello del listato precedente.

A proposito, \INS{movl \$3, 20(\%esp)} corrisponde a \INS{mov DWORD PTR [esp+20], 3} in sintassi Intel.
In sintassi AT\&T, il formato registro+offset per indirizzare memoria appare come \TT{offset(\%{register})}.
