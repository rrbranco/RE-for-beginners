\subsection{Berechnung der Maschinengenauigkeit}
Die Maschinengenauigkeit ist der kleinstmögliche Wert, mit dem die \ac{FPU} arbeiten kann.
Je mehr Bits für eine Fließkommazahl verwendet werden, desto kleiner ist die Maschinengenauigkeit.
Sie beträgt $2^{-23} = 1.19e-07$ für \Tfloat und $2^{-52} = 2.22e-16$ für \Tdouble.
Siehe auch:\href{http://link.yurichev.com/17367}{Wikipedia article}.

% TODO recheck values
Interessant ist, wie einfach die Maschinengenauigkeit berechnet werden kann:

\lstinputlisting[style=customc]{patterns/17_unions/epsilon/float.c}
Was wir hier machen ist, den Bruch in der IEEE 754 Zahl als Integer zu behandeln und 1 hinzuzuaddieren.
Die resultierende Fließkommazahl ist gleich $starting\_value+machine\_epsilon$, sodass wir nur den Startwert (in der
Fließkommaarithmetik) abziehen müssen um zu messen, welchen Unterschied ein Bit in der einfachen Genauigkeit (\Tfloat)
ausmacht.
Die \IT{union} dient hier als Mittel, um auf die IEEE 754 Zahl als regulären Integer zuzugreifen.
Die Addition von 1 entspricht hier einer Addition von 1 zum Bruch in der Zahl, aber natürlich kann hier ein Overflow
auftreten, was eine Addition von 1 zum Exponenten nach sich zieht.

\subsubsection{x86}

\lstinputlisting[caption=\Optimizing MSVC 2010,style=customasmx86]{patterns/17_unions/epsilon/float_MSVC_2010_Ox_DE.asm}
Der zweite \INS{FST} Befehl ist redundant: es besteht keine Notwendigkeit, den Eingabewert an derselben Stelle zu
speichern (der Compiler hat entschieden, die Variable $v$ an der gleichen Stelle im lokalen Stack anzulegen wie den
Eingabewert).
Der Wert wird mit \INS{INC} um 1 erhöht als wäre es einen normale Integervariable.
Danach wir der Wert als 32-Bit IEEE 754 Zahl in die FPU geladen, \INS{FSUBR} erledigt den Rest und das Ergebnis wird in
\TT{ST0} gespeichert.
Das letzte \INS{FSTP}/\INS{FLD} Befehlspaar ist redundant, aber der Compiler hat es hier nicht wegoptimiert.

\subsubsection{ARM64}

Erweitern wir unser Beispiel auf 64 Bit:

\lstinputlisting[label=machine_epsilon_double_c,style=customc]{patterns/17_unions/epsilon/double.c}
ARM64 kennt keinen Befehl, der eine Zahl zu einem FPU D-Register addieren kann, sodass der Eingabewert (in \TT{D0})
zunächst nach \ac{GPR} kopiert wird, dann inkrementiert wird und schließlich in das FPU Register \TT{D1} kopiert wird,
bevor die Subtraktion ausgeführt wird.

\lstinputlisting[caption=\Optimizing GCC 4.9
ARM64,style=customasmARM]{patterns/17_unions/epsilon/double_GCC49_ARM64_O3_DE.s}
Schauen Sie sich dieses Beispiel auch für x64 kompiliert mit SIMB Befehlen an:\myref{machine_epsilon_x64_and_SIMD}.

\subsubsection{MIPS}

\myindex{MIPS!\Instructions!MTC1}
Der neue Befehl ist hier \INS{MTC1} (\q{Move To Coprocessor 1}): er überträgt Daten von \ac{GPR} in die Register der
FPU.

\lstinputlisting[caption=\Optimizing GCC 4.4.5 (IDA),style=customasmMIPS]{patterns/17_unions/epsilon/MIPS_O3_IDA.lst}

\subsubsection{\Conclusion}
Es ist schwer zu sagen, ob jemand eine solche Trickserei in echtem Produktivcode benötigt, aber wie bereits mehrfach
erwähnt, ist dieses Beispiel gut geeignet, um das IEEE 754 Format und \IT{unions} in \CCpp zu erklären.

