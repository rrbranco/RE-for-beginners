\section[Linear congruential generator]{Linear congruential generator as pseudorandom number generator}
\myindex{\CStandardLibrary!rand()}
\label{LCG_simple}

Perhaps, the linear congruential generator is the simplest possible way to generate random numbers.

It's not in favour nowadays\footnote{Mersenne twister is better}, but it's so simple 
(just one multiplication, one addition and one AND operation), 
we can use it as an example.

\lstinputlisting[style=customc]{patterns/145_LCG/rand_EN.c}

There are two functions: the first one is used to initialize the internal state, and the second one is called
to generate pseudorandom numbers.

We see that two constants are used in the algorithm.
They are taken from
[William H. Press and Saul A. Teukolsky and William T. Vetterling and Brian P. Flannery, \IT{Numerical Recipes}, (2007)].

Let's define them using a \TT{\#define} \CCpp statement. It's a macro.

The difference between a \CCpp macro and a constant is that all macros are replaced 
with their value by \CCpp preprocessor,
and they don't take any memory, unlike variables.

In contrast, a constant is a read-only variable.

It's possible to take a pointer (or address) of a constant variable, but impossible to do so with a macro.

The last AND operation is needed because by C-standard \TT{my\_rand()} has to return a value in 
the 0..32767 range.

If you want to get 32-bit pseudorandom values, just omit the last AND operation.

\subsection{x86}

\lstinputlisting[caption=\Optimizing MSVC 2013,style=customasmx86]{patterns/145_LCG/rand_MSVC_2013_x86_Ox.asm}

Here we see it: both constants are embedded into the code.
There is no memory allocated for them.

The \TT{my\_srand()} function just copies its input value into the internal\\
\TT{rand\_state} variable.

\TT{my\_rand()} takes it, calculates the next \TT{rand\_state}, cuts it and leaves it in the EAX register.

The non-optimized version is more verbose:

\lstinputlisting[caption=\NonOptimizing MSVC 2013,style=customasmx86]{patterns/145_LCG/rand_MSVC_2013_x86.asm}

\subsection{x64}

The x64 version is mostly the same and uses 32-bit registers instead of 64-bit ones 
(because we are working with \Tint values here).

But \TT{my\_srand()} takes its input argument from the \ECX register rather than from stack:

\lstinputlisting[caption=\Optimizing MSVC 2013 x64,style=customasmx86]{patterns/145_LCG/rand_MSVC_2013_x64_Ox_EN.asm}

GCC compiler generates mostly the same code.

\subsection{32-bit ARM}

\lstinputlisting[caption=\OptimizingKeilVI (\ARMMode),style=customasmARM]{patterns/145_LCG/rand.s_Keil_ARM_O3_EN.s}

It's not possible to embed 32-bit constants into ARM instructions, so Keil has to place them externally and load them additionally.
One interesting thing is that it's not possible to embed the 0x7FFF constant as well.
So what Keil does is shifting \TT{rand\_state} left by 17 bits and then shifting it right by 17 bits.
This is analogous to the $(rand\_state \ll 17) \gg 17$ statement in \CCpp.
It seems to be useless operation, but what it does is clearing the high 17 bits, leaving the low 15 bits intact, and that's our goal after all. \\
\\
\Optimizing Keil for Thumb mode generates mostly the same code.

\subsection{MIPS}

\lstinputlisting[caption=\Optimizing GCC 4.4.5 (IDA),style=customasmMIPS]{patterns/145_LCG/MIPS_O3_IDA_EN.lst}

Wow, here we see only one constant (0x3C6EF35F or 1013904223).
Where is the other one (1664525)?

It seems that multiplication by 1664525 is performed by just using shifts and additions!
Let's check this assumption:

\lstinputlisting[style=customc]{patterns/145_LCG/test.c}

\lstinputlisting[caption=\Optimizing GCC 4.4.5 (IDA),style=customasmMIPS]{patterns/145_LCG/test_O3_MIPS.lst}

Indeed!

\subsubsection{MIPS relocations}

We will also focus on how such operations as load from memory and store to memory actually work.

The listings here are produced by IDA, which hides some details.

We'll run objdump twice: to get a disassembled listing and also relocations list:

\lstinputlisting[caption=\Optimizing GCC 4.4.5 (objdump)]{patterns/145_LCG/MIPS_O3_objdump.txt}

Let's consider the two relocations for the \TT{my\_srand()} function.

The first one, for address 0 has a type of \TT{R\_MIPS\_HI16}
and the second one for address 8 has a type of \TT{R\_MIPS\_LO16}.

That implies that address of the beginning of the .bss segment is to be written into the instructions at
address of 0 (high part of address) and 8 (low part of address).

The \TT{rand\_state} variable is at the very start of the .bss segment.

So we see zeros in the operands of instructions \LUI and \SW, because nothing is there yet---
the compiler don't know what to write there.

The linker will fix this, and the high part of the address will be written into the operand of \LUI and
the low part of the address---to the operand of \SW.

\SW will sum up the low part of the address and what is in register \$V0 (the high part is there).

It's the same story with the my\_rand() function: R\_MIPS\_HI16 relocation instructs the linker to write the high part
of the .bss segment address into instruction \LUI.

So the high part of the rand\_state variable address is residing in register \$V1.

The \LW instruction at address 0x10 sums up the high and low parts and loads the value of the rand\_state 
variable into \$V0.

The \SW instruction at address 0x54 do the summing again and then stores the new value 
to the rand\_state global variable.

IDA processes relocations while loading, thus hiding these details, but we should keep them in mind.

% TODO add example of compiled binary, GDB example, etc...


\subsection{Thread-safe version of the example}

The thread-safe version of the example is to be demonstrated later: \myref{LCG_TLS}.
