\myparagraph{\Optimizing MSVC 2010}

\lstinputlisting[caption=\Optimizing MSVC 2010,style=customasmx86]{patterns/12_FPU/3_comparison/x86/MSVC_Ox/MSVC_EN.asm}

\myindex{x86!\Instructions!FCOM}

\FCOM differs from \FCOMP in the sense that it just compares the values and doesn't change the FPU stack. 
Unlike the previous example, here the operands are in reverse order, 
which is why the result of the comparison in \CThreeBits is different:

\begin{itemize}
\item If $a>b$ in our example, then \CThreeBits bits are to be set as: 0, 0, 0.
\item If $b>a$, then the bits are: 0, 0, 1.
\item If $a=b$, then the bits are: 1, 0, 0.
\end{itemize}
% TODO: table?

The \INS{test ah, 65} instruction leaves just two bits~---\Cthree and \Czero. 
Both will be zero if $a>b$: in that case the \JNE jump will not be triggered. 
Then \INS{FSTP ST(1)} follows~---this instruction copies the value from \ST{0} to the operand and 
pops one value from the FPU stack.
In other words, the instruction copies \ST{0} (where the value of \GTT{\_a} is now) into \ST{1}.
After that, two copies of {\_a} are at the top of the stack. 
Then, one value is popped.
After that, \ST{0} contains {\_a} and the function is finishes.

The conditional jump \JNE is triggering in two cases: if $b>a$ or $a=b$. 
\ST{0} is copied into \ST{0}, it is just like an idle (\ac{NOP}) operation, then one value 
is popped from the stack and the top of the stack (\ST{0}) is contain what has been in \ST{1} before 
(that is {\_b}). 
Then the function finishes. 
The reason this instruction is used here probably is because the \ac{FPU} 
has no other instruction to pop a value from the stack and discard it.

\clearpage
\mysubparagraph{First \olly example: a=1.2 and b=3.4}
\myindex{\olly}

Both \FLD are executed:

\begin{figure}[H]
\centering
\myincludegraphics{patterns/12_FPU/3_comparison/x86/MSVC_Ox/olly1_1.png}
\caption{\olly: both \FLD are executed}
\label{fig:FPU_comparison_Ox_case1_olly1}
\end{figure}

\FCOM being executed: 
\olly shows the contents of \ST{0} and \ST{1} %
for convenience.

\clearpage
\FCOM has been executed:

\begin{figure}[H]
\centering
\myincludegraphics{patterns/12_FPU/3_comparison/x86/MSVC_Ox/olly1_2.png}
\caption{\olly: \FCOM has been executed}
\label{fig:FPU_comparison_Ox_case1_olly2}
\end{figure}

\Czero is set, all other condition flags are cleared.

\clearpage
\FNSTSW has been executed, \GTT{AX}=0x3100:

\begin{figure}[H]
\centering
\myincludegraphics{patterns/12_FPU/3_comparison/x86/MSVC_Ox/olly1_3.png}
\caption{\olly: \FNSTSW is executed}
\label{fig:FPU_comparison_Ox_case1_olly3}
\end{figure}

\clearpage
\TEST is executed:

\begin{figure}[H]
\centering
\myincludegraphics{patterns/12_FPU/3_comparison/x86/MSVC_Ox/olly1_4.png}
\caption{\olly: \TEST is executed}
\label{fig:FPU_comparison_Ox_case1_olly4}
\end{figure}

ZF=0, conditional jump is about to trigger now.

\clearpage
\INS{FSTP ST} (or \FSTP \ST{0}) has been executed~---1.2 has been popped from the stack, and 3.4 was left on top:

\begin{figure}[H]
\centering
\myincludegraphics{patterns/12_FPU/3_comparison/x86/MSVC_Ox/olly1_5.png}
\caption{\olly: \FSTP is executed}
\label{fig:FPU_comparison_Ox_case1_olly5}
\end{figure}

We see that the \INS{FSTP ST} 

instruction works just like popping one value from the FPU stack.

\clearpage
\mysubparagraph{Second \olly example: a=5.6 and b=-4}

Both \FLD are executed:

\begin{figure}[H]
\centering
\myincludegraphics{patterns/12_FPU/3_comparison/x86/MSVC_Ox/olly2_1.png}
\caption{\olly: both \FLD are executed}
\label{fig:FPU_comparison_Ox_case2_olly1}
\end{figure}

\FCOM is about to execute.

\clearpage
\FCOM has been executed:

\begin{figure}[H]
\centering
\myincludegraphics{patterns/12_FPU/3_comparison/x86/MSVC_Ox/olly2_2.png}
\caption{\olly: \FCOM is finished}
\label{fig:FPU_comparison_Ox_case2_olly2}
\end{figure}

All conditional flags are cleared.

\clearpage
\FNSTSW done, \GTT{AX}=0x3000:

\begin{figure}[H]
\centering
\myincludegraphics{patterns/12_FPU/3_comparison/x86/MSVC_Ox/olly2_3.png}
\caption{\olly: \FNSTSW has been executed}
\label{fig:FPU_comparison_Ox_case2_olly3}
\end{figure}

\clearpage
\TEST has been executed:

\begin{figure}[H]
\centering
\myincludegraphics{patterns/12_FPU/3_comparison/x86/MSVC_Ox/olly2_4.png}
\caption{\olly: \TEST has been executed}
\label{fig:FPU_comparison_Ox_case2_olly4}
\end{figure}

ZF=1, jump will not happen now.

\clearpage
\FSTP \ST{1} has been executed: a value of 5.6 is now at the top of the FPU stack.

\begin{figure}[H]
\centering
\myincludegraphics{patterns/12_FPU/3_comparison/x86/MSVC_Ox/olly2_5.png}
\caption{\olly: \FSTP has been executed}
\label{fig:FPU_comparison_Ox_case2_olly5}
\end{figure}

We now see that the \FSTP \ST{1} 
instruction works as follows: it leaves what has been at the top of the stack, but clears \ST{1}.


