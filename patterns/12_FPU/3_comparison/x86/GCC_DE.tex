\myparagraph{GCC 4.4.1}

\lstinputlisting[caption=GCC
4.4.1,style=customasmx86]{patterns/12_FPU/3_comparison/x86/GCC_DE.asm}

\myindex{x86!\Instructions!FUCOMPP}
\FUCOMPP{} ist fast wie like \FCOM, nimmt aber beide Werte vom Stand und
behandelt \q{undefinierte Zahlenwerte} anders.


\myindex{Non-a-numbers (NaNs)}
Ein wenig über \IT{undefinierte Zahlenwerte}.

\newcommand{\NANFN}{\footnote{\href{http://go.yurichev.com/17130}{wikipedia.org/wiki/NaN}}}
Die FPU ist in der Lage mit speziellen undefinieten Werten, den sogenannten
\IT{not-a-number}(kurz \gls{NaN})\NANFN umzugehen. Beispiele sind etwa der Wert
unendlich, das Ergebnis einer Division durch 0, etc. Undefinierte Werte können
entwder \q{quiet} oder \q{signaling} sein. Es ist möglich mit \q{quiet} NaNs zu
arbeiten, aber beim Versuch einen Befehl auf \q{signaling} NaNs auszuführen,
wird eine Exception geworfen. 

\myindex{x86!\Instructions!FCOM}
\myindex{x86!\Instructions!FUCOM}
\FCOM erzeugt eine Exception, falls irgendein Operand ein \gls{NaN} ist.
\FUCOM erzeugt eine Exception nur dann, wenn ein Operand eine \q{signaling}
\gls{NaN} (SNaN) ist.

\myindex{x86!\Instructions!SAHF}
\label{SAHF}
Der nächste Befehl ist \SAHF (\IT{Store AH into Flags})~---es handelt sich
hierbei um einen seltenen Befehl, der nicht mit der FPU zusammenhängt.
8 Bits aus AH werden in die niederen 8 Bit der CPU Flags in der folgenden
Reihenfolge verschoben:

\begin{center}
\begin{bytefield}[endianness=big,bitwidth=0.03\linewidth]{8}
\bitheader{7,6,4,2,0} \\
\bitbox{1}{SF} & 
\bitbox{1}{ZF} & 
\bitbox{1}{} & 
\bitbox{1}{AF} & 
\bitbox{1}{} & 
\bitbox{1}{PF} & 
\bitbox{1}{} & 
\bitbox{1}{CF}
\end{bytefield}
\end{center}


\myindex{x86!\Instructions!FNSTSW}
Erinnern wir uns, dass \FNSTSW die für uns interessanten Bits (\CThreeBits) auf
den Stellen 6,2,0 im AH Register setzt:

\begin{center}
\begin{bytefield}[endianness=big,bitwidth=0.03\linewidth]{8}
\bitheader{6,2,1,0} \\
\bitbox{1}{} & 
\bitbox{1}{C3} & 
\bitbox{3}{} & 
\bitbox{1}{C2} & 
\bitbox{1}{C1} & 
\bitbox{1}{C0}
\end{bytefield}
\end{center}

Mit anderen Worten: der Befehl \INS{fnstsw ax / sahf} verschiebt \CThreeBits
nach \ZF, \PF und \CF. 

Überlegen wir uns auch die Werte der \CThreeBits in unterschiedlichen Szenarien:

\begin{itemize} 
  \item Falls in unserem Beispiel $a$ größer als $b$ ist, dann werden die
  \CThreeBits auf 0,0,0 gesetzt.
  \item Falls $a$ kleiner als $b$ ist, werden die Bits auf 0,0,1 gesetzt.
  \item Falls $a=b$, dann werden die Bits auf 1,0,0 gesetzt.
\end{itemize}
% TODO: table?
Mit anderen Worten, die folgenden Zustände der CPU Flags sind nach drei
\FUCMPP/\FNSTSW/\SAHF Befehlen möglich:

\begin{itemize}
\item Falls $a>b$, werden die CPU Flags wie folgt gesetzt \GTT{ZF=0, PF=0,
CF=0}.
\item Falls $a<b$, werden die CPU Flags wie folgt gesetzt: \GTT{ZF=0, PF=0,
CF=1}.
\item Und falls $a=b$, dann gilt: \GTT{ZF=1, PF=0, CF=0}.
\end{itemize}
% TODO: table?

\myindex{x86!\Instructions!SETcc}
\myindex{x86!\Instructions!JNBE}
Abhängig von den CPU Flags und Bedingungen, speichert \SETNBE entweder 1 oder 0
in AL.
Es ist also quasi das Gegenstück von \JNBE mit dem Unterschied, dass \SETcc

Depending on the CPU flags and conditions, \SETNBE stores 1 or 0 to AL. 
It is almost the counterpart of \JNBE, with the exception that \SETcc
\footnote{\IT{cc} is \IT{condition code}} eine 1 oder 0 in \AL speichert, aber
\Jcc tatsächlich auch springt.
\SETNBE speicher 1 nur, falls \GTT{CF=0} und \GTT{ZF=0}.
Wenn dies nicht der Fall ist, dann wird 0 in \AL gespeichert.

Nur in einem Fall sind \CF und \ZF beide 0: falls $a>b$.

In diesem Fall wird 1 in \AL gespeichert, der nachfolgende \JZ Sprung wird nicht
ausgeführt und die Funktion liefert {\_a} zurück. In allen anderen Fällen wird
{\_b} zurückgegeben.