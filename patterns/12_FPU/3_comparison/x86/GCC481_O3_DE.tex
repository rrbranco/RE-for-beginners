\myparagraph{GCC 4.8.1 mit aktivierter \Othree Optimierung}
\label{gcc481_o3}
Mit der P6 Intel Familie\footnote{Beginnend mit Pentium Pro, Pentium-II, etc.}
wurden einige neue FPU Befehle hinzugefügt. 
\myindex{x86!\Instructions!FUCOMI}
Diese sind \INS{FUCOMI} (vergleiche Operanden und setze Flags der CPU) und 
\myindex{x86!\Instructions!FCMOVcc}
\INS{FCMOVcc} (arbeitet wie \INS{CMOVcc}, aber auf FPU Registern).
Offenbar haben sich die Verwalter von GCC dazu entschieden, den Support von
vor-P6 Intel CPUs (frühe Pentiums, 80486, etc.) einzustellen.

Außerdem ist die FPU nicht länger eine separate Einheit in der P6 Intel Familie,
sodass es nun auch möglich ist, die Flags der CPU von der FPU aus zu prüfen oder
zu verändern.

Wir erhalten also das Folgende:

\lstinputlisting[caption=\Optimizing GCC
4.8.1,style=customasmx86]{patterns/12_FPU/3_comparison/x86/GCC481_O3_DE.s}

Schwer zu sagen, warum \INS{FXCH} (vertausche Operanden) hier verwendet wird.

Es ist möglich, diesen Befehl loszuwerden, indem man die ersten beiden \FLD
Befehle vertauscht oder \INS{FCMOVBE} (\IT{below or equal}) durch \INS{FCMOVA}
(\IT{above}) ersetzt.
Wahrscheinlich handelt es sich hierbei um eine Ungenauigkeit im Compiler.

\INS{FUCOMI} vergleicht also \ST{0} ($a$) und \ST{1} ($b$) und setzt einige
Flags in der CPU. 
\INS{FCMOVBE} prüft die Flags und kopiert \ST{1} (in diesem Moment also $b$)
nach \ST{0} (hier: $a$), falls $ST0 (a) <= ST1 (b)$.
Andernfalls ($a>b$) wird $a$ in \ST{0} belassen.

Der letzte \FSTP Befehl belässt \ST{0} oben auf dem Stack und verwirft den
Inhalt von \ST{1}. 

Verfolgen wir den Funktionsverlauf in GDB:

\lstinputlisting[caption=\Optimizing GCC 4.8.1 and GDB,numbers=left]{patterns/12_FPU/3_comparison/x86/gdb.txt}

Unter Verwendung von \q{ni} führen wir die ersten beiden \FLD Befehle aus.

Sehen wir uns die FPU Register (Zeile 33) an.

Wie bereits erwähnt, bildet der FPU Registersatz einen Ringpuffer und keinen
Stack (\myref{FPU_is_rather_circular_buffer}).
Außerdem zeigt GDB nicht die \GTT{STx} Register, sondern die internen FPU
Register (\GTT{Rx}). 
Der Pfeil (in Zeile 35) zeigt auf das aktuell obere Ende des Stacks.

Wir sehen auch den Inhalt des \GTT{TOP} Registers in \IT{Status Word} (Zeile
44)--hier ist dieser 6, sodass das oberste Element im Stack also aktuell auf das
interne Register 6 zeigt.

Die Werte von $a$ und $b$ werden nach Ausführung von \INS{FXCH} (Zeile 54)
vertauscht.

\INS{FUCOMI} wird ausgeführt (Zeile 83).
Betrachten wir die Flags: \CF ist gesetzt (Zeile 95).

\INS{FCMOVBE} hat den Wert von $b$ kopiert (siehe Zeile 104).

\FSTP lässt einen Wert oben auf dem Stack (Zeile 136).
Der Wert von \GTT{TOP} beträgt jetzt 7, was bedeutet, dass das obere Ende des
FPU Stacks jetzt auf das interne Register 7 zeigt.
