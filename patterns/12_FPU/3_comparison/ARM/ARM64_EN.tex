\subsubsection{ARM64}

\myparagraph{\Optimizing GCC (Linaro) 4.9}

\lstinputlisting[style=customasmARM]{patterns/12_FPU/3_comparison/ARM/ARM64_GCC_O3_EN.lst}

The ARM64 \ac{ISA} has FPU-instructions 
which set \ac{APSR} the CPU flags instead of \ac{FPSCR} for convenience.
The\ac{FPU} is not a separate device here anymore (at least, logically).
\myindex{ARM!\Instructions!FCMPE}
Here we see \INS{FCMPE}. It compares the two values passed in \RegD{0} and \RegD{1} (which are the first and second arguments of the function)
and sets \ac{APSR} flags (N, Z, C, V).

\myindex{ARM!\Instructions!FCSEL}
\INS{FCSEL} (\IT{Floating Conditional Select}) copies the value of \RegD{0} or \RegD{1} into \RegD{0} depending on the condition (\GTT{GT}---Greater Than),
and again, it uses flags in \ac{APSR} register instead of \ac{FPSCR}.

This is much more convenient, compared to the instruction set in older CPUs.

If the condition is true (\GTT{GT}), then the value of \RegD{0} 
is copied into \RegD{0} (i.e., nothing happens).
If the condition is not true, the value of \RegD{1} 
is copied into \RegD{0}.

\myparagraph{\NonOptimizing GCC (Linaro) 4.9}

\lstinputlisting[style=customasmARM]{patterns/12_FPU/3_comparison/ARM/ARM64_GCC_EN.lst}

Non-optimizing GCC is more verbose.

First, the function saves its input argument values in the local stack (\IT{Register Save Area}).
Then the code reloads these values into registers
\RegX{0}/\RegX{1} and finally copies them to 
\RegD{0}/\RegD{1} to be compared using \INS{FCMPE}. 
A lot of redundant code, 
but that is how non-optimizing compilers work.
\INS{FCMPE} compares the values and sets the \ac{APSR} flags.
At this moment, 
the compiler is not thinking yet about the more convenient \INS{FCSEL} instruction, so it proceed using old methods: 
using the \INS{BLE} instruction (\IT{Branch if Less than or Equal}).
In the first case ($a>b$), the value of $a$ gets loaded 
into \RegX{0}.
In the other case ($a<=b$), the value of $b$ gets loaded into 
\RegX{0}.
Finally, the value from \RegX{0} gets copied into \RegD{0}, 
because the return value needs to be in this 
register.

\mysubparagraph{\Exercise}

As an exercise, you can try optimizing this piece of code 
manually by removing redundant instructions and not introducing new ones (including \INS{FCSEL}).

\myparagraph{\Optimizing GCC (Linaro) 4.9---float}

Let's also rewrite this example to use \Tfloat instead of \Tdouble.

\begin{lstlisting}[style=customc]
float f_max (float a, float b)
{
	if (a>b)
		return a;

	return b;
};
\end{lstlisting}

\lstinputlisting[style=customasmARM]{patterns/12_FPU/3_comparison/ARM/ARM64_GCC_O3_float_EN.lst}

It is the same code, but the S-registers are used instead of D- ones.
It's because numbers of type \Tfloat are passed in 32-bit S-registers (which are in fact the lower parts of the 64-bit D-registers).

