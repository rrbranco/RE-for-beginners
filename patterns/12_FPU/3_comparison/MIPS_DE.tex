\subsubsection{MIPS}

\myindex{MIPS!\Registers!FCCR}
Der Koprozessor des MIPS Prozessors hat ein Condition Bit, welches in der FPU
gesetzt und in der CPU geprüft werden kann.

Frühere MIPS haben nur ein Condition Bit (genannt FCC0), spätere haben deren 8
(genannt FCC7-FCC0). 

Diese(s) Bit(s) befinden sich im Register FCCR.

\lstinputlisting[caption=\Optimizing GCC 4.4.5
(IDA)]{patterns/12_FPU/3_comparison/MIPS_O3_IDA_DE.lst}

\myindex{MIPS!\Instructions!C.LT.D}
\INS{C.LT.D} vergleicht zwei Werte. 
\GTT{LT} ist die Bedingung \q{Less Than} (weniger als).
\GTT{D} impliziert einen Wert vom Typ \Tdouble.
Abhängig vom Ergebnis des Vergleichs wird das FCC0 Condition Bit entweder
gesetzt oder gelöscht.

\myindex{MIPS!\Instructions!BC1T}
\myindex{MIPS!\Instructions!BC1F}
\INS{BC1T} prüft das FCC0 Bit und sprint, falls das Bit gesetzt ist.
\GTT{T} bedeutet, dass der Sprung ausgeführt wird, wenn das Bit gesetzt
(\q{True}) ist.
Daneben gibt es auch den Befehl \INS{BC1F}, der springt, wenn das Bit gelöscht
(\q{FALSE}) ist.

Abhängig vom Sprung wird einer der Funktionsargument in \$F0 abgelegt.
