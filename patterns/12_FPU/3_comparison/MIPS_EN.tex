\subsubsection{MIPS}

\myindex{MIPS!\Registers!FCCR}
The co-processor of the MIPS processor has a condition bit which can be set in the FPU and checked in the CPU.

Earlier MIPS-es have only one condition bit (called FCC0), later ones have 8 (called FCC7-FCC0).

This bit (or bits) are located in the register called FCCR.

\lstinputlisting[caption=\Optimizing GCC 4.4.5 (IDA),style=customasmMIPS]{patterns/12_FPU/3_comparison/MIPS_O3_IDA_EN.lst}

\myindex{MIPS!\Instructions!C.LT.D}
\INS{C.LT.D} compares two values. 
\GTT{LT} is the condition \q{Less Than}.
\GTT{D} implies values of type \Tdouble.
Depending on the result of the comparison, the FCC0 condition bit is either set or cleared.

\myindex{MIPS!\Instructions!BC1T}
\myindex{MIPS!\Instructions!BC1F}
\INS{BC1T} checks the FCC0 bit and jumps if the bit is set.
\GTT{T} means that the jump is to be taken if the bit is set (\q{True}).
There is also the instruction \INS{BC1F} which jumps if the bit is cleared (\q{False}).

Depending on the jump, one of function arguments is placed into \$F0.
