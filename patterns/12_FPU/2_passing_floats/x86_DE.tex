\subsubsection{x86}
Schauen wir uns an, was wir in MSVC 2010 erhalten:

\lstinputlisting[caption=MSVC 2010,style=customasmx86]{patterns/12_FPU/2_passing_floats/MSVC_DE.asm}

\myindex{x86!\Instructions!FLD}
\myindex{x86!\Instructions!FSTP}
\FLD und \FSTP verschieben Variablen zwischen Datensegment und dem FPU
Stack.\GTT{pow()}\footnote{eine Standard-C-Funktion, die eine Zahl potenziert}
nimmt beide Werte vom Stack der FPU und gibt ihr Ergebnis über das \ST{0} Register zurück. 
Die Funktion \printf nimmt 8 Byte vom lokalen Stack und interpretiert diese als
Variable von Typ \Tdouble.

Übrigens könnte hier auch ein Paar \MOV Befehle verwendet werden, um die Werte
aus dem Speicher zu holen und auf den Stack zu legen, denn die Werte sind im
Speicher im IEEE 754 Format abgelegt und pow() arbeitet mit diesem Format,
sodass keine Umwandlung notwendig ist.
Genau so wird es im folgenden Beispiel für ARM auch
gemacht:\myref{FPU_passing_floats_ARM}
