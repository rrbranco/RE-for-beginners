\section{\FPUChapterName}
\label{sec:FPU}

\newcommand{\FNURLSTACK}{\footnote{\href{http://go.yurichev.com/17123}{wikipedia.org/wiki/Stack\_machine}}}
\newcommand{\FNURLFORTH}{\footnote{\href{http://go.yurichev.com/17124}{wikipedia.org/wiki/Forth\_(programming\_language)}}}
\newcommand{\FNURLIEEE}{\footnote{\href{http://go.yurichev.com/17125}{wikipedia.org/wiki/IEEE\_floating\_point}}}
\newcommand{\FNURLSP}{\footnote{\href{http://go.yurichev.com/17126}{wikipedia.org/wiki/Single-precision\_floating-point\_format}}}
\newcommand{\FNURLDP}{\footnote{\href{http://go.yurichev.com/17127}{wikipedia.org/wiki/Double-precision\_floating-point\_format}}}
\newcommand{\FNURLEP}{\footnote{\href{http://go.yurichev.com/17128}{wikipedia.org/wiki/Extended\_precision}}}

\ac{FPU}~--- блок в процессоре работающий с числами с плавающей запятой.

Раньше он назывался \q{сопроцессором} и он стоит немного в стороне от \ac{CPU}.

\subsection{IEEE 754}

Число с плавающей точкой в формате IEEE 754 состоит из \IT{знака}, \IT{мантиссы}\footnote{\IT{significand} или \IT{fraction} 
в англоязычной литературе} и \IT{экспоненты}.

\subsection{x86}

Перед изучением \ac{FPU} в x86 полезно ознакомиться с тем как работают стековые машины\FNURLSTACK 
или ознакомиться с основами языка Forth\FNURLFORTH.

\myindex{Intel!80486}
\myindex{Intel!FPU}
Интересен факт, что в свое время (до 80486) сопроцессор был отдельным чипом на материнской плате, 
и вследствие его высокой цены, он не всегда присутствовал. Его можно было докупить и установить отдельно
\footnote{Например, Джон Кармак использовал в своей игре Doom числа с фиксированной запятой 
(\href{http://go.yurichev.com/17357}{ru.wikipedia.org/wiki/Число\_с\_фиксированной\_запятой}), хранящиеся
в обычных 32-битных \ac{GPR} (16 бит на целую часть и 16 на дробную),
чтобы Doom работал на 32-битных компьютерах без FPU, т.е. 80386 и 80486 SX.}.
Начиная с 80486 DX в состав процессора всегда входит FPU.

\myindex{x86!\Instructions!FWAIT}
Этот факт может напоминать такой рудимент как наличие инструкции \INS{FWAIT}, которая заставляет
\ac{CPU} ожидать, пока \ac{FPU} закончит работу.
Другой рудимент это тот факт, что опкоды \ac{FPU}-инструкций начинаются с т.н. \q{escape}-опкодов 
(\GTT{D8..DF}) как опкоды, передающиеся в отдельный сопроцессор.

\myindex{IEEE 754}
\label{FPU_is_stack}
FPU имеет стек из восьми 80-битных регистров: \ST{0}..\ST{7}.
Для краткости, IDA и \olly отображают \ST{0} как \GTT{ST},
что в некоторых учебниках и документациях означает \q{Stack Top} (\q{вершина стека}).
Каждый регистр может содержать число в формате IEEE 754\FNURLIEEE.

\subsection{ARM, MIPS, x86/x64 SIMD}

В ARM и MIPS FPU это не стек, а просто набор регистров.

Такая же идеология применяется в расширениях SIMD в процессорах x86/x64.

\subsection{\CCpp}

\myindex{float}
\myindex{double}
В стандартных \CCpp имеются два типа для работы с числами с плавающей запятой: 
\Tfloat (\IT{число одинарной точности}\FNURLSP, 32 бита)
\footnote{Формат представления чисел с плавающей точкой одинарной точности затрагивается в разделе 
\IT{\WorkingWithFloatAsWithStructSubSubSectionName}~(\myref{sec:floatasstruct}).}
и \Tdouble (\IT{число двойной точности}\FNURLDP, 64 бита).

В \InSqBrackets{\TAOCPvolII 246} мы можем найти что \IT{single-precision} означает, что значение с плавающей точкой может быть
помещено в одно [32-битное] машинное слово, а \IT{doulbe-precision} означает, что оно размещено в двух словах (64 бита).

\myindex{long double}
GCC также поддерживает тип \IT{long double} (\IT{extended precision}\FNURLEP, 80 бит), но MSVC~--- нет.

Несмотря на то, что \Tfloat занимает столько же места, сколько и \Tint на 32-битной архитектуре, 
представление чисел, разумеется, совершенно другое.

\ifdefined\RUSSIAN
\subsection{Простой пример}

Рассмотрим простой пример:
\fi

\ifdefined\ENGLISH
\subsection{Simple example}

Let's consider this simple example:
\fi

\ifdefined\GERMAN
\subsection{\DEph{}}

\DEph{}

\fi

\lstinputlisting[style=customc]{patterns/12_FPU/1_simple/simple.c}

\subsubsection{x86}

% subsubsections
\EN{\myparagraph{MSVC}

Compile it in MSVC 2010:

\lstinputlisting[caption=MSVC 2010: \ttf{},style=customasmx86]{patterns/12_FPU/1_simple/MSVC_EN.asm}

\FLD takes 8 bytes from stack and loads the number into the \ST{0} register, automatically converting 
it into the internal 80-bit format (\IT{extended precision}).

\myindex{x86!\Instructions!FDIV}

\FDIV divides the value in \ST{0} by the number stored at address \\
\GTT{\_\_real@40091eb851eb851f}~---the value 3.14 is encoded there. 
The assembly syntax doesn't support floating point numbers, so 
what we see here is the hexadecimal representation of 3.14 in 64-bit IEEE 754 format.

After the execution of \FDIV \ST{0} holds the \gls{quotient}.

\myindex{x86!\Instructions!FDIVP}

By the way, there is also the \FDIVP instruction, which divides \ST{1} by \ST{0}, 
popping both these values from stack and then pushing the result. 
If you know the Forth language\FNURLFORTH,
you can quickly understand that this is a stack machine\FNURLSTACK.

The subsequent \FLD instruction pushes the value of $b$ into the stack.

After that, the quotient is placed in \ST{1}, and \ST{0} has the value of $b$.

\myindex{x86!\Instructions!FMUL}

The next \FMUL instruction does multiplication: $b$ from \ST{0} is multiplied by value at\\
\GTT{\_\_real@4010666666666666} (the number 4.1 is there) and leaves the result in the \ST{0} register.

\myindex{x86!\Instructions!FADDP}

The last \FADDP instruction adds the two values at top of stack, storing the result in \ST{1} 
and then popping the value of \ST{0}, thereby leaving the result at the top of the stack, in \ST{0}.

The function must return its result in the \ST{0} register, 
so there are no any other instructions except the function epilogue after \FADDP.

\input{patterns/12_FPU/1_simple/olly_EN.tex}
}
\RU{\myparagraph{MSVC}

Компилируем в MSVC 2010:

\lstinputlisting[caption=MSVC 2010: \ttf{},style=customasmx86]{patterns/12_FPU/1_simple/MSVC_RU.asm}

\FLD берет 8 байт из стека и загружает их в регистр \ST{0}, автоматически конвертируя во внутренний 
80-битный формат (\IT{extended precision}).

\myindex{x86!\Instructions!FDIV}
\FDIV делит содержимое регистра \ST{0} на число, лежащее по адресу \GTT{\_\_real@40091eb851eb851f}~--- 
там закодировано значение 3,14. Синтаксис ассемблера не поддерживает подобные числа, 
поэтому мы там видим шестнадцатеричное представление числа 3,14 в формате IEEE 754.

После выполнения \FDIV в \ST{0} остается \glslink{quotient}{частное}.

\myindex{x86!\Instructions!FDIVP}
Кстати, есть ещё инструкция \FDIVP, которая делит \ST{1} на \ST{0}, 
выталкивает эти числа из стека и заталкивает результат. 
Если вы знаете язык Forth\FNURLFORTH, то это как раз оно и есть~--- стековая машина\FNURLSTACK.

Следующая \FLD заталкивает в стек значение $b$.

После этого в \ST{1} перемещается результат деления, а в \ST{0} теперь $b$.

\myindex{x86!\Instructions!FMUL}
Следующий \FMUL умножает $b$ из \ST{0} на значение \\
\GTT{\_\_real@4010666666666666} --- там лежит число 4,1~--- и оставляет результат в \ST{0}.

\myindex{x86!\Instructions!FADDP}
Самая последняя инструкция \FADDP складывает два значения из вершины стека 
в \ST{1} и затем выталкивает значение, лежащее в \ST{0}. 
Таким образом результат сложения остается на вершине стека в \ST{0}.

Функция должна вернуть результат в \ST{0}, так что больше ничего здесь не производится, 
кроме эпилога функции.

\input{patterns/12_FPU/1_simple/olly_RU.tex}
}
\DE{\myparagraph{MSVC}

Kompilieren mit MSVC 2010 liefert:

\lstinputlisting[caption=MSVC 2010: \ttf{},style=customasmx86]{patterns/12_FPU/1_simple/MSVC_DE.asm}

\FLD nimmt 8 Byte vom Stack und lädt die Zahl in das \ST{0} Register, wobei
diese automatisch in das interne 80-bit-Format (\IT{erweiterte Genauigkeit})
konvertiert wird.

\myindex{x86!\Instructions!FDIV}
\FDIV teilt den Wert in \ST{0} durch die Zahl, die an der Adresse
\GTT{\_\_real@40091eb851eb851f} gespeichert ist~---der Wert 3.14 ist hier
kodiert.
Die Syntax des Assemblers erlaubt keine Fließkommazahlen, sodass wir hier die
hexadezimale Darstellung von 3.14 im 64-bit IEEE 754 Format finden.

Nach der Ausführung von \FDIV enthält \ST{0} den \glslink{quotient}{Quotienten}.

\myindex{x86!\Instructions!FDIVP}
Es gibt übrigens auch noch den \FDIVP Befehl, welcher \ST{1} durch \ST{0}
teilt, beide Werte vom Stack holt und das Ergebnis ebenfalls auf dem Stack
ablegt.
Wer mit der Sprache Forth \FNURLFORTH vertraut ist, erkennt schnell, dass es sich
hier um eine Stackmaschine\FNURLSTACK handelt.

Der nachfolgende \FLD Befehl speichert den Wert von $b$ auf dem Stack.

Anschließend wir der Quotient in \ST{1} abgelegt und \ST{0} enthält den Wert von
$b$.

\myindex{x86!\Instructions!FMUL}
Der nächste \FMUL Befehl führt folgende Multiplikation aus: $b$ aus Register
\ST{0} wird mit dem Wert an der Speicherstelle \GTT{\_\_real@4010666666666666}
(hier befindet sich die Zahl 4.1) multipliziert und hinterlässt das Ergebnis im
\ST{ß} Register.

\myindex{x86!\Instructions!FADDP}
Der letzte \FADDP Befehl addiert die beiden Werte, die auf dem Stack zuoberst
liegen, speichet das Ergebnis in \ST{1} und holt dann den Wert von \ST{0} vom
Stack, wobei das oberste Element auf dem Stack in \ST{0} gespeichert wird.

Die Funktion muss ihr Ergebnis im \ST{0} Register zurückgeben, sodass außer dem
Funktionsepilog nach \FADDP keine weiteren Befehle mehr folgen.

\input{patterns/12_FPU/1_simple/olly_DE.tex}
}

\EN{\myparagraph{GCC}

GCC 4.4.1 (with \Othree option) emits the same code, just slightly different:

\lstinputlisting[caption=\Optimizing GCC 4.4.1,style=customasmx86]{patterns/12_FPU/1_simple/GCC_EN.asm}

The difference is that, first of all, 3.14 is pushed to the stack (into \ST{0}), and then the value 
in \GTT{arg\_0} is divided by the value in \ST{0}.

\myindex{x86!\Instructions!FDIVR}

\FDIVR stands for \IT{Reverse Divide}~---to divide with divisor and dividend swapped with each other. 
There is no likewise instruction for multiplication since it is 
a commutative operation, so we just have \FMUL without its \GTT{-R} counterpart.

\myindex{x86!\Instructions!FADDP}

\FADDP adds the two values but also pops one value from the stack. 
After that operation, \ST{0} holds the sum.

}
\RU{\myparagraph{GCC}

GCC 4.4.1 (с опцией \Othree) генерирует похожий код, хотя и с некоторой разницей:

\lstinputlisting[caption=\Optimizing GCC 4.4.1,style=customasmx86]{patterns/12_FPU/1_simple/GCC_RU.asm}

Разница в том, что в стек сначала заталкивается 3,14 (в \ST{0}), а затем значение 
из \GTT{arg\_0} делится на то, что лежит в регистре \ST{0}.

\myindex{x86!\Instructions!FDIVR}
\FDIVR означает \IT{Reverse Divide}~--- делить, поменяв делитель и делимое местами. 
Точно такой же инструкции для умножения нет, потому что она была бы бессмысленна (ведь умножение 
операция коммутативная), так что остается только \FMUL без соответствующей ей \GTT{-R} инструкции.

\myindex{x86!\Instructions!FADDP}
\FADDP не только складывает два значения, но также и выталкивает из стека одно значение. 
После этого в \ST{0} остается только результат сложения.
}
\DE{\myparagraph{GCC}

GCC 4.4.1 (mit der Option \Othree) erzeugt fast den gleichen Code, nur leicht
verändert.

\lstinputlisting[caption=\Optimizing GCC 4.4.1,style=customasmx86]{patterns/12_FPU/1_simple/GCC_DE.asm} Der Unterschied
besteht darin, dass zuerst 3.14 auf dem Stack (in \ST{0}) abgelegt wird und danach der Wert in \GTT{arg\_0} durch den Wert in \ST{0}
geteilt wird.

\myindex{x86!\Instructions!FDIVR}
\FDIVR steht für \IT{Reverse Divide}~--teilen, wobei Dividend und Divisor
miteinander vertauscht werden. Da es sich bei der Multiplikation um eine
kommutative Operation handelt, gibt es keinen vergleichbaren Befehl für die
Multiplikation. Wir haben es lediglich \FMUL ohne \GTT{-R} Gegenstück zur
Verfügung.

\myindex{x86!\Instructions!FADDP}
\FADDP addiert die beiden Werte und holt auch einen Wert vom Stack. 
Nach der Ausführung steht die Summe in \ST{0}.

}


\EN{\subsubsection{ARM: \OptimizingXcodeIV (\ARMMode)}

Until ARM got standardized floating point support, several processor manufacturers added their own 
instructions extensions.
Then, VFP (\IT{Vector Floating Point}) was standardized.

One important difference from x86 is that in ARM, there
is no stack, you work just with registers.

\lstinputlisting[label=ARM_leaf_example10,caption=\OptimizingXcodeIV (\ARMMode),style=customasmARM]{patterns/12_FPU/1_simple/ARM/Xcode_ARM_O3_EN.asm}

\myindex{ARM!D-\registers{}}
\myindex{ARM!S-\registers{}}

So, we see here new some registers used, with D prefix.

These are 64-bit registers, there are 32 of them, and they can be used both for floating-point numbers 
(double) but also for SIMD (it is called NEON here in ARM).

There are also 32 32-bit S-registers, intended to be used for single precision 
floating pointer numbers (float).

It is easy to memorize: D-registers are for double precision numbers, while
S-registers---for single precision numbers.
More about it: \myref{ARM_VFP_registers}.

Both constants (3.14 and 4.1) are stored in memory in IEEE 754 format.

\myindex{ARM!\Instructions!VLDR}
\myindex{ARM!\Instructions!VMOV}
\INS{VLDR} and \INS{VMOV}, as it can be easily deduced, are analogous to the \INS{LDR} and \MOV instructions,
but they work with D-registers.

It has to be noted that these instructions, just like the D-registers, are intended not only for
floating point numbers, 
but can be also used for SIMD (NEON) operations and this will also be shown soon.

The arguments are passed to the function in a common way, via the R-registers, however
each number that has double precision has a size of 64 bits, so two R-registers are needed to pass each one.

\INS{VMOV D17, R0, R1} at the start, composes two 32-bit values from \Reg{0} and \Reg{1} into one 64-bit value
and saves it to \GTT{D17}.

\INS{VMOV R0, R1, D16} is the inverse operation: what has been in \GTT{D16} 
is split in two registers, \Reg{0} and \Reg{1}, because a double-precision number 
that needs 64 bits for storage, is returned in \Reg{0} and \Reg{1}.

\myindex{ARM!\Instructions!VDIV}
\myindex{ARM!\Instructions!VMUL}
\myindex{ARM!\Instructions!VADD}
\INS{VDIV}, \INS{VMUL} and \INS{VADD}, 
are instruction for processing floating point numbers that compute \gls{quotient}, 
\gls{product} and sum, respectively.

The code for Thumb-2 is same.

\subsubsection{ARM: \OptimizingKeilVI (\ThumbMode)}

\lstinputlisting[style=customasmARM]{patterns/12_FPU/1_simple/ARM/Keil_O3_thumb_EN.asm}

Keil generated code for a processor without FPU or NEON support.

The double-precision floating-point numbers are passed via generic R-registers,
and instead of FPU-instructions, service library functions are called\\
(like \GTT{\_\_aeabi\_dmul}, \GTT{\_\_aeabi\_ddiv}, \GTT{\_\_aeabi\_dadd})
which emulate multiplication, division and addition for floating-point numbers.

Of course, that is slower than FPU-coprocessor, but it's still better than nothing.

By the way, similar FPU-emulating libraries were very popular in the x86 world when coprocessors were rare
and expensive, and were installed only on expensive computers.

\myindex{ARM!soft float}
\myindex{ARM!armel}
\myindex{ARM!armhf}
\myindex{ARM!hard float}

The FPU-coprocessor emulation is called \IT{soft float} or \IT{armel} (\IT{emulation}) in the ARM world, 
while using the coprocessor's FPU-instructions is called \IT{hard float} or \IT{armhf}.

\iffalse
% TODO разобраться...
\myindex{Raspberry Pi}

For example, the Linux kernel for Raspberry Pi is compiled in two variants.

In the \IT{soft float} case, arguments are passed via R-registers, and in the \IT{hard float} case---via D-registers.

And that is what stops you from using armhf-libraries from armel-code or vice versa,
and that is why all the code in Linux distributions must be compiled according to a single convention.
\fi

\subsubsection{ARM64: \Optimizing GCC (Linaro) 4.9}

Very compact code:

\lstinputlisting[caption=\Optimizing GCC (Linaro) 4.9,style=customasmARM]{patterns/12_FPU/1_simple/ARM/ARM64_GCC_O3_EN.s}

\subsubsection{ARM64: \NonOptimizing GCC (Linaro) 4.9}

\lstinputlisting[caption=\NonOptimizing GCC (Linaro) 4.9,style=customasmARM]{patterns/12_FPU/1_simple/ARM/ARM64_GCC_O0_EN.s}

\NonOptimizing GCC is more verbose.

There is a lot of unnecessary value shuffling, including some clearly redundant code 
(the last two \INS{FMOV} instructions). Probably, GCC 4.9 is not yet good in generating ARM64 code.

What is worth noting is that ARM64 has 64-bit registers, and the D-registers are 64-bit ones as well.

So the compiler is free to save values of type \Tdouble in \ac{GPR}s instead of the local stack.
This isn't possible on 32-bit CPUs.

And again, as an exercise, you can try to optimize this function manually, without introducing
new instructions like \INS{FMADD}.
}
\RU{\subsubsection{ARM: \OptimizingXcodeIV (\ARMMode)}

Пока в ARM не было стандартного набора инструкций для работы с числами с плавающей точкой, разные производители процессоров
могли добавлять свои расширения для работы с ними.
Позже был принят стандарт VFP (\IT{Vector Floating Point}).

Важное отличие от x86 в том, что там вы работаете с FPU-стеком, а здесь стека нет, вы работаете просто с регистрами.

\lstinputlisting[label=ARM_leaf_example10,caption=\OptimizingXcodeIV (\ARMMode),style=customasmARM]{patterns/12_FPU/1_simple/ARM/Xcode_ARM_O3_RU.asm}

\myindex{ARM!D-\registers{}}
\myindex{ARM!S-\registers{}}
Итак, здесь мы видим использование новых регистров с префиксом D.

Это 64-битные регистры. Их 32 и их можно
использовать для чисел с плавающей точкой двойной точности (double) и для 
SIMD (в ARM это называется NEON).

Имеются также 32 32-битных S-регистра. Они применяются для работы с числами 
с плавающей точкой одинарной точности (float).

Запомнить легко: D-регистры предназначены для чисел double-точности, 
а S-регистры~--- для чисел single-точности.

Больше об этом: \myref{ARM_VFP_registers}.

Обе константы (3,14 и 4,1) хранятся в памяти в формате IEEE 754.

\myindex{ARM!\Instructions!VLDR}
\myindex{ARM!\Instructions!VMOV}
Инструкции \INS{VLDR} и \INS{VMOV}, как можно догадаться, это аналоги обычных \INS{LDR} и \MOV, но они работают с D-регистрами.

Важно отметить, что эти инструкции, как и D-регистры, предназначены не только для работы 
с числами с плавающей точкой, но пригодны также и для работы с SIMD (NEON), и позже это также будет видно.

Аргументы передаются в функцию обычным путем через R-регистры, однако 
каждое число, имеющее двойную точность, занимает 64 бита, так что для передачи каждого нужны два R-регистра.

\INS{VMOV D17, R0, R1} в самом начале составляет два 32-битных значения из \Reg{0} и \Reg{1} в одно 64-битное и сохраняет в \GTT{D17}.

\INS{VMOV R0, R1, D16} в конце это обратная процедура: то что было в \GTT{D16} 
остается в двух регистрах \Reg{0} и \Reg{1}, потому что число с двойной точностью, 
занимающее 64 бита, возвращается в паре регистров \Reg{0} и \Reg{1}.

\myindex{ARM!\Instructions!VDIV}
\myindex{ARM!\Instructions!VMUL}
\myindex{ARM!\Instructions!VADD}
\INS{VDIV}, \INS{VMUL} и \INS{VADD}, это инструкции для работы с числами 
с плавающей точкой, вычисляющие, соответственно, \glslink{quotient}{частное}, \glslink{product}{произведение} и сумму.

Код для Thumb-2 такой же.

\subsubsection{ARM: \OptimizingKeilVI (\ThumbMode)}

\lstinputlisting[style=customasmARM]{patterns/12_FPU/1_simple/ARM/Keil_O3_thumb_RU.asm}

Keil компилировал для процессора, в котором может и не быть поддержки FPU или NEON.
Так что числа с двойной точностью передаются в парах обычных R-регистров,
а вместо FPU-инструкций вызываются сервисные библиотечные функции\\
\GTT{\_\_aeabi\_dmul}, \GTT{\_\_aeabi\_ddiv}, \GTT{\_\_aeabi\_dadd}, эмулирующие умножение, деление и сложение чисел с плавающей точкой.

Конечно, это медленнее чем FPU-сопроцессор, но это лучше, чем ничего.

Кстати, похожие библиотеки для эмуляции сопроцессорных инструкций были очень распространены в x86 
когда сопроцессор был редким и дорогим и присутствовал далеко не во всех компьютерах.

\myindex{ARM!soft float}
\myindex{ARM!armel}
\myindex{ARM!armhf}
\myindex{ARM!hard float}
Эмуляция FPU-сопроцессора в ARM называется \IT{soft float} или \IT{armel} (\IT{emulation}),
а использование FPU-инструкций сопроцессора~--- \IT{hard float} или \IT{armhf}.

\iffalse
% TODO разобраться...
\myindex{Raspberry Pi}
Ядро Linux, например, для Raspberry Pi может поставляться в двух вариантах.

В случае \IT{soft float}, аргументы будут передаваться через R-регистры, 
а в случае \IT{hard float}, через D-регистры.


И это то, что помешает использовать, например, armhf-библиотеки
из armel-кода или наоборот, поэтому, весь код в дистрибутиве Linux должен быть скомпилирован
в соответствии с выбранным соглашением о вызовах.

\fi

\subsubsection{ARM64: \Optimizing GCC (Linaro) 4.9}

Очень компактный код:

\lstinputlisting[caption=\Optimizing GCC (Linaro) 4.9,style=customasmARM]{patterns/12_FPU/1_simple/ARM/ARM64_GCC_O3_RU.s}

\subsubsection{ARM64: \NonOptimizing GCC (Linaro) 4.9}

\lstinputlisting[caption=\NonOptimizing GCC (Linaro) 4.9,style=customasmARM]{patterns/12_FPU/1_simple/ARM/ARM64_GCC_O0_RU.s}

\NonOptimizing GCC более многословный.
Здесь много ненужных перетасовок значений, включая явно избыточный код 
(последние две инструкции \INS{GMOV}).
Должно быть, GCC 4.9 пока ещё не очень хорош для генерации кода под ARM64.
Интересно заметить что у ARM64 64-битные регистры и D-регистры так же 64-битные.
Так что компилятор может сохранять значения типа \Tdouble в \ac{GPR} вместо локального стека.
Это было невозможно на 32-битных CPU.
И снова, как упражнение, вы можете попробовать соптимизировать эту функцию вручную, без добавления
новых инструкций вроде \INS{FMADD}.

}
\DE{%TODO
\subsubsection{ARM: \OptimizingXcodeIV (\ARMMode)}
Bis die Unterstützung für Fließkommaarithmetik in ARM standardisiert wurde,
fügten einige Hersteller von Prozessoren ihre eigenen Befehlserweiterungen
hinzu. 
Schließlich wurde VFP (\IT{Vector Floating Point}) standardisiert.

Ein wichtiger Unterschied zum x86 ist, dass in es in ARM keinen Stack gibt,
sondern man nur mit den Registern arbeitet.

\lstinputlisting[label=ARM_leaf_example10,caption=\OptimizingXcodeIV (\ARMMode),style=customasmARM]{patterns/12_FPU/1_simple/ARM/Xcode_ARM_O3_DE.asm}

\myindex{ARM!D-\registers{}}
\myindex{ARM!S-\registers{}}

Hier sehen wir, dass einige neue Register mit einem D als Präfix verwendet
werden.

Bei diesen handelt es sich um 64-bit-Register; es gibt 32 von ihnen und sie
können sowohl für Fließkommazahlen (doppelte Genauigkeit (double)) als auch für
SIMD (heißt hier in ARM NEON) benutzt werden.

Es gibt also 32 32-bit-S-Register vorgesehen für Fließkommazahlen in einfacher
Genauigkeit (float).

Es ist leicht zu merken: D-Register sind für Zahlen in doppelter Genauigkeit,
während S-Register für einfache Genauigkeit (engl. single) vorgesehen sind.
Mehr dazu hier:\myref{ARM_VFP_registers}

Beide Konstanten (3.14 und 4.1) werden im IEEE 754 Format im Speicher abgelegt.

\myindex{ARM!\Instructions!VLDR}
\myindex{ARM!\Instructions!VMOV}
Wie man leicht sieht sind \INS{VLDR} und \INS{VMOV} analog zu den \INS{LDR} und
\MOV Befehlen, aber arbeiten auf D-Registern.

Es muss angemerkt werden, dass diese Befehle genau wie die D-Register nicht nur
für Fließkommazahlen vorgesehen sind, sondern ebenfalls für SIMD (NEON)
Operationen verwendet werden können, was wir im folgenden zeigen werden.

Die Paraemter werden der Funktion auf übliche Weise über die R-Register
übergeben, aber da jede Zahl in doppelter Genauigkeit eine Größe von 64 Bit hat
werden jeweils zwei R-Register benötigt, um eine Zahl zu übergeben.

Der Befehl \INS{VMOV D17, R0, R1} zu Beginn, fasst zwei 32-Bit-Werte aus
\Reg{0} und \Reg{1} zu einem 64-Bit-Wert zusammen und speichert diesen in
\GTT{D17}.

\INS{VMOV R0, R1, D16} ist die umgekehrte Operation: was vorher in \GTT{D16}
war, wird in zwei Register, \Reg{0} und \Reg{1} aufgeteilt, denn eine Zahl in
doppelter Genauigkeit, die 64 Bit Speicherplatz benötigt, wird über \Reg{0} und
\Reg{1} zurückgegeben.

\myindex{ARM!\Instructions!VDIV}
\myindex{ARM!\Instructions!VMUL}
\myindex{ARM!\Instructions!VADD}
\INS{VDIV}, \INS{VMUL} und \INS{VADD} sind Befehle zur Verarbeitung von
Fließkommazahlen, die \glslink{quotient}{Quotient}, \glslink{product}{Produkt} bzw. Summe berechnen.

Der Code für Thumb-2 ist identisch.

\subsubsection{ARM: \OptimizingKeilVI (\ThumbMode)}

\lstinputlisting[style=customasmARM]{patterns/12_FPU/1_simple/ARM/Keil_O3_thumb_DE.asm}

Keil erzeugte Code für einen Prozessor ohne FPU oder NEON Unterstützung.

Die Fließkommazahlen in doppelter Genauigkeit werden über die üblichen
R-Register übergeben und anstelle von FPU-Befehlen werden Programmbibliotheken
(wie z.B. \GTT{\_\_aeabi\_dmul}, \GTT{\_\_aeabi\_ddiv}, \GTT{\_\_aeabi\_dadd})
aufgerufen, welche Multiplikation, Division und Addition auf Fließkommazahlen
emulieren. 

Diese Vorgehensweise ist natürlich langsamer als der FPU-Koprozessor, aber es
ist besser als nichts.

Übrigens waren ähnliche FPU-emulierende Programmbibliotheken auch in der
x86-Welt sehr beliebt als Koprozessoren selten und teuer waren und nur auf
wertvollen Computern installiert waren.

\myindex{ARM!soft float}
\myindex{ARM!armel}
\myindex{ARM!armhf}
\myindex{ARM!hard float}
Die Emulation des FPU-Koprozessors wird \IT{soft float} oder \IT{armel} (in der
ARM-Welt) genannt, wohingegen die FPU-Befehle des Koprozessors \IT{hard float}
oder \IT{armhf} genannt werden.

\iffalse
% TODO разобраться...
\myindex{Raspberry Pi}
Der Linux Kernel des Raspberry Pi beispielsweise wird in zwei Varianten
kompiliert.

Im Falle von \IT{soft float} werden Parameter über R-Register übergeben und im
Falle von \IT{hard float} über D-Register.

Diese Tatsache hält einen davon ab armhf-Programmbibliotheken für armel-Code
oder umgekehrt zu verwenden und dies ist der Grund warum der gesamte Code in
Linux-Distributionen speziell für eine der beiden Konventionen kompiliert wird.
\fi

\subsubsection{ARM64: \Optimizing GCC (Linaro) 4.9}

Sehr kompakter Code:

\lstinputlisting[caption=\Optimizing GCC (Linaro) 4.9,style=customasmARM]{patterns/12_FPU/1_simple/ARM/ARM64_GCC_O3_DE.s}

\subsubsection{ARM64: \NonOptimizing GCC (Linaro) 4.9}

\lstinputlisting[caption=\NonOptimizing GCC (Linaro) 4.9,style=customasmARM]{patterns/12_FPU/1_simple/ARM/ARM64_GCC_O0_DE.s}

\NonOptimizing GCC ist geschwätziger.
Hier findet eine Menge unnützes Verschieben von Werten statt, inklusive einigem
eindeutig redundantem Code (die letzten beiden \INS{FMOV} Befehle). Vermutlich
ist GCC 4.9 noch nicht besonders gut im Erzeugen von ARM64 Code.

Bemerkenswert ist, dass ARM64 64-Bit-Register besitzt und die D-Register
ebenfalls 64 Bit breit sind.

Dadurch steht es dem Compiler frei Werte von Typ \Tdouble in \ac{GPR}s anstelle
auf dem lokalen Stack zu speichern. Dies ist in 32-bit-CPUs nicht möglich.

Wiederum kann man als Übung versuchen diese Funktion manuell zu optimieren ohne
neue Befehl wie \INS{FMADD} einzuführen. 
}



\EN{\subsubsection{MIPS}

MIPS can support several coprocessors (up to 4), 
the zeroth of which is a special control coprocessor,
and first coprocessor is the FPU.

As in ARM, the MIPS coprocessor is not a stack machine, it has 32 32-bit registers (\$F0-\$F31):
\myref{MIPS_FPU_registers}.

When one needs to work with 64-bit \Tdouble values, a pair of 32-bit F-registers is used.

\lstinputlisting[caption=\Optimizing GCC 4.4.5 (IDA),style=customasmMIPS]{patterns/12_FPU/1_simple/MIPS_O3_IDA_EN.lst}

The new instructions here are:

\myindex{MIPS!\Instructions!LWC1}
\myindex{MIPS!\Instructions!DIV.D}
\myindex{MIPS!\Instructions!MUL.D}
\myindex{MIPS!\Instructions!ADD.D}
\begin{itemize}

\item \INS{LWC1} loads a 32-bit word into a register of the first coprocessor (hence \q{1} in instruction name).
\myindex{MIPS!\Pseudoinstructions!L.D}

A pair of \INS{LWC1} instructions may be combined into a \INS{L.D} pseudo instruction.

\item \INS{DIV.D}, \INS{MUL.D}, \INS{ADD.D} do division, multiplication, and addition respectively 
(\q{.D} in the suffix stands for double precision, \q{.S} stands for single precision)

\end{itemize}

\myindex{MIPS!\Instructions!LUI}
\myindex{\CompilerAnomaly}
\label{MIPS_FPU_LUI}

There is also a weird compiler anomaly: the \INS{LUI} instructions that we've marked with a question mark.
It's hard for me to understand why load a part of a 64-bit constant of \Tdouble type into the \$V0 register.
These instructions has no effect.
% TODO did you try checking out compiler source code?
If someone knows more about it, please drop an email to author\footnote{\EMAIL}.

}
\RU{\subsubsection{MIPS}

MIPS может поддерживать несколько сопроцессоров (вплоть до 4), нулевой из которых это специальный
управляющий сопроцессор, а первый~--- это FPU.

Как и в ARM, сопроцессор в MIPS это не стековая машина. Он имеет 32 32-битных регистра (\$F0-\$F31):

\myref{MIPS_FPU_registers}.
Когда нужно работать с 64-битными значениями типа \Tdouble, используется пара 32-битных F-регистров.

\lstinputlisting[caption=\Optimizing GCC 4.4.5 (IDA),style=customasmMIPS]{patterns/12_FPU/1_simple/MIPS_O3_IDA_RU.lst}

Новые инструкции:

\myindex{MIPS!\Instructions!LWC1}
\myindex{MIPS!\Instructions!DIV.D}
\myindex{MIPS!\Instructions!MUL.D}
\myindex{MIPS!\Instructions!ADD.D}
\begin{itemize}

\item \INS{LWC1} загружает 32-битное слово в регистр первого сопроцессора (отсюда \q{1} в названии инструкции).

\myindex{MIPS!\Pseudoinstructions!L.D}
Пара инструкций \INS{LWC1} может быть объединена в одну псевдоинструкцию \INS{L.D}.

\item \INS{DIV.D}, \INS{MUL.D}, \INS{ADD.D} производят деление, умножение и сложение соответственно 
(\q{.D} в суффиксе означает двойную точность, \q{.S}~--- одинарную точность)

\end{itemize}

\myindex{MIPS!\Instructions!LUI}
\myindex{\CompilerAnomaly}
\label{MIPS_FPU_LUI}
Здесь также имеется странная аномалия компилятора: инструкция \INS{LUI} помеченная нами вопросительным знаком.%

Мне трудно понять, зачем загружать часть 64-битной константы типа \Tdouble в регистр \$V0.

От этих инструкций нет толка.
% TODO did you try checking out compiler source code?
Если кто-то об этом что-то знает, пожалуйста, напишите автору емейл \footnote{\EMAIL}.

}
\DE{\subsubsection{MIPS}

MIPS unterstütz mehrere Koprozessoren (bis zu 4); der nullte in ein spezeiller
Kontroll-Koprozessor und der erste Koprozessor ist die FPU.

Genau wie in ARM ist der MIPS Koprozessor keine Stackmaschine, sondern hat 32
32-bit-Register (\$F0-\$F31):
\myref{MIPS_FPU_registers}.

Muss man mit 64-bit \Tdouble Werten arbeiten, wird ein Paar 32-bit F-Register
hierfür verwendet.

\lstinputlisting[caption=\Optimizing GCC 4.4.5
(IDA)]{patterns/12_FPU/1_simple/MIPS_O3_IDA_DE.lst}

Die neuen Befehl sind im Einzelnen:

\myindex{MIPS!\Instructions!LWC1}
\myindex{MIPS!\Instructions!DIV.D}
\myindex{MIPS!\Instructions!MUL.D}
\myindex{MIPS!\Instructions!ADD.D}
\begin{itemize}

\item \INS{LWC1} lädt ein 32-bit-Wort in ein Register des ersten Koprozessors
(daher \q{1} im Namen des Befehls).
\myindex{MIPS!\Pseudoinstructions!L.D}

Ein Parr \INS{LWC1} Befehle kann zu einem \INS{L.D} Pseudobefehl zusammengefasst
werden.

\item \INS{DIV.D}, \INS{MUL.D}, \INS{ADD.D} führen Division, Multiplikation bzw.
Addition aus (das \q{D.} im Suffix steht für doppelte Genauigkeit, \q{S.}
bedeutet entsprechend einfache Genauigkeit).

\end{itemize}

\myindex{MIPS!\Instructions!LUI}
\myindex{\CompilerAnomaly}
\label{MIPS_FPU_LUI}
Es gibt ein verrückte Anomalie im Compiler: die \INS{LUI} Befehle, die wir mit
einem Fragezeichen versehen haben. 
Es ist sehr schwer zu verstehen, warum ein Teil einer
64-bit-Konstante vom Typ \Tdouble in das \$V0 Register geladen wird. 
Dieser Befehl hat keine Auswirkungen. 

% TODO did you try checking out compiler source code?
Sollte jemand mehr zu dieser Anomalie wissen, bittet der Autor um eine
Mail\footnote{\EMAIL}.

}


\subsection{\RU{Передача чисел с плавающей запятой в аргументах}\EN{Passing floating point numbers via arguments}\DEph{}}
\myindex{\CStandardLibrary!pow()}

\lstinputlisting[style=customc]{patterns/12_FPU/2_passing_floats/pow.c}

\EN{\subsubsection{x86}

Let's see what we get in (MSVC 2010):

\lstinputlisting[caption=MSVC 2010,style=customasmx86]{patterns/12_FPU/2_passing_floats/MSVC_EN.asm}

\myindex{x86!\Instructions!FLD}
\myindex{x86!\Instructions!FSTP}

\FLD and \FSTP move variables between the data segment and the FPU stack. 
\GTT{pow()}\footnote{a standard C function, raises a number to the given power (exponentiation)}
takes both values from the stack of the FPU and 
returns its result in the \ST{0} register.
\printf takes 8 bytes from the local stack and interprets them as \Tdouble type variable.

By the way, a pair of \MOV instructions could be used here for moving values from the memory
into the stack, because the values in memory are stored in IEEE 754 format, and pow() also takes them in this
format, so no conversion is necessary.
That's how it's done in the next example, for ARM: \myref{FPU_passing_floats_ARM}.

}
\RU{\subsubsection{x86}

Посмотрим, что у нас вышло (MSVC 2010):

\lstinputlisting[caption=MSVC 2010,style=customasmx86]{patterns/12_FPU/2_passing_floats/MSVC_RU.asm}

\myindex{x86!\Instructions!FLD}
\myindex{x86!\Instructions!FSTP}
\FLD и \FSTP перемещают переменные из сегмента данных в FPU-стек или обратно. 
\GTT{pow()}\footnote{стандартная функция Си, возводящая число в степень} достает оба значения из FPU-стека и 
возвращает результат в \ST{0}. 
\printf берет 8 байт из стека и трактует их как переменную типа \Tdouble.

Кстати, с тем же успехом можно было бы перекладывать эти два числа из памяти в стек при помощи пары \MOV:
 
ведь в памяти числа в формате IEEE 754, pow() также принимает их в том же
формате, и никакая конверсия не требуется.

Собственно, так и происходит в следующем примере с ARM: \myref{FPU_passing_floats_ARM}.

}
\DE{\subsubsection{x86}
Schauen wir uns an, was wir in MSVC 2010 erhalten:

\lstinputlisting[caption=MSVC 2010,style=customasmx86]{patterns/12_FPU/2_passing_floats/MSVC_DE.asm}

\myindex{x86!\Instructions!FLD}
\myindex{x86!\Instructions!FSTP}
\FLD und \FSTP verschieben Variablen zwischen Datensegment und dem FPU
Stack.\GTT{pow()}\footnote{eine Standard-C-Funktion, die eine Zahl potenziert}
nimmt beide Werte vom Stack der FPU und gibt ihr Ergebnis über das \ST{0} Register zurück. 
Die Funktion \printf nimmt 8 Byte vom lokalen Stack und interpretiert diese als
Variable von Typ \Tdouble.

Übrigens könnte hier auch ein Paar \MOV Befehle verwendet werden, um die Werte
aus dem Speicher zu holen und auf den Stack zu legen, denn die Werte sind im
Speicher im IEEE 754 Format abgelegt und pow() arbeitet mit diesem Format,
sodass keine Umwandlung notwendig ist.
Genau so wird es im folgenden Beispiel für ARM auch
gemacht:\myref{FPU_passing_floats_ARM}
}

\EN{\subsubsection{ARM + \NonOptimizingXcodeIV (\ThumbTwoMode)}
\label{FPU_passing_floats_ARM}

\lstinputlisting[style=customasmARM]{patterns/12_FPU/2_passing_floats/Xcode_thumb_O0.asm}

As it was mentioned before, 64-bit floating pointer numbers are passed in R-registers pairs.

This code is a bit redundant (certainly because optimization is turned off), 
since it is possible to load values into the R-registers directly without touching the D-registers.

So, as we see, the \GTT{\_pow} function receives its first argument in \Reg{0} and \Reg{1}, and its second one in \Reg{2} and \Reg{3}. 
The function leaves its result in \Reg{0} and \Reg{1}.
The result of \GTT{\_pow} is moved into \GTT{D16}, then in the \Reg{1} and \Reg{2} pair, from where \printf takes the resulting number.

\subsubsection{ARM + \NonOptimizingKeilVI (\ARMMode)}

\lstinputlisting[style=customasmARM]{patterns/12_FPU/2_passing_floats/Keil_ARM_O0.asm}

D-registers are not used here, just R-register pairs.

\subsubsection{ARM64 + \Optimizing GCC (Linaro) 4.9}

\lstinputlisting[caption=\Optimizing GCC (Linaro) 4.9,style=customasmARM]{patterns/12_FPU/2_passing_floats/ARM64_EN.s}

The constants are loaded into \RegD{0} and \RegD{1}: \TT{pow()} takes them from there.
The result will be in \RegD{0} after the execution of \TT{pow()}.
It is to be passed to \printf without any modification and moving, 
because \printf takes arguments of \glslink{integral type}{integral types} 
and pointers from X-registers, and floating point arguments from D-registers.

}
\RU{\subsubsection{ARM + \NonOptimizingXcodeIV (\ThumbTwoMode)}
\label{FPU_passing_floats_ARM}

\lstinputlisting[style=customasmARM]{patterns/12_FPU/2_passing_floats/Xcode_thumb_O0.asm}

Как уже было указано, 64-битные числа с плавающей точкой передаются в парах R-регистров.

Этот код слегка избыточен (наверное, потому что не включена оптимизация), ведь можно было бы 
загружать значения напрямую в R-регистры минуя загрузку в D-регистры.

Итак, видно, что функция \GTT{\_pow} получает первый аргумент в \Reg{0} и \Reg{1}, а второй в \Reg{2} и \Reg{3}. 
Функция оставляет результат в \Reg{0} и \Reg{1}.
Результат работы \GTT{\_pow} перекладывается в \GTT{D16}, 
затем в пару \Reg{1} и \Reg{2}, откуда 
\printf берет это число-результат.

\subsubsection{ARM + \NonOptimizingKeilVI (\ARMMode)}

\lstinputlisting[style=customasmARM]{patterns/12_FPU/2_passing_floats/Keil_ARM_O0.asm}

Здесь не используются D-регистры, используются только пары R-регистров.

\subsubsection{ARM64 + \Optimizing GCC (Linaro) 4.9}

\lstinputlisting[caption=\Optimizing GCC (Linaro) 4.9,style=customasmARM]{patterns/12_FPU/2_passing_floats/ARM64_RU.s}

Константы загружаются в \RegD{0} и \RegD{1}: 
функция \TT{pow()} берет их оттуда.
Результат в \RegD{0} после исполнения \TT{pow()}.
Он пропускается в \printf без всякой модификации и перемещений, 
потому что \printf берет аргументы \glslink{integral type}{интегральных типов} и указатели 
из X-регистров, а аргументы типа плавающей точки из D-регистров.

}
\DE{\subsubsection{ARM + \NonOptimizingXcodeIV (\ThumbTwoMode)}
\label{FPU_passing_floats_ARM}

\lstinputlisting[style=customasmARM]{patterns/12_FPU/2_passing_floats/Xcode_thumb_O0.asm}
Wie bereits vorher erwähnt werden Pointer auf 64-Bit-Fließkommazahlen über ein
Paar von R-Registern übergeben.

Dieser Code ist leicht redundant (sicherlich aufgrund der deaktivierten
Optimierung), da es möglich ist Werte direkt in die R-Register zu laden, ohne
die D-Register zu verwenden.

Wie wir also sehen erhält die \GTT{\_pow} Funktion ihr erster Argument in
\Reg{0} und \Reg{1} und das zweite in \Reg{2} und \Reg{3}. Die Funktion
speichert ihr Ergebnis in \Reg{0} und \Reg{1}. 
Das Ergebnis von \GTT{\_pow} wird zunächst nach \GTT{D16} und
anschließend in das Paar \Reg{1} und \Reg{2} verschoben, von wo aus \printf das
Ergebnis übernimmt. 

\subsubsection{ARM + \NonOptimizingKeilVI (\ARMMode)}

\lstinputlisting[style=customasmARM]{patterns/12_FPU/2_passing_floats/Keil_ARM_O0.asm}

Die D-Register werden hier nicht verwendet, sondern nur Paare von R-Registern.

\subsubsection{ARM64 + \Optimizing GCC (Linaro) 4.9}

\lstinputlisting[caption=\Optimizing GCC (Linaro) 4.9,style=customasmARM]{patterns/12_FPU/2_passing_floats/ARM64_DE.s}

Die Konstanten werden nach \RegD{0} und \RegD{1} geladen: \TT{pow()} übernimmt
sie von dort. Das Ergebnis befindet sich nach der Ausführung von \TT{pow()} in
\RegD{0}. 
Es wird ohne weitere Änderung oder Verschiebung an die Funktion \printf
übergeben, da \printf ganzzahlige Werte und Pointer aus X-Registern,
Fließkommaparameter jedoch aus D-Registern übernimmt.

}

\EN{\subsubsection{MIPS}

\lstinputlisting[caption=\Optimizing GCC 4.4.5 (IDA),style=customasmMIPS]{patterns/12_FPU/2_passing_floats/MIPS_O3_IDA_EN.lst}

And again, we see here \INS{LUI} loading a 32-bit part of a \Tdouble number into \$V0.
And again, it's hard to comprehend why.

\myindex{MIPS!\Instructions!MFC1}

The new instruction for us here is \INS{MFC1} (\q{Move From Coprocessor 1}).
The FPU is coprocessor number 1, hence \q{1} in the instruction name.
This instruction transfers values from the coprocessor's registers to the registers of the CPU (\ac{GPR}).
So at the end the result of \TT{pow()} is moved to registers \$A3 and \$A2, 
and \printf takes a 64-bit double value from this register pair.

}
\RU{\subsubsection{MIPS}

\lstinputlisting[caption=\Optimizing GCC 4.4.5 (IDA),style=customasmMIPS]{patterns/12_FPU/2_passing_floats/MIPS_O3_IDA_RU.lst}

И снова мы здесь видим, как \INS{LUI} загружает 32-битную часть числа типа \Tdouble в \$V0.
И снова трудно понять почему.

\myindex{MIPS!\Instructions!MFC1}
Новая для нас инструкция это \INS{MFC1} (\q{Move From Coprocessor 1}) (копировать из первого сопроцессора).
FPU это сопроцессор под номером 1, вот откуда \q{1} в имени инструкции.
Эта инструкция переносит значения из регистров сопроцессора в регистры основного CPU (\ac{GPR}).
Так что результат исполнения \TT{pow()} в итоге копируется в регистры \$A3 и \$A2
и из этой пары регистров \printf берет его как 64-битное значение типа \Tdouble.

}
\DE{\subsubsection{MIPS}

\lstinputlisting[caption=\Optimizing GCC 4.4.5
(IDA)]{patterns/12_FPU/2_passing_floats/MIPS_O3_IDA_DE.lst}
Und wieder sehen wir hier, dass der Befehl \INS{LUI} einen 32-Bit-Teil einer
\Tdouble Zahl nach \$V0 lädt.
Und wiederum ist es schwer nachzuvollziehen warum dies geschieht.

\myindex{MIPS!\Instructions!MFC1}
Der für uns neue Befehl an dieser Stelle ist \INS{MFC1}(\q{Move From Coprocessor
1}). Die Nummer des FPU-Koprozessors ist 1, daher die \q{1} im Namen des
Befehls. 
Dieser Befehl überträgt Werte aus den Registern des Koprozessors in die Register
der CPU (\ac{GPR}).
Auf diese Weise wird das Ergebnis von \TT{pow()} schließlich in die Register
\$A3 und \$A2 verschoben und \printf übernimmt einen 64-Bit-Wert von doppelter
Genauigkeit aus diesem Registerpaar.}


\subsection{\RU{Пример с сравнением}\EN{Comparison example}\DEph{}}

\RU{Попробуем теперь вот это:}\EN{Let's try this:}\DEph{}

\lstinputlisting[style=customc]{patterns/12_FPU/3_comparison/d_max.c}

\RU{Несмотря на кажущуюся простоту этой функции, понять, как она работает, будет чуть сложнее.}%
\EN{Despite the simplicity of the function, it will be harder to understand how it works.}%
\DEph{}

% subsections
\subsubsection{x86}

% subsubsections
\EN{\myparagraph{\NonOptimizing MSVC}

MSVC 2010 generates the following:

\lstinputlisting[caption=\NonOptimizing MSVC 2010,style=customasmx86]{patterns/12_FPU/3_comparison/x86/MSVC/MSVC_EN.asm}

\myindex{x86!\Instructions!FLD}

So, \FLD loads \GTT{\_b} into \ST{0}.

\label{Czero_etc}
\newcommand{\Czero}{\GTT{C0}\xspace}
\newcommand{\Ctwo}{\GTT{C2}\xspace}
\newcommand{\Cthree}{\GTT{C3}\xspace}
\newcommand{\CThreeBits}{\Cthree/\Ctwo/\Czero}

\myindex{x86!\Instructions!FCOMP}

\FCOMP compares the value in \ST{0} with what is in \GTT{\_a} 
and sets \CThreeBits bits in FPU status word register, accordingly. 
This is a 16-bit register that reflects the current state of the FPU.

After the bits are set, the \FCOMP instruction also pops one variable from the stack. 
This is what distinguishes it from \FCOM, which is just compares values, leaving the stack in the same state.

Unfortunately, CPUs before Intel P6
\footnote{Intel P6 is Pentium Pro, Pentium II, etc.} don't have any conditional 
jumps instructions which check the \CThreeBits bits. 
Perhaps, it is a matter of history (recall: FPU was a separate chip in past).\\
Modern CPU starting at Intel P6 have \FCOMI/\FCOMIP/\FUCOMI/\FUCOMIP 
instructions~---which do the same, but modify the \ZF/\PF/\CF CPU flags.

\myindex{x86!\Instructions!FNSTSW}

The \FNSTSW instruction copies FPU the status word register to \AX. 
\CThreeBits bits are placed at positions 14/10/8, 
they are at the same positions in the \AX register and all they are placed in the high part of \AX{}~---\AH{}.

\begin{itemize}
\item If $b>a$ in our example, then \CThreeBits bits are to be set as following: 0, 0, 0.
\item If $a>b$, then the bits are: 0, 0, 1.
\item If $a=b$, then the bits are: 1, 0, 0.
\item

If the result is unordered (in case of error), then the set bits are: 1, 1, 1.
\end{itemize}
% TODO: table here?

This is how \CThreeBits bits are located in the \AX register:

\input{C3_in_AX}

This is how \CThreeBits bits are located in the \AH register:

\input{C3_in_AH}

After the execution of \INS{test ah, 5}\footnote{5=101b}, 
only \Czero and \Ctwo bits (on 0 and 2 position) are considered, all other bits are just
ignored.

\label{parity_flag}
\myindex{x86!\Registers!\Flags!Parity flag}

Now let's talk about the \IT{parity flag}, another notable historical rudiment.

This flag is set to 1 if the number of ones in the result of the last calculation is even, and to 0 if it is odd.

Let's look into Wikipedia\footnote{\href{http://go.yurichev.com/17131}{wikipedia.org/wiki/Parity\_flag}}:

\begin{framed}
\begin{quotation}
One common reason to test the parity flag actually has nothing to do with parity. The FPU has four condition flags 
(C0 to C3), but they cannot be tested directly, and must instead be first copied to the flags register. 
When this happens, C0 is placed in the carry flag, C2 in the parity flag and C3 in the zero flag. 
The C2 flag is set when e.g. incomparable floating point values (NaN or unsupported format) are compared 
with the FUCOM instructions.
\end{quotation}
\end{framed}

As noted in Wikipedia, the parity flag used sometimes in FPU code, let's see how.

\myindex{x86!\Instructions!JP}

The \PF flag is to be set to 1 if both \Czero and \Ctwo are set to 0 or both are 1, in which case
the subsequent \JP (\IT{jump if PF==1}) is triggering. 
If we recall the values of \CThreeBits for various cases,
we can see that the conditional jump 
\JP is triggering in two cases: if $b>a$ or $a=b$ 
(\Cthree bit is not considered here, since it has been cleared by the \INS{test ah, 5} instruction).

It is all simple after that. 
If the conditional jump has been triggered, 
\FLD loads the value of \GTT{\_b} 
in \ST{0}, and if it hasn't been triggered, the value of \GTT{\_a} is loaded there.

\mysubparagraph{And what about checking \Ctwo?}

The \Ctwo flag is set in case of error (\gls{NaN}, etc.), but our code doesn't check it.

If the programmer cares about FPU errors, he/she must add additional checks.

\input{patterns/12_FPU/3_comparison/x86/MSVC/olly_EN.tex}
}
\RU{\myparagraph{\NonOptimizing MSVC}

Вот что выдал MSVC 2010:

\lstinputlisting[caption=\NonOptimizing MSVC 2010,style=customasmx86]{patterns/12_FPU/3_comparison/x86/MSVC/MSVC_RU.asm}

\myindex{x86!\Instructions!FLD}
Итак, \FLD загружает \GTT{\_b} в регистр \ST{0}.

\label{Czero_etc}
\newcommand{\Czero}{\GTT{C0}\xspace}
\newcommand{\Ctwo}{\GTT{C2}\xspace}
\newcommand{\Cthree}{\GTT{C3}\xspace}
\newcommand{\CThreeBits}{\Cthree/\Ctwo/\Czero}

\myindex{x86!\Instructions!FCOMP}
\FCOMP сравнивает содержимое \ST{0} с тем, что лежит в \GTT{\_a} и выставляет биты \CThreeBits в 
регистре статуса FPU. Это 16-битный регистр отражающий текущее состояние FPU.

После этого инструкция \FCOMP также выдергивает одно значение из стека. 
Это отличает её от \FCOM, которая просто сравнивает значения, оставляя стек в таком же состоянии.

К сожалению, у процессоров до Intel P6
\footnote{Intel P6 это Pentium Pro, Pentium II, и последующие модели} нет инструкций условного перехода,
проверяющих биты \CThreeBits.
Возможно, так сложилось исторически (вспомните о том, что FPU когда-то был вообще отдельным чипом).\\
А у Intel P6 появились инструкции \FCOMI/\FCOMIP/\FUCOMI/\FUCOMIP, делающие то же самое, 
только напрямую модифицирующие флаги \ZF/\PF/\CF.

\myindex{x86!\Instructions!FNSTSW}
Так что \FNSTSW копирует содержимое регистра статуса в \AX. 
Биты \CThreeBits занимают позиции, 
соответственно, 14, 10, 8. В этих позициях они и остаются в регистре \AX, 
и все они расположены в старшей части регистра~--- \AH.

\begin{itemize}
\item Если $b>a$ в нашем случае, то биты \CThreeBits должны быть выставлены так: 0, 0, 0.
\item Если $a>b$, то биты будут выставлены: 0, 0, 1.
\item Если $a=b$, то биты будут выставлены так: 1, 0, 0.
\item Если результат не определен (в случае ошибки), то биты будут выставлены так: 1, 1, 1.
\end{itemize}
% TODO: table here?

Вот как биты \CThreeBits расположены в регистре \AX:

\input{C3_in_AX}

Вот как биты \CThreeBits расположены в регистре \AH:

\input{C3_in_AH}

После исполнения \INS{test ah, 5}\footnote{5=101b} % FIXME: subscript here!
будут учтены только биты \Czero и \Ctwo (на позициях 0 и 2), остальные просто проигнорированы.

\label{parity_flag}
\myindex{x86!\Registers!\Flags!Флаг четности}
Теперь немного о \IT{parity flag}\footnote{флаг четности}. 
Ещё один замечательный рудимент эпохи.

Этот флаг выставляется в 1 если количество единиц в последнем результате четно. 
И в 0 если нечетно.

Заглянем в Wikipedia\footnote{\href{http://go.yurichev.com/17131}{wikipedia.org/wiki/Parity\_flag}}:

\begin{framed}
\begin{quotation}
One common reason to test the parity flag actually has nothing to do with parity. The FPU has four condition flags 
(C0 to C3), but they cannot be tested directly, and must instead be first copied to the flags register. 
When this happens, C0 is placed in the carry flag, C2 in the parity flag and C3 in the zero flag. 
The C2 flag is set when e.g. incomparable floating point values (NaN or unsupported format) are compared 
with the FUCOM instructions.
\end{quotation}
\end{framed}

Как упоминается в Wikipedia, флаг четности иногда используется в FPU-коде и сейчас мы увидим как.

\myindex{x86!\Instructions!JP}
Флаг \PF будет выставлен в 1, если \Czero и \Ctwo оба 1 или оба 0. 
И тогда сработает последующий \JP (\IT{jump if PF==1}). 
Если мы вернемся чуть назад и посмотрим значения \CThreeBits 
для разных вариантов, то увидим, что условный переход \JP сработает в двух случаях: если $b>a$ или если $a=b$ 
(ведь бит \Cthree перестал учитываться после исполнения \INS{test ah, 5}).

Дальше всё просто. Если условный переход сработал, то \FLD загрузит значение \INS{\_b} в \ST{0}, 
а если не сработал, то загрузится \GTT{\_a} и произойдет выход из функции.

\mysubparagraph{А как же проверка флага \Ctwo?}

Флаг \Ctwo включается в случае ошибки (\gls{NaN}, итд.), но наш код его не проверяет.

Если программисту нужно знать, не произошла ли FPU-ошибка, он должен позаботиться об этом
дополнительно, добавив соответствующие проверки.

\input{patterns/12_FPU/3_comparison/x86/MSVC/olly_RU.tex}
}

\EN{\myparagraph{\Optimizing MSVC 2010}

\lstinputlisting[caption=\Optimizing MSVC 2010,style=customasmx86]{patterns/12_FPU/3_comparison/x86/MSVC_Ox/MSVC_EN.asm}

\myindex{x86!\Instructions!FCOM}

\FCOM differs from \FCOMP in the sense that it just compares the values and doesn't change the FPU stack. 
Unlike the previous example, here the operands are in reverse order, 
which is why the result of the comparison in \CThreeBits is different:

\begin{itemize}
\item If $a>b$ in our example, then \CThreeBits bits are to be set as: 0, 0, 0.
\item If $b>a$, then the bits are: 0, 0, 1.
\item If $a=b$, then the bits are: 1, 0, 0.
\end{itemize}
% TODO: table?

The \INS{test ah, 65} instruction leaves just two bits~---\Cthree and \Czero. 
Both will be zero if $a>b$: in that case the \JNE jump will not be triggered. 
Then \INS{FSTP ST(1)} follows~---this instruction copies the value from \ST{0} to the operand and 
pops one value from the FPU stack.
In other words, the instruction copies \ST{0} (where the value of \GTT{\_a} is now) into \ST{1}.
After that, two copies of {\_a} are at the top of the stack. 
Then, one value is popped.
After that, \ST{0} contains {\_a} and the function is finishes.

The conditional jump \JNE is triggering in two cases: if $b>a$ or $a=b$. 
\ST{0} is copied into \ST{0}, it is just like an idle (\ac{NOP}) operation, then one value 
is popped from the stack and the top of the stack (\ST{0}) is contain what has been in \ST{1} before 
(that is {\_b}). 
Then the function finishes. 
The reason this instruction is used here probably is because the \ac{FPU} 
has no other instruction to pop a value from the stack and discard it.

\input{patterns/12_FPU/3_comparison/x86/MSVC_Ox/olly_EN.tex}
}
\RU{\myparagraph{\Optimizing MSVC 2010}

\lstinputlisting[caption=\Optimizing MSVC 2010,style=customasmx86]{patterns/12_FPU/3_comparison/x86/MSVC_Ox/MSVC_RU.asm}

\myindex{x86!\Instructions!FCOM}
\FCOM отличается от \FCOMP тем, что просто сравнивает значения и оставляет стек в том же состоянии. 
В отличие от предыдущего примера, операнды здесь в обратном порядке. 
Поэтому и результат сравнения в \CThreeBits будет отличаться:

\begin{itemize}
\item Если $a>b$, то биты \CThreeBits должны быть выставлены так: 0, 0, 0.
\item Если $b>a$, то биты будут выставлены так: 0, 0, 1.
\item Если $a=b$, то биты будут выставлены так: 1, 0, 0.
\end{itemize}
% TODO: table?

Инструкция \INS{test ah, 65} как бы оставляет только два бита~--- \Cthree и \Czero. 
Они оба будут нулями, если $a>b$: в таком случае переход \JNE не сработает. 
\myindex{ARM!\Instructions!FSTP}
Далее имеется инструкция \INS{FSTP ST(1)}~--- эта инструкция копирует 
значение \ST{0} в указанный операнд и выдергивает одно значение из стека. В данном случае, 
она копирует \ST{0} 
(где сейчас лежит~\GTT{\_a})~в~\ST{1}. 
После этого на вершине стека два раза лежит~\GTT{\_a}. Затем одно значение выдергивается. 
После этого в \ST{0} остается~\GTT{\_a} и функция завершается.

Условный переход \JNE сработает в двух других случаях: если $b>a$ или $a=b$. 
\ST{0} скопируется в \ST{0} (как бы холостая операция). 
Затем одно значение из стека вылетит и на вершине стека останется то, что 
до этого лежало в \ST{1} (то~есть~\GTT{\_b}). И функция завершится. 
Эта инструкция используется здесь видимо потому что в FPU 
нет другой инструкции, которая просто выдергивает 
значение из стека и выбрасывает его.

\input{patterns/12_FPU/3_comparison/x86/MSVC_Ox/olly_RU.tex}
}

\EN{\myparagraph{GCC 4.4.1}

\lstinputlisting[caption=GCC 4.4.1,style=customasmx86]{patterns/12_FPU/3_comparison/x86/GCC_EN.asm}

\myindex{x86!\Instructions!FUCOMPP}

\FUCOMPP{} is almost like \FCOM, but pops both values from the stack and handles
\q{not-a-numbers} differently.

\myindex{Non-a-numbers (NaNs)}
A bit about \IT{not-a-numbers}.

\newcommand{\NANFN}{\footnote{\href{http://go.yurichev.com/17130}{wikipedia.org/wiki/NaN}}}

The FPU is able to deal with special values which are \IT{not-a-numbers} or \gls{NaN}s\NANFN. 
These are infinity, result of division by 0, etc.
Not-a-numbers can be \q{quiet} and \q{signaling}. It is possible to continue to work with \q{quiet} NaNs, 
but if one tries to do any operation with \q{signaling} NaNs, an exception is to be raised.

\myindex{x86!\Instructions!FCOM}
\myindex{x86!\Instructions!FUCOM}

\FCOM raising an exception if any operand is \gls{NaN}. 
\FUCOM raising an exception only if any operand is a signaling \gls{NaN} (SNaN).

\myindex{x86!\Instructions!SAHF}
\label{SAHF}

The next instruction is \SAHF (\IT{Store AH into Flags})~---this is a rare 
instruction in code not related to the FPU. 
8 bits from AH are moved into the lower 8 bits of the CPU flags in the following order:

\input{SAHF_LAHF}

\myindex{x86!\Instructions!FNSTSW}

Let's recall that \FNSTSW moves the bits that interest us (\CThreeBits) into \AH 
and they are in positions 6, 2, 0 of the \AH register:

\input{C3_in_AH}

In other words, the \INS{fnstsw  ax / sahf} instruction pair moves \CThreeBits into \ZF, \PF and \CF.

Now let's also recall the values of \CThreeBits in different conditions:

\begin{itemize}
\item If $a$ is greater than $b$ in our example, then \CThreeBits are to be set to: 0, 0, 0.
\item if $a$ is less than $b$, then the bits are to be set to: 0, 0, 1.
\item If $a=b$, then: 1, 0, 0.
\end{itemize}
% TODO: table?

In other words, these states of the CPU flags are possible
after three \\
\FUCOMPP/\FNSTSW/\SAHF instructions:

\begin{itemize}
\item If $a>b$, the CPU flags are to be set as: \GTT{ZF=0, PF=0, CF=0}.
\item If $a<b$, then the flags are to be set as: \GTT{ZF=0, PF=0, CF=1}.
\item And if $a=b$, then: \GTT{ZF=1, PF=0, CF=0}.
\end{itemize}
% TODO: table?

\myindex{x86!\Instructions!SETcc}
\myindex{x86!\Instructions!JNBE}

Depending on the CPU flags and conditions, \SETNBE stores 1 or 0 to AL. 
It is almost the counterpart of \JNBE, with the exception that \SETcc 
\footnote{\IT{cc} is \IT{condition code}} stores 1 or 0 in \AL, 
but \Jcc does actually jump or not. 
\SETNBE stores 1 only if \GTT{CF=0} and \GTT{ZF=0}. 
If it is not true, 0 is to be stored into \AL.

Only in one case both \CF and \ZF are 0: if $a>b$.

Then 1 is to be stored to \AL, the subsequent \JZ is not to be triggered and the function will return {\_a}. 
In all other cases, {\_b} is to be returned.

}
\RU{\myparagraph{GCC 4.4.1}

\lstinputlisting[caption=GCC 4.4.1,style=customasmx86]{patterns/12_FPU/3_comparison/x86/GCC_RU.asm}

\myindex{x86!\Instructions!FUCOMPP}
\FUCOMPP~--- это почти то же что и \FCOM, только выкидывает из стека оба значения после сравнения, 
а также несколько иначе реагирует на \q{не-числа}.

\myindex{Не-числа (NaNs)}
Немного о \IT{не-числах}.

\newcommand{\NANFN}{\footnote{\href{http://go.yurichev.com/17129}{ru.wikipedia.org/wiki/NaN}}}

FPU умеет работать со специальными переменными, которые числами не являются и называются \q{не числа} или 
\gls{NaN}\NANFN. 
Это бесконечность, результат деления на ноль, и так далее. Нечисла бывают \q{тихие} и \q{сигнализирующие}. 
С первыми можно продолжать работать и далее, а вот если вы попытаетесь совершить какую-то операцию 
с сигнализирующим нечислом, то сработает исключение.

\myindex{x86!\Instructions!FCOM}
\myindex{x86!\Instructions!FUCOM}
Так вот, \FCOM вызовет исключение если любой из операндов какое-либо нечисло.
\FUCOM же вызовет исключение только если один из операндов именно \q{сигнализирующее нечисло}.

\myindex{x86!\Instructions!SAHF}
\label{SAHF}
Далее мы видим \SAHF (\IT{Store AH into Flags})~--- это довольно редкая инструкция в коде, не использующим FPU. 
8 бит из \AH перекладываются в младшие 8 бит регистра статуса процессора в таком порядке:

\input{SAHF_LAHF}

\myindex{x86!\Instructions!FNSTSW}
Вспомним, что \FNSTSW перегружает интересующие нас биты \CThreeBits в \AH, 
и соответственно они будут в позициях 6, 2, 0 в регистре \AH:

\input{C3_in_AH}

Иными словами, пара инструкций \INS{fnstsw  ax / sahf} перекладывает биты \CThreeBits в флаги \ZF, \PF, \CF.

Теперь снова вспомним, какие значения бит \CThreeBits будут при каких результатах сравнения:

\begin{itemize}
\item Если $a$ больше $b$ в нашем случае, то биты \CThreeBits должны быть выставлены так: 0, 0, 0.
\item Если $a$ меньше $b$, то биты будут выставлены так: 0, 0, 1.
\item Если $a=b$, то так: 1, 0, 0.
\end{itemize}
% TODO: table?

Иными словами, после трех инструкций \FUCOMPP/\FNSTSW/\SAHF возможны такие состояния флагов:

\begin{itemize}
\item Если $a>b$ в нашем случае, то флаги будут выставлены так: \GTT{ZF=0, PF=0, CF=0}.
\item Если $a<b$, то флаги будут выставлены так: \GTT{ZF=0, PF=0, CF=1}.
\item Если $a=b$, то так: \GTT{ZF=1, PF=0, CF=0}.
\end{itemize}
% TODO: table?

\myindex{x86!\Instructions!SETcc}
\myindex{x86!\Instructions!JNBE}
Инструкция \SETNBE выставит в \AL единицу или ноль в зависимости от флагов и условий. 
Это почти аналог \JNBE, за тем лишь исключением, что \SETcc
\footnote{\IT{cc} это \IT{condition code}}
выставляет 1 или 0 в \AL, а \Jcc делает переход или нет. 
\SETNBE запишет 1 только если \GTT{CF=0} и \GTT{ZF=0}. Если это не так, то запишет 0 в \AL.

\CF будет 0 и \ZF будет 0 одновременно только в одном случае: если $a>b$.

Тогда в \AL будет записана 1, последующий условный переход \JZ выполнен не будет 
и функция вернет~\GTT{\_a}. 
В остальных случаях, функция вернет~\GTT{\_b}.
}

\EN{\myparagraph{\Optimizing GCC 4.4.1}

\lstinputlisting[caption=\Optimizing GCC 4.4.1,style=customasmx86]{patterns/12_FPU/3_comparison/x86/GCC_O3_EN.asm}

\myindex{x86!\Instructions!JA}

It is almost the same except that \JA is used after \SAHF. 
Actually, conditional jump instructions that check \q{larger}, \q{lesser} or \q{equal} for unsigned number comparison 
(these are \JA, \JAE, \JB, \JBE, \JE/\JZ, \JNA, \JNAE, \JNB, \JNBE, \JNE/\JNZ) check only flags \CF and \ZF.\\
\\
Let's recall where bits \CThreeBits are located in the \GTT{AH} register after the execution of \INS{FSTSW}/\FNSTSW:

\input{C3_in_AH}

Let's also recall, how the bits from \GTT{AH} are stored into the CPU flags the execution of \SAHF:

\input{SAHF_LAHF}

After the comparison, the \Cthree and \Czero bits are moved into \ZF and \CF, so the conditional jumps are able work after. \JA is triggering if both \CF are \ZF zero.

Thereby, the conditional jumps instructions listed here can be used after a \FNSTSW/\SAHF instruction pair.

Apparently, the FPU \CThreeBits status bits were placed there intentionally, to easily map them to base CPU flags without additional permutations?

}
\RU{\myparagraph{\Optimizing GCC 4.4.1}

\lstinputlisting[caption=\Optimizing GCC 4.4.1,style=customasmx86]{patterns/12_FPU/3_comparison/x86/GCC_O3_RU.asm}

\myindex{x86!\Instructions!JA}

Почти всё что здесь есть, уже описано мною, кроме одного: использование \JA после \SAHF. 
Действительно, инструкции условных переходов \q{больше}, \q{меньше} и \q{равно} для сравнения беззнаковых чисел 
(а это \JA, \JAE, \JB, \JBE, \JE/\JZ, \JNA, \JNAE, \JNB, \JNBE, \JNE/\JNZ) проверяют только флаги \CF и \ZF.\\
\\
Вспомним, как биты \CThreeBits располагаются в регистре \GTT{AH} после исполнения \INS{FSTSW}/\FNSTSW:

\input{C3_in_AH}

Вспомним также, как располагаются биты из \GTT{AH} во флагах CPU после исполнения \SAHF:

\input{SAHF_LAHF}

Биты \Cthree и \Czero после сравнения перекладываются в флаги \ZF и \CF так, что перечисленные инструкции переходов могут работать. \JA сработает, если \CF и \ZF обнулены.

Таким образом, перечисленные инструкции условного перехода можно использовать после инструкций \FNSTSW/\SAHF.

Может быть, биты статуса FPU \CThreeBits преднамеренно были размещены таким образом, чтобы переноситься на базовые флаги процессора без перестановок?

}

\EN{\myparagraph{GCC 4.8.1 with \Othree optimization turned on}
\label{gcc481_o3}

Some new FPU instructions were added in the P6 Intel family\footnote{Starting at Pentium Pro, Pentium-II, etc.}.
\myindex{x86!\Instructions!FUCOMI}
These are \INS{FUCOMI} (compare operands and set flags of the main CPU) and 
\myindex{x86!\Instructions!FCMOVcc}
\INS{FCMOVcc} (works like \INS{CMOVcc}, but on FPU registers).

Apparently, the maintainers of GCC decided to drop support of pre-P6 Intel CPUs (early Pentiums, 80486, etc.).

And also, the FPU is no longer separate unit in P6 Intel family, so now it is possible to modify/check flags of the main CPU from the FPU.

So what we get is:

\lstinputlisting[caption=\Optimizing GCC 4.8.1,style=customasmx86]{patterns/12_FPU/3_comparison/x86/GCC481_O3_EN.s}

Hard to guess why \INS{FXCH} (swap operands) is here.

It's possible to get rid of it easily by swapping the first two \FLD instructions or by replacing 
\INS{FCMOVBE} (\IT{below or equal}) by \INS{FCMOVA} (\IT{above}).
Probably it's a compiler inaccuracy.

So \INS{FUCOMI} compares \ST{0} ($a$) and \ST{1} ($b$) 
and then sets some flags in the main CPU.
\INS{FCMOVBE} checks the flags and copies \ST{1} 
($b$ here at the moment) to 
\ST{0} ($a$ here) if $ST0 (a) <= ST1 (b)$.
Otherwise ($a>b$), it leaves $a$ in \ST{0}.

The last \FSTP leaves \ST{0} on top of the stack, discarding the contents of \ST{1}.

Let's trace this function in GDB:

\lstinputlisting[caption=\Optimizing GCC 4.8.1 and GDB,numbers=left]{patterns/12_FPU/3_comparison/x86/gdb.txt}

Using \q{ni}, 
let's execute the first two \FLD instructions.

Let's examine the FPU registers (line 33).

As it was mentioned before, the FPU registers set is a circular buffer rather than a stack (\myref{FPU_is_rather_circular_buffer}).
And GDB doesn't show \GTT{STx} registers, but internal the FPU registers (\GTT{Rx}). 
The arrow (at line 35) points to the current top of the stack.

You can also see the \GTT{TOP} register contents in \IT{Status Word} (line 44)---it is 6 now, 
so the stack top is now pointing to internal register 6.

The values of $a$ and $b$ are swapped after \INS{FXCH} is executed (line 54).

\INS{FUCOMI} is executed (line 83). 
Let's see the flags: \CF is set (line 95).

\INS{FCMOVBE} has copied the value of $b$ (see line 104).

\FSTP leaves one value at the top of stack (line 136). 
The value of \GTT{TOP} is now 7, so the FPU stack top is pointing to internal register 7.

}
\RU{\myparagraph{GCC 4.8.1 с оптимизацией \Othree}
\label{gcc481_o3}

В линейке процессоров P6 от Intel 
появились новые FPU-инструкции\footnote{Начиная с Pentium Pro, Pentium-II, итд.}.
\myindex{x86!\Instructions!FUCOMI}
Это \INS{FUCOMI} (сравнить операнды и выставить флаги основного CPU) и
\myindex{x86!\Instructions!FCMOVcc}
\INS{FCMOVcc} (работает как \INS{CMOVcc}, но на регистрах FPU).
Очевидно, разработчики GCC решили отказаться от поддержки процессоров до линейки P6 (ранние Pentium, 80486, итд.).

И кстати, FPU уже давно не отдельная часть процессора в линейке P6, так что флаги основного CPU можно модифицировать из FPU.

Вот что имеем:

\lstinputlisting[caption=\Optimizing GCC 4.8.1,style=customasmx86]{patterns/12_FPU/3_comparison/x86/GCC481_O3_RU.s}

Не совсем понимаю, зачем здесь \INS{FXCH} (поменять местами операнды).

От нее легко избавиться поменяв местами инструкции \FLD либо заменив 
\INS{FCMOVBE} (\IT{below or equal}~--- меньше или равно) на 
\INS{FCMOVA} (\IT{above}~--- больше).

Должно быть, неаккуратность компилятора.

Так что \INS{FUCOMI} сравнивает \ST{0} ($a$) и \ST{1} ($b$) 
и затем устанавливает флаги основного CPU.
\INS{FCMOVBE} проверяет флаги и копирует \ST{1} 
(в тот момент там находится $b$) в 
\ST{0} (там $a$) если $ST0 (a) <= ST1 (b)$.
В противном случае ($a>b$), она оставляет $a$ в \ST{0}.

Последняя \FSTP оставляет содержимое \ST{0} на вершине стека, выбрасывая содержимое \ST{1}.

Попробуем оттрасировать функцию в GDB:

\lstinputlisting[caption=\Optimizing GCC 4.8.1 and GDB,numbers=left]{patterns/12_FPU/3_comparison/x86/gdb.txt}

Используя \q{ni}, дадим первым двум инструкциям \FLD исполниться.

Посмотрим регистры FPU (строка 33).

Как уже было указано ранее, регистры FPU это скорее кольцевой буфер, нежели стек (\myref{FPU_is_rather_circular_buffer}).
И GDB показывает не регистры \GTT{STx}, а внутренние регистры FPU (\GTT{Rx}). 
Стрелка (на строке 35) указывает на текущую вершину стека.

Вы можете также увидеть содержимое регистра \GTT{TOP} в \q{Status Word} (строка 44). Там сейчас 6, так что
вершина стека сейчас указывает на внутренний регистр 6.

Значения $a$ и $b$ меняются местами после исполнения \INS{FXCH} (строка 54).

\INS{FUCOMI} исполнилась (строка 83).
Посмотрим флаги: \CF выставлен (строка 95).

\INS{FCMOVBE} действительно скопировал значение $b$ (см. строку 104).

\FSTP оставляет одно значение на вершине стека (строка 136). 
Значение \GTT{TOP} теперь 7, так что вершина FPU-стека указывает на внутренний регистр 7.
}


\EN{\subsubsection{ARM}

\myparagraph{\OptimizingXcodeIV (\ARMMode)}

\lstinputlisting[caption=\OptimizingXcodeIV (\ARMMode),style=customasmARM]{patterns/12_FPU/3_comparison/ARM/Xcode_ARM_EN.lst}

\myindex{ARM!\Registers!APSR}
\myindex{ARM!\Registers!FPSCR}
A very simple case.
The input values are placed into the \GTT{D17} and \GTT{D16} registers and then compared using the \INS{VCMPE} instruction.

Just like in the x86 coprocessor, the ARM coprocessor has its own status and flags register (\ac{FPSCR}),
since there is a necessity to store coprocessor-specific flags.
% TODO -> расписать регистр по битам
\myindex{ARM!\Instructions!VMRS}
And just like in x86, there are no conditional jump instruction in ARM, 
that can check bits in the status register of the coprocessor. 
So there is \INS{VMRS}, which copies 4 bits (N, Z, C, V) from the coprocessor status word into bits of the \IT{general} status register (\ac{APSR}).

\myindex{ARM!\Instructions!VMOVGT}
\INS{VMOVGT} is the analog of the \INS{MOVGT}, 
instruction for D-registers, it executes if one operand is greater than the other while comparing (\IT{GT---Greater Than}). 

If it gets executed, the value of $a$ is to be written into \GTT{D16} (that is currently stored in in \GTT{D17}).
Otherwise the value of $b$ stays in the \GTT{D16} register.

\myindex{ARM!\Instructions!VMOV}

The penultimate instruction \INS{VMOV} prepares the value in the \GTT{D16} register for returning it via the \Reg{0} and \Reg{1}
register pair.

\myparagraph{\OptimizingXcodeIV (\ThumbTwoMode)}

\begin{lstlisting}[caption=\OptimizingXcodeIV (\ThumbTwoMode),style=customasmARM]
VMOV            D16, R2, R3 ; b
VMOV            D17, R0, R1 ; a
VCMPE.F64       D17, D16
VMRS            APSR_nzcv, FPSCR
IT GT 
VMOVGT.F64      D16, D17
VMOV            R0, R1, D16
BX              LR
\end{lstlisting}

Almost the same as in the previous example, however slightly different.
As we already know, many instructions in ARM mode can be supplemented by condition predicate.
But there is no such thing in Thumb mode. 
There is no space in the 16-bit instructions for 4 more bits in which conditions can be encoded.

\myindex{ARM!\ThumbTwoMode}

However, Thumb-2 was extended to make it possible to specify predicates to old Thumb instructions.
Here, in the \IDA-generated listing, we see the \INS{VMOVGT} instruction, as in previous example.

In fact, the usual \INS{VMOV} is encoded there, but \IDA adds the \GTT{-GT} suffix to it, 
since there is a \INS{IT GT} instruction placed right before it.

\label{ARM_Thumb_IT}
\myindex{ARM!\Instructions!IT}
\myindex{ARM!if-then block}
The \INS{IT} instruction defines a so-called \IT{if-then block}. 

After the instruction it is possible to place up to 4 instructions, 
each of them has a predicate suffix.
In our example, \INS{IT GT} implies that the next instruction is to be executed, if the \IT{GT} (\IT{Greater Than}) condition is true.

\myindex{Angry Birds}
Here is a more complex code fragment, by the way, from Angry Birds (for iOS):

\begin{lstlisting}[caption=Angry Birds Classic,style=customasmARM]
...
ITE NE
VMOVNE          R2, R3, D16
VMOVEQ          R2, R3, D17
BLX             _objc_msgSend ; not suffixed
...
\end{lstlisting}

\INS{ITE} stands for \IT{if-then-else} 

and it encodes suffixes for the next two instructions.

The first instruction executes if the condition encoded in \INS{ITE} (\IT{NE, not equal}) is true at, and the second---if the condition is not true.
(The inverse condition of \GTT{NE} is \GTT{EQ} (\IT{equal})).

The instruction followed after the second \INS{VMOV} (or \INS{VMOVEQ}) is a normal one, not suffixed (\INS{BLX}).

\myindex{Angry Birds}
One more that's slightly harder, which is also from Angry Birds:

\begin{lstlisting}[caption=Angry Birds Classic,style=customasmARM]
...
ITTTT EQ
MOVEQ           R0, R4
ADDEQ           SP, SP, #0x20
POPEQ.W         {R8,R10}
POPEQ           {R4-R7,PC}
BLX             ___stack_chk_fail ; not suffixed
...
\end{lstlisting}

Four \q{T} symbols in the instruction mnemonic mean that the four subsequent instructions are to be executed if the condition is true.

That's why \IDA adds the \GTT{-EQ} suffix to each one of them. 

And if there was, for example, \INS{ITEEE EQ} (\IT{if-then-else-else-else}), 
then the suffixes would have been set as follows:

\begin{lstlisting}
-EQ
-NE
-NE
-NE
\end{lstlisting}

\myindex{Angry Birds}
Another fragment from Angry Birds:

\begin{lstlisting}[caption=Angry Birds Classic,style=customasmARM]
...
CMP.W           R0, #0xFFFFFFFF
ITTE LE
SUBLE.W         R10, R0, #1
NEGLE           R0, R0
MOVGT           R10, R0
MOVS            R6, #0         ; not suffixed
CBZ             R0, loc_1E7E32 ; not suffixed
...
\end{lstlisting}

\INS{ITTE} (\IT{if-then-then-else}) 

implies that the 1st and 2nd instructions are to be executed if the \GTT{LE} (\IT{Less or Equal})
condition is true, and the 3rd---if the inverse condition (\GTT{GT}---\IT{Greater Than}) 
is true.

Compilers usually don't generate all possible combinations.
\myindex{Angry Birds}

For example, in the mentioned Angry Birds game (\IT{classic} version for iOS)
only these variants of the \INS{IT} instruction are used: 
\INS{IT}, \INS{ITE}, \INS{ITT}, \INS{ITTE}, \INS{ITTT}, \INS{ITTTT}.
\myindex{\GrepUsage}
How to learn this?
In \IDA It is possible to produce listing files, so it was created with an option to show 4 bytes for each opcode.
Then, knowing the high part of the 16-bit opcode (\INS{IT} is \GTT{0xBF}), we do the following using \GTT{grep}:

\begin{lstlisting}
cat AngryBirdsClassic.lst | grep " BF" | grep "IT" > results.lst
\end{lstlisting}

\myindex{ARM!\ThumbTwoMode}

By the way, if you program in ARM assembly language manually for Thumb-2 mode, 
and you add conditional suffixes,
the assembler will add the \INS{IT} instructions automatically with the required flags where it is necessary.

\myparagraph{\NonOptimizingXcodeIV (\ARMMode)}

\begin{lstlisting}[caption=\NonOptimizingXcodeIV (\ARMMode),style=customasmARM]
b               = -0x20
a               = -0x18
val_to_return   = -0x10
saved_R7        = -4

                STR             R7, [SP,#saved_R7]!
                MOV             R7, SP
                SUB             SP, SP, #0x1C
                BIC             SP, SP, #7
                VMOV            D16, R2, R3
                VMOV            D17, R0, R1
                VSTR            D17, [SP,#0x20+a]
                VSTR            D16, [SP,#0x20+b]
                VLDR            D16, [SP,#0x20+a]
                VLDR            D17, [SP,#0x20+b]
                VCMPE.F64       D16, D17
                VMRS            APSR_nzcv, FPSCR
                BLE             loc_2E08
                VLDR            D16, [SP,#0x20+a]
                VSTR            D16, [SP,#0x20+val_to_return]
                B               loc_2E10

loc_2E08
                VLDR            D16, [SP,#0x20+b]
                VSTR            D16, [SP,#0x20+val_to_return]

loc_2E10
                VLDR            D16, [SP,#0x20+val_to_return]
                VMOV            R0, R1, D16
                MOV             SP, R7
                LDR             R7, [SP+0x20+b],#4
                BX              LR
\end{lstlisting}

Almost the same as we already saw, 
but there is too much redundant code because the $a$ and $b$ variables are stored in the local stack, as well
as the return value.

\myparagraph{\OptimizingKeilVI (\ThumbMode)}

\begin{lstlisting}[caption=\OptimizingKeilVI (\ThumbMode),style=customasmARM]
                PUSH    {R3-R7,LR}
                MOVS    R4, R2
                MOVS    R5, R3
                MOVS    R6, R0
                MOVS    R7, R1
                BL      __aeabi_cdrcmple
                BCS     loc_1C0
                MOVS    R0, R6
                MOVS    R1, R7
                POP     {R3-R7,PC}

loc_1C0
                MOVS    R0, R4
                MOVS    R1, R5
                POP     {R3-R7,PC}
\end{lstlisting}


Keil doesn't generate FPU-instructions since it cannot rely on them being
supported on the target CPU, and it cannot be done by straightforward bitwise comparing.
%TODO1: why?
So it calls an external library function to do the comparison: \GTT{\_\_aeabi\_cdrcmple}. 
\myindex{ARM!\Instructions!BCS}

N.B. The result of the comparison is to be left in the flags by this function, so the following
\INS{BCS} (\IT{Carry set---Greater than or equal})
instruction can work without any additional code.

}
\RU{\subsubsection{ARM}

\myparagraph{\OptimizingXcodeIV (\ARMMode)}

\lstinputlisting[caption=\OptimizingXcodeIV (\ARMMode),style=customasmARM]{patterns/12_FPU/3_comparison/ARM/Xcode_ARM_RU.lst}

\myindex{ARM!\Registers!APSR}
\myindex{ARM!\Registers!FPSCR}
Очень простой случай.
Входные величины помещаются в \GTT{D17} и \GTT{D16} и сравниваются при помощи инструкции \INS{VCMPE}.
Как и в сопроцессорах x86, сопроцессор в ARM имеет свой собственный регистр статуса и флагов (\ac{FPSCR}),
потому что есть необходимость хранить специфичные для его работы флаги.

% TODO -> расписать регистр по битам
\myindex{ARM!\Instructions!VMRS}
И так же, как и в x86, 
в ARM нет инструкций условного перехода, проверяющих биты в регистре статуса сопроцессора. 
Поэтому имеется инструкция \INS{VMRS}, копирующая 4 бита (N, Z, C, V) 
из статуса сопроцессора в биты \IT{общего} статуса (регистр \ac{APSR}).

\myindex{ARM!\Instructions!VMOVGT}
\INS{VMOVGT} это аналог \INS{MOVGT}, инструкция для D-регистров, срабатывающая, если при сравнении один операнд был больше чем второй
(\IT{GT --- Greater Than}). 

Если она сработает, 
в \GTT{D16} запишется значение $a$, лежащее в тот момент в \GTT{D17}.
В обратном случае в \GTT{D16} остается значение $b$.


\myindex{ARM!\Instructions!VMOV}
Предпоследняя инструкция \INS{VMOV} готовит то, что было в \GTT{D16}, для возврата через 
пару регистров \Reg{0} и \Reg{1}.

\myparagraph{\OptimizingXcodeIV (\ThumbTwoMode)}

\begin{lstlisting}[caption=\OptimizingXcodeIV (\ThumbTwoMode),style=customasmARM]
VMOV            D16, R2, R3 ; b
VMOV            D17, R0, R1 ; a
VCMPE.F64       D17, D16
VMRS            APSR_nzcv, FPSCR
IT GT 
VMOVGT.F64      D16, D17
VMOV            R0, R1, D16
BX              LR
\end{lstlisting}

Почти то же самое, что и в предыдущем примере, за парой отличий.
Как мы уже знаем, многие инструкции в режиме ARM можно дополнять условием.
Но в режиме Thumb такого нет.
В 16-битных инструкций просто нет места для лишних 4 битов, при помощи
которых можно было бы закодировать условие выполнения.

\myindex{ARM!\ThumbTwoMode}
Поэтому в Thumb-2 добавили возможность дополнять \\
Thumb-инструкции условиями.
В листинге, сгенерированном при помощи \IDA, мы видим инструкцию \INS{VMOVGT}, 
такую же как и в предыдущем примере.

В реальности там закодирована обычная инструкция \INS{VMOV}, просто \IDA добавила суффикс \GTT{-GT} к ней, 
потому что перед этой инструкцией стоит \INS{IT GT}.

\label{ARM_Thumb_IT}
\myindex{ARM!\Instructions!IT}
\myindex{ARM!if-then block}
Инструкция \INS{IT} определяет так называемый \IT{if-then block}. 
После этой инструкции можно указывать до четырех инструкций, 
к каждой из которых будет добавлен суффикс условия.

В нашем примере \INS{IT GT} означает,
что следующая за ней инструкция будет исполнена, если условие
\IT{GT} (\IT{Greater Than}) справедливо.

\myindex{Angry Birds}
Теперь более сложный пример. Кстати, из 
Angry Birds (для iOS):

\begin{lstlisting}[caption=Angry Birds Classic,style=customasmARM]
...
ITE NE
VMOVNE          R2, R3, D16
VMOVEQ          R2, R3, D17
BLX             _objc_msgSend ; без суффикса
...
\end{lstlisting}

\INS{ITE} означает \IT{if-then-else} 
и кодирует суффиксы для двух следующих за ней инструкций.

Первая из них исполнится, если условие, закодированное в \INS{ITE} (\IT{NE, not equal}) будет в тот момент справедливо,
а вторая~--- если это условие не сработает.
(Обратное условие от \GTT{NE} это \GTT{EQ} (\IT{equal})).

Инструкция следующая за второй \INS{VMOV} (или VMOEQ) нормальная, без суффикса (\INS{BLX}).

\myindex{Angry Birds}
Ещё чуть сложнее, и снова этот фрагмент из Angry Birds:

\begin{lstlisting}[caption=Angry Birds Classic,style=customasmARM]
...
ITTTT EQ
MOVEQ           R0, R4
ADDEQ           SP, SP, #0x20
POPEQ.W         {R8,R10}
POPEQ           {R4-R7,PC}
BLX             ___stack_chk_fail ; без суффикса
...
\end{lstlisting}

Четыре символа \q{T} в инструкции означают, что четыре последующие инструкции будут исполнены если условие соблюдается.
Поэтому \IDA добавила ко всем четырем инструкциям суффикс \GTT{-EQ}. 
А если бы здесь было, например,
\INS{ITEEE EQ} (\IT{if-then-else-else-else}), 
тогда суффиксы для следующих четырех инструкций были бы расставлены так:

\begin{lstlisting}
-EQ
-NE
-NE
-NE
\end{lstlisting}

\myindex{Angry Birds}
Ещё фрагмент из Angry Birds:

\begin{lstlisting}[caption=Angry Birds Classic,style=customasmARM]
...
CMP.W           R0, #0xFFFFFFFF
ITTE LE
SUBLE.W         R10, R0, #1
NEGLE           R0, R0
MOVGT           R10, R0
MOVS            R6, #0         ; без суффикса
CBZ             R0, loc_1E7E32 ; без суффикса
...
\end{lstlisting}

\INS{ITTE} (\IT{if-then-then-else}) 
означает, что первая и вторая инструкции исполнятся, если условие \GTT{LE} (\IT{Less or Equal})
справедливо, а третья~--- если справедливо обратное условие (\GTT{GT} --- \IT{Greater Than}).

Компиляторы способны генерировать далеко не все варианты.

\myindex{Angry Birds}
Например, в вышеупомянутой игре Angry Birds (версия \IT{classic} для iOS)

встречаются только такие варианты инструкции \INS{IT}: 
\INS{IT}, \INS{ITE}, \INS{ITT}, \INS{ITTE}, \INS{ITTT}, \INS{ITTTT}.
\myindex{\GrepUsage}
Как это узнать?
В \IDA можно сгенерировать листинг (что и было сделано), только в опциях был установлен показ 4 байтов для каждого опкода.

Затем, зная что старшая часть 16-битного опкода (\INS{IT} это \GTT{0xBF}), сделаем при помощи \GTT{grep} это:

\begin{lstlisting}
cat AngryBirdsClassic.lst | grep " BF" | grep "IT" > results.lst
\end{lstlisting}

\myindex{ARM!\ThumbTwoMode}
Кстати, если писать на ассемблере для режима Thumb-2 вручную, и дополнять инструкции суффиксами
условия, то ассемблер автоматически будет добавлять инструкцию \INS{IT} с соответствующими флагами там,
где надо.

\myparagraph{\NonOptimizingXcodeIV (\ARMMode)}

\begin{lstlisting}[caption=\NonOptimizingXcodeIV (\ARMMode),style=customasmARM]
b               = -0x20
a               = -0x18
val_to_return   = -0x10
saved_R7        = -4

                STR             R7, [SP,#saved_R7]!
                MOV             R7, SP
                SUB             SP, SP, #0x1C
                BIC             SP, SP, #7
                VMOV            D16, R2, R3
                VMOV            D17, R0, R1
                VSTR            D17, [SP,#0x20+a]
                VSTR            D16, [SP,#0x20+b]
                VLDR            D16, [SP,#0x20+a]
                VLDR            D17, [SP,#0x20+b]
                VCMPE.F64       D16, D17
                VMRS            APSR_nzcv, FPSCR
                BLE             loc_2E08
                VLDR            D16, [SP,#0x20+a]
                VSTR            D16, [SP,#0x20+val_to_return]
                B               loc_2E10

loc_2E08
                VLDR            D16, [SP,#0x20+b]
                VSTR            D16, [SP,#0x20+val_to_return]

loc_2E10
                VLDR            D16, [SP,#0x20+val_to_return]
                VMOV            R0, R1, D16
                MOV             SP, R7
                LDR             R7, [SP+0x20+b],#4
                BX              LR
\end{lstlisting}

Почти то же самое, что мы уже видели, 
но много избыточного кода из-за хранения $a$ и $b$, 
а также выходного значения, в локальном стеке.


\myparagraph{\OptimizingKeilVI (\ThumbMode)}

\begin{lstlisting}[caption=\OptimizingKeilVI (\ThumbMode),style=customasmARM]
                PUSH    {R3-R7,LR}
                MOVS    R4, R2
                MOVS    R5, R3
                MOVS    R6, R0
                MOVS    R7, R1
                BL      __aeabi_cdrcmple
                BCS     loc_1C0
                MOVS    R0, R6
                MOVS    R1, R7
                POP     {R3-R7,PC}

loc_1C0
                MOVS    R0, R4
                MOVS    R1, R5
                POP     {R3-R7,PC}
\end{lstlisting}

Keil не генерирует FPU-инструкции, потому что не 
рассчитывает на то, что они будет поддерживаться, а простым сравнением побитово здесь не обойтись.

%TODO1: why?
Для сравнения вызывается библиотечная функция \GTT{\_\_aeabi\_cdrcmple}. 
\myindex{ARM!\Instructions!BCS}

N.B. Результат сравнения эта функция оставляет в флагах, чтобы следующая за вызовом инструкция
\INS{BCS} (\IT{Carry set~--- Greater than or equal})
могла работать без дополнительного кода.

}
\DE{\subsubsection{ARM}

\myparagraph{\OptimizingXcodeIV (\ARMMode)}

\lstinputlisting[caption=\OptimizingXcodeIV
(\ARMMode),style=customasmARM]{patterns/12_FPU/3_comparison/ARM/Xcode_ARM_DE.lst}

\myindex{ARM!\Registers!APSR}
\myindex{ARM!\Registers!FPSCR}
Ein recht einfacher Fall.
Die Eingabewerte werden in die Register \GTT{D17} und \GTT{D16} geladen und dann mit dem Befehl \INS{VCMPE} verglichen. 

Genau wie der x86-Koprozessor besitzt auch der ARM-Koprozessor seine eigenen Status und Flag Register (\ac{FPSCR}), denn
es gibt auch hier die Notwendigkeit die spezifischen Flags des Koprozessors zu speichern.

% TODO -> расписать регистр по битам
\myindex{ARM!\Instructions!VMRS}
Und genau wie beim x86 gibt es auch in ARM keine Befehle für bedingte Sprünge, die die Bits im Statusregister des
Koprozessors abfragen können. So gibt es den Befehl \INS{VMRS}, um 4 Bits (N, Z, C, V) vom Statusregister des
Koprozessors in das \IT{allgemeine} Statusregister (\ac{APSR}) zu kopieren.

\myindex{ARM!\Instructions!VMOVGT}
\INS{VMOVGT} ist das Analogon zum \INS{MOVGT} Befehl für D-Register: er wird ausgeführt, wenn ein Operand bezüglich
eines \IT{GT---Greater Than (dt. größer als)} Vergleichs größer ist als der andere.

Wenn er ausgeführt wird, wird der Wert von $a$ nach \GTT{D16} geschrieben (dieser wird aktuell in \GTT{D17}
gespeichert). Andernfalls bleibt der Wert von $b$ im \GTT{D16} Register.

\myindex{ARM!\Instructions!VMOV}
Der vorletzte Befehl \INS{VMOV} bereitet den Wert im \GTT{D16} Register für die Rückgabe über das Registerpaar \Reg{0}
und \Reg{1} vor.

\myparagraph{\OptimizingXcodeIV (\ThumbTwoMode)}

\begin{lstlisting}[caption=\OptimizingXcodeIV (\ThumbTwoMode),style=customasmARM]
VMOV            D16, R2, R3 ; b
VMOV            D17, R0, R1 ; a
VCMPE.F64       D17, D16
VMRS            APSR_nzcv, FPSCR
IT GT 
VMOVGT.F64      D16, D17
VMOV            R0, R1, D16
BX              LR
\end{lstlisting}

Fast das gleiche wie im vorherigen Beispiel, aber in gewisser Weise dennoch unterschiedlich.
Wie wir bereits wissen, können viele Befehl im ARM mode durch bedingte Prädikate unterstützt werden.
Im Thumb mode dagegen gibt es nichts Vergleichbares.
Es gibt keinen Platz in den 16-Bit-Befehlen für 4 weitere Bits, in denen Bedingungen kodiert werden könnten.

\myindex{ARM!\ThumbTwoMode}
Thumb-2 wurde erweitert, um zu ermöglichen alten Thumb-Befehlen nachträglich Prädikate zuzuweisen. Hier, im von \IDA
erzeugten Listing finden wir den \INS{VMOVGT} Befehl aus dem vorherigen Beispiel wieder.

Tatsächlich ist hier das gewöhnliche \INS{VMOV} kodiert, aber \IDA ergänzt den Suffix \GTT{-GT}, da sich direkt davor
eine \INS{IT GT} Befehl befindet.

\label{ARM_Thumb_IT}
\myindex{ARM!\Instructions!IT}
\myindex{ARM!if-then block}
Der \INS{IT} Befehl definiert einen sogenannten \IT{If-Then-Block}.

Nach dem Befehl können bis zu 4 weitere Befehle, jeder mit einem beschreibenden Suffix, verwendet werden.
In unserem Beispiel impliziert \INS{IT GT}, dass der Folgebefehl genau dann ausgeführt werden soll, wenn die IT{GT}
(\IT{Greater Than}) Bedingung wahr ist.

\myindex{Angry Birds}
Hier ist ein komplexeres Codefragment, welches aus Angry Birds (für iOS) stammt:

\begin{lstlisting}[caption=Angry Birds Classic,style=customasmARM]
...
ITE NE
VMOVNE          R2, R3, D16
VMOVEQ          R2, R3, D17
BLX             _objc_msgSend ; ohne Suffix
...
\end{lstlisting}

\INS{ITE} steht für \IT{if-then-else} und kodiert Suffixe für die beiden folgenden Befehle.

Der erste Befehl wird ausgeführt, wenn die durch \INS{ITE} (\IT{NE, not ewual}, dt. ungleich) kodierte Bedingung wahr
ist und der zweite wenn die Bedingung falsch ist (die inverse Bedingung zu \GTT{NE} ist \GTT{EQ} (\IT{equal}, dt.
gleich)).

Der dem zweiten Befehl folgende \INS{VMOV} (oder \INS{VMOVEQ}) ist ein gewöhnlicher Befehl ohne Suffix (\INS{BLX}).

\myindex{Angry Birds}
Ein weiteres etwas schwieriger verständliches Codefragment, ebenfalls aus Angry Birds:

\begin{lstlisting}[caption=Angry Birds Classic,style=customasmARM]
...
ITTTT EQ
MOVEQ           R0, R4
ADDEQ           SP, SP, #0x20
POPEQ.W         {R8,R10}
POPEQ           {R4-R7,PC}
BLX             ___stack_chk_fail ; ohne Suffix
...
\end{lstlisting}
Vier \q{T} Symbole in der Mnemonik des Befehls bedeuten, dass die vier folgenden Befehle alle ausgeführt werden, wenn
die Bedingung wahr ist. 

Aus diesem Grund fügt \IDA den \GTT{-EQ} Suffix an jeden der vier Befehle an.

Gäbe es beispielsweise \INS{ITEEE EQ} (\IT{if-then-else-else-else}), dann würden wie folgt Suffixe angehängt werden:

\begin{lstlisting}
-EQ
-NE
-NE
-NE
\end{lstlisting}

\myindex{Angry Birds}
Ein weiteres Fragment aus Angry Birds:

\begin{lstlisting}[caption=Angry Birds Classic,style=customasmARM]
...
CMP.W           R0, #0xFFFFFFFF
ITTE LE
SUBLE.W         R10, R0, #1
NEGLE           R0, R0
MOVGT           R10, R0
MOVS            R6, #0         ; ohne Suffix
CBZ             R0, loc_1E7E32 ; ohne Suffix
...
\end{lstlisting}

\INS{ITTE} (\IT{if-then-then-else}) impliziert, dass der erste und zweite Befehl ausgeführt werden, wenn die \GTT{LE}
(\IT{Less or Equal}, dt. mindestens) Bedingung wahr ist und der dritte, wenn die inverse Bedingung
(\GTT{GT}---\IT{Greater Than}, dt. mehr als) wahr ist.

Für gewöhnlich erzeugen Compiler nicht alle denkbaren Kombinationen.
\myindex{Angry Birds}
Im betrachteten Spiel Angry Birds beispielsweise (\IT{classic} Version für iOS) werden nur die folgenden Variationen des
\INS{IT} Befehls verwendet:
\INS{IT}, \INS{ITE}, \INS{ITT}, \INS{ITTE}, \INS{ITTT}, \INS{ITTTT}.
\myindex{\GrepUsage}
Bleibt die Frage, wie man dies lernen kann. 
In \IDA ist es mögliche Listing-Dateien zu erzeugen mit der Option 4 Bytes für jeden Opcode anzuzeigen. 
Dadurch können wir bei Kenntnis des höherwertigen Teils des 16-Bit-Opcodes (\INS{IT} entspricht \GTT{0xBF}) unter
Zuhilfenahme von \GTT{grep} wie folgt vorgehen:

\begin{lstlisting}
cat AngryBirdsClassic.lst | grep " BF" | grep "IT" > results.lst
\end{lstlisting}

\myindex{ARM!\ThumbTwoMode}
Übrigens, wenn man von Hand Assemblerprogramme für ARM in Thumb-2 mode schreibt und man die Suffixe für die Bedingungen
selbst anhängt, wird der Assemblierer die entsprechenden \INS{IT} Befehle inklusive der benötigten Flags automatisch an
den benötigten Stellen hinzufügen.

\myparagraph{\NonOptimizingXcodeIV (\ARMMode)}

\begin{lstlisting}[caption=\NonOptimizingXcodeIV (\ARMMode),style=customasmARM]
b               = -0x20
a               = -0x18
val_to_return   = -0x10
saved_R7        = -4

                STR             R7, [SP,#saved_R7]!
                MOV             R7, SP
                SUB             SP, SP, #0x1C
                BIC             SP, SP, #7
                VMOV            D16, R2, R3
                VMOV            D17, R0, R1
                VSTR            D17, [SP,#0x20+a]
                VSTR            D16, [SP,#0x20+b]
                VLDR            D16, [SP,#0x20+a]
                VLDR            D17, [SP,#0x20+b]
                VCMPE.F64       D16, D17
                VMRS            APSR_nzcv, FPSCR
                BLE             loc_2E08
                VLDR            D16, [SP,#0x20+a]
                VSTR            D16, [SP,#0x20+val_to_return]
                B               loc_2E10

loc_2E08
                VLDR            D16, [SP,#0x20+b]
                VSTR            D16, [SP,#0x20+val_to_return]

loc_2E10
                VLDR            D16, [SP,#0x20+val_to_return]
                VMOV            R0, R1, D16
                MOV             SP, R7
                LDR             R7, [SP+0x20+b],#4
                BX              LR
\end{lstlisting}
Fast identisch mit dem, was wir schon gesehen haben, aber hier gibt es zu viel redundanten Code, weil die Variablen $a$
und $b$ im lokalen Stack und zusätzlich als Rückgabewerte gespeichert werden.

\myparagraph{\OptimizingKeilVI (\ThumbMode)}

\begin{lstlisting}[caption=\OptimizingKeilVI (\ThumbMode),style=customasmARM]
                PUSH    {R3-R7,LR}
                MOVS    R4, R2
                MOVS    R5, R3
                MOVS    R6, R0
                MOVS    R7, R1
                BL      __aeabi_cdrcmple
                BCS     loc_1C0
                MOVS    R0, R6
                MOVS    R1, R7
                POP     {R3-R7,PC}

loc_1C0
                MOVS    R0, R4
                MOVS    R1, R5
                POP     {R3-R7,PC}
\end{lstlisting}

Keil erzeugt keine FPU-Befehle, da er sich sich darauf verlassen kann, dass diese von der Ziel-CPU unterstützt werden
und dies nicht durch einfache bitweisen Vergleich erledigt werden kann.

%TODO1: why?
Keil ruft also eine Funktion einer externen Programmbibliothek (\GTT{\_\_aeabi\_cdrcmple}) auf, um den Vergleich
durchzuführen.
\myindex{ARM!\Instructions!BCS}
Das Ergebnis des Vergleich wird von der Funktion in den Flags belassen, sodass der folgende \INS{BCS} (\IT{Carry
set---Greater than or equal}) Befehl ohne zusätzlichen Code funktioniert.

}

\EN{\subsubsection{ARM64}

\myparagraph{\Optimizing GCC (Linaro) 4.9}

\lstinputlisting[style=customasmARM]{patterns/12_FPU/3_comparison/ARM/ARM64_GCC_O3_EN.lst}

The ARM64 \ac{ISA} has FPU-instructions 
which set \ac{APSR} the CPU flags instead of \ac{FPSCR} for convenience.
The\ac{FPU} is not a separate device here anymore (at least, logically).
\myindex{ARM!\Instructions!FCMPE}
Here we see \INS{FCMPE}. It compares the two values passed in \RegD{0} and \RegD{1} (which are the first and second arguments of the function)
and sets \ac{APSR} flags (N, Z, C, V).

\myindex{ARM!\Instructions!FCSEL}
\INS{FCSEL} (\IT{Floating Conditional Select}) copies the value of \RegD{0} or \RegD{1} into \RegD{0} depending on the condition (\GTT{GT}---Greater Than),
and again, it uses flags in \ac{APSR} register instead of \ac{FPSCR}.

This is much more convenient, compared to the instruction set in older CPUs.

If the condition is true (\GTT{GT}), then the value of \RegD{0} 
is copied into \RegD{0} (i.e., nothing happens).
If the condition is not true, the value of \RegD{1} 
is copied into \RegD{0}.

\myparagraph{\NonOptimizing GCC (Linaro) 4.9}

\lstinputlisting[style=customasmARM]{patterns/12_FPU/3_comparison/ARM/ARM64_GCC_EN.lst}

Non-optimizing GCC is more verbose.

First, the function saves its input argument values in the local stack (\IT{Register Save Area}).
Then the code reloads these values into registers
\RegX{0}/\RegX{1} and finally copies them to 
\RegD{0}/\RegD{1} to be compared using \INS{FCMPE}. 
A lot of redundant code, 
but that is how non-optimizing compilers work.
\INS{FCMPE} compares the values and sets the \ac{APSR} flags.
At this moment, 
the compiler is not thinking yet about the more convenient \INS{FCSEL} instruction, so it proceed using old methods: 
using the \INS{BLE} instruction (\IT{Branch if Less than or Equal}).
In the first case ($a>b$), the value of $a$ gets loaded 
into \RegX{0}.
In the other case ($a<=b$), the value of $b$ gets loaded into 
\RegX{0}.
Finally, the value from \RegX{0} gets copied into \RegD{0}, 
because the return value needs to be in this 
register.

\mysubparagraph{\Exercise}

As an exercise, you can try optimizing this piece of code 
manually by removing redundant instructions and not introducing new ones (including \INS{FCSEL}).

\myparagraph{\Optimizing GCC (Linaro) 4.9---float}

Let's also rewrite this example to use \Tfloat instead of \Tdouble.

\begin{lstlisting}[style=customc]
float f_max (float a, float b)
{
	if (a>b)
		return a;

	return b;
};
\end{lstlisting}

\lstinputlisting[style=customasmARM]{patterns/12_FPU/3_comparison/ARM/ARM64_GCC_O3_float_EN.lst}

It is the same code, but the S-registers are used instead of D- ones.
It's because numbers of type \Tfloat are passed in 32-bit S-registers (which are in fact the lower parts of the 64-bit D-registers).

}
\RU{\subsubsection{ARM64}

\myparagraph{\Optimizing GCC (Linaro) 4.9}

\lstinputlisting[style=customasmARM]{patterns/12_FPU/3_comparison/ARM/ARM64_GCC_O3_RU.lst}

В ARM64 \ac{ISA} теперь есть FPU-инструкции, устанавливающие флаги CPU \ac{APSR} вместо \ac{FPSCR} для удобства.
\ac{FPU} больше не отдельное устройство (по крайней мере логически).
\myindex{ARM!\Instructions!FCMPE}
Это \INS{FCMPE}. Она сравнивает два значения, переданных в \RegD{0} и \RegD{1} 
(а это первый и второй аргументы функции) и выставляет флаги в \ac{APSR} (N, Z, C, V).

\myindex{ARM!\Instructions!FCSEL}
\INS{FCSEL} (\IT{Floating Conditional Select}) копирует значение \RegD{0} или
\RegD{1} в \RegD{0} в зависимости от условия 
(\GTT{GT} --- Greater Than --- больше чем),
и снова, она использует флаги в регистре \ac{APSR} вместо \ac{FPSCR}.
Это куда удобнее, если сравнивать с тем набором инструкций, что был в процессорах раньше.

Если условие верно (\GTT{GT}), тогда значение из \RegD{0} копируется в \RegD{0} (т.е. ничего не происходит).
Если условие не верно, то значение \RegD{1} копируется в \RegD{0}.

\myparagraph{\NonOptimizing GCC (Linaro) 4.9}

\lstinputlisting[style=customasmARM]{patterns/12_FPU/3_comparison/ARM/ARM64_GCC_RU.lst}

Неоптимизирующий GCC более многословен.
В начале функция сохраняет значения входных аргументов в локальном стеке (\IT{Register Save Area}).
Затем код перезагружает значения в регистры
\RegX{0}/\RegX{1} и наконец копирует их в 
\RegD{0}/\RegD{1} для сравнения инструкцией \INS{FCMPE}. 
Много избыточного кода, но так работают неоптимизирующие компиляторы.
\INS{FCMPE} сравнивает значения и устанавливает флаги в \ac{APSR}.
В этот момент компилятор ещё не думает о более удобной инструкции \INS{FCSEL}, так что он работает старым 
методом: 
использует инструкцию \INS{BLE} (\IT{Branch if Less than or Equal} (переход если меньше или равно)).
В одном случае ($a>b$) значение $a$ перезагружается в \RegX{0}.
В другом случае ($a<=b$) значение $b$ загружается в \RegX{0}.
Наконец, значение из \RegX{0} копируется в \RegD{0}, 
потому что возвращаемое значение оставляется в этом регистре.

\mysubparagraph{\Exercise}

Для упражнения вы можете попробовать оптимизировать этот фрагмент кода вручную, удалив избыточные инструкции,
но не добавляя новых (включая \INS{FCSEL}).

\myparagraph{\Optimizing GCC (Linaro) 4.9: float}

Перепишем пример. Теперь здесь \Tfloat вместо \Tdouble.

\begin{lstlisting}[style=customc]
float f_max (float a, float b)
{
	if (a>b)
		return a;

	return b;
};
\end{lstlisting}

\lstinputlisting[style=customasmARM]{patterns/12_FPU/3_comparison/ARM/ARM64_GCC_O3_float_RU.lst}

Всё то же самое, только используются S-регистры вместо D-.
Так что числа типа \Tfloat передаются в 32-битных S-регистрах (а это младшие части 64-битных D-регистров).

}
\DE{\subsubsection{ARM64}

\myparagraph{\Optimizing GCC (Linaro) 4.9}

\lstinputlisting[style=customasmARM]{patterns/12_FPU/3_comparison/ARM/ARM64_GCC_O3_DE.lst}
Der ARM64 \ac{ISA} verfügt über FPU-Befehle, die der Einfachheit halber die Flags der CPU \ac{APSR} anstelle von
\ac{FPSCR} setzen.
Die \ac{FPU} ist hier kein separates Gerät mehr (zumindest logisch).

\myindex{ARM!\Instructions!FCMPE}
Wir finden hier \INS{FCMPE}. Er vergleicht die beiden über \RegD{0} und \RegD{1} übergebenen Werte (dabei handelt es
sich um das erste und zweite Argument der Funktion) und setzt \ac{APSR} die Flags (N, Z, C, V).

\myindex{ARM!\Instructions!FCSEL}
\INS{FCSEL} (\IT{Floating Conditional Select}) kopiert den Wert von \RegD{0} oder \RegD{1} nach \RegD{0}, abhängig von
der Bedingung (\GTT{GT}---Greater Than), und verwendet wiederum Flags im \ac{APSR} Register anstatt derer von
\ac{FPSCR}.

Dies ist im Vergleich zum Befehlssatz alter CPUs deutlich praktischer.

Falls die Bedingung wahr ist (\GTT{GT}), dann wird der Wert von \RegD{0} nach \RegD{0} kopiert (d.h. es geschieht
nichts).
Falls die Bedingung falsch ist, wird der Wert von \RegD{1} nach \RegD{0} kopiert.

\myparagraph{\NonOptimizing GCC (Linaro) 4.9}

\lstinputlisting[style=customasmARM]{patterns/12_FPU/3_comparison/ARM/ARM64_GCC_DE.lst}
Der nicht optimierende GCC ist weniger kompakt.

Zunächst speichert die Funktion ihre Eingabewerte auf dem lokalen Stack (\IT{Register Save Area}), danach lädt der Code
die Werte erneut in die Register \RegX{0}/\RegX{1} und kopiert sie schließlich nach \RegD{0}/\RegD{1}, um sie mittels
\INS{FCMPE} zu vergleichen.
Eine Menge redundanter Code, aber so arbeitet ein nicht optimierender Compiler nun einmal.
\INS{FCMPE} vergleich die Werte und setzt die \ac{APSR} Flags.
Zu diesem Zeitpunkt entscheidet sich der Compiler noch nicht für den praktischeren \INS{FCSEL} Befehl und arbeitet
stattdessen mit herkömmlichen Methoden:
er verwendet den \INS{BLE} Befehl (\IT{Branch if Less than or Equal}).
Im ersten Fall ($a>b$) wird der Wert von $a$ nach \RegX{0} geladen. 
Im anderen Fall ($a<=b$) wird der Wert von $b$ nach \RegX{0} geladen.
Schließlich wird der Wert aus \RegX{0} nach \RegD{0} kopiert, denn der Rückgabewert muss sich in diesem Register
befinden.


\mysubparagraph{\Exercise}
Dem Leser bleibt als Übung, den vorstehenden Code zu optimieren, indem manuell die redundanten Instruktionen entfernt
werden ohne dabei neue einzuführen (wie etwa \INS{FCSEL}).

\myparagraph{\Optimizing GCC (Linaro) 4.9---float}
Wir wollen nun dieses Beispiel umschreiben, indem wir \Tfloat anstelle von \Tdouble verwenden.

\begin{lstlisting}[style=customc]
float f_max (float a, float b)
{
	if (a>b)
		return a;

	return b;
};
\end{lstlisting}

\lstinputlisting[style=customasmARM]{patterns/12_FPU/3_comparison/ARM/ARM64_GCC_O3_float_DE.lst}
Es ist der gleiche Code, aber hier werden die S-Register anstelle der D-Register verwendet.
Das ist darauf zurückzuführen, dass der \Tfloat Typ in 32-Bit-S-Registern übergeben wird (welche in Wirklichkeit nichts
anderes als die niederen Teile der 64-Bit-D-Register sind).
}

\EN{\subsubsection{MIPS}

\myindex{MIPS!\Registers!FCCR}
The co-processor of the MIPS processor has a condition bit which can be set in the FPU and checked in the CPU.

Earlier MIPS-es have only one condition bit (called FCC0), later ones have 8 (called FCC7-FCC0).

This bit (or bits) are located in the register called FCCR.

\lstinputlisting[caption=\Optimizing GCC 4.4.5 (IDA),style=customasmMIPS]{patterns/12_FPU/3_comparison/MIPS_O3_IDA_EN.lst}

\myindex{MIPS!\Instructions!C.LT.D}
\INS{C.LT.D} compares two values. 
\GTT{LT} is the condition \q{Less Than}.
\GTT{D} implies values of type \Tdouble.
Depending on the result of the comparison, the FCC0 condition bit is either set or cleared.

\myindex{MIPS!\Instructions!BC1T}
\myindex{MIPS!\Instructions!BC1F}
\INS{BC1T} checks the FCC0 bit and jumps if the bit is set.
\GTT{T} means that the jump is to be taken if the bit is set (\q{True}).
There is also the instruction \INS{BC1F} which jumps if the bit is cleared (\q{False}).

Depending on the jump, one of function arguments is placed into \$F0.
}
\RU{\subsubsection{MIPS}

\myindex{MIPS!\Registers!FCCR}

В сопроцессоре MIPS есть бит результата, который устанавливается в FPU и проверяется в CPU.

Ранние MIPS имели только один бит (с названием FCC0), а у поздних их 8 (с названием FCC7-FCC0).
Этот бит (или биты) находятся в регистре с названием FCCR.

\lstinputlisting[caption=\Optimizing GCC 4.4.5 (IDA),style=customasmMIPS]{patterns/12_FPU/3_comparison/MIPS_O3_IDA_RU.lst}

\myindex{MIPS!\Instructions!C.LT.D}
\INS{C.LT.D} сравнивает два значения. 
\GTT{LT} это условие \q{Less Than} (меньше чем).
\GTT{D} означает переменные типа \Tdouble.

В зависимости от результата сравнения, бит FCC0 устанавливается или очищается.

\myindex{MIPS!\Instructions!BC1T}
\myindex{MIPS!\Instructions!BC1F}
\INS{BC1T} проверяет бит FCC0 и делает переход, если бит выставлен.
\GTT{T} означает, что переход произойдет если бит выставлен (\q{True}).
Имеется также инструкция \INS{BC1F} которая сработает, если бит сброшен (\q{False}).

В зависимости от перехода один из аргументов функции помещается в регистр \$F0.

}
\DE{\subsubsection{MIPS}

\myindex{MIPS!\Registers!FCCR}
Der Koprozessor des MIPS Prozessors hat ein Condition Bit, welches in der FPU
gesetzt und in der CPU geprüft werden kann.

Frühere MIPS haben nur ein Condition Bit (genannt FCC0), spätere haben deren 8
(genannt FCC7-FCC0). 

Diese(s) Bit(s) befinden sich im Register FCCR.

\lstinputlisting[caption=\Optimizing GCC 4.4.5
(IDA)]{patterns/12_FPU/3_comparison/MIPS_O3_IDA_DE.lst}

\myindex{MIPS!\Instructions!C.LT.D}
\INS{C.LT.D} vergleicht zwei Werte. 
\GTT{LT} ist die Bedingung \q{Less Than} (weniger als).
\GTT{D} impliziert einen Wert vom Typ \Tdouble.
Abhängig vom Ergebnis des Vergleichs wird das FCC0 Condition Bit entweder
gesetzt oder gelöscht.

\myindex{MIPS!\Instructions!BC1T}
\myindex{MIPS!\Instructions!BC1F}
\INS{BC1T} prüft das FCC0 Bit und sprint, falls das Bit gesetzt ist.
\GTT{T} bedeutet, dass der Sprung ausgeführt wird, wenn das Bit gesetzt
(\q{True}) ist.
Daneben gibt es auch den Befehl \INS{BC1F}, der springt, wenn das Bit gelöscht
(\q{FALSE}) ist.

Abhängig vom Sprung wird einer der Funktionsargument in \$F0 abgelegt.
}


\subsection{Некоторые константы}

В Wikipedia легко найти представление некоторых констант в IEEE 754.
Любопытно узнать, что 0.0 в IEEE 754 представляется как 32 нулевых бита (для одинарной точности) или 64 нулевых бита
(для двойной).
Так что, для записи числа 0.0 в переменную в памяти или регистр, можно пользоваться инструкцией \MOV, или \TT{XOR reg, reg}.
\myindex{\CStandardLibrary!memset()}
Это тем может быть удобно, что если в структуре есть много переменных разных типов, то обычной ф-ций memset()
можно установить все целочисленные переменные в 0, все булевы переменные в \IT{false}, все указатели в NULL,
и все переменные с плавающей точкой (любой точности) в 0.0.

\subsection{Копирование}

По инерции можно подумать, что для загрузки и сохранения (и, следовательно, копирования) чисел в формате
IEEE 754 нужно использовать пару инструкций \INS{FLD}/\INS{FST}.
Тем не менее, этого куда легче достичь используя обычную инструкцию \INS{MOV},
которая, конечно же, просто копирует значения побитово.

\subsection{Стек, калькуляторы и обратная польская запись}

\myindex{Обратная польская запись}
Теперь понятно, почему некоторые старые калькуляторы используют обратную польскую запись
\footnote{\href{http://go.yurichev.com/17355}{ru.wikipedia.org/wiki/Обратная\_польская\_запись}}.

Например для сложения 12 и 34 нужно было набрать 12, потом 34, потом нажать знак \q{плюс}.

Это потому что старые калькуляторы просто реализовали стековую машину и это было куда проще, чем обрабатывать сложные выражения со скобками.

\subsection{x64}

О том, как происходит работа с числами с плавающей запятой в x86-64, читайте здесь: \myref{floating_SIMD}.

% sections
\subsection{\Exercises}

\begin{itemize}
	\item \url{http://challenges.re/60}
	\item \url{http://challenges.re/61}
\end{itemize}



