\myparagraph{MSVC}

Compile it in MSVC 2010:

\lstinputlisting[caption=MSVC 2010: \ttf{},style=customasmx86]{patterns/12_FPU/1_simple/MSVC_EN.asm}

\FLD takes 8 bytes from stack and loads the number into the \ST{0} register, automatically converting 
it into the internal 80-bit format (\IT{extended precision}).

\myindex{x86!\Instructions!FDIV}

\FDIV divides the value in \ST{0} by the number stored at address \\
\GTT{\_\_real@40091eb851eb851f}~---the value 3.14 is encoded there. 
The assembly syntax doesn't support floating point numbers, so 
what we see here is the hexadecimal representation of 3.14 in 64-bit IEEE 754 format.

After the execution of \FDIV \ST{0} holds the \gls{quotient}.

\myindex{x86!\Instructions!FDIVP}

By the way, there is also the \FDIVP instruction, which divides \ST{1} by \ST{0}, 
popping both these values from stack and then pushing the result. 
If you know the Forth language\FNURLFORTH,
you can quickly understand that this is a stack machine\FNURLSTACK.

The subsequent \FLD instruction pushes the value of $b$ into the stack.

After that, the quotient is placed in \ST{1}, and \ST{0} has the value of $b$.

\myindex{x86!\Instructions!FMUL}

The next \FMUL instruction does multiplication: $b$ from \ST{0} is multiplied by value at\\
\GTT{\_\_real@4010666666666666} (the number 4.1 is there) and leaves the result in the \ST{0} register.

\myindex{x86!\Instructions!FADDP}

The last \FADDP instruction adds the two values at top of stack, storing the result in \ST{1} 
and then popping the value of \ST{0}, thereby leaving the result at the top of the stack, in \ST{0}.

The function must return its result in the \ST{0} register, 
so there are no any other instructions except the function epilogue after \FADDP.

\clearpage
\myparagraph{MSVC + \olly}
\myindex{\olly}

2 pairs of 32-bit words are marked by red in the stack. 
Each pair is a double-number in IEEE 754 format and is passed from \main.

We see how the first \FLD loads a value ($1.2$) from the stack and puts it into \ST{0}:

\begin{figure}[H]
\centering
\myincludegraphics{patterns/12_FPU/1_simple/olly1.png}
\caption{\olly: the first \FLD has been executed}
\label{fig:FPU_simple_olly_1}
\end{figure}

Because of unavoidable conversion errors from 64-bit IEEE 754 floating point to 80-bit
(used internally in the FPU), here we see 1.999\ldots, which is close to 1.2.

\EIP now points to the next instruction (\FDIV), which loads a double-number (a constant) from memory.
For convenience, \olly shows its value: 3.14

\clearpage
Let's trace further. 
\FDIV has been executed, now \ST{0} contains 0.382\ldots
(\gls{quotient}):

\begin{figure}[H]
\centering
\myincludegraphics{patterns/12_FPU/1_simple/olly2.png}
\caption{\olly: \FDIV has been executed}
\label{fig:FPU_simple_olly_2}
\end{figure}

\clearpage
Third step: the next \FLD 
has been executed, loading 3.4 into \ST{0} (here we see the approximate value 3.39999\ldots): 

\begin{figure}[H]
\centering
\myincludegraphics{patterns/12_FPU/1_simple/olly3.png}
\caption{\olly: the second \FLD has been executed}
\label{fig:FPU_simple_olly_3}
\end{figure}

At the same time, \gls{quotient} \IT{is pushed} into \ST{1}.
Right now, \EIP points to the next instruction: \FMUL. 
It loads the constant 4.1 from memory, which \olly shows.

\clearpage
Next: \FMUL has been executed, so now the \gls{product} is in \ST{0}:

\begin{figure}[H]
\centering
\myincludegraphics{patterns/12_FPU/1_simple/olly4.png}
\caption{\olly: the \FMUL has been executed}
\label{fig:FPU_simple_olly_4}
\end{figure}

\clearpage
Next: the \FADDP has been executed, now the result of the addition is in \ST{0}, and \ST{1} is cleared:

\begin{figure}[H]
\centering
\myincludegraphics{patterns/12_FPU/1_simple/olly5.png}
\caption{\olly: \FADDP has been executed}
\label{fig:FPU_simple_olly_5}
\end{figure}

The result is left in \ST{0}, because the function returns its value in \ST{0}.

\main takes this value from the register later.

We also see something unusual: the 13.93\ldots value is now located in \ST{7}.
Why?

\label{FPU_is_rather_circular_buffer}

As we have read some time before in this book, the \ac{FPU} registers are a stack: \myref{FPU_is_stack}. 
But this is a simplification.

Imagine if it was implemented \IT{in hardware} as it's described, then all 7 register's
contents must be moved (or copied) to adjacent registers during pushing and popping, 
and that's a lot of work.

In reality, the \ac{FPU} has just 8 registers and a pointer (called \GTT{TOP}) which contains a register number,
which is the current \q{top of stack}.

When a value is pushed to the stack, \GTT{TOP} is pointed to the next available register,
and then a value is written to that register.

The procedure is reversed if a value is popped, however, the register which has been freed is not cleared
(it could possibly be cleared, but this is more work which can degrade performance).
So that's what we see here. 

It can be said that \FADDP saved the sum in the stack, and then popped one element.

But in fact, this instruction saved the sum and then shifted \GTT{TOP}.

More precisely, the registers of the \ac{FPU} are a circular buffer.

