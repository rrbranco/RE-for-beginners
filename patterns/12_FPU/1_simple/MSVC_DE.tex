\myparagraph{MSVC}

Kompilieren mit MSVC 2010 liefert:

\lstinputlisting[caption=MSVC 2010: \ttf{},style=customasmx86]{patterns/12_FPU/1_simple/MSVC_DE.asm}

\FLD nimmt 8 Byte vom Stack und lädt die Zahl in das \ST{0} Register, wobei
diese automatisch in das interne 80-bit-Format (\IT{erweiterte Genauigkeit})
konvertiert wird.

\myindex{x86!\Instructions!FDIV}
\FDIV teilt den Wert in \ST{0} durch die Zahl, die an der Adresse
\GTT{\_\_real@40091eb851eb851f} gespeichert ist~---der Wert 3.14 ist hier
kodiert.
Die Syntax des Assemblers erlaubt keine Fließkommazahlen, sodass wir hier die
hexadezimale Darstellung von 3.14 im 64-bit IEEE 754 Format finden.

Nach der Ausführung von \FDIV enthält \ST{0} den \glslink{quotient}{Quotienten}.

\myindex{x86!\Instructions!FDIVP}
Es gibt übrigens auch noch den \FDIVP Befehl, welcher \ST{1} durch \ST{0}
teilt, beide Werte vom Stack holt und das Ergebnis ebenfalls auf dem Stack
ablegt.
Wer mit der Sprache Forth \FNURLFORTH vertraut ist, erkennt schnell, dass es sich
hier um eine Stackmaschine\FNURLSTACK handelt.

Der nachfolgende \FLD Befehl speichert den Wert von $b$ auf dem Stack.

Anschließend wir der Quotient in \ST{1} abgelegt und \ST{0} enthält den Wert von
$b$.

\myindex{x86!\Instructions!FMUL}
Der nächste \FMUL Befehl führt folgende Multiplikation aus: $b$ aus Register
\ST{0} wird mit dem Wert an der Speicherstelle \GTT{\_\_real@4010666666666666}
(hier befindet sich die Zahl 4.1) multipliziert und hinterlässt das Ergebnis im
\ST{ß} Register.

\myindex{x86!\Instructions!FADDP}
Der letzte \FADDP Befehl addiert die beiden Werte, die auf dem Stack zuoberst
liegen, speichet das Ergebnis in \ST{1} und holt dann den Wert von \ST{0} vom
Stack, wobei das oberste Element auf dem Stack in \ST{0} gespeichert wird.

Die Funktion muss ihr Ergebnis im \ST{0} Register zurückgeben, sodass außer dem
Funktionsepilog nach \FADDP keine weiteren Befehle mehr folgen.

\clearpage
\myparagraph{MSVC + \olly}
\myindex{\olly}
Zwei Paare aus 32-bit Worten sind im Stack rot markiert.
Jedes Paar ist eine double-Zahl im IEEE 754 Format und wurde von \main
übergeben.

Wie sehen wie zunächst \FLD einen Wert ($1.2$) von Stack lädt und diesen in
\ST{0} ablegt:

\begin{figure}[H]
\centering
\myincludegraphics{patterns/12_FPU/1_simple/olly1.png}
\caption{\olly: der erste \FLD wurde ausgeführt}
\label{fig:FPU_simple_olly_1}
\end{figure}
Aufgrund der unvermeidlichen Konversionsfehler von der 64-bit IEEE 754
Fließkommazahl in ein 80-bit-Format (das intern in der FPU verwendet wird),
sehen wird hier 1.1999\ldots, was näherungsweise 1.2 entspricht.

\EIP zeigt nun auf den nächsten Befehl (\FDIV), der eine double-Zahl (eine
Konstante) aus dem Speicher lädt. Zur besseren Übersicht zeigt \olly deren Wert
an: 3.14

\clearpage
Verfolgen wir das ganze etwas weiter.
\FDIV wurde ausgeführt, nun enthält \ST{0} also 0.382\ldots (\glslink{quotient}{Quotient}):

\begin{figure}[H]
\centering
\myincludegraphics{patterns/12_FPU/1_simple/olly2.png}
\caption{\olly: \FDIV wurde ausgeführt}
\label{fig:FPU_simple_olly_2}
\end{figure}

\clearpage
Dritter Schritt: der nächste \FLD Befehl wurde ausgeführt; er lud 3.4 nach
\ST{0} (wir sehen hier den Näherungswert 3.39999\ldots):

\begin{figure}[H]
\centering
\myincludegraphics{patterns/12_FPU/1_simple/olly3.png}
\caption{\olly: der zweite \FLD wurde ausgeführt}
\label{fig:FPU_simple_olly_3}
\end{figure}
Gleichzeitig wird der \glslink{quotient}{Quotient} nach \ST{1} verschoben.
In diesem Moment zeigt \EIP auf den nächsten Befehl: \FMUL.
Dieser lädt die Konstante 4.1 aus dem Speicher, wie \olly zeigt.

\clearpage
Dann: \FMUL wurde ausgeführt, sodass das \glslink{product}{Produkt} jetzt in \ST{0} liegt.

\begin{figure}[H]
\centering
\myincludegraphics{patterns/12_FPU/1_simple/olly4.png}
\caption{\olly: \FMUL wurde ausgeführt}
\label{fig:FPU_simple_olly_4}
\end{figure}

\clearpage
Dann: der Befehl \FADDP wurde ausgeführt, sodass sich in \ST{0} nunmehr das
Ergebnis der Addition befindet und \ST{1} wird gelöscht:

\begin{figure}[H]
\centering
\myincludegraphics{patterns/12_FPU/1_simple/olly5.png}
\caption{\olly: \FADDP wurde ausgeführt}
\label{fig:FPU_simple_olly_5}
\end{figure}
Das Ergebnis bleibt in \ST{0}, denn die Funktion liefert ihren Rückgabewert über
\ST{0} zurück.

Später liest \main diesen Wert aus dem Register.

Wir sehen außerdem etwas Ungewöhnliches: der Wert 13.93\ldots befindet sich nun
in \ST{7}. Warum?

\label{FPU_is_rather_circular_buffer}
Wie bereits vorher in diesem Buch beschrieben, bilden die \ac{FPU} einen Stack:
\myref{FPU_is_stack}. Dabei handelt es sich jedoch um eine vereinfachte
Darstellung.

Stellen wir uns vor, es wäre genau wie beschrieben in \IT{Hardware}
implementiert, dann müssten die Inhalte aller 7 Register während der push und
pop-Befehle in jeweils benachbarte Register verschoben (oder kopiert) werden und
das würde eine Menge Aufwand bedeuten.

In Wirklichkeit hat die \ac{FPU} nur 8 Register und einen Pointer (\GTT{TOP}
genannt), der die Registernummer enthält, die derzeit oben auf dem Stack liegt.

Wenn ein Wert auf dem Stack abgelegt wird, zeigt \GTT{TOP} auf das nächste
verfügbare Register und dann wird der Wert in dieses Register geschrieben.

Dieser Vorgang läuft umgekehrt ab, wenn ein Wert vom Stack geholt wird, aber das
freigewordene Register wird nicht gelöscht (es könnnte möglicherweise gelöscht
werden, aber dies würde einen Mehraufwand bedeuten und die Performance
herabsetzen). 
Genau das sehen wir hier.

Man kann sagen, dass \FADDP die Summe auf dem Stack gespeichert hat und dann ein
Element vom Stack geholt hat.

Aber in Wirklichkeit hat der Befehl die Summe gespeichert und dann \GTT{TOP}
verschoben.

Genauer gesagt bilden die Register der \ac{FPU} einen Ringpuffer. 

