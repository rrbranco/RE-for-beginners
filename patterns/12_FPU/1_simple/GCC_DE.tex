\myparagraph{GCC}

GCC 4.4.1 (mit der Option \Othree) erzeugt fast den gleichen Code, nur leicht
verändert.

\lstinputlisting[caption=\Optimizing GCC 4.4.1,style=customasmx86]{patterns/12_FPU/1_simple/GCC_DE.asm} Der Unterschied
besteht darin, dass zuerst 3.14 auf dem Stack (in \ST{0}) abgelegt wird und danach der Wert in \GTT{arg\_0} durch den Wert in \ST{0}
geteilt wird.

\myindex{x86!\Instructions!FDIVR}
\FDIVR steht für \IT{Reverse Divide}~--teilen, wobei Dividend und Divisor
miteinander vertauscht werden. Da es sich bei der Multiplikation um eine
kommutative Operation handelt, gibt es keinen vergleichbaren Befehl für die
Multiplikation. Wir haben es lediglich \FMUL ohne \GTT{-R} Gegenstück zur
Verfügung.

\myindex{x86!\Instructions!FADDP}
\FADDP addiert die beiden Werte und holt auch einen Wert vom Stack. 
Nach der Ausführung steht die Summe in \ST{0}.

