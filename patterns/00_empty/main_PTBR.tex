\section{Uma Função Vazia}
\label{empty_func}

A função mais simples possivel é, indiscutivelmente, aquela que não faz nada:

\lstinputlisting[caption=\EN{\CCpp Code},style=customc]{patterns/00_empty/1.c}

Vamos compilar isto!

\subsection{x86}

Aqui o que ambos compiladores GCC e MSVC produzem em uma plataforma x86:

\lstinputlisting[caption=\Otimizando GCC/MSVC (\assemblyOutput),style=customasmx86]{patterns/00_empty/1.s}

\myindex{x86!\Instructions!RET}
Há somente uma instrução: \RET, que retorna a execução para o \gls{caller}.

\subsection{ARM}

\lstinputlisting[caption=\OtimizandoKeilVI (\ARMMode) \assemblyOutput,style=customasmARM]{patterns/00_empty/1_Keil_ARM_O3.s}

O endereço de retorno não é salvo na pilha local no ARM \ac{ISA}, mas no link register,
então a instrução \INS{BX LR} faz a execução pular para este endereço---efetivamente retornando a execução
ao \gls{caller}.

\subsection{MIPS}

Há duas convenções de nomeação no mundo MIPS quando nomeando registradores: 
por número (de \$0 a \$31) ou por pseudônimo (\$V0, \$A0, etc.).

Abaixo é a saída em assembly do GCC que lista registradores por número:

\lstinputlisting[caption=\Optimizing GCC 4.4.5 (\assemblyOutput),style=customasmMIPS]{patterns/00_empty/MIPS.s}

\dots enquanto \IDA faz isso por pseudônimo:

\lstinputlisting[caption=\Optimizing GCC 4.4.5 (IDA),style=customasmMIPS]{patterns/00_empty/MIPS_IDA.lst}

\myindex{MIPS!\Instructions!J}
A primeira instrução é a instrução jump (J or JR) que retorna o fluxo de execução ao \gls{caller},
pulando para o endereço no registrador \$31 (ou \$RA).

Este é o registrador analogo ao \ac{LR} no ARM.

A segunda instrução é \ac{NOP}, que não faz nada.
Podemos ignorar isso por enquanto.

\subsubsection{Uma Nota Sobre Instruções MIPS e Nomes de Registradores}

Registradores e nomes de instruções no mundo MIPS são tradicionalmente escritas em caixa baixa.
Entretanto, para fins de consistencia, este livro irá usar caixa alta,
como é a convenção seguida por todos os outros \ac{ISA}s presentes nesse livro.

\subsection{Funções Vazias na Prática}

Tirando o fato que funções vazias parecem inúteis, elas são bem frequentes em código de baixo nível.

Primeiro de tudo, elas são bem populares em funções de debug, como esta:

\lstinputlisting[caption=\EN{Código \CCpp},style=customc]{patterns/00_empty/dbg_print.c}

Em uma versão sem debug (como na versão ''final''), \TT{\_DEBUG} não é definido,
assim como a função \TT{dbg\_print()}, tirando o fato de ser chamada durante a execução,
será vazia.

Similarmente, um método popular de proteção de software é fazer uma versão completa e uma versão de demo.
Uma demo pode deixar de ter funções importantes, como esta do exemplo:

\lstinputlisting[caption=\EN{\CCpp Code},style=customc]{patterns/00_empty/demo.c}

A função \TT{save\_file()} pode ser chamada quando o usuário clica em \TT{Arquivo->Salvar} no menu.
A versão demo pode ser entregue com esse item do menu desativado, mas mesmo se um cracker ativá-la, 
somente uma função vazia sem código usável será chamado.

IDA marca tais funções com nomes como \TT{nullsub\_00}, \TT{nullsub\_01}, etc.

