\section{Leere Funktion}
\label{empty_func}

Die denkbar einfachste Funktion ist sicher eine die nichts macht:

\lstinputlisting[caption=\EN{\CCpp Code}\DE{\CCpp-Code},style=customc]{patterns/00_empty/1.c}

Kompilieren wir diesen Quellcode!

\subsection{x86}
Nachfolgende die Ausgabe die sowohl der optimierende GCC als auch der MSVC-Compiler auf
einer x86-Plattform produziert:

\lstinputlisting[caption=\Optimizing GCC/MSVC (\assemblyOutput),style=customasmx86]{patterns/00_empty/1.s}

\myindex{x86!\Instructions!RET}
Es gibt lediglich eine Anweisung \RET, welche die Ausführung zurück an \gls{caller} übergibt.

\subsection{ARM}
\lstinputlisting[caption=\OptimizingKeilVI (\ARMMode) \assemblyOutput,style=customasmARM]{patterns/00_empty/1_Keil_ARM_O3.s}

Die Rücksprungadresse ist im ARM-\ac{ISA} nicht auf dem lokalen Stack gesichert sondern im Link-Register.
Die \INS{BX LR}-Anweisung führt dazu, dass die an diese Adresse gesprungen wird----also die Ausführung
wieder an den \gls{caller} übergeben wird.

\subsection{MIPS}
Es gibt zwei Namenskonventionen für Register in der MIPS-Welt:
durch Zahlen (von\$0 bis \$31) oder Pseudonamen (\$V0, \$A0, usw.).

Die Assembler-Ausgabe von GCC unten listet die Register mit Nummern auf:

\lstinputlisting[caption=\Optimizing GCC 4.4.5 (\assemblyOutput),style=customasmMIPS]{patterns/00_empty/MIPS.s}

\dots während \IDA Pseudonamen nutzt:

\lstinputlisting[caption=\Optimizing GCC 4.4.5 (IDA),style=customasmMIPS]{patterns/00_empty/MIPS_IDA.lst}

\myindex{MIPS!\Instructions!J}
Die erste Anweisung ist die Sprunganweisung die die Ausführung durch Springen an die Adresse
in Register \$31 (oder \$RA) wieder an den \gls{caller} übergibt.

Dies ist das Register analog zu \ac{LR} in ARM.

Die zweite Anweisung ist \ac{NOP} und macht gar nichts; sie kann hier erst mal ignoriert werden.

\subsubsection{Eine Anmerkung zu MIPS-Anweisungen und Registernamen}

Register und Anweisungsnamen sind in Bezug auf MIPS traditionellerweise klein geschrieben.
Um einen einheitlichen Stil zu haben, werden in diesem Buch jedoch Großbuchstaben genutzt.
Dies gilt für alle \ac{ISA}s die in diesem Buch besprochen werden.

\subsection{Leere Funktionen in der Praxis}

Abgesehen von der Tatsache, dass leere Funktionen nutzlos sind, begegnet man ihnen
in Low-Level-Code häufiger.

Zum Einen sind Debugging-Funktionen wie die Folgende recht verbreitet:

\lstinputlisting[caption=\EN{\CCpp Code}\DE{\CCpp-Code},style=customc]{patterns/00_empty/dbg_print.c}

Bei einem Nicht-Debug-Build (d.h. ``Release''), ist \TT{\_DEBUG} nicht definiert.
Die Funktion \TT{dbg\_print()} ist, obwohl sie bei der Ausführung aufgerufen wird, ohne Inhalt.

Eine verbreitete Art des Software-Schutzes ist es verschiedene Build-Versionen zu erstellen:
eine für legale Kunden und eine Demo-Version. Der Letzteren können zum Beispiel wichtige
Funktionen fehlen, wie folgt:

\lstinputlisting[caption=\EN{\CCpp Code}\DE{\CCpp-Code},style=customc]{patterns/00_empty/demo.c}

Die Funktion\TT{save\_file()} kann vom Nutzer über den Menüpunkt \TT{File->Save} aufgerufen werden.
Die Demo-Version könnte mit einem deaktivierten Menüpunkt ausgeliefert werden, selbst wenn ein
Cracker in der Lage ist diesen wieder zu aktivieren, wird die leere Funktion ohne sinnvollen
Inhalt ausgeführt

\IDA markiert solche Funktionen mit Namen wie \TT{nullsub\_00}, \TT{nullsub\_01}, usw.
