\section{Function prologue and epilogue}
\label{sec:prologepilog}
\myindex{Function epilogue}
\myindex{Function prologue}

A function prologue is a sequence of instructions at the start of a function. It often looks something like the following code fragment:

\begin{lstlisting}[style=customasmx86]
    push    ebp
    mov     ebp, esp
    sub     esp, X
\end{lstlisting}

What these instruction do: save the value in the \EBP register,
set the value of the \EBP register to the value of the \ESP and then allocate space on the stack 
for local variables.

The value in the \EBP stays the same over the period of the function execution and is to be used for local variables and 
arguments access. 
For the same purpose one can use \ESP, but since it changes over time this approach is not too convenient.

The function epilogue frees the allocated space in the stack, returns the value in the \EBP register back to its initial state 
and returns the control flow to the \gls{caller}:

\begin{lstlisting}[style=customasmx86]
    mov    esp, ebp
    pop    ebp
    ret    0
\end{lstlisting}

% what about calling convention?
Function prologues and epilogues are usually detected in disassemblers for function delimitation.

\subsection{\Recursion}

\myindex{\Recursion}
Epilogues and prologues can negatively affect the recursion performance.

More about recursion in this book: \myref{Recursion_and_tail_call}.

