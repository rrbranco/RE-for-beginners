\subsubsection{ARM: \OptimizingKeilVI (\ARMMode)}
\myindex{\CLanguageElements!switch}

\lstinputlisting[style=customasmARM]{patterns/08_switch/1_few/few_ARM_ARM_O3.asm}
Auch hier können wir bei Untersuchung des Code nicht sagen, ob im Quellcode ein switch() oder eine Folge von
if()-Ausdrücken vorliegt.


\myindex{ARM!\Instructions!ADRcc}
Wir finden hier Befehle mit Prädikaten wieder (wie \ADREQ (\IT{Equal})), welcher nur dann ausgeführt wird, wenn $R0=0$
und dann die Adresse des Strings IT{<<zero\textbackslash{}n>>} nach \Reg{0} lädt.

\myindex{ARM!\Instructions!BEQ}
Der folgende \ac{BEQ} Befehl übergibt den Control Flow an \TT{loc\_170}, falls $R0=0$.
Ein aufmerksamer Leser könnte sich fragen, ob \ac{BEQ} korrekt ausgelöst wird, da \ADREQ das \Reg{0} Register bereits
mit einem anderen Wert befüllt hat.
Es wird korrekt ausgelöst, da \ac{BEQ} die Flags, die vom \CMP Befehl gesetzt wurden, prüft und \ADREQ die Flags nicht
verändert.

Die übrigen Befehle kennen wir bereits.
Es gibt nur einen Aufruf von \printf am Ende und wir haben diesen Trick bereits hier
kennengelernt~(\myref{ARM_B_to_printf}). Am Ende gibt es drei Wege zur Ausführung von \printf.

\myindex{ARM!\Instructions!ADRcc}
\myindex{ARM!\Instructions!CMP}
Der letzte Befehl, \TT{CMP R0, \#2}, wird benötigt, um zu prüfen, ob $a=2$.
Wenn dies nicht der Fall ist, lädt \ADRNE einen Pointer auf den String \IT{<<something unknown \textbackslash{}n>>} nach
\Reg{0}, da $a$ bereits auf Gleichheit mit 0 oder 1 geprüft wurde und wir können sicher sein, dass die Variable $a$ an
dieser Stelle keinen dieser beiden Werte enthält.
Falls $R0=2$ ist, lädt \ADREQ einen Pointer auf den String \IT{<<two\textbackslash{}n>>} nach \Reg{0}. 

\subsubsection{ARM: \OptimizingKeilVI (\ThumbMode)}

\lstinputlisting[style=customasmARM]{patterns/08_switch/1_few/few_ARM_thumb_O3.asm}

% FIXME а каким можно? к каким нельзя? \myref{} ->
Wie bereits erwähnt ist es bei den meisten Befehlen im Thumb mode nicht möglich Prädikate für Bedingungen hinzuzufügen,
sodass der Thumb-Code hier dem leicht verständlichen x86 \ac{CISC}-style Code sehr ähnlich ist.

\subsubsection{ARM64: \NonOptimizing GCC (Linaro) 4.9}

\lstinputlisting[style=customasmARM]{patterns/08_switch/1_few/ARM64_GCC_O0_DE.lst}
Der Datentyp des Eingabewertes ist \Tint, deshalb wird das Register \RegW{0} anstatt des \RegX{0} Registers verwendet,
um ihn aufzunehmen.

Die Pointer auf die Strings werden an \puts mit einem \INS{ADRP}/\INS{ADD} Befehlspaar übergeben, genauso wie wir es im
\q{\HelloWorldSectionName} Beispiel gezeigt haben:~\myref{pointers_ADRP_and_ADD}.

\subsubsection{ARM64: \Optimizing GCC (Linaro) 4.9}

\lstinputlisting[style=customasmARM]{patterns/08_switch/1_few/ARM64_GCC_O3_DE.lst}
Ein besser optimiertes Stück Code. 
Der Befehl \TT{CBZ} (\IT{Compare and Branch on Zero}) springt, falls \RegW{0} gleich null ist.
Es gibt auch einen direkten Sprung zu \puts anstelle eines Aufrufs, so wie bereits hier
erklärt:~\myref{JMP_instead_of_RET}.

