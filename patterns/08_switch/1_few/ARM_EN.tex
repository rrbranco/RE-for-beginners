\subsubsection{ARM: \OptimizingKeilVI (\ARMMode)}
\myindex{\CLanguageElements!switch}

\lstinputlisting[style=customasmARM]{patterns/08_switch/1_few/few_ARM_ARM_O3.asm}

Again, by investigating this code we cannot say if it was a switch() in the original source code, 
or just a pack of if() statements.

\myindex{ARM!\Instructions!ADRcc}

Anyway, we see here predicated instructions again (like \ADREQ (\IT{Equal}))
which is triggered only in case $R0=0$, and then loads the address of the string \IT{<<zero\textbackslash{}n>>}
into \Reg{0}.
\myindex{ARM!\Instructions!BEQ}
The next instruction \ac{BEQ} redirects control flow to \TT{loc\_170}, if $R0=0$.

An astute reader may ask, will \ac{BEQ} trigger correctly since \ADREQ it
has already filled the \Reg{0} register before with another value?

Yes, it will since \ac{BEQ} checks the flags set by the \CMP instruction, 
and \ADREQ does not modify any flags at all.

The rest of the instructions are already familiar to us. 
There is only one call to \printf , at the end, and we have already examined this trick here~(\myref{ARM_B_to_printf}).
At the end, there are three paths to \printf{}.

\myindex{ARM!\Instructions!ADRcc}
\myindex{ARM!\Instructions!CMP}
The last instruction, \TT{CMP R0, \#2}, is needed to check if $a=2$.

If it is not true, then \ADRNE loads a pointer to the string \IT{<<something unknown \textbackslash{}n>>}
into \Reg{0}, since $a$ has already been checked to be equal to 0 or 1,
and we can sure that the $a$ variable is not equal to these numbers at this point.
And if $R0=2$, 
a pointer to the string \IT{<<two\textbackslash{}n>>}
will be loaded by \ADREQ into \Reg{0}.

\subsubsection{ARM: \OptimizingKeilVI (\ThumbMode)}

\lstinputlisting[style=customasmARM]{patterns/08_switch/1_few/few_ARM_thumb_O3.asm}

% FIXME а каким можно? к каким нельзя? \myref{} ->

As was already mentioned, it is not possible to add conditional predicates to most instructions in Thumb
mode, so the Thumb-code here is somewhat similar to the easily understandable x86 \ac{CISC}-style code.

\subsubsection{ARM64: \NonOptimizing GCC (Linaro) 4.9}

\lstinputlisting[style=customasmARM]{patterns/08_switch/1_few/ARM64_GCC_O0_EN.lst}

The type of the input value is \Tint, hence register \RegW{0} is used to hold it instead of the whole
\RegX{0} register.

The string pointers are passed to \puts using an \INS{ADRP}/\INS{ADD} instructions pair just like it was demonstrated in the 
\q{\HelloWorldSectionName} example:~\myref{pointers_ADRP_and_ADD}.

\subsubsection{ARM64: \Optimizing GCC (Linaro) 4.9}

\lstinputlisting[style=customasmARM]{patterns/08_switch/1_few/ARM64_GCC_O3_EN.lst}

Better optimized piece of code.
\TT{CBZ} (\IT{Compare and Branch on Zero}) instruction does jump if \RegW{0} is zero.
There is also a direct jump to \puts instead of calling it, like it was explained before:~\myref{JMP_instead_of_RET}.

