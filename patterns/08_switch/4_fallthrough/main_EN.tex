\subsection{Fall-through}

Another popular usage of \TT{switch()} operator is so-called \q{fallthrough}.
Here is simple example\footnote{Copypasted from \url{https://github.com/azonalon/prgraas/blob/master/prog1lib/lecture_examples/is_whitespace.c}}:

\lstinputlisting[numbers=left,style=customc]{patterns/08_switch/4_fallthrough/fallthrough1.c}

Slightly harder, from Linux kernel\footnote{Copypasted from \url{https://github.com/torvalds/linux/blob/master/drivers/media/dvb-frontends/lgdt3306a.c}}:

\lstinputlisting[numbers=left,style=customc]{patterns/08_switch/4_fallthrough/fallthrough2.c}

\lstinputlisting[caption=Optimizing GCC 5.4.0 x86,numbers=left,style=customasmx86]{patterns/08_switch/4_fallthrough/fallthrough2.s}

We can get to \TT{.L5} label if there is number 3250 at function's input.
But we can get to this label from the other side:
we see that there are no jumps between \printf call and \TT{.L5} label.

Now we can understand why \IT{switch()} statement is sometimes a source of bugs:
one forgotten \IT{break} will transform your
\IT{switch()} statement into \IT{fallthrough} one, and several blocks will be executed instead of single one.

