\subsubsection{MIPS}

\lstinputlisting[caption=\Optimizing GCC 4.4.5 (IDA),style=customasmMIPS]{patterns/08_switch/2_lot/MIPS_O3_IDA_RU.lst}

\myindex{MIPS!\Instructions!SLTIU}
Новая для нас инструкция здесь это \INS{SLTIU} (\q{Set on Less Than Immediate Unsigned}~--- установить,
если меньше чем значение, беззнаковое сравнение).

\myindex{MIPS!\Instructions!SLTU}
На самом деле, это то же что и \INS{SLTU} (\q{Set on Less Than Unsigned}), но \q{I} означает \q{immediate},
т.е. число может быть задано в самой инструкции.

\myindex{MIPS!\Instructions!BNEZ}
\INS{BNEZ} это \q{Branch if Not Equal to Zero} (переход если не равно нулю).

Код очень похож на код для других \ac{ISA}.
\myindex{MIPS!\Instructions!SLL}
\INS{SLL} (\q{Shift Word Left Logical}~--- логический сдвиг влево) совершает умножение на 4.
MIPS всё-таки это 32-битный процессор, так что все адреса в таблице переходов (\IT{jumptable}) 32-битные.

