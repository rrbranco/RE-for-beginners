\subsection{De nombreux cas}

Si une déclaration \TT{switch()} contient beaucoup de cas, il n'est pas très pratique
pour le compilateur de générer un trop gros code avec de nombreuses instructions
\JE/\JNE.

\lstinputlisting[label=switch_lot_c,style=customc]{patterns/08_switch/2_lot/lot.c}

\subsubsection{x86}

\myparagraph{MSVC \NonOptimizing}

Nous obtenons (MSVC 2010):

\lstinputlisting[caption=MSVC 2010,style=customasmx86]{patterns/08_switch/2_lot/lot_msvc_FR.asm}

\myindex{jumptable}

Ce que nous voyons ici est un ensemble d'appels à \printf avec des arguments variés.
Ils ont tous, non seulement des adresses dans la mémoire du procesus, mais aussi
des labels symboliques internes assignés par le compilateur.
Tous ces labels ont aussi mentionnés dans la table interne \TT{\$LN11@f}.

Au début de la fonctions, si $a$ est supérieur à 4, l'exécution est passée au
labal \TT{\$LN1@f}, oú \printf est appelé avec l'argument \TT{'something unknown'}.

Mais si la valeur de $a$ est inférieure ou égale à 4, elle est alors multipliée
par 4 et ajoutée à l'adresse de la table \TT{\$LN11@f}. C'est ainsi qu'une adresse
à l'intérieur de la table est construite, pointant exactement sur l'élément dont
nous avons besoin. Par exemple, supposons que $a$ soit égale à 2. $2*4 = 8$ (tous
les éléments de la table sont adressés dans un processus 32-bit et c'est pourquoi
les éléments ont une taille de 4 octets).
L'adresse de la table \TT{\$LN11@f} + 8 est celle de l'élément de la table oú
le label \TT{\$LN4@f} est stocké.
\JMP prend l'adresse de \TT{\$LN4@f} dans la table et y saute.

Cette table est quelquefois appelée \IT{jumptable} (table de saut) ou \IT{branch table}
(table de branchement)\footnote{L'ensemble de la méthode était appelé \IT{computed
GOTO} (GOTO calculés) dans les premières versions de ForTran:
\href{http://go.yurichev.com/17122}{wikipedia}.
Pas très pertinent de nos jours, mais quel terme!}.

Le \printf correspondant est appelé avec l'argument \TT{'two'}.\\
Littéralement, l'instruction \TT{jmp DWORD PTR \$LN11@f[ecx*4]} signifie
\IT{sauter au DWORD qui est stocké à l'adresse} \TT{\$LN11@f + ecx * 4}.

\TT{npad} (\myref{sec:npad}) est une macro du langage d'assemblage qui aligne le
label suivant de telle sorte qu'il soit stocké à une adresse alignée sur une limite
de 4 octets (ou 16 octets).
C'est très adapté pour le processeur puisqu'il est capable d'aller chercher des
valeurs 32-bit dans la mémoire à travers le bus mémoire, la mémoire cache, etc.,
de façons beaucoup plus éfficace si c'est aligné.

\clearpage
\mysubparagraph{\olly}
\myindex{\olly}

Essayons cet exemple dans \olly.
La valeur d'entrée de la fonction (2) est chargée dans \EAX:

\begin{figure}[H]
\centering
\myincludegraphics{patterns/08_switch/2_lot/olly1.png}
\caption{\olly: la valeur d'entrée de la fonction est chargée dans \EAX}
\label{fig:switch_lot_olly1}
\end{figure}

\clearpage
La valeur entrée est testée, est-elle plus grande que 4?
Si non, le saut par \q{défaut} n'est pas pris:
\begin{figure}[H]
\centering
\myincludegraphics{patterns/08_switch/2_lot/olly2.png}
\caption{\olly: 2 n'est pas plus grand que 4: le saut n'est pas pris}
\label{fig:switch_lot_olly2}
\end{figure}

\clearpage
Ici, nous voyons une table des sauts:

\begin{figure}[H]
\centering
\myincludegraphics{patterns/08_switch/2_lot/olly3.png}
\caption{\olly: calcul de l'adresse de destination en utilisant la table des sauts}
\label{fig:switch_lot_olly3}
\end{figure}

Ici, nous avons cliqué \q{Follow in Dump} $\rightarrow$ \q{Address constant}, donc
nous voyons maintenant la \IT{jumptable} dans la fenêtre des données.
Il y a 5 valeurs 32-bit\footnote{Elles sont soulignées par \olly car ce sont aussi
des FIXUPs: \myref{subsec:relocs}, nous y reviendrons plus tard}.
\ECX contient maintenant 2, donc le troisième élément (peut être indexé par 2\footnote{Á
propos des index de tableaux, lire aussi: \ref{arrays_at_one}}) de la table va être
utilisé.
Il est également possible de cliquer sur \q{Follow in Dump} $\rightarrow$ \q{Memory
address} et \olly va montrer l'élément adressé par l'instruction \JMP.
Il s'agit de \TT{0x010B103A}.

\clearpage
Après le saut, nous sommes en \TT{0x010B103A}: le code qui affiche \q{two} va être
exécuté:

\begin{figure}[H]
\centering
\myincludegraphics{patterns/08_switch/2_lot/olly4.png}
\caption{\olly: maintenant nous sommes au \IT{cas:} label}
\label{fig:switch_lot_olly4}
\end{figure}


\myparagraph{GCC \NonOptimizing}
\label{switch_lot_GCC}

Voyons ce que GCC 4.4.1 génère:

\lstinputlisting[caption=GCC 4.4.1,style=customasmx86]{patterns/08_switch/2_lot/lot_gcc.asm}

\myindex{x86!\Registers!JMP}

C'est presque la même chose, avec une petite nuance: l'argument \TT{arg\_0} est multiplié
par 4 en décalant de 2 bits vers la gauche (c'est preque comme multiplier par 4)~(\myref{SHR}).
Ensuite l'adresse du label est prise depuis le tableau \TT{off\_804855C}, stockée
dans \EAX, et ensuite \TT{JMP EAX} effectue le saut réel.


\subsubsection{ARM: \OptimizingKeilVI (\ARMMode)}
\label{sec:SwitchARMLot}

\lstinputlisting[caption=\OptimizingKeilVI (\ARMMode),style=customasmARM]{patterns/08_switch/2_lot/lot_ARM_ARM_O3.asm}

Ce code utilise les caractéristiques du mode ARM dans lequel toutes les instructions
ont une taille fixe de 4 octets.

Gardons à l'esprit que la valeur maximale de $a$ est 4 et que toute autre valeur
supérieure provoquera l'affichage de la chaîne \IT{<<something unknown\textbackslash{}n>>}

\myindex{ARM!\Instructions!CMP}
\myindex{ARM!\Instructions!ADDCC}
La première instruction \TT{CMP R0, \#5} compare la valeur entrée dans $a$ avec 5.

\footnote{ADD---addition}
L'instruction suivante, \TT{ADDCC PC, PC, R0,LSL\#2}, est exécutée si et seulement
si $R0 < 5$ (\IT{CC=Carry clear / Less than} retenue vide, inférieur à).
Par conséquent, si \TT{ADDCC} n'est pas exécutée (c'est le cas $R0 \geq 5$), un
saut au label \IT{default\_case} se produit.

Mais si $R0 < 5$ et que \TT{ADDCC} est exécuté, voici ce qui se produit:

La valeur dans \Reg{0} est multipliée par 4.
En fait, le suffixe de l'instruction \TT{LSL\#2} signifie \q{décalage à gauche de 2 bits}.
Mais comme nous le verrons plus tard~(\myref{division_by_shifting}) dans la section
\q{\ShiftsSectionName}, décaler de 2 bits vers la gauche est équivalent à multiplier
par 4.

Puis, nous ajoutons $R0*4$ à la valeur courante du \ac{PC}, et sautons à l'une
des instructions \TT{B} (\IT{Branch}) situées plus bas.

Au moment de l'exécution de \TT{ADDCC}, la valeur du \ac{PC} est en avance de 8
octets (\TT{0x180}) sur l'adresse à laquelle l'instruction \TT{ADDCC} se trouve
(\TT{0x178}), ou, autrement dit, en avance de 2 instructions.

\myindex{ARM!Pipeline}

C'est ainsi que le pipeline des processeurs ARM fonctionne: lorsque \TT{ADDCC} est
exécutée, le processeur, à ce moment, commence à préparer les instructions après
la suivante, c'est pourquoi \ac{PC} pointe ici. Cela doit être mémorisé.

Si $a=0$, elle sera ajoutée à la valeur de \ac{PC}, et la valeur courante de \ac{PC}
sera écrite dans \ac{PC} (qui est 8 octets en avant) et un saut au label \IT{loc\_180}
sera effectué, qui est 8 octets en avant du point où l'instruction se trouve.

Si $a=1$, alors $PC+8+a*4 = PC+8+1*4 = PC+12 = 0x184$ sera écrit dans \ac{PC}, qui
est l'adresse du label \IT{loc\_184}.

A chaque fois que l'on ajoute 1 à $a$, le \ac{PC} résultant est incrémenté de
4.

4 est la taille des instructions en mode ARM, et donc, la longueur de chaque instruction
\TT{B} desquelles il y a 5 à la suite.

Chacune de ces cinq instructions \TT{B} passe le contrôle plus loin, à ce qui a
été programmé dans le \IT{switch()}.

Le chargement du pointeur sur la chaîne correspondante se produit ici, etc.

\subsubsection{ARM: \OptimizingKeilVI (\ThumbMode)}

\lstinputlisting[caption=\OptimizingKeilVI (\ThumbMode),style=customasmARM]{patterns/08_switch/2_lot/lot_ARM_thumb_O3.asm}

\myindex{ARM!\ThumbMode}
\myindex{ARM!\ThumbTwoMode}

On ne peut pas être sûr que toutes ces instructions en mode Thumb et Thumb-2 ont
la même taille. On peut même dire que les intructions dans ces modes ont une longueur
variable, tout comme en x86.

\myindex{jumptable}

Donc, une table spéciale est ajoutée, qui contient des informations sur le nombre
de cas (sans inclure celui par défaut), et un offset pour chaque label auquel le
contrôle doit être passé dans chaque cas.

\myindex{ARM!Mode switching}
\myindex{ARM!\Instructions!BX}

Une fonction spéciale est présente ici qui s'occupe de la table et du passage du
contrôle, appelée \IT{\_\_ARM\_common\_switch8\_thumb}.
Elle commence avec \TT{BX PC}, dont la fonction est de passer le mode du processeur
en ARM.
Ensuite, vous voyez la fonction pour le traitement de la table.

C'est trop avancé pour être détaillé ici, donc passons cela.
% TODO explain it...

\myindex{ARM!\Registers!Link Register}

Il est intéressant de noter que la fonction utilise le regsitre \ac{LR} comme un
pointeur sur la table.

En effet, après l'appel de cette fonction, \ac{LR} contient l'adresse après\\
l'instruciton \TT{BL \_\_ARM\_common\_switch8\_thumb}, oú la table commence.

Il est intéressant de noter que le code est généré comme une fonction indépendante
afin de la ré-utiliser, donc le compilateur ne génèrera pas le même code pour chaque
déclaration switch().

\IDA l'a correctement identifié comme une fonction de service et une table, et a
ajouté un commentaire au label comme\\
\TT{jumptable 000000FA case 0}.


\subsubsection{MIPS}

\lstinputlisting[caption=GCC 4.4.5 \Optimizing (IDA),style=customasmMIPS]{patterns/08_switch/2_lot/MIPS_O3_IDA_FR.lst}

\myindex{MIPS!\Instructions!SLTIU}

La nouvelle instruction pour nous est \INS{SLTIU} (\q{Set on Less Than Immediate Unsigned}
Mettre si inférieur à la valeur immédiate non signée).
\myindex{MIPS!\Instructions!SLTU}

Ceci est la même que \INS{SLTU} (\q{Set on Less Than Unsigned}), mais \q{I} signifie
\q{immediate}, i.e., un nombre doit être spécifié dans l'instruction elle-même.

\myindex{MIPS!\Instructions!BNEZ}
\INS{BNEZ} est \q{Branch if Not Equal to Zero}.

Le code est très proche de l'autre \ac{ISA}s.
\myindex{MIPS!\Instructions!SLL}
\INS{SLL} (\q{Shift Word Left Logical}) effectue une multiplication par 4.

MIPS est un CPU 32-bit après tout, donc toutes les adresses de la \IT{jumtable}
sont 32-bits.



\subsubsection{\Conclusion{}}

Squelette grossier d'un \IT{switch()}:

% TODO: ARM, MIPS skeleton
\lstinputlisting[caption=x86,style=customasmx86]{patterns/08_switch/2_lot/skel1_FR.lst}

Le saut a une adresse de la table de saut peut aussi être implémenté en utilisant
cette instruction: \\
\TT{JMP jump\_table[REG*4]}.
Ou \TT{JMP jump\_table[REG*8]} en x64.

Une table de saut est juste un tableau de pointeurs, comme celle décrite plus
loin: \myref{array_of_pointers_to_strings}. 
