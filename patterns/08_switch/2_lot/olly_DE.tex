\clearpage
\mysubparagraph{\olly}
\myindex{\olly}

Untersuchen wir das Beispiel in \olly.
Der Eingabewert der Funktion (2) wird nach \EAX geladen: 

\begin{figure}[H]
\centering
\myincludegraphics{patterns/08_switch/2_lot/olly1.png}
\caption{\olly: das Funktionsargument wird nach \EAX geladen}
\label{fig:switch_lot_olly1}
\end{figure}

\clearpage
Es wird geprüft, ob der Eingabewert größer als 4 ist.
Falls nicht, wird der \q{default} Sprung nicht ausgeführt:
\begin{figure}[H]
\centering
\myincludegraphics{patterns/08_switch/2_lot/olly2.png}
\caption{\olly: 2 ist nicht größer als 4: kein Sprung wird ausgeführt}
\label{fig:switch_lot_olly2}
\end{figure}

\clearpage
Hier sehen wir eine Jumptable:

\begin{figure}[H]
\centering
\myincludegraphics{patterns/08_switch/2_lot/olly3.png}
\caption{\olly: Zieladresse mit Jumptable berechnen}
\label{fig:switch_lot_olly3}
\end{figure}
Wir haben \q{Follow in Dump}$\rightarrow$ \q{Address constant} geklickt, sodass wir jetzt die \IT{Jumptable} im
Datenfenster sehen. Hier sind 5 32-Bit-Werte\footnote{Diese werden von \olly unterstrichen, da
es auch FIXUPs sind: \myref{subsec:relocs}, wir kommen später darauf zurück}.
\ECX ist jetzt 2, sodass das zweite Element (beginnend bei null) der Tabelle verwendet wird.
% TBT
Es ist auch möglich durch Klicken auf q{Follow in Dump} $\rightarrow$ 
\q{Memory address} in \olly das Element, das durch den \JMP Befehl angesteuert wird, anzeigen zu lassen. Dieses Element
ist hier \TT{0x010B103A}.

\clearpage
Nach dem Sprung sind wir an der Stelle \TT{0x010B103A}: der Code zur Ausgabe von \q{two} wird jetzt ausgeführt:

\begin{figure}[H]
\centering
\myincludegraphics{patterns/08_switch/2_lot/olly4.png}
\caption{\olly: jetzt sind wir am \IT{case:} Label}
\label{fig:switch_lot_olly4}
\end{figure}
