\subsubsection{ARM: \OptimizingKeilVI (\ARMMode)}
\label{sec:SwitchARMLot}

\lstinputlisting[caption=\OptimizingKeilVI (\ARMMode),style=customasmARM]{patterns/08_switch/2_lot/lot_ARM_ARM_O3.asm}
Dieser Code verwendet das ARM mode Feature, das alle Befehle eine feste Länge von 4 Byte haben.

Vergessen wir nicht, dass der Maximalwert für $a$ 4 beträgt und jeder größere Wert zur Ausgabe des \IT{<<something
unknown\textbackslash{}n>>} Strings führt.

\myindex{ARM!\Instructions!CMP}
\myindex{ARM!\Instructions!ADDCC}
Der erste \TT{CMP R0, \#5} Befehl vergleich den Eingabewert $a$ mit 5.

\footnote{ADD---Addition}
Der nächste \TT{ADDCC PC, PC, R0,LSL\#2} Befehl wird nur ausgeführt, falls $R0 < 5$ (\IT{CC=Carry clear / kleiner als}).
Wenn \TT{ADDCC} nicht ausgeführt wird (d.h. $R0\geq 5$), wird ein Sprung zum \IT{default\_case} Label ausgeführt.

Aber wenn $R0 < 5$ und \TT{ADDCC} ausgeführt wird, wird das Folgende geschehen:

Der Wert in \Reg{0} wird mit 4 multipliziert.
Der Suffix \TT{LSL2} am Befehl steht dabei für \q{shift left by 2 bits}.
Aber wie wir später~(\myref{division_by_shifting}) im Abschnitt \q{\ShiftsSectionName} sehen werden, ist eine
Verschiebung um 2 Bits nach links äquivalent zu einer Multiplikation mit 4.

Danach addieren wir $R0\cdot 4$ zum aktuellen Wert in \ac{PC} und springen dadurch zu einem der unteren \TT{B}
(\IT{Branch}) Befehle.

Im Moment der Ausführung von\TT{ADDCC} ist der Wert von \ac{PC} (\TT{0x180}) 8 Bytes - oder mit anderen Worten: 2
Befehle - größer als die Adresse, an der sich der \TT{ADDCC} Befehl befindet (\TT{0x178})

\myindex{ARM!Pipeline}
So funktioniert die Pipeline in ARM Prozessoren: wenn \TT{ADDCC} ausgeführt wird, beginnt der Prozessor den Befehl
nach dem nächsten abzuarbeiten und deshalb zeigt \ac{PC} hierher. Das müssen wir im Kopf behalten.

Wenn $a=0$, dann wird dies zum Wert in \ac{PC} addiert und der aktuelle Wert des \ac{PC} wird nach \ac{PC} geschrieben
(welcher 8 Byte größer ist) und es wird zum Label \IT{loc\_180} gesprungen, welches 8 Byte größer ist als die Adresse
des \TT{ADDCC} Befehls.

Wenn $a=1$, dann wird $PC+8+a\cdot 4 = PC+8+1\cdot 4 = PC+12 = 0x184$ nach \ac{PC} geschrieben,was der Adresse des
\IT{loc\_184} Labels entspricht.

Jedes Mal wenn $a$ um 1 erhöht wird, erhöht sich der \ac{PC} um 4.

Dabei ist 4 die Länge eines Befehls im ARM mode und auch die Länge jedes \TT{B} Befehls, von denen sich hier 5 befinden.

Jeder dieser fünf \TT{B} Befehle gibt den Control Flow weiter so wie es im \IT{switch()} Ausdruck programmiert wurde.

Hier werden jeweils die Pointer auf die zugehörigen Strings geladen, etc.

\subsubsection{ARM: \OptimizingKeilVI (\ThumbMode)}

\lstinputlisting[caption=\OptimizingKeilVI (\ThumbMode),style=customasmARM]{patterns/08_switch/2_lot/lot_ARM_thumb_O3.asm}

\myindex{ARM!\ThumbMode}
\myindex{ARM!\ThumbTwoMode}
Man kann sich nicht sicher sein, dass alle Befehle im Thumb und Thumb-2 mode dieselbe Größe haben.
Man kann sogar sagen, dass die Befehle hier genau wie in x86 variable Längen haben.

\myindex{jumptable}
Deshalb wird hier eine spezielle Tabelle verwendet, die Informationen darüber enthält wie viele Fälle vorliegen (ohne
den Default-Case) und es wird für jeden Fall ein Label mit einem Offset für den Control Flow im zugehörigen Fall
angegeben.


\myindex{ARM!Mode switching}
\myindex{ARM!\Instructions!BX}
Hier taucht eine spezielle Funktion namens \IT{\_\_ARM\_common\_switch8\_thumb} auf, die mit der Tabelle und der
Übergabe des Control Flows umgeht.
Sie beginnt mit \TT{BX PC}, dessen Aufgabe es ist, den Prozessor in den ARM mode zu versetzen.
Danach finden wir die Funktion für den Umgang mit der Tabelle.

Es ist hier zu fortgeschritten um weiter ins Details zu gehen, daher lassen wir es für den Moment hierbei bewenden. 

% TODO explain it...

\myindex{ARM!\Registers!Link Register}
Ist ist interessant festzustellen, dass die Funktion das \ac{LR} Register als Pointer auf die Tabelle verwendet.

Tatsächlich enthält \ac{LR} nach dem Aufruf der Funktion die Adresse nach dem Befehl\\
\TT{BL \_\_ARM\_common\_switch8\_thumb}, an dem die Tabelle beginnt.

Es ist auch bemerkenswert, dass der Code als eine separate Funktion erzeugt wird, um wiederverwendet werden zu können,
sodass der Compiler nicht für jeden switch() Ausdruck den gleichen Code erzeugen muss.

\IDA hat erfolgreich ermittelt, dass es sich um eine Servicefunktion und eine Tabelle handelt und hat Kommentare wie
etwa \TT{jumptable 000000FA case 0} zu den Labels hinzugefügt.


