\subsection{Viele Fälle}
Wenn ein \TT{switch()} Ausdruck viele Fälle enthält, ist es für den Compiler nicht günstig sehr großen Code mit vielen
\JE/\JNE Befehlen zu erzeugen.

\lstinputlisting[label=switch_lot_c,style=customc]{patterns/08_switch/2_lot/lot.c}

\subsubsection{x86}

\myparagraph{\NonOptimizing MSVC}

Wir erhalten (MSVC 2010):

\lstinputlisting[caption=MSVC 2010,style=customasmx86]{patterns/08_switch/2_lot/lot_msvc_DE.asm}

\myindex{jumptable}
Was wir hier sehen ist eine Ansammlung von Aufrufen von \printf mit diversen Argumenten.
Alle haben nicht Adressen im Speicher des Prozesses, sondern auch interne symbolische Labels, die ihnen vom Compiler
zugewiesen werden.
Alle diese Labels werden auch in der internen Tabelle \TT{\$LN11@f} aufgeführt. 

Zu Beginn der Funktion wird der Control Flow an das Label \TT{\$LN1@f} abgegeben, wenn $a$ größer ist als 4. An diesem
Label wird \printf mit dem Argument \TT{'something unknown'} aufgerufen.

Wenn aber der Wert von $a$ kleiner gleich 4 ist, dann wird dieser mit 4 multipliziert und mit der Tabellenadresse
\TT{\$LN11@f} addiert. Auf diese Weise wird die Adresse innerhalb der Tabelle konstruiert und zeigt genau auf das
gewünschte Element. Nehmen wir zum Beispiel an, dass $a$ gleich 2 ist. $2\cdot 4=8$ (alle Tabellenelemente sind
Adressen in einem 32-Bit-Prozess und haben daher eine Breite von 4 Bytes).
Die Adresse an der Stelle \TT{\$LN11@f} + 8 ist das Tabellenelement, an dem das Label \TT{\$LN4@f} gespeichert ist.
\JMP holt die Adresse \TT{\$LN4@f} aus der Tabelle und springt dorthin.

Diese Tabelle wird manchmal \IT{Jumptable} oder \IT{Verzweigungstabelle} genannt\footnote{Die ganze Methode wurde
in früheren Versionen von Fortran \IT{berechnetes GOTO} genannt:
\href{http://go.yurichev.com/17122}{wikipedia}.
Heutzutage zwar nicht mehr relevant, aber welch ein Ausdruck!}.

Dann wird das zugehörige \printf mit dem Argument \TT{'two'} aufgerufen.\\
Der Befehl TT{jmp DWORD PTR \$LN11@f[ecx*4]} bedeutet dabei \IT{springe zum an dieser Stelle gespeicherten
DWORD}\TT{\$LN11@f + ecx * 4}.

\TT{npad} (\myref{sec:npad}) ist ein Assemblermakro, dass das nächste Label so angeordnet, dass es an einer 4 Byte
(oder 16 Bit) Adressgrenze gespeichert wird. Das ist für den Prozessor sehr praktisch, da er die 32-Bit-Werte aus dem
Speicher durch den Speicherbus, den Cache, etc. in effektiverer Weise laden kann.

\clearpage
\mysubparagraph{\olly}
\myindex{\olly}

Untersuchen wir das Beispiel in \olly.
Der Eingabewert der Funktion (2) wird nach \EAX geladen: 

\begin{figure}[H]
\centering
\myincludegraphics{patterns/08_switch/2_lot/olly1.png}
\caption{\olly: das Funktionsargument wird nach \EAX geladen}
\label{fig:switch_lot_olly1}
\end{figure}

\clearpage
Es wird geprüft, ob der Eingabewert größer als 4 ist.
Falls nicht, wird der \q{default} Sprung nicht ausgeführt:
\begin{figure}[H]
\centering
\myincludegraphics{patterns/08_switch/2_lot/olly2.png}
\caption{\olly: 2 ist nicht größer als 4: kein Sprung wird ausgeführt}
\label{fig:switch_lot_olly2}
\end{figure}

\clearpage
Hier sehen wir eine Jumptable:

\begin{figure}[H]
\centering
\myincludegraphics{patterns/08_switch/2_lot/olly3.png}
\caption{\olly: Zieladresse mit Jumptable berechnen}
\label{fig:switch_lot_olly3}
\end{figure}
Wir haben \q{Follow in Dump}$\rightarrow$ \q{Address constant} geklickt, sodass wir jetzt die \IT{Jumptable} im
Datenfenster sehen. Hier sind 5 32-Bit-Werte\footnote{Diese werden von \olly unterstrichen, da
es auch FIXUPs sind: \myref{subsec:relocs}, wir kommen später darauf zurück}.
\ECX ist jetzt 2, sodass das zweite Element (beginnend bei null) der Tabelle verwendet wird.
% TBT
Es ist auch möglich durch Klicken auf q{Follow in Dump} $\rightarrow$ 
\q{Memory address} in \olly das Element, das durch den \JMP Befehl angesteuert wird, anzeigen zu lassen. Dieses Element
ist hier \TT{0x010B103A}.

\clearpage
Nach dem Sprung sind wir an der Stelle \TT{0x010B103A}: der Code zur Ausgabe von \q{two} wird jetzt ausgeführt:

\begin{figure}[H]
\centering
\myincludegraphics{patterns/08_switch/2_lot/olly4.png}
\caption{\olly: jetzt sind wir am \IT{case:} Label}
\label{fig:switch_lot_olly4}
\end{figure}


\myparagraph{\NonOptimizing GCC}
\label{switch_lot_GCC}

Schauen wir was GCC 4.4.1 erzeugt:

\lstinputlisting[caption=GCC 4.4.1,style=customasmx86]{patterns/08_switch/2_lot/lot_gcc.asm}

\myindex{x86!\Registers!JMP}
Es ist bis auf eine Nuance das gleiche: das Argument \TT{arg\_0} wird mit 4 multipliziert durch eine Verschiebung von 2
Bits nach links (dies entspricht einer Multiplikation mit 4)~(\myref{SHR}).
Dann wird die Adresse des Labels vom \TT{off\_804855C} genommen, die in \EAX gespeichert wird, und dann wird mit
\TT{JMP EAX} der eigentliche Sprung durchgeführt.



\subsubsection{ARM: \OptimizingKeilVI (\ARMMode)}
\label{sec:SwitchARMLot}

\lstinputlisting[caption=\OptimizingKeilVI (\ARMMode),style=customasmARM]{patterns/08_switch/2_lot/lot_ARM_ARM_O3.asm}
Dieser Code verwendet das ARM mode Feature, das alle Befehle eine feste Länge von 4 Byte haben.

Vergessen wir nicht, dass der Maximalwert für $a$ 4 beträgt und jeder größere Wert zur Ausgabe des \IT{<<something
unknown\textbackslash{}n>>} Strings führt.

\myindex{ARM!\Instructions!CMP}
\myindex{ARM!\Instructions!ADDCC}
Der erste \TT{CMP R0, \#5} Befehl vergleich den Eingabewert $a$ mit 5.

\footnote{ADD---Addition}
Der nächste \TT{ADDCC PC, PC, R0,LSL\#2} Befehl wird nur ausgeführt, falls $R0 < 5$ (\IT{CC=Carry clear / kleiner als}).
Wenn \TT{ADDCC} nicht ausgeführt wird (d.h. $R0\geq 5$), wird ein Sprung zum \IT{default\_case} Label ausgeführt.

Aber wenn $R0 < 5$ und \TT{ADDCC} ausgeführt wird, wird das Folgende geschehen:

Der Wert in \Reg{0} wird mit 4 multipliziert.
Der Suffix \TT{LSL2} am Befehl steht dabei für \q{shift left by 2 bits}.
Aber wie wir später~(\myref{division_by_shifting}) im Abschnitt \q{\ShiftsSectionName} sehen werden, ist eine
Verschiebung um 2 Bits nach links äquivalent zu einer Multiplikation mit 4.

Danach addieren wir $R0\cdot 4$ zum aktuellen Wert in \ac{PC} und springen dadurch zu einem der unteren \TT{B}
(\IT{Branch}) Befehle.

Im Moment der Ausführung von\TT{ADDCC} ist der Wert von \ac{PC} (\TT{0x180}) 8 Bytes - oder mit anderen Worten: 2
Befehle - größer als die Adresse, an der sich der \TT{ADDCC} Befehl befindet (\TT{0x178})

\myindex{ARM!Pipeline}
So funktioniert die Pipeline in ARM Prozessoren: wenn \TT{ADDCC} ausgeführt wird, beginnt der Prozessor den Befehl
nach dem nächsten abzuarbeiten und deshalb zeigt \ac{PC} hierher. Das müssen wir im Kopf behalten.

Wenn $a=0$, dann wird dies zum Wert in \ac{PC} addiert und der aktuelle Wert des \ac{PC} wird nach \ac{PC} geschrieben
(welcher 8 Byte größer ist) und es wird zum Label \IT{loc\_180} gesprungen, welches 8 Byte größer ist als die Adresse
des \TT{ADDCC} Befehls.

Wenn $a=1$, dann wird $PC+8+a\cdot 4 = PC+8+1\cdot 4 = PC+12 = 0x184$ nach \ac{PC} geschrieben,was der Adresse des
\IT{loc\_184} Labels entspricht.

Jedes Mal wenn $a$ um 1 erhöht wird, erhöht sich der \ac{PC} um 4.

Dabei ist 4 die Länge eines Befehls im ARM mode und auch die Länge jedes \TT{B} Befehls, von denen sich hier 5 befinden.

Jeder dieser fünf \TT{B} Befehle gibt den Control Flow weiter so wie es im \IT{switch()} Ausdruck programmiert wurde.

Hier werden jeweils die Pointer auf die zugehörigen Strings geladen, etc.

\subsubsection{ARM: \OptimizingKeilVI (\ThumbMode)}

\lstinputlisting[caption=\OptimizingKeilVI (\ThumbMode),style=customasmARM]{patterns/08_switch/2_lot/lot_ARM_thumb_O3.asm}

\myindex{ARM!\ThumbMode}
\myindex{ARM!\ThumbTwoMode}
Man kann sich nicht sicher sein, dass alle Befehle im Thumb und Thumb-2 mode dieselbe Größe haben.
Man kann sogar sagen, dass die Befehle hier genau wie in x86 variable Längen haben.

\myindex{jumptable}
Deshalb wird hier eine spezielle Tabelle verwendet, die Informationen darüber enthält wie viele Fälle vorliegen (ohne
den Default-Case) und es wird für jeden Fall ein Label mit einem Offset für den Control Flow im zugehörigen Fall
angegeben.


\myindex{ARM!Mode switching}
\myindex{ARM!\Instructions!BX}
Hier taucht eine spezielle Funktion namens \IT{\_\_ARM\_common\_switch8\_thumb} auf, die mit der Tabelle und der
Übergabe des Control Flows umgeht.
Sie beginnt mit \TT{BX PC}, dessen Aufgabe es ist, den Prozessor in den ARM mode zu versetzen.
Danach finden wir die Funktion für den Umgang mit der Tabelle.

Es ist hier zu fortgeschritten um weiter ins Details zu gehen, daher lassen wir es für den Moment hierbei bewenden. 

% TODO explain it...

\myindex{ARM!\Registers!Link Register}
Ist ist interessant festzustellen, dass die Funktion das \ac{LR} Register als Pointer auf die Tabelle verwendet.

Tatsächlich enthält \ac{LR} nach dem Aufruf der Funktion die Adresse nach dem Befehl\\
\TT{BL \_\_ARM\_common\_switch8\_thumb}, an dem die Tabelle beginnt.

Es ist auch bemerkenswert, dass der Code als eine separate Funktion erzeugt wird, um wiederverwendet werden zu können,
sodass der Compiler nicht für jeden switch() Ausdruck den gleichen Code erzeugen muss.

\IDA hat erfolgreich ermittelt, dass es sich um eine Servicefunktion und eine Tabelle handelt und hat Kommentare wie
etwa \TT{jumptable 000000FA case 0} zu den Labels hinzugefügt.



\subsubsection{MIPS}

\lstinputlisting[caption=\Optimizing GCC 4.4.5 (IDA),style=customasmMIPS]{patterns/08_switch/2_lot/MIPS_O3_IDA_DE.lst}

\myindex{MIPS!\Instructions!SLTIU}
Der für uns neue Befehl ist \INS{SLTIU} (\q{Set on Less Than Immediate Unsigned}).

\myindex{MIPS!\Instructions!SLTU}
Dies ist das gleiche wie \INS{SLTU} (\q{Set on Less Than Unsigned}); das \q{I} steht dabei für \q{immediate}, d.h.
für den Befehl muss eine Zahl angegeben werden. 

\myindex{MIPS!\Instructions!BNEZ}
\INS{BNEZ} ist \q{Branch if Not Equal to Zero}.

Der Code ist den anderen \ac{ISA}s sehr ähnlich.
\myindex{MIPS!\Instructions!SLL}
\INS{SLL} (\q{Shift Word Left Logical}) führt eine Multiplikation mit 4 durch.
Da MIPS eine 32-Bit CPU ist, sind auch die Adressen in der \IT{Jumptable} 32 Bit groß.

\subsubsection{\Conclusion{}}

Das grobe Gerüst eines \IT{switch()}:

% TODO: ARM, MIPS skeleton
\lstinputlisting[caption=x86,style=customasmx86]{patterns/08_switch/2_lot/skel1_DE.lst}
Der Sprung zur Adresse in der Jumptable kann auch durch den folgenden Befehl realisiert werden:\\
\TT{JMP jump\_table[REG*4]}
oder \TT{JMP jump\_table[REG*8]} in x64.

Eine \IT{Jumptable} ist nur ein Array von Pointern, genau wie das hier beschriebene:
\myref{array_of_pointers_to_strings}.
