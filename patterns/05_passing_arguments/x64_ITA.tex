\subsection{x64}

\myindex{x86-64}

La situazione e' leggemente diversa in x86-64. Gli argomenti della funzione (i primi 4 o 6) 
sono passati tramite i registri. La funzione chiamata (\gls{callee}) legge quindi i parametri dai registri anziche' dallo stack.

\subsubsection{MSVC}

\Optimizing MSVC:

\lstinputlisting[caption=\Optimizing MSVC 2012 x64,style=customasmx86]{patterns/05_passing_arguments/x64_MSVC_Ox_EN.asm}

Come possiamo vedere, la piccola funzione \ttf prende tutti i suoi argomenti dai registri.

L'istruzione \LEA qui e' usata per l'addizione. Apparentemente il compilatore l'ha ritenuta piu' veloce di \TT{ADD}.
\myindex{x86!\Instructions!LEA}

\LEA e' anche usata nella funzione \main per preparare il primo e il tezo argomento di \ttf.
Il compilatore deve aver deciso che questo approccio e' piu' veloce del modo tradizionale di caricare valori nei registri usando l'istruzione \MOV.

Diamo un'occhiata all'output di MSVC senza ottimizzazioni:

\lstinputlisting[caption=MSVC 2012 x64,style=customasmx86]{patterns/05_passing_arguments/x64_MSVC_IDA_EN.asm}

L'output puo' lasciarci un po' perplessi in quanto tutti i 3 argomenti nei registri sono anche salvati nello stack per qualche motivo. 
\myindex{Shadow space}
\label{shadow_space}
Cio' e' detto \q{shadow space}
\footnote{\href{http://go.yurichev.com/17256}{MSDN}}: 
ogni Win64 potrebbe (ma non deve necessariamente farlo) salvare tutti i 4 vallori dei registri in questo spazio.
E questo avviene per due ragioni: 
1) e' eccessivo allocare un intero registro (o addirittura 4) per un argomento in input, pertanto sara' acceduto tramite lo stack.
2) il debugger sa sempre dove trovare gli argomenti della funzione ad un break
\footnote{\href{http://go.yurichev.com/17257}{MSDN}}.

Quindi, alcune funzioni piuttosto estese potrebbero salvare i loro argomenti nello \q{shadows space} nel caso in cui 
abbiano necessita' di utilizzarli durante l'esecuzione della funzione. Altre funzioni piu' piccole (come la nostra) potrebbero non farlo.

Allorace spazio nello \q{shadow space} e' responsabilita' del chiamante (\gls{caller}).

\subsubsection{GCC}

\Optimizing GCC genera codice piu' o meno comprensibile:

\lstinputlisting[caption=\Optimizing GCC 4.4.6 x64,style=customasmx86]{patterns/05_passing_arguments/x64_GCC_O3_EN.s}

\NonOptimizing GCC:

\lstinputlisting[caption=GCC 4.4.6 x64,style=customasmx86]{patterns/05_passing_arguments/x64_GCC_EN.s}

\myindex{Shadow space}

In System V *NIX (\SysVABI) non e' richiesto lo \q{shadow space}, ma la funizone chiamata (\gls{callee}) potrebbe aver bisogno di salvare
i suoi argomenti da qualche parte in caso di scarsita' di registri a disposizione.

\subsubsection{GCC: uint64\_t instead of int}

Il nostro esempio utilizza \Tint a 32-bit, motivo per cui viene usata la parte a 32-bit del registro (con prefisso \TT{E-}).

Puo' essere leggermente modificato per utilizzare valori a 64-bit:

\lstinputlisting[style=customc]{patterns/05_passing_arguments/ex64.c}

\lstinputlisting[caption=\Optimizing GCC 4.4.6 x64,style=customasmx86]{patterns/05_passing_arguments/ex64_GCC_O3_IDA_EN.asm}

Il codice e' lo stesso, ma in questo caso vengono usati i registri \IT{completi} (con prefisso \TT{R-}).
