\subsubsection{\NonOptimizingKeilVI (\ARMMode)}

\begin{lstlisting}[style=customasmARM]
.text:000000A4 00 30 A0 E1      MOV     R3, R0
.text:000000A8 93 21 20 E0      MLA     R0, R3, R1, R2
.text:000000AC 1E FF 2F E1      BX      LR
...
.text:000000B0             main
.text:000000B0 10 40 2D E9      STMFD   SP!, {R4,LR}
.text:000000B4 03 20 A0 E3      MOV     R2, #3
.text:000000B8 02 10 A0 E3      MOV     R1, #2
.text:000000BC 01 00 A0 E3      MOV     R0, #1
.text:000000C0 F7 FF FF EB      BL      f
.text:000000C4 00 40 A0 E1      MOV     R4, R0
.text:000000C8 04 10 A0 E1      MOV     R1, R4
.text:000000CC 5A 0F 8F E2      ADR     R0, aD_0        ; "%d\n"
.text:000000D0 E3 18 00 EB      BL      __2printf
.text:000000D4 00 00 A0 E3      MOV     R0, #0
.text:000000D8 10 80 BD E8      LDMFD   SP!, {R4,PC}
\end{lstlisting}

Die \main Funktion ruft einfach zwei weitere Funktionen auf, mit diesen drei werten die dann
der ersten Funktion übergeben werden.

Wie bereits angemerkt, auf ARM werden die erste 4 Werte in den ersten vier Registern übergeben (\Reg{0}-\Reg{3}).

Die \ttf Funktion, benutzt augenscheinlich die ersten drei Register (\Reg{0}-\Reg{2}) als Argumente

\myindex{ARM!\Instructions!MLA}

Die \TT{MLA} (\IT{Multiplikation Akkumulierung})
Instruktion multipliziert den ersten der beiden Operanden (\Reg{3} und \Reg{1}), addiert den dritten Operanden
(\Reg{2}) zum Produkt und speichert das Ergebnis in das nullte Register (\Reg{0}), wohin per Standard definiert
Funktionen ihre Rückgabe werte speichern.

\myindex{Fused multiply–add}
Multiplikation und Addition in einem\footnote{\WPMAO} (\IT{Fused multiply-add})
ist ist eine sehr nützliche Instruktion. Nebenbei bemerkt gab es eine solche Instruktion 
auf x86 nicht, bis FMA-Instruktionen in SIMD implementiert wurden.
\footnote{\href{http://go.yurichev.com/17103}{wikipedia}}.

Die erste Instruktion \TT{MOV R3, R0},
ist anscheinen redundant (es hätte anstatt eine einzelne \TT{MLA} Instruktion benutzt werden können).
Der Compiler hat die Instruktion nicht weg optimiert, da das Programm ohne Optimierungen compiliert wurde.

\myindex{ARM!Mode switching}
\myindex{ARM!\Instructions!BX}

Die \TT{BX} Instruktion gibt die Kontrolle an die Adresse zurück die im \ac{LR} Register gespeichert ist.
Falls nötig switcht die Instruktion den Prozessor Modus von Thumb zu ARM oder umgekehrt.
Das kann nötig sein da die \ttf Funktion nicht weiß von welcher Art Code sie vielleicht aufgerufen wird,
ARM oder Thumb.
Wenn die Funktion von Thumb Code aufgerufen wird gibt \TT{BX} nicht nur 
die Kontrolle an die aufrufende Funktion zurück, sondern switcht auch den Prozessor Modus auf 
Thumb. Es wird kein switch ausgeführt, falls die Funktion von ARM Code aufgerufen wurde \InSqBrackets{\ARMSevenRef A2.3.2} 
