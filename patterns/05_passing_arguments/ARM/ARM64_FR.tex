\subsubsection{ARM64}

\myparagraph{GCC (Linaro) 4.9 \Optimizing}

\myindex{Multiplication-addition fusionnées} % TODO FIXME verify
\myindex{ARM!\Instructions!MADD}
Tout ce qu'il y a ici est simple.
\TT{MADD} est juste une intruction qui effectue une multiplication/addition fusionnées
(similaire à l'instruction \TT{MLA} que nous avons déjà vue).
Tous les 3 arguments sont passés dans la partie 32-bit de registres-X.
Effectivement, le type des arguments est 32-bit \IT{int}.
Le résultat est renvoyé dans \TT{W0}.

\lstinputlisting[caption=GCC (Linaro) 4.9 \Optimizing,style=customasmARM]{patterns/05_passing_arguments/ARM/ARM64_O3_FR.s}

Étendons le type de toutes les données à 64-bit \TT{uint64\_t} et testons:

\lstinputlisting[style=customc]{patterns/05_passing_arguments/ex64.c}

\begin{lstlisting}[style=customasmARM]
f:
	madd	x0, x0, x1, x2
	ret
main:
	mov	x1, 13396
	adrp	x0, .LC8
	stp	x29, x30, [sp, -16]!
	movk	x1, 0x27d0, lsl 16
	add	x0, x0, :lo12:.LC8
	movk	x1, 0x122, lsl 32
	add	x29, sp, 0
	movk	x1, 0x58be, lsl 48
	bl	printf
	mov	w0, 0
	ldp	x29, x30, [sp], 16
	ret

.LC8:
	.string	"%lld\n"
\end{lstlisting}

La fonction \ttf{} est la même, seulement les registres-X 64-bit sont utilisés entièrement
maintenant.
Les valeurs longues sur 64-bit sont chargées dans les registres par partie, c'est
également décrit ici: \myref{ARM_big_constants_loading}.

\myparagraph{GCC (Linaro) 4.9 \NonOptimizing}

Le code sans optimisation est plus redondant:

\begin{lstlisting}[style=customasmARM]
f:
	sub	sp, sp, #16
	str	w0, [sp,12]
	str	w1, [sp,8]
	str	w2, [sp,4]
	ldr	w1, [sp,12]
	ldr	w0, [sp,8]
	mul	w1, w1, w0
	ldr	w0, [sp,4]
	add	w0, w1, w0
	add	sp, sp, 16
	ret
\end{lstlisting}

Le code sauve ses arguments en entrée dans la pile locale, dans le cas oú quelqu'un
(ou quelque chose) dans cette fonction aurait besoin d'utiliser les registres \TT{W0...W2}.
Cela évite d'écraser les arguments originels de la fonction, qui pourraient être
de nouveau utilisés par la suite.

Cela est appelé \IT{Zone de sauvegarde de registre.} (\ARMPCS).
L'appelée, toutefois, n'est pas obligée de les sauvegarder.
C'est un peu similaire au \q{Shadow Space}: \myref{shadow_space}.

Pourquoi est-ce que GCC 4.9 avec l'option d'optimisation supprime ce code de sauvegarde?
Parce qu'il a fait plus d'optimisation et en a conclu que les arguments de la fonction
n'allaient pas être utilisés par la suite et donc que les registres \TT{W0...W2}
ne vont pas être utilisés.

\myindex{ARM!\Instructions!MUL}
\myindex{ARM!\Instructions!ADD}

Nous avons donc une paire d'instructions \TT{MUL}/\TT{ADD} au lieu d'un seul \TT{MADD}.
