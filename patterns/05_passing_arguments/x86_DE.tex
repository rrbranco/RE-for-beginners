\subsection{x86}

\subsubsection{MSVC}

Das ist das Ergebnis nach dem kompilieren (MSVC 2010 Express):

\lstinputlisting[label=src:passing_arguments_ex_MSVC_cdecl,caption=MSVC 2010 Express]{patterns/05_passing_arguments/msvc_DE.asm}

\myindex{x86!\Registers!EBP}

Was wir hier sehen ist das die \main Funktion drei Zahlen auf den Stack schiebt und \TT{f(int,int,int).} aufruft

Der Argument zugriff innerhalb von \ttf wird organisiert mit der Hilfe von Makros wie zum Beispiel:\\
\TT{\_a\$ = 8}, 
auf die gleiche weise wie Lokale Variablen allerdings mit positiven Offsets (adressiert mit \IT{plus}).

Also adressieren wir die \IT{äussere} Seite des \glslink{stack frame}{Stack frame} indem wir \TT{\_a\$} Makros zum Wert des \EBP Registers addieren  

\myindex{x86!\Instructions!IMUL}
\myindex{x86!\Instructions!ADD}

Dann wird der Wert von $a$ in \EAX gespeichert. Nachdem die \IMUL Instruktion ausgeführt wurde, ist
der Wert in \EAX ein Produkt des Wertes aus \EAX und dem Inhalt von \TT{\_b}.

Nun addiert \ADD den Wert in \TT{\_c} auf \EAX

Der Wert in \EAX muss nicht verschoben werden: Der Wert von \EAX befindet sich schon wo er sein muss

Beim zurück kehren zur \gls{caller} Funktion, wird der Wert aus \EAX genommen und als Argument 
für den \printf Aufruf benutzt.


\subsubsection{MSVC + \olly}
\myindex{\olly}
Lasst uns die Darstellung in \olly betrachten

Wenn wir die erste Instruktion tracen in \ttf das auf eines der Argumente
zugreift (das erste), können wir sehen das \EBP auf den \gls{stack frame} zeigt,
dieser Frame wird mit dem roten Rechteck markiert dargestellt.

Das erste Element des \gls{stack frame} ist der gespeicherte Wert von \EBP,
das zweite Element ist \ac{RA}, das dritte Element ist das erste Funktions Argument, dann
folgt das zweite und dritte Funktions Argument.

Um auf das erste Funktions Argument zu zugreifen, muss man lediglich exakt 8 (2 32-Bit Wörter) zu 
\EBP addieren.

\olly erkennt diesen Umstand, und Kommentare zu den entsprechenden Stack Elementen hinzugefügt zum Beispiel:

\q{RETURN from} und \q{Arg1 = \dots}, etc.

Beachte: Funktions Argumente sind keine Mitglieder des Funktions Stack Frame, sie sind eher
Mitglieder des Stack Frame der \gls{caller} Funktion.

Deswegen, hat \olly die \q{Arg} Elemente als Mitglied eines anderen Stackframes identifiziert.

\begin{figure}[H]
\centering
\myincludegraphics{patterns/05_passing_arguments/olly.png}
\caption{\olly: inside of \ttf{} function}
\label{fig:passing_arguments_olly}
\end{figure}



\subsubsection{GCC}


Lasst uns das gleiche in GCC kompilieren und die Ergebnisse in \IDA betrachten:

\lstinputlisting[caption=GCC 4.4.1]{patterns/05_passing_arguments/gcc_DE.asm}

Das Ergebnis ist fast das gleiche aber mit kleineren Unterschieden die wir bereits früher
besprochen haben.

Der \gls{stack pointer} wird nicht zurück gesetzt nach den beiden Funktion aufrufen (f und printf),
weil die vorletzte \TT{LEAVE} Instruktion (\myref{x86_ins:LEAVE}) sich um das zurück setzen kümmert.
