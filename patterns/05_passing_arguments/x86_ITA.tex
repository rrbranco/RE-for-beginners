\subsection{x86}

\subsubsection{MSVC}

Ecco il risultato della compilazione ocn MSVC 2010 Express:

\lstinputlisting[label=src:passing_arguments_ex_MSVC_cdecl,caption=MSVC 2010 Express,style=customasmx86]{patterns/05_passing_arguments/msvc_EN.asm}

\myindex{x86!\Registers!EBP}

Vediamo che la funzione \main fa il push di 3 numeri sullo stack e chiama \TT{f(int,int,int).} 

L'accesso agli argomenti all'interno della funzione \ttf e' gestito con l'aiuto di macro come: \TT{\_a\$ = 8}, 
allo stesso modo delle variabili locali, ma con offset positivi.
Si sta quindi indirizzando il lato \IT{esterno} dello \gls{stack frame} sommando la macro \TT{\_a\$} al valore contenuto nel registro \EBP.

\myindex{x86!\Instructions!IMUL}
\myindex{x86!\Instructions!ADD}

Successivamente il valore di $a$ e' memorizzato in \EAX. A seguito dell'esecuzione dell'istruzione \IMUL, il valore in \EAX e' 
il \gls{prodotto} del valore in \EAX e del contenuto di \TT{\_b}.

Infine, \ADD aggiunge il valore in \TT{\_c} a \EAX.

Il valore \EAX non necessita di essere spostato: si trova gia' nel posto giusto.
Al termine, la funzione chiamante (\gls{caller}) prende il valore di \EAX e lo usa come argomento di \printf.

\subsubsection{MSVC + \olly}
\myindex{\olly}
Illustriamo il funzionamento con \olly.
Quando raggiungiamo la prima istruzione in \ttf che usa uno degli argomenti (il primo) 
notiamo che \EBP punta allo \gls{stack frame}, indentificato dal riquadro rosso.

Il primo elemento dello \gls{stack frame} e' il valore salvato di \EBP, 
il secondo e' il \ac{RA}, il terzo rappresenta il primo argomento della funzione, seguito dal secondo e terzo argomento.

Per accedere al primo argomento della funzione bisogna aggiungere esattamente 8 (2 word a 32-bit) a \EBP.

\olly e' in grado di distinguere gli argomenti in quesot modo, ed ha aggiunto dei commenti agli elementi dello stack, ad esempio:

\q{RETURN from} and \q{Arg1 = \dots}, etc.

N.B.: Gli argomenti della funzione non sono membri dello stack frame della funzione chiamata, appartengono allo stack frame della
funzione chiamante (\gls{caller}).

Pertanto \olly ha contrassegnato gli elementi \q{Arg} come membri di un altro stack frame.

\begin{figure}[H]
\centering
\myincludegraphics{patterns/05_passing_arguments/olly.png}
\caption{\olly: inside of \ttf{} function}
\label{fig:passing_arguments_olly}
\end{figure}


\subsubsection{GCC}

Compiliamo lo stesso esempio con GCC 4.4.1 ed osserviamo il risultato con \IDA:

\lstinputlisting[caption=GCC 4.4.1,style=customasmx86]{patterns/05_passing_arguments/gcc_EN.asm}

Il risultato e' pressoche' identico, a meno di piccole differenze gia' discusse in precedenza.

Lo \gls{stack pointer} non viene ripristinato dopo le due chiamate a funzione(f and printf), 
poiche' se ne occupa la penultima istruzione \TT{LEAVE} (\myref{x86_ins:LEAVE}) alla fine della funizone.
