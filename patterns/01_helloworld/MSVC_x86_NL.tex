\subsubsection{MSVC}

We compileren het in MSVC 2010:

\begin{lstlisting}
cl 1.cpp /Fa1.asm
\end{lstlisting}

(\TT{/Fa} optie zorgt ervoor dat de compiler het bestand met assembly listing genereert)

\begin{lstlisting}[caption=MSVC 2010,style=customasmx86]
CONST	SEGMENT
$SG3830	DB	'hello, world', 0AH, 00H
CONST	ENDS
PUBLIC	_main
EXTRN	_printf:PROC
; Function compile flags: /Odtp
_TEXT	SEGMENT
_main	PROC
	push	ebp
	mov	ebp, esp
	push	OFFSET $SG3830
	call	_printf
	add	esp, 4
	xor	eax, eax
	pop	ebp
	ret	0
_main	ENDP
_TEXT	ENDS
\end{lstlisting}

\NLph{}
Het verschil tussen Intel-syntax en AT\&T-syntax zal besproken worden in: \myref{ATT_syntax}.

De compiler heeft het bestand, \TT{1.obj} gegenereerd, hetwelk gelinkt wordt tot \TT{1.exe}.
In ons geval bevat het bestand twee segmenten: \TT{CONST} (voor data constanten) en \TT{\_TEXT}(voor code).

\myindex{\CLanguageElements!const}
\label{string_is_const_char}
De string \TT{hello, world} in \CCpp is van het type \TT{const char[]}\InSqBrackets{\TCPPPL p176, 7.3.2}, maar heeft geen eigen naam.
De compiler moet een manier hebben om met de string om te kunnen, en definieert er daarom de interne naam \TT{\$SG3830} voor.

Daarom kan het voorbeeld herschreven worden als volgt:

\lstinputlisting[style=customc]{patterns/01_helloworld/hw_2.c}

Laten we terug gaan naar de assembly listing. Zoals je kan zien, wordt de string beeindigd door een nul-byte. Dit is standaard voor \CCpp strings.
Meer over \CCpp strings: \myref{C_strings}.

In het code segment, \TT{\_TEXT}, is er slechts een functie tot nu toe: \main{}.
De functie \main begint met een proloog code en eindigt met een epiloog code (zoals bijna elke functie)
\footnote{Je kan hier meer over lezen in de sectie over functieprologen en epilogen ~(\myref{sec:prologepilog}).}.

\myindex{x86!\Instructions!CALL}
Na de functie proloog zien we de call naar de \printf{} functie: \INS{CALL \_printf}.
\myindex{x86!\Instructions!PUSH}
Voor de call wordt het adres van de string (of een pointer ernaar) die onze begroeting bevat, op de stack geplaatsd met de hulp van de \PUSH instructie.

Wanneer de \printf functie de controle teruggeeft aan de \main functie, staat het string adres (of de pointer ernaar) nog steeds op de stack.
Aangezien we dit niet meer nodig hebben, moet de \gls{stack pointer} (het \ESP register) gecorrigeerd worden.

\myindex{x86!\Instructions!ADD}
\INS{ADD ESP, 4} betekent dat er 4 wordt opgeteld bij de \ESP registerwaarde.

Waarom 4? Aangezien dit een 32-bit programma is, hebben we exact 4 bytes nodig om een adres door te geven via de stack. als het x64 code was, zouden we 8 bytes nodig gehad hebben.
\INS{ADD ESP, 4} is een effectief equivalent voor \TT{POP register} maar zonder gebruik van een register\footnote{CPU flags worden echter wel aangepast}.

\myindex{Intel C++}
\myindex{\oracle}
\myindex{x86!\Instructions!POP}

Met dezelfde reden zullen sommige compilers (zoals de Intel C++ Compiler) gebruik maken van \TT{POP ECX}
in plaats van \ADD (een dergelijk patroon kan waargenomen worden in de \oracle{} code aangezien deze gecompileerd is met de Intel C++ compiler).
Deze instructie heeft bijna hetzelfde effect, maar de inhoud van het \ECX register zal overschreven worden.
De Intel C++ Compiler gebruikt waarschijnlijk \TT{POP ECX} aangezien de opcode van deze instructie korter is dan \TT{ADD ESP, x} (1 byte voor \TT{POP} tegen 3 voor \TT{ADD}).

Hier is een voorbeeld van het gebruik van \POP in plaats van \ADD van \oracle{}:

\begin{lstlisting}[caption=\oracle 10.2 Linux (app.o bestand),style=customasmx86]
.text:0800029A                 push    ebx
.text:0800029B                 call    qksfroChild
.text:080002A0                 pop     ecx
\end{lstlisting}

%Lees meer over de stack in de sectie ~(\myref{sec:stack}).
\myindex{\CLanguageElements!return}
Na \printf aan te roepen, bevat de originele \CCpp code het statement \TT{return 0}~---return 0 als resultaat van de \main functie.

\myindex{x86!\Instructions!XOR}
In de gegenereerde code wordt dit geimplementeerd door de instructie \INS{XOR EAX, EAX}.

\myindex{x86!\Instructions!MOV}

\XOR is feitelijk simpelweg \q{eXclusive OR}\footnote{\href{http://go.yurichev.com/17118}{wikipedia}} maar de compilers gebruiken het vaak in plaats van
\INS{MOV EAX, 0} --- wederom omdat de opcode hiervoor iets korter is (2 bytes voor \XOR tegenover 5 voor \MOV).

\myindex{x86!\Instructions!SUB}
Sommige compilers gebruiken \INS{SUB EAX, EAX}, wat staat voor \IT{verminder de waarde in} \EAX \IT{met de waarde in} \EAX, wat in elke situatie resulteert in nul.

\myindex{x86!\Instructions!RET}
De laatste instructie \RET geeft de controle terug aan de \gls{caller}. Gewoonlijk is dit \CCpp \ac{CRT} code, die op zijn beurt de controle teruggeeft aan het \ac{OS}.

