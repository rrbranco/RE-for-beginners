\subsubsection{GCC: x86-64}

\myindex{x86-64}
Essayons GCC sur un Linux 64-bit:

\lstinputlisting[caption=GCC 4.4.6 x64,style=customasmx86]{patterns/01_helloworld/GCC_x64_FR.s}

Une méthode de passage des arguments à la fonction dans des registres est aussi utilisée sur Linux, *BSD et
\MacOSX est \SysVABI.

Les 6 premiers arguments sont passés dans les registres \RDI, \RSI, \RDX, \RCX, \Reg{8}, \Reg{9} et les autres---par
la pile.

Donc le pointeur sur la chaîne est passé dans \EDI (la partie 32-bit du registre).
Mais pourquoi ne pas utiliser la partie 64-bit, \RDI?

Il est important de garder à l'esprit que toutes les instructions \MOV en mode 64-bit qui écrivent quelque chose
dans la partie 32-bit inférieur du registre efface également les 32-bit supérieur (comme indiqué dans les manuels Intel:
\myref{x86_manuals}).\\
I.e., l'instruction \INS{MOV EAX, 011223344h} écrit correctement une valeur dans \RAX, puisque que les bits supérieurs
sont mis à zéro.

Si nous ouvrons le fichier objet compilé (.o), nous pouvons voir tous les opcodes des instructions
\footnote{Ceci doit être activé dans \textbf{Options $\rightarrow$ Disassembly $\rightarrow$ Number of opcode bytes}}:

\lstinputlisting[caption=GCC 4.4.6 x64,style=customasmx86]{patterns/01_helloworld/GCC_x64.lst}

\label{hw_EDI_instead_of_RDI}
Comme on le voit, l'instruction qui écrit dans \EDI en \TT{0x4004D4} occupe 5 octets.
La même instruction qui écrit une valeur sur 64-bit dans \RDI occupe 7 octets.
Il semble que GCC essaye d'économiser un peu d'espace.
En outre, cela permet d'être sûr que le segment de données contenant la chaîne ne sera pas alloué à une adresse supérieure
à 4 \gls{GiB}.

\label{SysVABI_input_EAX}
Nous voyons aussi que le registre \EAX est mis à zéro avant l'appel à la fonction \printf.
Ceci, car conformément à l' \ac{ABI} standard mentionnée plus haut,
le nombre de registres vectoriel utilisés est passé dans \EAX sur les systèmes *NIX en x86-64.

