\section{\HelloWorldSectionName}
\label{sec:helloworld}

We bekijken het beroemde voorbeeld uit het boek \KRBook:

\lstinputlisting[style=customc]{patterns/01_helloworld/hw.c}

\subsection{x86}

\EN{\subsubsection{MSVC}

Let's compile it in MSVC 2010:

\begin{lstlisting}
cl 1.cpp /Fa1.asm
\end{lstlisting}

(\TT{/Fa} option instructs the compiler to generate assembly listing file)

\begin{lstlisting}[caption=MSVC 2010,style=customasmx86]
CONST	SEGMENT
$SG3830	DB	'hello, world', 0AH, 00H
CONST	ENDS
PUBLIC	_main
EXTRN	_printf:PROC
; Function compile flags: /Odtp
_TEXT	SEGMENT
_main	PROC
	push	ebp
	mov	ebp, esp
	push	OFFSET $SG3830
	call	_printf
	add	esp, 4
	xor	eax, eax
	pop	ebp
	ret	0
_main	ENDP
_TEXT	ENDS
\end{lstlisting}

MSVC produces assembly listings in Intel-syntax.
The difference between Intel-syntax and AT\&T-syntax will be discussed in \myref{ATT_syntax}.

The compiler generated the file, \TT{1.obj}, which is to be linked into \TT{1.exe}.
In our case, the file contains two segments: \TT{CONST} (for data constants) and \TT{\_TEXT} (for code).

\myindex{\CLanguageElements!const}
\label{string_is_const_char}
The string \TT{hello, world} in \CCpp has type \TT{const char[]}\InSqBrackets{\TCPPPL p176, 7.3.2}, but it does not have its own name.
The compiler needs to deal with the string somehow so it defines the internal name \TT{\$SG3830} for it.

That is why the example may be rewritten as follows:

\lstinputlisting[style=customc]{patterns/01_helloworld/hw_2.c}

Let's go back to the assembly listing. As we can see, the string is terminated by a zero byte, which is standard for \CCpp strings.
More about \CCpp strings: \myref{C_strings}.

In the code segment, \TT{\_TEXT}, there is only one function so far: \main{}.
The function \main starts with prologue code and ends with epilogue code (like almost any function)
\footnote{You can read more about it in the section about function prologues and epilogues ~(\myref{sec:prologepilog}).}.

\myindex{x86!\Instructions!CALL}
After the function prologue we see the call to the \printf{} function:\\
\INS{CALL \_printf}.
\myindex{x86!\Instructions!PUSH}
Before the call, a string address (or a pointer to it) containing our greeting is placed on the stack with the help of the \PUSH instruction.

When the \printf function returns the control to the \main function, the string address (or a pointer to it) is still on the stack.
Since we do not need it anymore, the \gls{stack pointer} (the \ESP register) needs to be corrected.

\myindex{x86!\Instructions!ADD}
\INS{ADD ESP, 4} means add 4 to the \ESP register value.

Why 4? Since this is a 32-bit program, we need exactly 4 bytes for address passing through the stack. If it was x64 code we would need 8 bytes.
\INS{ADD ESP, 4} is effectively equivalent to \INS{POP register} but without using any register\footnote{CPU flags, however, are modified}.

\myindex{Intel C++}
\myindex{\oracle}
\myindex{x86!\Instructions!POP}

For the same purpose, some compilers (like the Intel C++ Compiler) may emit \TT{POP ECX} 
instead of \ADD (e.g., such a pattern can be observed in the \oracle{} code as it is compiled with the Intel C++ compiler).
This instruction has almost the same effect but the \ECX register contents will be overwritten.
The Intel C++ compiler supposedly uses \TT{POP ECX} since this instruction's opcode is shorter than \TT{ADD ESP, x} (1 byte for \TT{POP} against 3 for \TT{ADD}).

Here is an example of using \POP instead of \ADD from \oracle{}:

\begin{lstlisting}[caption=\oracle 10.2 Linux (app.o file),style=customasmx86]
.text:0800029A                 push    ebx
.text:0800029B                 call    qksfroChild
.text:080002A0                 pop     ecx
\end{lstlisting}

%Read more about the stack in section
% ~(\myref{sec:stack}).
\myindex{\CLanguageElements!return}
After calling \printf, the original \CCpp code contains the statement \TT{return 0}~---return 0 as the result of the \main function.

\myindex{x86!\Instructions!XOR}
In the generated code this is implemented by the instruction \INS{XOR EAX, EAX}.

\myindex{x86!\Instructions!MOV}

\XOR is in fact just \q{eXclusive OR}\footnote{\href{http://go.yurichev.com/17118}{wikipedia}} but the compilers often use it instead of
\INS{MOV EAX, 0}---again because it is a slightly shorter opcode (2 bytes for \XOR against 5 for \MOV).

\myindex{x86!\Instructions!SUB}
Some compilers emit \INS{SUB EAX, EAX}, which means \IT{SUBtract the value in the} \EAX \IT{from the value in} \EAX. That in any case will results in zero.

\myindex{x86!\Instructions!RET}
The last instruction \RET returns the control to the \gls{caller}. Usually, this is \CCpp \ac{CRT} code which in turn returns control to the \ac{OS}.

}
\FR{\subsubsection{MSVC}

Compilons-le avec MSVC 2010:

\begin{lstlisting}
cl 1.cpp /Fa1.asm
\end{lstlisting}

(L'option \TT{/Fa} indique au compilateur de générer un fichier avec le listing en assembleur)

\begin{lstlisting}[caption=MSVC 2010,style=customasmx86]
CONST	SEGMENT
$SG3830	DB	'hello, world', 0AH, 00H
CONST	ENDS
PUBLIC	_main
EXTRN	_printf:PROC
; Function compile flags: /Odtp
_TEXT	SEGMENT
_main	PROC
	push	ebp
	mov	ebp, esp
	push	OFFSET $SG3830
	call	_printf
	add	esp, 4
	xor	eax, eax
	pop	ebp
	ret	0
_main	ENDP
_TEXT	ENDS
\end{lstlisting}

MSVC génère des listings assembleur avec la syntaxe Intel.
La différence entre la syntaxe Intel et la syntaxe AT\&T sera discutée dans \myref{ATT_syntax}.

Le compilateur a généré le fichier object \TT{1.obj}, qui sera lié dans l'exécutable \TT{1.exe}.
Dans notre cas, le fichier contient deux segments: \TT{CONST} (pour les données constantes)
 et \TT{\_TEXT} (pour le code).

\myindex{\CLanguageElements!const}
\label{string_is_const_char}
La chaîne \TT{hello, world} en \CCpp a le type \TT{const char[]}\InSqBrackets{\TCPPPL p176, 7.3.2}, mais
elle n'a pas de nom en propre.
Le compilateur doit pouvoir l'utiliser et lui défini donc un nom interne \TT{\$SG3830} à cette fin.

C'est pourquoi l'exemple pourrait être récrit comme suit:

\lstinputlisting[style=customc]{patterns/01_helloworld/hw_2.c}

Retournons au listing assembleur. Comme nous le voyons, la chaîne est terminée avec un octet à zéro, ce qui
est le standard pour les chaînes \CCpp.

Dans le segment de code, \TT{\_TEXT}, il n'y a qu'une seule fonction: \main{}.
La fonction \main débute par le code du prologue et se termine par le code de l'épilogue 
(comme presque toutes les fonctions)
\footnote{Vous pouvez en lire plus dans la section concerant les prologues et épilogues de
fonctions ~(\myref{sec:prologepilog}).}.

\myindex{x86!\Instructions!CALL}
Après le prologue de la fonction nous voyons l'appel à la fonction \printf{}:\\
\INS{CALL \_printf}.
\myindex{x86!\Instructions!PUSH}
Avant l'appel, l'adresse de la chaîne (ou un pointeur sur elle) contenant notre message
 est placée sur la pile avec l'aide de l'instruction \PUSH.

Lorsque la fonction \printf rend le contrôle à la fonction \main, l'adresse de la chaîne (ou un pointeur sur elle)
est toujours sur la pile.
Comme nous n'en avons plus besoin, le pointeur de pile (\gls{stack pointer} le registre \ESP) doit être corrigé.

\myindex{x86!\Instructions!ADD}
\INS{ADD ESP, 4} signifie ajouter 4 à la valeur du registre \ESP.

Pourquoi 4? puisqu'il s'agit d'un programme 32-bit, nous avons besoin d'exactement 4 octets pour passer une adresse
par la pile. S'il s'agissait d'un code x64, nous aurions besoin de 8 octets.
\INS{ADD ESP, 4} est effectivement équivalent à \INS{POP register} mais sans utiliser de registre\footnote{Les
flags du CPU, toutefois, sont modifiés}.

\myindex{Intel C++}
\myindex{\oracle}
\myindex{x86!\Instructions!POP}

Pour la même raison, certains compilateurs (comme le compilateur C++ d'Intel) peuvent générer \TT{POP ECX}
à la place de \ADD (e.g., ce comportement peut être observé dans le code d'\oracle{} car il est
compilé avec le compilateur C++ d'Intel.
Cette instruction a presque le même effet mais le contenu du registre \ECX sera écrasé.
Le compilateur C++ d'Intel utilise probablement \TT{POP ECX} car l'opcode de cette instruction est plus
 court que celui de \TT{ADD ESP, x} (1 octet pour \TT{POP} contre 3 pour \TT{ADD}).

Voici un exemple d'utilisation de \POP à la place de \ADD dans \oracle{}:

\begin{lstlisting}[caption=\oracle 10.2 Linux (app.o file),style=customasmx86]
.text:0800029A                 push    ebx
.text:0800029B                 call    qksfroChild
.text:080002A0                 pop     ecx
\end{lstlisting}

%Lisez en plus sur la pile dans la section
% ~(\myref{sec:stack}).

\myindex{\CLanguageElements!return}
Après l'appel de \printf, le code \CCpp original contient la déclaration \TT{return 0}~---renvoie 0 comme valeur de retour de la fonction \main.

\myindex{x86!\Instructions!XOR}
Dans le code généré cela est implémenté par l'instruction \INS{XOR EAX, EAX}.

\myindex{x86!\Instructions!MOV}

\XOR est en fait un simple \q{OU exclusif (eXclusive OR}\footnote{\href{http://go.yurichev.com/17118}{wikipedia}} mais
les compilateurs l'utilisent souvent à la place de \INS{MOV EAX, 0}---à nouveau parce que l'opcode est légèrement plus
court (2 octets pour \XOR contre 5 pour \MOV).

\myindex{x86!\Instructions!SUB}
Certains compilateurs génèrent \INS{SUB EAX, EAX}, qui signifie \IT{Soustraire la valeur dans} \EAX \IT{de la valeur dans} \EAX,
 ce qui, dans tous les cas, donne zéro.

\myindex{x86!\Instructions!RET}
La dernière instruction \RET redonne le contrôle à l'appelant (\gls{caller}). c'est du code \CCpp \ac{CRT}, qui, à son tour, redonne le contrôle à l'\ac{OS}.

}
\ITA{\subsubsection{MSVC}

Compiliamolo in MSVC 2010:

\begin{lstlisting}
cl 1.cpp /Fa1.asm
\end{lstlisting}

(l'opzione \TT{/Fa} indica al compilatore di generare un file con il listato assembly)

\begin{lstlisting}[caption=MSVC 2010,style=customasmx86]
CONST	SEGMENT
$SG3830	DB	'hello, world', 0AH, 00H
CONST	ENDS
PUBLIC	_main
EXTRN	_printf:PROC
; Function compile flags: /Odtp
_TEXT	SEGMENT
_main	PROC
	push	ebp
	mov	ebp, esp
	push	OFFSET $SG3830
	call	_printf
	add	esp, 4
	xor	eax, eax
	pop	ebp
	ret	0
_main	ENDP
_TEXT	ENDS
\end{lstlisting}

\ITAph{}
La differenza tra le sintassi Intel e AT\&T-syntax sarà discussa al \myref{ATT_syntax}.

Il compilatore ha generato il file, \TT{1.obj}, che deve essere linkato (collegato) in \TT{1.exe}.
Nel nostro caso, il file contiene due segmentu: \TT{CONST} (per i dati constanti) e \TT{\_TEXT} (per il codice).

\myindex{\CLanguageElements!const}
\label{string_is_const_char}
La stringa \TT{hello, world} in \CCpp ha tipo \TT{const char[]}\InSqBrackets{\TCPPPL p176, 7.3.2}, ma non ha un nome proprio.
Il compilatore deve in qualche modo aver a che fare con la stringa, e la definisce quindi con il nome interno \TT{\$SG3830}.

Questo è il motivo per cui l'esempio potrebbe essere riscritto nel modo seguente:

\lstinputlisting[style=customc]{patterns/01_helloworld/hw_2.c}

Torniamo al listato assembly. Come possiamo vedere, la stringa è terminata con un byte zero, che è lo standard per la terminazione delle stringhe \CCpp.
\ITAph{}: \myref{C_strings}.

Nel code segment, \TT{\_TEXT}, esiste fino ad ora solo una funzione: \main{}.
La funzione \main inizia con il codice di prologo (prologue code) e termina con il codice di epilogo (epilogue code) (come quasi qualunque funzione)
\footnote{Maggiori informazioni si trovano nella sezione su prologo ed epilogo delle funzioni ~(\myref{sec:prologepilog}).}.

\myindex{x86!\Instructions!CALL}
Dopo il prologo della funzione, notiamo la chiamata alla funzione \printf{} : \INS{CALL \_printf}.
\myindex{x86!\Instructions!PUSH}
Prima della chiamata, l'indirizzo della stringa (o un puntatore ad essa) contenente il saluto viene messo sullo stack con l'aiuto dell'istruzione \PUSH.

Quando la funzione \printf restituisce il controllo alla funzione \main , l'indirizzo della stringa (o il puntatore) si trova ancora sullo stack.
Poichè non ne abbiamo più bisogno, lo \gls{stack pointer} (il registro \ESP ) deve essere corretto.

\myindex{x86!\Instructions!ADD}
\INS{ADD ESP, 4} significa aggiungi 4 al valore del registro \ESP.

Perchè 4? Essendo questo un programma a 32-bit, abbiamo esattamente bisogno di 4 bytes per passare un indirizzo attraverso lo stack. Se fosse stato codice x64 ne sarebbero serviti 8.
\INS{ADD ESP, 4} è a tutti gli effetti equivalente a \TT{POP register} ma senza usare alcun registro\footnote{i flag CPU vengono comunque modificati}.

\myindex{Intel C++}
\myindex{\oracle}
\myindex{x86!\Instructions!POP}

Per lo stesso scopo, alcuni compilatori (come l'Intel C++ Compiler) potrebbero emettere l'istruzione \TT{POP ECX} 
invece di \ADD (ad esempio, queto tipo di pattern può essere nel codice di \oracle{} che è compilato con l' Intel C++ compiler).
Questa istruzione ha pressoché lo stesso effetto ma il contenuto del registro \ECX sarà sovrascritto.
Il compilatore Intel C++ usa probabilmente \TT{POP ECX} poichè l'opcode di questa istruzione è più corto di \TT{ADD ESP, x} (1 byte per \TT{POP} contro 3 per \TT{ADD}).

Ecco un esempio dell'uso di \POP al posto di \ADD da \oracle{}:

\begin{lstlisting}[caption=\oracle 10.2 Linux (\ITAph{}),style=customasmx86]
.text:0800029A                 push    ebx
.text:0800029B                 call    qksfroChild
.text:080002A0                 pop     ecx
\end{lstlisting}

\myindex{\CLanguageElements!return}
Dopo la chiamata a \printf, il codice \CCpp originale contiene la direttiva \TT{return 0}~---restituisci 0 come risultato dalla funzione \main.

\myindex{x86!\Instructions!XOR}
Nel codice generato, questa è implementata dall'istruzione \INS{XOR EAX, EAX}.

\myindex{x86!\Instructions!MOV}

\XOR è infatti semplicemente \q{eXclusive OR, ovvero OR esclusivo}\footnote{\href{http://go.yurichev.com/17118}{wikipedia}} ma i compilatori lo usano spesso al posto di 
\INS{MOV EAX, 0}---ancora una volta poichè è un opcode leggermente più corto (2 byte per \XOR contro 5 per \MOV).

\myindex{x86!\Instructions!SUB}
Alcuni compilatori emettono l'istruzione \INS{SUB EAX, EAX}, che significa \IT{sottrai (SUBtract) il valore nel registro} \EAX \IT{dal valore nel registro} \EAX, che, in ogni caso, risulta uguale a zero.

\myindex{x86!\Instructions!RET}
L'ultima istruzione \RET restituisce il controllo al chiamante (\gls{caller}). Solitamente, questo è codice \CCpp \ac{CRT} , che, a sua volta, restituisce il controllo all' \ac{OS}.

}
\NL{\subsubsection{MSVC}

We compileren het in MSVC 2010:

\begin{lstlisting}
cl 1.cpp /Fa1.asm
\end{lstlisting}

(\TT{/Fa} optie zorgt ervoor dat de compiler het bestand met assembly listing genereert)

\begin{lstlisting}[caption=MSVC 2010,style=customasmx86]
CONST	SEGMENT
$SG3830	DB	'hello, world', 0AH, 00H
CONST	ENDS
PUBLIC	_main
EXTRN	_printf:PROC
; Function compile flags: /Odtp
_TEXT	SEGMENT
_main	PROC
	push	ebp
	mov	ebp, esp
	push	OFFSET $SG3830
	call	_printf
	add	esp, 4
	xor	eax, eax
	pop	ebp
	ret	0
_main	ENDP
_TEXT	ENDS
\end{lstlisting}

\NLph{}
Het verschil tussen Intel-syntax en AT\&T-syntax zal besproken worden in: \myref{ATT_syntax}.

De compiler heeft het bestand, \TT{1.obj} gegenereerd, hetwelk gelinkt wordt tot \TT{1.exe}.
In ons geval bevat het bestand twee segmenten: \TT{CONST} (voor data constanten) en \TT{\_TEXT}(voor code).

\myindex{\CLanguageElements!const}
\label{string_is_const_char}
De string \TT{hello, world} in \CCpp is van het type \TT{const char[]}\InSqBrackets{\TCPPPL p176, 7.3.2}, maar heeft geen eigen naam.
De compiler moet een manier hebben om met de string om te kunnen, en definieert er daarom de interne naam \TT{\$SG3830} voor.

Daarom kan het voorbeeld herschreven worden als volgt:

\lstinputlisting[style=customc]{patterns/01_helloworld/hw_2.c}

Laten we terug gaan naar de assembly listing. Zoals je kan zien, wordt de string beeindigd door een nul-byte. Dit is standaard voor \CCpp strings.
Meer over \CCpp strings: \myref{C_strings}.

In het code segment, \TT{\_TEXT}, is er slechts een functie tot nu toe: \main{}.
De functie \main begint met een proloog code en eindigt met een epiloog code (zoals bijna elke functie)
\footnote{Je kan hier meer over lezen in de sectie over functieprologen en epilogen ~(\myref{sec:prologepilog}).}.

\myindex{x86!\Instructions!CALL}
Na de functie proloog zien we de call naar de \printf{} functie: \INS{CALL \_printf}.
\myindex{x86!\Instructions!PUSH}
Voor de call wordt het adres van de string (of een pointer ernaar) die onze begroeting bevat, op de stack geplaatsd met de hulp van de \PUSH instructie.

Wanneer de \printf functie de controle teruggeeft aan de \main functie, staat het string adres (of de pointer ernaar) nog steeds op de stack.
Aangezien we dit niet meer nodig hebben, moet de \gls{stack pointer} (het \ESP register) gecorrigeerd worden.

\myindex{x86!\Instructions!ADD}
\INS{ADD ESP, 4} betekent dat er 4 wordt opgeteld bij de \ESP registerwaarde.

Waarom 4? Aangezien dit een 32-bit programma is, hebben we exact 4 bytes nodig om een adres door te geven via de stack. als het x64 code was, zouden we 8 bytes nodig gehad hebben.
\INS{ADD ESP, 4} is een effectief equivalent voor \TT{POP register} maar zonder gebruik van een register\footnote{CPU flags worden echter wel aangepast}.

\myindex{Intel C++}
\myindex{\oracle}
\myindex{x86!\Instructions!POP}

Met dezelfde reden zullen sommige compilers (zoals de Intel C++ Compiler) gebruik maken van \TT{POP ECX}
in plaats van \ADD (een dergelijk patroon kan waargenomen worden in de \oracle{} code aangezien deze gecompileerd is met de Intel C++ compiler).
Deze instructie heeft bijna hetzelfde effect, maar de inhoud van het \ECX register zal overschreven worden.
De Intel C++ Compiler gebruikt waarschijnlijk \TT{POP ECX} aangezien de opcode van deze instructie korter is dan \TT{ADD ESP, x} (1 byte voor \TT{POP} tegen 3 voor \TT{ADD}).

Hier is een voorbeeld van het gebruik van \POP in plaats van \ADD van \oracle{}:

\begin{lstlisting}[caption=\oracle 10.2 Linux (app.o bestand),style=customasmx86]
.text:0800029A                 push    ebx
.text:0800029B                 call    qksfroChild
.text:080002A0                 pop     ecx
\end{lstlisting}

%Lees meer over de stack in de sectie ~(\myref{sec:stack}).
\myindex{\CLanguageElements!return}
Na \printf aan te roepen, bevat de originele \CCpp code het statement \TT{return 0}~---return 0 als resultaat van de \main functie.

\myindex{x86!\Instructions!XOR}
In de gegenereerde code wordt dit geimplementeerd door de instructie \INS{XOR EAX, EAX}.

\myindex{x86!\Instructions!MOV}

\XOR is feitelijk simpelweg \q{eXclusive OR}\footnote{\href{http://go.yurichev.com/17118}{wikipedia}} maar de compilers gebruiken het vaak in plaats van
\INS{MOV EAX, 0} --- wederom omdat de opcode hiervoor iets korter is (2 bytes voor \XOR tegenover 5 voor \MOV).

\myindex{x86!\Instructions!SUB}
Sommige compilers gebruiken \INS{SUB EAX, EAX}, wat staat voor \IT{verminder de waarde in} \EAX \IT{met de waarde in} \EAX, wat in elke situatie resulteert in nul.

\myindex{x86!\Instructions!RET}
De laatste instructie \RET geeft de controle terug aan de \gls{caller}. Gewoonlijk is dit \CCpp \ac{CRT} code, die op zijn beurt de controle teruggeeft aan het \ac{OS}.

}
\RU{\subsubsection{MSVC}

Компилируем в MSVC 2010:

\begin{lstlisting}
cl 1.cpp /Fa1.asm
\end{lstlisting}

(Ключ \TT{/Fa} означает сгенерировать листинг на ассемблере)

\begin{lstlisting}[caption=MSVC 2010,style=customasmx86]
CONST	SEGMENT
$SG3830	DB	'hello, world', 0AH, 00H
CONST	ENDS
PUBLIC	_main
EXTRN	_printf:PROC
; Function compile flags: /Odtp
_TEXT	SEGMENT
_main	PROC
	push	ebp
	mov	ebp, esp
	push	OFFSET $SG3830
	call	_printf
	add	esp, 4
	xor	eax, eax
	pop	ebp
	ret	0
_main	ENDP
_TEXT	ENDS
\end{lstlisting}

MSVC выдает листинги в синтаксисе Intel.
Разница между синтаксисом Intel и AT\&T будет рассмотрена немного позже:

Компилятор сгенерировал файл \TT{1.obj}, который впоследствии будет слинкован линкером в \TT{1.exe}.
В нашем случае этот файл состоит из двух сегментов: \TT{CONST} (для данных-констант) и \TT{\_TEXT} (для кода).

\myindex{\CLanguageElements!const}
\label{string_is_const_char}
Строка \TT{hello, world} в \CCpp имеет тип \TT{const char[]}\InSqBrackets{\TCPPPL p176, 7.3.2}, однако не имеет имени.
Но компилятору нужно как-то с ней работать, поэтому он дает ей внутреннее имя \TT{\$SG3830}.

Поэтому пример можно было бы переписать вот так:

\lstinputlisting[style=customc]{patterns/01_helloworld/hw_2.c}

Вернемся к листингу на ассемблере. Как видно, строка заканчивается нулевым байтом~--- это требования стандарта \CCpp для строк.
Больше о строках в \CCpp: \myref{C_strings}.

В сегменте кода \TT{\_TEXT} находится пока только одна функция: \main{}.
Функция \main, как и практически все функции, начинается с пролога и заканчивается эпилогом
\footnote{Об этом смотрите подробнее в разделе о прологе и эпилоге функции ~(\myref{sec:prologepilog}).}.

\myindex{x86!\Instructions!CALL}
Далее следует вызов функции \printf{}: \INS{CALL \_printf}.
\myindex{x86!\Instructions!PUSH}
Перед этим вызовом адрес строки (или указатель на неё) с нашим приветствием (``Hello, world!'') при помощи инструкции \PUSH помещается в стек.

После того, как функция \printf возвращает управление в функцию \main, адрес строки (или указатель на неё) всё ещё лежит в стеке.
Так как он больше не нужен, то \glslink{stack pointer}{указатель стека} (регистр \ESP) корректируется.

\myindex{x86!\Instructions!ADD}
\INS{ADD ESP, 4} означает прибавить 4 к значению в регистре \ESP.

Почему 4? Так как это 32-битный код, для передачи адреса нужно 4 байта. В x64-коде это 8 байт.\\
\INS{ADD ESP, 4} эквивалентно \TT{POP регистр}, но без использования какого-либо регистра\footnote{Флаги процессора, впрочем, модифицируются}.

\myindex{Intel C++}
\myindex{\oracle}
\myindex{x86!\Instructions!POP}

Некоторые компиляторы, например, Intel C++ Compiler, в этой же ситуации могут вместо 
\ADD сгенерировать \TT{POP ECX} (подобное можно встретить, например, в коде \oracle{}, им скомпилированном),
что почти то же самое, только портится значение в регистре \ECX.
Возможно, компилятор применяет \TT{POP ECX}, потому что эта инструкция короче (1 байт у \TT{POP} против 3 у \TT{ADD}).

Вот пример использования \POP вместо \ADD из \oracle{}:

\begin{lstlisting}[caption=\oracle 10.2 Linux (файл app.o),style=customasmx86]
.text:0800029A                 push    ebx
.text:0800029B                 call    qksfroChild
.text:080002A0                 pop     ecx
\end{lstlisting}

%О стеке можно прочитать в соответствующем разделе
% ~(\myref{sec:stack}).
\myindex{\CLanguageElements!return}
После вызова \printf в оригинальном коде на \CCpp указано \TT{return 0}~--- вернуть 0 в качестве результата функции \main.

\myindex{x86!\Instructions!XOR}
В сгенерированном коде это обеспечивается инструкцией \\
\INS{XOR EAX, EAX}.

\myindex{x86!\Instructions!MOV}

\XOR, как легко догадаться~--- \q{исключающее ИЛИ}\footnote{\href{http://go.yurichev.com/17118}{wikipedia}}, но компиляторы часто используют его вместо простого
\INS{MOV EAX, 0} --- снова потому, что опкод короче (2 байта у \XOR против 5 у \MOV).

\myindex{x86!\Instructions!SUB}
Некоторые компиляторы генерируют \INS{SUB EAX, EAX}, что значит \IT{отнять значение в} \EAX \IT{от значения в }\EAX, что в любом случае даст 0 в результате.

\myindex{x86!\Instructions!RET}
Самая последняя инструкция \RET возвращает управление в вызывающую функцию. Обычно это код \CCpp \ac{CRT}, который, в свою очередь, вернёт управление в \ac{OS}.

}
\PTBR{\subsubsection{MSVC}

Vamos compilar esse código no MSVC 2010:

\begin{lstlisting}
cl 1.cpp /Fa1.asm
\end{lstlisting}

(A opção \TT{/Fa} instrui o compilador para gerar o arquivo de listagem em assembly)

\begin{lstlisting}[caption=MSVC 2010,style=customasmx86]
CONST	SEGMENT
$SG3830	DB	'hello, world', 0AH, 00H
CONST	ENDS
PUBLIC	_main
EXTRN	_printf:PROC
; Function compile flags: /Odtp
_TEXT	SEGMENT
_main	PROC
	push	ebp
	mov	ebp, esp
	push	OFFSET $SG3830
	call	_printf
	add	esp, 4
	xor	eax, eax
	pop	ebp
	ret	0
_main	ENDP
_TEXT	ENDS
\end{lstlisting}

\PTBRph{} \myref{ATT_syntax}.

O compilador gerou o arquivo \TT{1.obj}, que está ligado a \TT{1.exe}.
No nosso caso, o arquivo contém dois segmentos: \TT{CONST} (para informações que são constantes) e \TT{\_TEXT} (para o código).

\myindex{\CLanguageElements!const}
\label{string_is_const_char}
A string \TT{hello, word} em \CCpp tem seu tipo const \TT{const char[]} \InSqBrackets{\TCPPPL p176, 7.3.2}, mas não tem um nome.
O compilador precisa lidar com essa string de alguma maneira, definindo então o nome de \TT{\$SG3830} para ela.

Assim, o código pode ser reescrito da seguinte maneira:

\lstinputlisting[style=customc]{patterns/01_helloworld/hw_2.c}

Vamos voltar para a listagem em assembly. Como podemos ver, a string é delimitada por um byte de valor zero, o que é padrão para strings em \CCpp.
Mais sobre strings em \CCpp: \myref{C_strings}.

No segmento de código \TT{\_TEXT}, só há uma função por enquanto: \main{}.
A função \main{} começa com um código como cabeçalho e termina com outro como rodapé (quase como qualquer outra função)
\footnote{\PTBRph{} ~(\myref{sec:prologepilog}).}.

\myindex{x86!\Instructions!CALL}
Depois do cabeçalho da função, podemos ver a chamada para a função \printf{}: \INS{CALL \_printf}.
\myindex{x86!\Instructions!PUSH}
Antes da chamada, o endereço da string (ou um ponteiro para o mesmo) contendo nossa saudação (``Hello, world!'') é colocado na stack com a ajuda a instrução \PUSH.

Quando o a função printf() retorna o controle para a função main(), o endereço da string (ou o ponteiro para a mesma) ainda está na stack.
Como não precisamos mais dela, o apontador da stack (registrador \ESP) precisa ser corrigido.

\myindex{x86!\Instructions!ADD}
\INS{ADD ESP, 4} significa adicionar 4 para o valor do registrador \ESP.

Mas por que 4? Como esse é um programa de 32-bits, nós precisamos exatamente 4 bytes para endereço passando pela stack.
\INS{ADD ESP, 4} é equivalente a um POP mas sem precisar de nenhum registrador\footnote{\ac{TBT}: CPU flags worden echter wel aangepast}.

\myindex{Intel C++}
\myindex{\oracle}
\myindex{x86!\Instructions!POP}

Pelos mesmos motivos, alguns compiladores (como o Intel C++ Compiler) podem emitir \TT{POP ECX} ao invés de \ADD (esse padrão pode ser observado no código do \oracle{} pois ele é compilado com o Intel C++ Compiler).
Essa instrução tem quase o mesmo efeito mas o conteúdo de ECX seria apagado.
O Intel C++ provavelmente usa \TT{POP ECX} pois o opcode é menor do que \TT{ADD ESP, x} (1 byte para \POP ao invés de 3 para \ADD).

Aqui está um exemplo do uso de \POP ao invés de \ADD do \oracle{}:

\begin{lstlisting}[caption=\oracle 10.2 Linux (app.o file),style=customasmx86]
.text:0800029A                 push    ebx
.text:0800029B                 call    qksfroChild
.text:080002A0                 pop     ecx
\end{lstlisting}

\myindex{\CLanguageElements!return}
Depois de chamar \printf{}, o código original em \CCpp contém a declaração \TT{return 0} --- return 0 como o resultado da função \main{}.

\myindex{x86!\Instructions!XOR}
No código gerado, isso é implementado pela instrução \INS{XOR EAX, EAX}.

\myindex{x86!\Instructions!MOV}

\XOR é a condição lógica ``ou exclusivo''\footnote{\href{http://go.yurichev.com/17118}{wikipedia}} que os compiladores geralmente usam ao invés de 
\INS{MOV EAX, 0} --- de novo por causa de um pequeno decréscimo no número de bytes necessários (2 bytes para \XOR contra 5 para a instrução \MOV).

\myindex{x86!\Instructions!SUB}
Alguns compiladores também usam \INS{SUB EAX, EAX}, que significa, SUBtrair o valor contido em \EAX do valor em \EAX, que, em qualquer caso, resultará em zero.

\myindex{x86!\Instructions!RET}
A última instrução \RET retorna o controle para onde a função foi chamada. Geralmente, isso é código \CCpp \ac{CRT}, que retorna o controle para o sistema operacional.

}
\DE{\subsubsection{MSVC}

Das Beispiel wird jezt in MSVC 2010 kompiliert:

\begin{lstlisting}
cl 1.cpp /Fa1.asm
\end{lstlisting}

(Die \TT{/Fa}-Option weist den Compiler an, Assembler-Code auszugeben.)

\begin{lstlisting}[caption=MSVC 2010,style=customasmx86]
CONST	SEGMENT
$SG3830	DB	'hello, world', 0AH, 00H
CONST	ENDS
PUBLIC	_main
EXTRN	_printf:PROC
; Function compile flags: /Odtp
_TEXT	SEGMENT
_main	PROC
	push	ebp
	mov	ebp, esp
	push	OFFSET $SG3830
	call	_printf
	add	esp, 4
	xor	eax, eax
	pop	ebp
	ret	0
_main	ENDP
_TEXT	ENDS
\end{lstlisting}

MSVC erstellt Assembler-Code im Intel-Syntax.
Der Unterschied zum AT\&T-Syntax wird später in \myref{ATT_syntax} behandelt.

Der Compiler generiert die Datei \TT{1.obj}, die anschließend zu \TT{1.exe} gelinkt wird.
In diesem Fall besteht die Datei aus zwei Segmenten: \TT{CONST} (für konstante Daten) und \TT{\_TEXT} (für Quellcode).

\myindex{\CLanguageElements!const}
\label{string_is_const_char}
Die Zeichenkette \TT{hello, world} hat in \CCpp den Typ \TT{const char[]}\InSqBrackets{\TCPPPL p176, 7.3.2}, aber keinen eigenen Bezeichner.
Da der Compiler jedoch irgendwie auf diese Zeichenkette zugreifen muss, definiert er den internen Namen \TT{\$SG3830}.

Aus diesem Grund kann das Beispiel auch wie folgt geschrieben werden:

\lstinputlisting[style=customc]{patterns/01_helloworld/hw_2.c}

Nochmal zurück zum Assembler-Listing: wie man sehen kann ist die Zeichenkette gemäß dem \CCpp-Standard mit einem 0-Byte abgeschlossen.
Mehr über \CCpp-Zeichenketten ist im Abschnitt \myref{C_strings} zu finden.

% Not sure what the technically precise translation of prologue and epilogue is
In dem Code-Segment \TT{\_TEXT} ist lediglich eine Funktion: \main{}.
Diese startet mit einem Prolog-Teil und endet mit einem Epilog-Teil (wie fast alle Funktionen)
\footnote{Mehr darüber in dem Abschnitt über Funktions-Prologe und -Epiloge ~(\myref{sec:prologepilog}).}.

\myindex{x86!\Instructions!CALL}
Nach dem Funktions-Prolog ist der Aufruf der \printf{}-Funktion zu sehen:\\
\INS{CALL \_printf}.
\myindex{x86!\Instructions!PUSH}
Vor dem Aufruf wird die Adresse der Zeichenkette (oder ein Zeiger darauf) mit dem Inhalt unserer Begrüßung
auf dem Stack gespeichert. Dies geschieht durch die \PUSH-Anweisung.

Wenn \printf die Ausführung wieder an \main übergibt, befindet sich die Adresse der Zeichenkette (oder ein Zeiger darauf) immer noch auf dem Stack.
Da diese jedoch nicht mehr benötigt wird, muss der \gls{stack pointer} (das \ESP-Register) korrigiert werden.

\myindex{x86!\Instructions!ADD}
\INS{ADD ESP, 4} bedeutet, dass der Wert 4 zu dem \ESP-Rregister-Wert addiert wird.

Warum 4? Da dies ein 32-Bit-Programm ist, werden exakt 4 Byte benötigt um Adressen auf dem Stack abzulegen. Wenn dies x64-Code wäre, würden 8 Byte benötigt.
\INS{ADD ESP, 4} ist quasi gleichbedeutend mit \INS{POP Register} jedoch ohne die Verwendung von Registern\footnote{Statusregister der CPU können sich jedoch ändern}.

\myindex{Intel C++}
\myindex{\oracle}
\myindex{x86!\Instructions!POP}

Aus dem gleichen Grund generieren einige Compiler (wie der Intel C++-Compiler) die Anweisung \TT{POP ECX}
anstatt \ADD (dieses Muster kann zum Beispiel im \oracle{}-Code gefunden werden, da dieser mit dem Intel-Compiler erstellt wurde).
Diese Anweisung hat nahezu den gleichen Effekt, nur dass die Inhalte des \ECX-Registers überschrieben werden.
Der Intel C++-Compiler nutzt \TT{POP ECX} vermutlich, da der OpCode für diese Anweisung kürzer ist als \TT{ADD ESP, x} (1 Byte für \TT{POP} und 3 Byte für \TT{ADD}),

Nachfolgend ein Beispiel unter der Verwendung von \POP anstatt \ADD aus \oracle{}:

\begin{lstlisting}[caption=\oracle 10.2 Linux (app.o file),style=customasmx86]
.text:0800029A                 push    ebx
.text:0800029B                 call    qksfroChild
.text:080002A0                 pop     ecx
\end{lstlisting}

%Mehr über den Stack gibt es im Abschnitt
% ~(\myref{sec:stack}).
\myindex{\CLanguageElements!return}
Nachdem \printf aufgerufen wurde, enthält der Original-\CCpp-Code die Anweisung \TT{return 0} als Rückgabewert der \main-Funktion.

\myindex{x86!\Instructions!XOR}
In dem hier gezeigten Code ist dies durch die Anweisung \INS{XOR EAX, EAX} realisiert.

\myindex{x86!\Instructions!MOV}

XOR ist lediglich ein \q{exklusives Oder}\footnote{\href{http://go.yurichev.com/17118}{wikipedia}} aber der Compiler nutzt dies oft anstatt
\INS{MOV EAX, 0}---auch hier wieder aufgrund des leicht kürzeren OpCodes (2 Byte für \XOR und 5 Byte für \MOV).

\myindex{x86!\Instructions!SUB}
Einige Compiler erzeugen \INS{SUB EAX, EAX}, was \IT{Subtrahiere den Wert in} \EAX \IT{vom Wert in} \EAX bedeutet.
In jedem Fall erzeugt dies einen Wert von Null.

\myindex{x86!\Instructions!RET}
Die letzte Anweisung \RET gibt die Ausführungskontrolle wieder an die aufrufende Funktion \gls{caller}.
Üblicherweise ist dies \CCpp \ac{CRT}-Code welcher wiederum die Kontrolle an das Betriebssystem (\ac{OS}) übergibt.
}

\EN{\subsubsection{GCC}

Now let's try to compile the same \CCpp code in the GCC 4.4.1 compiler in Linux: \TT{gcc 1.c -o 1}.
Next, with the assistance of the \IDA disassembler, let's see how the \main function was created.
\IDA, like MSVC, uses Intel-syntax\footnote{We could also have GCC produce assembly listings in Intel-syntax by applying the options \TT{-S -masm=intel}.}.

\begin{lstlisting}[caption=code in \IDA,style=customasmx86]
main            proc near

var_10          = dword ptr -10h

                push    ebp
                mov     ebp, esp
                and     esp, 0FFFFFFF0h
                sub     esp, 10h
                mov     eax, offset aHelloWorld ; "hello, world\n"
                mov     [esp+10h+var_10], eax
                call    _printf
                mov     eax, 0
                leave
                retn
main            endp
\end{lstlisting}

\myindex{Function prologue}
\myindex{x86!\Instructions!AND}
The result is almost the same.
The address of the \TT{hello, world} string (stored in the data segment) is loaded in the \EAX register first and then it is saved onto the stack. \\
In addition, the function prologue has \INS{AND ESP, 0FFFFFFF0h}~---this 
instruction aligns the \ESP register value on a 16-byte boundary.
This results in all values in the stack being aligned the same way (The CPU performs better if the values it is dealing with are located in memory at addresses aligned 
on a 4-byte or 16-byte boundary)\footnote{\URLWPDA}.

\myindex{x86!\Instructions!SUB}
\INS{SUB ESP, 10h} allocates 16 bytes on the stack. Although, as we can see hereafter, only 4 are necessary here.

This is because the size of the allocated stack is also aligned on a 16-byte boundary.

% TODO1: rewrite.
\myindex{x86!\Instructions!PUSH}
The string address (or a pointer to the string) is then stored directly onto the stack without using the \PUSH instruction.
\IT{var\_10}~---is a local variable and is also an argument for \printf{}.
Read about it below.

Then the \printf function is called.

Unlike MSVC, when GCC is compiling without optimization turned on, it emits \TT{MOV EAX, 0} instead of a shorter opcode.

\myindex{x86!\Instructions!LEAVE}
The last instruction, \LEAVE~---is the equivalent of the \TT{MOV ESP, EBP} and \TT{POP EBP} instruction pair~---in other words, this instruction sets the \gls{stack pointer} (\ESP) back and restores the \EBP register to its initial state.
This is necessary since we modified these register values (\ESP and \EBP) at the beginning of the function (by executing \INS{MOV EBP, ESP} / \INS{AND ESP, \ldots}).

\subsubsection{GCC: \ATTSyntax}
\label{ATT_syntax}

Let's see how this can be represented in assembly language AT\&T syntax.
This syntax is much more popular in the UNIX-world.

\begin{lstlisting}[caption=let's compile in GCC 4.7.3]
gcc -S 1_1.c
\end{lstlisting}

We get this:

\lstinputlisting[caption=GCC 4.7.3,style=customasmx86]{patterns/01_helloworld/GCC.s}

The listing contains many macros (beginning with dot). These are not interesting for us at the moment.

For now, for the sake of simplification, we can ignore them (except the \IT{.string} macro which
encodes a null-terminated character sequence just like a C-string). Then we'll see this
\footnote{This GCC option can be used to eliminate \q{unnecessary} macros: \IT{-fno-asynchronous-unwind-tables}}:

\lstinputlisting[caption=GCC 4.7.3,style=customasmx86]{patterns/01_helloworld/GCC_refined.s}

\myindex{\ATTSyntax}
\myindex{\IntelSyntax}
Some of the major differences between Intel and AT\&T syntax are:

\begin{itemize}

\item
Source and destination operands are written in opposite order.

In Intel-syntax: <instruction> <destination operand> <source operand>.

In AT\&T syntax: <instruction> <source operand> <destination operand>.

\myindex{\CStandardLibrary!memcpy()}
\myindex{\CStandardLibrary!strcpy()}
Here is an easy way to memorize the difference:
when you deal with Intel-syntax, you can imagine that there is an equality sign ($=$) between operands
and when you deal with AT\&T-syntax imagine there is a right arrow ($\rightarrow$)
\footnote{By the way, in some C standard functions (e.g., memcpy(), strcpy()) the arguments
are listed in the same way as in Intel-syntax: first the pointer to the destination memory block, and then
the pointer to the source memory block.}.

\item
AT\&T: Before register names, a percent sign must be written (\%) and before numbers a dollar sign (\$).
Parentheses are used instead of brackets.

\item
AT\&T: A suffix is added to instructions to define the operand size:

\begin{itemize}
\item q --- quad (64 bits)
\item l --- long (32 bits)
\item w --- word (16 bits)
\item b --- byte (8 bits)
\end{itemize}

% TODO1 simple example may be? \RU{Например mov\textbf{l}, movb, movw представляют различые версии инсструкция mov} \EN {For example: movl, movb, movw are variations of the mov instruction}

\end{itemize}

Let's go back to the compiled result: it is identical to what we saw in \IDA.
With one subtle difference: \TT{0FFFFFFF0h} is presented as \TT{\$-16}.
It is the same thing: \TT{16} in the decimal system is \TT{0x10} in hexadecimal.
\TT{-0x10} is equal to \TT{0xFFFFFFF0} (for a 32-bit data type).

\myindex{x86!\Instructions!MOV}
One more thing: the return value is to be set to 0 by using the usual \MOV, not \XOR.
\MOV just loads a value to a register.
Its name is a misnomer (data is not moved but rather copied). In other architectures, this instruction is named \q{LOAD} or \q{STORE} or something similar.

}
\FR{\subsubsection{GCC}

Maintenant essayons de compiler le même code \CCpp avec le compilateur GCC 4.4.1 sur Linux: \TT{gcc 1.c -o 1}.
Ensuite, avec l'aide du désassembleur \IDA, regardons comment la fonction \main a été créée.
\IDA, comme MSVC, utilise la syntaxe Intel\footnote{GCC peut aussi produire un listing assembleur utilisant la syntaxe Intel en lui passant les options \TT{-S -masm=intel}.}.

\begin{lstlisting}[caption=code in \IDA,style=customasmx86]
main            proc near

var_10          = dword ptr -10h

                push    ebp
                mov     ebp, esp
                and     esp, 0FFFFFFF0h
                sub     esp, 10h
                mov     eax, offset aHelloWorld ; "hello, world\n"
                mov     [esp+10h+var_10], eax
                call    _printf
                mov     eax, 0
                leave
                retn
main            endp
\end{lstlisting}

\myindex{Prologue de fonction}
\myindex{x86!\Instructions!AND}
Le résultat est presque le même.
L'adresse de la chaîne \TT{hello, world} (stockée dans le segment de donnée) est d'abord chargée dans
 le registre \EAX puis sauvée dans la pile. \\
En plus, le prologue de la fonction a l'instruction \INS{AND ESP, 0FFFFFFF0h}~---cette
instruction aligne le registre \ESP sur une limite de 16-octet.
Ainsi, toutes les valeurs sur la pile seront alignées de la même manière (Le CPU
est plus performant si les adresses avec lesquelles il travaille en mémoire sont
alignées sur des limites de 4-octet ou 16-octet)\footnote{\URLWPDA}.

\myindex{x86!\Instructions!SUB}
\INS{SUB ESP, 10h} réserve 16 octets sur la pile. Pourtant, comme nous allons le voir, seuls 4 sont nécessaires ici.

C'est parce que la taille de la pile allouée est alignée sur une limite de 16-octet.

% TODO1: rewrite.
\myindex{x86!\Instructions!PUSH}
L'adresse de la chaîne est (ou un pointeur vers la chaîne) est stockée directement sur la pile sans utiliser
l'instruction \PUSH.
\IT{var\_10}~---est une variable locale et est aussi un argument pour \printf{}.
Lisez à ce propos en dessous.

Ensuite la fonction \printf est appelée.

Contrairement à MSVC, lorsque GCC compile sans optimisation, il génère \TT{MOV EAX, 0} au lieu d'un opcode plus court.

\myindex{x86!\Instructions!LEAVE}
La dernière instruction, \LEAVE~---est équivalente à la paire d'instruction \TT{MOV ESP, EBP} et \TT{POP EBP}~---en d'autres mots,
cette instruction déplace le pointeur de pile (\gls{stack pointer}) (\ESP) et remet le registre \EBP dans son état initial.
Cela est nécessaire puisque nous avons modifié les valeurs de ces registres (\ESP et \EBP) au début de la fonction (en exécutant \INS{MOV EBP, ESP} / \INS{AND ESP, \ldots}).

\subsubsection{GCC: \ATTSyntax}
\label{ATT_syntax}

Voyons comment cela peut-être représenté en langage d'assemblage avec la syntaxe AT\&T.
Cette syntaxe est bien plus populaire dans le monde UNIX.

\begin{lstlisting}[caption=compilons avec GCC 4.7.3]
gcc -S 1_1.c
\end{lstlisting}

Nous obtenons ceci:

\lstinputlisting[caption=GCC 4.7.3,style=customasmx86]{patterns/01_helloworld/GCC.s}

Le listing contient beaucoup de macros (qui commencent avec un point). Cela ne nous intéresse pas pour le moment.

Pour l'instant, dans un soucis de simplification, nous pouvons les ignorer (excepté la macro \IT{.string}
qui encode une séquence de caractères terminée par un octet nul, comme une chaîne C).
Ensuite nous verrons cela\footnote{Cette option de GCC peut être utilisée pour éliminer les macros \q{non necéssaire}:
\IT{-fno-asynchronous-unwind-tables}}:

\lstinputlisting[caption=GCC 4.7.3,style=customasmx86]{patterns/01_helloworld/GCC_refined.s}

\myindex{\ATTSyntax}
\myindex{\IntelSyntax}
Quelques-une des différences majeures entre la syntax Intel et AT\&T sont:

\begin{itemize}

\item
Opérandes source et destination sont écrites dans l'ordre inverse.

En syntaxe Intel: <instruction> <opérande de destination> <opérande source>.

En syntaxe AT\&T: <instruction> <opérande source> <opérande de destination>.

\myindex{\CStandardLibrary!memcpy()}
\myindex{\CStandardLibrary!strcpy()}
Voici un moyen simple de mémoriser la différence:
lorsque vous avez affaire avec la syntaxe Intel, vous pouvez imaginer qu'il y a un signe
égal ($=$) entre les opérandes et lorsque vous avez affaire avec la syntaxe AT\&T imaginez
qu'il y a un flèche droite ($\rightarrow$)
\footnote{A propos, dans certaine fonction C standard (e.g., memcpy(), strcpy()) les arguments
sont listés de la même manière que dans la syntaxe Intel: en premier se trouve le pointeur du
bloc mémoire de destination, et ensuite le pointeur sur le bloc mémoire source.}.

\item
AT\&T: Avant les noms de registres, un signe pourcent doit être écrit (\%) et avant les nombres, un signe dollar (\$).
Les parenthèses sont utilisées à la place des crochets.

\item
AT\&T: un suffixe est ajouté à l'instruction pour définir la taille de l'opérande:

\begin{itemize}
\item q --- quad (64 bits)
\item l --- long (32 bits)
\item w --- word (16 bits)
\item b --- byte (8 bits)
\end{itemize}

% TODO1 simple example may be? \RU{Например mov\textbf{l}, movb, movw представляют различые версии инсструкция mov} \EN {For example: movl, movb, movw are variations of the mov instruction}
% \FR {Par exemple: movl, movb, movw sont des variantes de l'instruction mov}

\end{itemize}

Retournons au résultat compilé: il est identique à ce que l'on voit dans \IDA.
Avec une différence subtile: \TT{0FFFFFFF0h} est représenté avec \TT{\$-16}.
C'est la même chose: \TT{16} dans le système décimal est \TT{0x10} en hexadécimal.
\TT{-0x10} est équivalent à \TT{0xFFFFFFF0} (pour un type de donnée sur 32-bit).

\myindex{x86!\Instructions!MOV}
Encore une chose: la valeur de retour est mise à 0 en utilisant un \MOV usuel, pas un \XOR.
\MOV charge seulement la valeur dans le registre.
Le nom est mal choisi (la donnée n'est pas déplacée, mais plutôt copiée). Dans d'autres architectures, cette instruction
est nommée \q{LOAD} ou \q{STORE} ou quelque chose de similaire.

}
\RU{\subsubsection{GCC}

Теперь скомпилируем то же самое компилятором GCC 4.4.1 в Linux: \TT{gcc 1.c -o 1}.
Затем при помощи \IDA посмотрим как скомпилировалась функция \main.
\IDA, как и MSVC, показывает код в синтаксисе Intel\footnote{Мы также можем заставить GCC генерировать листинги в этом формате при помощи ключей \TT{-S -masm=intel}.}.

\begin{lstlisting}[caption=код в \IDA,style=customasmx86]
main            proc near

var_10          = dword ptr -10h

                push    ebp
                mov     ebp, esp
                and     esp, 0FFFFFFF0h
                sub     esp, 10h
                mov     eax, offset aHelloWorld ; "hello, world\n"
                mov     [esp+10h+var_10], eax
                call    _printf
                mov     eax, 0
                leave
                retn
main            endp
\end{lstlisting}

\myindex{Function prologue}
\myindex{x86!\Instructions!AND}
Почти то же самое. 
Адрес строки \TT{hello, world}, лежащей в сегменте данных, вначале сохраняется в \EAX, затем записывается в стек.
А ещё в прологе функции мы видим \TT{AND ESP, 0FFFFFFF0h}~--- 
эта инструкция выравнивает значение в \ESP по 16-байтной границе, делая все значения 
в стеке также выровненными по этой границе (процессор более эффективно работает с переменными, расположенными
в памяти по адресам кратным 4 или 16)\footnote{\URLWPDA}.

\myindex{x86!\Instructions!SUB}
\INS{SUB ESP, 10h} выделяет в стеке 16 байт. Хотя, как будет видно далее, здесь достаточно только 4.

Это происходит потому, что количество выделяемого места в локальном стеке тоже выровнено по 16-байтной границе.

% TODO1: rewrite.
\myindex{x86!\Instructions!PUSH}
Адрес строки (или указатель на строку) затем записывается прямо в стек без помощи инструкции \PUSH.
\IT{var\_10} одновременно и локальная переменная и аргумент для \printf{}. Подробнее об этом будет ниже.

Затем вызывается \printf.

В отличие от MSVC, GCC в компиляции без включенной оптимизации генерирует \TT{MOV EAX, 0} вместо более короткого опкода.

\myindex{x86!\Instructions!LEAVE}
Последняя инструкция \LEAVE~--- это аналог команд \TT{MOV ESP, EBP} и \TT{POP EBP}~--- то есть возврат \glslink{stack pointer}{указателя стека} и регистра \EBP в первоначальное состояние.
Это необходимо, т.к. в начале функции мы модифицировали регистры \ESP и \EBP{}\\
(при помощи \INS{MOV EBP, ESP} / \INS{AND ESP, \ldots}).

\subsubsection{GCC: \ATTSyntax}
\label{ATT_syntax}

Попробуем посмотреть, как выглядит то же самое в синтаксисе AT\&T языка ассемблера.
Этот синтаксис больше распространен в UNIX-мире.

\begin{lstlisting}[caption=компилируем в GCC 4.7.3]
gcc -S 1_1.c
\end{lstlisting}

Получим такой файл:

\lstinputlisting[caption=GCC 4.7.3,style=customasmx86]{patterns/01_helloworld/GCC.s}

Здесь много макросов (начинающихся с точки). Они нам пока не интересны.

Пока что, ради упрощения, мы можем 
их игнорировать (кроме макроса \IT{.string}, при помощи которого кодируется последовательность символов, 
оканчивающихся нулем~--- такие же строки как в Си). И тогда получится следующее
\footnote{Кстати, для уменьшения генерации \q{лишних} макросов, можно использовать такой ключ GCC: \IT{-fno-asynchronous-unwind-tables}}:

\lstinputlisting[caption=GCC 4.7.3,style=customasmx86]{patterns/01_helloworld/GCC_refined.s}

\myindex{\ATTSyntax}
\myindex{\IntelSyntax}
Основные отличия синтаксиса Intel и AT\&T следующие:

\begin{itemize}

\item
Операнды записываются наоборот.

В Intel-синтаксисе: \\
<инструкция> <операнд назначения> <операнд-источник>.

В AT\&T-синтаксисе: \\
<инструкция> <операнд-источник> <операнд назначения>.

\myindex{\CStandardLibrary!memcpy()}
\myindex{\CStandardLibrary!strcpy()}
Чтобы легче понимать разницу, можно запомнить следующее:
когда вы работаете с синтаксисом Intel~--- можете в уме ставить знак равенства ($=$) между операндами,
а когда с синтаксисом AT\&T~--- мысленно ставьте стрелку направо ($\rightarrow$)
\footnote{Кстати, в некоторых стандартных функциях библиотеки Си (например, memcpy(), strcpy()) также применяется 
расстановка аргументов как в синтаксисе Intel: вначале указатель в памяти на блок назначения, 
затем указатель на блок-источник.}.

\item
AT\&T: Перед именами регистров ставится символ процента (\%), а перед числами символ доллара (\$).
Вместо квадратных скобок используются круглые.

\item
AT\&T: К каждой инструкции добавляется специальный символ, определяющий тип данных:

\begin{itemize}
\item q --- quad (64 бита)
\item l --- long (32 бита)
\item w --- word (16 бит)
\item b --- byte (8 бит)
\end{itemize}

% TODO1 simple example may be? \RU{Например mov\textbf{l}, movb, movw представляют различые версии инсструкция mov} \EN {For example: movl, movb, movw are variations of the mov instruciton}

\end{itemize}

Возвращаясь к результату компиляции: он идентичен тому, который мы посмотрели в \IDA.
Одна мелочь: \TT{0FFFFFFF0h} записывается как \TT{\$-16}.
Это то же самое: \TT{16} в десятичной системе это \TT{0x10} в шестнадцатеричной.
\TT{-0x10} будет как раз \TT{0xFFFFFFF0} (в рамках 32-битных чисел).

\myindex{x86!\Instructions!MOV}
Возвращаемый результат устанавливается в 0 обычной инструкцией \MOV, а не \XOR.
\MOV просто загружает значение в регистр.
Её название не очень удачное (данные не перемещаются, а копируются). В других архитектурах подобная инструкция обычно носит название \q{LOAD} или \q{STORE} или что-то в этом роде.

}
\NL{\subsubsection{GCC}

Nu zullen we dezelfde \CCpp code compileren in de GCC 4.4.1 compiler in Linux: \TT{gcc 1.c -o 1}.
Vervolgens, met de assistentie van de \IDA disassembler, zullen we kijken hoe de \main functie gemaakt is.
\IDA, maakt net als MSVC gebruik van de Intel-syntax\footnote{We hadden GCC ook assembly listings kunnen laten gereren in Intel-syntax door gebruik te maken van de opties \TT{-S -masm=intel}.}.

\begin{lstlisting}[caption=code in \IDA,style=customasmx86]
main            proc near

var_10          = dword ptr -10h

                push    ebp
                mov     ebp, esp
                and     esp, 0FFFFFFF0h
                sub     esp, 10h
                mov     eax, offset aHelloWorld ; "hello, world\n"
                mov     [esp+10h+var_10], eax
                call    _printf
                mov     eax, 0
                leave
                retn
main            endp
\end{lstlisting}

\myindex{Function prologue}
\myindex{x86!\Instructions!AND}
Het resultaat is bijna hetzelfde.
Het adres van de \TT{hello, world} string (opgeslagen in het data segment) wordt eerst ingeladen in het \EAX register en wordt daarna opgeslagen op de stack.
Daarbovenop vind je in de functie proloog hetvolgende terug: \TT{AND ESP, 0FFFFFFF0h}~---
deze instructie lijnt de \ESP registerwaarde uit op een 16-byte begrenzing.
Dit resulteert in het feit dat alle waarden op de stack op dezelfde manier uitgelijnd worden.
De CPU presteert beter als de waarden die hij moet behandelen gelokaliseerd zijn in het geheugen op adressen die gealigneerd zijn op een 4-byte of 16-byte begrenzing.\footnote{URLWPDA}.

\myindex{x86!\Instructions!SUB}
\INS{SUB ESP, 10h} reserveert 16 bytes op de stack. Zoals we hierna echter kunnen zijn, zijn er in dit geval slechts 4 nodig.

Dit komt doordat de grootte van de gereserveerde stack ook uitgelijnd is op een 16-byte begrenzing.

% TODO1: rewrite.
\myindex{x86!\Instructions!PUSH}
Het string adres (of een pointer naar de string) wordt dan rechtstreeks op de stack geplaatst zonder gebruik te maken van de \PUSH instructie.
\IT{var\_10}~---is een lokale variabele en is ook een argument voor \printf{}.
Lees er hieronder meer over.

\NLph{}

In tegenstelling tot MSVC, wanneer GCC compileert zonder optimizatie, maakt het gebruik van \TT{MOV EAX, 0} in plaats van kortere opcodes.

\myindex{x86!\Instructions!LEAVE}
De laatste instructie, \LEAVE~---is het equivalent van het \TT{MOV ESP, EBP} en \TT{POP EBP} instructiepaar.
Met andere woorden, deze instructie zet de \gls{stack pointer} (\ESP) terug, en herstelt het \EBP register
terug tot zijn oorspronkelijke staat.
Dit is nodig aangezien we deze registerwaarden hebben gewijzigd (\ESP en \EBP) in het begin van de functie (door het uitvoeren van \INS{MOV EBP, ESP} / \INS{AND ESP, \ldots}).

\subsubsection{GCC: \ATTSyntax}
\label{ATT_syntax}

Laat ons eens kijken hoe dit kan weergegeven worden in assembly in de AT\&T syntax.
Deze syntax is veel populairder in de UNIX-wereld.

\begin{lstlisting}[caption=\NLph{} GCC 4.7.3]
gcc -S 1_1.c
\end{lstlisting}

We krijgen dit resultaat:

\lstinputlisting[caption=GCC 4.7.3,style=customasmx86]{patterns/01_helloworld/GCC.s}

De lijst bevat vele macros (die beginnen met een punt). Maar deze zijn niet interessant voor ons momenteel.

Voorlopig, om het simpel te houden, kunnen we deze negeren (buiten de \IT{.string} macro, dewelke
een null-terminated karakter reeks encodeert net als een C-string). Daarna zien we dit
\footnote{Deze GCC optie kan gebruikt worden om alle \q{onnodige} macros te elimineren: \IT{-fno-asynchronous-unwind-tables}}:

\lstinputlisting[caption=GCC 4.7.3,style=customasmx86]{patterns/01_helloworld/GCC_refined.s}

\myindex{\ATTSyntax}
\myindex{\IntelSyntax}
Sommige grote verschillen tussen de Intel en AT\&T syntax zijn:

\begin{itemize}

\item
\NLph{}

In Intel-syntax: <instructie> <doel> <bron>.

In AT\&T syntax: <instructie> <bron> <doel>.

\myindex{\CStandardLibrary!memcpy()}
\myindex{\CStandardLibrary!strcpy()}
Een gemakkelijke manier om dit verschil te onthouden is: 
Wanneer je met Intel-syntax te doen krijgt, kan je je inbeelden dat er een gelijkheidsteken ($=$) staat tussen de operands
en met AT\&T-syntax beeld je je in dat er een pijl naar rechts staat ($\rightarrow$)
\footnote{Trouwens, in sommige C standaard functies (bv. memcpy(), strcpy()) worden
de argumenten opgelijst op dezelfde manier als in Intel-syntax: eerst een pointer naar het bestemmings geheugen block, 
gevolgd door een pointer naar de bron.}.

\item
AT\&T: Voor registernamen moet een percentteken geschreven worden (\%) en voor cijfers een dollarteken (\$).
Ronde haakjes worden gebruikt in plaats van haakjes.

\item
AT\&T: Een suffix wordt toegevoegd aan de instructies om de operand grootte te bepalen:

\begin{itemize}
\item q --- quad (64 bits)
\item l --- long (32 bits)
\item w --- word (16 bits)
\item b --- byte (8 bits)
\end{itemize}

\end{itemize}

Laten we even terugblikken op het gecompileerde resultaat: dit is identiek als wat we gezien hebben in \IDA.
Met een klein verschil: \TT{0FFFFFFF0h} wordt weergegeven als \TT{\$-16}.
Dit is hetzelfde: \TT{16} in het decimaalsysteem is \TT{0x10} in hexadecimal.
\TT{-0x10} is gelijk aan \TT{0xFFFFFFF0} (voor een 32-bit data type).

\myindex{x86!\Instructions!MOV}
Nog een ding: de return value wordt best op 0 gezet door gebruik te maken van \MOV, niet van \XOR.
\MOV laadt gewoon een waarde in het register.
De naam is een foute noemer (data wordt niet verplaatst, maar eerder gekopieerd). In andere architecturen wordt deze instructie \q{LOAD} of \q{STORE} of iets soortgelijks genoemd.

}
\ITA{\subsubsection{GCC}

Proviamo adesso a compilare lo stesso codice \CCpp con il compilatore GCC 4.4.1 su Linux: \TT{gcc 1.c -o 1}.
Successivamente, con l'aiuto del disassembler \IDA, vediamo come è stata creata la funzione \main .
\IDA, come MSVC, utilizza la sintassi Intel\footnote{Possiamo anche fare in modo che GCC produca un listato assembly con la sintassi Intel tramite l'opzione \TT{-S -masm=intel}.}.

\begin{lstlisting}[caption=codice in \IDA,style=customasmx86]
main            proc near

var_10          = dword ptr -10h

                push    ebp
                mov     ebp, esp
                and     esp, 0FFFFFFF0h
                sub     esp, 10h
                mov     eax, offset aHelloWorld ; "hello, world\n"
                mov     [esp+10h+var_10], eax
                call    _printf
                mov     eax, 0
                leave
                retn
main            endp
\end{lstlisting}

\myindex{Function prologue}
\myindex{x86!\Instructions!AND}
Il risultato è pressoché lo stesso.
L'indirizzo della stringa \TT{hello, world} (memorizzato nel data segment) è caricato prima nel registro \EAX e successivamente salvato sullo stack.
Inoltre, il prologo della funzione contiene \TT{AND ESP, 0FFFFFFF0h}~---questa 
istruzione allinea il valore del registro \ESP a 16-byte.
Ciò fa sì che tutti i valori sullo stack siano allineati allo stesso modo (la CPU è più efficiente se i valori che tratta sono collocati in memoria ad indirizzi allineati a, ovvero multipli di, 4 o 16 byte)\footnote{\URLWPDA}.

\myindex{x86!\Instructions!SUB}
\INS{SUB ESP, 10h} alloca 16 byte sullo stack. Tuttavia, come vedremo a breve, solo 4 sono necessari in questo caso.

Ciò è dovuto al fatto che la dimensione dello stack allocato è anch'essa allineata a 16 byte.

% TODO1: rewrite.
\myindex{x86!\Instructions!PUSH}
L'indirizzo della stringa (o un puntatore alla stringa) è quindi memorizzato direttamente sullo stack senza utilizzare l'istruzione \PUSH .
\IT{var\_10}~--- è una variabile locale ed è anche un argomento di \printf{}.
Maggiori dettagli in seguito.

Infine viene chiamata la funzione \printf.

Diversamente da MSVC, quando GCC compila senza ottimizzazione emette \TT{MOV EAX, 0} invece di un opcode più breve.

\myindex{x86!\Instructions!LEAVE}
L'ultima istruzione, \LEAVE~---è l'equivalente della coppia di istruzioni \TT{MOV ESP, EBP} e \TT{POP EBP} ~---in altre parole, questa istruzione riporta indietro lo \gls{stack pointer} (\ESP) e ripristina il registro \EBP al suo stato iniziale.
Ciò è necessario poiché abbiamo modificato i valori di questi registri (\ESP and \EBP) all'inizio della funzione ( eseguendo \INS{MOV EBP, ESP} / \INS{AND ESP, \ldots}).

\subsubsection{GCC: \ATTSyntax}
\label{ATT_syntax}

Vediamo come tutto questo può essere rappresentato nella sintassi assembly AT\&T.
Questa sintassi è molto più popolare nel mondo UNIX.

\begin{lstlisting}[caption=compiliamo in GCC 4.7.3]
gcc -S 1_1.c
\end{lstlisting}

Otteniamo questo:

\lstinputlisting[caption=GCC 4.7.3,style=customasmx86]{patterns/01_helloworld/GCC.s}

Il listato contiene molte macro (iniziano con il punto). Per il momento non ci interessano.

Per il momento, e solo per una questione di semplificazione, possiamo ignorarle (fatta eccezione per la macro \IT{.string} che codifica una sequenza di caratteri che termina con il null-byte (zero) proprio come una stringa C). Consideriamo soltanto questo
\footnote{Questa opzione di GCC può essere usata per eliminare le macro \q{superflue}: \IT{-fno-asynchronous-unwind-tables}}:

\lstinputlisting[caption=GCC 4.7.3,style=customasmx86]{patterns/01_helloworld/GCC_refined.s}

\myindex{\ATTSyntax}
\myindex{\IntelSyntax}
Alcune delle differenze maggiori tra la sintassi Intel e quella AT\&T sono:

\begin{itemize}

\item
\ITAph{}

Sintassi Intel: <istruzione> <operando di destinazione> <operando di origine>.

Sintassi AT\&T: <istruzione> <operando di origine> <operando di destinazione>.

\myindex{\CStandardLibrary!memcpy()}
\myindex{\CStandardLibrary!strcpy()}
Ecco un modo facile per memorizzare la differenza:
quando si tratta di sintassi Intel immagina che ci sia un segno di uguaglianza ($=$) tra i due operandi, quando si tratta di sintassi AT\&T immagina una freccia da sinistra a destra ($\rightarrow$)
\footnote{A proposito, in alcune funzioni standard C(es., memcpy(), strcpy()) gli argomenti sono elencati nello stesso modo della sintassi Intel: prima il puntatore al blocco di memoria di destinazione, e poi il puntatore al blocco di memoria di origine.}.

\item
AT\&T: Il simbolo di percentuale (\%) deve essere scritto prima del nome di un registro, e il dollaro (\$) prima dei numeri.

\item
AT\&T: All'istruzione si aggiunge un suffisso che definisce le dimensioni dell'operando:

\begin{itemize}
\item q --- quad (64 bit)
\item l --- long (32 bit)
\item w --- word (16 bit)
\item b --- byte (8 bit)
\end{itemize}

\end{itemize}

Torniamo al risultato compilato: è identico a quello che abbiamo visto in \IDA.
Con una piccola differenza: \TT{0FFFFFFF0h} è presentato come \TT{\$-16}.
E' la stessa cosa: \TT{16} nel sistema decimale è \TT{0x10} in esadecimale.
\TT{-0x10} è uguale a \TT{0xFFFFFFF0} (per un tipo di dato a 32-bit).

\myindex{x86!\Instructions!MOV}
Ancora una cosa: il valore di ritorno viene settato a 0 usando \MOV, non \XOR.
\MOV semplicemente carica un valore in un registro.
Il suo nome è fuorviante (il dato non viene spostato, bensì copiato). In altre architectures questa istruzione è chiamata \q{LOAD} o \q{STORE} o qualcosa di simile.

}
\DE{\subsubsection{GCC}

Als nächstes wird der gleiche \CCpp-Code mit GCC 4.4.1 unter Linux kompiliert: \TT{gcc 1.c -o 1}.
Mithilfe des \IDA-Disassemblers wird untersucht, wie die \main-Funktion erzeugt wurde.
\IDA nutzt, genau wie MSVX den Intel-Syntax\footnote{GCC kann Assembler-Ausgaben im Intel-Syntax erzeugen mit der Options \TT{-S -masm=intel}.}.

\begin{lstlisting}[caption=Code in \IDA,style=customasmx86]
main            proc near

var_10          = dword ptr -10h

                push    ebp
                mov     ebp, esp
                and     esp, 0FFFFFFF0h
                sub     esp, 10h
                mov     eax, offset aHelloWorld ; "hello, world\n"
                mov     [esp+10h+var_10], eax
                call    _printf
                mov     eax, 0
                leave
                retn
main            endp
\end{lstlisting}

\myindex{Function prologue}
\myindex{x86!\Instructions!AND}
Das Ergebnis ist fast das gleiche.
Die Adresse der \TT{hello, world}-Zeichenkette (im Daten-Segment) wird zunächst in das \EAX-Register geladen und anschließend auf dem Stack gesichert.\\
Zusätzlich beinhaltet der Funktions-Prolog \INS{AND ESP, 0FFFFFFF0h}~---diese
Anweisung richtet den \ESP-Register-Wert an eine 16-Byte-Grenze aus.
Dies führt dazu, dass alle Werte im Stack auf die gleiche Weise ausgerichtet sind.
Die CPU kann Anweisungen schneller ausführen, wenn die zu verarbeitenden Daten auf einer an 4- oder 16-Byte-Grenzen ausgerichteten Adresse liegen\footnote{\URLWPDA}.

\myindex{x86!\Instructions!SUB}
\INS{SUB ESP, 10h} reserviert 16 Byte auf dem Stack, auch wenn - wie später gezeigt wird - nur 4 Byte benötigt werden.

Der Grund liegt darin, dass auch die Größe des Stacks an eine 16-Byte-Grenze ausgerichtet ist.

% TODO1: rewrite.
\myindex{x86!\Instructions!PUSH}
Die Adresse der Zeichenkette (oder ein Zeiger darauf) wird anschließend direkt ohne die \PUSH-Anweisung auf dem Stack gespeichert.
IT{var\_10}~---ist eine lokale Variable und ein Argument für \printf{}.
Mehr dazu später.

Anschließend wird die \printf-Funktion aufgerufen.

Anders als MSVC erzeugt GCC ohne Optimierung Die Anweisung \TT{MOV EAX, 0} anstatt des kürzeren OpCodes.

\myindex{x86!\Instructions!LEAVE}
Die letzte Anweisung \LEAVE ist ein Äquivalent zu der Kombination aus \TT{MOV ESP, EBP} und \TT{POP EBP}.
Mit anderen Worten: diese Anweisung setzt den \gls{stack pointer} (\ESP) zurück und stellt die initalen Werte des \EBP-Registers wieder her.
Dies ist notwendig weil die Registerwerte (\ESP und \EBP) zu Beginn der Funktion (durch \INS{MOV EBP, ESP} / \INS{AND ESP, \ldots}).

\subsubsection{GCC: \ATTSyntax}
\label{ATT_syntax}

Im nächsten Beispiel ist sichtbar, wie dies im AT\%T-Syntax dargestellt werden kann.
Dieser Syntax ist sehr viel populärer in der UNIX-Welt.

\begin{lstlisting}[caption=Das Beispiel kompiliert mit GCC 4.7.3]
gcc -S 1_1.c
\end{lstlisting}

Das Ergebnis ist wie folgt:

\lstinputlisting[caption=GCC 4.7.3,style=customasmx86]{patterns/01_helloworld/GCC.s}

Der Quellcode beinhaltet Makros (beginnend mit einem Punkt), die hier aber nicht von Belang sind.

An dieser Stelle werden aus Gründen der Übersichtlichkeit alle Makros au0er \IT{.string}
ignoriert. Letzeres kodiert eine Null-terminierte Zeichenkette, die einem C-String entspricht.

Die resultierende Ausgabe ist diese
\footnote{Um die \q{unnötigen} Makros zu unterdrücen kann die GCC-Option \IT{-fno-asynchronous-unwind-tables} genutzt werden}:

\lstinputlisting[caption=GCC 4.7.3,style=customasmx86]{patterns/01_helloworld/GCC_refined.s}

\myindex{\ATTSyntax}
\myindex{\IntelSyntax}
Einige der Hauptunterschiede zwischen Intel und AT\&T-Syntax sin:

\begin{itemize}

\item
Quell- und Zieloperanden sind in umgekehrter Reihenfolge angegeben.

Im Intel-Syntax: <Anweisung> <Ziel-Operand> <Quell-Operand>.

Im AT\&T-Syntax: <Anweisung> <Quell-Operand> <Ziel-Operand>.

\myindex{\CStandardLibrary!memcpy()}
\myindex{\CStandardLibrary!strcpy()}
Hier ist eine einfache Möglichkeit um sich den Unterschied zu merken:
Beim Umgang mit dem Intel-Syntax, kann man sich ein Gleichheitszeichen ($=$) zwischen den Operanden vorstellen
und beim AT\&T-Syntax einen Pfeil nach rechts ($\rightarrow$)
\footnote{Einige C-Standard-Funktionen (z.B. memcpy(), strcpy()) sind die Parameter ebenfalls wie im
Intel-Syntax aufgelistet: erst der Zeiger zum Ziel, dann der Zeiger auf die Speicher-Quelle)}.

\item
AT\&T: Vor einem Register-Namen muss ein Prozentzeichen (\%) und vor Zahlen ein Dollarzeichen (\$) stehen.
Statt eckigen werden runde Klammern genutzt.

\item
AT\&T: An eine Anweisung ist ein Suffix angehängt, der die Operandengröße angibt:

\begin{itemize}
\item q --- quad (64 bits)
\item l --- long (32 bits)
\item w --- word (16 bits)
\item b --- byte (8 bits)
\end{itemize}

% TODO1 simple example may be? \RU{Например mov\textbf{l}, movb, movw представляют различые версии инсструкция mov} \EN {For example: movl, movb, movw are variations of the mov instruciton} \DE {Zum Beispiel sind movl, movb und movw Variationen der mov-Anweisung}

\end{itemize}

Nochmals zu dem kompilierten Ergebnis: Dieses ist identisch mit der Anzeige in \IDA,
jedoch mit einem kleinen Unterschied: \TT{0FFFFFFF0h} wird als \TT{\$-16} angezeigt.
Der eigentliche Wert ist der selbe: \TT{16} im Dezimalsystem ist \TT{0x10} im Hexadezimalsystem.
Für 32-Bit-Datentypen ist \TT{-0x10} identisch mit \TT{0xFFFFFFF0}.

\myindex{x86!\Instructions!MOV}
Eine weitere Sache: der Rückgabewert ist mittels \MOV auf Null gesetzt, nicht mit \XOR.
\MOV läd lediglich einen Wert in ein Register.
Der Name ist irreführend, da die Daten nicht verschoben, sondern kopiert werden.
In anderen Architekturen ist wird dieser Befehl \q{LOAD} oder \q{STORE} oder ähnlich genannt.
}


\subsection{x86-64}
\EN{\subsubsection{MSVC: x86-64}

\myindex{x86-64}
Let's also try 64-bit MSVC:

\lstinputlisting[caption=MSVC 2012 x64,style=customasmx86]{patterns/01_helloworld/MSVC_x64.asm}

\myindex{fastcall}

In x86-64, all registers were extended to 64-bit and now their names have an \TT{R-} prefix.
In order to use the stack less often (in other words, to access external memory/cache less often), there exists
a popular way to pass function arguments via registers (\IT{fastcall}) \myref{fastcall}.
I.e., a part of the function arguments is passed in registers, the rest---via the stack.
In Win64, 4 function arguments are passed in the \RCX, \RDX, \Reg{8}, \Reg{9} registers.
That is what we see here: a pointer to the string for \printf is now passed not in the stack, but in the \RCX register.
The pointers are 64-bit now, so they are passed in the 64-bit registers (which have the \TT{R-} prefix).
However, for backward compatibility, it is still possible to access the 32-bit parts, using the \TT{E-} prefix.
This is how the \RAX/\EAX/\AX/\AL register looks like in x86-64:

\RegTableOne{RAX}{EAX}{AX}{AH}{AL}

The \main function returns an \Tint{}-typed value, which is, in \CCpp, for better backward compatibility
and portability, still 32-bit, so that is why the \EAX register is cleared at the function end (i.e., the 32-bit
part of the register) instead of \RAX{}.
There are also 40 bytes allocated in the local stack.
This is called the \q{shadow space}, about which we are going to talk later: \myref{shadow_space}.
}
\FR{\subsubsection{MSVC: x86-64}

\myindex{x86-64}
Essayons MSVC 64-bit:

\lstinputlisting[caption=MSVC 2012 x64,style=customasmx86]{patterns/01_helloworld/MSVC_x64.asm}

\myindex{fastcall}

En x86-64, tous les registres ont été étendus à 64-bit et leurs noms ont maintenant le préfixe \TT{R-}.
Afin d'utiliser la pile moins souvent (en d'autres termes, pour accèder moins souvent à la mémoire externe/au cache),
il existe un moyen commun de passer les arguments aux fonctions par les registres (\IT{fastcall}) \myref{fastcall}.
I.e., une partie des arguments de la fonction est passée par les registres, le reste---par la pile.
En Win64, 4 arguments de fonction sont passés dans les registres \RCX, \RDX, \Reg{8}, \Reg{9}.
C'est ce que l'on voit ci-dessus: un pointeur sur la chaîne pour \printf est passé non pas par la pile,
mais par le registre \RCX.
Les pointeurs font maintenant 64-bit, ils sont donc passés dans les registres 64-bit (qui ont le préfixe \TT{R-}).
Toutefois, pour la rétrocompatibilité, il est toujours possible d'accèder à la partie 32-bits des registres,
en utilisant le prefixe \TT{E-}.
Voici à quoi ressemblent les registres \RAX/\EAX/\AX/\AL en x86-64:

\RegTableOne{RAX}{EAX}{AX}{AH}{AL}

La fonction \main renvoie un type \Tint{}, qui est, en \CCpp, pour une meilleure rétrocompatibilité
et portabilité, toujours 32-bit, c'est pourquoi le registre \EAX est mis à zéro à la fin de la fonction (i.e., la
partie 32-bit du registre) au lieu de \RAX{}.
Il y aussi 40 octets alloués sur la pile locale.
Cela est appelé le \q{shadow space}, dont nous parlerons plus tard: \myref{shadow_space}.
}
\ITA{\subsubsection{MSVC: x86-64}

\myindex{x86-64}
Proviamo anche con MSVC a 64-bit:

\lstinputlisting[caption=MSVC 2012 x64,style=customasmx86]{patterns/01_helloworld/MSVC_x64.asm}

\myindex{fastcall}

In x86-64, tutti i registri sono stati estesi a 64-bit ed il loro nome ha il prefisso \TT{R-}.
Per usare lo stack meno spesso (in altre parole, per accedere meno spesso alla memoria esterna/cache), esiste un metodo molto diffuso per passare gli argomenti delle funzioni tramite i registri (\IT{fastcall})
\myref{fastcall}.
Ovvero, una parte degli argomenti è passata attraverso i registri, il resto ---attraverso lo stack.
In Win64, 4 argpmenti di funzione sono passati nei registri \RCX, \RDX, \Reg{8}, \Reg{9}.
Questo è ciò che vediamo qui: un puntatore alla stringa per \printf è adesso passato nel registro \RCX anziché tramite lo stack.
I puntatori adesso sono a 64-bit , quindi sono passati nei registri a 64-bit (aventi il prefisso \TT{R-}).
E' comunque possibile, per retrocompatibilità, accedere alle parti a 32-bit parts, usando il prefisso \TT{E-}.
I registri \RAX/\EAX/\AX/\AL in x86-64 appaiono così:

\RegTableOne{RAX}{EAX}{AX}{AH}{AL}

La funzione \main restituisce un valore di tipo \Tint{}, che in \CCpp, per migliore retrocompatibilità e portabilità, resta ancora a 32-bit, motivo per cui il registro \EAX viene svuotato invece di \RAX{} alla fine della funzione (i.e., la parte a 32-bit
del registro).
Ci sono anche 40 byte allocati nello stack locale.
Questo spazio è detto \q{shadow space}, di cui parleremo più avanti: \myref{shadow_space}.

}
\NL{\subsubsection{MSVC: x86-64}

\myindex{x86-64}
Laat ons ook eens kijken naar 64-bit MSVC:

\lstinputlisting[caption=MSVC 2012 x64,style=customasmx86]{patterns/01_helloworld/MSVC_x64.asm}

\myindex{fastcall}

In x86-64 zijn alle registers uitgebreid tot 64-bit, en hebben hun namen een \TT{R-} prefix gekregen.
Om de stack minder te gebruiken (met andere woorden, om het externe geheugen/cache minder vaak te benaderen), bestaat
er een populaire manier om functies parameters door te geven via registers (\IT{fastcall}) \myref{fastcall}.
Bv., een deel van de parameters wordt doorgegeven via het register, de rest --- via de stack.
In Win64, worden 4 functie parameters doorgegeven via de \RCX, \RDX, \Reg{8}, \Reg{9} registers.
Dat is wat we hier zien: een pointer naar de string voor \printf wordt doorgegeven, niet via de stack, maar via het \RCX register.
De pointers zijn 64-bit nu, dus worden ze doorgegeven in de 64-bit registers (dewelke de \TT{R-} prefix hebben).
Voor backward compatibility is het echter nog steeds mogelijk om de 32-bit gedeelten aan te spreken, door gebruik te maken van de \TT{E-} prefix.
Dit is hoe de \RAX/\EAX/\AX/\AL registers eruit zien in x86-64:

\RegTableOne{RAX}{EAX}{AX}{AH}{AL}

De \main functie geeft een \Tint{}-typed waarde terug, hetwelk, in \CCpp, voor betere backward compatibiliteit
en portabiliteit, nog steeds 32-bit is. Daarom wordt het \EAX register ook leeggemaakt bij het einde van de functie
(het 32-bit gedeelte van het register) in plaats van \RAX{}.
Er zijn ook 40 bytes gealloceerd op de lokale stack.
Dit wordt de \q{shadow space} genoemd, waarover we het later nog gaan hebben: \myref{shadow_space}.

}
\RU{\subsubsection{MSVC: x86-64}

\myindex{x86-64}
Попробуем также 64-битный MSVC:

\lstinputlisting[caption=MSVC 2012 x64,style=customasmx86]{patterns/01_helloworld/MSVC_x64.asm}

\myindex{fastcall}

В x86-64 все регистры были расширены до 64-х бит и теперь имеют префикс \TT{R-}.
Чтобы поменьше задействовать стек (иными словами, поменьше обращаться кэшу и внешней памяти), уже давно имелся
довольно популярный метод передачи аргументов функции через регистры (\IT{fastcall}) \myref{fastcall}.
Т.е. часть аргументов функции передается через регистры и часть ---через стек.
В Win64 первые 4 аргумента функции передаются через регистры \RCX, \RDX, \Reg{8}, \Reg{9}.
Это мы здесь и видим: указатель на строку в \printf теперь передается не через стек, а через регистр \RCX.
Указатели теперь 64-битные, так что они передаются через 64-битные части регистров (имеющие префикс \TT{R-}).
Но для обратной совместимости можно обращаться и к нижним 32 битам регистров используя префикс \TT{E-}.
Вот как выглядит регистр \RAX/\EAX/\AX/\AL в x86-64:

\RegTableOne{RAX}{EAX}{AX}{AH}{AL}

Функция \main возвращает значение типа \Tint, который в \CCpp, надо полагать, для лучшей совместимости и переносимости,
оставили 32-битным. Вот почему в конце функции \main обнуляется не \RAX, а \EAX, т.е. 32-битная часть регистра.
Также видно, что 40 байт выделяются в локальном стеке.
Это \q{shadow space} которое мы будем рассматривать позже: \myref{shadow_space}.
}
\PTBR{\subsubsection{MSVC: x86-64}

\myindex{x86-64}
Vamos tentar também o MSVC 64-bits:

\lstinputlisting[caption=MSVC 2012 x64,style=customasmx86]{patterns/01_helloworld/MSVC_x64.asm}

\myindex{fastcall}

No x86-64, todos os registradores foram extendidos para 64-bits e agora seus nomes contém um \TT{R-} no prefixo.
A fim de diminuir a frequência com que a stack (pilha) é usada (em outras palavras, para acessar memória externa/cache menos vezes),
existe uma maneira popular de passar argumentos para funções através dos registradores (\IT{fastcall}) \myref{fastcall}.
Por exemplo, uma parte dos argumentos da função é passada nos registradores, o resto pela stack.
No Win64, 4 argumentos de funções são passados através dos registradores \RCX, \RDX, \Reg{8}, \Reg{9}.
Que é o que nós vemos, um ponteiro para a string para o printf() não é passado pela stack, mas no registrador \RCX.
Os ponteiros são 64-bits agora, então, eles são passados através dos registradores de 64-bits (que tem prefixo \TT{R-}).
Entretanto, para compatibilidade, ainda é possível acessar partes de 32-bits, usando o prefixo \TT{E-}.
É assim que os registradores \RAX/\EAX/\AX/\AL se parecem no x86-64:

\RegTableOne{RAX}{EAX}{AX}{AH}{AL}

A função \main retorna um valor do tipo inteiro, que em \CCpp é melhor para compatibilidade com versões anteriores e portabilidade,
de 32-bits, por isso o registrador \EAX é limpo no final da função (a parte de 32-bits do registrador) ao invés de \RAX.
Há também 40 bytes alocados na pilha local.
Que é chamado de ``shadow space'', o qual falaremos mais tarde: \myref{shadow_space}.

}
\DE{\subsubsection{MSVC: x86-64}

\myindex{x86-64}
Hier das gleiche Beispiel mit der 64-Bit-Variante von MSVC kompiliert:

\lstinputlisting[caption=MSVC 2012 x64,style=customasmx86]{patterns/01_helloworld/MSVC_x64.asm}

\myindex{fastcall}

In x86-64 wurden alle Regeister auf 64-Bit erweitert und die Registernamen mit einem \TT{R-}Prefix versehen.
Um den Stack weniger oft zu nutzen (also um auf externen Speicher / Cache selterner zuzugreifen), existiert
ein verbreiteter Weg um Funktionsargumente per Register (\IT{fastcall}) \myref{fastcall} zu übergeben.
Das heißt ein Teil der Funktionsargumente wird in Registern übergeben, der Rest---über den Stack.
In Win64 werden vier Funktionsargumente in den Registern \RCX, \RDX, \Reg{8} und \Reg{9} übergeben.
Das ist was hier sichtbar ist: der Zeiger zu der Zeichenkette für \printf ist jetzt nicht im Stack übergeben sondern im \RCX-Register.
Die Zeiger sind nun 64-Bit breit, also werden sie in den 64-Bit-Registern übergeben (die jetz den \TT{R-}Prefix haben).
Aus Gründen der Rückwärtskompatibilität ist es aber immer noch möglich mit dem \TT{E-}Prefix auf 32-Bit-Teile zuzugrifen.
Nachfolgend, der Aufbau der \RAX/\EAX/\AX/\AL-Register in x86-64:

\RegTableOne{RAX}{EAX}{AX}{AH}{AL}

Die \main-Funktion gibt einen Wert vom Typ \Tint{} zurück, der in \CCpp aus Gründen der Kompatibilität und
Portabilität immernoch 32 Bit breit ist. Daher wird am Ende der Funktion das \EAX-Register auf Null gesetzt
(das heißt der 32-Bit-Part des Registers) anstatt \RAX{}.
Auf dem lokalen Stack sind zusätzliche 40 Byte reserviert.
Dieser Bereich wird \q{shadow space} genannt und wird in Abschnitt \myref{shadow_space} noch genauer betrachtet.
}

\EN{\subsubsection{GCC: x86-64}

\myindex{x86-64}
Let's also try GCC in 64-bit Linux:

\lstinputlisting[caption=GCC 4.4.6 x64,style=customasmx86]{patterns/01_helloworld/GCC_x64_EN.s}

A method to pass function arguments in registers is also used in Linux, *BSD and \MacOSX is \SysVABI.

The first 6 arguments are passed in the \RDI, \RSI, \RDX, \RCX, \Reg{8}, \Reg{9}  registers, and the rest---via the stack.

So the pointer to the string is passed in \EDI (the 32-bit part of the register).
But why not use the 64-bit part, \RDI?

It is important to keep in mind that all \MOV instructions in 64-bit mode that write something into the lower 32-bit register part also clear the higher 32-bits (as stated in Intel manuals: \myref{x86_manuals}).\\
I.e., the \INS{MOV EAX, 011223344h} writes a value into \RAX correctly, since the higher bits will be cleared.

If we open the compiled object file (.o), we can also see all the instructions' opcodes
\footnote{This must be enabled in \textbf{Options $\rightarrow$ Disassembly $\rightarrow$ Number of opcode bytes}}:

\lstinputlisting[caption=GCC 4.4.6 x64,style=customasmx86]{patterns/01_helloworld/GCC_x64.lst}

\label{hw_EDI_instead_of_RDI}
As we can see, the instruction that writes into \EDI at \TT{0x4004D4} occupies 5 bytes.
The same instruction writing a 64-bit value into \RDI occupies 7 bytes.
Apparently, GCC is trying to save some space.
Besides, it can be sure that the data segment containing the string will not be allocated at the addresses higher than 4\gls{GiB}.

\label{SysVABI_input_EAX}
We also see that the \EAX register has been cleared before the \printf function call.
This is done because according to \ac{ABI} standard mentioned above,
the number of used vector registers is passed in \EAX in *NIX systems on x86-64.

}
\FR{\subsubsection{GCC: x86-64}

\myindex{x86-64}
Essayons GCC sur un Linux 64-bit:

\lstinputlisting[caption=GCC 4.4.6 x64,style=customasmx86]{patterns/01_helloworld/GCC_x64_FR.s}

Une méthode de passage des arguments à la fonction dans des registres est aussi utilisée sur Linux, *BSD et
\MacOSX est \SysVABI.

Les 6 premiers arguments sont passés dans les registres \RDI, \RSI, \RDX, \RCX, \Reg{8}, \Reg{9} et les autres---par
la pile.

Donc le pointeur sur la chaîne est passé dans \EDI (la partie 32-bit du registre).
Mais pourquoi ne pas utiliser la partie 64-bit, \RDI?

Il est important de garder à l'esprit que toutes les instructions \MOV en mode 64-bit qui écrivent quelque chose
dans la partie 32-bit inférieur du registre efface également les 32-bit supérieur (comme indiqué dans les manuels Intel:
\myref{x86_manuals}).\\
I.e., l'instruction \INS{MOV EAX, 011223344h} écrit correctement une valeur dans \RAX, puisque que les bits supérieurs
sont mis à zéro.

Si nous ouvrons le fichier objet compilé (.o), nous pouvons voir tous les opcodes des instructions
\footnote{Ceci doit être activé dans \textbf{Options $\rightarrow$ Disassembly $\rightarrow$ Number of opcode bytes}}:

\lstinputlisting[caption=GCC 4.4.6 x64,style=customasmx86]{patterns/01_helloworld/GCC_x64.lst}

\label{hw_EDI_instead_of_RDI}
Comme on le voit, l'instruction qui écrit dans \EDI en \TT{0x4004D4} occupe 5 octets.
La même instruction qui écrit une valeur sur 64-bit dans \RDI occupe 7 octets.
Il semble que GCC essaye d'économiser un peu d'espace.
En outre, cela permet d'être sûr que le segment de données contenant la chaîne ne sera pas alloué à une adresse supérieure
à 4 \gls{GiB}.

\label{SysVABI_input_EAX}
Nous voyons aussi que le registre \EAX est mis à zéro avant l'appel à la fonction \printf.
Ceci, car conformément à l' \ac{ABI} standard mentionnée plus haut,
le nombre de registres vectoriel utilisés est passé dans \EAX sur les systèmes *NIX en x86-64.

}
\RU{\subsubsection{GCC: x86-64}

\myindex{x86-64}
Попробуем GCC в 64-битном Linux:

\lstinputlisting[caption=GCC 4.4.6 x64,style=customasmx86]{patterns/01_helloworld/GCC_x64_RU.s}

В Linux, *BSD и \MacOSX для x86-64 также принят способ передачи аргументов функции через регистры \SysVABI.

6 первых аргументов передаются через регистры \RDI, \RSI, \RDX, \RCX, \Reg{8}, \Reg{9}, а остальные --- через стек.

Так что указатель на строку передается через \EDI (32-битную часть регистра).
Но почему не через 64-битную часть, \RDI?

Важно запомнить, что в 64-битном режиме все инструкции \MOV, записывающие что-либо в младшую 32-битную часть регистра, обнуляют старшие 32-бита (это можно найти в документации от Intel: \myref{x86_manuals}).
То есть, инструкция \INS{MOV EAX, 011223344h} корректно запишет это значение в \RAX, старшие биты сбросятся в ноль.

Если посмотреть в \IDA скомпилированный объектный файл (.o), увидим также опкоды всех инструкций
\footnote{Это нужно задать в \textbf{Options $\rightarrow$ Disassembly $\rightarrow$ Number of opcode bytes}}:

\lstinputlisting[caption=GCC 4.4.6 x64,style=customasmx86]{patterns/01_helloworld/GCC_x64.lst}

\label{hw_EDI_instead_of_RDI}
Как видно, инструкция, записывающая в \EDI по адресу \TT{0x4004D4}, занимает 5 байт.
Та же инструкция, записывающая 64-битное значение в \RDI, занимает 7 байт.
Возможно, GCC решил немного сэкономить.
К тому же, вероятно, он уверен, что сегмент данных, где хранится строка, никогда не будет расположен в адресах выше 4\gls{GiB}.

\label{SysVABI_input_EAX}
Здесь мы также видим обнуление регистра \EAX перед вызовом \printf.
Это делается потому что по упомянутому выше стандарту передачи аргументов в *NIX для x86-64 в \EAX передается количество задействованных векторных регистров.

}
\NL{\subsubsection{GCC: x86-64}

\myindex{x86-64}
Laat ons ook eens kijken naar GCC in 64-bit Linux:

% TODO translate:
\lstinputlisting[caption=GCC 4.4.6 x64,style=customasmx86]{patterns/01_helloworld/GCC_x64_EN.s}

Een methode om functieargumenten door te geven in registers wordt ook gebruikt in Linux, *BSD en \MacOSX \SysVABI.

De eerste 6 argumenten worden doorgegeven in de \RDI, \RSI, \RDX, \RCX, \Reg{8}, \Reg{9} registers, en de rest --- via de stack.

De pointer naar de string wordt dus doorgegeven via \EDI (het 32-bit gedeelte van het register).
Maar waarom gebruikt men niet het 64-bit gedeelte, \RDI?

Het is belangrijk te onthouden dat alle \MOV instructies in 64-bit modus, die iets schrijven in het onderste 32-bit gedeelte van het register, ook het bovenste 32-bit gedeelte leegmaken.
\INS{MOV EAX, 011223344h} schrijft een waarde correct weg in \RAX, aangezien de bovenste bits zullen worden leeggemaakt.

Als we het gecompileerde object-bestand (.o) openen, kunnen we ook de opcodes zien van alle instructies
\footnote{Dit moet ook geactiveerd worden in \textbf{Options $\rightarrow$ Disassembly $\rightarrow$ Number of opcode bytes}}:

\lstinputlisting[caption=GCC 4.4.6 x64,style=customasmx86]{patterns/01_helloworld/GCC_x64.lst}

\label{hw_EDI_instead_of_RDI}
Zoals je kan zien, bezet de instructie die in \EDI schrijft op \TT{0x4004D4} 5 bytes.
Dezelfde instructie die een 64-bit waarde in \RDI schrijft, bezet 7 bytes.
Blijkbaar probeert GCC wat plaats te besparen.
Daarnaast kunnen we met zekerheid zeggen dat het data segment dat de string bevat, niet zal gealloceerd worden op de adressen hoger dan 4\gls{GiB}.

\label{SysVABI_input_EAX}
We zien ook dat het \EAX register leeggemaakt is voor de \printf functie call.
Dit wordt gedaan omdat het aantal gebruikte vector registers wordt doorgegeven in \EAX in *NIX systemen op x86-64.

}
\ITA{\subsubsection{GCC: x86-64}

\myindex{x86-64}
\ITAph{}:

% TODO: translate:
\lstinputlisting[caption=GCC 4.4.6 x64,,style=customasmx86]{patterns/01_helloworld/GCC_x64_EN.s}

Un metodo per passare argomenti di funzione nei registri usato anche in Linux, *BSD and \MacOSX è \SysVABI.

I primi 6 argomenti sono passati nei registri \RDI, \RSI, \RDX, \RCX, \Reg{8}, \Reg{9}  , ed il resto---tramite lo stack.

Quindi il puntatore alla stringa viene passato in \EDI (la parte a 32-bit del registro).
Ma perchè no nusare la parte a 64-bit \RDI?

E' importante ricordare che tutte le istruzioni \MOV in modalità 64-bit che scrivono qualcosa nella parte bassa a 32-bit di un registro, azzera anche la parte alta a 32-bits.
Ad esempio, \INS{MOV EAX, 011223344h} scrive un valore in \RAX correttamente, poichè i bit della parte alta saranno azzerati.

Se apriamo il file oggetto compilato (.o), possiamo anche vedere gli opcode di tutte le istruzioni
\footnote{Deve essere abilitato in \textbf{Options $\rightarrow$ Disassembly $\rightarrow$ Number of opcode bytes}}:

\lstinputlisting[caption=GCC 4.4.6 x64,style=customasmx86]{patterns/01_helloworld/GCC_x64.lst}

\label{hw_EDI_instead_of_RDI}
Come possiamo notare, l'istruzione che scrive dentro \EDI a \TT{0x4004D4} occupa 5 byte.
La stessa istruzione che scrive un valore a 64-bit dentro \RDI occupa 7 bytes.
Apparentemente, GCC sta cercando di risparmiare un po' di spazio.
Inoltre, può essere sicuro che il segmento dati contenente la stringa non sarà allocato ad indirizzi maggiori di 4\gls{GiB}.

\label{SysVABI_input_EAX}
Notiamo anche che il registro \EAX è stato azzerato prima della chiamata alla funzione \printf .
Ciò avviene perché il numbero dei registri vettore usati viene passato in \EAX nei sistemi *NIX x86-64.

}
\DE{\subsubsection{GCC: x86-64}

\myindex{x86-64}
Nachfolgend das Beispiel unter einem 64 Bit-Linux-System mit GCC kompoliert:

\lstinputlisting[caption=GCC 4.4.6 x64,style=customasmx86]{patterns/01_helloworld/GCC_x64_DE.s}

Eine Methode im Funktionsargumente in Registern zu übergeben, wird auch in Linux, *BSD und \MacOSX genutzt und heißt \SysVABI.

Die ersten sechs Argumente sind in den Registern \RDI, \RSI, \RDX, \RCX,\Reg{8} und \Reg{9} übergeben und der Rest---über den Stack.

Der Zeiger zu der Zeichenkette ist in \EDI (also, dem 32-Bit-Teil) gesichert.
Warum wird nicht der 64-Bit-Teil \RDI genutz?

Es ist wichtig sich zu vergegenwertigen, dass alle \MOV-Anweisungen im 64-Bit-Modus, die etwas in den niederwertigen 32-Bit-Teil eines Registers schreiben,
auch den höherwertigen 32-Bit-Teil des Registers löschen (siehe Intel-Handbücher: \myref{x86_manuals}).\\
Die Anweisung \INS{MOV EAX, 011223344h} schreibt also den richtigen Wert in \RAX, weil die höherwetigen Bits auf Null gesetzt werden.

In der Objekt-Datei (.o) eines Kompilats sind ebenfalls alles OpCodes der verwendeten Anweisungen zu sehen.
\footnote{Dies muss aktiviert werden: \textbf{Optionen $\rightarrow$ Disassembly $\rightarrow$ Number of opcode bytes}}:

\lstinputlisting[caption=GCC 4.4.6 x64,style=customasmx86]{patterns/01_helloworld/GCC_x64.lst}

\label{hw_EDI_instead_of_RDI}
Wie man sehen kann verändert die Anweisung zum Schreiben in \EDI an der Adresse \TT{0x4004D4} fünf Byte.
Dieselbe Anweisung die einen 64-Bit-Wert in \RDI schreibt, verändert 7 Bytes.
Offenstichtlich versucht GCC etwas Speicherplatz zu sparen.
Nebenbei ist es sicher, dass das Datensegment, welches die Zeichenkette entählt niemals an Adressen höher 4\gls{GiB} reserviert wird.

\label{SysVABI_input_EAX}
Es ist auch erkennbar, dass das \EAX-Register vor dem Aufruf von \printf zurückgesetzt wurde.
Dies geschieht, aufgrund der Konvention in der oben genannten \ac{ABI}, dass in *NIX-Systemen auf x86-64-Architektur
die Anzahl der genutzten Vektor-Register in \EAX übergeben wird.
}


\section{GCC\EMDASH{}\EN{one more thing}\RU{ещё кое-что}\PTBR{mais uma coisa}}
\label{use_parts_of_C_strings}

\RU{Тот факт, что \IT{анонимная} Си-строка имеет тип}\EN{The fact that an \IT{anonymous} C-string has}\PTBR{O fato de que uma \IT{anonimo} C-string tenha} 
\IT{const}\EN{ type}\PTBR{tipo} (\myref{string_is_const_char}), 
\RU{и тот факт, что выделенные в сегменте констант Си-строки гаратировано неизменяемые (immutable), 
ведет к интересному следствию}\EN{and
that C-strings allocated in constants segment are guaranteed to be immutable, has an interesting consequence}\PTBR{e a string alocada no segumento das contantes possuem garantias de não serem alteradas, isso tem uma consequencia interessante}:
\RU{компилятор может использовать определенную часть строки}\EN{the compiler may use a specific part of the string}\PTBR{o compilador pode usar um trecho especifico da string}.

\RU{Вот простой пример}\EN{Let's try this example}\PTBR{Vejamos este exemplo}:

\begin{lstlisting}
#include <stdio.h>

int f1()
{
	printf ("world\n");
}

int f2()
{
	printf ("hello world\n");
}

int main()
{
	f1();
	f2();
}
\end{lstlisting}

\RU{Среднестатистический компилятор с \CCpp (включая MSVC) выделит место для двух строк, 
но вот что делает GCC 4.8.1}%
\EN{Common \CCpp{}-compilers (including MSVC) allocate two strings, 
but let's see what GCC 4.8.1 does}\PTBR{Compiladores comuns baseados no \CCpp{} (incluindo MSVC) alocam duas strings, 
mas vejamos o que o GCC 4.8.1 faz}:

\begin{lstlisting}[caption=GCC 4.8.1 + \RU{листинг в }IDA\EN{ listing}]
f1              proc near

s               = dword ptr -1Ch

                sub     esp, 1Ch
                mov     [esp+1Ch+s], offset s ; "world\n"
                call    _puts
                add     esp, 1Ch
                retn
f1              endp

f2              proc near

s               = dword ptr -1Ch

                sub     esp, 1Ch
                mov     [esp+1Ch+s], offset aHello ; "hello "
                call    _puts
                add     esp, 1Ch
                retn
f2              endp

aHello          db 'hello '
s               db 'world',0xa,0
\end{lstlisting}

\RU{Действительно, когда мы выводим строку}\EN{Indeed: when we print the \q{hello world} string}\PTBR{Fato: quando imprimimos a string \q{hello world}, 
\RU{эти два слова расположены в памяти впритык друг к другу и \puts, вызываясь из функции f2(), вообще не знает,
что эти строки разделены}\EN{these two words are positioned in memory adjacently and \puts called from f2() 
function is not aware that this string is divided}\PTBR{as duas palavras são posisionadas na mem\'oria em posições adjacentes e \puts \'e chamado da funcao f2() não sabe que a string esta dividida}. \RU{Они и не разделены на самом деле, они разделены
только \q{виртуально}, в нашем листинге}\EN{In fact, it's not divided; it's divided only \q{virtually}, in this
listing}\PTBR{Na verdade, não est\'a dividida; est\'a separada apenas \q{virtualmente}, nesta listagem}.

\RU{Когда}\EN{When}\PTBR{Quando} \puts \RU{вызывается из f1(), он использует строку}\EN{is called from f1(), it uses the}\PTBR{e chamado de f1(), usa o} 
\q{world} \RU{плюс нулевой байт}\EN{string plus a zero byte}\PTBR{string mais um byte zerado}. \puts \RU{не знает, что там ещё есть какая-то строка
перед этой}\EN{is not aware that there is something before this string}\PTBR{e não sabe que h\'a mais alguma coisa antes da string}!

\RU{Этот трюк часто используется (по крайней мере в GCC) и может сэкономить немного памяти.}
\EN{This clever trick is often used by at least GCC and can save some memory.}
\EN{Este \'e um truque geralmente utilizado pelo GCC e pode economizar um pouco de memoria.}

\EN{\subsection{ARM}
\label{sec:hw_ARM}

\myindex{\idevices}
\myindex{Raspberry Pi}
\myindex{Xcode}
\myindex{LLVM}
\myindex{Keil}
For my experiments with ARM processors, several compilers were used:

\begin{itemize}
\item Popular in the embedded area: Keil Release 6/2013.

\item Apple Xcode 4.6.3 IDE with the LLVM-GCC 4.2 compiler
\footnote{It is indeed so: Apple Xcode 4.6.3 uses open-source GCC as front-end compiler and LLVM 
code generator}.

\item GCC 4.9 (Linaro) (for ARM64), available as win32-executables at \url{http://go.yurichev.com/17325}.

\end{itemize}

32-bit ARM code is used (including Thumb and Thumb-2 modes) in all cases in this book, if not mentioned otherwise.
When we talk about 64-bit ARM here, we call it ARM64.

% subsections
\subsubsection{\NonOptimizingKeilVI (\ARMMode)}

Let's start by compiling our example in Keil:

\begin{lstlisting}
armcc.exe --arm --c90 -O0 1.c 
\end{lstlisting}

\myindex{\IntelSyntax}
The \IT{armcc} compiler produces assembly listings in Intel-syntax, but it has high-level ARM-processor related macros
\footnote{e.g. ARM mode lacks \PUSH/\POP instructions}, 
but it is more important for us to see the instructions \q{as is} so let's see the compiled result in \IDA.

\begin{lstlisting}[caption=\NonOptimizingKeilVI (\ARMMode) \IDA,style=customasmARM]
.text:00000000             main
.text:00000000 10 40 2D E9    STMFD   SP!, {R4,LR}
.text:00000004 1E 0E 8F E2    ADR     R0, aHelloWorld ; "hello, world"
.text:00000008 15 19 00 EB    BL      __2printf
.text:0000000C 00 00 A0 E3    MOV     R0, #0
.text:00000010 10 80 BD E8    LDMFD   SP!, {R4,PC}

.text:000001EC 68 65 6C 6C+aHelloWorld  DCB "hello, world",0    ; DATA XREF: main+4
\end{lstlisting}

In the example, we can easily see each instruction has a size of 4 bytes.
Indeed, we compiled our code for ARM mode, not for Thumb.

\myindex{ARM!\Instructions!STMFD}
\myindex{ARM!\Instructions!POP}
The very first instruction, \INS{STMFD SP!, \{R4,LR\}}\footnote{\ac{STMFD}}, 
works as an x86 \PUSH instruction, writing the values of two registers (\Reg{4} and \ac{LR}) into the stack.

Indeed, in the output listing from the \IT{armcc} compiler, for the sake of simplification, 
actually shows the \INS{PUSH \{r4,lr\}} instruction.
But that is not quite precise. The \PUSH instruction is only available in Thumb mode.
So, to make things less confusing, we're doing this in \IDA.

This instruction first \glspl{decrement} the \ac{SP} so it points to the place in the stack
that is free for new entries, then it saves the values of the \Reg{4} and \ac{LR} registers at the address
stored in the modified \ac{SP}.

This instruction (like the \PUSH instruction in Thumb mode) is able to save several register values at once which can be very useful.
By the way, this has no equivalent in x86.
It can also be noted that the \TT{STMFD} instruction is a generalization 
of the \PUSH instruction (extending its features), since it can work with any register, not just with \ac{SP}.
In other words, \TT{STMFD} may be used for storing a set of registers at the specified memory address.

\myindex{\PICcode}
\myindex{ARM!\Instructions!ADR}
The \INS{ADR R0, aHelloWorld}
instruction adds or subtracts the value in the \ac{PC} register to the offset where the \TT{hello, world} string is located.
How is the \TT{PC} register used here, one might ask?
This is called \q{\PICcode}\footnote{Read more about it in relevant section~(\myref{sec:PIC})}.

Such code can be executed at a non-fixed address in memory.
In other words, this is \ac{PC}-relative addressing.
The \INS{ADR} instruction takes into account the difference between the address of this instruction and the address where the string is located.
This difference (offset) is always to be the same, no matter at what address our code is loaded by the \ac{OS}.
That's why all we need is to add the address of the current instruction (from \ac{PC}) in order to get the absolute memory address of our C-string.

\myindex{ARM!\Registers!Link Register}
\myindex{ARM!\Instructions!BL}
\INS{BL \_\_2printf}\footnote{Branch with Link} instruction calls the \printf function. 
Here's how this instruction works: 

\begin{itemize}
\item store the address following the \INS{BL} instruction (\TT{0xC}) into the \ac{LR};
\item then pass the control to \printf by writing its address into the \ac{PC} register.
\end{itemize}

When \printf finishes its execution it must have information about where it needs to return the control to.
That's why each function passes control to the address stored in the \ac{LR} register.

That is a difference between \q{pure} \ac{RISC}-processors like ARM and \ac{CISC}-processors like x86,
where the return address is usually stored on the stack.
Read more about this in next section~(\myref{sec:stack}).

By the way, an absolute 32-bit address or offset cannot be encoded in the 32-bit \TT{BL} instruction because
it only has space for 24 bits.
As we may recall, all ARM-mode instructions have a size of 4 bytes (32 bits).
Hence, they can only be located on 4-byte boundary addresses.
This implies that the last 2 bits of the instruction address (which are always zero bits) may be omitted.
In summary, we have 26 bits for offset encoding. This is enough to encode $current\_PC \pm{} \approx{}32M$.

\myindex{ARM!\Instructions!MOV}
Next, the \INS{MOV R0, \#0}\footnote{Meaning MOVe} instruction just writes 0 into the \Reg{0} register.
That's because our C-function returns 0 and the return value is to be placed in the \Reg{0} register.

\myindex{ARM!\Registers!Link Register}
\myindex{ARM!\Instructions!LDMFD}
\myindex{ARM!\Instructions!POP}
The last instruction \INS{LDMFD SP!, {R4,PC}}\footnote{\ac{LDMFD} is an inverse instruction of \ac{STMFD}}.
It loads values from the stack (or any other memory place) in order to save them into \Reg{4} and \ac{PC}, and \glslink{increment}{increments} the \gls{stack pointer} \ac{SP}.
It works like \POP here.\\
N.B. The very first instruction \TT{STMFD} saved the \Reg{4} and \ac{LR} registers pair on the stack, but \Reg{4} and \ac{PC} are \IT{restored} during the \TT{LDMFD} execution.

As we already know, the address of the place where each function must return control to is usually saved in the \ac{LR} register.
The very first instruction saves its value in the stack because the same register will be used by our
\main function when calling \printf.
In the function's end, this value can be written directly to the \ac{PC} register, thus passing control to where our function has been called.

Since \main is usually the primary function in \CCpp,
the control will be returned to the \ac{OS} loader or to a point in a \ac{CRT},
or something like that.

All that allows omitting the \INS{BX LR} instruction at the end of the function.

\myindex{ARM!DCB}
\TT{DCB} is an assembly language directive defining an array of bytes or ASCII strings, akin to the DB directive 
in the x86-assembly language.


\subsubsection{\NonOptimizingKeilVI (\ThumbMode)}

Let's compile the same example using Keil in Thumb mode:

\begin{lstlisting}
armcc.exe --thumb --c90 -O0 1.c 
\end{lstlisting}

We are getting (in \IDA):

\begin{lstlisting}[caption=\NonOptimizingKeilVI (\ThumbMode) + \IDA,style=customasmARM]
.text:00000000             main
.text:00000000 10 B5          PUSH    {R4,LR}
.text:00000002 C0 A0          ADR     R0, aHelloWorld ; "hello, world"
.text:00000004 06 F0 2E F9    BL      __2printf
.text:00000008 00 20          MOVS    R0, #0
.text:0000000A 10 BD          POP     {R4,PC}

.text:00000304 68 65 6C 6C+aHelloWorld  DCB "hello, world",0    ; DATA XREF: main+2
\end{lstlisting}

We can easily spot the 2-byte (16-bit) opcodes. This is, as was already noted, Thumb.
\myindex{ARM!\Instructions!BL}
The \TT{BL} instruction, however, consists of two 16-bit instructions.
This is because it is impossible to load an offset for the \printf function while using the small space in one 16-bit opcode.
Therefore, the first 16-bit instruction loads the higher 10 bits of the offset and the second instruction loads 
the lower 11 bits of the offset.

% TODO:
% BL has space for 11 bits, so if we don't encode the lowest bit,
% then we should get 11 bits for the upper half, and 12 bits for the lower half.
% And the highest bit encodes the sign, so the destination has to be within
% \pm 4M of current_PC.
% This may be less if adding the lower half does not carry over,
% but I'm not sure --all my programs have 0 for the upper half,
% and don't carry over for the lower half.
% It would be interesting to check where __2printf is located relative to 0x8
% (I think the program counter is the next instruction on a multiple of 4
% for THUMB).
% The lower 11 bytes of the BL instructions and the even bit are
% 000 0000 0110 | 001 0010 1110 0 = 000 0000 0110 0010 0101 1100 = 0x00625c,
% so __2printf should be at 0x006264.
% But if we only have 10 and 11 bits, then the offset would be:
% 00 0000 0110 | 01 0010 1110 0 = 0 0000 0011 0010 0101 1100 = 0x00325c,
% so __2printf should be at 0x003264.
% In this case, though, the new program counter can only be 1M away,
% because of the highest bit is used for the sign.

As was noted, all instructions in Thumb mode have a size of 2 bytes (or 16 bits).
This implies it is impossible for a Thumb-instruction to be at an odd address whatsoever.
Given the above, the last address bit may be omitted while encoding instructions.

In summary, the \TT{BL} Thumb-instruction can encode an address in $current\_PC \pm{}\approx{}2M$.

\myindex{ARM!\Instructions!PUSH}
\myindex{ARM!\Instructions!POP}
As for the other instructions in the function: \PUSH and \POP work here just like the described \TT{STMFD}/\TT{LDMFD} only the \ac{SP} register is not mentioned explicitly here.
\TT{ADR} works just like in the previous example.
\TT{MOVS} writes 0 into the \Reg{0} register in order to return zero.


\subsubsection{\OptimizingXcodeIV (\ARMMode)}

Xcode 4.6.3 without optimization turned on produces a lot of redundant code so we'll study optimized output, where the instruction count is as small as possible, setting the compiler switch \Othree.

\begin{lstlisting}[caption=\OptimizingXcodeIV (\ARMMode),style=customasmARM]
__text:000028C4             _hello_world
__text:000028C4 80 40 2D E9   STMFD           SP!, {R7,LR}
__text:000028C8 86 06 01 E3   MOV             R0, #0x1686
__text:000028CC 0D 70 A0 E1   MOV             R7, SP
__text:000028D0 00 00 40 E3   MOVT            R0, #0
__text:000028D4 00 00 8F E0   ADD             R0, PC, R0
__text:000028D8 C3 05 00 EB   BL              _puts
__text:000028DC 00 00 A0 E3   MOV             R0, #0
__text:000028E0 80 80 BD E8   LDMFD           SP!, {R7,PC}

__cstring:00003F62 48 65 6C 6C+aHelloWorld_0  DCB "Hello world!",0
\end{lstlisting}

The instructions \TT{STMFD} and \TT{LDMFD} are already familiar to us.

\myindex{ARM!\Instructions!MOV}

The \MOV instruction just writes the number \TT{0x1686} into the \Reg{0} register.
This is the offset pointing to the \q{Hello world!} string.

The \TT{R7} register (as it is standardized in \IOSABI) is a frame pointer. More on that below.

\myindex{ARM!\Instructions!MOVT}
The \TT{MOVT R0, \#0} (MOVe Top) instruction writes 0 into higher 16 bits of the register.
The issue here is that the generic \MOV instruction in ARM mode may write only the lower 16 bits of the register.

Keep in mind, all instruction opcodes in ARM mode are limited in size to 32 bits. Of course, this limitation is not related to moving data between registers.
That's why an additional instruction \TT{MOVT} exists for writing into the higher bits (from 16 to 31 inclusive).
Its usage here, however, is redundant because the \TT{MOV R0, \#0x1686} instruction above cleared the higher part of the register.
This is supposedly a shortcoming of the compiler.
% TODO:
% I think, more specifically, the string is not put in the text section,
% ie. the compiler is actually not using position-independent code,
% as mentioned in the next paragraph.
% MOVT is used because the assembly code is generated before the relocation,
% so the location of the string is not yet known,
% and the high bits may still be needed.

\myindex{ARM!\Instructions!ADD}
The \TT{ADD R0, PC, R0} instruction adds the value in the \ac{PC} to the value in the \Reg{0}, to calculate the absolute address of the \q{Hello world!} string. 
As we already know, it is \q{\PICcode} so this correction is essential here.

The \INS{BL} instruction calls the \puts function instead of \printf.

\label{puts}
\myindex{\CStandardLibrary!puts()}
\myindex{puts() instead of printf()}

GCC replaced the first \printf call with \puts.
Indeed: \printf with a sole argument is almost analogous to \puts. 

\IT{Almost}, because the two functions are producing the same result only in case the 
string does not contain printf format identifiers starting with \IT{\%}. 
In case it does, the effect of these two functions would be different
\footnote{It has also to be noted the \puts does not require a `\textbackslash{}n' new line symbol 
at the end of a string, so we do not see it here.}.

Why did the compiler replace the \printf with \puts? Presumably because \puts is faster
\footnote{\href{http://go.yurichev.com/17063}{ciselant.de/projects/gcc\_printf/gcc\_printf.html}}. 

Because it just passes characters to \gls{stdout} without comparing every one of them with the \IT{\%} symbol.

Next, we see the familiar \TT{MOV R0, \#0} instruction intended to set the \Reg{0} register to 0.

\subsubsection{\OptimizingXcodeIV (\ThumbTwoMode)}

By default Xcode 4.6.3 generates code for Thumb-2 in this manner:

\begin{lstlisting}[caption=\OptimizingXcodeIV (\ThumbTwoMode),style=customasmARM]
__text:00002B6C                   _hello_world
__text:00002B6C 80 B5          PUSH            {R7,LR}
__text:00002B6E 41 F2 D8 30    MOVW            R0, #0x13D8
__text:00002B72 6F 46          MOV             R7, SP
__text:00002B74 C0 F2 00 00    MOVT.W          R0, #0
__text:00002B78 78 44          ADD             R0, PC
__text:00002B7A 01 F0 38 EA    BLX             _puts
__text:00002B7E 00 20          MOVS            R0, #0
__text:00002B80 80 BD          POP             {R7,PC}

...

__cstring:00003E70 48 65 6C 6C 6F 20+aHelloWorld  DCB "Hello world!",0xA,0
\end{lstlisting}

% Q: If you subtract 0x13D8 from 0x3E70,
% you actually get a location that is not in this function, or in _puts.
% How is PC-relative addressing done in THUMB2?
% A: it's not Thumb-related. there are just mess with two different segments. TODO: rework this listing.

\myindex{\ThumbTwoMode}
\myindex{ARM!\Instructions!BL}
\myindex{ARM!\Instructions!BLX}

The \TT{BL} and \TT{BLX} instructions in Thumb mode, as we recall, are encoded as a pair of 16-bit instructions.
In Thumb-2 these \IT{surrogate} opcodes are extended in such a way so that new instructions may be encoded here as 32-bit instructions.

That is obvious considering that the opcodes of the Thumb-2 instructions always begin with \TT{0xFx} or \TT{0xEx}.

But in the \IDA listing
the opcode bytes are swapped because for ARM processor the instructions are encoded as follows: 
last byte comes first and after that comes the first one (for Thumb and Thumb-2 modes) 
or for instructions in ARM mode the fourth byte comes first, then the third,
then the second and finally the first (due to different \gls{endianness}).

So that is how bytes are located in IDA listings:
\begin{itemize}
\item for ARM and ARM64 modes: 4-3-2-1;
\item for Thumb mode: 2-1;
\item for 16-bit instructions pair in Thumb-2 mode: 2-1-4-3.
\end{itemize}

\myindex{ARM!\Instructions!MOVW}
\myindex{ARM!\Instructions!MOVT.W}
\myindex{ARM!\Instructions!BLX}

So as we can see, the \TT{MOVW}, \TT{MOVT.W} and \TT{BLX} instructions begin with \TT{0xFx}.

One of the Thumb-2 instructions is \TT{MOVW R0, \#0x13D8} ~---it stores a 16-bit value into the lower part of the \Reg{0} register, clearing the higher bits.

Also, \TT{MOVT.W R0, \#0} ~works just like \TT{MOVT} from the previous example only it works in Thumb-2.

\myindex{ARM!mode switching}
\myindex{ARM!\Instructions!BLX}

Among the other differences, the \TT{BLX} instruction is used in this case instead of the \TT{BL}.

The difference is that, besides saving the \ac{RA} in the \ac{LR} register and passing control 
to the \puts function, the processor is also switching from Thumb/Thumb-2 mode to ARM mode (or back).

This instruction is placed here since the instruction to which control is passed looks like (it is encoded in ARM mode):

\begin{lstlisting}[style=customasmARM]
__symbolstub1:00003FEC _puts           ; CODE XREF: _hello_world+E
__symbolstub1:00003FEC 44 F0 9F E5     LDR  PC, =__imp__puts
\end{lstlisting}

This is essentially a jump to the place where the address of \puts is written in the imports' section.

So, the observant reader may ask: why not call \puts right at the point in the code where it is needed?

Because it is not very space-efficient.

\myindex{Dynamically loaded libraries}
Almost any program uses external dynamic libraries (like DLL in Windows, .so in *NIX or .dylib in \MacOSX).
The dynamic libraries contain frequently used library functions, including the standard C-function \puts.

\myindex{Relocation}
In an executable binary file (Windows PE .exe, ELF or Mach-O) an import section is present.
This is a list of symbols (functions or global variables) imported from external modules along with the names of the modules themselves.

The \ac{OS} loader loads all modules it needs and, while enumerating import symbols in the primary module, determines the correct addresses of each symbol.

In our case, \IT{\_\_imp\_\_puts} is a 32-bit variable used by the \ac{OS} loader to store the correct address of the function in an external library. 
Then the \TT{LDR} instruction just reads the 32-bit value from this variable and writes it into the \ac{PC} register, passing control to it.

So, in order to reduce the time the \ac{OS} loader needs for completing this procedure, 
it is good idea to write the address of each symbol only once, to a dedicated place.

\myindex{thunk-functions}
Besides, as we have already figured out, it is impossible to load a 32-bit value into a register 
while using only one instruction without a memory access.

Therefore, the optimal solution is to allocate a separate function working in ARM mode with the sole 
goal of passing control to the dynamic library and then to jump to this short one-instruction function (the so-called \gls{thunk function}) from the Thumb-code.

\myindex{ARM!\Instructions!BL}
By the way, in the previous example (compiled for ARM mode) the control is passed by the \TT{BL} to the 
same \gls{thunk function}.
The processor mode, however, is not being switched (hence the absence of an \q{X} in the instruction mnemonic).

\myparagraph{More about thunk-functions}
\myindex{thunk-functions}

Thunk-functions are hard to understand, apparently, because of a misnomer.
The simplest way to understand it as adaptors or convertors of one type of jack to another.
For example, an adaptor allowing the insertion of a British power plug into an American wall socket, or vice-versa. 
Thunk functions are also sometimes called \IT{wrappers}.

Here are a couple more descriptions of these functions:

\begin{framed}
\begin{quotation}
“A piece of coding which provides an address:”, according to P. Z. Ingerman, 
who invented thunks in 1961 as a way of binding actual parameters to their formal 
definitions in Algol-60 procedure calls. If a procedure is called with an expression 
in the place of a formal parameter, the compiler generates a thunk which computes 
the expression and leaves the address of the result in some standard location.

\dots

Microsoft and IBM have both defined, in their Intel-based systems, a “16-bit environment” 
(with bletcherous segment registers and 64K address limits) and a “32-bit environment” 
(with flat addressing and semi-real memory management). The two environments can both be 
running on the same computer and OS (thanks to what is called, in the Microsoft world, 
WOW which stands for Windows On Windows). MS and IBM have both decided that the process 
of getting from 16- to 32-bit and vice versa is called a “thunk”; for Windows 95, 
there is even a tool, THUNK.EXE, called a “thunk compiler”.
\end{quotation}
\end{framed}
% TODO FIXME move to bibliography and quote properly above the quote
( \href{http://go.yurichev.com/17362}{The Jargon File} )

\myindex{LAPACK}
\myindex{FORTRAN}
Another example we can find in LAPACK library---a ``Linear Algebra PACKage'' written in FORTRAN.
\CCpp developers also want to use LAPACK, but it's insane to rewrite it to \CCpp and then maintain several versions.
So there are short C functions callable from \CCpp environment, which are, in turn, call FORTRAN functions,
and do almost anything else:

\begin{lstlisting}[style=customc]
double Blas_Dot_Prod(const LaVectorDouble &dx, const LaVectorDouble &dy)
{
    assert(dx.size()==dy.size());
    integer n = dx.size();
    integer incx = dx.inc(), incy = dy.inc();

    return F77NAME(ddot)(&n, &dx(0), &incx, &dy(0), &incy);
}
\end{lstlisting}

Also, functions like that are called ``wrappers''.


\subsubsection{ARM64}

\myparagraph{GCC}

Let's compile the example using GCC 4.8.1 in ARM64:

\lstinputlisting[numbers=left,label=hw_ARM64_GCC,caption=\NonOptimizing GCC 4.8.1 + objdump,style=customasmARM]{patterns/01_helloworld/ARM/hw.lst}

There are no Thumb and Thumb-2 modes in ARM64, only ARM, so there are 32-bit instructions only.
The Register count is doubled: \myref{ARM64_GPRs}.
64-bit registers have \TT{X-} prefixes, while its 32-bit parts---\TT{W-}.

\myindex{ARM!\Instructions!STP}
The \TT{STP} instruction (\IT{Store Pair}) 
saves two registers in the stack simultaneously: \RegX{29} and \RegX{30}.

Of course, this instruction is able to save this pair at an arbitrary place in memory, 
but the \ac{SP} register is specified here, so the pair is saved in the stack.

ARM64 registers are 64-bit ones, each has a size of 8 bytes, so one needs 16 bytes for saving two registers.

The exclamation mark (``!'') after the operand means that 16 is to be subtracted from \ac{SP} first, and only then
are values from register pair to be written into the stack.
This is also called \IT{pre-index}.
About the difference between \IT{post-index} and \IT{pre-index} 
read here: \myref{ARM_postindex_vs_preindex}.

Hence, in terms of the more familiar x86, the first instruction is just an analogue to a pair of
\TT{PUSH X29} and \TT{PUSH X30}.
\RegX{29} is used as \ac{FP} in ARM64, and \RegX{30} 
as \ac{LR}, so that's why they are saved in the function prologue and restored in the function epilogue.

The second instruction copies \ac{SP} in \RegX{29} (or \ac{FP}).
This is made so to set up the function stack frame.

\label{pointers_ADRP_and_ADD}
\myindex{ARM!\Instructions!ADRP/ADD pair}
\TT{ADRP} and \ADD instructions are used to fill the 
address of the string \q{Hello!} into the \RegX{0} register, 
because the first function argument is passed
in this register.
There are no instructions, whatsoever, in ARM that can store a large number into a register (because the instruction
length is limited to 4 bytes, read more about it here: \myref{ARM_big_constants_loading}).
So several instructions must be utilized. The first instruction (\TT{ADRP}) writes the address of the 4KiB page, where the string is
located, into \RegX{0}, 
and the second one (\ADD) just adds the remainder to the address.
More about that in: \myref{ARM64_relocs}.

\TT{0x400000 + 0x648 = 0x400648}, and we see our \q{Hello!} C-string in the \TT{.rodata} data segment at this address.

\myindex{ARM!\Instructions!BL}

\puts is called afterwards using the \TT{BL} instruction. This was already discussed: \myref{puts}.

\MOV writes 0 into \RegW{0}. 
\RegW{0} is the lower 32 bits of the 64-bit \RegX{0} register:

\input{ARM_X0_register}

The function result is returned via \RegX{0} and \main returns 0, so that's how the return result is prepared.
But why use the 32-bit part?

Because the \Tint data type in ARM64, just like in x86-64, is still 32-bit, for better compatibility.

So if a function returns a 32-bit \Tint, only the lower 32 bits of \RegX{0} register have to be filled.

In order to verify this, let's change this example slightly and recompile it.
Now \main returns a 64-bit value:

\begin{lstlisting}[caption=\main returning a value of \TT{uint64\_t} type,style=customc]
#include <stdio.h>
#include <stdint.h>

uint64_t main()
{
        printf ("Hello!\n");
        return 0;
}
\end{lstlisting}

The result is the same, but that's how \MOV at that line looks like now:

\begin{lstlisting}[caption=\NonOptimizing GCC 4.8.1 + objdump]
  4005a4:       d2800000        mov     x0, #0x0      // #0
\end{lstlisting}

\myindex{ARM!\Instructions!LDP}

\INS{LDP} (\IT{Load Pair}) then restores the \RegX{29} and \RegX{30} registers.

There is no exclamation mark after the instruction: this implies that the values are first loaded from the stack,
and only then is \ac{SP} increased by 16.
This is called \IT{post-index}.

\myindex{ARM!\Instructions!RET}
A new instruction appeared in ARM64: \RET. 
It works just as \TT{BX LR}, only a special \IT{hint} bit is added, informing the \ac{CPU}
that this is a return from a function, not just another jump instruction, so it can execute it more optimally.

Due to the simplicity of the function, optimizing GCC generates the very same code.


}
\FR{\subsection{ARM}
\label{sec:hw_ARM}

\myindex{\idevices}
\myindex{Raspberry Pi}
\myindex{Xcode}
\myindex{LLVM}
\myindex{Keil}
Pour mes expérimentations avec les processeurs ARM, différents compilateurs ont été utilisés:

\begin{itemize}
\item Populaire dans le monde de l'embarqué: Keil Release 6/2013.

\item Apple Xcode 4.6.3 IDE avec le compilateur LLVM-GCC 4.2
\footnote{C'est ainsi: Apple Xcode 4.6.3 utilise les composants open-source GCC comme front-end et LLVM
comme générateur de code} % TODO clarify

\item GCC 4.9 (Linaro) (pour ARM64), disponible comme exécutable win32 ici \url{http://go.yurichev.com/17325}.

\end{itemize}

C'est du code ARM 32-bit qui est utilisé (également pour les modes Thumb et Thumb-2) dans tous les
cas dans ce livre, sauf mention contraire.

% subsections
\subsubsection{\NonOptimizingKeilVI (\ARMMode)}

Commençons par compiler notre exemple en Keil:

\begin{lstlisting}
armcc.exe --arm --c90 -O0 1.c 
\end{lstlisting}

\myindex{\IntelSyntax}
Le compilateur \IT{armcc} produit un listing assembleur en syntaxe Intel, mais il dispose de macros
de haut niveau liées au processeur ARM\footnote{e.g. les instructions \PUSH/\POP manquent en mode
ARM}, mais il est plus important pour nous de voir les instructions \q{telles quelles} donc
regardons le résultat compilé dans \IDA.

\begin{lstlisting}[caption=\NonOptimizingKeilVI (\ARMMode) \IDA,style=customasmARM]
.text:00000000             main
.text:00000000 10 40 2D E9    STMFD   SP!, {R4,LR}
.text:00000004 1E 0E 8F E2    ADR     R0, aHelloWorld ; "hello, world"
.text:00000008 15 19 00 EB    BL      __2printf
.text:0000000C 00 00 A0 E3    MOV     R0, #0
.text:00000010 10 80 BD E8    LDMFD   SP!, {R4,PC}

.text:000001EC 68 65 6C 6C+aHelloWorld  DCB "hello, world",0    ; DATA XREF: main+4
\end{lstlisting}

Dans l'exemple, nous voyons facilement que chaque instruction a une taille de 4 octets.
En effet, nous avons compilé notre code en mode ARM, pas pout Thumb.

\myindex{ARM!\Instructions!STMFD}
\myindex{ARM!\Instructions!POP}
La toute première instruction, \INS{STMFD SP!, \{R4,LR\}}\footnote{\ac{STMFD}},
fonctionne comme une instruction \PUSH en x86, écrivant la valeur de deux registres
(\Reg{4} et \ac{LR}) sur la pile.

En effet, dans le listing de la sortie du compilateur \IT{armcc}, dans un souci
de simplification, il montre l'instruction \INS{PUSH \{r4,lr\}}.
Mais ce n'est pas très précis. L'instruction \PUSH est seulement disponible dans
le mode Thumb.  Donc, pour rendre les choses moins confuses, nous faisons cela
dans \IDA.

Cette instruction décrémente (\glspl{decrement}) d'abord le pointeur de pile \ac{SP}
pour qu'il pointe sur de l'espace libre pour de nouvelles entrées, ensuite elle
sauve les valeurs des registres \Reg{4} et \ac{LR} à cette adresse.

Cette instruction (comme l'instruction \PUSH en mode Thumb) est capable de
sauvegarder plusieurs valeurs de registre à la fois, ce qui peut être très utile.
A propos, elle n'a pas d'équivalent en x86.
On peut noter que l'instruction \TT{STMFD} est une généralisation de l'instruction
\PUSH (étendant ses fonctionnalités), puisqu'elle peut travailler avec n'importe
quel registre, pas seulement avec \ac{SP}.
En d'autres mots, l'instruction \TT{STMFD} peut être utilisée pour stocker un
ensemble de registres à une adresse donnée.

\myindex{\PICcode}
\myindex{ARM!\Instructions!ADR}
L'instruction \INS{ADR R0, aHelloWorld}
ajoute ou soustrait la valeur dans le registre \ac{PC} à l'offset où la chaîne
\TT{hello, world} se trouve.
On peut se demander comment le registre \TT{PC} est utilisé ici ?
C'est appelé du \q{\PICcode}\footnote{Lire à ce propos la section(\myref{sec:PIC})}.

Un tel code peut être exécuté à n'importe quelle adresse en mémoire.
En d'autres mots, c'est un adressage \ac{PC}-relatif. %TODO relatif au \ac{PC} ?
L'instruction \INS{ADR} prend en compte la différence entre l'adresse de cette
instruction et l'adresse où est située la chaîne.
Cette différence (offset) est toujours la même, peu importe à quelle adresse
notre code est chargé par l'\ac{OS}.
C'est pourquoi tout ce dont nous avons besoin est d'ajouter l'adresse de l'instruction
courante (du \ac{PC}) pour obtenir l'adresse absolue en mémoire de notre chaîne C.

\myindex{ARM!\Registers!Link Register}
\myindex{ARM!\Instructions!BL}
L'instruction \INS{BL \_\_2printf}\footnote{Branch with Link} appelle la fonction \printf.
Voici comment fonctionne cette instruction:

\begin{itemize}
\item sauve l'adresse suivant l'instruction \INS{BL} (\TT{0xC}) dans \ac{LR};
\item puis passe le contrôle à \printf en écrivant son adresse dans le registre \ac{PC}.
\end{itemize}

Lorsque la fonction \printf termine son exécution elle doit avoir savoir où elle
doit redonner le contrôle.
C'est pourquoi chaque fonction passe le contrôle à l'adresse se trouvant dans le registre \ac{LR}.

C'est une différence entre un processeur \ac{RISC} \q{pure} comme ARM et un
processeur \ac{CISC} comme x86, où l'adresse de retour est en général sauvée
sur la pile.
En lire plus à ce propos dans la section~(\myref{sec:stack}) suivante.

A propos, une adresse absolue ou un offset de 32-bit ne peuvent être encodés
dans l'instruction 32-bit \TT{BL} car il n'y a qu'un espace de 24 bits.
Comme nous devons nous en souvenir, toutes les instructions ont une taille de
4 octets (32 bits).
Par conséquent, elles ne peuvent se trouver qu'à des adresses alignées dur des
limites de 4 octets.
Cela implique que les 2 derniers bits de l'adresse d'une instruction (qui sont
toujours des bits à zéro) peuvent être omis.
En résumé, nous avons 26 bits pour encoder l'offset. C'est assez pour encoder
$current\_PC \pm{} \approx{}32M$.

\myindex{ARM!\Instructions!MOV}
Ensuite, l'instruction \INS{MOV R0, \#0}\footnote{Signifiant MOVe} écrit juste
0 dans le registre \Reg{0}.
C'est parce que notre fonction C renvoie 0 et la valeur de retour doit être
mise dans le registre \Reg{0}.

\myindex{ARM!\Registers!Link Register}
\myindex{ARM!\Instructions!LDMFD}
\myindex{ARM!\Instructions!POP}
La dernière instruction \INS{LDMFD SP!, {R4,PC}}\footnote{\ac{LDMFD} est
l'instruction inverse de \ac{STMFD}}.
Elle prend des valeurs sur la pile (ou de toute autre endroit en mémoire)
afin de les sauver dans \Reg{4} et \ac{PC}, et \glslink{increment}{incrémente}
le pointeur de pile \gls{stack pointer} \ac{SP}.
Cela fonctionne ici comme \POP.\\
N.B. La toute première instruction \TT{STMFD} a sauvée la paire de registres
\Reg{4} et \ac{LR} sur la pile, mais \Reg{4} et \ac{PC} sont \IT{restaurés}
 pendant l'exécution de \TT{LDMFD}.

Comme nous le savons déjà, l'adresse où chaque fonction doit redonner le
contrôle est usuellement sauvée dans le registre \ac{LR}.
La toute première instruction sauve sa valeur sur la pile car le même
registre va être utilisé par notre fonction \main lors de l'appel à \printf.
A la fin de la fonction, cette valeur peut être écrite directement dans le
registre \ac{PC}, passant ainsi le contrôle là où notre fonction a été appelée.
Comme \main est en général la première fonction en \CCpp, le contrôle sera
redonné au chargeur de l'\ac{OS} ou a un point dans un \ac{CRT}, ou quelque
chose comme ça.

Tout cela permet d'omettre l'instruction \INS{BX LR} à la fin de la fonction.

\myindex{ARM!DCB}
\TT{DCB} est une directive du langage d'assemblage définissant un tableau d'octets
ou des chaînes ASCII, proche de la directive DB dans le langage d'assemblage x86.


\subsubsection{\NonOptimizingKeilVI (\ThumbMode)}

Compilons le même exemple en utilisant keil en mode Thumb:

\begin{lstlisting}
armcc.exe --thumb --c90 -O0 1.c 
\end{lstlisting}

Nous obtenons (dans \IDA):

\begin{lstlisting}[caption=\NonOptimizingKeilVI (\ThumbMode) + \IDA,style=customasmARM]
.text:00000000             main
.text:00000000 10 B5          PUSH    {R4,LR}
.text:00000002 C0 A0          ADR     R0, aHelloWorld ; "hello, world"
.text:00000004 06 F0 2E F9    BL      __2printf
.text:00000008 00 20          MOVS    R0, #0
.text:0000000A 10 BD          POP     {R4,PC}

.text:00000304 68 65 6C 6C+aHelloWorld  DCB "hello, world",0    ; DATA XREF: main+2
\end{lstlisting}

Nous pouvons repérer facilement les opcodes sur 2 octets (16-bit). C'est, comme déjà noté, Thumb.
\myindex{ARM!\Instructions!BL}
L'instruction \TT{BL}, toutefois, consiste en deux instructions 16-bit.
C'est parce qu'il est impossible de charger un offset pour la fonction \printf
en utilisant seulement le petit espace dans un opcode 16-bit.
Donc, la première instruction 16-bit charge les 10 bits supérieurs de l'offset
et la seconde instruction les 11 bits inférieurs de l'offset.

% TODO:
% BL has space for 11 bits, so if we don't encode the lowest bit,
% then we should get 11 bits for the upper half, and 12 bits for the lower half.
% And the highest bit encodes the sign, so the destination has to be within
% \pm 4M of current_PC.
% This may be less if adding the lower half does not carry over,
% but I'm not sure --all my programs have 0 for the upper half,
% and don't carry over for the lower half.
% It would be interesting to check where __2printf is located relative to 0x8
% (I think the program counter is the next instruction on a multiple of 4
% for THUMB).
% The lower 11 bytes of the BL instructions and the even bit are
% 000 0000 0110 | 001 0010 1110 0 = 000 0000 0110 0010 0101 1100 = 0x00625c,
% so __2printf should be at 0x006264.
% But if we only have 10 and 11 bits, then the offset would be:
% 00 0000 0110 | 01 0010 1110 0 = 0 0000 0011 0010 0101 1100 = 0x00325c,
% so __2printf should be at 0x003264.
% In this case, though, the new program counter can only be 1M away,
% because of the highest bit is used for the sign.

Comme il a été écrit, toutes les instructions en mode Thumb ont une taille de 2
octets (ou 16 bits).
Cela implique qu'il impossible pour une instruction Thumb d'être à une adresse
impaire, quelle qu'elle soit.
En tenant compte de cela, le dernier bit de l'adresse peut être omis lors de
l'encodage des instructions.

En résumé, l'instruction Thumb \TT{BL} peut encoder une adresse en $current\_PC \pm{}\approx{}2M$.

\myindex{ARM!\Instructions!PUSH}
\myindex{ARM!\Instructions!POP}
Comme pour les autres instructions dans la fonction: \PUSH et \POP fonctionnent
ici comme les instructions décrites \TT{STMFD}/\TT{LDMFD} seul le
registre \ac{SP} n'est pas mentionné explicitement ici.
\TT{ADR} fonctionne comme dans l'exemple précédent.
\TT{MOVS} écrit 0 dans le registre \Reg{0} afin de renvoyer zéro.


\subsubsection{\OptimizingXcodeIV (\ARMMode)}

Xcode 4.6.3 sans l'option d'optimisation produit beaucoup de code redondant c'est
pourquoi nous allons étudier le code généré avec optimisation, où le nombre
d'instruction est aussi petit que possible, en mettant l'option \Othree du
compilateur.

\begin{lstlisting}[caption=\OptimizingXcodeIV (\ARMMode),style=customasmARM]
__text:000028C4             _hello_world
__text:000028C4 80 40 2D E9   STMFD           SP!, {R7,LR}
__text:000028C8 86 06 01 E3   MOV             R0, #0x1686
__text:000028CC 0D 70 A0 E1   MOV             R7, SP
__text:000028D0 00 00 40 E3   MOVT            R0, #0
__text:000028D4 00 00 8F E0   ADD             R0, PC, R0
__text:000028D8 C3 05 00 EB   BL              _puts
__text:000028DC 00 00 A0 E3   MOV             R0, #0
__text:000028E0 80 80 BD E8   LDMFD           SP!, {R7,PC}

__cstring:00003F62 48 65 6C 6C+aHelloWorld_0  DCB "Hello world!",0
\end{lstlisting}

Les instructions \TT{STMFD} et \TT{LDMFD} nous sont déjà familières.

\myindex{ARM!\Instructions!MOV}

L'instruction \MOV écrit simplement le nombre \TT{0x1686} dans le registre \Reg{0}.
C'est l'offset pointant sur la chaîne \q{Hello world!}.

Le registre \TT{R7} (tel qu'il est standardisé dans \IOSABI) est un pointeur de frame. Voir plus loin.

\myindex{ARM!\Instructions!MOVT}
L'instruction \TT{MOVT R0, \#0} (MOVe Top) écrit 0 dans les 16 bits de poids
fort du registre.
Le problème ici est que l'instruction générique \MOV en mode ARM peut n'écrire
que dans les 16 bits de poids faible du registre.

Il faut garder à l'esprit que tout les opcodes d'instruction en mode ARM sont
limités en taille à 32 bits. Bien sûr, cette limitation n'est pas relative
au déplacement de données entre registres.
C'est pourquoi une instruction supplémentaire existe \TT{MOVT} pour écrire dans
les bits de la partie haute (de 16 à 31 inclus).
Son usage ici, toutefois, est redondant car l'instruction \TT{MOV R0, \#0x1686}
ci dessus a éffacé la partie haute du registre.
C'est soi-disant un défaut du compilateur.
% TODO:
% I think, more specifically, the string is not put in the text section,
% ie. the compiler is actually not using position-independent code,
% as mentioned in the next paragraph.
% MOVT is used because the assembly code is generated before the relocation,
% so the location of the string is not yet known,
% and the high bits may still be needed.

\myindex{ARM!\Instructions!ADD}
L'instruction \TT{ADD R0, PC, R0} ajoute la valeur dans \ac{PC} à celle de
\Reg{0}, pour calculer l'adresse absolue de la chaîne \q{Hello world!}.
Comme nous l'avons déjà vu, il s'agit de \q{\PICcode} donc la correction
est essentielle ici.

L'instruction \INS{BL} appelle la fonction \puts au lieu de \printf.

\label{puts}
\myindex{\CStandardLibrary!puts()}
\myindex{puts() instead of printf()}

GCC a remplacé le premier appel à \printf par un à \puts.
Effectivement: \printf avec un unique argument est presque analogue à \puts.

\IT{Presque}, car les deux fonctions produisent le même résultat uniquement dans
le cas où la chaîne ne contient pas d'identifiants de format débutant par \IT{\%}.
Dans le cas où elle en contient, l'effet de ces deux fonctions est différent.
\footnote{Il est à noter que \puts ne nécessite pas un `\textbackslash{}n'
symbole de retour à la ligne à la fin de la chaîne, donc nous ne le voyons pas ici.}.

Pourquoi est-ce que le compilateur a remplacé \printf par \puts? Probablement car
\puts est plus rapide.
\footnote{\href{http://go.yurichev.com/17063}{ciselant.de/projects/gcc\_printf/gcc\_printf.html}}. 

Car il envoie seulement les caractères dans \glslink{stdout}{sortie standard}
sans comparer chacun d'entre eux avec le symbole \IT{\%}.

Ensuite, nous voyons l'instruction familière \TT{MOV R0, \#0} pour mettre le
registre \Reg{0} à 0.

\subsubsection{\OptimizingXcodeIV (\ThumbTwoMode)}

Par défaut Xcode 4.6.3 génère du code pour Thumb-2 de cette manière:

\begin{lstlisting}[caption=\OptimizingXcodeIV (\ThumbTwoMode),style=customasmARM]
__text:00002B6C                   _hello_world
__text:00002B6C 80 B5          PUSH            {R7,LR}
__text:00002B6E 41 F2 D8 30    MOVW            R0, #0x13D8
__text:00002B72 6F 46          MOV             R7, SP
__text:00002B74 C0 F2 00 00    MOVT.W          R0, #0
__text:00002B78 78 44          ADD             R0, PC
__text:00002B7A 01 F0 38 EA    BLX             _puts
__text:00002B7E 00 20          MOVS            R0, #0
__text:00002B80 80 BD          POP             {R7,PC}

...

__cstring:00003E70 48 65 6C 6C 6F 20+aHelloWorld  DCB "Hello world!",0xA,0
\end{lstlisting}

% Q: If you subtract 0x13D8 from 0x3E70,
% you actually get a location that is not in this function, or in _puts.
% How is PC-relative addressing done in THUMB2?
% A: it's not Thumb-related. there are just mess with two different segments. TODO: rework this listing.

\myindex{\ThumbTwoMode}
\myindex{ARM!\Instructions!BL}
\myindex{ARM!\Instructions!BLX}

Les instructions \TT{BL} et \TT{BLX} en mode Thumb, comme on s'en souviens, sont
encodées comme une paire d'instructions 16 bits.
En Thumb-2 ces opcodes \IT{substituts} sont étendus de telle sorte que les nouvelles
instructions puissent être encodées comme des instructions 32-bit.

C'est évident en considérant que les opcodes des instructions Thumb-2 commencent
toujours avec \TT{0xFx} ou \TT{0xEx}.

Mais dans le listing d'\IDA les octets d'opcodes sont échangés car pour le processeur
ARM les intructions sont encodées comme ceci:
dernier octet en premier et ensuite le premier (pour les modes Thumb et Thumb-2)
ou pour les intructions en mode ARM le quatrième octet vient en premier, ensuite
le troisième, puis le second et enfin le premier (à cause des différents \gls{endianness}).

C'est ainsi que les octets se trouvent dans le listing d'IDA:
\begin{itemize}
\item pour les modes ARM et ARM64: 4-3-2-1;
\item pour le mode Thumb: 2-1;
\item pour les paires d'instructions 16-bit en mode Thumb-2: 2-1-4-3.
\end{itemize}

\myindex{ARM!\Instructions!MOVW}
\myindex{ARM!\Instructions!MOVT.W}
\myindex{ARM!\Instructions!BLX}

Donc, comme on peut le voir, les instructions \TT{MOVW}, \TT{MOVT.W} et \TT{BLX}
commencent par \TT{0xFx}.

Une des instructions Thumb-2 est \TT{MOVW R0, \#0x13D8} ~---elle stocke une valeur
16-bit dans la partie inférieure du registre \Reg{0}, éffaçant les bits supérieurs.

Aussi, \TT{MOVT.W R0, \#0} ~fonctionne comme \TT{MOVT} de l'exemple précédent
mais il fonctionne en Thumb-2.

\myindex{ARM!mode switching}
\myindex{ARM!\Instructions!BLX}

Parmi les autres différences, l'instruction \TT{BLX} est utilisée dans ce cas à
à la place de \TT{BL}.

La différence est que, en plus de sauver \ac{RA} dans le registre \ac{LR} et de
passer le contrôle à la fonction \puts, le processeur change du mode Thumb/Thumb-2
au mode ARM (ou inversement).

Cette instruction est placée ici, car l'instruction à laquelle est passée le contrôle
ressemble à (c'est encodé en mode ARM):

\begin{lstlisting}[style=customasmARM]
__symbolstub1:00003FEC _puts           ; CODE XREF: _hello_world+E
__symbolstub1:00003FEC 44 F0 9F E5     LDR  PC, =__imp__puts
\end{lstlisting}

Il s'agit principalement d'un saut à l'endroit où l'adresse de \puts est écrit
dans la section import.

Mais alors, le lecteur attentif pourrait demander: pourquoi ne pas appeler \puts
depuis l'endroit dans le code où on en a besoin ?

Parce que ce n'est pas très éfficace en terme d'espace.

\myindex{Dynamically loaded libraries}
Presque tous les programmes utilisent des bibliothèques dynamiques externes
(comme les DELL sous Windows, les .so sous *NIX ou les .dylib sous \MacOSX).
Les bibliothèques dynamiques contiennent les bibliothèques fréquemment utilisées,
incluant la fonction C standard \puts.

\myindex{Relocation}
Dans un fichier binaire exécutable (Windows PE .exe, ELF ou Mach-O) se trouve
une section d'import.
Il s'agit d'une liste des symboles (fonctions ou variables globales) importées
depuis des modules externes avec le nom des modules eux-même.

Le chargeur de l'\ac{OS} charge tous les modules dont il a besoin, tout en énumérant
les symboles d'import dans le module primaire, il détermine l'adresse correcte de
chaque symbole.

Dans notre cas, \IT{\_\_imp\_\_puts} est une variable 32-bit utilisée par le
chargeur de l'\ac{OS} pour sauver l'adresse correcte d'une fonction dans une
bibliothèque externe.
Ensuite l'instruction \TT{LDR} lit la valeur 32-bit depuis cette variable et
l'écrit dans le registre \ac{PC}, lui passant le contrôle.

Donc, pour réduire le temps dont le chargeur de l'\ac{OS} à besoin pour réaliser
cette procédure, c'est une bonne idée d'écrire l'adresse de chaque symbole une
seule fois, à une place dédiée.

\myindex{thunk-functions}

A côté de ça, comme nous l'avons déjà compris, il est impossible de charger
une valeur 32-bit dans un registre en utilisant seulement une instruction
sans un accès mémoire.

Donc, la solution optimale est d'allouer une fonction séparée fonctionnant en
mode ARM avec le seul but de passer le contrôle à l bibliothèque dynamique et
ensuite de sauter à cette petite fonction d'une instruction (ainsi appelée
\glslink{thunk function}{fonction thunk}) depuis le code Thumb.

\myindex{ARM!\Instructions!BL}
A propos, dans l'exemple précédent (compilé en mode ARM), le contrôle est
passé par \TT{BL} à la même \glslink{thunk function}{fonction thunk}.
Le mode du processeur, toutefois, n'est pas échangé (d'où l'absence d'un \q{X}
dans le mnémonique de l'instruction).

\myparagraph{Plus à propos des fonctions thunk}
\myindex{thunk-functions}

Les fonctions thunk sont difficile à comprendre, apparement, à cause d'un
mauvais nom.
La manière la plus simple est de les voir comme des adaptateurs ou des
convertisseurs d'un type jack à un autre.
Par exemple, un adaptateur permettant l'insertion d'un cordon électrique
britannique sur une prise murale américaine, ou vice-versa.
Les fonctions thunk sont parfois appelées \IT{wrappers}.

Voici quelques autres descriptions de ces fonctions:

\begin{framed}
\begin{quotation}
“Un morceau de code qui fournit une adresse:”, d'après P. Z. Ingerman, qui inventa
thunk en 1961 comme un moyen de lier les paramètres réels à leur définition
formelle dans les appels de procédures en Algol-60. Si une procédure est appelée
avec une expression à la place d'un paramètre formel, le compilateur génère un
thunk qui calcule l'expression et laisse l'adresse du résultat dans une place
standard.

\dots

Microsoft et IBM ont tous les deux défini, dans systèmes basés sur Intel, un
"environnement 16-bit" (avec leurs horribles registres de segment et la limite des
adresses à 64K) et un "environnement 32-bit" (avec un adressage linéaire et une
gestion semi-réelle de la mémoire). Les deux environnements peuvent fonctionner
sur le même ordinateur et OS (grâce à ce qui est appelé, dans le monde
Microsoft, WOW qui signifie Windows dans Windows). MS et IBM ont tous deux décidé
que le procédé de passer de 16-bit à 32-bit et vice-versa est appelé un "thunk";
pour Window 95, il y a même un outil, THUNK.EXE, appelé un "compilateur thunk".
\end{quotation}
\end{framed}
% TODO FIXME move to bibliography and quote properly above the quote
( \href{http://go.yurichev.com/17362}{The Jargon File} )

\myindex{LAPACK}
\myindex{FORTRAN}
Nous pouvons trouver un autre exemple dans la bibliothèque LAPCAK---un ``Linear Algebra PACKage''
écrit en FORTRAN.
Les développeurs \CCpp veulent aussi utiliser LAPACK, mais c'est un non-sens de
la récrire en \CCpp et de maintenir plusieurs versions.
Donc, il y a des petites fonctions que l'on peut invoquer depuis un environnement
\CCpp, qui font, à leur tour, des appels aux fonctions FORTRAN, et qui font
presque tout le reste:

\begin{lstlisting}[style=customc]
double Blas_Dot_Prod(const LaVectorDouble &dx, const LaVectorDouble &dy)
{
    assert(dx.size()==dy.size());
    integer n = dx.size();
    integer incx = dx.inc(), incy = dy.inc();

    return F77NAME(ddot)(&n, &dx(0), &incx, &dy(0), &incy);
}
\end{lstlisting}

Donc, ce genre de fonctions est appelé ``wrappers''.


\subsubsection{ARM64}

\myparagraph{GCC}

Compilons l'exemple en utilisant GCC 4.8.1 en ARM64:

\lstinputlisting[numbers=left,label=hw_ARM64_GCC,caption=GCC 4.8.1 \NonOptimizing + objdump,style=customasmARM]{patterns/01_helloworld/ARM/hw.lst}

Il n'y a pas de mode Thumb ou Thumb-2 en ARM64, seulement en ARM, donc il n'y a que des
instructions 32-bit.
Le nombre de registres a doublé: \myref{ARM64_GPRs}.
Les registres 64-bit ont le préfixe \TT{X-}, tandis que leurs partie 32-bit basse---\TT{W-}.

\myindex{ARM!\Instructions!STP}
L'instruction \TT{STP} (\IT{Store Pair} stocke une paire)
sauve deux registres sur la pile simultanément: \RegX{29} et \RegX{30}.

Bien sûr, cette instruction peut sauvegarder cette paire à n'importe quelle endroit en mémoire,
mais le registre \ac{SP} est spécifié ici, donc la paire est sauvé sur le pile.

Les registres ARM64 font 64-bit, chacun a une taille de 8 octets, donc il faut 16 octets pour sauver
deux registres.

Le point d'exclamation (``!'') après l'opérande signifie que 16 octets doivent d'abord être soustrait de \ac{SP},
et ensuite les valeurs de la paire de registres peuvent être écrites sur la pile.
Ceci est appelé le \IT{pre-index}.
A propos de la différence entre \IT{post-index} et \IT{pre-index}
lisez ceci: \myref{ARM_postindex_vs_preindex}.

Dans la gamme plus connue du x86, la première instruction est analogue à la paire
\TT{PUSH X29} et \TT{PUSH X30}.
En ARM64, \RegX{29} est utilisé comme \ac{FP} et \RegX{30} comme \ac{LR}, c'est pourquoi ils sont
sauvegardés dans le prologue de la fonction et remis dans l'épilogue.

La seconde instruction copie \ac{SP} dans \RegX{29} (ou \ac{FP}).
Cela sert à préparer la pile de la fonction.

\label{pointers_ADRP_and_ADD}
\myindex{ARM!\Instructions!ADRP/ADD pair}
Les instructions \TT{ADRP} et \ADD sont utilisées pour remplir l'adresse de
la chaîne \q{Hello!} dans le registre \RegX{0},
car le premier argument de la fonction est passé dans ce registre.
Il n'y a pas d'instruction, quelqu'elle soit, en ARM qui puisse stocker un nombre large
dans un registre (car la longueur des instructions est limitée à 4 octets, cf: \myref{ARM_big_constants_loading}).
Plusieurs instructions doivent donc être utilisées. La première instruction (\TT{ADRP}) écrit l'adresse de
la page de 4KiB, où se trouve la chaîne, dans \RegX{0}, et la seconde (\ADD) ajoute simplement
le reste de l'adresse.
Plus d'information ici: \myref{ARM64_relocs}.

\TT{0x400000 + 0x648 = 0x400648}, et nous voyons notre chaîne C \q{Hello!} dans le \TT{.rodata} segment
des données à cette adresse.

\myindex{ARM!\Instructions!BL}

\puts est appelée après en utilisant l'instruction \TT{BL}. Cela a déjà été discuté: \myref{puts}.

\MOV écrit 0 dans \RegW{0}.
\RegW{0} est la partie basse 32 bits du registre 64-bit \RegX{0}:

\input{ARM_X0_register}

Le résultat de la fonction est retourné via \RegX{0} et main renvoie 0, donc c'est ainsi que la valeur
de retour est préparée.
Mais pourquoi utiliser la partie 32-bit?

Parce que ls type de donnée \Tint en ARM64, tout comme en x86-64, est toujours 32-bit, pour une
meilleure compatibilité.

Donc si la fonction renvoie un \Tint 32-bit, seul les 32 premiers bits du registre \RegX{0} doivent
être rempli.

Pour vérifier ceci, changer un peu cet exemple et recompilons le.
Maintenant, \main renvoie une valeur sur 64-bit:

\begin{lstlisting}[caption=\main renvoie une valeur de type \TT{uint64\_t} type,style=customc]
#include <stdio.h>
#include <stdint.h>

uint64_t main()
{
        printf ("Hello!\n");
        return 0;
}
\end{lstlisting}

%%The result is the same, but that's how \MOV at that line looks like now:
Le résultat est le même, mais voilà à quoi ressemble \MOV à cette ligne maintenant:

\begin{lstlisting}[caption=GCC 4.8.1 \NonOptimizing + objdump]
  4005a4:       d2800000        mov     x0, #0x0      // #0
\end{lstlisting}

\myindex{ARM!\Instructions!LDP}

\INS{LDP} (\IT{Load Pair}) remet les registres \RegX{29} et \RegX{30}.

Il n'y a pas de point d'exclamation après l'instruction: celui signifie que les valeurs sont
d'abord chargées depuis la pile, et ensuite \ac{SP} est incrémenté de 16.
Cela est appelé \IT{post-index}.

\myindex{ARM!\Instructions!RET}
Une nouvelle instruction est apparue en ARM64: \RET.
Elle fonctionne comme \TT{BX LR}, un \IT{hint} bit particulier est ajouté, qui informe le \ac{CPU}
qu'il s'agit d'un retour de fonction, et pas d'une autre instruction de saut, et il peut l'exécuter
de manière plus optimale. 

A cause de la simplicité de la fonction, GCC avec l'option d'optimisation génère le même code.


}
\RU{\subsection{ARM}
\label{sec:hw_ARM}

\myindex{\idevices}
\myindex{Raspberry Pi}
\myindex{Xcode}
\myindex{LLVM}
\myindex{Keil}
Для экспериментов с процессором ARM было использовано несколько компиляторов:

\begin{itemize}
\item Популярный в embedded-среде Keil Release 6/2013.

\item Apple Xcode 4.6.3 с компилятором LLVM-GCC 4.2
\footnote{Это действительно так: Apple Xcode 4.6.3 использует опен-сорсный GCC как компилятор переднего плана и кодогенератор LLVM}.

\item GCC 4.9 (Linaro) (для ARM64), 
доступный в виде исполняемого файла для win32 на \url{http://go.yurichev.com/17325}.

\end{itemize}

Везде в этой книге, если не указано иное, идет речь о 32-битном ARM (включая режимы Thumb и Thumb-2).
Когда речь идет о 64-битном ARM, он называется здесь ARM64.

% subsections
\subsubsection{\NonOptimizingKeilVI (\ARMMode)}

Для начала скомпилируем наш пример в Keil:

\begin{lstlisting}
armcc.exe --arm --c90 -O0 1.c 
\end{lstlisting}

\myindex{\IntelSyntax}
Компилятор \IT{armcc} генерирует листинг на ассемблере в формате Intel.
Этот листинг содержит некоторые высокоуровневые макросы, связанные с ARM
\footnote{например, он показывает инструкции \PUSH/\POP, отсутствующие в режиме ARM},
а нам важнее увидеть инструкции \q{как есть}, так что посмотрим скомпилированный результат в \IDA.

\begin{lstlisting}[caption=\NonOptimizingKeilVI (\ARMMode) \IDA,style=customasmARM]
.text:00000000             main
.text:00000000 10 40 2D E9    STMFD   SP!, {R4,LR}
.text:00000004 1E 0E 8F E2    ADR     R0, aHelloWorld ; "hello, world"
.text:00000008 15 19 00 EB    BL      __2printf
.text:0000000C 00 00 A0 E3    MOV     R0, #0
.text:00000010 10 80 BD E8    LDMFD   SP!, {R4,PC}

.text:000001EC 68 65 6C 6C+aHelloWorld  DCB "hello, world",0    ; DATA XREF: main+4
\end{lstlisting}

В вышеприведённом примере можно легко увидеть, что каждая инструкция имеет размер 4 байта.
Действительно, ведь мы же компилировали наш код для режима ARM, а не Thumb.

\myindex{ARM!\Instructions!STMFD}
\myindex{ARM!\Instructions!POP}
Самая первая инструкция, \INS{STMFD SP!, \{R4,LR\}}\footnote{\ac{STMFD}},
работает как инструкция \PUSH в x86, записывая значения двух регистров (\Reg{4} и \ac{LR}) в стек.
Действительно, в выдаваемом листинге на ассемблере компилятор \IT{armcc} для упрощения указывает здесь инструкцию
\INS{PUSH \{r4,lr\}}.
Но это не совсем точно, инструкция \PUSH доступна только в режиме Thumb, поэтому,
во избежание путаницы, я предложил работать в \IDA.

Итак, эта инструкция уменьшает \ac{SP}, чтобы он указывал на место в стеке, свободное для записи
новых значений, затем записывает значения регистров \Reg{4} и \ac{LR} 
по адресу в памяти, на который указывает измененный регистр \ac{SP}.

Эта инструкция, как и инструкция \PUSH в режиме Thumb, может сохранить в стеке одновременно несколько значений регистров, что может быть очень удобно.
Кстати, такого в x86 нет.
Также следует заметить, что \TT{STMFD}~--- генерализация инструкции \PUSH (то есть расширяет её возможности), потому что может работать с любым регистром, а не только с \ac{SP}.
Другими словами, \TT{STMFD} можно использовать для записи набора регистров в указанном месте памяти.

\myindex{\PICcode}
\myindex{ARM!\Instructions!ADR}
Инструкция \INS{ADR R0, aHelloWorld} прибавляет или отнимает значение регистра \ac{PC} к смещению, где хранится строка
\TT{hello, world}.
Причем здесь \ac{PC}, можно спросить? Притом, что это так называемый \q{\PICcode}
\footnote{Читайте больше об этом в соответствующем разделе ~(\myref{sec:PIC})}.
Он предназначен для исполнения будучи не привязанным к каким-либо адресам в памяти.
Другими словами, это относительная от \ac{PC} адресация.
В опкоде инструкции \TT{ADR} указывается разница между адресом этой инструкции и местом, где хранится строка.
Эта разница всегда будет постоянной, вне зависимости от того, куда был загружен \ac{OS} наш код.
Поэтому всё, что нужно~--- это прибавить адрес текущей инструкции (из \ac{PC}), чтобы получить текущий абсолютный адрес нашей Си-строки.

\myindex{ARM!\Registers!Link Register}
\myindex{ARM!\Instructions!BL}
Инструкция \INS{BL \_\_2printf}\footnote{Branch with Link} вызывает функцию \printf.
Работа этой инструкции состоит из двух фаз:

\begin{itemize}
\item записать адрес после инструкции \INS{BL} (\TT{0xC}) в регистр \ac{LR};
\item передать управление в \printf, записав адрес этой функции в регистр \ac{PC}.
\end{itemize}

Ведь когда функция \printf закончит работу, нужно знать, куда вернуть управление, поэтому закончив работу, всякая функция передает управление по адресу, записанному в регистре \ac{LR}.

В этом разница между \q{чистыми} \ac{RISC}-процессорами вроде ARM и \ac{CISC}-процессорами как x86,
где адрес возврата обычно записывается в стек ~(\myref{sec:stack}).

Кстати, 32-битный абсолютный адрес (либо смещение) невозможно закодировать в 32-битной инструкции \INS{BL}, в ней есть место только для 24-х бит.
Поскольку все инструкции в режиме ARM имеют длину 4 байта (32 бита) и инструкции могут находится только по адресам кратным 4, то последние 2 бита (всегда нулевых) можно не кодировать.
В итоге имеем 26 бит, при помощи которых можно закодировать $current\_PC \pm{} \approx{}32M$.

\myindex{ARM!\Instructions!MOV}
Следующая инструкция \INS{MOV R0, \#0}\footnote{Означает MOVe}
просто записывает 0 в регистр \Reg{0}.
Ведь наша Си-функция возвращает 0, а возвращаемое значение всякая функция оставляет в \Reg{0}.

\myindex{ARM!\Registers!Link Register}
\myindex{ARM!\Instructions!LDMFD}
\myindex{ARM!\Instructions!POP}
Последняя инструкция \INS{LDMFD SP!, {R4,PC}}\footnote{\ac{LDMFD}~--- это инструкция, обратная \ac{STMFD}}.
Она загружает из стека (или любого другого места в памяти) значения для сохранения их в \Reg{4} и \ac{PC}, увеличивая \glslink{stack pointer}{указатель стека} \ac{SP}.
Здесь она работает как аналог \POP.\\
N.B. Самая первая инструкция \TT{STMFD} сохранила в стеке \Reg{4} и \ac{LR}, а \IT{восстанавливаются} во время исполнения \TT{LDMFD} регистры \Reg{4} и \ac{PC}.

Как мы уже знаем, в регистре \ac{LR} обычно сохраняется адрес места, куда нужно всякой функции вернуть управление.
Самая первая инструкция сохраняет это значение в стеке, потому что наша функция \main позже будет сама пользоваться этим регистром в момент вызова \printf.
А затем, в конце функции, это значение можно сразу записать прямо в \ac{PC}, таким образом, передав управление туда, откуда была вызвана наша функция.

Так как функция \main обычно самая главная в \CCpp, управление будет возвращено в загрузчик \ac{OS}, либо куда-то в \ac{CRT} 
или что-то в этом роде.

Всё это позволяет избавиться от инструкции \INS{BX LR} в самом конце функции.

\myindex{ARM!DCB}
\TT{DCB}~--- директива ассемблера, описывающая массивы байт или ASCII-строк, аналог директивы DB в x86-ассемблере.


\subsubsection{\NonOptimizingKeilVI (\ThumbMode)}

Скомпилируем тот же пример в Keil для режима Thumb:

\begin{lstlisting}
armcc.exe --thumb --c90 -O0 1.c 
\end{lstlisting}

Получим (в \IDA):

\begin{lstlisting}[caption=\NonOptimizingKeilVI (\ThumbMode) + \IDA,style=customasmARM]
.text:00000000             main
.text:00000000 10 B5          PUSH    {R4,LR}
.text:00000002 C0 A0          ADR     R0, aHelloWorld ; "hello, world"
.text:00000004 06 F0 2E F9    BL      __2printf
.text:00000008 00 20          MOVS    R0, #0
.text:0000000A 10 BD          POP     {R4,PC}

.text:00000304 68 65 6C 6C+aHelloWorld  DCB "hello, world",0    ; DATA XREF: main+2
\end{lstlisting}

Сразу бросаются в глаза двухбайтные (16-битные) опкоды --- это, как уже было отмечено, Thumb.

\myindex{ARM!\Instructions!BL}
Кроме инструкции \TT{BL}.
Но на самом деле она состоит из двух 16-битных инструкций.
Это потому что в одном 16-битном опкоде слишком мало места для задания смещения, по которому находится функция \printf.
Так что первая 16-битная инструкция загружает старшие 10 бит смещения, а вторая~--- младшие 11 бит смещения.

% TODO:
% BL has space for 11 bits, so if we don't encode the lowest bit,
% then we should get 11 bits for the upper half, and 12 bits for the lower half.
% And the highest bit encodes the sign, so the destination has to be within
% \pm 4M of current_PC.
% This may be less if adding the lower half does not carry over,
% but I'm not sure --all my programs have 0 for the upper half,
% and don't carry over for the lower half.
% It would be interesting to check where __2printf is located relative to 0x8
% (I think the program counter is the next instruction on a multiple of 4
% for THUMB).
% The lower 11 bytes of the BL instructions and the even bit are
% 000 0000 0110 | 001 0010 1110 0 = 000 0000 0110 0010 0101 1100 = 0x00625c,
% so __2printf should be at 0x006264.
% But if we only have 10 and 11 bits, then the offset would be:
% 00 0000 0110 | 01 0010 1110 0 = 0 0000 0011 0010 0101 1100 = 0x00325c,
% so __2printf should be at 0x003264.
% In this case, though, the new program counter can only be 1M away,
% because of the highest bit is used for the sign.

Как уже было упомянуто, все инструкции в Thumb-режиме имеют длину 2 байта (или 16 бит).
Поэтому невозможна такая ситуация, когда Thumb-инструкция начинается по нечетному адресу.

Учитывая сказанное, последний бит адреса можно не кодировать.
Таким образом, в Thumb-инструкции \TT{BL} можно закодировать адрес $current\_PC \pm{}\approx{}2M$.

\myindex{ARM!\Instructions!PUSH}
\myindex{ARM!\Instructions!POP}
Остальные инструкции в функции (\PUSH и \POP) здесь работают почти так же, как и описанные \TT{STMFD}/\TT{LDMFD}, только регистр \ac{SP} здесь не указывается явно.
\INS{ADR} работает так же, как и в предыдущем примере.
\INS{MOVS} записывает 0 в регистр \Reg{0} для возврата нуля.


\subsubsection{\OptimizingXcodeIV (\ARMMode)}

Xcode 4.6.3 без включенной оптимизации выдает слишком много лишнего кода, поэтому включим оптимизацию компилятора (ключ \Othree), потому что там меньше инструкций.

\begin{lstlisting}[caption=\OptimizingXcodeIV (\ARMMode),style=customasmARM]
__text:000028C4             _hello_world
__text:000028C4 80 40 2D E9   STMFD           SP!, {R7,LR}
__text:000028C8 86 06 01 E3   MOV             R0, #0x1686
__text:000028CC 0D 70 A0 E1   MOV             R7, SP
__text:000028D0 00 00 40 E3   MOVT            R0, #0
__text:000028D4 00 00 8F E0   ADD             R0, PC, R0
__text:000028D8 C3 05 00 EB   BL              _puts
__text:000028DC 00 00 A0 E3   MOV             R0, #0
__text:000028E0 80 80 BD E8   LDMFD           SP!, {R7,PC}

__cstring:00003F62 48 65 6C 6C+aHelloWorld_0  DCB "Hello world!",0
\end{lstlisting}

Инструкции \TT{STMFD} и \TT{LDMFD} нам уже знакомы.

\myindex{ARM!\Instructions!MOV}
Инструкция \MOV просто записывает число \TT{0x1686} в регистр \Reg{0}~--- это смещение, указывающее на строку \q{Hello world!}.

Регистр \Reg{7} (по стандарту, принятому в \IOSABI) это frame pointer, о нем будет рассказано позже.

\myindex{ARM!\Instructions!MOVT}
Инструкция \TT{MOVT R0, \#0} (MOVe Top) записывает 0 в старшие 16 бит регистра.
Дело в том, что обычная инструкция \MOV в режиме ARM может записывать какое-либо значение только в младшие 16 бит регистра, ведь в ней нельзя закодировать больше.
Помните, что в режиме ARM опкоды всех инструкций ограничены длиной в 32 бита. Конечно, это ограничение не касается перемещений данных между регистрами.

Поэтому для записи в старшие биты (с 16-го по 31-й включительно) существует дополнительная команда \INS{MOVT}.
Впрочем, здесь её использование избыточно, потому что инструкция \INS{MOV R0, \#0x1686} выше и так обнулила старшую часть регистра.
Возможно, это недочет компилятора.
% TODO:
% I think, more specifically, the string is not put in the text section,
% ie. the compiler is actually not using position-independent code,
% as mentioned in the next paragraph.
% MOVT is used because the assembly code is generated before the relocation,
% so the location of the string is not yet known,
% and the high bits may still be needed.

\myindex{ARM!\Instructions!ADD}
Инструкция \TT{ADD R0, PC, R0} прибавляет \ac{PC} к \Reg{0} для вычисления действительного адреса строки \q{Hello world!}. Как нам уже известно, это \q{\PICcode}, поэтому такая корректива необходима.

Инструкция \TT{BL} вызывает \puts вместо \printf.

\label{puts}
\myindex{\CStandardLibrary!puts()}
\myindex{puts() вместо printf()}
Компилятор заменил вызов \printf на \puts. 
Действительно, \printf с одним аргументом это почти аналог \puts.
 
\IT{Почти}, если принять условие, что в строке не будет управляющих символов \printf, 
начинающихся со знака процента. Тогда эффект от работы этих двух функций будет разным
\footnote{Также нужно заметить, что \puts не требует символа перевода строки `\textbackslash{}n' в конце строки,
поэтому его здесь нет.}.

Зачем компилятор заменил один вызов на другой? Наверное потому что \puts работает быстрее
\footnote{\href{http://go.yurichev.com/17063}{ciselant.de/projects/gcc\_printf/gcc\_printf.html}}. 
Видимо потому что \puts проталкивает символы в \gls{stdout} не сравнивая каждый со знаком процента.

Далее уже знакомая инструкция \TT{MOV R0, \#0}, служащая для установки в 0 возвращаемого значения функции.


\subsubsection{\OptimizingXcodeIV (\ThumbTwoMode)}

По умолчанию Xcode 4.6.3 генерирует код для режима Thumb-2 примерно в такой манере:

\begin{lstlisting}[caption=\OptimizingXcodeIV (\ThumbTwoMode),style=customasmARM]
__text:00002B6C                   _hello_world
__text:00002B6C 80 B5          PUSH            {R7,LR}
__text:00002B6E 41 F2 D8 30    MOVW            R0, #0x13D8
__text:00002B72 6F 46          MOV             R7, SP
__text:00002B74 C0 F2 00 00    MOVT.W          R0, #0
__text:00002B78 78 44          ADD             R0, PC
__text:00002B7A 01 F0 38 EA    BLX             _puts
__text:00002B7E 00 20          MOVS            R0, #0
__text:00002B80 80 BD          POP             {R7,PC}

...

__cstring:00003E70 48 65 6C 6C 6F 20+aHelloWorld  DCB "Hello world!",0xA,0
\end{lstlisting}

% Q: If you subtract 0x13D8 from 0x3E70,
% you actually get a location that is not in this function, or in _puts.
% How is PC-relative addressing done in THUMB2?
% A: it's not Thumb-related. there are just mess with two different segments. TODO: rework this listing.

\myindex{\ThumbTwoMode}
\myindex{ARM!\Instructions!BL}
\myindex{ARM!\Instructions!BLX}
Инструкции \TT{BL} и \TT{BLX} в Thumb, как мы помним, кодируются как пара 16-битных инструкций, 
а в Thumb-2 эти \IT{суррогатные} опкоды расширены так, что новые инструкции кодируются здесь как 
32-битные инструкции.
Это можно заметить по тому что опкоды Thumb-2 инструкций всегда начинаются с \TT{0xFx} либо с \TT{0xEx}.
Но в листинге \IDA байты опкода переставлены местами.
Это из-за того, что в процессоре ARM инструкции кодируются так:
в начале последний байт, потом первый (для Thumb и Thumb-2 режима), либо, 
(для инструкций в режиме ARM) в начале четвертый байт, затем третий, второй и первый 
(т.е. другой \gls{endianness}).

Вот так байты следуют в листингах IDA:

\begin{itemize}
\item для режимов ARM и ARM64: 4-3-2-1;
\item для режима Thumb: 2-1;
\item для пары 16-битных инструкций в режиме Thumb-2: 2-1-4-3.
\end{itemize}

\myindex{ARM!\Instructions!MOVW}
\myindex{ARM!\Instructions!MOVT.W}
\myindex{ARM!\Instructions!BLX}
Так что мы видим здесь что инструкции \TT{MOVW}, \TT{MOVT.W} и \TT{BLX} начинаются с \TT{0xFx}.

Одна из Thumb-2 инструкций это
\TT{MOVW R0, \#0x13D8}~--- она записывает 16-битное число в младшую часть регистра \Reg{0}, очищая старшие биты.

Ещё \TT{MOVT.W R0, \#0}~--- эта инструкция работает так же, как и \TT{MOVT} из предыдущего примера, но она работает в Thumb-2.

\myindex{ARM!переключение режимов}
\myindex{ARM!\Instructions!BLX}
Помимо прочих отличий, здесь используется инструкция \TT{BLX} вместо \TT{BL}.
Отличие в том, что помимо сохранения адреса возврата в регистре \ac{LR} и передаче управления 
в функцию \puts, происходит смена режима процессора с Thumb/Thumb-2 на режим ARM (либо назад).
Здесь это нужно потому, что инструкция, куда ведет переход, выглядит так (она закодирована в режиме ARM):

\begin{lstlisting}[style=customasmARM]
__symbolstub1:00003FEC _puts           ; CODE XREF: _hello_world+E
__symbolstub1:00003FEC 44 F0 9F E5     LDR  PC, =__imp__puts
\end{lstlisting}

Это просто переход на место, где записан адрес \puts в секции импортов.
Итак, внимательный читатель может задать справедливый вопрос: почему бы не вызывать \puts сразу в 
том же месте кода, где он нужен?
Но это не очень выгодно из-за экономии места и вот почему.

\myindex{Динамически подгружаемые библиотеки}
Практически любая программа использует внешние динамические библиотеки (будь то DLL в Windows, .so в *NIX 
либо .dylib в \MacOSX).
В динамических библиотеках находятся часто используемые библиотечные функции, в том числе стандартная функция Си \puts.

\myindex{Relocation}
В исполняемом бинарном файле 
(Windows PE .exe, ELF либо Mach-O) имеется секция импортов, список символов (функций либо глобальных переменных) импортируемых из внешних модулей, а также названия самих модулей.
Загрузчик \ac{OS} загружает необходимые модули и, перебирая импортируемые символы в основном модуле, проставляет правильные адреса каждого символа.
В нашем случае, \IT{\_\_imp\_\_puts} 
это 32-битная переменная, куда загрузчик \ac{OS} запишет правильный адрес этой же функции во внешней библиотеке. 
Так что инструкция \TT{LDR} просто берет 32-битное значение из этой переменной, и, записывая его в регистр \ac{PC}, просто передает туда управление.
Чтобы уменьшить время работы загрузчика \ac{OS}, нужно чтобы ему пришлось записать адрес каждого символа только один раз, в соответствующее, выделенное для них, место.

\myindex{thunk-функции}
К тому же, как мы уже убедились, нельзя одной инструкцией загрузить в регистр 32-битное число без обращений к памяти.
Так что наиболее оптимально выделить отдельную функцию, работающую в режиме ARM, 
чья единственная цель~--- передавать управление дальше, в динамическую библиотеку.
И затем ссылаться на эту короткую функцию из одной инструкции (так называемую \glslink{thunk function}{thunk-функцию}) из Thumb-кода.

\myindex{ARM!\Instructions!BL}
Кстати, в предыдущем примере (скомпилированном для режима ARM), переход при помощи инструкции \TT{BL} ведет 
на такую же \glslink{thunk function}{thunk-функцию}, однако режим процессора не переключается (отсюда отсутствие \q{X} в мнемонике инструкции).

\myparagraph{Еще о thunk-функциях}
\myindex{thunk-функции}

Thunk-функции трудновато понять, скорее всего, из-за путаницы в терминах.
Проще всего представлять их как адаптеры-переходники из одного типа разъемов в другой.
Например, адаптер позволяющее вставить в американскую розетку британскую вилку, или наоборот.
Thunk-функции также иногда называются \IT{wrapper-ами}. \IT{Wrap} в английском языке это \IT{обертывать}, \IT{завертывать}.
Вот еще несколько описаний этих функций:

\begin{framed}
\begin{quotation}
“A piece of coding which provides an address:”, according to P. Z. Ingerman, 
who invented thunks in 1961 as a way of binding actual parameters to their formal 
definitions in Algol-60 procedure calls. If a procedure is called with an expression 
in the place of a formal parameter, the compiler generates a thunk which computes 
the expression and leaves the address of the result in some standard location.

\dots

Microsoft and IBM have both defined, in their Intel-based systems, a “16-bit environment” 
(with bletcherous segment registers and 64K address limits) and a “32-bit environment” 
(with flat addressing and semi-real memory management). The two environments can both be 
running on the same computer and OS (thanks to what is called, in the Microsoft world, 
WOW which stands for Windows On Windows). MS and IBM have both decided that the process 
of getting from 16- to 32-bit and vice versa is called a “thunk”; for Windows 95, 
there is even a tool, THUNK.EXE, called a “thunk compiler”.
\end{quotation}
\end{framed}
% TODO FIXME move to bibliography and quote properly above the quote
( \href{http://go.yurichev.com/17362}{The Jargon File} )

\myindex{LAPACK}
\myindex{FORTRAN}
Еще один пример мы можем найти в библиотеке LAPACK --- (``Linear Algebra PACKage'') написанная на FORTRAN.
Разработчики на \CCpp также хотят использовать LAPACK, но переписывать её на \CCpp, а затем поддерживать несколько версий,
это безумие.
Так что имеются короткие функции на Си вызываемые из \CCpp{}-среды, которые, в свою очередь, вызывают функции на FORTRAN,
и почти ничего больше не делают:

\begin{lstlisting}[style=customc]
double Blas_Dot_Prod(const LaVectorDouble &dx, const LaVectorDouble &dy)
{
    assert(dx.size()==dy.size());
    integer n = dx.size();
    integer incx = dx.inc(), incy = dy.inc();

    return F77NAME(ddot)(&n, &dx(0), &incx, &dy(0), &incy);
}
\end{lstlisting}

Такие ф-ции еще называют ``wrappers'' (т.е., ``обертка'').


\subsubsection{ARM64}

\myparagraph{GCC}

Компилируем пример в GCC 4.8.1 для ARM64:

\lstinputlisting[numbers=left,label=hw_ARM64_GCC,caption=\NonOptimizing GCC 4.8.1 + objdump,style=customasmARM]{patterns/01_helloworld/ARM/hw.lst}

В ARM64 нет режима Thumb и Thumb-2, только ARM, так что тут только 32-битные инструкции.

Регистров тут в 2 раза больше: \myref{ARM64_GPRs}.
64-битные регистры теперь имеют префикс 
\TT{X-}, а их 32-битные части --- \TT{W-}.

\myindex{ARM!\Instructions!STP}
Инструкция \TT{STP} (\IT{Store Pair}) 
сохраняет в стеке сразу два регистра: \RegX{29} и \RegX{30}.
Конечно, эта инструкция может сохранять эту пару где угодно в памяти, но здесь указан регистр \ac{SP}, так что
пара сохраняется именно в стеке.

Регистры в ARM64 64-битные, каждый имеет длину в 8 байт, так что для хранения двух регистров нужно именно 16 байт.

Восклицательный знак (``!'') после операнда означает, что сначала от \ac{SP} будет отнято 16 и только затем
значения из пары регистров будут записаны в стек.

Это называется \IT{pre-index}.
Больше о разнице между \IT{post-index} и \IT{pre-index} 
описано здесь: \myref{ARM_postindex_vs_preindex}.

Таким образом, в терминах более знакомого всем процессора x86, первая инструкция~--- это просто аналог 
пары инструкций \TT{PUSH X29} и \TT{PUSH X30}.
\RegX{29} в ARM64 используется как \ac{FP}, а \RegX{30} 
как \ac{LR}, поэтому они сохраняются в прологе функции и
восстанавливаются в эпилоге.

Вторая инструкция копирует \ac{SP} в \RegX{29} (или \ac{FP}).
Это нужно для установки стекового фрейма функции.

\label{pointers_ADRP_and_ADD}
\myindex{ARM!\Instructions!ADRP/ADD pair}
Инструкции \TT{ADRP} и \ADD нужны для формирования адреса строки \q{Hello!} в регистре \RegX{0}, 
ведь первый аргумент функции передается через этот регистр.
Но в ARM нет инструкций, при помощи которых можно записать в регистр длинное число 
(потому что сама длина инструкции ограничена 4-я байтами. Больше об этом здесь: \myref{ARM_big_constants_loading}).
Так что нужно использовать несколько инструкций.
Первая инструкция (\TT{ADRP}) записывает в \RegX{0} адрес 4-килобайтной страницы где находится строка, 
а вторая (\ADD) просто прибавляет к этому адресу остаток.
Читайте больше об этом: \myref{ARM64_relocs}.

\TT{0x400000 + 0x648 = 0x400648}, и мы видим, что в секции данных \TT{.rodata} по этому адресу как раз находится наша
Си-строка \q{Hello!}.

\myindex{ARM!\Instructions!BL}
Затем при помощи инструкции \TT{BL} вызывается \puts. Это уже рассматривалось ранее: \myref{puts}.

Инструкция \MOV записывает 0 в \RegW{0}. 
\RegW{0} это младшие 32 бита 64-битного регистра \RegX{0}:

\input{ARM_X0_register}

А результат функции возвращается через \RegX{0}, и \main возвращает 0, 
так что вот так готовится возвращаемый результат.

Почему именно 32-битная часть?
Потому в ARM64, как и в x86-64, тип \Tint оставили 32-битным, для лучшей совместимости.

Следовательно, раз уж функция возвращает 32-битный \Tint, то нужно заполнить только 32 младших бита регистра \RegX{0}.

Для того, чтобы удостовериться в этом, немного отредактируем этот пример и перекомпилируем его.%

Теперь \main возвращает 64-битное значение:

\begin{lstlisting}[caption=\main возвращающая значение типа \TT{uint64\_t},style=customc]
#include <stdio.h>
#include <stdint.h>

uint64_t main()
{
        printf ("Hello!\n");
        return 0;
}
\end{lstlisting}

Результат точно такой же, только \MOV в той строке теперь выглядит так:

\begin{lstlisting}[caption=\NonOptimizing GCC 4.8.1 + objdump]
  4005a4:       d2800000        mov     x0, #0x0      // #0
\end{lstlisting}

\myindex{ARM!\Instructions!LDP}
Далее при помощи инструкции \INS{LDP} (\IT{Load Pair}) восстанавливаются регистры \RegX{29} и \RegX{30}.

Восклицательного знака после инструкции нет. Это означает, что сначала значения достаются из стека, и только потом \ac{SP} увеличивается на 16.

Это называется \IT{post-index}.

\myindex{ARM!\Instructions!RET}
В ARM64 есть новая инструкция: \RET. 
Она работает так же как и \INS{BX LR}, но там добавлен специальный бит,
подсказывающий процессору, что это именно выход из функции, а не просто переход, чтобы процессор
мог более оптимально исполнять эту инструкцию.

Из-за простоты этой функции оптимизирующий GCC генерирует точно такой же код.



}
\ITA{\subsection{ARM}
\label{sec:hw_ARM}

\myindex{\idevices}
\myindex{Raspberry Pi}
\myindex{Xcode}
\myindex{LLVM}
\myindex{Keil}
Per gli esperimenti con i processori ARM, sono stati utilizzati diversi compilatori:

\begin{itemize}
\item Diffuso nel settore embedded: Keil Release 6/2013.

\item Apple Xcode 4.6.3 IDE con il compilatore LLVM-GCC 4.2
\footnote{Apple Xcode 4.6.3 utilizza il compilatore open-source GCC come compilatore front-end ed il generatore di codice LLVM}.

\item GCC 4.9 (Linaro) (\ITAph{} ARM64), \ITAph{} \url{http://go.yurichev.com/17325}.

\end{itemize}

In qesto libro, se non diversamente specificato, viene utilizzato codice ARM a 32-bit ARM (incluse le modalita' Thumb e Thumb-2).
\ITAph{}

% subsections
\subsubsection{\NonOptimizingKeilVI (\ARMMode)}

Iniziamo a compilare il nostro esempio in Keil:

\begin{lstlisting}
armcc.exe --arm --c90 -O0 1.c 
\end{lstlisting}

\myindex{\IntelSyntax}
Il compilatore \IT{armcc} produce un listato assembly con sintassi Intel,
e utilizza macro di alto livello legate al processore ARM
\footnote{ad esempio, l' ARM mode 'e privo delle istruzioni \PUSH/\POP},
tuttavia e' piu' importante per noi vedere le istruzioni \q{cosi' come sono}, quindi guardiamo il risultato compilato con \IDA.

\begin{lstlisting}[caption=\NonOptimizingKeilVI (\ARMMode) \IDA,style=customasmARM]
.text:00000000             main
.text:00000000 10 40 2D E9    STMFD   SP!, {R4,LR}
.text:00000004 1E 0E 8F E2    ADR     R0, aHelloWorld ; "hello, world"
.text:00000008 15 19 00 EB    BL      __2printf
.text:0000000C 00 00 A0 E3    MOV     R0, #0
.text:00000010 10 80 BD E8    LDMFD   SP!, {R4,PC}

.text:000001EC 68 65 6C 6C+aHelloWorld  DCB "hello, world",0    ; DATA XREF: main+4
\end{lstlisting}

Nell'esempio possiamo facilmente vedere che ogni istruzione ha lunghezza pari a 4 byte.
Difatti abbiamo compilato il codice per la modalita' ARM e non Thumb.

\myindex{ARM!\Instructions!STMFD}
\myindex{ARM!\Instructions!POP}
La prima istruzione, \TT{STMFD SP!, \{R4,LR\}}\footnote{\ac{STMFD}},
funzione come l' istruzione \PUSH in x86, scrivendo i valori di due registri (\Reg{4} \ITAph{} \ac{LR}) nello stack.
Infatti il listato di output prodotto dal compilatore \IT{armcc}, per semplificazione, mostra l'istruzione \INS{PUSH \{r4,lr\}}.
Ma cio' non e' del tutto esatto. L'istruzione\PUSH e' disponibile solo in modalita' Thumb. Utilizziamo quindi \IDA per non fare confusione.

Questa istruzione dapprima \glslink{decrement}{decrementa} il valore di \ac{SP} cosi' da farlo puntare alla porzione dello stack che' e' libera di ospitare nuovi dati, quindi salva il valore dei registri \Reg{4} e \ac{LR} all'indirizzo memorizzato nel registro \ac{SP} appena modificato.

Questa istruzione (esattamente come \PUSH in Thumb mode) e' in grado di salvare il valore di piu' registri contemporaneamente, cosa che e' puo' risultare molto utile. 
A proposito, non ha un equivalente in x86.
Si puo' notare anche che l'istruzione \TT{STMFD} e' una generalizzazione dell'istruzione \PUSH (che estende le sue funzionalita'), poiche' puo' funzionare con qualunque registro, e non solo \ac{SP}.
In altre parole, \TT{STMFD} puo' essere usata per memorizzare un insieme di registri all'indirizzo di memoria specificato.

\myindex{\PICcode}
\myindex{ARM!\Instructions!ADR}
L'istruzione \INS{ADR R0, aHelloWorld}
aggiunge o sottrae il valore nel registro \ac{PC} all'offset dove e' memorizzata la stringa \TT{hello, world}.
Ci si potrebbe chiedere, come e' utilizzato qui il registro \TT{PC}?
Cio' e' detto \q{\PICcode}
\footnote{Maggiori informazioni sono fornite nella sezione~(\myref{sec:PIC})}.
Questo tipo di codice puo' essere eseguito a indirizzi non fissi (variabili)in memoria.
In altre parole, e' un indirizzamento relativo a \ac{PC} (\ac{PC}-relative addressing).
L'istruzione \TT{ADR} tiene conto della differenza tra l'indirizzo di questa istruzione e l'indirizzo dove si trova la stringa.
Questa differenza (offset) sara' sempre la stessa, a prescindere dall'indirizzo in cui nostro codice sara' caricato dall'\ac{OS}.
Cio' spiega perche' bisogna soltanto aggiungere l'indirizzo dell'istruzione corrente (from \ac{PC}) per ottenere l'indirizzo assoluto in memoria della nostra stringa C.

\myindex{ARM!\Registers!Link Register}
\myindex{ARM!\Instructions!BL}
L'istruzione \INS{BL \_\_2printf}\footnote{Branch with Link} chiama la funzione \printf. 
Questa istruzione funziona cosi': 
\begin{itemize}
\item memorizza l'indirizzo successivo all'istruzione \INS{BL} (\TT{0xC}) nel registro \ac{LR};
\item quindi passa il controllo a \printf scrivendo il suo indirizzo nel registro \ac{PC}.
\end{itemize}

Quando la funzione \printf termina la sua esecuzione, deve sapere a chi restituire il controllo (dove ritornare). Per questo motivo ogni funzione passa il controllo all'indirizzo memorizzato nel registro \ac{LR}.

Questa e' una differenza tra processori \ac{RISC} \q{puri} come ARM e processori simili a \ac{CISC} come x86, nei quali il return address e' solitamente memorizzato nello stack
\footnote{Maggiori informazioni si trovano nella prossima sezione~(\myref{sec:stack})}.

A proposito, un indirizzo assoluto o un offset a 32-bit non puo' essere codificato nell'istruzione a 32-bit \TT{BL} poiche' ha solo spazio per 24 bit.
Come potremmo ricordare, tutte le istruzioni in ARM-mode hanno dimensione fissa di 4 byte (32 bit).
Dunque possono essere collocate solo su indirizzi allineati a 4-byte.
Cio' implica che gli ultimi 2 bits dell'indirizzo dell'istruzione (che sono sempre zero) possono essere omessi.
Abbiamo in definitiva 26 bit per la codifica dell'offset (offset encoding). E cio e' sufficiente per codificare $current\_PC \pm{} \approx{}32M$.

\myindex{ARM!\Instructions!MOV}
L'istruzione successiva, \INS{MOV R0, \#0}\footnote{\ITAph{} MOVe} scrive soltanto 0 nel registro \Reg{0}.
Questo succede perche' la nostra funzione C restituisce 0, ed il valore di ritorno deve essere memorizzato nel registro \Reg{0}.

\myindex{ARM!\Registers!Link Register}
\myindex{ARM!\Instructions!LDMFD}
\myindex{ARM!\Instructions!POP}
L'ultima istruzione \INS{LDMFD SP!, {R4,PC}}\footnote{\ac{LDMFD} e' l'istruzione inversa rispetto a  \ac{STMFD}}.
Carica i valori dallo stack (o qualunque altra zona di memoria) per salvarli nei registri \Reg{4} e \ac{PC}, e \glslink{increment}{incrementa} lo \gls{stack pointer} \ac{SP}.
In questo caso funziona come \POP.\\
N.B. La prima istruzione \TT{STMFD} aveva salvato la coppia di registri \Reg{4} e \ac{LR} sullo stack, ma \Reg{4} e \ac{PC} vengono \IT{ripristinati} durante l'esecuzione di \TT{LDMFD}.

Come gia' sappiamo, l'indirizzo del posto a cui ogni funzione devere restituire il controllo e' solitamente salvato nel registro \ac{LR}.
La prima istruzione salva il suo valore nello stack perche' lo stesso registro sara' usato dalla nostra funzione \main per la chiamata a \printf.
Al termine della funzione, questo valore puo' essere scritto direttamente nel registro \ac{PC}, passando di fatti il controllo al punto in cui la nostra funzione era stata chiamata.

Dal momento che \main e' solitamente la funzione principale in \CCpp,
il controllo sara' restituito al loader dell' \ac{OS} oppure ad un punto in una \ac{CRT},
o qualcosa del genere.

Tutto cio' consente di omettere l'istruzione \TT{BX LR} alla fine della funzione.

\myindex{ARM!DCB}
\TT{DCB} e' una direttiva assembly che definisce un array di byte o una stringa ASCII, analoga alla direttiva DB in linguaggio assembly x86.


% TODO translate: \subsubsection{\NonOptimizingKeilVI (\ThumbMode)}

Compiliamo lo stesso esempio usando Keil in Thumb mode:

\begin{lstlisting}
armcc.exe --thumb --c90 -O0 1.c 
\end{lstlisting}

Otteniamo (in \IDA):

\begin{lstlisting}[caption=\NonOptimizingKeilVI (\ThumbMode) + \IDA,style=customasmARM]
.text:00000000             main
.text:00000000 10 B5          PUSH    {R4,LR}
.text:00000002 C0 A0          ADR     R0, aHelloWorld ; "hello, world"
.text:00000004 06 F0 2E F9    BL      __2printf
.text:00000008 00 20          MOVS    R0, #0
.text:0000000A 10 BD          POP     {R4,PC}

.text:00000304 68 65 6C 6C+aHelloWorld  DCB "hello, world",0    ; DATA XREF: main+2
\end{lstlisting}

Possiamo facilmente individuare gli opcode a 2-byte (16-bit). Questo e', come gia' detto, Thumb.
\myindex{ARM!\Instructions!BL}
L'istruzione \TT{BL} , tuttavia, e' fatta da due istruzioni a 16-bit.
Cio' accade perche' e' impossibile caricare un offset per la funzione \printf usando il poco spazio a disposizione in un opcode a 16-bit.
Pertanto la prima istruzione a 16-bit carica i 10 bit alti ( the higher 10 bits) dell' offset e la seconda istruzione carica
gli 11 bit bassi (the lower 11 bits) dell'offset.

% TODO:
% BL has space for 11 bits, so if we don't encode the lowest bit,
% then we should get 11 bits for the upper half, and 12 bits for the lower half.
% And the highest bit encodes the sign, so the destination has to be within
% \pm 4M of current_PC.
% This may be less if adding the lower half does not carry over,
% but I'm not sure --all my programs have 0 for the upper half,
% and don't carry over for the lower half.
% It would be interesting to check where __2printf is located relative to 0x8
% (I think the program counter is the next instruction on a multiple of 4
% for THUMB).
% The lower 11 bytes of the BL instructions and the even bit are
% 000 0000 0110 | 001 0010 1110 0 = 000 0000 0110 0010 0101 1100 = 0x00625c,
% so __2printf should be at 0x006264.
% But if we only have 10 and 11 bits, then the offset would be:
% 00 0000 0110 | 01 0010 1110 0 = 0 0000 0011 0010 0101 1100 = 0x00325c,
% so __2printf should be at 0x003264.
% In this case, though, the new program counter can only be 1M away,
% because of the highest bit is used for the sign.

Come gia' detto, tutte le istruzione in Thumb mode hanno dimensione pari a 2 bytes (o 16 bits).
Cio' implica che e' impossibile trovare un'istruzione Thumb-instruction ad un indirizzo dispari.
Secondo quanto detto, l'ultimo bit dell'indirizzo puo' essere omesso nell'encoding delle istruzioni.

Per riassumere, l'istruzione Thumb \TT{BL} puo' codificare un indirizzo in $current\_PC \pm{}\approx{}2M$.

\myindex{ARM!\Instructions!PUSH}
\myindex{ARM!\Instructions!POP}
Riguardo le altre istruzioni nella funzione: \PUSH and \POP qui funzionano come \TT{STMFD}/\TT{LDMFD} con
l'unica differenza che il registro \ac{SP} in questo caso non e' menzionato esplicitamente.
\TT{ADR} funziona esattamente come nell'esempio precedente.
\TT{MOVS} scrive 0 nel registro \Reg{0} per restituire zero.

% TODO translate: \subsubsection{\OptimizingXcodeIV (\ARMMode)}

Xcode 4.6.3 senza ottimizzazioni produce un sacco di codice ridondante, percio' studieremo l'output ottimizzato in cui le 
le istruzioni sono il meno possibile, settando lo switch del compilatore \Othree.

\begin{lstlisting}[caption=\OptimizingXcodeIV (\ARMMode),style=customasmARM]
__text:000028C4             _hello_world
__text:000028C4 80 40 2D E9   STMFD           SP!, {R7,LR}
__text:000028C8 86 06 01 E3   MOV             R0, #0x1686
__text:000028CC 0D 70 A0 E1   MOV             R7, SP
__text:000028D0 00 00 40 E3   MOVT            R0, #0
__text:000028D4 00 00 8F E0   ADD             R0, PC, R0
__text:000028D8 C3 05 00 EB   BL              _puts
__text:000028DC 00 00 A0 E3   MOV             R0, #0
__text:000028E0 80 80 BD E8   LDMFD           SP!, {R7,PC}

__cstring:00003F62 48 65 6C 6C+aHelloWorld_0  DCB "Hello world!",0
\end{lstlisting}

Le istruzioni \TT{STMFD} e \TT{LDMFD} ci sono gia' familiari.

\myindex{ARM!\Instructions!MOV}

L'istruzione \MOV scrive il numero \TT{0x1686} nel registro \Reg{0} .
Questo e' l'offset che punta alla stringa \q{Hello world!} .

Il registro \TT{R7} (per come standardizzato in \IOSABI) e' un frame pointer. Maggiori informazioni in basso.

\myindex{ARM!\Instructions!MOVT}
L'istruzione \TT{MOVT R0, \#0} (MOVe Top) scrive 0 nei 16 bit alti (higher 16 bits) del registro.
Il problema qui e' che l'istruzione generico \MOV in ARM mode potrebbe scrivere solo i 16 bit bassi del registro.

Ricorda che tutti gli opcode delle istruzioni in ARM mode sono limitati ad una lunghezza di 32 bit. Ovviamente questa limitazione non riguarda lo spostamento dei dati tra registri.
E questo spiega perche' esiste l'istruzione aggiuntiva \TT{MOVT} per scrivere nelle parti alte (da 16 a 31, inclusi).
Il suo uso qui e' comunque ridondante, perche' l'istruzione \TT{MOV R0, \#0x1686} di sopra ha azzerato la parte alta del registro.
E' probabilmente un difetto/svista del compilatore.
% TODO:
% I think, more specifically, the string is not put in the text section,
% ie. the compiler is actually not using position-independent code,
% as mentioned in the next paragraph.
% MOVT is used because the assembly code is generated before the relocation,
% so the location of the string is not yet known,
% and the high bits may still be needed.

\myindex{ARM!\Instructions!ADD}
L'istruzione \TT{ADD R0, PC, R0} aggiunge il valore in \ac{PC} al valore in \Reg{0}, per calcolare l'indirizzo assoluto della stringa \q{Hello world!}. 
Come sappiamo, si tratta di \q{\PICcode} e quindi questa correzione risulta essenziale in questo caso.

L'istruzione \INS{BL} chiama la funzione \puts ivece di \printf.

\label{puts}
\myindex{\CStandardLibrary!puts()}
\myindex{puts() instead of printf()}

GCC ha sostituito la prima chiamata a \printf con \puts.
Infatti: \printf con un solo argomento e' quasi analoga a \puts. 

\IT{Quasi}, perche' le due funzioni producono lo stesso risultato solo nel caso in cui la stringa non contiene 
identificatori di formato (format identifiers) che iniziano con \IT{\%}. 
In caso contrario l'effetto di queste due funzioni sarebbe diverso
\footnote{Bisogna anche notare che \puts non richiede un simbolo new line `\textbackslash{}n' alla fine della stringa,
per questo non lo vediamo qui.}.

Perche' il compilatore ha sostituito \printf con \puts? Probabilment perche' \puts e' piu' veloce
\footnote{\href{http://go.yurichev.com/17063}{ciselant.de/projects/gcc\_printf/gcc\_printf.html}}. 

Poiche' passa direttamente i caratteri a \gls{stdout} senza confrontare ciascuno di essi con il simbolo \IT{\%}.

Andando avanti, vediamo la familiare istruzione \TT{MOV R0, \#0} che imposta il registro \Reg{0} a to 0.

% TODO translate: \subsubsection{\OptimizingXcodeIV (\ThumbTwoMode)}

Di default Xcode 4.6.3 genera codice per Thumb-2 in questo modo:

\begin{lstlisting}[caption=\OptimizingXcodeIV (\ThumbTwoMode),style=customasmARM]
__text:00002B6C                   _hello_world
__text:00002B6C 80 B5          PUSH            {R7,LR}
__text:00002B6E 41 F2 D8 30    MOVW            R0, #0x13D8
__text:00002B72 6F 46          MOV             R7, SP
__text:00002B74 C0 F2 00 00    MOVT.W          R0, #0
__text:00002B78 78 44          ADD             R0, PC
__text:00002B7A 01 F0 38 EA    BLX             _puts
__text:00002B7E 00 20          MOVS            R0, #0
__text:00002B80 80 BD          POP             {R7,PC}

...

__cstring:00003E70 48 65 6C 6C 6F 20+aHelloWorld  DCB "Hello world!",0xA,0
\end{lstlisting}

% Q: If you subtract 0x13D8 from 0x3E70,
% you actually get a location that is not in this function, or in _puts.
% How is PC-relative addressing done in THUMB2?
% A: it's not Thumb-related. there are just mess with two different segments. TODO: rework this listing.

\myindex{\ThumbTwoMode}
\myindex{ARM!\Instructions!BL}
\myindex{ARM!\Instructions!BLX}

Le istruzioni \TT{BL} e \TT{BLX} in Thumb mode, come ricordiamo, sono codificate con una coppia di istruzioni 16-bit.
In Thumb-2 questi opcode \IT{surrogati} sono estesi in modo tale che le nuove istruzioni possano essere codificate in istruzioni a 32-bit.

Cio' appare ovvio considerando che che gli opcodes delle istruzioni Thumb-2 iniziano sempre con \TT{0xFx} o \TT{0xEx}.

Ma nel listato \IDA 
i byte degli opcode sono invertiti poiche' per i processori ARM le istruzioni sono codificate secondo il seguente principio: 
l'ultimo byte viene prima ed e' seguito dal primo byte ( per le modalita' Thumb e Thumb-2 ) 
oppure, per istruzioni in ARM mode il quarto byte viene prima, seguito dal terzo, dal secondo ed infine dal primo (a causa
della diversa \gls{endianness}).

Quindi i byte nei listati IDA sono collocati cosi':
\begin{itemize}
\item per ARM and ARM64 modes: 4-3-2-1;
\item per Thumb mode: 2-1;
\item per coppie di istruzioni a 16-bit in Thumb-2 mode: 2-1-4-3.
\end{itemize}

\myindex{ARM!\Instructions!MOVW}
\myindex{ARM!\Instructions!MOVT.W}
\myindex{ARM!\Instructions!BLX}

Come possiamo vedere, le istruzioni \TT{MOVW}, \TT{MOVT.W} e \TT{BLX} iniziano con \TT{0xFx}.

Ona delle istruzioni Thumb-2 e' \TT{MOVW R0, \#0x13D8} ~---memorizza un valore a 16-bit nella parte bassa del registro \Reg{0} ,
azzerando i bit piu' alti.

Allo stesso modo, \TT{MOVT.W R0, \#0} ~funziona come \TT{MOVT} nel precedente esempio, ma in Thumb-2.

\myindex{ARM!mode switching}
\myindex{ARM!\Instructions!BLX}

Tra le altre differenze, l'istruzione \TT{BLX} in questo caso e' usata al posto di \TT{BL}.

La differenza sta nel fatto che, oltre a salvare \ac{RA} nel registro \ac{LR} e passare il controllo alla funzione \puts,
il processore passa dalla modalita' Thumb/Thumb-2 alla modalita' ARM mode (o viceversa).

Questa istruzione e' posta qui poiche' l'istruzione a cui il controllo viene passato appare cosi' (e' codificata in ARM mode):

\begin{lstlisting}[style=customasmARM]
__symbolstub1:00003FEC _puts           ; CODE XREF: _hello_world+E
__symbolstub1:00003FEC 44 F0 9F E5     LDR  PC, =__imp__puts
\end{lstlisting}

E' essenzialmente un jump alla zona dove l'indirizzo di \puts e' scritto nella imports section.

Il lettore attento potrebbe chiedere: perche' non chiamare \puts proprio nel punto del codice, dove serve effettivamente?

Perche' non e' efficiente in termini di spazio.

\myindex{Dynamically loaded libraries}
Quasi tutti i programmi utilizzano librerie esterne dinamiche (come le DLL in Windows, .so in *NIX o .dylib in \MacOSX).
Le librerie dinamiche contengono funzioni usate di frequente, inclusa la funzione C standard \puts.

\myindex{Relocation}
In un file eseguibile (Windows PE .exe, ELF o Mach-O) e' presente una import section.
E' una lista di simboli (symbols - funzioni o variabili globali) importata da moduli esterni insieme ai nomi dei moduli stessi.

Il loade dell' \ac{OS} carica tutti i moduli necessari e, mentre enumera gli import symbols nel modulo primario, determina gli indirizzi
corretti per ciascun simbolo.

Nel nostro caso, \IT{\_\_imp\_\_puts} e' una variabile a 32-bit usata dal loader dell'\ac{OS} per memorizzare l'indirizzo corretto
della funzione in una libreria esterna. 
Successivamente l'istruzione \TT{LDR} legge semplicemente il valore a 32-bit da questa variabile e lo scrive nel registro \ac{PC},
passando il controllo ad esso.

Quindi, per ridurre il tempo necessario al loader dell'\ac{OS} per completare questa procedura, e' una buona idea scrivere l'indirizzo
di ogni simbolo solo una volta, in un punto dedicato.

\myindex{thunk-functions}
Inoltre, come abbiamo gia' capito, e' impossibile caricare un valore a 32-bit value in un registro utilizzando solo una istruzione senza
accesso alla memoria.

Pertanto, la soluzione ottimale e' quella di allocare una funzione separata, che funziona in ARM mode, con il solo scopo di passare
il controllo alla libreria dinamica e quindi saltare dal codice Thumb a questa piccola funzione di una sola istruzione (and then to jump to this short one-instruction)
(la cosiddetta \gls{thunk function}).

\myindex{ARM!\Instructions!BL}
A proposito, nel precedente esempio (compilato per ARM mode) il controllo e' passato da \TT{BL} alla stessa \gls{thunk function}.
La modalita' del processore pero' non viene cambiata (da cui l'assenza di una \q{X} nella instruction mnemonic).

\myparagraph{More about thunk-functions}
\myindex{thunk-functions}

Le thunk-functions sono difficili da comprendere, apparentemente, a causa di una denominazione impropria.
Il modo migliore per capirle e' pensarle come adattatori o convertitori da un tipo di jack ad un altro.
Ad esempio, un adattatore che consente l'inserimento di una spina elettrica Inglese in una presa Americana, o viceversa.
Le thunk functions sono a volte anche dette \IT{wrappers}.

Seguono altre descrizioni di queste funzioni:

\begin{framed}
\begin{quotation}
“Un pezzo di codice che fornisce un indirizzo:”, secondo to P. Z. Ingerman, 
che ha inventato le thunks nel 1961 come un modo di legare i parameters alle loro definizioni formali 
nelle chiamate a procedura in Algol-60. Se una procedura e' chiamata con un'espressione al posto di un parametro formale,
il compilatore genera una thunk che calcola l'espressione e lascia l'indirizzo del risultato in una posizione standard.

\dots

Microsoft e IBM hanno definito, nei loro sistemi basati su Intel, un “ambiente a 16-bit” 
(con orrendi segment registers e limitazioni di indirizzi a 64K) e un “ambiente a 32-bit” 
(con indirizzamento piatto e gestione della memoria semi reale). I due ambienti possono girare contemporaneamente
sullo stesso computer e OS (grazie a quello che, nel mondo Microsoft, e' detto WOW, acronimo per Windows On Windows).
MS e IBM hanno entrambi deciso che il processo di di passare da 16- a 32-bit e viceversa e' detto un “thunk”; in Windows 95, 
esiste anche un tool, THUNK.EXE, detto “thunk compiler”.
\end{quotation}
\end{framed}
% TODO FIXME move to bibliography and quote properly above the quote
( \href{http://go.yurichev.com/17362}{The Jargon File} )

% TODO translate: \subsubsection{ARM64}

\myparagraph{GCC}

Compiliamo l'esempio con GCC 4.8.1 per ARM64:

\lstinputlisting[numbers=left,label=hw_ARM64_GCC,caption=\NonOptimizing GCC 4.8.1 + objdump,style=customasmARM]{patterns/01_helloworld/ARM/hw.lst}

In ARM64 non ci sono le modalita' Thumb e Thumb-2, solo ARM, quindi soltanto istruzioni a 32-bit.
Il numero di registri e' raddoppiato: \myref{ARM64_GPRs}.
I registri a 64-bit hanno il prefisso \TT{X-} prefixes, mentre le loro parti a 32-bit hanno il prefisso --- \TT{W-}.

\myindex{ARM!\Instructions!STP}
L'istruzione \TT{STP} (\IT{Store Pair}) 
salva simultaneamente due registri nello stack: \RegX{29} e \RegX{30}.

Questa istruzione puo' ovviamente salvare la coppia di valori in una posizione arbitraria in memoria, tuttavia in questo caso e'
specificato il registro \ac{SP} , e di conseguenza la coppia viene salvata nello stack.

I registri ARM64 sono a 64-bit, ognuno di essi ha dimensione pari a 8 bytes, quindi sono necessari 16 bytes per salvare i due registri.

Il punto esclamativo (``!'') dopo l'operando sta a significare che 16 deve esse prima sottratto da \ac{SP} , e solo successivamente
i valori devono essere scritti nello stack.

Questo e' anche detto \IT{pre-index}.
Per le differenze tra \IT{post-index} e \IT{pre-index} 
leggere qui: \myref{ARM_postindex_vs_preindex}.

Quindi, in termini del piu' familiare x86, la prima istruzione e' semplicamente l'analogo della coppia
\TT{PUSH X29} e \TT{PUSH X30}.
\RegX{29} in ARM64 e' usato come \ac{FP} , e \RegX{30} 
come \ac{LR}, e questo spiega perche' sono salvati nel prologo di funzione e ripristinati nell'epilogo.

La seconda istruzione copia \ac{SP} in \RegX{29} (o \ac{FP}).
Cio' viene fatto per impostare lo stack frame della funzione.

\label{pointers_ADRP_and_ADD}
\myindex{ARM!\Instructions!ADRP/ADD pair}
Le istruzioni \TT{ADRP} e \ADD sono usate per inserire
l'indirizzo della stringa \q{Hello!} nel registro \RegX{0} , 
poiche' il primo argomento della funzione viene passato in questo registro.

Non esiste alcun tipo di istruzione in ARM in grado di salvare un numero molto grande in un registro (perche' la lunghezza delle
istruzioni e' limitata a 4 byte, maggiori informazioni qui: \myref{ARM_big_constants_loading}).
Percio' devono essere utilizzate piu' istruzioni. La prima (\TT{ADRP}) , scrive l'indirizzo della pagina di 4KiB (4KiB page)
,in cui si trova la stringa, nel registro \RegX{0}, e la seconda (\ADD) aggiunge semplicemente il resto dell'indirizzo.
Maggiori informazioni su questo tema: \myref{ARM64_relocs}.

\TT{0x400000 + 0x648 = 0x400648}, e vediamo la nostra C-string \q{Hello!} nel \TT{.rodata} data segment a questo indirizzo.

\myindex{ARM!\Instructions!BL}

\puts viene chiamata subito dopo usando l'istruzione \TT{BL}. Questo e' stato gia' discusso: \myref{puts}.

\MOV scrive 0 in \RegW{0}. 
\RegW{0} e' la parte bassa a 32 bits del registro a 64-bit \RegX{0}:

\input{ARM_X0_register}

Il risultato della funzione e' restituito tramite \RegX{0} e \main restituisce 0, quindi e' cosi' che viene preparato 
il valore da restituire.
Ma perche' usare la parte a 32-bit?

Perche' il tipo \Tint in ARM64, esattamente come in x86-64, e' a 32-bit, per maggiore compatibilita'.
Quindi se una funzione restituisce un \Tint a 32-bit, solo la parte piu' bassa a 32 bits del registro \RegX{0} sara' valorizzata.

Per verificare quanto detto, cambiamo leggermente l'esempio e ricompiliamolo.
Adesso \main restituisce un valore a 64-bit:

\begin{lstlisting}[caption=\main returning a value of \TT{uint64\_t} type,style=customc]
#include <stdio.h>
#include <stdint.h>

uint64_t main()
{
        printf ("Hello!\n");
        return 0;
}
\end{lstlisting}

Il risultato e' lo stesso, ma quell'istruzione MOV adesso appare cosi': 

\begin{lstlisting}[caption=\NonOptimizing GCC 4.8.1 + objdump]
  4005a4:       d2800000        mov     x0, #0x0      // #0
\end{lstlisting}

\myindex{ARM!\Instructions!LDP}

\INS{LDP} (\IT{Load Pair}) infine riprisina i registri \RegX{29} e \RegX{30}.

Non c'e' il punto esclamativo dopo l'istruzione: cio' implica che il valore viene prima caricato dallo stack, e solo successivamente 
\ac{SP} e' incrementato di 16.
Cio' e' detto \IT{post-index}.

\myindex{ARM!\Instructions!RET}
Una nuova istruzione e' apparsa in ARM64: \RET. 
Funziona esattamente come \TT{BX LR}, con l'aggiunta di uno speciale \IT{hint} bit, che informa la \ac{CPU}
del fatto che si tratta di un ritorno da una funzione, e non soltanto una normale istruzione jump, cosi' che possa 
essere eseguita in modo ottimale.

A causa della semplicita' della funzione, GCC con le opzioni di ottimizzazione genera esattamente lo stesso codice.


}
\DE{\subsection{ARM}
\label{sec:hw_ARM}

\myindex{\idevices}
\myindex{Raspberry Pi}
\myindex{Xcode}
\myindex{LLVM}
\myindex{Keil}
Für die Experimente mit ARM-Prozessoren wurden verschiedene Compiler genutzt:

\begin{itemize}
\item Verbreitet im Embedded-Bereich: Keil Release 6/2013.

\item Apple Xcode 4.6.3 IDE mit dem LLVM-GCC 4.2-Compiler
\footnote{Tatsächlich nutzt Apple Xcode 4.6.3 GCC als Front-End-Compiler und LLVM 
Code Generator}.

\item GCC 4.9 (Linaro) (für ARM64), verfügbar als Win32-Executable unter \url{http://go.yurichev.com/17325}.

\end{itemize}
Wenn nicht anders angegeben wird immer der 32-Bit ARM-Code (inklusive Thumb und Thumb-2-Mode) genutzt.
Wenn von 64-Bit ARM die Rede ist, dann wird ARM64 geschrieben.

% subsections
\subsubsection{\NonOptimizingKeilVI (\ARMMode)}

Beginnen wir mit dem Kompilieren des Beispiels mit Keil:

\begin{lstlisting}
armcc.exe --arm --c90 -O0 1.c 
\end{lstlisting}

\myindex{\IntelSyntax}
Der \IT{armcc}-Compiler erstellt Assembler-Quelltext im Intel-Syntax, hat aber High-Level-Makros
bezüglich der ARM-Prozessoren\footnote{d.h. der ARM-Mode hat keine \PUSH/\POP-Anweisungen}.
Es ist hier wichtig die \q{richtigen} Anweisungen zu sehen, deswegen ist hier das Ergebnis mit
\IDA kompiliert.

\begin{lstlisting}[caption=\NonOptimizingKeilVI (\ARMMode) \IDA,style=customasmARM]
.text:00000000             main
.text:00000000 10 40 2D E9    STMFD   SP!, {R4,LR}
.text:00000004 1E 0E 8F E2    ADR     R0, aHelloWorld ; "hello, world"
.text:00000008 15 19 00 EB    BL      __2printf
.text:0000000C 00 00 A0 E3    MOV     R0, #0
.text:00000010 10 80 BD E8    LDMFD   SP!, {R4,PC}

.text:000001EC 68 65 6C 6C+aHelloWorld  DCB "hello, world",0    ; DATA XREF: main+4
\end{lstlisting}

Im ersten Beispiel ist zu erkennen, dass jede Anweisung 4 Byte groß ist.
Tatsächlich wurde der Code für den ARM- und nicht den Thumb-Mode erstellt.

\myindex{ARM!\Instructions!STMFD}
\myindex{ARM!\Instructions!POP}
Die erste Anweisung, \INS{STMFD SP!, \{R4,LR\}}\footnote{\ac{STMFD}}, arbeitet wie eine x86-\PUSH-Anweisung
um die Werte der beiden Register (\Reg{4} and \ac{LR}) auf den Stack zu legen.

Die Ausgabe des \IT{armcc}-Compilers, zeigt, aus Gründen der Einfachheit, die \INS{PUSH \{r4,lr\}}-Anweisung.
Dies ist nicht vollständig präzise. Die \PUSH-Anweisung ist nur im Thumb-Mode verfügbar.
Um die Dinge nicht zu verwirrend zu machen, wird der Code in \IDA kompiliert.

Die Anweisung dekrementiert zunächst den \ac{SP}, so dass er auf den Bereich im Stack zeigt, der
für neue Einträge frei ist. Anschließend werden die Werte der Register \Reg{4} und \ac{LR} an der Adresse
gespeichert auf den der (modifizierte) \ac{SP} zeigt.

Diese Anweisungen (wie \PUSH im Thumb-Mode) ist in der Lage mehrere Register-Werte auf einmal zu speichern,
was sehr nützlich sein kann. Übrigens: in x86 gibt es dazu kein Äquivalent.
Außerdem ist erwähnenswert, dass die \TT{STMFD}-Anweisung eine Generalisierung der \PUSH-Anweisung
(ohne deren Eigenschaften) ist, weil sie auf jedes Register angewandt werden kann und nicht nur auf \ac{SP}.
Mit anderen Worten kann \TT{STMFD} genutzt werden um eine Reihen von Registern an einer angegebenen
Speicher-Adresse zu sichern.

\myindex{\PICcode}
\myindex{ARM!\Instructions!ADR}
Die \INS{ADR R0, aHelloWorld}-Anweisung addiert oder subtrahiert den Wert im \ac{PC}-Register zum Offset
an dem die \TT{hello, world}-Zeichenkette ist.
Man kann sich nun fragen, wie das \TT{PC}-Register hier genutzt wird.
Dies wird \q{\PICcode}\footnote{mehr darüber in der entsprechenden Sektion~(\myref{sec:PIC})} genannt.

Code dieser Art kann an nicht-festen Adressen im Speicher ausgeführt werden.
Mit anderen Worten: dies ist \ac{PC}-relative Adressierung.
Die \INS{ADR}-Anweisung berücksichtigt den Unterschied zwischen der Adresse dieser Anweisung und der Adresse
an dem die Zeichenkette gespeichert ist.
Der Unterschied (Offset) ist immer gleich, egal an welcher Adresse der Code vom \ac{OS} geladen wurden.
Dementsprechend ist alles was gemacht werden muss, die Adresse der aktuellen Anweisung (vom \ac{PC})
zu addieren um die absolute Speicheradresse der Zeichenkette zu bekommen.

\myindex{ARM!\Registers!Link Register}
\myindex{ARM!\Instructions!BL}
\INS{BL \_\_2printf}\footnote{Branch with Link}-Anweisung ruft die \printf-Funktion auf. 
Die Anweisung funktioniert wie folgt:

\begin{itemize}
\item Speichere die Adresse hinter der \INS{BL}-Anweisung (\TT{0xC}) in \ac{LR};
\item anschließend wird übergebe die Kontrolle an \printf indem dessen Adresse ins \ac{PC}-Register geschrieben wird.
\end{itemize}

Wenn \printf die Ausführung beendet, müssen Informationen vorliegen, wo die Ausführung weitergehen soll.
Das ist der Grund warum jede Funktion die Kontrolle an die Adresse, gespeichert im \ac{LR}-Register übergibt.

Dies ist ein Unterschied zwischen einem \q{reinem} \ac{RISC}-Prozessor wie ARM und \ac{CISC}-Prozessoren wie x86,
bei denen die Rücksprungadresse in der Regel auf dem Stack gespeichert wird.
Mehr dazu ist im nächsten Abschnitt zu lesen~(\myref{sec:stack}).

Übrigens eine absolute 32-Bit-Adresse oder -Offset kann nicht in einer 32-Bit-\TT{BL}-Anweisung kodiert werden,
weil diese nur für 24 Bit Platz bietet. Wie bereits erwähnt haben alle ARM-Mode-Anweisungen eine Größe
von 4 Byte (32 Bit). Aus diesem Grund können diese nur an 4-Byte-Grenzen des Speichers platziert werden.
Dies heißt auch, das die letzten zwei Bit der Anweisungsadresse (die immer Null sind) entfallen können.
Zusammenfassend, stehen 26 Bit für die Offset-Kodierung zur Verfügung. Dies ist genug für
$current\_PC \pm{} \approx{}32M$.

\myindex{ARM!\Instructions!MOV}
Als nächstes schreibt die Anweisung \INS{MOV R0, \#0}\footnote{das heißt MOVe} lediglich 0 in
das \Reg{0}-Register weil der Rückgabewert hier gespeichert wird und die gezeigte C-Funktion 0
als Argument für die return-Anweisung hat.

\myindex{ARM!\Registers!Link Register}
\myindex{ARM!\Instructions!LDMFD}
\myindex{ARM!\Instructions!POP}
Die letzte Anweisung \INS{LDMFD SP!, {R4,PC}}\footnote{\ac{LDMFD} ist eine inverse Anweisung von \ac{STMFD}}
lädt die Werte nacheinander vom Stack (oder eine andere Speicheradresse) um sie in die Register \Reg{4} und \ac{PC}
zu sichern. Außerdem wird der Stack Pointer \ac{SP} inkrementiert. Hier arbeitet der Befehl wie \POP.

Die erste Anweisung \TT{STMFD} sichert das Register-Paar \Reg{4} und \ac{LR} auf dem Stack, jedoch werden \Reg{4} und \ac{PC}
während der Ausführung von \TT{LDMFD} \IT{wiederhergestellt}.

Wie bereits bekannt, wird die Adresse die nach der Ausführung einer Funktion angesprungen wird in dem \ac{LR}-Register gesichert.
Die allererste Anweisung sichert diese Wert auf dem Stack weil das gleiche Register von der \main-Funktion genutzt wird,
wenn \printf aufgerufen wird.
Am Ende der Funktion kann dieser Wert direkt in das \ac{PC}-Register geschrieben werden und so die Ausführung an der
Stelle fortgesetzt werden an der die Funktion aufgerufen wurde.

Da \main in der Regel die erste Funktion in \CCpp ist, wird die Kontrolle an das \ac{OS} oder einen Punkt in der
\ac{CRT} übergeben.

All dies erlaubt das Auslassen der \INS{BX LR}-Anweisung am Ende der Funktion.

\myindex{ARM!DCB}
\TT{DCB} ist eine Assemblerdirektive die ein Array von Bytes oder ASCII anlegt, ähnlich der DB-Direktive
in der x86-Assembler-Sprache.

\subsubsection{\NonOptimizingKeilVI (\ThumbMode)}

Nachfolgend das gleiche Beispiel mit dem Keil-Compiler im Thumb-Mode erstellt:

\begin{lstlisting}
armcc.exe --thumb --c90 -O0 1.c 
\end{lstlisting}

In \IDA wird folgende Ausgabe erzeugt:

\begin{lstlisting}[caption=\NonOptimizingKeilVI (\ThumbMode) + \IDA,style=customasmARM]
.text:00000000             main
.text:00000000 10 B5          PUSH    {R4,LR}
.text:00000002 C0 A0          ADR     R0, aHelloWorld ; "hello, world"
.text:00000004 06 F0 2E F9    BL      __2printf
.text:00000008 00 20          MOVS    R0, #0
.text:0000000A 10 BD          POP     {R4,PC}

.text:00000304 68 65 6C 6C+aHelloWorld  DCB "hello, world",0    ; DATA XREF: main+2
\end{lstlisting}

Leicht zu erkennen sind die 2-Byte (16 Bit) OpCodes, die wie bereits erwähnt Thumb-Anweisungen sind.
\myindex{ARM!\Instructions!BL}
Die \TT{BL}-Anweisung besteht aus zwei 16-Bit-Anweisungen, weil es für die \printf-Funktion unmöglich ist
einen Offset zu laden, wenn der kleine Speicherbereich in einem 16-Bit-Opcode genutzt wird.
Aus diesem Grund lädt die erste 16-Bit-Anweisung die höherwertigen 10 Bit des Offsets und die zweite
Anweisung die niederwertigen 11 Bit.

% TODO:
% BL has space for 11 bits, so if we don't encode the lowest bit,
% then we should get 11 bits for the upper half, and 12 bits for the lower half.
% And the highest bit encodes the sign, so the destination has to be within
% \pm 4M of current_PC.
% This may be less if adding the lower half does not carry over,
% but I'm not sure --all my programs have 0 for the upper half,
% and don't carry over for the lower half.
% It would be interesting to check where __2printf is located relative to 0x8
% (I think the program counter is the next instruction on a multiple of 4
% for THUMB).
% The lower 11 bytes of the BL instructions and the even bit are
% 000 0000 0110 | 001 0010 1110 0 = 000 0000 0110 0010 0101 1100 = 0x00625c,
% so __2printf should be at 0x006264.
% But if we only have 10 and 11 bits, then the offset would be:
% 00 0000 0110 | 01 0010 1110 0 = 0 0000 0011 0010 0101 1100 = 0x00325c,
% so __2printf should be at 0x003264.
% In this case, though, the new program counter can only be 1M away,
% because of the highest bit is used for the sign.

Wie erwähnt haben alle Anweisungen im Thumb-Mode eine Größe von 2 Byte (16 Bit).
Dies bedeutet, dass es unmöglich ist an einer ungeraden Adresse einen Anweisung unterzubringen.
Das hat auch zur Folge, dass das letzte Bit der Adresse bei der Kodierung der
Anweisungen weggelassen werden kann.

Zusammenfassend kann die \TT{BL}-Thumb-Anweisung eine Adresse bis $current\_PC \pm{}\approx{}2M$ kodieren.

\myindex{ARM!\Instructions!PUSH}
\myindex{ARM!\Instructions!POP}
Wie für die anderen Anweisungen in dieser Funktion arbeiten \PUSH und \POP wie die beschriebenden \TT{STMFD}/\TT{LDMFD},
nur dass das \ac{SP}-Register hier nicht explizit genannt wird.
\TT{ADR} arbeitet genau wie in dem vorherigen Beispiel.
\TT{MOVS} schreibt 0 in das Register \Reg{0} um 0 zurückzugeben.

\subsubsection{\OptimizingXcodeIV (\ARMMode)}

Xcode 4.6.3 ohne Optimierung produziert eine Menge redundanten Code, so dass im Folgenden die
optimierte Ausgabe gelistet ist bei der die Anzahl der Anweisungen so klein wie möglich ist.
Der Compiler-Schalter ist \Othree.

\begin{lstlisting}[caption=\OptimizingXcodeIV (\ARMMode),style=customasmARM]
__text:000028C4             _hello_world
__text:000028C4 80 40 2D E9   STMFD           SP!, {R7,LR}
__text:000028C8 86 06 01 E3   MOV             R0, #0x1686
__text:000028CC 0D 70 A0 E1   MOV             R7, SP
__text:000028D0 00 00 40 E3   MOVT            R0, #0
__text:000028D4 00 00 8F E0   ADD             R0, PC, R0
__text:000028D8 C3 05 00 EB   BL              _puts
__text:000028DC 00 00 A0 E3   MOV             R0, #0
__text:000028E0 80 80 BD E8   LDMFD           SP!, {R7,PC}

__cstring:00003F62 48 65 6C 6C+aHelloWorld_0  DCB "Hello world!",0
\end{lstlisting}

Die Anweisungen \TT{STMFD} und \TT{LDMFD} sind bereits bekannt.

\myindex{ARM!\Instructions!MOV}

Die \MOV-Anweisung schreibt lediglich die Nummer \TT{0x1686} in das Register \Reg{0}.
Dies ist der Offset der auf die Zeichenkette \q{Hello world!} zeigt.

Das Register \TT{R7} (spezifiziert in \IOSABI) ist ein Frame Pointer. Mehr darüber folgt später.

\myindex{ARM!\Instructions!MOVT}
Die \TT{MOVT R0, \#0} (MOVe Top)-Anweisung schreibt 0 in die höherwertigen 16 Bit des Registers.
Das Problem ist hier, dass die generische \MOV-Anweisung im ARM-Mode nur die niederwertigen 16 Bit
des Registers beschreibt.

Dran denken: alle Opcodes im ARM-Mode sind in der Größe auf 32 Bit begrenzt. Natürlich gilt diese
Begrenzung nicht für das Verschieben von Daten zwischen Registern.
Aus diesem Grund existiert die zusätzliche Anweisung  \TT{MOVT} um in die höherwertigen Bits
(von 16 bis einschließlich 31) zu beschreiben.
Die Benutzung ist in diesem Fall redundant, weil die Anweisung \TT{MOV R0, \#0x1686} darüpber
den höherwertigen Teil des Registers zurückgesetzt hat.
Dies ist vermutlich ein Mangel des Compilers.

% TODO:
% I think, more specifically, the string is not put in the text section,
% ie. the compiler is actually not using position-independent code,
% as mentioned in the next paragraph.
% MOVT is used because the assembly code is generated before the relocation,
% so the location of the string is not yet known,
% and the high bits may still be needed.

\myindex{ARM!\Instructions!ADD}
Die Anweisung \TT{ADD R0, PC, R0} addiert den Wert im \ac{PC} zum Wert im Register \Reg{0}
um die absolute Adresse der \q{Hello world!}-Zeichenkette zu berechnen.
Wie bereits bekannt ist dies \q{\PICcode}, so dass diese Korrektur hier unbedingt notwendig ist.

Die \INS{BL}-Anweisung ruft \puts anstatt \printf auf.

\label{puts}
\myindex{\CStandardLibrary!puts()}
\myindex{puts() anstatt printf()}

GCC ersetzt den ersten \printf-Aufruf mit \puts. In der Tat ist \printf mit nur einem
Argument identisch mit \puts.

Die beiden Funktionen produzieren lediglich das gleiche Ergebnis, weil printf keine
Formatkennzeichner, beginnend mit \IT{\%}, enhält.
Sollte dies jedoch der Fall sein, wäre die Auswirkung der beiden Funktionen
unterschiedlich\footnote{Des weiteren benötigt \puts kein '\textbackslash{}n'
für den Zeilenumbruch am Ende der Zeichenkette, weswegen wir dies hier nicht sehen.}.

Warum hat der Compiler diese Ersetzung durchgeführt? Vermutlich hat dies Vorteile bei
der Geschwindigkeit, weil \puts schneller ist
\footnote{\href{http://go.yurichev.com/17063}{ciselant.de/projects/gcc\_printf/gcc\_printf.html}}
und lediglich die Zeichen zu \gls{stdout} übergibt, anstatt jedes Zeichen mit \IT{\%} zu vergleichen.

Als nächstes ist die bekannte Anweisung \TT{MOV R0, \#0} zu sehen um das Register \Reg{0} auf 0 zu setzen.

\subsubsection{\OptimizingXcodeIV (\ThumbTwoMode)}

Standardmäßig generiert Xcode 4.6.3 den Thumb-2-Code auf folgende Weise:

\begin{lstlisting}[caption=\OptimizingXcodeIV (\ThumbTwoMode),style=customasmARM]
__text:00002B6C                   _hello_world
__text:00002B6C 80 B5          PUSH            {R7,LR}
__text:00002B6E 41 F2 D8 30    MOVW            R0, #0x13D8
__text:00002B72 6F 46          MOV             R7, SP
__text:00002B74 C0 F2 00 00    MOVT.W          R0, #0
__text:00002B78 78 44          ADD             R0, PC
__text:00002B7A 01 F0 38 EA    BLX             _puts
__text:00002B7E 00 20          MOVS            R0, #0
__text:00002B80 80 BD          POP             {R7,PC}

...

__cstring:00003E70 48 65 6C 6C 6F 20+aHelloWorld  DCB "Hello world!",0xA,0
\end{lstlisting}

% Q: If you subtract 0x13D8 from 0x3E70,
% you actually get a location that is not in this function, or in _puts.
% How is PC-relative addressing done in THUMB2?
% A: it's not Thumb-related. there are just mess with two different segments. TODO: rework this listing.

\myindex{\ThumbTwoMode}
\myindex{ARM!\Instructions!BL}
\myindex{ARM!\Instructions!BLX}

Die \TT{BL}- und \TT{BLX}-Anweisung im Thumb-Mode ist als Paar von 16-Bit-Anweisungen kodiert.
In Thumb-2 sind diese \IT{Ersatz}-Opcodes so erweitert, dass neue Anweisungen hier mit 32 Bit
kodiert werden können.

Offensichtlich beginnen die Opcodes der Thumb-2-Anweisungen immer mit \TT{0xFx} oder \TT{0xEx}.

Im \IDA-Listing jedoch sind die Bytes der Opcodes vertauscht weil für den ARM-Prozessor die
Anweisungen wie folgt kodiert werden:
Das letzte Byte kommt zuerst und danach das erste (für Thumb- und Thum-2-Mode) oder für
Anweisungen im ARM-Mode kommt das vierte Byte zuerst, dann das dritte, dann das zweite und
zum Schluss das erste (aufgrund des unterschiedlichen \gls{endianness}).

Die Bytes sind also im \IDA-Listing wie folgt angeordnet:
\begin{itemize}
\item für ARM und ARM64 Mode: 4-3-2-1;
\item für Thumb Mode: 2-1;
\item für 16-Bit-Anweisungspaar in Thumb-2 Mode: 2-1-4-3.
\end{itemize}

\myindex{ARM!\Instructions!MOVW}
\myindex{ARM!\Instructions!MOVT.W}
\myindex{ARM!\Instructions!BLX}

Wie zu sehen ist, beginnend die Anweisungen \TT{MOVW}, \TT{MOVT.W} und \TT{BLX} mit \TT{0xFx}.

Eine der Thumb-2-Anweisungen ist \TT{MOVW R0, \#0x13D8} ~---sie speichert einen 16-Bit-Wert in den
niederwertigeren Teil des \Reg{0}-Registers und setzt die höherwertigen Bits auf 0.

Des weiteren funktioniert \TT{MOVT.W R0, \#0} genau wie \TT{MOVT} aus dem vorherigen Beispiel,
jedoch nur für Thumb-2.

\myindex{ARM!mode switching}
\myindex{ARM!\Instructions!BLX}

Neben den anderen Unterschieden wird in diesem Fall die \TT{BLX}-Anweisung anstatt \TT{BL} genutzt.

Der Unterschied ist, dass, neben dem Speichern von \ac{RA} in das \ac{LR}-Register und die Übergabe
der Ausführungskontrolle an die \puts-Funktion, der Prozessor auch vom Thumb/Thumb-2-Mode in den
ARM-Mode (oder zurück) wechselt.

Diese Anweisung ist hier eingefügt weil die Anweisung mit der die Kontrolle abgegeben wird wie folgt
aussieht (im ARM-Mode kodiert):

\begin{lstlisting}[style=customasmARM]
__symbolstub1:00003FEC _puts           ; CODE XREF: _hello_world+E
__symbolstub1:00003FEC 44 F0 9F E5     LDR  PC, =__imp__puts
\end{lstlisting}

Dies ist im Endeffekt ein Sprung an die Stelle an der die Adresse von \puts in der import-Sektion geschriben wird.

Der aufmerksame Leser mag fragen: warum wird \puts nicht direkt an der Stelle im Code aufgerufen,
an der es benötigt wird? Dies wäre nicht sehr speicherplatzeffizient.

\myindex{Dynamically loaded libraries}
Fast jedes Programm nutzt externe, dynamische Bibliotheken (wie DLL in Windows, .so in *NIX oder.dylib in \MacOSX).
Diese Bibliotheken beinhalten häufig genutzte Funktion wie die Standard-C-Funktion \puts.

\myindex{Relocation}
In einer ausführbaren Binärdatei (Windows PE .exe, ELF oder Mach-O) existiert eine import-Sektion.
Dies ist eine Liste von Symbolen (Funktionen oder globale Variablen) die, zusammen mit den Namen, von
externen Modulen importiert werden.

Der \ac{OS}-Loader läd alle Module die gebraucht werden und bestimmt die korrekten Adressen von jedem Symbol,
während diese in dem primärem Modul aufgelistet werden.

In dem vorliegenden Fall ist \IT{\_\_imp\_\_puts} eine 32-Bit-Variable die vom \ac{OS}-Loader genutzt wird
um die korrekte Adresse der Funktion in der externen Bibliothek zu speichern.
Anschließend liest die \TT{LDR}-Anweisung den 32-Bit-Wert dieser Variable und schreibt ihn in das \ac{PC}-Register
bevor die Ausführkontrolle dorthin übergeben wird.

Um also die Zeit zu reduzieren die der \ac{OS}-Loader für dieses Vorgehen benötigt, ist es eine gute Idee
die Adressen für jedes Symbol einmalig an eine geeignete Stelle zu schreiben.

\myindex{thunk-functions}
Daneben wurde bereits erwähnt, dass es unmöglich ist einen 32-Bit-Wert in ein Register zu laden wenn
nur eine Anweisung ohne Speicher-Zugriff genutzt wird.

Aus diesem Grund ist die optimale Lösung, eine separate Funktion im ARM-Mode zu allozieren die lediglich
die Aufgabe hat die Ausführkontrolle an die dynamische Bibliothek zu übergeben und dann in diese kurze
Funktion mit einer Anweisung (so genannte \gls{thunk function}) aus dem Thumb-Code auszuführen.

\myindex{ARM!\Instructions!BL}
Übrigens: in dem vorherigen Beispiel (für ARM-Mode kompiliert) wird die Ausführkontrolle durch \TT{BL}
an die gleiche \gls{thunk function} übergeben.
Der Prozessor-Modus wird hier jedoch aufgrund des Fehlens eines \q{X} im Anweisungsnamen nicht gewechselt.

\myparagraph{Mehr über Thunk-Funktionen}
\myindex{thunk-functions}

Thunk-Funktionen sind aufgrund der irrtümlichen Bezeichnung schwierig zu verstehen.
Der einfachste Weg ist es sie als Adapter oder Konverter zwischen verschiedenen Anschlüssen aufzufassen.
Zum Beispiel wie einen Adapter zwischen einer britischen und einer amerikanischen Steckdose oder andersherum.
Thunk-Funktionen werden manchmal auch \IT{Wrapper} genannt.

Hier sind einige weitere Beschreibung dieser Funktionstypen:

\begin{framed}
\begin{quotation}
"Ein Teil der Software um Adressen zur Verfügung zu stellen:" nach P. Z. Ingerman,
der 1961 Thunk-Funktionen als Möglichkeit zum Binden von Aktualparametern zu deren
formalen Definitionen in Algol-60-Prozedur-Aufrufen. Wenn eine Prozedur mit einem Ausdruck anstatt
der formalen Parameter aufgerufen wird, generiert der Compiler eine Thunk-Funktion die den Ausdruck
errechnet und die Adresse des Ergebnisses an eine Standard-Stelle speichert.

\dots

% TODO: the english version is kind of misleading -- I could not find the word "bletcherous"
Microsoft und IBM haben beide in ihrem Intel-basierten System eine "16-Bit Umgebung"
und eine "32-Bit-Umgebung" definiert. Beide können auf dem selben Computer und demselben
Betriebssystem laufen (dank dem was Microsoft \q{ Windows On Windows} (WOW) nennt).
Sowohl MS als auch IBM haben entschieden, den Vorgang der zwischen 16- und 32-Bit wechselt
"Thunk" zu nennen; für Windows 95 existiert sogar ein Tool THUNK.EXE, das Thunk-Compiler
genannt wird.\end{quotation}
\end{framed}
% TODO FIXME move to bibliography and quote properly above the quote
( \href{http://go.yurichev.com/17362}{The Jargon File} )

\subsubsection{ARM64}

\myparagraph{GCC}

Das Beispiel wird im Folgenden mit GCC 4.1.8 in ARM64 kompiliert:

\lstinputlisting[numbers=left,label=hw_ARM64_GCC,caption=\NonOptimizing GCC 4.8.1 + objdump,style=customasmARM]{patterns/01_helloworld/ARM/hw.lst}

Es gibt keine Thumb- oder Thumb-2-Modes in ARM64, sondern nur ARM, also 32-Bit-Anweisungen.
Die Register-Anzahl ist verdoppelt: \myref{ARM64_GPRs}.
64-Bit-Register haben einen \TT{X-}Prefix, 32-Bit-Teile ein \TT{W-}.

\myindex{ARM!\Instructions!STP}
Die \TT{STP}-Anweisung (\IT{Store Pair}) speichert zwei Register auf dem Stack gleichzeitig:
\RegX{29} und \RegX{30}.

Natürlich kann diese Anweisung dieses Registerpaar an einer beliebigen Stelle im Speicher
sichern, aber da hier das \ac{SP}-Register angegeben ist, wird das Paar auf dem Stack gesichert.

ARM64-Register sind 64 Bit breit, jedes von ihnen ist 8 Byte groß. Dementsprechend werden 16 Byte
für das Speichern zweier Register benötigt.

Das Ausrufungszeichen (``!'')  nach dem Operanden bedeutet, dass zunächst der Wert 16 vom \ac{SP}
subtrahiert werden muss und erst dann die Werte vom Register-Paar auf den Stack geschrieben werden.
Dies wird auch \IT{pre-index} genannt.
Mehr über den Unterschied von \IT{post-index} und \IT{pre-index} ist im Abschnitt
\myref{ARM_postindex_vs_preindex} zu finden.

Im Sprachgebrauch des gebräuchlicheren x86, ist die erste Anweisung analog zu den Anweisungen
\TT{PUSH X29} und \TT{PUSH X30} zu verstehen.
\RegX{29} wird als \ac{FP} in ARM64 genutzt, und \RegX{30} als \ac{LR}, weswegen sie am Anfang der
Funktion gesichert und am Ende wiederhergestellt werden.

Die zweite Anweisung kopiert \ac{SP} in \RegX{29} (oder \ac{FP}) um den Stack Frame vorzubereiten.

\label{pointers_ADRP_and_ADD}
\myindex{ARM!\Instructions!ADRP/ADD pair}
\TT{ADRP} und \ADD-Anweisungen werden genutzt um die Adresse der Zeichenkette \q{Hello!} in das
Register \RegX{0} zu schreiben, da das erste Funktionsargument in an dieser Stelle übergeben wird. 

Es gibt in ARM keine Anweisung, die eine große Zahl in einem Register sichern kann, weil die Länge der
Anweisungen auf 4 Byte begrenzt ist. Siehe dazu auch \myref{ARM_big_constants_loading}).
Aus diesem Grund müssen mehrere Anweisungen genutzt werden. Die erste (\TT{ADRP}) schreibt die Adresse
der 4KiB-Page in der die Zeichenkette sich befindet in das Register \RegX{0}.
Die zweite (\ADD) addiert lediglich den Rest der Adresse.
Siehe dazu auch \myref{ARM64_relocs}.

\TT{0x400000 + 0x648 = 0x400648}, und die Zeichenkette \q{Hello!} ist im \TT{.rodata} Daten-Segmet
an dieser Adresse zu sehen.

\myindex{ARM!\Instructions!BL}

\puts wird anschließend mit der \TT{BL}-Anweisung aufgerufen. Dies wurde bereits diskutiert: \myref{puts}.

\MOV schreibt 0 in \RegW{0}.
\RegW{0} sind die niederwertigeren 32 Bit des 64-Bit-Registers \RegX{0}:

\input{ARM_X0_register}

Das Ergebnis der Funktion wird über \RegX{0} zurückgegeben und \main gibt 0 zurück.
Dies ist also der Weg wie das Ergebnis vorbereitet wird.
Der 32-Bit-Teil wird genutzt,weil der \Tint-Datentyp in ARM64 aus Kompatibilitätsgründen,
wie in x86-64, 32 Bit breit ist.

Da die Funktion einen 32-Bit \Tint-Wert zurück gibt, müssen lediglich die unteren 32 Bits des
\RegX{0}-Registers gefüllt werden.

Um dies zu überprüfen wird das Beispiel leicht verändert und neu kompiliert.
\main soll nun einen 64-Bit-Wert zurück geben:

\begin{lstlisting}[caption=\main gibt einen \TT{uint64\_t}-Datentyp zurück,style=customc]
#include <stdio.h>
#include <stdint.h>

uint64_t main()
{
        printf ("Hello!\n");
        return 0;
}
\end{lstlisting}

Das Ergebnis ist das gleiche, allerdings sieht \MOV nun wie folgt aus:

\begin{lstlisting}[caption=\NonOptimizing GCC 4.8.1 + objdump]
  4005a4:       d2800000        mov     x0, #0x0      // #0
\end{lstlisting}

\myindex{ARM!\Instructions!LDP}

\INS{LDP} (\IT{Load Pair}) stellt anschließend die Register \RegX{29} und \RegX{30} wieder her.

An dieser Stelle steht kein Ausrufungszeichen nach der Anweisung: dies impliziert, dass der Wert zunächst
vom Stack gelesen wird und erst dann wird \ac{SP} um den Wert 16 verringert.
Dies wird \IT{post-index} genannt.

\myindex{ARM!\Instructions!RET}
Eine neue Anweisung taucht hier in ARM64 auf \RET.
Diese arbeitet wie \TT{BX LR}, jedoch wird ein spezielles \IT{Hinweis-}Bit hinzugefügt, welches die \ac{CPU}
darüber informiert, dass dies ein Rücksprung aus einer Funktion ist und kein anderer Sprung, so dass die
Ausführung optimiert werden kann.

Aufgrund der Einfachheit dieser Funktion, erstellt der optimierende GCC den gleichen Code.

}

\EN{\subsection{MIPS}

\subsubsection{A word about the \q{global pointer}}
\label{MIPS_GP}

\myindex{MIPS!\GlobalPointer}

One important MIPS concept is the \q{global pointer}.
As we may already know, each MIPS instruction has a size of 32 bits, so it's impossible to embed a 32-bit
address into one instruction: a pair has to be used for this 
(like GCC did in our example for the text string address loading).
It's possible, however, to load data from the address in the range of $register-32768...register+32767$ using one
single instruction (because 16 bits of signed offset could be encoded in a single instruction).
So we can allocate some register for this purpose and also allocate a 64KiB area of most used data.
This allocated register is called a \q{global pointer} and it points to the middle of the 64KiB area.
This area usually contains global variables and addresses of imported functions like \printf, 
because the GCC developers decided that getting the address of some function must be as fast as a single instruction
execution instead of two.
In an ELF file this 64KiB area is located partly in sections .sbss (\q{small \ac{BSS}}) for uninitialized data and 
.sdata (\q{small data}) for initialized data.
This implies that the programmer may choose what data he/she wants to be accessed fast and place it into 
.sdata/.sbss.
Some old-school programmers may recall the MS-DOS memory model \myref{8086_memory_model} 
or the MS-DOS memory managers like XMS/EMS where all memory was divided in 64KiB blocks.

\myindex{PowerPC}

This concept is not unique to MIPS. At least PowerPC uses this technique as well.

\subsubsection{\Optimizing GCC}

Lets consider the following example, which illustrates the \q{global pointer} concept.

\lstinputlisting[caption=\Optimizing GCC 4.4.5 (\assemblyOutput),numbers=left,style=customasmMIPS]{patterns/01_helloworld/MIPS/hw_O3_EN.s}

As we see, the \$GP register is set in the function prologue to point to the middle of this area.
The \ac{RA} register is also saved in the local stack.
\puts is also used here instead of \printf.
\myindex{MIPS!\Instructions!LW}
The address of the \puts function is loaded into \$25 using \INS{LW} the instruction (\q{Load Word}).
\myindex{MIPS!\Instructions!LUI}
\myindex{MIPS!\Instructions!ADDIU}
Then the address of the text string is loaded to \$4 using \INS{LUI} (\q{Load Upper Immediate}) and 
\INS{ADDIU} (\q{Add Immediate Unsigned Word}) instruction pair.
\INS{LUI} sets the high 16 bits of the register (hence \q{upper} word in instruction name) and \INS{ADDIU} adds
the lower 16 bits of the address.

\INS{ADDIU} follows \INS{JALR} (haven't you forgot \IT{branch delay slots} yet?).
The register \$4 is also called \$A0, which is used for passing the first function argument
\footnote{The MIPS registers table is available in appendix \myref{MIPS_registers_ref}}.

\myindex{MIPS!\Instructions!JALR}

\INS{JALR} (\q{Jump and Link Register}) jumps to the address stored in the \$25 register (address of \puts) 
while saving the address of the next instruction (LW) in \ac{RA}.
This is very similar to ARM.
Oh, and one important thing is that the address saved in \ac{RA} is not the address of the next instruction (because
it's in a \IT{delay slot} and is executed before the jump instruction),
but the address of the instruction after the next one (after the \IT{delay slot}).
Hence, $PC + 8$ is written to \ac{RA} during the execution of \TT{JALR}, in our case, this is the address of the \INS{LW}
instruction next to \INS{ADDIU}.

\INS{LW} (\q{Load Word}) at line 20 restores \ac{RA} from the local stack (this instruction is actually part of the function epilogue).

\myindex{MIPS!\Pseudoinstructions!MOVE}

\INS{MOVE} at line 22 copies the value from the \$0 (\$ZERO) register to \$2 (\$V0).
\label{MIPS_zero_register}

MIPS has a \IT{constant} register, which always holds zero.
Apparently, the MIPS developers came up with the idea that zero is in fact the busiest constant in the computer programming,
so let's just use the \$0 register every time zero is needed.

Another interesting fact is that MIPS lacks an instruction that transfers data between registers.
In fact, \TT{MOVE DST, SRC} is \TT{ADD DST, SRC, \$ZERO} ($DST=SRC+0$), which does the same.
Apparently, the MIPS developers wanted to have a compact opcode table.
This does not mean an actual addition happens at each \INS{MOVE} instruction.
Most likely, the \ac{CPU} optimizes these pseudo instructions and the \ac{ALU} is never used.

\myindex{MIPS!\Instructions!J}

\INS{J} at line 24 jumps to the address in \ac{RA}, which is effectively performing a return from the function.
\INS{ADDIU} after \INS{J} is in fact executed before \INS{J} (remember \IT{branch delay slots}?) and is part of the function epilogue.
Here is also a listing generated by \IDA. Each register here has its own pseudo name:

\lstinputlisting[caption=\Optimizing GCC 4.4.5 (\IDA),numbers=left,style=customasmMIPS]{patterns/01_helloworld/MIPS/hw_O3_IDA_EN.lst}

The instruction at line 15 saves the GP value into the local stack, and this instruction is missing mysteriously from the GCC output listing, maybe by a GCC error
\footnote{Apparently, functions generating listings are not so critical to GCC users, so some unfixed errors may still exist.}.
The GP value has to be saved indeed, because each function can use its own 64KiB data window.
The register containing the \puts address is called \$T9, because registers prefixed with T- are called
\q{temporaries} and their contents may not be preserved.

\subsubsection{\NonOptimizing GCC}

\NonOptimizing GCC is more verbose.

\lstinputlisting[caption=\NonOptimizing GCC 4.4.5 (\assemblyOutput),numbers=left,style=customasmMIPS]{patterns/01_helloworld/MIPS/hw_O0_EN.s}

We see here that register FP is used as a pointer to the stack frame.
We also see 3 \ac{NOP}s.
The second and third of which follow the branch instructions.
Perhaps the GCC compiler always adds \ac{NOP}s (because of \IT{branch delay slots}) after branch
instructions and then, if optimization is turned on, maybe eliminates them.
So in this case they are left here.

Here is also \IDA listing:

\lstinputlisting[caption=\NonOptimizing GCC 4.4.5 (\IDA),numbers=left,style=customasmMIPS]{patterns/01_helloworld/MIPS/hw_O0_IDA_EN.lst}

\myindex{MIPS!\Pseudoinstructions!LA}

Interestingly, \IDA recognized the \INS{LUI}/\INS{ADDIU} instructions pair and coalesced them into one 
\INS{LA} (\q{Load Address}) pseudo instruction at line 15.
We may also see that this pseudo instruction has a size of 8 bytes!
This is a pseudo instruction (or \IT{macro}) because it's not a real MIPS instruction, but rather
a handy name for an instruction pair.

\myindex{MIPS!\Pseudoinstructions!NOP}
\myindex{MIPS!\Instructions!OR}

Another thing is that \IDA doesn't recognize \ac{NOP} instructions, so here they are at lines 22, 26 and 41.
It is \TT{OR \$AT, \$ZERO}.
Essentially, this instruction applies the OR operation to the contents of the \$AT register
with zero, which is, of course, an idle instruction.
MIPS, like many other \ac{ISA}s, doesn't have a separate \ac{NOP} instruction.

\subsubsection{Role of the stack frame in this example}

The address of the text string is passed in the register.
Why setup a local stack anyway?
The reason for this lies in the fact that the values of registers \ac{RA} and GP have to be saved somewhere 
(because \printf is called), and the local stack is used for this purpose.
If this was a \gls{leaf function}, it would have been possible to get rid of the function prologue and epilogue,
for example: \myref{MIPS_leaf_function_ex1}.

\subsubsection{\Optimizing GCC: load it into GDB}

\myindex{GDB}
\lstinputlisting[caption=sample GDB session]{patterns/01_helloworld/MIPS/O3_GDB.txt}

}
\RU{\subsection{MIPS}

\subsubsection{О \q{глобальном указателе} (\q{global pointer})}
\label{MIPS_GP}

\myindex{MIPS!\GlobalPointer}
\q{Глобальный указатель} (\q{global pointer})~--- это важная концепция в MIPS.
Как мы уже возможно знаем, каждая инструкция в MIPS имеет размер 32 бита, поэтому невозможно
закодировать 32-битный адрес внутри одной инструкции. Вместо этого нужно использовать пару инструкций
(как это сделал GCC для загрузки адреса текстовой строки в нашем примере).
С другой стороны, используя только одну инструкцию, 
возможно загружать данные по адресам в пределах $register-32768...register+32767$, потому что 16 бит
знакового смещения можно закодировать в одной инструкции).
Так мы можем выделить какой-то регистр для этих целей и ещё выделить буфер в 64KiB для самых 
часто используемых данных.
Выделенный регистр называется \q{глобальный указатель} (\q{global pointer}) и он указывает на середину
области 64KiB.
Эта область обычно содержит глобальные переменные и адреса импортированных функций вроде \printf,
потому что разработчики GCC решили, что получение адреса функции должно быть как можно более быстрой операцией,
исполняющейся за одну инструкцию вместо двух.
В ELF-файле эта 64KiB-область находится частично в секции .sbss (\q{small \ac{BSS}}) для неинициализированных
данных и в секции .sdata (\q{small data}) для инициализированных данных.
Это значит что программист может выбирать, к чему нужен как можно более быстрый доступ, и затем расположить
это в секциях .sdata/.sbss.
Некоторые программисты \q{старой школы} могут вспомнить модель памяти в MS-DOS \myref{8086_memory_model} 
или в менеджерах памяти вроде XMS/EMS, где вся память делилась на блоки по 64KiB.

\myindex{PowerPC}
Эта концепция применяется не только в MIPS. По крайней мере PowerPC также использует эту технику.

\subsubsection{\Optimizing GCC}

Рассмотрим следующий пример, иллюстрирующий концепцию \q{глобального указателя}.

\lstinputlisting[caption=\Optimizing GCC 4.4.5 (\assemblyOutput),numbers=left,style=customasmMIPS]{patterns/01_helloworld/MIPS/hw_O3_RU.s}

Как видно, регистр \$GP в прологе функции выставляется в середину этой области.
Регистр \ac{RA} сохраняется в локальном стеке.
Здесь также используется \puts вместо \printf.
\myindex{MIPS!\Instructions!LW}
Адрес функции \puts загружается в \$25 инструкцией \INS{LW} (\q{Load Word}).
\myindex{MIPS!\Instructions!LUI}
\myindex{MIPS!\Instructions!ADDIU}
Затем адрес текстовой строки загружается в \$4 парой инструкций \INS{LUI} (\q{Load Upper Immediate}) и
\INS{ADDIU} (\q{Add Immediate Unsigned Word}).
\INS{LUI} устанавливает старшие 16 бит регистра (поэтому в имени инструкции присутствует \q{upper}) и \INS{ADDIU}
прибавляет младшие 16 бит к адресу.
\INS{ADDIU} следует за \INS{JALR} (помните о \IT{branch delay slots}?).
Регистр \$4 также называется \$A0, который используется для передачи первого аргумента функции
\footnote{Таблица регистров в MIPS доступна в приложении \myref{MIPS_registers_ref}}.
\myindex{MIPS!\Instructions!JALR}
\INS{JALR} (\q{Jump and Link Register}) делает переход по адресу в регистре \$25 (там адрес \puts) 
при этом сохраняя адрес следующей инструкции (\INS{LW}) в \ac{RA}.
Это так же как и в ARM.
И ещё одна важная вещь: адрес сохраняемый в \ac{RA} это адрес не следующей инструкции (потому что
это \IT{delay slot} и исполняется перед инструкцией перехода),
а инструкции после неё (после \IT{delay slot}).
Таким образом во время исполнения \INS{JALR} в \ac{RA} записывается $PC + 8$.
В нашем случае это адрес инструкции \INS{LW} следующей после \INS{ADDIU}.

\INS{LW} (\q{Load Word}) в строке 20 восстанавливает \ac{RA} из локального стека (эта инструкция скорее часть эпилога функции).

\myindex{MIPS!\Pseudoinstructions!MOVE}
\INS{MOVE} в строке 22 копирует значение из регистра \$0 (\$ZERO) в \$2 (\$V0).

\label{MIPS_zero_register}
В MIPS есть \IT{константный} регистр, всегда содержащий ноль.
Должно быть, разработчики MIPS решили, что 0 это самая востребованная константа в программировании,
так что пусть будет использоваться регистр \$0, всякий раз, когда будет нужен 0.
Другой интересный факт: в MIPS нет инструкции, копирующей значения из регистра в регистр.
На самом деле, \TT{MOVE DST, SRC} это \TT{ADD DST, SRC, \$ZERO} ($DST=SRC+0$), которая делает тоже самое.
Очевидно, разработчики MIPS хотели сделать как можно более компактную таблицу опкодов.
Это не значит, что сложение происходит во время каждой инструкции \INS{MOVE}.
Скорее всего, эти псевдоинструкции оптимизируются в \ac{CPU} и \ac{ALU} никогда не используется.

\myindex{MIPS!\Instructions!J}
\INS{J} в строке 24 делает переход по адресу в \ac{RA}, и это работает как выход из функции.
\INS{ADDIU} после \INS{J} на самом деле исполняется перед \INS{J} (помните о \IT{branch delay slots}?) 
и это часть эпилога функции.

Вот листинг сгенерированный \IDA. Каждый регистр имеет свой псевдоним:

\lstinputlisting[caption=\Optimizing GCC 4.4.5 (\IDA),numbers=left,style=customasmMIPS]{patterns/01_helloworld/MIPS/hw_O3_IDA_RU.lst}

Инструкция в строке 15 сохраняет GP в локальном стеке. Эта инструкция мистическим образом отсутствует
в листинге от GCC, может быть из-за ошибки в самом GCC\footnote{Очевидно, функция вывода листингов не так критична
для пользователей GCC, поэтому там вполне могут быть неисправленные ошибки.}.
Значение GP должно быть сохранено, потому что всякая функция может работать со своим собственным окном данных
размером 64KiB.
Регистр, содержащий адрес функции \puts называется \$T9, потому что регистры с префиксом T- называются
\q{temporaries} и их содержимое можно не сохранять.

\subsubsection{\NonOptimizing GCC}

\NonOptimizing GCC более многословный.

\lstinputlisting[caption=\NonOptimizing GCC 4.4.5 (\assemblyOutput),numbers=left,style=customasmMIPS]{patterns/01_helloworld/MIPS/hw_O0_RU.s}

Мы видим, что регистр FP используется как указатель на фрейм стека.
Мы также видим 3 \ac{NOP}-а.
Второй и третий следуют за инструкциями перехода.
Видимо, компилятор GCC всегда добавляет \ac{NOP}-ы (из-за \IT{branch delay slots})
после инструкций переходов и затем, если включена оптимизация, от них может избавляться.
Так что они остались здесь.

Вот также листинг от \IDA:

\lstinputlisting[caption=\NonOptimizing GCC 4.4.5 (\IDA),numbers=left,style=customasmMIPS]{patterns/01_helloworld/MIPS/hw_O0_IDA_RU.lst}

\myindex{MIPS!\Pseudoinstructions!LA}
Интересно что \IDA распознала пару инструкций \INS{LUI}/\INS{ADDIU} и собрала их в одну псевдоинструкцию 
\INS{LA} (\q{Load Address}) в строке 15.
Мы также видим, что размер этой псевдоинструкции 8 байт!
Это псевдоинструкция (или \IT{макрос}), потому что это не настоящая инструкция MIPS, а скорее
просто удобное имя для пары инструкций.

\myindex{MIPS!\Pseudoinstructions!NOP}
\myindex{MIPS!\Instructions!OR}
Ещё кое что: \IDA не распознала \ac{NOP}-инструкции в строках 22, 26 и 41.

Это \TT{OR \$AT, \$ZERO}.
По своей сути это инструкция, применяющая операцию \IT{ИЛИ} к содержимому регистра \$AT с нулем,
что, конечно же, холостая операция.
MIPS, как и многие другие \ac{ISA}, не имеет отдельной \ac{NOP}-инструкции.

\subsubsection{Роль стекового фрейма в этом примере}

Адрес текстовой строки передается в регистре.
Так зачем устанавливать локальный стек?
Причина в том, что значения регистров \ac{RA} и GP должны быть сохранены где-то
(потому что вызывается \printf) и для этого используется локальный стек.

Если бы это была \gls{leaf function}, тогда можно было бы избавиться от пролога и эпилога функции. Например:
 \myref{MIPS_leaf_function_ex1}.

\subsubsection{\Optimizing GCC: загрузим в GDB}

\myindex{GDB}
\lstinputlisting[caption=пример сессии в GDB]{patterns/01_helloworld/MIPS/O3_GDB.txt}

}
\ITA{\subsection{MIPS}

\subsubsection{Qualche parola sul \q{global pointer}}
\label{MIPS_GP}

\myindex{MIPS!\GlobalPointer}

Un importante concetto MIPS e' il \q{global pointer}.
Come potremmo gia' sapere, ogni ustrizione MIPS ha lunghezza pari a 32 bit, quindi e' impossibile inserire un indirizzo a 32-bit
in una sola istruzione: deve essere usata una coppia (come ha fatto GCC nell'esempio per il caricamento dell'indirizzo della stringa).
E' comunque possibile caricare dati da un indirizzo nell'intervallo $register-32768...register+32767$ utilizzando una singola istruzione
(perche' 16 bit di un signed offset possono essere codificati in una singola istruzione).
Possiamo quindi allocare per questo scopo un registro e allocare anche un'area a 64KiB per i dati piu' usati.
Questo registro dedicato e' detto \q{global pointer} e punta in mezzo dall'area di 64KiB.
Questa area solitamente contiene variabili globali e indirizzi di funzioni importate come \printf, perche' gli sviluppatori di GCC hanno deciso
che il recupero dell'indirizzo di una funzione deve essere veloce tanto quanto l'esecuzione di una singola istruzione invece di due.

In un file ELF questa area di 64KiB e' collocata parzialmente nelle sezioni .sbss (\q{small \ac{BSS}}) per dati non inizializzati
e .sdata (\q{small data}) per dati inizializzati.

Cio' implica che il programmatore puo' scegliere a quale dati si possa accedere piu' velocemente e piazzarli nelle sezioni .sdata/.sbss.
Alcuni programmatori old-school potrebbero ricordarsi del memory model MS-DOS \myref{8086_memory_model} 
o dei memory manger MS-DOS come XMS/EMS, in cui tutta la memoria era divisa in blocchi da 64KiB.

\myindex{PowerPC}

Questo concetto non e' unicamente di MIPS. Anche PowerPC usa la stessa tecnica.

\subsubsection{\Optimizing GCC}

Consideriamo il seguente esempio che illustra il concetto di \q{global pointer}.

\lstinputlisting[caption=\Optimizing GCC 4.4.5 (\assemblyOutput),numbers=left,style=customasmMIPS]{patterns/01_helloworld/MIPS/hw_O3_EN.s}

Come possiamo vedere, il registro \$GP e' settato nel prologo della funzione affinche' punti nel mezzo di questa area.
Il registro \ac{RA} viene anche salvato sullo stack locale.
\puts e' usata anche qui al posto di \printf.
\myindex{MIPS!\Instructions!LW}
L'indirizzo della funzione \puts e' caticato in \$25 usando \INS{LW} , l'istruzione (\q{Load Word}).
\myindex{MIPS!\Instructions!LUI}
\myindex{MIPS!\Instructions!ADDIU}
Successivamente l'indirizzo della stringa viene caricato in \$4 usando la coppia di istruzioni \INS{LUI} (\q{Load Upper Immediate}) e 
\INS{ADDIU} (\q{Add Immediate Unsigned Word}).
\INS{LUI} setta i 16 bit alti del registro (da cui la parola \q{upper} nel nome dell'istruzione) e \INS{ADDIU} aggiunge
i 16 bit piu' bassi dell'indirizzo.

\INS{ADDIU} segue \INS{JALR} (ti ricordi dei \IT{branch delay slots}?).
Il registro \$4 e' anche detto \$A0, e' usato per passare il primo argomento di una funzione
\footnote{La tabella dei registri MIPS e' riportata in appendice \myref{MIPS_registers_ref}}.

\myindex{MIPS!\Instructions!JALR}

\INS{JALR} (\q{Jump and Link Register}) salta all'indirizzo memorizzato nel registro \$25 register (indirizzo di \puts) 
salvando l'indirizzo della prossima istruzione (LW) in \ac{RA}.
Questo e' molto simile ad ARM.
Oh, e una cosa importate e' che l'indirizzo salvato in \ac{RA} non e' l'indirizzo della prossima istruzione (perche' e' in un 
\IT{delay slot} e viene eseguito prima prima dell'istruzione jump),
ma l'indirizzo dell'istruzione dopo la prossima (dopo il \IT{delay slot}).

Quindi, $PC + 8$ viene scritto in \ac{RA} durante l'esecuzione di \TT{JALR}, nel nostro caso, questo e' l'indirizzo dell'istruzione 
\INS{LW} successiva a \INS{ADDIU}.

\INS{LW} (\q{Load Word}) a riga 20 ripristina \ac{RA} dallo stack locale (questa istruzione e' in effetti parte
dell'epilogo della funzione).

\myindex{MIPS!\Pseudoinstructions!MOVE}

\INS{MOVE} a riga 22 copia il valore dal registro \$0 (\$ZERO) al \$2 (\$V0).
\label{MIPS_zero_register}

MIPS ha un registro \IT{costante}, il cui valore e' sempre zero.
Apparentemente, gli sviluppatori MIPS hanno pensato che zero e' la costante piu' usata in programmazione, quindi usiamo il registro \$0
 ogni volta che serve il valore zero.

Un altro fatto interessante e' che in MIPS non c'e' un'istruzione che trasferisce dati tra registri.
Infatti, \TT{MOVE DST, SRC} e' \TT{ADD DST, SRC, \$ZERO} ($DST=SRC+0$), che fa la stessa cosa.
Apparentemente gli sviluppatori MIPS desideravano avere una tabella di opcode compatta.
Questo non significa che un'addizione si verifichi per ogni istruzione \INS{MOVE}.
Molto probabilmente, la \ac{CPU} ottimizza queste pseudoistruzioni e la \ac{ALU} non viene mai usata.

\myindex{MIPS!\Instructions!J}

\INS{J} a riga 24 salta all'indirizzo in \ac{RA}, effettuando di fatti il ritorno dalla funzione.
\INS{ADDIU} dopo \INS{J} e' in effetti eseguita prima di J (ricordi i \IT{branch delay slots}?) e fa parte dell'epilogo della funzione.
Ecco anche il listato generato da \IDA. Ogni registro qui ha il suo pseudonimo:

\lstinputlisting[caption=\Optimizing GCC 4.4.5 (\IDA),numbers=left,style=customasmMIPS]{patterns/01_helloworld/MIPS/hw_O3_IDA_EN.lst}

L'istruzione alla riga 15 salva il valore di GP sullo stack locale, e questa istruzione manca misteriosamente dal listato prodotto da GCC, forse per un errore di GCC
\footnote{Apparentemente, le funzioni che generano i listati non sono fondamentali per gli utenti GCC, quindi qualche errore
non ancora corretto puo' esserci.}.
Il valore di GP deve essere infatti salvato, perche' ogni funzione puo' usare la sua finestra dati da 64KiB.
Il registro contenente l'indirizzo di \puts e' chiamato \$T9, perche' i registri con il prefisso T- sono detti 
\q{temporaries} ed il loro contenuto puo' non essere preservato.

\subsubsection{\NonOptimizing GCC}

\NonOptimizing GCC e' piu' verboso.

\lstinputlisting[caption=\NonOptimizing GCC 4.4.5 (\assemblyOutput),numbers=left,style=customasmMIPS]{patterns/01_helloworld/MIPS/hw_O0_EN.s}

Qui vediamo che il registro FP e' usato come un puntatore allo stack frame.
Vediamo anche 3 \ac{NOP}s.
Di cui il secondo e terzo seguono all'istruzione branch.
Forse GCC aggiunge sempre \ac{NOP}s (a causa dei \IT{branch delay slots}) dopo le istruzioni branch
e successivamente, se le ottimizzazioni sono attivate, forse li elimina.
Quindi in questo caso sono rimasti.

Ecco anche il listato \IDA:

\lstinputlisting[caption=\NonOptimizing GCC 4.4.5 (\IDA),numbers=left,style=customasmMIPS]{patterns/01_helloworld/MIPS/hw_O0_IDA_EN.lst}

\myindex{MIPS!\Pseudoinstructions!LA}

E' interessante notare che \IDA ha riconosciuto la coppia di istruzioni \INS{LUI}/\INS{ADDIU} e le ha fusae in un'unica pseudoistruzione 
\INS{LA} (\q{Load Address}) a riga 15.
Possiamo anche vedere che questa pseudoistruzione e' lunga 8 bytes!
Questa e' una pseudoistruzione (o \IT{macro}) in quanto non e' una vera istruzione MIPS , ma soltanto un nome comodo per una coppia
di istruzioni.

\myindex{MIPS!\Pseudoinstructions!NOP}
\myindex{MIPS!\Instructions!OR}

Un'altra cosa e' che \IDA non ha riconosciuto le istruzioni \ac{NOP} , che sono alle righe 22, 26 e 41.
E' \TT{OR \$AT, \$ZERO}.
Essenzialmente, questa istruzione applica l'operazione OR al contenuto del registro \$AT 
con zero, che e', ovviamente, un'istruzione nulla/inutile.
MIPS, come molte altre \ac{ISA}, non ha un'istruzione \ac{NOP} propria.

\subsubsection{Ruolo dello the stack frame in questo esempio}

L'indirizzo della stringa e' passato nel registro.
Perche' impostare allora ugualmente uno stack locale?
La ragione sta nel fatto che i valori dei registri \ac{RA} e GP devono essere salvati da qualche parte 
(poiche' viene chiamata \printf ), e lo stack locale e' usato proprio per questo scopo.
Se fosse stata una \gls{leaf function}, sarebbe stato possibile fare a meno (disfarsi) del prologo e dell'epilogo,
ad esempio: \myref{MIPS_leaf_function_ex1}.

\subsubsection{\Optimizing GCC: carichiamolo in GDB}

\myindex{GDB}
\lstinputlisting[caption=sample GDB session]{patterns/01_helloworld/MIPS/O3_GDB.txt}
}
\DE{\subsection{MIPS}

\subsubsection{ein Wort über \q{globale Zeiger}}
\label{MIPS_GP}

\myindex{MIPS!\GlobalPointer}

Ein wichtiges Konzept bei MIPS ist der \q{globale Zeiger}.
Wie bereits bekannt, besteht besteht jede MIPS-Anweisung aus 32 Bit, so dass es
nicht möglich ist eine 32-Bit-Anweisung darin unterzubringen: ein Anweisungspaar
wird verwendet (wie GCC dies in dem Beispiel zum Laden der Zeichenkettenadresse
getan hat).

Es ist jedoch möglich, Daten aus dem Adressbereich $register-32768...register+32767$
mit nur einer Anweisung zuladen, weil ein 16 Bit vorzeichenbehafteter Offset
in einer einzelnen Anweisung kodiert werden kann.
Es können also einige Register zu diesen Zweck alloziert werden und 64KiB-Bereiche
für die am häufigsten genutzten Daten
Dieses allozierte Register wird \q{globaler Zeiger} genannt und zeigt in die Mitte
des 64KiB-Bereichs.

Dieser Bereich enthält in der Regel globale Variablen und Adressen von importierten
Funktionen wie \printf, weil die GCC-Entwickler entschieden, dass das Laden einiger
Funktionsadressen so schnell sein sollte wie eine einzelne Anweisung anstatt zwei.
In einer ELF-Datei ist dieser 64KiB-Bereich teils in der Sektion .sbss (\q{small \ac{BSS}})
für uninitialiserte Daten und teil in .sdata (\q{small data}) für initialisierte
Daten zu finden.

Dies impliziert, dass der Programmierer entscheiden kann, auf welche Daten ein schneller
Zugriff (durch das Platzieren in .sdata/.sbss) möglich sein soll.
Einige Programmierer \q{der alten Schule} erinnern sich vielleicht an das MS-DOS
Speichermodell \myref{8086_memory_model} oder MS-DOS Speicherverwaltungen wie XMS/EMS
bei denen der komplette Speicher in 64KiB-Blöcke unterteilt war.

\myindex{PowerPC}

Dieses Konzept ist nicht nur bei MIPS vorhanden. Zumindest der PowerPC nutzt es ebenfalls.

\subsubsection{\Optimizing GCC}

Nachfolgen ein Beispiel welches das Konzept der \q{globalen Zeiger} veranschaulichen soll.

\lstinputlisting[caption=\Optimizing GCC 4.4.5 (\assemblyOutput),numbers=left,style=customasmMIPS]{patterns/01_helloworld/MIPS/hw_O3_DE.s}

Wie zu sehen wird das \$GP-Register im Funktionsprolog so gesetzt, dass auf die Mitte
dieses Bereichs gezeigt wird.
Das \ac{RA}-Register wird ebenfalls auf dem lokalen Stack gesichert.
Anstelle von \printf wird wieder \puts aufgerufen.
\myindex{MIPS!\Instructions!LW}
Die Adresse der Funktion \puts wird mit der \INS{LW}-Anweisung (\q{Load Word}) in \$25 geladen.
\myindex{MIPS!\Instructions!LUI}
\myindex{MIPS!\Instructions!ADDIU}
Anschließend wird die Adressse der Zeichenkette mit dem Anweisungspaar \INS{LUI} (\q{Load Upper Immediate})
und \INS{ADDIU} (\q{Add Immediate Unsigned Word}) in \$4 geladen.
\INS{LUI} setzt die  oberen 16 Bit des Registers (deswegen \q{upper} im Anweisungsnamen)
und \INS{ADDIU} addiert die unteren 16 Bit der Adresse.

\INS{ADDIU} folgt \INS{JALR} (zur Erinnerung: \IT{branch delay slots}).
Das Register \$4 wird auch \$A0 genannt und für das Übergeben des ersten Funktionsarguments
genutzt\footnote{Die MIPS-Register-Tabelle ist im Anhang verfügbar \myref{MIPS_registers_ref}}.

\myindex{MIPS!\Instructions!JALR}

\INS{JALR} (\q{Jump and Link Register}) springt zu der Adresse die im Register \$25
gespeichert ist (Adresse von \puts) und speichert die Adresse der übernächsten Anweisung
(LW) in \ac{RA}. Dies ist sehr ähnlich zu ARM.
Eine wichtige Sache ist, dass die Adresse in \ac{RA} nicht die Adresse der nächsten
Anweisung ist (da dies ein \IT{delay slot} ist und vor der Sprunganweisung ausgeführt wird),
sondern die Adresse der darauf folgenden Anweisung (nach dem \IT{delay slot}).
Da in diesem Fall während der Ausführung von \TT{JALR} der Wert $PC + 8$ in \ac{RA}
geschrieben wird, ist dies die Adresse der \INS{LW}-Anweisung nach \INS{ADDIU}.

\INS{LW} (\q{Load Word}) in Zeile 20 stellt \ac{RA} wieder vom lokalen Stack her.
Diese Anweisung ist tatsächlich ein Teil des Funktionsepilogs.

\myindex{MIPS!\Pseudoinstructions!MOVE}

\INS{MOVE} in Zeile 22 kopiert der Wert vom \$0 (\$ZERO)-Register in \$2 (\$V0).
\label{MIPS_zero_register}

MIPS besitzt ein \IT{konstantes} Register, welches immer eine Null beinhaltet.
Anscheinend hatten die MIPS-Entwickler die Idee, dass eine Null die beliebteste
Konstante in der Programmierung ist, also wird in Zukunft immer das \$0-Register
genutzt wenn eine Null benötigt wird.

Eine weitere interessante Tatsache in MIPS ist das Fehlen einer Anweisung zum Transferieren
von Daten zwischen zwei Registern.
Die Anweisung \TT{MOVE DST, SRC} entspricht jedoch \TT{ADD DST, SRC, \$ZERO} ($DST=SRC+0$),
und bewirkt genau das gleiche.
Anscheinend wollten die MIPS-Entwickler eine kompakte Opcode-Tabelle haben.
Das bedeutet nicht, dass dies bei jeder \INS{MOVE}-Anweisung passiert.
Sehr wahrscheinlich optimiert die \ac{CPU} diese Pseudo-Anweisung und die \ac{ALU}
wird niemals genutzt.

\myindex{MIPS!\Instructions!J}

\INS{J} in Zeile 24 springt zu der Adresse in \ac{RA}, was im Endeffekt einem Sprung aus einer Funktion entspricht.
\INS{ADDIU} nach \INS{J} wird tatsächlich bevor \INS{J} ausgeführt (siehe \IT{branch delay slots}) und ist
ein Teil des Funktions-Epilogs.
Hier ist die Ausgabe, die \IDA generiert. Jedes Register hat einen eigenen Pseudo-Namen:

\lstinputlisting[caption=\Optimizing GCC 4.4.5 (\IDA),numbers=left,style=customasmMIPS]{patterns/01_helloworld/MIPS/hw_O3_IDA_DE.lst}

Die Anweisung in Zeile 15 speichert den GP-Wert auf dem lokalen Stack. Diese Anweisung fällt seltsamerweise
beim GCC, was vielleicht auf einen Fehler des Compilers hinweist. \footnote{Anscheinend sind Funktionen die
Listings erzeugen nicht so kritisch für GCC-Nutzer, so dass vielleicht noch unbehobene Fehler existieren.}.
Der GP-Wert muss auch gespeichert werden weil jede Funktion ihren eigenen 64KiB-Datenbereich nutzen kann.
Das Register mit der Adresse von \puts wird \$T9, da Register mit Präfix T- temporäre Register sind deren
Inhalte nicht erhalten werden müssen.

\subsubsection{\NonOptimizing GCC}

\NonOptimizing GCC ist ausführlicher.

\lstinputlisting[caption=\NonOptimizing GCC 4.4.5 (\assemblyOutput),numbers=left,style=customasmMIPS]{patterns/01_helloworld/MIPS/hw_O0_DE.s}

Es ist zu sehen, dass das FP-Register als Zeiger zum Stack Frame genutzt wird.
Außerdem sind im Listing drei \ac{NOP}-Anweisungen.
Die zweite und dritte welche der Sprunganweisung folgt.
Möglicherweise fügt der GCC-Compiler immer \ac{NOP}-Anweisungen nach einer Sprung hinzu
(wegen der \IT{branch delay slots}) und entfernt diese wenn die Optimierung eingeschaltet ist.
In diesem Fall bleiben sie also bestehen.

Nachfolgend das \IDA-Listing:

\lstinputlisting[caption=\NonOptimizing GCC 4.4.5 (\IDA),numbers=left,style=customasmMIPS]{patterns/01_helloworld/MIPS/hw_O0_IDA_DE.lst}

\myindex{MIPS!\Pseudoinstructions!LA}

Interessanterweise kennt \IDA das Anweisungspaar \INS{LUI}/\INS{ADDIU} und fasst diese zu einer einzigen
Pseudoanweisung \INS{LA} (\q{Load Address}) zusammen (Zeile 15).
Es ist auch zu sehen, dass diese Pseudoanweisung eine Größe von 8 Byte hat!
Dies ist eine Pseudoanweisung (oder \IT{Makro}) weil es sich hier nicht um eine echte MIPS-Anweisung
handelt, sondern eher um einen handlichen Namen für ein Anweisungspaar.

\myindex{MIPS!\Pseudoinstructions!NOP}
\myindex{MIPS!\Instructions!OR}

Eine weitere Sache ist, dass \IDA keine \ac{NOP}-Anweisung kennt.
Also ist in den Zeilen 22, 26 und 41 \TT{OR \$AT, \$ZERO}.
Im Wesentlichen führt diese Anweisung eine ODER-Operation auf die Inhalte des \$AT-Register aus,
welche 0 ist. Dies entspricht natürlich einer Idle-Anweisung.
MIPS hat wie viele andere \ac{ISA} keine separate \ac{NOP}-Anweisung.

\subsubsection{Aufgabe des Stack Frames in diesem Beispiel}

Die Adresse dieser Zeichenkette ist in einem Register übergeben.
Warum wird dennoch der lokale Stack vorbereitet?
Der Grund dafür liegt in der Tatsache, dass die Werte der Register \ac{RA} und GP
wegen des Aufrugs von \printf irgendwo gesichert werden müssen und hier eben der
lokale Stack dafür genutzt wird.
Wenn dies eine \gls{leaf function} wäre, bestünde die Möglichkeit den Funktionsepilog
und -prolog wegzulassen, wie hier: \myref{MIPS_leaf_function_ex1}.

\subsubsection{\Optimizing GCC: in GDB laden}

\myindex{GDB}
\lstinputlisting[caption=sample GDB session]{patterns/01_helloworld/MIPS/O3_GDB.txt}
}
\FR{\subsection{MIPS}

\subsubsection{Un mot à propos du \q{pointeur global}}
\label{MIPS_GP}

\myindex{MIPS!\GlobalPointer}

Un concept MIPS important est le \q{pointeur global}.
Comme nous le savons déjà, chaque instruction MIPS a une taille de 32bits, donc
il est impossible d'avoir une adresse 32-bit dans une instruction: il faut pour
cela utiliser une paire.
(comme le fait GCC dans notre exemple pour le chargement de l'adresse de la chaîne
de texte).
Il est possible, toutefois, de charger des données depuis une adresse dans l'interval
$register-32768...register+32767$ en utilisant une seule instruction (car un offset
signé de 16 bits peut être encodé dans une seule instruction).
Nous pouvons alors allouer un registre dans ce but et dédier un block de 64KiB
pour les données les plus utilisées.
Ce registre dédié est appelé un \q{pointeur global} et il pointe au milieu du
block de 64 KiB.
Ce block contient en général les variables globales et les adresses des fonctions
importées, comme \printf, car les développeurs de GCC ont décidé qu'obtenir
l'adresse d'une fonction devait se faire en une instruction au lieu de deux.
Dans un fichier ELF ce block de 64KiB se trouve en partie dans une section .sbss
(\q{small \ac{BSS}}) pour les données non initialisées et .sdata (\q{small data})
pour celles initialisées.
Cela implique que le programmeur peut choisir quelle donnée il/elle souhaite rendre
accesible rapidement et doit les stocker dans .sdata/.sbss.
Certains programmeurs old-school peuvent se souvenir du modèle de mémoire MS-DOS
\myref{8086_memory_model} ou des gestionnaires de mémoire MS-DOS comme XMS/EMS
où toute la mémoire était divisée en bloc de 64KiB.

\myindex{PowerPC}

Ce concept n'est pas restreint à MIPS. Au moins les PowerPC utilisent aussi cette
technique.

\subsubsection{GCC \Optimizing}

Considérons l'exemple suivant, qui illustre le concept de \q{pointeur global}.

\lstinputlisting[caption=GCC 4.4.5 \Optimizing (\assemblyOutput),numbers=left,style=customasmMIPS]{patterns/01_helloworld/MIPS/hw_O3_FR.s}

Comme on le voit, le registre \$GP est défini dans le prologue de la fonction
pour pointer au milieu de ce block.
Le registre \ac{RA} est sauvé sur la pile locale.
\puts est utilisé ici au lieu de \printf.
\myindex{MIPS!\Instructions!LW}
L'adresse de la fonction \puts est chargée dans \$25 en utilisant l'instruction \INS{LW} (\q{Load Word}).
\myindex{MIPS!\Instructions!LUI}
\myindex{MIPS!\Instructions!ADDIU}
Ensuite l'adresse de la chaîne de texte est chargée dans \$4 avec la paire
d'instructions \INS{LUI} ((\q{Load Upper Immediate}) et \INS{ADDIU}
(\q{Add Immediate Unsigned Word}).
\INS{LUI} défini les 16 bits de poids fort du registre (d'où le mot \q{upper}
dans le nom de l'instruction) et \INS{ADDIU} ajoute les 16 bits de poids faible
de l'adresse.

\INS{ADDIU} suit \INS{JALR} (vous n'avez pas déjà oublié le \IT{slot de
retard de branchement} ?).
Le registre \$4 est aussi appelé \$A0, qui est utilisé pour passer le premier
argument d'une fonction \footnote{La table des registres MIPS est disponible en
appendice \myref{MIPS_registers_ref}}.

\myindex{MIPS!\Instructions!JALR}

\INS{JALR} (\q{Jump and Link Register}) saute à l'adresse stockée dans le registre
\$25 (adresse de \puts) en sauvant l'adresse de la prochaine instruction (LW)
dans \ac{RA}.
C'est très similaire à ARM.
Oh, encore une chose importante, l'adresse sauvée dans \ac{RA} n'est pas
l'adresse de l'instruction suivante (car c'est celle du \IT{slot de délai} et
elle est exécutée avant l'instruction de saut), mais l'adresse de l'instruction
après la suivante (après le \IT{slot de délai}).
Par conséquent, $PC + 8$ est écrit dans \ac{RA} pendant l'exécution de \TT{JALR},
dans notre cas, c'est l'adresse de l'instruction \INS{LW} après \INS{ADDIU}.

\INS{LW} (\q{Load Word}) à la ligne 20 restaure \ac{RA} depuis la pile locale
(cette instruction fait partie de l'épilogue de la fonction).

\myindex{MIPS!\Pseudoinstructions!MOVE}

\INS{MOVE} à la ligne 22 copie la valeur du registre \$0 (\$ZERO) dans \$2 (\$V0).
\label{MIPS_zero_register}

MIPS a un registre \IT{constant}, qui contient toujours zéro.
Apparemment, les développeurs de MIPS avaient à l'esprit que zéro est la constante
la plus utilisée en programmation, utilisons donc le registre \$0 à chaque fois
que zéro est requis.

Un autre fait intéressant est qu'il manque en MIPS une instruction qui transfère
des données entre des registres.
En fait,  \TT{MOVE DST, SRC} est \TT{ADD DST, SRC, \$ZERO} ($DST=SRC+0$), qui
fait la même chose.
Apparemment, les développeurs de MIPS voulaient une table des opcodes compacte.
Cela ne siginifie pas qu'il y a une addition à chaque instruction \INS{MOVE}.
Très probablement, le \ac{CPU} optimise ces pseudo instructions et l'\ac{ALU} n'est
jamais utilisé.

\myindex{MIPS!\Instructions!J}

\INS{J} à la ligne 24 saute à l'adresse dans \ac{RA}, qui effectue effectivement
un retour de la fonction.
\INS{ADDIU} après \INS{J} est en fait exécutée avant \INS{J} (vous vous rappeler
du \IT{slot de délai de branchement}?) et fait partie de l'épilogue de la fonction.
Voici un listing généré par \IDA. Chaque registre a son propre pseudo nom:

\lstinputlisting[caption=GCC 4.4.5 \Optimizing (\IDA),numbers=left,style=customasmMIPS]{patterns/01_helloworld/MIPS/hw_O3_IDA_FR.lst}

L'instruction à la ligne 15 sauve la valeur de GP sur la pile locale, et cette
instruction manque mystérieusement dans le listing de sortie de GCC, peut-être
une erreur de GCC
\footnote{Apparamment, les fonctions générant les listings ne sont pas si critique
pour les utilisateurs de GCC, donc des erreurs peuvent toujours subsister.}.
La valeur de GP doit effectivement être sauvée, car chaque fonction utilise sa
propre fenêtre de 64KiB.
Le registre contenant l'adresse de \puts est appelé \$T9, car les registres
préfixés avec T- sont appelés \q{temporaires} et leur contenu ne doit pas être
préservé. 

\subsubsection{GCC \NonOptimizing}

GCC \NonOptimizing est plus verbeux.

\lstinputlisting[caption=GCC 4.4.5 \NonOptimizing (\assemblyOutput),numbers=left,style=customasmMIPS]{patterns/01_helloworld/MIPS/hw_O0_FR.s}

Nous voyons ici que le registre FP est utilisé comme un pointeur sur la pile.
Nous voyons aussi 3 \ac{NOP}s.
Le second et le troisième suivent une instruction de branchement.
Peut-être que le compilateur GCC ajoute toujours des \ac{NOP}s (à cause du
\IT{slot de retard de branchement}) après les instructions de branchement, et
alors, si l'optimisation est demandée, peut-être qu'il les élimine.
Donc, dans ce cas, ils sont laissés en place.

Voici le listing \IDA:

\lstinputlisting[caption=GCC 4.4.5 \NonOptimizing (\IDA),numbers=left,style=customasmMIPS]{patterns/01_helloworld/MIPS/hw_O0_IDA_FR.lst}

\myindex{MIPS!\Pseudoinstructions!LA}

Intéressant, \IDA a reconnu les intructions \INS{LUI}/\INS{ADDIU} et les a concaténées
en une pseudo instruction \INS{LA} (\q{Load Address}) à la ligne 15.
Nous pouvons voir que cette pseudo instruction a une taille de 8 octets!
C'est une pseudo intruction (ou \IT{macro}) car ce n'est pas une instruction MIPS
réelle, mais plutôt un nom pratique pour une paire d'instructions.

\myindex{MIPS!\Pseudoinstructions!NOP}
\myindex{MIPS!\Instructions!OR}

Une autre chose est qu'\IDA ne reconnait pas les instructions \ac{NOP}, donc ici
elles se trouvent aux lignes 22, 26 et 41.
C'est \TT{OR \$AT, \$ZERO}.
Essentiellement, cette instruction applique l'opération OR au contenu du registre
\$AT avec zéro, ce qui, bien sûr, est une instruction sans effet.
MIPS, comme beaucoup d'autres \ac{ISA}s, n'a pas une instruction \ac{NOP}.

\subsubsection{Rôle de la pile dans cet exemple}

L'adresse de la chaîne de texte est passée dans le registre.
Pourquoi définir une pile locale quand même?
La raison de cela est que la valeur des registres \ac{RA} et GP doit être sauvée
quelque part (car \printf est appelée), et que la pile locale est utilisée pour cela.
Si cela avait été une \glslink{leaf function}{fonction leaf}, il aurait été
possible de se passer du prologue et de l'épilogue de la fonction, par
exemple: \myref{MIPS_leaf_function_ex1}.

\subsubsection{GCC \Optimizing: chargeons-le dans GDB}

\myindex{GDB}
\lstinputlisting[caption=extrait d'une session GDB]{patterns/01_helloworld/MIPS/O3_GDB.txt}

}


\subsection{\Conclusion{}}

Het grootste verschil tussen x86/ARM en x64/ARM64 code is dat de pointer naar de string nu 64-bits in lengte is.
De meeste moderne \ac{CPU}s zijn tegenwoordig 64-bit wegens zowel de verminderde gebruik van geheugen, als de grote vraag ervoor door moderne applicaties.
We kunnen hierdoor veel meer geheugen aan onze computers toevoegen dan dat 32-bit pointers kunnen aanspreken.
Bijgevolg zijn alle pointers nu 64-bit.

% sections
\subsection{\Exercises}

\begin{itemize}
	\item \url{http://challenges.re/48}
	\item \url{http://challenges.re/49}
\end{itemize}


