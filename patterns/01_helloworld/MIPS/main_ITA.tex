\subsection{MIPS}

\subsubsection{Qualche parola sul \q{global pointer}}
\label{MIPS_GP}

\myindex{MIPS!\GlobalPointer}

Un importante concetto MIPS e' il \q{global pointer}.
Come potremmo gia' sapere, ogni ustrizione MIPS ha lunghezza pari a 32 bit, quindi e' impossibile inserire un indirizzo a 32-bit
in una sola istruzione: deve essere usata una coppia (come ha fatto GCC nell'esempio per il caricamento dell'indirizzo della stringa).
E' comunque possibile caricare dati da un indirizzo nell'intervallo $register-32768...register+32767$ utilizzando una singola istruzione
(perche' 16 bit di un signed offset possono essere codificati in una singola istruzione).
Possiamo quindi allocare per questo scopo un registro e allocare anche un'area a 64KiB per i dati piu' usati.
Questo registro dedicato e' detto \q{global pointer} e punta in mezzo dall'area di 64KiB.
Questa area solitamente contiene variabili globali e indirizzi di funzioni importate come \printf, perche' gli sviluppatori di GCC hanno deciso
che il recupero dell'indirizzo di una funzione deve essere veloce tanto quanto l'esecuzione di una singola istruzione invece di due.

In un file ELF questa area di 64KiB e' collocata parzialmente nelle sezioni .sbss (\q{small \ac{BSS}}) per dati non inizializzati
e .sdata (\q{small data}) per dati inizializzati.

Cio' implica che il programmatore puo' scegliere a quale dati si possa accedere piu' velocemente e piazzarli nelle sezioni .sdata/.sbss.
Alcuni programmatori old-school potrebbero ricordarsi del memory model MS-DOS \myref{8086_memory_model} 
o dei memory manger MS-DOS come XMS/EMS, in cui tutta la memoria era divisa in blocchi da 64KiB.

\myindex{PowerPC}

Questo concetto non e' unicamente di MIPS. Anche PowerPC usa la stessa tecnica.

\subsubsection{\Optimizing GCC}

Consideriamo il seguente esempio che illustra il concetto di \q{global pointer}.

\lstinputlisting[caption=\Optimizing GCC 4.4.5 (\assemblyOutput),numbers=left,style=customasmMIPS]{patterns/01_helloworld/MIPS/hw_O3_EN.s}

Come possiamo vedere, il registro \$GP e' settato nel prologo della funzione affinche' punti nel mezzo di questa area.
Il registro \ac{RA} viene anche salvato sullo stack locale.
\puts e' usata anche qui al posto di \printf.
\myindex{MIPS!\Instructions!LW}
L'indirizzo della funzione \puts e' caticato in \$25 usando \INS{LW} , l'istruzione (\q{Load Word}).
\myindex{MIPS!\Instructions!LUI}
\myindex{MIPS!\Instructions!ADDIU}
Successivamente l'indirizzo della stringa viene caricato in \$4 usando la coppia di istruzioni \INS{LUI} (\q{Load Upper Immediate}) e 
\INS{ADDIU} (\q{Add Immediate Unsigned Word}).
\INS{LUI} setta i 16 bit alti del registro (da cui la parola \q{upper} nel nome dell'istruzione) e \INS{ADDIU} aggiunge
i 16 bit piu' bassi dell'indirizzo.

\INS{ADDIU} segue \INS{JALR} (ti ricordi dei \IT{branch delay slots}?).
Il registro \$4 e' anche detto \$A0, e' usato per passare il primo argomento di una funzione
\footnote{La tabella dei registri MIPS e' riportata in appendice \myref{MIPS_registers_ref}}.

\myindex{MIPS!\Instructions!JALR}

\INS{JALR} (\q{Jump and Link Register}) salta all'indirizzo memorizzato nel registro \$25 register (indirizzo di \puts) 
salvando l'indirizzo della prossima istruzione (LW) in \ac{RA}.
Questo e' molto simile ad ARM.
Oh, e una cosa importate e' che l'indirizzo salvato in \ac{RA} non e' l'indirizzo della prossima istruzione (perche' e' in un 
\IT{delay slot} e viene eseguito prima prima dell'istruzione jump),
ma l'indirizzo dell'istruzione dopo la prossima (dopo il \IT{delay slot}).

Quindi, $PC + 8$ viene scritto in \ac{RA} durante l'esecuzione di \TT{JALR}, nel nostro caso, questo e' l'indirizzo dell'istruzione 
\INS{LW} successiva a \INS{ADDIU}.

\INS{LW} (\q{Load Word}) a riga 20 ripristina \ac{RA} dallo stack locale (questa istruzione e' in effetti parte
dell'epilogo della funzione).

\myindex{MIPS!\Pseudoinstructions!MOVE}

\INS{MOVE} a riga 22 copia il valore dal registro \$0 (\$ZERO) al \$2 (\$V0).
\label{MIPS_zero_register}

MIPS ha un registro \IT{costante}, il cui valore e' sempre zero.
Apparentemente, gli sviluppatori MIPS hanno pensato che zero e' la costante piu' usata in programmazione, quindi usiamo il registro \$0
 ogni volta che serve il valore zero.

Un altro fatto interessante e' che in MIPS non c'e' un'istruzione che trasferisce dati tra registri.
Infatti, \TT{MOVE DST, SRC} e' \TT{ADD DST, SRC, \$ZERO} ($DST=SRC+0$), che fa la stessa cosa.
Apparentemente gli sviluppatori MIPS desideravano avere una tabella di opcode compatta.
Questo non significa che un'addizione si verifichi per ogni istruzione \INS{MOVE}.
Molto probabilmente, la \ac{CPU} ottimizza queste pseudoistruzioni e la \ac{ALU} non viene mai usata.

\myindex{MIPS!\Instructions!J}

\INS{J} a riga 24 salta all'indirizzo in \ac{RA}, effettuando di fatti il ritorno dalla funzione.
\INS{ADDIU} dopo \INS{J} e' in effetti eseguita prima di J (ricordi i \IT{branch delay slots}?) e fa parte dell'epilogo della funzione.
Ecco anche il listato generato da \IDA. Ogni registro qui ha il suo pseudonimo:

\lstinputlisting[caption=\Optimizing GCC 4.4.5 (\IDA),numbers=left,style=customasmMIPS]{patterns/01_helloworld/MIPS/hw_O3_IDA_EN.lst}

L'istruzione alla riga 15 salva il valore di GP sullo stack locale, e questa istruzione manca misteriosamente dal listato prodotto da GCC, forse per un errore di GCC
\footnote{Apparentemente, le funzioni che generano i listati non sono fondamentali per gli utenti GCC, quindi qualche errore
non ancora corretto puo' esserci.}.
Il valore di GP deve essere infatti salvato, perche' ogni funzione puo' usare la sua finestra dati da 64KiB.
Il registro contenente l'indirizzo di \puts e' chiamato \$T9, perche' i registri con il prefisso T- sono detti 
\q{temporaries} ed il loro contenuto puo' non essere preservato.

\subsubsection{\NonOptimizing GCC}

\NonOptimizing GCC e' piu' verboso.

\lstinputlisting[caption=\NonOptimizing GCC 4.4.5 (\assemblyOutput),numbers=left,style=customasmMIPS]{patterns/01_helloworld/MIPS/hw_O0_EN.s}

Qui vediamo che il registro FP e' usato come un puntatore allo stack frame.
Vediamo anche 3 \ac{NOP}s.
Di cui il secondo e terzo seguono all'istruzione branch.
Forse GCC aggiunge sempre \ac{NOP}s (a causa dei \IT{branch delay slots}) dopo le istruzioni branch
e successivamente, se le ottimizzazioni sono attivate, forse li elimina.
Quindi in questo caso sono rimasti.

Ecco anche il listato \IDA:

\lstinputlisting[caption=\NonOptimizing GCC 4.4.5 (\IDA),numbers=left,style=customasmMIPS]{patterns/01_helloworld/MIPS/hw_O0_IDA_EN.lst}

\myindex{MIPS!\Pseudoinstructions!LA}

E' interessante notare che \IDA ha riconosciuto la coppia di istruzioni \INS{LUI}/\INS{ADDIU} e le ha fusae in un'unica pseudoistruzione 
\INS{LA} (\q{Load Address}) a riga 15.
Possiamo anche vedere che questa pseudoistruzione e' lunga 8 bytes!
Questa e' una pseudoistruzione (o \IT{macro}) in quanto non e' una vera istruzione MIPS , ma soltanto un nome comodo per una coppia
di istruzioni.

\myindex{MIPS!\Pseudoinstructions!NOP}
\myindex{MIPS!\Instructions!OR}

Un'altra cosa e' che \IDA non ha riconosciuto le istruzioni \ac{NOP} , che sono alle righe 22, 26 e 41.
E' \TT{OR \$AT, \$ZERO}.
Essenzialmente, questa istruzione applica l'operazione OR al contenuto del registro \$AT 
con zero, che e', ovviamente, un'istruzione nulla/inutile.
MIPS, come molte altre \ac{ISA}, non ha un'istruzione \ac{NOP} propria.

\subsubsection{Ruolo dello the stack frame in questo esempio}

L'indirizzo della stringa e' passato nel registro.
Perche' impostare allora ugualmente uno stack locale?
La ragione sta nel fatto che i valori dei registri \ac{RA} e GP devono essere salvati da qualche parte 
(poiche' viene chiamata \printf ), e lo stack locale e' usato proprio per questo scopo.
Se fosse stata una \gls{leaf function}, sarebbe stato possibile fare a meno (disfarsi) del prologo e dell'epilogo,
ad esempio: \myref{MIPS_leaf_function_ex1}.

\subsubsection{\Optimizing GCC: carichiamolo in GDB}

\myindex{GDB}
\lstinputlisting[caption=sample GDB session]{patterns/01_helloworld/MIPS/O3_GDB.txt}
