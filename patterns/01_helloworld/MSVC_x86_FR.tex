\subsubsection{MSVC}

Compilons-le avec MSVC 2010:

\begin{lstlisting}
cl 1.cpp /Fa1.asm
\end{lstlisting}

(L'option \TT{/Fa} indique au compilateur de générer un fichier avec le listing en assembleur)

\begin{lstlisting}[caption=MSVC 2010,style=customasmx86]
CONST	SEGMENT
$SG3830	DB	'hello, world', 0AH, 00H
CONST	ENDS
PUBLIC	_main
EXTRN	_printf:PROC
; Function compile flags: /Odtp
_TEXT	SEGMENT
_main	PROC
	push	ebp
	mov	ebp, esp
	push	OFFSET $SG3830
	call	_printf
	add	esp, 4
	xor	eax, eax
	pop	ebp
	ret	0
_main	ENDP
_TEXT	ENDS
\end{lstlisting}

MSVC génère des listings assembleur avec la syntaxe Intel.
La différence entre la syntaxe Intel et la syntaxe AT\&T sera discutée dans \myref{ATT_syntax}.

Le compilateur a généré le fichier object \TT{1.obj}, qui sera lié dans l'exécutable \TT{1.exe}.
Dans notre cas, le fichier contient deux segments: \TT{CONST} (pour les données constantes)
 et \TT{\_TEXT} (pour le code).

\myindex{\CLanguageElements!const}
\label{string_is_const_char}
La chaîne \TT{hello, world} en \CCpp a le type \TT{const char[]}\InSqBrackets{\TCPPPL p176, 7.3.2}, mais
elle n'a pas de nom en propre.
Le compilateur doit pouvoir l'utiliser et lui défini donc un nom interne \TT{\$SG3830} à cette fin.

C'est pourquoi l'exemple pourrait être récrit comme suit:

\lstinputlisting[style=customc]{patterns/01_helloworld/hw_2.c}

Retournons au listing assembleur. Comme nous le voyons, la chaîne est terminée avec un octet à zéro, ce qui
est le standard pour les chaînes \CCpp.

Dans le segment de code, \TT{\_TEXT}, il n'y a qu'une seule fonction: \main{}.
La fonction \main débute par le code du prologue et se termine par le code de l'épilogue 
(comme presque toutes les fonctions)
\footnote{Vous pouvez en lire plus dans la section concerant les prologues et épilogues de
fonctions ~(\myref{sec:prologepilog}).}.

\myindex{x86!\Instructions!CALL}
Après le prologue de la fonction nous voyons l'appel à la fonction \printf{}:\\
\INS{CALL \_printf}.
\myindex{x86!\Instructions!PUSH}
Avant l'appel, l'adresse de la chaîne (ou un pointeur sur elle) contenant notre message
 est placée sur la pile avec l'aide de l'instruction \PUSH.

Lorsque la fonction \printf rend le contrôle à la fonction \main, l'adresse de la chaîne (ou un pointeur sur elle)
est toujours sur la pile.
Comme nous n'en avons plus besoin, le pointeur de pile (\gls{stack pointer} le registre \ESP) doit être corrigé.

\myindex{x86!\Instructions!ADD}
\INS{ADD ESP, 4} signifie ajouter 4 à la valeur du registre \ESP.

Pourquoi 4? puisqu'il s'agit d'un programme 32-bit, nous avons besoin d'exactement 4 octets pour passer une adresse
par la pile. S'il s'agissait d'un code x64, nous aurions besoin de 8 octets.
\INS{ADD ESP, 4} est effectivement équivalent à \INS{POP register} mais sans utiliser de registre\footnote{Les
flags du CPU, toutefois, sont modifiés}.

\myindex{Intel C++}
\myindex{\oracle}
\myindex{x86!\Instructions!POP}

Pour la même raison, certains compilateurs (comme le compilateur C++ d'Intel) peuvent générer \TT{POP ECX}
à la place de \ADD (e.g., ce comportement peut être observé dans le code d'\oracle{} car il est
compilé avec le compilateur C++ d'Intel.
Cette instruction a presque le même effet mais le contenu du registre \ECX sera écrasé.
Le compilateur C++ d'Intel utilise probablement \TT{POP ECX} car l'opcode de cette instruction est plus
 court que celui de \TT{ADD ESP, x} (1 octet pour \TT{POP} contre 3 pour \TT{ADD}).

Voici un exemple d'utilisation de \POP à la place de \ADD dans \oracle{}:

\begin{lstlisting}[caption=\oracle 10.2 Linux (app.o file),style=customasmx86]
.text:0800029A                 push    ebx
.text:0800029B                 call    qksfroChild
.text:080002A0                 pop     ecx
\end{lstlisting}

%Lisez en plus sur la pile dans la section
% ~(\myref{sec:stack}).

\myindex{\CLanguageElements!return}
Après l'appel de \printf, le code \CCpp original contient la déclaration \TT{return 0}~---renvoie 0 comme valeur de retour de la fonction \main.

\myindex{x86!\Instructions!XOR}
Dans le code généré cela est implémenté par l'instruction \INS{XOR EAX, EAX}.

\myindex{x86!\Instructions!MOV}

\XOR est en fait un simple \q{OU exclusif (eXclusive OR}\footnote{\href{http://go.yurichev.com/17118}{wikipedia}} mais
les compilateurs l'utilisent souvent à la place de \INS{MOV EAX, 0}---à nouveau parce que l'opcode est légèrement plus
court (2 octets pour \XOR contre 5 pour \MOV).

\myindex{x86!\Instructions!SUB}
Certains compilateurs génèrent \INS{SUB EAX, EAX}, qui signifie \IT{Soustraire la valeur dans} \EAX \IT{de la valeur dans} \EAX,
 ce qui, dans tous les cas, donne zéro.

\myindex{x86!\Instructions!RET}
La dernière instruction \RET redonne le contrôle à l'appelant (\gls{caller}). c'est du code \CCpp \ac{CRT}, qui, à son tour, redonne le contrôle à l'\ac{OS}.

