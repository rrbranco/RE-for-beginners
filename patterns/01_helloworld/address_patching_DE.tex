\subsubsection{Adress-Patching (Win64)}

Wenn das Beispiel in MSCV2013 mit der Option \TT{\textbackslash{}MD} kompiliert wird
(was zu einer kleineren ausfühbaren Datei durch das linken mit \TT{MSVCR*.DLL} führt),
kommt zuerst die \main-Funktion und kann einfach gefunden werden:

\begin{figure}[H]
\centering
\myincludegraphics{patterns/01_helloworld/hiew_incr1.png}
\caption{Hiew}
\label{}
\end{figure}

Als Experiment kann die Adresse des Pointers um 1 \gls{increment} werden:

\begin{figure}[H]
\centering
\myincludegraphics{patterns/01_helloworld/hiew_incr2.png}
\caption{Hiew}
\label{}
\end{figure}

Hiew zeigt \q{ello, world} als Zeichenkette und beim Ausführen der gepatchten
Datei wird eben dieser Text ausgegeben.

\subsubsection{Aussuchen einer anderen Zeichenkette einer Binärdatei (Linux x64)}

Die Binärdatei die beim Kompilieren des Beispiels mit GCC 5.4.0 unter Linux x64
entsteht, beinhaltet noch viele andere Zeichenketten: die meisten sind importierte
Funktions- und Bibliotheksnamen.

Mit objdump können die Inhalte aller Sektionen der kompilierten Datei ausgegeben werden:

\begin{lstlisting}[basicstyle=\ttfamily, mathescape]
$\$$ objdump -s a.out

a.out:     file format elf64-x86-64

Contents of section .interp:
 400238 2f6c6962 36342f6c 642d6c69 6e75782d  /lib64/ld-linux-
 400248 7838362d 36342e73 6f2e3200           x86-64.so.2.
Contents of section .note.ABI-tag:
 400254 04000000 10000000 01000000 474e5500  ............GNU.
 400264 00000000 02000000 06000000 20000000  ............ ...
Contents of section .note.gnu.build-id:
 400274 04000000 14000000 03000000 474e5500  ............GNU.
 400284 fe461178 5bb710b4 bbf2aca8 5ec1ec10  .F.x[.......^...
 400294 cf3f7ae4                             .?z.

...
\end{lstlisting}

Es ist kein Problem die Adresse der Zeichenkette \q{/lib64/ld-linux-x86-64.so.2}
an \TT{printf()} zu übergeben:

\begin{lstlisting}[style=customc]
#include <stdio.h>

int main()
{
    printf(0x400238);
    return 0;
}
\end{lstlisting}

Schwer zu glauben, aber dieser Code gibt die erwähnte Zeichenkette aus.

Beim Ändern der Adresse zu \TT{0x400260} wird die Zeichenkette \q{GNU} ausgegeben.
Diese Adresse gilt für die hier verwendete GCC-Version, Toolkonfiguration und so weiter.
Auf anderen Systemen kann die ausfühbare Datei leicht unterschiedlich sein, was auch
die Adressen verändern kann.
Auch das Hinzufügen und Entfernen von Quellcode kann Adressen vor- und zurückschieben.
