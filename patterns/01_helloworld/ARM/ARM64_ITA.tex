\subsubsection{ARM64}

\myparagraph{GCC}

Compiliamo l'esempio con GCC 4.8.1 per ARM64:

\lstinputlisting[numbers=left,label=hw_ARM64_GCC,caption=\NonOptimizing GCC 4.8.1 + objdump,style=customasmARM]{patterns/01_helloworld/ARM/hw.lst}

In ARM64 non ci sono le modalita' Thumb e Thumb-2, solo ARM, quindi soltanto istruzioni a 32-bit.
Il numero di registri e' raddoppiato: \myref{ARM64_GPRs}.
I registri a 64-bit hanno il prefisso \TT{X-} prefixes, mentre le loro parti a 32-bit hanno il prefisso --- \TT{W-}.

\myindex{ARM!\Instructions!STP}
L'istruzione \TT{STP} (\IT{Store Pair}) 
salva simultaneamente due registri nello stack: \RegX{29} e \RegX{30}.

Questa istruzione puo' ovviamente salvare la coppia di valori in una posizione arbitraria in memoria, tuttavia in questo caso e'
specificato il registro \ac{SP} , e di conseguenza la coppia viene salvata nello stack.

I registri ARM64 sono a 64-bit, ognuno di essi ha dimensione pari a 8 bytes, quindi sono necessari 16 bytes per salvare i due registri.

Il punto esclamativo (``!'') dopo l'operando sta a significare che 16 deve esse prima sottratto da \ac{SP} , e solo successivamente
i valori devono essere scritti nello stack.

Questo e' anche detto \IT{pre-index}.
Per le differenze tra \IT{post-index} e \IT{pre-index} 
leggere qui: \myref{ARM_postindex_vs_preindex}.

Quindi, in termini del piu' familiare x86, la prima istruzione e' semplicamente l'analogo della coppia
\TT{PUSH X29} e \TT{PUSH X30}.
\RegX{29} in ARM64 e' usato come \ac{FP} , e \RegX{30} 
come \ac{LR}, e questo spiega perche' sono salvati nel prologo di funzione e ripristinati nell'epilogo.

La seconda istruzione copia \ac{SP} in \RegX{29} (o \ac{FP}).
Cio' viene fatto per impostare lo stack frame della funzione.

\label{pointers_ADRP_and_ADD}
\myindex{ARM!\Instructions!ADRP/ADD pair}
Le istruzioni \TT{ADRP} e \ADD sono usate per inserire
l'indirizzo della stringa \q{Hello!} nel registro \RegX{0} , 
poiche' il primo argomento della funzione viene passato in questo registro.

Non esiste alcun tipo di istruzione in ARM in grado di salvare un numero molto grande in un registro (perche' la lunghezza delle
istruzioni e' limitata a 4 byte, maggiori informazioni qui: \myref{ARM_big_constants_loading}).
Percio' devono essere utilizzate piu' istruzioni. La prima (\TT{ADRP}) , scrive l'indirizzo della pagina di 4KiB (4KiB page)
,in cui si trova la stringa, nel registro \RegX{0}, e la seconda (\ADD) aggiunge semplicemente il resto dell'indirizzo.
Maggiori informazioni su questo tema: \myref{ARM64_relocs}.

\TT{0x400000 + 0x648 = 0x400648}, e vediamo la nostra C-string \q{Hello!} nel \TT{.rodata} data segment a questo indirizzo.

\myindex{ARM!\Instructions!BL}

\puts viene chiamata subito dopo usando l'istruzione \TT{BL}. Questo e' stato gia' discusso: \myref{puts}.

\MOV scrive 0 in \RegW{0}. 
\RegW{0} e' la parte bassa a 32 bits del registro a 64-bit \RegX{0}:

\begin{center}
\begin{tabular}{ | l | l | }
\hline
\RU{Старшие 32 бита}\EN{High 32-bit part}\ES{Parte alta de 32 bits}\PTBRph{}\PLph{}\ITAph{}\DEph{}\THAph{}\NLph{}\FR{Partie 32 bits haute} & \RU{младшие 32 бита}\EN{low 32-bit part}\ES{parte baja de 32 bits}\PTBRph{}\PLph{}\ITAph{}\DEph{}\THAph{}\NLph{}\FR{Partie 32 bits basse} \\
\hline
\multicolumn{2}{ | c | }{X0} \\
\hline
\multicolumn{1}{ | c | }{} & \multicolumn{1}{ c | }{W0} \\
\hline
\end{tabular}
\end{center}


Il risultato della funzione e' restituito tramite \RegX{0} e \main restituisce 0, quindi e' cosi' che viene preparato 
il valore da restituire.
Ma perche' usare la parte a 32-bit?

Perche' il tipo \Tint in ARM64, esattamente come in x86-64, e' a 32-bit, per maggiore compatibilita'.
Quindi se una funzione restituisce un \Tint a 32-bit, solo la parte piu' bassa a 32 bits del registro \RegX{0} sara' valorizzata.

Per verificare quanto detto, cambiamo leggermente l'esempio e ricompiliamolo.
Adesso \main restituisce un valore a 64-bit:

\begin{lstlisting}[caption=\main returning a value of \TT{uint64\_t} type,style=customc]
#include <stdio.h>
#include <stdint.h>

uint64_t main()
{
        printf ("Hello!\n");
        return 0;
}
\end{lstlisting}

Il risultato e' lo stesso, ma quell'istruzione MOV adesso appare cosi': 

\begin{lstlisting}[caption=\NonOptimizing GCC 4.8.1 + objdump]
  4005a4:       d2800000        mov     x0, #0x0      // #0
\end{lstlisting}

\myindex{ARM!\Instructions!LDP}

\INS{LDP} (\IT{Load Pair}) infine riprisina i registri \RegX{29} e \RegX{30}.

Non c'e' il punto esclamativo dopo l'istruzione: cio' implica che il valore viene prima caricato dallo stack, e solo successivamente 
\ac{SP} e' incrementato di 16.
Cio' e' detto \IT{post-index}.

\myindex{ARM!\Instructions!RET}
Una nuova istruzione e' apparsa in ARM64: \RET. 
Funziona esattamente come \TT{BX LR}, con l'aggiunta di uno speciale \IT{hint} bit, che informa la \ac{CPU}
del fatto che si tratta di un ritorno da una funzione, e non soltanto una normale istruzione jump, cosi' che possa 
essere eseguita in modo ottimale.

A causa della semplicita' della funzione, GCC con le opzioni di ottimizzazione genera esattamente lo stesso codice.
