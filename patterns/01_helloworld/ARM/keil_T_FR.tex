\subsubsection{\NonOptimizingKeilVI (\ThumbMode)}

Compilons le même exemple en utilisant keil en mode Thumb:

\begin{lstlisting}
armcc.exe --thumb --c90 -O0 1.c 
\end{lstlisting}

Nous obtenons (dans \IDA):

\begin{lstlisting}[caption=\NonOptimizingKeilVI (\ThumbMode) + \IDA,style=customasmARM]
.text:00000000             main
.text:00000000 10 B5          PUSH    {R4,LR}
.text:00000002 C0 A0          ADR     R0, aHelloWorld ; "hello, world"
.text:00000004 06 F0 2E F9    BL      __2printf
.text:00000008 00 20          MOVS    R0, #0
.text:0000000A 10 BD          POP     {R4,PC}

.text:00000304 68 65 6C 6C+aHelloWorld  DCB "hello, world",0    ; DATA XREF: main+2
\end{lstlisting}

Nous pouvons repérer facilement les opcodes sur 2 octets (16-bit). C'est, comme déjà noté, Thumb.
\myindex{ARM!\Instructions!BL}
L'instruction \TT{BL}, toutefois, consiste en deux instructions 16-bit.
C'est parce qu'il est impossible de charger un offset pour la fonction \printf
en utilisant seulement le petit espace dans un opcode 16-bit.
Donc, la première instruction 16-bit charge les 10 bits supérieurs de l'offset
et la seconde instruction les 11 bits inférieurs de l'offset.

% TODO:
% BL has space for 11 bits, so if we don't encode the lowest bit,
% then we should get 11 bits for the upper half, and 12 bits for the lower half.
% And the highest bit encodes the sign, so the destination has to be within
% \pm 4M of current_PC.
% This may be less if adding the lower half does not carry over,
% but I'm not sure --all my programs have 0 for the upper half,
% and don't carry over for the lower half.
% It would be interesting to check where __2printf is located relative to 0x8
% (I think the program counter is the next instruction on a multiple of 4
% for THUMB).
% The lower 11 bytes of the BL instructions and the even bit are
% 000 0000 0110 | 001 0010 1110 0 = 000 0000 0110 0010 0101 1100 = 0x00625c,
% so __2printf should be at 0x006264.
% But if we only have 10 and 11 bits, then the offset would be:
% 00 0000 0110 | 01 0010 1110 0 = 0 0000 0011 0010 0101 1100 = 0x00325c,
% so __2printf should be at 0x003264.
% In this case, though, the new program counter can only be 1M away,
% because of the highest bit is used for the sign.

Comme il a été écrit, toutes les instructions en mode Thumb ont une taille de 2
octets (ou 16 bits).
Cela implique qu'il impossible pour une instruction Thumb d'être à une adresse
impaire, quelle qu'elle soit.
En tenant compte de cela, le dernier bit de l'adresse peut être omis lors de
l'encodage des instructions.

En résumé, l'instruction Thumb \TT{BL} peut encoder une adresse en $current\_PC \pm{}\approx{}2M$.

\myindex{ARM!\Instructions!PUSH}
\myindex{ARM!\Instructions!POP}
Comme pour les autres instructions dans la fonction: \PUSH et \POP fonctionnent
ici comme les instructions décrites \TT{STMFD}/\TT{LDMFD} seul le
registre \ac{SP} n'est pas mentionné explicitement ici.
\TT{ADR} fonctionne comme dans l'exemple précédent.
\TT{MOVS} écrit 0 dans le registre \Reg{0} afin de renvoyer zéro.

