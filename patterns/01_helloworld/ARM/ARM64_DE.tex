\subsubsection{ARM64}

\myparagraph{GCC}

Das Beispiel wird im Folgenden mit GCC 4.1.8 in ARM64 kompiliert:

\lstinputlisting[numbers=left,label=hw_ARM64_GCC,caption=\NonOptimizing GCC 4.8.1 + objdump,style=customasmARM]{patterns/01_helloworld/ARM/hw.lst}

Es gibt keine Thumb- oder Thumb-2-Modes in ARM64, sondern nur ARM, also 32-Bit-Anweisungen.
Die Register-Anzahl ist verdoppelt: \myref{ARM64_GPRs}.
64-Bit-Register haben einen \TT{X-}Prefix, 32-Bit-Teile ein \TT{W-}.

\myindex{ARM!\Instructions!STP}
Die \TT{STP}-Anweisung (\IT{Store Pair}) speichert zwei Register auf dem Stack gleichzeitig:
\RegX{29} und \RegX{30}.

Natürlich kann diese Anweisung dieses Registerpaar an einer beliebigen Stelle im Speicher
sichern, aber da hier das \ac{SP}-Register angegeben ist, wird das Paar auf dem Stack gesichert.

ARM64-Register sind 64 Bit breit, jedes von ihnen ist 8 Byte groß. Dementsprechend werden 16 Byte
für das Speichern zweier Register benötigt.

Das Ausrufungszeichen (``!'')  nach dem Operanden bedeutet, dass zunächst der Wert 16 vom \ac{SP}
subtrahiert werden muss und erst dann die Werte vom Register-Paar auf den Stack geschrieben werden.
Dies wird auch \IT{pre-index} genannt.
Mehr über den Unterschied von \IT{post-index} und \IT{pre-index} ist im Abschnitt
\myref{ARM_postindex_vs_preindex} zu finden.

Im Sprachgebrauch des gebräuchlicheren x86, ist die erste Anweisung analog zu den Anweisungen
\TT{PUSH X29} und \TT{PUSH X30} zu verstehen.
\RegX{29} wird als \ac{FP} in ARM64 genutzt, und \RegX{30} als \ac{LR}, weswegen sie am Anfang der
Funktion gesichert und am Ende wiederhergestellt werden.

Die zweite Anweisung kopiert \ac{SP} in \RegX{29} (oder \ac{FP}) um den Stack Frame vorzubereiten.

\label{pointers_ADRP_and_ADD}
\myindex{ARM!\Instructions!ADRP/ADD pair}
\TT{ADRP} und \ADD-Anweisungen werden genutzt um die Adresse der Zeichenkette \q{Hello!} in das
Register \RegX{0} zu schreiben, da das erste Funktionsargument in an dieser Stelle übergeben wird. 

Es gibt in ARM keine Anweisung, die eine große Zahl in einem Register sichern kann, weil die Länge der
Anweisungen auf 4 Byte begrenzt ist. Siehe dazu auch \myref{ARM_big_constants_loading}).
Aus diesem Grund müssen mehrere Anweisungen genutzt werden. Die erste (\TT{ADRP}) schreibt die Adresse
der 4KiB-Page in der die Zeichenkette sich befindet in das Register \RegX{0}.
Die zweite (\ADD) addiert lediglich den Rest der Adresse.
Siehe dazu auch \myref{ARM64_relocs}.

\TT{0x400000 + 0x648 = 0x400648}, und die Zeichenkette \q{Hello!} ist im \TT{.rodata} Daten-Segmet
an dieser Adresse zu sehen.

\myindex{ARM!\Instructions!BL}

\puts wird anschließend mit der \TT{BL}-Anweisung aufgerufen. Dies wurde bereits diskutiert: \myref{puts}.

\MOV schreibt 0 in \RegW{0}.
\RegW{0} sind die niederwertigeren 32 Bit des 64-Bit-Registers \RegX{0}:

\begin{center}
\begin{tabular}{ | l | l | }
\hline
\RU{Старшие 32 бита}\EN{High 32-bit part}\ES{Parte alta de 32 bits}\PTBRph{}\PLph{}\ITAph{}\DEph{}\THAph{}\NLph{}\FR{Partie 32 bits haute} & \RU{младшие 32 бита}\EN{low 32-bit part}\ES{parte baja de 32 bits}\PTBRph{}\PLph{}\ITAph{}\DEph{}\THAph{}\NLph{}\FR{Partie 32 bits basse} \\
\hline
\multicolumn{2}{ | c | }{X0} \\
\hline
\multicolumn{1}{ | c | }{} & \multicolumn{1}{ c | }{W0} \\
\hline
\end{tabular}
\end{center}


Das Ergebnis der Funktion wird über \RegX{0} zurückgegeben und \main gibt 0 zurück.
Dies ist also der Weg wie das Ergebnis vorbereitet wird.
Der 32-Bit-Teil wird genutzt,weil der \Tint-Datentyp in ARM64 aus Kompatibilitätsgründen,
wie in x86-64, 32 Bit breit ist.

Da die Funktion einen 32-Bit \Tint-Wert zurück gibt, müssen lediglich die unteren 32 Bits des
\RegX{0}-Registers gefüllt werden.

Um dies zu überprüfen wird das Beispiel leicht verändert und neu kompiliert.
\main soll nun einen 64-Bit-Wert zurück geben:

\begin{lstlisting}[caption=\main gibt einen \TT{uint64\_t}-Datentyp zurück,style=customc]
#include <stdio.h>
#include <stdint.h>

uint64_t main()
{
        printf ("Hello!\n");
        return 0;
}
\end{lstlisting}

Das Ergebnis ist das gleiche, allerdings sieht \MOV nun wie folgt aus:

\begin{lstlisting}[caption=\NonOptimizing GCC 4.8.1 + objdump]
  4005a4:       d2800000        mov     x0, #0x0      // #0
\end{lstlisting}

\myindex{ARM!\Instructions!LDP}

\INS{LDP} (\IT{Load Pair}) stellt anschließend die Register \RegX{29} und \RegX{30} wieder her.

An dieser Stelle steht kein Ausrufungszeichen nach der Anweisung: dies impliziert, dass der Wert zunächst
vom Stack gelesen wird und erst dann wird \ac{SP} um den Wert 16 verringert.
Dies wird \IT{post-index} genannt.

\myindex{ARM!\Instructions!RET}
Eine neue Anweisung taucht hier in ARM64 auf \RET.
Diese arbeitet wie \TT{BX LR}, jedoch wird ein spezielles \IT{Hinweis-}Bit hinzugefügt, welches die \ac{CPU}
darüber informiert, dass dies ein Rücksprung aus einer Funktion ist und kein anderer Sprung, so dass die
Ausführung optimiert werden kann.

Aufgrund der Einfachheit dieser Funktion, erstellt der optimierende GCC den gleichen Code.
