\subsubsection{\OptimizingXcodeIV (\ARMMode)}

Xcode 4.6.3 sans l'option d'optimisation produit beaucoup de code redondant c'est
pourquoi nous allons étudier le code généré avec optimisation, où le nombre
d'instruction est aussi petit que possible, en mettant l'option \Othree du
compilateur.

\begin{lstlisting}[caption=\OptimizingXcodeIV (\ARMMode),style=customasmARM]
__text:000028C4             _hello_world
__text:000028C4 80 40 2D E9   STMFD           SP!, {R7,LR}
__text:000028C8 86 06 01 E3   MOV             R0, #0x1686
__text:000028CC 0D 70 A0 E1   MOV             R7, SP
__text:000028D0 00 00 40 E3   MOVT            R0, #0
__text:000028D4 00 00 8F E0   ADD             R0, PC, R0
__text:000028D8 C3 05 00 EB   BL              _puts
__text:000028DC 00 00 A0 E3   MOV             R0, #0
__text:000028E0 80 80 BD E8   LDMFD           SP!, {R7,PC}

__cstring:00003F62 48 65 6C 6C+aHelloWorld_0  DCB "Hello world!",0
\end{lstlisting}

Les instructions \TT{STMFD} et \TT{LDMFD} nous sont déjà familières.

\myindex{ARM!\Instructions!MOV}

L'instruction \MOV écrit simplement le nombre \TT{0x1686} dans le registre \Reg{0}.
C'est l'offset pointant sur la chaîne \q{Hello world!}.

Le registre \TT{R7} (tel qu'il est standardisé dans \IOSABI) est un pointeur de frame. Voir plus loin.

\myindex{ARM!\Instructions!MOVT}
L'instruction \TT{MOVT R0, \#0} (MOVe Top) écrit 0 dans les 16 bits de poids
fort du registre.
Le problème ici est que l'instruction générique \MOV en mode ARM peut n'écrire
que dans les 16 bits de poids faible du registre.

Il faut garder à l'esprit que tout les opcodes d'instruction en mode ARM sont
limités en taille à 32 bits. Bien sûr, cette limitation n'est pas relative
au déplacement de données entre registres.
C'est pourquoi une instruction supplémentaire existe \TT{MOVT} pour écrire dans
les bits de la partie haute (de 16 à 31 inclus).
Son usage ici, toutefois, est redondant car l'instruction \TT{MOV R0, \#0x1686}
ci dessus a éffacé la partie haute du registre.
C'est soi-disant un défaut du compilateur.
% TODO:
% I think, more specifically, the string is not put in the text section,
% ie. the compiler is actually not using position-independent code,
% as mentioned in the next paragraph.
% MOVT is used because the assembly code is generated before the relocation,
% so the location of the string is not yet known,
% and the high bits may still be needed.

\myindex{ARM!\Instructions!ADD}
L'instruction \TT{ADD R0, PC, R0} ajoute la valeur dans \ac{PC} à celle de
\Reg{0}, pour calculer l'adresse absolue de la chaîne \q{Hello world!}.
Comme nous l'avons déjà vu, il s'agit de \q{\PICcode} donc la correction
est essentielle ici.

L'instruction \INS{BL} appelle la fonction \puts au lieu de \printf.

\label{puts}
\myindex{\CStandardLibrary!puts()}
\myindex{puts() instead of printf()}

GCC a remplacé le premier appel à \printf par un à \puts.
Effectivement: \printf avec un unique argument est presque analogue à \puts.

\IT{Presque}, car les deux fonctions produisent le même résultat uniquement dans
le cas où la chaîne ne contient pas d'identifiants de format débutant par \IT{\%}.
Dans le cas où elle en contient, l'effet de ces deux fonctions est différent.
\footnote{Il est à noter que \puts ne nécessite pas un `\textbackslash{}n'
symbole de retour à la ligne à la fin de la chaîne, donc nous ne le voyons pas ici.}.

Pourquoi est-ce que le compilateur a remplacé \printf par \puts? Probablement car
\puts est plus rapide.
\footnote{\href{http://go.yurichev.com/17063}{ciselant.de/projects/gcc\_printf/gcc\_printf.html}}. 

Car il envoie seulement les caractères dans \glslink{stdout}{sortie standard}
sans comparer chacun d'entre eux avec le symbole \IT{\%}.

Ensuite, nous voyons l'instruction familière \TT{MOV R0, \#0} pour mettre le
registre \Reg{0} à 0.
