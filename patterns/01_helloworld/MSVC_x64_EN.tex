\subsubsection{MSVC: x86-64}

\myindex{x86-64}
Let's also try 64-bit MSVC:

\lstinputlisting[caption=MSVC 2012 x64,style=customasmx86]{patterns/01_helloworld/MSVC_x64.asm}

\myindex{fastcall}

In x86-64, all registers were extended to 64-bit and now their names have an \TT{R-} prefix.
In order to use the stack less often (in other words, to access external memory/cache less often), there exists
a popular way to pass function arguments via registers (\IT{fastcall}) \myref{fastcall}.
I.e., a part of the function arguments is passed in registers, the rest---via the stack.
In Win64, 4 function arguments are passed in the \RCX, \RDX, \Reg{8}, \Reg{9} registers.
That is what we see here: a pointer to the string for \printf is now passed not in the stack, but in the \RCX register.
The pointers are 64-bit now, so they are passed in the 64-bit registers (which have the \TT{R-} prefix).
However, for backward compatibility, it is still possible to access the 32-bit parts, using the \TT{E-} prefix.
This is how the \RAX/\EAX/\AX/\AL register looks like in x86-64:

\RegTableOne{RAX}{EAX}{AX}{AH}{AL}

The \main function returns an \Tint{}-typed value, which is, in \CCpp, for better backward compatibility
and portability, still 32-bit, so that is why the \EAX register is cleared at the function end (i.e., the 32-bit
part of the register) instead of \RAX{}.
There are also 40 bytes allocated in the local stack.
This is called the \q{shadow space}, about which we are going to talk later: \myref{shadow_space}.
