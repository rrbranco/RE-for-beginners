\myparagraph{32-bit ARM}

\mysubparagraph{\NonOptimizingXcodeIV (\ARMMode)}

\lstinputlisting[caption=\NonOptimizingXcodeIV
(\ARMMode),label=ARM_leaf_example7,style=customasmARM]{patterns/10_strings/1_strlen/ARM/xcode_ARM_O0_DE.asm}

Der nicht optimierende LLVM erzeugt zu viel Code, aber wir können wir erkennen
wir die Funktion mit lokalen Variablen auf dem Stack arbeitet. 
Es gibt nur zwei lokale Variablen in unserer Funktion: \IT{eos} und \IT{str}.
In folgenden von \IDA erzeugten Listing, sind die Variablen \IT{var\_8} und
\IT{var\_4} in \IT{eos} bzw. \IT{str} umbenannt.

Der erste Befehl speichert lediglich bei Eingabewerte in \IT{str} und \IT{eos}.

Der Körper der Schleife startet beim Label \IT{loc\_2CB8}.

Die ersten drei Befehle des Schleifenkörpers (\TT{LDR}, \ADD, \TT{STR}) laden
den Wert von \IT{eos} nach \Reg{0}. 
Anschließend wird der Wert erhöht und zurück in \IT{eos} auf den Stack
geschrieben.

\myindex{ARM!\Instructions!LDRSB}
Der folgende Befehl, \TT{LDRSB R0, [R0]} (\q{Load Register Signed Byte}), lädt
ein Byte aus dem Speicher von der Adresse in \Reg{0} und erweitert es mit
Vorzeichen auf 32-bit.
\footnote{Der Keil Compiler behandelt den Typ \Tchar als signed, genau wie
MSVC und GCC.}.
\myindex{x86!\Instructions!MOVSX}
Dies ist vergleichbar zum \MOVSX Befehl in x86.

Der Compiler behandelt dieses Byte als signed, das der \Tchar Typ nach dem
C-Standard ebenfalls signed ist. 
Dies wurde in Bezug auf x86 in diesem Abschnitt bereits in~(\myref{MOVSX})
beschrieben.

\myindex{Intel!8086}
\myindex{Intel!8080}
\myindex{ARM}

Man beachte, dass es in ARM unmöglich ist, einen 8- oder 16-bit-Teil eines
32-bit-Registers alleine zu verwenden, anderes als in x86.

Dies rührt daher, dass x86 eine große Bandbreite and Kompatibilität mit
Vorgängerversionen besitzt, bis hin zum 16-bit 8086 oder sogar dem 8-bit 8080,
ARM auf der anderen Seite jedoch von Beginn an als 32-bit RISC-Prozessor geplant
wurde.

Infolgedessen müssen auch um einzelne Bytes in ARM zu verarbeiten, stets
komplette 32-bit-Register verwendet werden.

Der Befehl \TT{LDRSB} lädt nun die Bytes des String einzeln nach \Reg{0}.
Die nachfolgenden \CMP und \ac{BEQ} Befehle prüfen, ob das aktuelle Byte 0 ist.
Wenn nicht, beginnt der Körper der Schleife erneut.
Und wenn das aktuelle Byte 0 ist, dann wird die Schleife beendet.

Am Ende der Funktion wird die Differenz zwischen \IT{eos} und \IT{str}
berechnet, 1 vom Ergebnis abgezogen und das Resultat über das Register \Reg{0}
zurückgegeben.

N.B. Register wurden in dieser Funktion nicht gespeichert.
\myindex{ARM!\Registers!scratch registers}
Das liegt daran, dass gemäß der ARM Aufrufkonventionen die Register \Reg{0} bis
\Reg{3} sogenannte \q{scratch register} sind, vorgesehen für Parameterübergaben.
Deshalb ist es nicht notwendig ihren Inhalt am Ende der Funktion
wiederherzustellen, denn die aufrufende Funktion wird diese Werte nicht weiter
verwenden. 
Im weiteren können diese für alles Mögliche benutzt werden.

Es werden hier keine weiteren Register verwendet, sodass wir nichts auf dem
Stack speichern müssen. 

Dadurch kann der control flow über einen einfachen Sprung (\TT{BX}) an die
aufrufende Funktion an der Adresse im \ac{LR} Register übergeben werden.

\mysubparagraph{\OptimizingXcodeIV (\ThumbMode)}

\lstinputlisting[caption=\OptimizingXcodeIV (\ThumbMode),style=customasmARM]{patterns/10_strings/1_strlen/ARM/xcode_thumb_O3.asm}
Der optimierende LLVM entscheidet also, dass \IT{eos} und \IT{str} keinen Platz
auf dem Stack benötigen, sondern stets in Registern gespeichert werden können. 

Vor dem Anfang des Schleifenkörpers befindet sich \IT{str} stets in \Reg{0} und
\IT{eos} in \Reg{1}.

\myindex{ARM!\Instructions!LDRB.W}
Der Befehl \TT{LDRB.W R2, [R1],\#1} lädt ein Byte aus dem Speicher von der
Adresse aus \Reg{1} nach \Reg{2}, erweitert es zum einem signed 32-bit-Wert und
mehr noch: Das \TT{\#1} am Ende des Befehl bewirkt \q{Post-indexed addressing},
was bedeutet, dass 1 zum Register \Reg{1} addiert wird, nachdem das Byte geladen
wurde.
Mehr zum Thema:\myref{ARM_postindex_vs_preindex}. 

Des Weiteren finden wir \CMP und \ac{BNE} im Körper der Schleife; diese Befehle
werden durchlaufen, bis 0 im String gefunden wurde.

\myindex{ARM!\Instructions!MVNS}
\myindex{x86!\Instructions!NOT}
\TT{MVNS}\footnote{MoVe Not} (invertiert alle Bits wie \NOT in x86) und \ADD
Befehle berechnen $eos - str - 1$.
Tatsächlich berechnen diese beiden Befehle $R0 =
~str + eos$, was äquivalent zur Formulierung im Quellcode ist und die Begründung
dazu wurde bereits hier gegeben ~(\myref{strlen_NOT_ADD}).

Offenbar befindet LLVM genau wie GCC, dass diese Code kürzer (oder schneller)
ist.

\mysubparagraph{\OptimizingKeilVI (\ARMMode)}

\lstinputlisting[caption=\OptimizingKeilVI (\ARMMode),label=ARM_leaf_example6,style=customasmARM]{patterns/10_strings/1_strlen/ARM/Keil_ARM_O3.asm}

\myindex{ARM!\Instructions!SUBEQ}

Fast das gleiche wie zuvor, mit der Änderung, dass der $str - eos - 1$
Ausdruck nicht am Ende der Funktion, sondern mitten in der Schleife berechnet
wird. Wir erinnern uns, dass der \TT{-EQ} Suffix bedeutet, dass die dieser
Befehl nur dann ausgeführt wird, wenn die Operanden im \CMP direkt davor gleich
waren.
Dadurch werden beide \TT{SUBEQ} Befehle ausgeführt, falls das \Reg{0} Register 0
enthält und das Ergebnis verbleibt in \Reg{0}.
