\myparagraph{ARM64}

\mysubparagraph{\Optimizing GCC (Linaro) 4.9}

\lstinputlisting[style=customasmARM]{patterns/10_strings/1_strlen/ARM/ARM64_GCC_O3_EN.lst}

The algorithm is the same as in \myref{strlen_MSVC_Ox}: 
find a zero 
byte, calculate the difference between the pointers and decrement the result by 1.
Some comments were added by the author of this book.

The only thing worth noting is that our example is somewhat wrong: \\
\TT{my\_strlen()} returns 32-bit \Tint, while it has to return \TT{size\_t} or another 64-bit type.

The reason is that, theoretically, \TT{strlen()} can be called for a huge blocks in memory that exceeds
4GB, so it must able to return a 64-bit value on 64-bit platforms.

Because of my mistake, the last \SUB instruction operates on a 32-bit part of register, while the penultimate
\SUB instruction works on full the 64-bit register (it calculates the difference between the pointers).

It's my mistake, it is better to leave it as is, as an example of how the code could look like in such case.

\mysubparagraph{\NonOptimizing GCC (Linaro) 4.9}

\lstinputlisting[style=customasmARM]{patterns/10_strings/1_strlen/ARM/ARM64_GCC_O0_EN.lst}

It's more verbose.
The variables are often tossed here to and from memory (local stack).
The same mistake here: the decrement operation happens on a 32-bit register part.

