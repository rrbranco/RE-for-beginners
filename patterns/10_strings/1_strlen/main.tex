\subsection{strlen()}
\myindex{\CStandardLibrary!strlen()}

\ifdefined\ENGLISH
Let's talk about loops one more time. Often, the \TT{strlen()} 
function
\footnote{counting the characters in a string in the C language} 
is implemented using a \TT{while()} statement.
Here is how it is done in the MSVC standard libraries:
\fi

\ifdefined\RUSSIAN
Ещё немного о циклах. Часто функция \TT{strlen()}
\footnote{подсчет длины строки в Си}
реализуется при помощи \TT{while()}.
Например, вот как это сделано в стандартных библиотеках MSVC:
\fi

\ifdefined\GERMAN
\DEph{}
\fi

\lstinputlisting[style=customc]{patterns/10_strings/1_strlen/ex1.c}

% subsections
\EN{\subsubsection{x86}

\myparagraph{\NonOptimizing MSVC}

Let's compile:

\lstinputlisting[style=customasmx86]{patterns/10_strings/1_strlen/10_1_msvc_EN.asm}

\myindex{x86!\Instructions!MOVSX}
\myindex{x86!\Instructions!TEST}

We get two new instructions here: \MOVSX and \TEST.

\label{MOVSX}

The first one---\MOVSX---takes a byte from an address in memory and stores the value in a 32-bit register. 
\MOVSX stands for \IT{MOV with Sign-Extend}. 
\MOVSX sets the rest of the bits, from the 8th to the 31th, 
to 1 if the source byte is \IT{negative} or to 0 if is \IT{positive}.

And here is why.

By default, the \Tchar type is signed in MSVC and GCC. If we have two values of which one is \Tchar 
and the other is \Tint, (\Tint is signed too), and if the first value contain -2 (coded as \TT{0xFE}) 
and we just copy this byte into the \Tint container, it makes \TT{0x000000FE}, and this 
from the point of signed \Tint view is 254, but not -2. In signed int, -2 is coded as \TT{0xFFFFFFFE}. 
So if we have to transfer \TT{0xFE} from a variable of \Tchar type to \Tint, 
we have to identify its sign and extend it. That is what \MOVSX does.

You can also read about it in \q{\IT{\SignedNumbersSectionName}} section~(\myref{sec:signednumbers}).

It's hard to say if the compiler needs to store a \Tchar variable in \EDX, it could just take a 8-bit register part 
(for example \DL). Apparently, the compiler's \gls{register allocator} works like that.

\myindex{ARM!\Instructions!TEST}

Then we see \TT{TEST EDX, EDX}. 
You can read more about the \TEST instruction in the section about bit fields~(\myref{sec:bitfields}).
Here this instruction just checks if the value in \EDX equals to 0.

\myparagraph{\NonOptimizing GCC}

Let's try GCC 4.4.1:

\lstinputlisting[style=customasmx86]{patterns/10_strings/1_strlen/10_3_gcc.asm}

\label{movzx}
\myindex{x86!\Instructions!MOVZX}

The result is almost the same as in MSVC, but here we see \MOVZX instead of \MOVSX. 
\MOVZX stands for \IT{MOV with Zero-Extend}. 
This instruction copies a 8-bit or 16-bit value into a 32-bit register and sets the rest of the bits to 0. 
In fact, this instruction is convenient only because it enable us to replace this instruction pair:\\
\TT{xor eax, eax / mov al, [...]}.

On the other hand, it is obvious that the compiler could produce this code: \\
\TT{mov al, byte ptr [eax] / test al, al}---it is almost the same, however, 
the highest bits of the \EAX register will contain random noise. 
But let's think it is compiler's drawback---it cannot produce more understandable code. 
Strictly speaking, the compiler is not obliged to emit understandable (to humans) code at all.

\myindex{x86!\Instructions!SETcc}

The next new instruction for us is \SETNZ. 
Here, if \AL doesn't contain zero, \TT{test al, al} 
sets the \ZF flag to 0, but \SETNZ, if \TT{ZF==0} (\IT{NZ} stands for \IT{not zero}) sets \AL to 1.
Speaking in natural language, \IT{if \AL is not zero, let's jump to loc\_80483F0}. 
The compiler emits some redundant code, but let's not forget that the optimizations are turned off.

\myparagraph{\Optimizing MSVC}
\label{strlen_MSVC_Ox}

Now let's compile all this in MSVC 2012, with optimizations turned on (\Ox):

\lstinputlisting[caption=\Optimizing MSVC 2012 /Ob0,style=customasmx86]{patterns/10_strings/1_strlen/10_2_EN.asm}

Now it is all simpler.
Needless to say, the compiler could use registers with such efficiency
only in small functions with a few local variables.

\myindex{x86!\Instructions!INC}
\myindex{x86!\Instructions!DEC}
\INC/\DEC---are \gls{increment}/\gls{decrement} instructions, in other words: add or subtract 1 to/from a variable.

\clearpage
\myparagraph{\Optimizing MSVC + \olly}
\myindex{\olly}

We can try this (optimized) example in \olly.  Here is the first iteration:

\begin{figure}[H]
\centering
\myincludegraphics{patterns/10_strings/1_strlen/olly1.png}
\caption{\olly: first iteration start}
\label{fig:strlen_olly_1}
\end{figure}

We see that \olly found a loop and, for convenience, \IT{wrapped} its instructions in brackets.
By clicking the right button on \EAX, we can choose 
\q{Follow in Dump} and the memory window scrolls to the right place.
Here we can see the string \q{hello!} in memory.
There is at least
one zero byte after it and then random garbage.

If \olly sees a register with a valid address in it, that points to some string, 
it is shown as a string.

\clearpage
Let's press F8 (\stepover) a few times, to get to the start of the body of the loop:

\begin{figure}[H]
\centering
\myincludegraphics{patterns/10_strings/1_strlen/olly2.png}
\caption{\olly: second iteration start}
\label{fig:strlen_olly_2}
\end{figure}

We see that \EAX contains the address of the second character in the string.

\clearpage

We have to press F8 enough number of times in order to escape from the loop:

\begin{figure}[H]
\centering
\myincludegraphics{patterns/10_strings/1_strlen/olly3.png}
\caption{\olly: pointers difference to be calculated now}
\label{fig:strlen_olly_3}
\end{figure}

We see that \EAX now contains the address of zero byte that's right after the string.
Meanwhile, \EDX hasn't changed,
so it still pointing to the start of the string.

The difference between these two addresses is being calculated now.

\clearpage
The \SUB instruction just got executed:

\begin{figure}[H]
\centering
\myincludegraphics{patterns/10_strings/1_strlen/olly4.png}
\caption{\olly: \EAX to be decremented now}
\label{fig:strlen_olly_4}
\end{figure}

The difference of pointers is in the \EAX register now---7.
Indeed, the length of the \q{hello!} string is 6, 
but with the zero byte included---7.
But \TT{strlen()} must return the number of non-zero characters in the string.
So the decrement executes and then the function returns.


\myparagraph{\Optimizing GCC}

Let's check GCC 4.4.1 with optimizations turned on (\Othree key):

\lstinputlisting[style=customasmx86]{patterns/10_strings/1_strlen/10_3_gcc_O3.asm}
 
Here GCC is almost the same as MSVC, except for the presence of \MOVZX.
However, here \MOVZX could be replaced with\\
\TT{mov dl, byte ptr [eax]}.

Perhaps it is simpler for GCC's code generator to \IT{remember} 
the whole 32-bit \EDX register 
is allocated for a \Tchar variable and it then can be sure that the highest bits has no any noise 
at any point.

\label{strlen_NOT_ADD}
\myindex{x86!\Instructions!NOT}
\myindex{x86!\Instructions!XOR}

After that we also see a new instruction---\NOT. This instruction inverts all bits in the operand. \\
You can say that it is a synonym to the \TT{XOR ECX, 0ffffffffh} instruction. 
\NOT and the following \ADD calculate the pointer difference and subtract 1, just in a different way. 
At the start \ECX, where the pointer to \IT{str} is stored, gets inverted and 1 is subtracted from it.

See also: \q{\SignedNumbersSectionName}~(\myref{sec:signednumbers}).
 
In other words, at the end of the function just after loop body, these operations are executed:

\begin{lstlisting}[style=customc]
ecx=str;
eax=eos;
ecx=(-ecx)-1; 
eax=eax+ecx
return eax
\end{lstlisting}

\dots~and this is effectively equivalent to:

\begin{lstlisting}[style=customc]
ecx=str;
eax=eos;
eax=eax-ecx;
eax=eax-1;
return eax
\end{lstlisting}

Why did GCC decide it would be better? Hard to guess. 
But perhaps the both variants are equivalent in efficiency.
}
\RU{\subsubsection{x86}

\myparagraph{\NonOptimizing MSVC}

Итак, компилируем:

\lstinputlisting[style=customasmx86]{patterns/10_strings/1_strlen/10_1_msvc_RU.asm}

\myindex{x86!\Instructions!MOVSX}
\myindex{x86!\Instructions!TEST}
Здесь две новых инструкции: \MOVSX и \TEST.

\label{MOVSX}
О первой. \MOVSX предназначена для того, чтобы взять байт из какого-либо места в памяти и положить его, 
в нашем случае, в регистр \EDX. 
Но регистр \EDX~--- 32-битный. \MOVSX означает \IT{MOV with Sign-Extend}. 
Оставшиеся биты с 8-го по 31-й \MOVSX сделает единицей, если исходный байт в памяти имеет знак \IT{минус}, 
или заполнит нулями, если знак \IT{плюс}.

И вот зачем всё это.

По умолчанию в MSVC и GCC тип \Tchar~--- знаковый. Если у нас есть две переменные, одна \Tchar, а другая \Tint 
(\Tint тоже знаковый), и если в первой переменной лежит -2 (что кодируется как \TT{0xFE}) и мы просто 
переложим это в \Tint, 
то там будет \TT{0x000000FE}, а это, с точки зрения \Tint, даже знакового, будет 254, но никак не -2. 
-2 в переменной \Tint кодируется как \TT{0xFFFFFFFE}. Для того чтобы значение \TT{0xFE} из переменной типа 
\Tchar переложить 
в знаковый \Tint с сохранением всего, нужно узнать его знак и затем заполнить остальные биты. 
Это делает \MOVSX.

См. также об этом раздел
 \q{\IT{\SignedNumbersSectionName}}~(\myref{sec:signednumbers}).

Хотя конкретно здесь компилятору вряд ли была особая надобность хранить значение \Tchar в регистре \EDX, 
а не его восьмибитной части, скажем \DL. Но получилось, как получилось. Должно быть 
\gls{register allocator} компилятора сработал именно так.

\myindex{ARM!\Instructions!TEST}
Позже выполняется \TT{TEST EDX, EDX}. 
Об инструкции \TEST читайте в разделе о битовых полях~(\myref{sec:bitfields}).
Конкретно здесь эта инструкция просто проверяет состояние регистра \EDX на 0.

\myparagraph{\NonOptimizing GCC}

Попробуем GCC 4.4.1:

\lstinputlisting[style=customasmx86]{patterns/10_strings/1_strlen/10_3_gcc.asm}

\label{movzx}
\myindex{x86!\Instructions!MOVZX}
Результат очень похож на MSVC, только здесь используется \MOVZX, а не \MOVSX. 
\MOVZX означает \IT{MOV with Zero-Extend}. Эта инструкция перекладывает какое-либо значение 
в регистр и остальные биты выставляет в 0.
Фактически, преимущество этой инструкции только в том, что она позволяет 
заменить две инструкции сразу:\\
\TT{xor eax, eax / mov al, [...]}.

С другой стороны, нам очевидно, что здесь можно было бы написать вот так: \\
\TT{mov al, byte ptr [eax] / test al, al}~--- это тоже самое, хотя старшие биты \EAX будут \q{замусорены}. 
Но будем считать, что это погрешность компилятора~--- 
он не смог сделать код более экономным или более понятным. 
Строго говоря, компилятор вообще не нацелен на то, чтобы генерировать понятный (для человека) код.

\myindex{x86!\Instructions!SETcc}
Следующая новая инструкция для нас~--- \SETNZ. В данном случае, если в \AL был не ноль, 
то \TT{test al, al} выставит флаг \ZF в 0, а \SETNZ, если \TT{ZF==0} 
(\IT{NZ} значит \IT{not zero}) выставит 1 в \AL. 
Смысл этой процедуры в том, что 
\IT{если AL не ноль, выполнить переход на} \TT{loc\_80483F0}.
Компилятор выдал немного избыточный код, но не будем забывать, что оптимизация выключена.

\myparagraph{\Optimizing MSVC}
\label{strlen_MSVC_Ox}

Теперь скомпилируем всё то же самое в MSVC 2012, но с включенной оптимизацией (\Ox):

\lstinputlisting[caption=\Optimizing MSVC 2012 /Ob0,style=customasmx86]{patterns/10_strings/1_strlen/10_2_RU.asm}

Здесь всё попроще стало. Но следует отметить, что компилятор обычно может так хорошо использовать регистры 
только на небольших функциях с небольшим количеством локальных переменных.

\myindex{x86!\Instructions!INC}
\myindex{x86!\Instructions!DEC}
\INC/\DEC --- это инструкции \glslink{increment}{инкремента}-\glslink{decrement}{декремента}. Попросту говоря~--- 
увеличить на единицу или уменьшить.

\clearpage
\myparagraph{\Optimizing MSVC + \olly}
\myindex{\olly}

Можем попробовать этот (соптимизированный) пример в \olly.  Вот самая первая итерация:

\begin{figure}[H]
\centering
\myincludegraphics{patterns/10_strings/1_strlen/olly1.png}
\caption{\olly: начало первой итерации}
\label{fig:strlen_olly_1}
\end{figure}

Видно, что \olly обнаружил цикл и, для удобства, \IT{свернул} инструкции тела цикла в скобке.

Нажав правой кнопкой на \EAX, можно выбрать \q{Follow in Dump} 
и позиция в окне памяти будет как раз там, где надо.

Здесь мы видим в памяти строку \q{hello!}.
После неё имеется как минимум 1 нулевой байт, затем случайный мусор.
Если \olly видит, что в регистре содержится адрес какой-то строки, он показывает эту строку.

\clearpage
Нажмем F8 (\stepover) столько раз, чтобы текущий адрес снова был в начале тела цикла:

\begin{figure}[H]
\centering
\myincludegraphics{patterns/10_strings/1_strlen/olly2.png}
\caption{\olly: начало второй итерации}
\label{fig:strlen_olly_2}
\end{figure}

Видно, что \EAX уже содержит адрес второго символа в строке.

\clearpage
Будем нажимать F8 достаточное количество раз, чтобы выйти из цикла:

\begin{figure}[H]
\centering
\myincludegraphics{patterns/10_strings/1_strlen/olly3.png}
\caption{\olly: сейчас будет вычисление разницы указателей}
\label{fig:strlen_olly_3}
\end{figure}

Увидим, что \EAX теперь содержит адрес нулевого байта, следующего сразу за строкой.

А \EDX так и не менялся~--- он всё ещё указывает на начало строки.
Здесь сейчас будет вычисляться разница между этими двумя адресами.

\clearpage
Инструкция \SUB исполнилась:

\begin{figure}[H]
\centering
\myincludegraphics{patterns/10_strings/1_strlen/olly4.png}
\caption{\olly: сейчас будет декремент \EAX}
\label{fig:strlen_olly_4}
\end{figure}

Разница указателей сейчас в регистре \EAX~--- 7.

Действительно, длина строки \q{hello!}~--- 6, 
но вместе с нулевым байтом --- 7.
Но \TT{strlen()} должна возвращать количество ненулевых символов в строке.
Так что сейчас будет исполняться декремент и выход из функции.



\myparagraph{\Optimizing GCC}

Попробуем GCC 4.4.1 с включенной оптимизацией (ключ \Othree):

\lstinputlisting[style=customasmx86]{patterns/10_strings/1_strlen/10_3_gcc_O3.asm}

Здесь GCC не очень отстает от MSVC за исключением наличия \MOVZX. 

Впрочем, \MOVZX здесь явно можно заменить на\\
\TT{mov dl, byte ptr [eax]}.

Но возможно, компилятору GCC просто проще помнить, что у него под переменную типа \Tchar отведен целый 
32-битный регистр \EDX и быть уверенным в том, что старшие биты регистра не будут замусорены.

\label{strlen_NOT_ADD}
\myindex{x86!\Instructions!NOT}
\myindex{x86!\Instructions!XOR}
Далее мы видим новую для нас инструкцию \NOT. Эта инструкция инвертирует все биты в операнде. 
Можно сказать, что здесь это синонимично инструкции \TT{XOR ECX, 0ffffffffh}. 
\NOT и следующая за ней инструкция \ADD вычисляют разницу указателей и отнимают от результата единицу. 
Только происходит это слегка по-другому. Сначала \ECX, где хранится указатель на \IT{str}, 
инвертируется и от него отнимается единица.
См. также раздел: \q{\SignedNumbersSectionName}~(\myref{sec:signednumbers}).

Иными словами, в конце функции, после цикла, происходит примерно следующее: 

\begin{lstlisting}[style=customc]
ecx=str;
eax=eos;
ecx=(-ecx)-1; 
eax=eax+ecx
return eax
\end{lstlisting}

\dots~что эквивалентно:

\begin{lstlisting}[style=customc]
ecx=str;
eax=eos;
eax=eax-ecx;
eax=eax-1;
return eax
\end{lstlisting}

Но почему GCC решил, что так будет лучше? Трудно угадать.
Но наверное, оба эти варианта работают примерно одинаково в плане эффективности и скорости.
}
\DE{\subsubsection{x86}

\myparagraph{\NonOptimizing MSVC}

Kompilieren wir es:

\lstinputlisting[style=customasmx86]{patterns/10_strings/1_strlen/10_1_msvc_DE.asm}

\myindex{x86!\Instructions!MOVSX}
\myindex{x86!\Instructions!TEST}

Wir finden hier zwei neue Befehle: \MOVSX und \TEST.

\label{MOVSX}

Der erste --\MOVSX--nimmt ein Byte aus einer Speicheradresse und speichert den
Wert in einem 32-bit-Register.
\MOVSX steht für \IT{MOV with Sign-Extend}.
\MOVSX setzt die übrigen Bits vom 8. bis zum 31. auf 1, falls das Quellbyte
\IT{negativ} ist oder auf 0, falls es \IT{positiv} ist.

Und hier ist der Grund dafür.

Standardmäßig ist der \Tchar Datentyp in MSVC und GCC vorzeichenbehaftet
(signed). Wenn wir zwei Werte haben, einen \Tchar und einen \Tint, (\Tint ist
ebenfalls vorzeichenbehaftet) und der erste Wert enthält -2 (kodiert als
\TT{0xFE}) und wir kopieren dieses Byte in den \Tint Container, erhalten wir
\TT{0x000000FE} und dies entspricht als signed \Tint 254, aber nicht -2. Der
signed \Tint -2 wird als \TT{0xFFFFFFFE} dargestellt. Wenn wir also \TT{0xFE}
vom Datentyp \Tchar nach \Tint übertragen wollen, müssen wir das Vorzeichen
identifizieren und den Wert entsprechend erweitern. Genau dies tut der Befehl
\MOVSX.

Weitere Informationen dazu finden sich im Abschnitt
\q{\IT{\SignedNumbersSectionName}} ~(\myref{sec:signednumbers}).

Es ist schwer zu sagen, ob der Compiler tatsächlich eine \Tchar Variable in \EDX
speichern muss, er könnte auch einen 8-Bit-Registerteil (z.B. \DL) dafür
verwenden . Offenbar arbeitet der \gls{register allocator} des Compilers auf
diese Art.

\myindex{ARM!\Instructions!TEST}

Wir finden im Weiteren den Befehl \TT{TEST EDX, EDX}. 
Für mehr Informationen zum \TEST Befehl siehe auch den Abschnitt über
Bitfelder~(\myref{sec:bitfields}).
In unserem Fall überprüft der Befehl lediglich, ob der Wert im Register \EDX
gleich 0 ist.

\myparagraph{\NonOptimizing GCC}

Schauen wir uns GCC 4.4.1 an:

\lstinputlisting[style=customasmx86]{patterns/10_strings/1_strlen/10_3_gcc.asm}

\label{movzx}
\myindex{x86!\Instructions!MOVZX}

Das Ergebnis ist fast identisch mit dem von MSVC, aber hier finden wir \MOVZX
anstelle von \MOVSX. 
\MOVZX steht für \IT{MOV with Zero-Extend}. 
Dieser Befehl kopiert einen 8-Bit- oder 16-Bit-Wert in ein 32-Bit-Register und
setzt die übrigen Bits auf 0.
Tatsächlich findet dieser Befehl vor allem deshalb Anwendung, weil er es uns
erlaubt, folgendes Befehlspaar zu ersetzen:\\
\TT{xor eax, eax / mov al, [...]}.

Andererseits ist offensichtlich, dass der Compiler folgenden Code erzeugen kann:
\\
\TT{mov al, byte ptr [eax] / test al, al}--es ist fast das gleiche, aber die
oberen Bits des \EAX Registers enthalten hier Zufallswerte bzw.
sogenanntes Zufallsrauschen.
Aber bedenken wir den Nachteil des Compilers--er kann nicht leichter
verständlichen Code erzeugen. 
Genau genommen, ist der Compiler überhaupt nicht daran gebunden, (Menschen)
verständlichen Code zu erzeugen.

\myindex{x86!\Instructions!SETcc}

Der nächste neue Befehl für uns ist \SETNZ.
In diesem Fall setzt \TT{test al,al} das \ZF flag auf 0, falls \AL nicht 0
enthät, aber \SETNZ setzt \AL auf 1, falls \TT{ZF==0} (IT{NZ} steht für
\IT{non zero}).
In natürlicher Sprache, \IT{falls \AL ungleich 0, springe zu loc\_80483F0}. 
Der Compiler erzeugt leicht redundanten Code, aber bedenken wir, dass die
Optimierung hier deaktiviert ist.

\myparagraph{\Optimizing MSVC}
\label{strlen_MSVC_Ox}

Kompilieren wir nun alles in MSVC 2012 mit aktivierter Optimierung (\Ox):

\lstinputlisting[caption=\Optimizing MSVC 2012 /Ob0,style=customasmx86]{patterns/10_strings/1_strlen/10_2_DE.asm}

Jetzt ist alles einfacher.
Unnötig zu erwähnen, dass der Compiler Register mit solcher Effizienz nur in
kleinen Funktionen mit einigen wenigen lokalen Variablen verwenden kann.

\myindex{x86!\Instructions!INC}
\myindex{x86!\Instructions!DEC}
\INC/\DEC---sind \glslink{increment}{inkrement}/\glslink{decrement}{dekrement} Befehle; mit anderen Worten:
addiere oder subtrahiere 1 zu bzw. von einer Variable. 

\clearpage
\myparagraph{\Optimizing MSVC + \olly}
\myindex{\olly}

We can try this (optimized) example in \olly.  Here is the first iteration:

\begin{figure}[H]
\centering
\myincludegraphics{patterns/10_strings/1_strlen/olly1.png}
\caption{\olly: first iteration start}
\label{fig:strlen_olly_1}
\end{figure}

We see that \olly found a loop and, for convenience, \IT{wrapped} its instructions in brackets.
By clicking the right button on \EAX, we can choose 
\q{Follow in Dump} and the memory window scrolls to the right place.
Here we can see the string \q{hello!} in memory.
There is at least
one zero byte after it and then random garbage.

If \olly sees a register with a valid address in it, that points to some string, 
it is shown as a string.

\clearpage
Let's press F8 (\stepover) a few times, to get to the start of the body of the loop:

\begin{figure}[H]
\centering
\myincludegraphics{patterns/10_strings/1_strlen/olly2.png}
\caption{\olly: second iteration start}
\label{fig:strlen_olly_2}
\end{figure}

We see that \EAX contains the address of the second character in the string.

\clearpage

We have to press F8 enough number of times in order to escape from the loop:

\begin{figure}[H]
\centering
\myincludegraphics{patterns/10_strings/1_strlen/olly3.png}
\caption{\olly: pointers difference to be calculated now}
\label{fig:strlen_olly_3}
\end{figure}

We see that \EAX now contains the address of zero byte that's right after the string.
Meanwhile, \EDX hasn't changed,
so it still pointing to the start of the string.

The difference between these two addresses is being calculated now.

\clearpage
The \SUB instruction just got executed:

\begin{figure}[H]
\centering
\myincludegraphics{patterns/10_strings/1_strlen/olly4.png}
\caption{\olly: \EAX to be decremented now}
\label{fig:strlen_olly_4}
\end{figure}

The difference of pointers is in the \EAX register now---7.
Indeed, the length of the \q{hello!} string is 6, 
but with the zero byte included---7.
But \TT{strlen()} must return the number of non-zero characters in the string.
So the decrement executes and then the function returns.


\myparagraph{\Optimizing GCC}

Schauen wir uns GCC 4.4.1 mit aktiverter Optimierung (\Othree key) an:

\lstinputlisting[style=customasmx86]{patterns/10_strings/1_strlen/10_3_gcc_O3.asm}
 
Hier erzeugt GCC fast identischen Code zu MSVC, außer dass hier ein \MOVZX
auftritt. 
In der Tat könnte \MOVZX hier durch \TT{mov dl, byte ptr [eax]} ersetzt werden.
 
Möglicherweise ist es einfacher für den GCC Code Generator sich daran zu
\IT{erinnern}, dass das gesamte 32-bit-\EDX Register für eine \Tchar Variable
reserviert ist und so sicherzustellen, dass die oberen Bits zu keinem Zeitpunkt
Zufallsrauschen enthalten.

\label{strlen_NOT_ADD}
\myindex{x86!\Instructions!NOT}
\myindex{x86!\Instructions!XOR}

Danach finden wir also einen neuen Befehl--\NOT. Dieser Befehl kippt alle Bits
in seinem Operanden.\\
Man kann sagen, dass es sich um ein Synonym zum Befehl \TT{XOR ECX, 0ffffffffh}
handelt. 
\NOT und das darauf folgende \ADD berechnen die Differenz im Pointer und
subtrahieren 1, nur auf eine andere Art und Weise. 
Zu Beginn wird \ECX, in dem der Pointer auf \IT{str} gespeichert ist, invertiert
und vom Ergebnis wird 1 abgezogen.

Hierzu siehe auch: \q{\SignedNumbersSectionName}~(\myref{sec:signednumbers}).
 
Mit anderen Worten, am Ende der Funktion, direkt nach dem Schleifenkörper,
werden die folgenden Befehle ausgeführt:

\begin{lstlisting}[style=customc]
ecx=str;
eax=eos;
ecx=(-ecx)-1; 
eax=eax+ecx
return eax
\end{lstlisting}

\dots~und das ist äquivalent zu:

\begin{lstlisting}[style=customc]
ecx=str;
eax=eos;
eax=eax-ecx;
eax=eax-1;
return eax
\end{lstlisting}

Warum GCC entschieden hat, dass das eine besser ist als das andere? Schwer zu
sagen.
Möglicherweise sind aber beide Variante gleichermaßen effizient.
}
\subsubsection{ARM}

% subsubsections
\EN{\myparagraph{32-bit ARM}

\mysubparagraph{\NonOptimizingXcodeIV (\ARMMode)}

\lstinputlisting[caption=\NonOptimizingXcodeIV (\ARMMode),label=ARM_leaf_example7,style=customasmARM]{patterns/10_strings/1_strlen/ARM/xcode_ARM_O0_EN.asm}

Non-optimizing LLVM generates too much code, however, here we can see how the function works with 
local variables in the stack.
There are only two local variables in our function: \IT{eos} and \IT{str}.
In this listing, generated by \IDA, we have manually renamed \IT{var\_8} and \IT{var\_4} to \IT{eos} and \IT{str}.

The first instructions just saves the input values into both \IT{str} and \IT{eos}.

The body of the loop starts at label \IT{loc\_2CB8}.

The first three instruction in the loop body (\TT{LDR}, \ADD, \TT{STR}) load the value of \IT{eos} into \Reg{0}. 
Then the value is \glslink{increment}{incremented} and saved back into \IT{eos}, which is located in the stack.

\myindex{ARM!\Instructions!LDRSB}
The next instruction,  \TT{LDRSB R0, [R0]} (\q{Load Register Signed Byte}), loads a byte from memory at the address stored in \Reg{0} and sign-extends it to 32-bit
\footnote{The Keil compiler treats the \Tchar type as signed, just like MSVC and GCC.}.
\myindex{x86!\Instructions!MOVSX}
This is similar to the \MOVSX instruction in x86.

The compiler treats this byte as signed since the \Tchar type is signed according to the C standard.
It was already written about it~(\myref{MOVSX}) in this section, in relation to x86.

\myindex{Intel!8086}
\myindex{Intel!8080}
\myindex{ARM}

It has to be noted that it is impossible to use 8- or 16-bit part 
of a 32-bit register in ARM separately of the whole register,
as it is in x86.

Apparently, it is because x86 has a huge history of backwards compatibility with its ancestors 
up to the 16-bit 8086 and even 8-bit 8080,
but ARM was developed from scratch as a 32-bit RISC-processor.

Consequently, in order to process separate bytes in ARM, one has to use 32-bit registers anyway.

So, \TT{LDRSB} loads bytes from the string into \Reg{0}, one by one.
The following \CMP and \ac{BEQ} instructions check if the loaded byte is 0.
If it's not 0, control passes to the start of the body of the loop.
And if it's 0, the loop ends.

At the end of the function, the difference between 
\IT{eos} and \IT{str} is calculated, 1 is subtracted from it, and resulting value is returned
via \Reg{0}.

N.B. Registers were not saved in this function.
\myindex{ARM!\Registers!scratch registers}

That's because in the ARM calling convention registers \Reg{0}-\Reg{3} are \q{scratch registers}, 
intended for arguments passing,
and we're not required to restore their value when the function exits, 
since the calling function will not use them anymore.
Consequently, they may be used for anything we want.

No other registers are used here, so that is why we have nothing to save on the stack.

Thus, control may be returned back to calling function by a simple jump (\TT{BX}),
to the address in the \ac{LR} register.

\mysubparagraph{\OptimizingXcodeIV (\ThumbMode)}

\lstinputlisting[caption=\OptimizingXcodeIV (\ThumbMode),style=customasmARM]{patterns/10_strings/1_strlen/ARM/xcode_thumb_O3.asm}

As optimizing LLVM concludes, \IT{eos} and \IT{str} do not need space on the stack, and can always be stored in registers.

Before the start of the loop body, \IT{str} is always in \Reg{0}, 
and \IT{eos}---in \Reg{1}.

\myindex{ARM!\Instructions!LDRB.W}
The \TT{LDRB.W R2, [R1],\#1} instruction loads a byte from the memory at the address stored in \Reg{1}, to \Reg{2}, sign-extending it to a 32-bit value, but not just that.
\TT{\#1} at the instruction's end is implies \q{Post-indexed addressing}, which means that 1 is to be added to \Reg{1} after the byte is loaded.
Read more about it: \myref{ARM_postindex_vs_preindex}.

Then you can see \CMP and \ac{BNE} in the body of the loop, these instructions continue looping until 0 is found in the string.

\myindex{ARM!\Instructions!MVNS}
\myindex{x86!\Instructions!NOT}
\TT{MVNS}\footnote{MoVe Not} (inverts all bits, like \NOT in x86) and \ADD instructions compute $eos - str - 1$.
In fact, these two instructions compute $R0 = ~str + eos$, 
which is effectively equivalent to what was in the source code, and why it is so, was already explained here
~(\myref{strlen_NOT_ADD}).

Apparently, LLVM, just like GCC, concludes that this code can be shorter (or faster).

\mysubparagraph{\OptimizingKeilVI (\ARMMode)}

\lstinputlisting[caption=\OptimizingKeilVI (\ARMMode),label=ARM_leaf_example6,style=customasmARM]{patterns/10_strings/1_strlen/ARM/Keil_ARM_O3.asm}

\myindex{ARM!\Instructions!SUBEQ}

Almost the same as what we saw before, with the exception that the $str - eos - 1$ 
expression can be computed not at the function's end, but right in the body of the loop.
The \TT{-EQ} suffix, as we may recall, implies that the instruction executes only if the operands in
the \CMP that has been executed before were equal to each other.
Thus, if \Reg{0} contains 0, both \TT{SUBEQ} instructions executes and result is left in the \Reg{0} register.

}
\RU{\myparagraph{32-битный ARM}

\mysubparagraph{\NonOptimizingXcodeIV (\ARMMode)}

\lstinputlisting[caption=\NonOptimizingXcodeIV (\ARMMode),label=ARM_leaf_example7,style=customasmARM]{patterns/10_strings/1_strlen/ARM/xcode_ARM_O0_RU.asm}

Неоптимизирующий LLVM генерирует слишком много кода. Зато на этом примере можно посмотреть, 
как функции работают с локальными переменными в стеке.

В нашей функции только локальных переменных две --- это два указателя:
\IT{eos} и \IT{str}.
В этом листинге сгенерированном при помощи \IDA мы переименовали \IT{var\_8} и \IT{var\_4} в \IT{eos} и \IT{str} вручную.%

Итак, первые несколько инструкций просто сохраняют входное значение в обоих переменных \IT{str} и \IT{eos}.

С метки \IT{loc\_2CB8} начинается тело цикла.

Первые три инструкции в теле цикла (\TT{LDR}, \ADD, \TT{STR}) 
загружают значение \IT{eos} в \Reg{0}. 
Затем происходит инкремент значения и оно сохраняется в локальной переменной \IT{eos} расположенной 
в стеке.

\myindex{ARM!\Instructions!LDRSB}
Следующая инструкция \TT{LDRSB R0, [R0]} (\q{Load Register Signed Byte}) 
загружает байт из памяти по адресу \Reg{0}, расширяет его до 32-бит считая его знаковым (signed) 
и сохраняет в \Reg{0}
\footnote{Компилятор Keil считает тип \Tchar знаковым, как и MSVC и GCC.}.
\myindex{x86!\Instructions!MOVSX}
Это немного похоже на инструкцию \MOVSX в x86.
Компилятор считает этот байт знаковым (signed), потому что тип \Tchar по стандарту Си~--- знаковый.

Об этом уже было немного написано~(\myref{MOVSX}) в этой же секции, но посвященной x86.

\myindex{Intel!8086}
\myindex{Intel!8080}
\myindex{ARM}
Следует также заметить, что в ARM нет возможности использовать 8-битную или 16-битную часть 
регистра, как это возможно в x86.

Вероятно, это связано с тем, что за x86 тянется длинный шлейф совместимости со своими предками, 
вплоть до 16-битного 8086 и даже 8-битного 8080, 
а ARM разрабатывался с чистого листа как 32-битный RISC-процессор.

Следовательно, чтобы работать с отдельными байтами на ARM, так или иначе придется использовать 
32-битные регистры.

Итак, \TT{LDRSB} загружает символы из строки в \Reg{0}, по одному.

Следующие инструкции \CMP и \ac{BEQ} проверяют, является ли этот символ 0.

Если не 0, то происходит переход на начало тела цикла.
А если 0, выходим из цикла.

В конце функции вычисляется разница между 
\IT{eos} и \IT{str}, вычитается единица, и вычисленное 
значение возвращается через \Reg{0}.

N.B. В этой функции не сохранялись регистры.
\myindex{ARM!\Registers!scratch registers}
По стандарту регистры \Reg{0}-\Reg{3} называются также \q{scratch registers}.
Они предназначены для передачи аргументов и 
их значения не нужно восстанавливать при выходе из функции, потому что они больше не нужны в вызывающей функции.
Таким образом, их можно использовать как захочется.

А~так~как никакие больше регистры не используются, то и сохранять нечего.

Поэтому управление можно вернуть вызывающей функции 
простым переходом (\TT{BX}) по адресу в регистре \ac{LR}.

\mysubparagraph{\OptimizingXcodeIV (\ThumbMode)}

\lstinputlisting[caption=\OptimizingXcodeIV (\ThumbMode),style=customasmARM]{patterns/10_strings/1_strlen/ARM/xcode_thumb_O3.asm}

Оптимизирующий LLVM решил, что под переменные \IT{eos} и \IT{str} выделять место в стеке не обязательно,
и эти переменные можно хранить прямо в регистрах.

Перед началом тела цикла \IT{str} будет находиться в \Reg{0}, 
а \IT{eos} --- в \Reg{1}.

\myindex{ARM!\Instructions!LDRB.W}
Инструкция \TT{LDRB.W R2, [R1],\#1} загружает в \Reg{2} байт из памяти по адресу \Reg{1}, 
расширяя его как знаковый (signed), до 32-битного
значения, но не только это.

\TT{\#1} в конце инструкции означает \q{Post-indexed addressing},
т.е. после загрузки байта к \Reg{1} добавится единица.

Читайте больше об этом: \myref{ARM_postindex_vs_preindex}.

Далее в теле цикла можно увидеть \CMP и \ac{BNE}. Они продолжают работу цикла до тех пор, 
пока не будет встречен 0.

\myindex{ARM!\Instructions!MVNS}
\myindex{x86!\Instructions!NOT}
После конца цикла \TT{MVNS}\footnote{MoVe Not} (инвертирование всех бит, \NOT в x86) и \ADD вычисляют $eos - str - 1$.
На самом деле, эти две инструкции вычисляют $R0 = ~str + eos$, 
что эквивалентно тому, что было в исходном коде. Почему это так, уже было описано чуть раньше, здесь 
~(\myref{strlen_NOT_ADD}).

Вероятно, LLVM, как и GCC, посчитал, что такой код может быть короче (или быстрее).

\mysubparagraph{\OptimizingKeilVI (\ARMMode)}

\lstinputlisting[caption=\OptimizingKeilVI (\ARMMode),label=ARM_leaf_example6,style=customasmARM]{patterns/10_strings/1_strlen/ARM/Keil_ARM_O3.asm}

\myindex{ARM!\Instructions!SUBEQ}
Практически то же самое, что мы уже видели, за тем исключением, что выражение $str - eos - 1$ 
может быть вычислено не в самом конце функции, а прямо в теле цикла.

Суффикс \TT{-EQ} означает, что инструкция будет выполнена только
если операнды в исполненной перед этим инструкции \CMP были равны.

Таким образом, если в \Reg{0} будет 0,
обе инструкции \TT{SUBEQ} исполнятся и результат останется в \Reg{0}.

}
\DE{\myparagraph{32-bit ARM}

\mysubparagraph{\NonOptimizingXcodeIV (\ARMMode)}

\lstinputlisting[caption=\NonOptimizingXcodeIV
(\ARMMode),label=ARM_leaf_example7,style=customasmARM]{patterns/10_strings/1_strlen/ARM/xcode_ARM_O0_DE.asm}

Der nicht optimierende LLVM erzeugt zu viel Code, aber wir können wir erkennen
wir die Funktion mit lokalen Variablen auf dem Stack arbeitet. 
Es gibt nur zwei lokale Variablen in unserer Funktion: \IT{eos} und \IT{str}.
In folgenden von \IDA erzeugten Listing, sind die Variablen \IT{var\_8} und
\IT{var\_4} in \IT{eos} bzw. \IT{str} umbenannt.

Der erste Befehl speichert lediglich bei Eingabewerte in \IT{str} und \IT{eos}.

Der Körper der Schleife startet beim Label \IT{loc\_2CB8}.

Die ersten drei Befehle des Schleifenkörpers (\TT{LDR}, \ADD, \TT{STR}) laden
den Wert von \IT{eos} nach \Reg{0}. 
Anschließend wird der Wert erhöht und zurück in \IT{eos} auf den Stack
geschrieben.

\myindex{ARM!\Instructions!LDRSB}
Der folgende Befehl, \TT{LDRSB R0, [R0]} (\q{Load Register Signed Byte}), lädt
ein Byte aus dem Speicher von der Adresse in \Reg{0} und erweitert es mit
Vorzeichen auf 32-bit.
\footnote{Der Keil Compiler behandelt den Typ \Tchar als signed, genau wie
MSVC und GCC.}.
\myindex{x86!\Instructions!MOVSX}
Dies ist vergleichbar zum \MOVSX Befehl in x86.

Der Compiler behandelt dieses Byte als signed, das der \Tchar Typ nach dem
C-Standard ebenfalls signed ist. 
Dies wurde in Bezug auf x86 in diesem Abschnitt bereits in~(\myref{MOVSX})
beschrieben.

\myindex{Intel!8086}
\myindex{Intel!8080}
\myindex{ARM}

Man beachte, dass es in ARM unmöglich ist, einen 8- oder 16-bit-Teil eines
32-bit-Registers alleine zu verwenden, anderes als in x86.

Dies rührt daher, dass x86 eine große Bandbreite and Kompatibilität mit
Vorgängerversionen besitzt, bis hin zum 16-bit 8086 oder sogar dem 8-bit 8080,
ARM auf der anderen Seite jedoch von Beginn an als 32-bit RISC-Prozessor geplant
wurde.

Infolgedessen müssen auch um einzelne Bytes in ARM zu verarbeiten, stets
komplette 32-bit-Register verwendet werden.

Der Befehl \TT{LDRSB} lädt nun die Bytes des String einzeln nach \Reg{0}.
Die nachfolgenden \CMP und \ac{BEQ} Befehle prüfen, ob das aktuelle Byte 0 ist.
Wenn nicht, beginnt der Körper der Schleife erneut.
Und wenn das aktuelle Byte 0 ist, dann wird die Schleife beendet.

Am Ende der Funktion wird die Differenz zwischen \IT{eos} und \IT{str}
berechnet, 1 vom Ergebnis abgezogen und das Resultat über das Register \Reg{0}
zurückgegeben.

N.B. Register wurden in dieser Funktion nicht gespeichert.
\myindex{ARM!\Registers!scratch registers}
Das liegt daran, dass gemäß der ARM Aufrufkonventionen die Register \Reg{0} bis
\Reg{3} sogenannte \q{scratch register} sind, vorgesehen für Parameterübergaben.
Deshalb ist es nicht notwendig ihren Inhalt am Ende der Funktion
wiederherzustellen, denn die aufrufende Funktion wird diese Werte nicht weiter
verwenden. 
Im weiteren können diese für alles Mögliche benutzt werden.

Es werden hier keine weiteren Register verwendet, sodass wir nichts auf dem
Stack speichern müssen. 

Dadurch kann der control flow über einen einfachen Sprung (\TT{BX}) an die
aufrufende Funktion an der Adresse im \ac{LR} Register übergeben werden.

\mysubparagraph{\OptimizingXcodeIV (\ThumbMode)}

\lstinputlisting[caption=\OptimizingXcodeIV (\ThumbMode),style=customasmARM]{patterns/10_strings/1_strlen/ARM/xcode_thumb_O3.asm}
Der optimierende LLVM entscheidet also, dass \IT{eos} und \IT{str} keinen Platz
auf dem Stack benötigen, sondern stets in Registern gespeichert werden können. 

Vor dem Anfang des Schleifenkörpers befindet sich \IT{str} stets in \Reg{0} und
\IT{eos} in \Reg{1}.

\myindex{ARM!\Instructions!LDRB.W}
Der Befehl \TT{LDRB.W R2, [R1],\#1} lädt ein Byte aus dem Speicher von der
Adresse aus \Reg{1} nach \Reg{2}, erweitert es zum einem signed 32-bit-Wert und
mehr noch: Das \TT{\#1} am Ende des Befehl bewirkt \q{Post-indexed addressing},
was bedeutet, dass 1 zum Register \Reg{1} addiert wird, nachdem das Byte geladen
wurde.
Mehr zum Thema:\myref{ARM_postindex_vs_preindex}. 

Des Weiteren finden wir \CMP und \ac{BNE} im Körper der Schleife; diese Befehle
werden durchlaufen, bis 0 im String gefunden wurde.

\myindex{ARM!\Instructions!MVNS}
\myindex{x86!\Instructions!NOT}
\TT{MVNS}\footnote{MoVe Not} (invertiert alle Bits wie \NOT in x86) und \ADD
Befehle berechnen $eos - str - 1$.
Tatsächlich berechnen diese beiden Befehle $R0 =
~str + eos$, was äquivalent zur Formulierung im Quellcode ist und die Begründung
dazu wurde bereits hier gegeben ~(\myref{strlen_NOT_ADD}).

Offenbar befindet LLVM genau wie GCC, dass diese Code kürzer (oder schneller)
ist.

\mysubparagraph{\OptimizingKeilVI (\ARMMode)}

\lstinputlisting[caption=\OptimizingKeilVI (\ARMMode),label=ARM_leaf_example6,style=customasmARM]{patterns/10_strings/1_strlen/ARM/Keil_ARM_O3.asm}

\myindex{ARM!\Instructions!SUBEQ}

Fast das gleiche wie zuvor, mit der Änderung, dass der $str - eos - 1$
Ausdruck nicht am Ende der Funktion, sondern mitten in der Schleife berechnet
wird. Wir erinnern uns, dass der \TT{-EQ} Suffix bedeutet, dass die dieser
Befehl nur dann ausgeführt wird, wenn die Operanden im \CMP direkt davor gleich
waren.
Dadurch werden beide \TT{SUBEQ} Befehle ausgeführt, falls das \Reg{0} Register 0
enthält und das Ergebnis verbleibt in \Reg{0}.
}

\EN{\myparagraph{ARM64}

\mysubparagraph{\Optimizing GCC (Linaro) 4.9}

\lstinputlisting[style=customasmARM]{patterns/10_strings/1_strlen/ARM/ARM64_GCC_O3_EN.lst}

The algorithm is the same as in \myref{strlen_MSVC_Ox}: 
find a zero 
byte, calculate the difference between the pointers and decrement the result by 1.
Some comments were added by the author of this book.

The only thing worth noting is that our example is somewhat wrong: \\
\TT{my\_strlen()} returns 32-bit \Tint, while it has to return \TT{size\_t} or another 64-bit type.

The reason is that, theoretically, \TT{strlen()} can be called for a huge blocks in memory that exceeds
4GB, so it must able to return a 64-bit value on 64-bit platforms.

Because of my mistake, the last \SUB instruction operates on a 32-bit part of register, while the penultimate
\SUB instruction works on full the 64-bit register (it calculates the difference between the pointers).

It's my mistake, it is better to leave it as is, as an example of how the code could look like in such case.

\mysubparagraph{\NonOptimizing GCC (Linaro) 4.9}

\lstinputlisting[style=customasmARM]{patterns/10_strings/1_strlen/ARM/ARM64_GCC_O0_EN.lst}

It's more verbose.
The variables are often tossed here to and from memory (local stack).
The same mistake here: the decrement operation happens on a 32-bit register part.

}
\RU{\myparagraph{ARM64}

\mysubparagraph{\Optimizing GCC (Linaro) 4.9}

\lstinputlisting[style=customasmARM]{patterns/10_strings/1_strlen/ARM/ARM64_GCC_O3_RU.lst}

Алгоритм такой же как и в \myref{strlen_MSVC_Ox}: 
найти нулевой байт, затем вычислить разницу между указателями, затем отнять 1 от результата.
Комментарии добавлены автором книги.

Стоит добавить, что наш пример имеет ошибку: \TT{my\_strlen()}
возвращает 32-битный \Tint, тогда как должна возвращать \TT{size\_t} или иной 64-битный тип.

Причина в том, что теоретически, \TT{strlen()} можно вызывать для огромных блоков в памяти,
превышающих 4GB, так что она должна иметь возможность вернуть 64-битное значение на 64-битной платформе.

Так что из-за моей ошибки, последняя инструкция \SUB работает над 32-битной частью регистра, тогда
как предпоследняя \SUB работает с полными 64-битными частями (она вычисляет разницу между указателями).

Это моя ошибка, но лучше оставить это как есть, как пример кода, который возможен в таком случае.

\mysubparagraph{\NonOptimizing GCC (Linaro) 4.9}

\lstinputlisting[style=customasmARM]{patterns/10_strings/1_strlen/ARM/ARM64_GCC_O0_RU.lst}

Более многословно.
Переменные часто сохраняются в память и загружаются назад (локальный стек).
Здесь та же ошибка: операция декремента происходит над 32-битной частью регистра.

}
\DE{\myparagraph{ARM64}

\mysubparagraph{\Optimizing GCC (Linaro) 4.9}

\lstinputlisting[style=customasmARM]{patterns/10_strings/1_strlen/ARM/ARM64_GCC_O3_DE.lst}

Der Algorithmus ist der gleiche wie in \myref{strlen_MSVC_Ox}: 
finde ein Nullbyte, berechne die Differenz zwischen den Pointern und subtrahiere
1 vom Ergebnis. 
Einige Kommentare wurden vom Autor hinzugefügt.

Die einzig bemerkenswerte Sache ist, dass unser Beispiel in gewisser Weise
fehlerhaft ist:\\
\TT{my\_strlen()} liefert einen 32-bit \Tint, obwohl es \TT{size\_t} oder
einen anderen 64-bit Typ zurückliefern müsste.

Der Grund dafür ist, dass \TT{strlen()} theoretisch für einen sehr großen
Speicherblock, größer als 4GB, aufgerufen werden könnte und deshalb auf einer
64-bit-Plattform in der Lage sein muss, einen 64-bit-Wert zurückzuliefern.

Aufgrund meines Fehlers, arbeitet der letzte \SUB Befehl nur mit einem
32-bit-Teil des Registers, wohingegen der vorletzte \SUB Befehl mit dem
kompletten 64-bit-Register arbeitet (und die Differenz zwischen den Pointer
berechnet).

Es handelt sich um einen Fehler von mir, und es ist besser es so zu lassen, als
ein Lehrbeispiel wie Code in einem derartigen Fall aussehen kann.

\mysubparagraph{\NonOptimizing GCC (Linaro) 4.9}

\lstinputlisting[style=customasmARM]{patterns/10_strings/1_strlen/ARM/ARM64_GCC_O0_DE.lst}

Es ist umfangreicher. 
Die Variablen werden hier viel im Speicher (lokaler Stack) herumgeschoben.
Der obige Fehler findet sich auch hier: das Dekrementieren geschieht nur in
einem 32-bit-Teil des Registers.
}


\EN{\subsubsection{MIPS}

\lstinputlisting[caption=\Optimizing GCC 4.4.5 (IDA),style=customasmMIPS]{patterns/10_strings/1_strlen/MIPS_O3_IDA_EN.lst}

\myindex{MIPS!\Instructions!NOR}
\myindex{MIPS!\Pseudoinstructions!NOT}

MIPS lacks a \NOT instruction, but has \NOR which is \TT{OR~+~NOT} operation.

This operation is widely used in digital electronics\footnote{NOR is called \q{universal gate}}.
\index{Apollo Guidance Computer}
For example, the Apollo Guidance Computer used in the Apollo program, 
was built by only using 5600 NOR gates:
[Jens Eickhoff, \IT{Onboard Computers, Onboard Software and Satellite Operations: An Introduction}, (2011)].
But NOR element isn't very popular in computer programming.

So, the NOT operation is implemented here as \TT{NOR~DST,~\$ZERO,~SRC}.

From fundamentals \myref{sec:signednumbers} we know that bitwise inverting a signed number is the same 
as changing its sign and subtracting 1 from the result.

So what \NOT does here is to take the value of $str$ and transform it into $-str-1$.
The addition operation that follows prepares result.

}
\RU{\subsubsection{MIPS}

\lstinputlisting[caption=\Optimizing GCC 4.4.5 (IDA),style=customasmMIPS]{patterns/10_strings/1_strlen/MIPS_O3_IDA_RU.lst}

\myindex{MIPS!\Instructions!NOR}
\myindex{MIPS!\Pseudoinstructions!NOT}
В MIPS нет инструкции \NOT, но есть \NOR~--- операция \TT{OR~+~NOT}.

Эта операция широко применяется в цифровой электронике\footnote{\NOR называют \q{универсальным элементом}}.
\index{Apollo Guidance Computer}
Например, космический компьютер Apollo Guidance Computer использовавшийся в программе \q{Аполлон} был
построен исключительно на 5600 элементах \NOR: 
[Jens Eickhoff, \IT{Onboard Computers, Onboard Software and Satellite Operations: An Introduction}, (2011)].
Но элемент NOR не очень популярен в программировании.

Так что операция \NOT реализована здесь как \TT{NOR~DST,~\$ZERO,~SRC}.

Из фундаментальных знаний \myref{sec:signednumbers}, мы можем знать, что побитовое инвертирование знакового
числа это то же что и смена его знака с вычитанием 1 из результата.

Так что \NOT берет значение $str$ и трансформирует его в $-str-1$.

Следующая операция сложения готовит результат.

}
\DE{\subsubsection{MIPS}

\lstinputlisting[caption=\Optimizing GCC 4.4.5
(IDA)]{patterns/10_strings/1_strlen/MIPS_O3_IDA_DE.lst}

\myindex{MIPS!\Instructions!NOR}
\myindex{MIPS!\Pseudoinstructions!NOT}

MIPS besitzt keinen \NOT Befehl, dafür aber den Befehl \NOT, welcher der
Funktion \TT{OR~+~NOT} entspricht.

Diese Funktion wird häufig in der Digitaltechnik verwendet\footnote{NOR wird
 \q{universelles Gatter} genannt}.

\index{Apollo Guidance Computer}
Der Apollo Guidance Computer, der im Apollo Programm der NASA verwendet wurde,
bestand beispielsweise ausschließlich aus 5600 \NOR Gattern:
[Jens Eickhoff, \IT{Onboard Computers, Onboard Software and Satellite
Operations: An Introduction}, (2011)].
In der Programmierung ist die Funktion \NOT nicht besonders beliebt. 

Die \NOT Funktion ist hier also durch \TT{NOR~DST,~\$ZERO,~SRC} implementiert.

Aus dem Grundlagenteil \myref{sec:signednumbers} wissen wir, dass das bitweise
invertieren einer vorzeichenbehafteten Zahl gerade einem Wechsel des Vorzeichens
mit anschließender Subtraktion von 1 entspricht. 

Was \NOT hier also tut, ist, den Wert von $str$ in $-str-1$ umzuwandeln. 
Die folgende Addition bereitet das Ergebnis vor.
}
