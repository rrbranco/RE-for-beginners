\subsection{Limites de strings}

\myindex{win32!GetOpenFileName}
É interessante notar como parametros são passados para a função win32 \IT{GetOpenFileName()}.
Afim de usá-la, deve-se inserir uma lista de extensões permitidas:

\begin{lstlisting}[style=customc]
    OPENFILENAME *LPOPENFILENAME;
    ...
    char * filter = "Text files (*.txt)\0*.txt\0MS Word files (*.doc)\0*.doc\0\0";
    ...
    LPOPENFILENAME = (OPENFILENAME *)malloc(sizeof(OPENFILENAME));
    ...
    LPOPENFILENAME->lpstrFilter = filter;
    ...

    if(GetOpenFileName(LPOPENFILENAME))
    {
        ...
\end{lstlisting}

O que acontece aqui é que a lista de strings é passada para \IT{GetOpenFileName()}.
Não é dificil parseá-la: onde você encontrar um único byte zero, é um item.
Onde você encontrar dois bytes zero, é o fim da lista.
Se você passar esta string para \printf, ele irá tratar o primeiro item como uma única string.

Então isto é uma string ou...?
É melhor dizer que é um buffer contendo algumas C-strings terminadas em zero, que podem ser guardadas e 
processadas como um todo.

\myindex{\CStandardLibrary!strtok}
Outro exemplo é a função \IT{strtok()}. Ela recebe uma string e escreve bytes zero no meio dela.
Ela transforma a string de entrada em uma forma de buffer, que tem algumas C-strings terminadas em zero.

