% TODO translate to RU
\subsection{Sistemas de Numeração}

Nós humanos estamos acostumados com o sistema de numeração decimal, provavelmente porque a maioria de nós temos 10 dedos.
Todavia, o número 10 não tem nenhum significado especial na matemática ou ciências.
No campo da eletrônica digital, o sistema de numeração é naturalmente o sistema binário: onde 0 é a representação para falta de corrente e 1 para a presença de corrente elétrica.
O número 10 em binário representa o 2 no sistema decimal; 100 em binário é 4 no decimal e assim por diante.

Se o sistema de numeração decimal tem 10 dígitos, ele tem \IT{radix} (ou \IT{base}) 10.
Sistema de numeração binária tem \IT{radix}  2.

Importante lembrar:
1) \IT{number} é um número, enquanto \IT{digit} is a term of writing system and is usually one character;
2) O número nao muda quando convertido para outra base, a notação escrita é (e sua forma de representação na \ac{RAM}).

Como converter um número de uma base para outra?

Positional notation is used almost everywhere, this means, a digit has some weight depending on where it is placed inside of number.
If 2 is placed at the rightmost place, it's 2.
If it is placed at the place one digit before rightmost, it's 20.

O que $1234$ significa?

$10^3 \cdot 1 + 10^2 \cdot 2 + 10^1 \cdot 3 + 1 \cdot 4$ = 1234 or
$1000 \cdot 1 + 100 \cdot 2 + 10 \cdot 3 + 4 = 1234$

A mesma coisa ocorre para números binários, mas na base 2 ao invés de 10.
O que 0b101011 siginifica?

$2^5 \cdot 1 + 2^4 \cdot 0 + 2^3 \cdot 1 + 2^2 \cdot 0 + 2^1 \cdot 1 + 2^0 \cdot 1 = 43$ or
$32 \cdot 1 + 16 \cdot 0 + 8 \cdot 1 + 4 \cdot 0 + 2 \cdot 1 + 1 = 43$

Positional notation can be opposed to non-positional notation such as Roman numeric system
\footnote{About numeric system evolution, see \InSqBrackets{\TAOCPvolII{}, 195--213.}}.
Perhaps, humankind switched to positional notation because it's easier to do basic operations (addition, multiplication, etc.) on paper by hand.

Assim, números binários podem ser adicionados, subtraídos da mesma forma que é ensinada nas escolas, porém, apenas dois dígitos estão disponíveis.

Números binários são agrupados e representados no código fonte, assim é utilizado o sistema de numeração hexadecimal.

A base hexadecimal utiliza digitos de 0..9 e também 6 caractéres latínos: A..F.

Cada dígito hexadecimal usa 4 bits ou 4 digitos binários, então é muito fácil converter um número binário para hexadecimal e vice-versa, até manualmente de cabeça.

\begin{center}
\begin{longtable}{ | l | l | l | }
\hline
\HeaderColor hexadecimal & \HeaderColor binário & \HeaderColor decimal \\
\hline
0	&0000	&0 \\
1	&0001	&1 \\
2	&0010	&2 \\
3	&0011	&3 \\
4	&0100	&4 \\
5	&0101	&5 \\
6	&0110	&6 \\
7	&0111	&7 \\
8	&1000	&8 \\
9	&1001	&9 \\
A	&1010	&10 \\
B	&1011	&11 \\
C	&1100	&12 \\
D	&1101	&13 \\
E	&1110	&14 \\
F	&1111	&15 \\
\hline
\end{longtable}
\end{center}

How to understand, which radix is used in a specific place?

Decimal numbers are usually written as is, i.e., 1234. But some assemblers allows to make emphasis on decimal radix and this number can be written with "d" suffix: 1234d.

Binary numbers sometimes prepended with "0b" prefix: 0b100110111 (\ac{GCC} has non-standard language extension for this\footnote{\url{https://gcc.gnu.org/onlinedocs/gcc/Binary-constants.html}}).
There is also another way: "b" suffix, for example: 100110111b.
I'll try to stick to "0b" prefix throughout the book for binary numbers.

Hexadecimal numbers are prepended with "0x" prefix in \CCpp and other \ac{PL}s: 0x1234ABCD.
Or they are has "h" suffix: 1234ABCDh---this is common way of representing them in assemblers and debuggers.
If the number is started with A..F digit, 0 is to be added before: 0ABCDEFh.
There was also convention that was popular in 8-bit home computers era, using \$ prefix, like \$ABCD.
I'll try to stick to "0x" prefix throughout the book for hexadecimal numbers.

Should one learn to convert numbers in mind? A table of 1-digit hexadecimal numbers can easily be memorized.
As of larger numbers, probably, it's not worth to torment yourself.

Perhaps, the most visible to all people hexadecimal numbers are in \ac{URL}s.
This is the way how non-Latin characters are encoded.
For example:
\url{https://en.wiktionary.org/wiki/na\%C3\%AFvet\%C3\%A9} is the \ac{URL} of Wiktionary article about \q{naïveté} word.

\subsubsection{Base Octal}

Outro sistema de numeração manual que foi largamente utilizado no passado é o octal: Existem 8 digitos (0..7) e cada um é mapeado em grupos de 3 bits, então é fácil realizar a conversão.
Este sistema foi substituido pelo sistema hexadecimal em quase todos os lugares, no entando, surpreendentemente, existe um utilitário *NIX usado por muitas pessoas que usa a numeração octal como argumento:\TT{chmod}.

\myindex{UNIX!chmod}
Como muitos usuários *NIX sabem, os arguments do \TT{chmod} podem ser um número de 3 dígitos. O primeiro dígito dita o permissionamento do dono do arquivo, o segundo para o grupo(que o arquivo pertence), e o terceiro para todos os outros.
E cada dígito pode ser representado da seguinte forma em binário:

\begin{center}
\begin{longtable}{ | l | l | l | }
\hline
\HeaderColor decimal & \HeaderColor binário & \HeaderColor significado \\
\hline
7	&111	&\textbf{rwx} \\
6	&110	&\textbf{rw-} \\
5	&101	&\textbf{r-x} \\
4	&100	&\textbf{r-{}-} \\
3	&011	&\textbf{-wx} \\
2	&010	&\textbf{-w-} \\
1	&001	&\textbf{-{}-x} \\
0	&000	&\textbf{-{}-{}-} \\
\hline
\end{longtable}
\end{center}

Desta forma, cada bit é mapeado para uma flag: ler(Read)/escrever(Write)/eXecutar.

Now the reason why I'm talking about \TT{chmod} here is that the whole number in argument can be represented as octal number.
Let's take for example, 644.
When you run \TT{chmod 644 file}, you set read/write permissions for owner, read permissions for group and again, read permissions for everyone else.
Let's convert 644 octal number to binary, this will be \TT{110100100}, or (in groups of 3 bits) \TT{110 100 100}.

Now we see that each triplet describe permissions for owner/group/others: first is \TT{rw-}, second is \TT{r--} and third is \TT{r--}.

Octal numeral system was also popular on old computers like PDP-8, because word there could be 12, 24 or 36 bits, and these numbers are divisible by 3, so octal system was natural on that environment.
Nowadays, all popular computers employs word/address size of 16, 32 or 64 bits, and these numbers are divisible by 4, so hexadecimal system is more natural here.

Octal numeral system is supported by all standard \CCpp compilers.
This is source of confusion sometimes, because octal numbers are encoded with zero prepended, for example, 0377 is 255.
And sometimes, you may make a typo and write "09" instead of 9, and the compiler would report error.
GCC may report something like that:\\
\TT{error: invalid digit "9" in octal constant}.

\subsubsection{Divisibility}

When you see a decimal number like 120, you can quickly deduce that it's divisible by 10, because the last digit is zero.
In the same way, 123400 is divisible by 100, because two last digits are zeros.

Likewise, hexadecimal number 0x1230 is divisible by 0x10 (or 16), 0x123000 is divisible by 0x1000 (or 4096), etc.

Binary number 0b1000101000 is divisible by 0b1000 (8), etc.

This property can be used often to realize quickly if a size of some block in memory is padded to some boundary.
For example, sections in \ac{PE} files are almost always started at addresses ending with 3 hexadecimal zeros: 0x41000, 0x10001000, etc.
The reason behind this is in the fact that almost all \ac{PE} sections are padded to boundary of 0x1000 (4096) bytes.

\subsubsection{Multi-precision arithmetic and radix}

\index{RSA}
Multi-precision arithmetic can use huge numbers, and each one may be stored in several bytes.
For example, RSA keys, both public and private, are spanning up to 4096 bits and maybe even more.

In \InSqBrackets{\TAOCPvolII, 265} we can find the following idea: when you store multi-precision number in several bytes,
the whole number can be represented as having a radix of $2^8=256$, and each digit goes to corresponding byte.
Likewise, if you store multi-precision number in several 32-bit integer values, each digit goes to each 32-bit slot,
and you may think about this number as stored in radix of $2^{32}$.

\subsubsection{Pronouncement}

Numbers in non-decimal base are usually pronounced by one digit: ``one-zero-zero-one-one-...''.
Words like ``ten'', ``thousand'', etc, are usually not pronounced, because it will be confused with decimal base then.

\subsubsection{Números de ponto flutuante}

A diferença entre números inteiros e de ponto flutuante é que números de ponto flutuante em geral são escritos com ``.0'' no final, como $0.0$, $123.0$, etc.