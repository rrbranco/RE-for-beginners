% TODO to be resynced with EN version
\subsection{Sistemi di numerazione}

Gli umani si sono abituati ad usare il sistema numerico decimale probabilmente perche' quasi tutti hanno 10 dita.
Ciononostante, il numero 10 non ha alcun significato rilevante in scienze e in matematica.
Il sistema di numerazione naturale nell'elettronica digitale e' il binario: 0 per l'assenza di corrente nel filo e 1 per la presenza.
10 in binario e' 2 in decimale, 100 in binary is 4 in decimale e cosi' via.

Come si fa a convertire un numero da un sistema ad un altro?

La notazione posizionale e' usata quasi dappertutto. Cio' vuol dire che la cifra (numero in un singolo carattere) ha un "peso" diverso in base alla posizone in cui si trova. 
Se 2 si trova nella parte piu' a destra del numero, e' 2.
Se si trova di una posizione piu' a destra, e' 20.

Per cosa sta 1234?

$10^3 \cdot 1 + 10^2 \cdot 2 + 10^1 \cdot 3 + 1 \cdot 4$ = 1234 or 
$1000 \cdot 1 + 100 \cdot 2 + 10 \cdot 3 + 4 = 1234$

Lo stesso vale per i numeri binari, ma la base e' 2 invece di 10.
Per cosa sta 101011?

$2^5 \cdot 1 + 2^4 \cdot 0 + 2^3 \cdot 1 + 2^2 \cdot 0 + 2^1 \cdot 1 + 2^0 \cdot 1 = 43$ or
$32 \cdot 1 + 16 \cdot 0 + 8 \cdot 1 + 4 \cdot 0 + 2 \cdot 1 + 1 = 43$

La notazione posizionale e' opposta a quella non-posizionale come il sistema numerico Romano.
Forse l'umanita' ha deciso di passare alla notazione posizionale perche' e' piu' facile effettuare semplici operazioni (addizione, moltiplicazione, etc.) a mano su carta.

I numeri binari possono essere addizionati, sotratti e cosi' via, nello stesso modo in cui ci e' stato insegnato a scuola, con l'unica differenza chec si sono solo 2 cifre disponibili.

I numeri binary possono risultare ingombranti quando usati in codici sorgenti e dump, ed in questi casi puo' essere usato il sistema esadecimale.

Il sistema esadecimale usa i numeri 0..9 e 6 lettere latine: A..F.
Ogni cifra esadecimale occupa 4bit o 4 binary digits, ed e; quindi molto facile convertire da binario a esadecimale e viceversa, anche a mente,

\begin{center}
\begin{longtable}{ | l | l | l | }
\hline
\HeaderColor hexadecimal & \HeaderColor binary & \HeaderColor decimal \\
\hline
0	&0000	&0 \\
1	&0001	&1 \\
2	&0010	&2 \\
3	&0011	&3 \\
4	&0100	&4 \\
5	&0101	&5 \\
6	&0110	&6 \\
7	&0111	&7 \\
8	&1000	&8 \\
9	&1001	&9 \\
A	&1010	&10 \\
B	&1011	&11 \\
C	&1100	&12 \\
D	&1101	&13 \\
E	&1110	&14 \\
F	&1111	&15 \\
\hline
\end{longtable}
\end{center}

Come identificare quale sistema viene usato?

I numeri decimali sono solitamente scritti cosi' come sono, es. 1234. Alcuni assembler consentono pero' di aggiungere enfasi al sistema decimale ed il numero puo essere scritto con il suffisso "d": 1234d.

I numeri binari sono a volte preceduti dal prefisso "0b": 0b100110111 (\ac{GCC} non ha un'estensione del linguaggio standard per questo\footnote{\url{https://gcc.gnu.org/onlinedocs/gcc/Binary-constants.html}}).
C'e' anche un altro modo: il suffisso "b" suffix, ad esempio: 100110111b.
Cerchero' di usare il prefisso "0b" per identificare i numeri binari all'interno del libro.

I numeri esadecimali sono preceduti dal prefisso "0x" in \CCpp e altri \ac{PL}s: 0x1234ABCD.
Oppure hanno il suffisso "h": 1234ABCDh - questo e' il modo comune di rapressentarli negli assembler e nei debugger.
Se il numero inizia con una cifra A..F, 0 dovrebbe essere aggiunto all'inizio: 0ABCDEFh.
% TBT
Cerchero' di usare il prefisso "0x" per identificare i numeri esadecimali all'interno del libro.

E' il caso di imparare a convertire i numeri a mente? Una tabella di numeri decimali ad una cifra puo' essere memorizzata facilmente.
Per numeri piu' grandi, probabilmente, non e' il caso di tormentarsi.

\subsubsection{Sistema di numerazione ottale}

Un altro sistema di numerazione molto usato in pasato in programmazione e' l'ottale: ci sono 8 cifre (0..7) e ciascuna e' associata a 3 bit, quindi e' facile convertire da una parte all'altra.
E' stato comunque rimpiazzato dal sistema esadecimale quasi ovunque, ma sorprendentemente c'e' una utility *NIX usata spesso e da molte persone che ha per argomento un numero ottale: \TT{chmod}.

\myindex{UNIX!chmod}
Come molti utenti *NIX sanno, l'argomento \TT{chmod} puo' essere un numero di 3 cifre. La prima rappresenta i diritti del proprietario del file, la seconda i diritti del gruppo (a cui il file appartiene), la terza i diritti per chiunque altro.
E ogni cifra puo' essere rappresentata in forma binaria:

\begin{center}
\begin{longtable}{ | l | l | l | }
\hline
\HeaderColor decimale & \HeaderColor binario & \HeaderColor significato \\
\hline
7	&111	&\textbf{rwx} \\
6	&110	&\textbf{rw-} \\
5	&101	&\textbf{r-x} \\
4	&100	&\textbf{r-{}-} \\
3	&011	&\textbf{-wx} \\
2	&010	&\textbf{-w-} \\
1	&001	&\textbf{-{}-x} \\
0	&000	&\textbf{-{}-{}-} \\
\hline
\end{longtable}
\end{center}

Quindi ogni bit corrisponde ad un flag: read/write/execute.

La ragione per cui sto parlando di \TT{chmod} e' che l'intero numero dell'argomento puo' essere rappresentato in ottale.
Prendiamo per esempio 644.
Quando si esegue \TT{chmod 644 file}, si impostano i permessi di lettura/scrittura per il proprietario, il permesso di lettura per il gruppo, e il permesso di lettura per tutti gli altri.
Convertiamo il numero 644 in ottale a binario, sara' \TT{110100100}, o (in gruppi di 3 bit) \TT{110 100 100}.

Notiamo che ogni tripletta descrive i permessi per proprietario/gruppo/altri (owner/group/others): il primo e' \TT{rw-}, il secondo e' \TT{r--} ed il terzo e' \TT{r--}.

Il sistema ottale era anche molto diffuso nei vecchi computer come PDP-8, perche' una word poteva essere 12, 24 o 36 bit, e questi numeri sono tutti divisibili per 3, quindi il sistema ottale era naturale in quell'ambiente.
Oggi tutti i computer comuni utilizano word/indirizzi lunghi 16, 32 o 64 bit, e questi numeri sono divisibili per 4, di conseguenza l'esadecimale risulta piu' naturale.

Il sistema ottale e' supportato da tutti i compilatori \CCpp standard.
Talvolta cio' da' vita a confusione, perche' i numeri ottali sono codificati preponendo uno zero, per esempio 0377 e' 255.
A volte si potrebbe scrivere per errore "09" invece di 9 , e il compilatore non lo consentirebbe.
GCC potrebbe presentare un errore del genere:\\
\TT{error: invalid digit "9" in octal constant}.

\subsubsection{Divisibilita'}

Quando si vede un numero decimale come 120, si puo' velocemente dedurre che e' divisibile per 10, perche' l'ultima cifra e' zero.
Allo stesso modo, 123400 e' divisibile per 100 perche' le ultime due cifre sono zeri.


In maniera simile, il numero esadecimale 0x1230 e' divisible per 0x10 (ovvero 16), 0x123000 e' divisibile per 0x1000 (ovvero 4096), etc.

Il numero binario 0b1000101000 e' dibisibile per 0b1000 (8), etc.

Questa proprieta' puo' essere spesso usata per capire velocemente se la dimensione di un dato blocco di memoria e' "allungata" (padded) per raggiungere un certo confine.
Per esempio, le sezioni in un file \ac{PE} iniziano quasi sempre ad indirizzi che terminano con 3 zeri esadecimali: 0x41000, 0x10001000, etc.
La ragione per cui questo accade risiede nel fatto che quasi tutte le sezioni di un \ac{PE} sono allungate per raggiungere un confine di 0x1000 (4096) bytes.

% TBT
