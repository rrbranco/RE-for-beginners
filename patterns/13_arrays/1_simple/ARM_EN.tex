\subsubsection{ARM}

\myparagraph{\NonOptimizingKeilVI (\ARMMode)}

\lstinputlisting[style=customasmARM]{patterns/13_arrays/1_simple/simple_Keil_ARM_O0_EN.asm}

\Tint type requires 32 bits for storage (or 4 bytes),

so to store 20 \Tint variables 80 (\TT{0x50}) bytes are needed.
So that is why the \INS{SUB SP, SP, \#0x50} 

instruction in the function's prologue allocates exactly this amount of space in the stack.

In both the first and second loops, the loop iterator \var{i} is placed in the \Reg{4} register.

\myindex{ARM!Optional operators!LSL}

The number that is to be written into the array is calculated as $i*2$, which is effectively equivalent 
to shifting it left by one bit,\\
so \INS{MOV R0, R4,LSL\#1} instruction does this.

\myindex{ARM!\Instructions!STR}
\INS{STR R0, [SP,R4,LSL\#2]} writes the contents of \Reg{0} into the array.

Here is how a pointer to array element is calculated: \ac{SP} points to the start of the array, \Reg{4} is $i$.

So shifting $i$ left by 2 bits is effectively equivalent to multiplication by 4
(since each array element has a size of 4 bytes) and then it's added to the address of the start of the array.

\myindex{ARM!\Instructions!LDR}

The second loop has an inverse \INS{LDR R2, [SP,R4,LSL\#2]}
instruction. It loads the value we need from the array, and the pointer to it is calculated likewise.

\myparagraph{\OptimizingKeilVI (\ThumbMode)}

\lstinputlisting[style=customasmARM]{patterns/13_arrays/1_simple/simple_Keil_thumb_O3_EN.asm}

Thumb code is very similar.
\myindex{ARM!\Instructions!LSLS}

Thumb mode has special instructions for bit shifting (like \TT{LSLS}),
which calculates the value to be written into the array and the address of each element in the array as well.

The compiler allocates slightly more space in the local stack, however, the last 4 bytes are not used.

\myparagraph{\NonOptimizing GCC 4.9.1 (ARM64)}

\lstinputlisting[caption=\NonOptimizing GCC 4.9.1 (ARM64),style=customasmARM]{patterns/13_arrays/1_simple/ARM64_GCC491_O0_EN.s}

