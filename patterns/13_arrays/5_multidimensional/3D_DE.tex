\subsubsection{Beispiel: dreidimensionales Array}

Mit multidimensionalen Arrays ist es das gleiche.

Wir werden nun mit einem Array vom Typ \Tint arbeiten: jedes Element benötigt 4 Byte Speicherplatz.

Sehen wir es uns an:

\lstinputlisting[caption=simple example,style=customc]{patterns/13_arrays/5_multidimensional/multi.c}

\myparagraph{x86}

Wir erhalten das Folgende (MSVC 2010):

\lstinputlisting[caption=MSVC 2010,style=customasmx86]{patterns/13_arrays/5_multidimensional/multi_msvc_DE.asm}
Nichts Außergewöhnliches. Zur Berechnung des Index' werden in der Formel $address=600 \cdot 4 \cdot x + 30 \cdot 4 \cdot
y + 4z$ drei Eingabewerte verwendet, um das multidimensionale Array zu repräsentieren.
Vergessen wir nicht, dass der \Tint Typ 32 Bit (4 Byte) breit ist, sodass alle Koeffizienten mit 4 multipliziert werden
müssen.

\lstinputlisting[caption=GCC 4.4.1,style=customasmx86]{patterns/13_arrays/5_multidimensional/multi_gcc_DE.asm}
Der GCC Compiler arbeitet anders.

Für eine der Operationen in der Berechnung ($30y$) produziet GCC Code ohne Multiplikationsbefehle.
Das funktioniert wie folgt:
$(y+y) \ll 4 - (y+y) = (2y) \ll 4 - 2y = 2 \cdot 16 \cdot y - 2y = 32y - 2y = 30y$.

So werden für die $30y$ Berechnung nur ein Addierbefehl, eine bitweiser Verschiebebefehl und ein Subtraktionsbefehl
verwendet. So geht es schneller.

\myparagraph{ARM + \NonOptimizingXcodeIV (\ThumbMode)}

\lstinputlisting[caption=\NonOptimizingXcodeIV
(\ThumbMode),style=customasmARM]{patterns/13_arrays/5_multidimensional/multi_Xcode_thumb_O0_DE.asm}

\NonOptimizing LLVM speichert alle Variablen auf dem lokalen Stack, was redundant ist.

Die Adresse des Arrayelements wird über die eben gezeigte Formel berechnet.

\myparagraph{ARM + \OptimizingXcodeIV (\ThumbMode)}

\lstinputlisting[caption=\OptimizingXcodeIV
(\ThumbMode),style=customasmARM]{patterns/13_arrays/5_multidimensional/multi_Xcode_thumb_O3_DE.asm}
Die Tricks für das Ersetzen der Multiplikation durch Verschieben, Addieren und Subtrahieren, die wir bereits
kennengelernt haben, kommen hier auch vor.

\myindex{ARM!\Instructions!RSB}
\myindex{ARM!\Instructions!SUB}
Hier finden wir auch einen für uns neuen Befehl: \RSB (\IT{Reverse Subtract}).

Er arbeitet genau wie \SUB, aber vertauscht die Operanden vor der Ausführung. Warum?

\myindex{ARM!Optional operators!LSL}
\SUB und \RSB  sind Befehle, bei denen auf den zweiten Operanden eine bitweise Verschiebung angewendet werden kann:
(\INS{LSL\#4}).
Dieser Koeffizient kann aber nur auf den zweiten Operanden angewendet werden.

Das ist günstig für kommutative Operationen wie Addition und Multiplikation (die Operanden können vertauscht werden,
ohne das Ergebnis zu verändern).

Subtraktion dagegen ist nicht kommutativ, weshalb für diese Fälle \RSB existiert.

\myparagraph{MIPS}

\myindex{MIPS!Global Pointer}
Das Beispiel ist sehr klein, sodass der GCC Compiler entschieden hat das Array $a$ im 64KiB Platz abzulegen, um es durch
den globalen Pointer zugreifbar zu machen.

\lstinputlisting[caption=\Optimizing GCC 4.4.5
(IDA),style=customasmMIPS]{patterns/13_arrays/5_multidimensional/multi_MIPS_O3_IDA_DE.lst}

