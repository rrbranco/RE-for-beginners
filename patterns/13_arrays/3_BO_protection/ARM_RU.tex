\subsubsection{\OptimizingXcodeIV (\ThumbTwoMode)}

Возвращаясь к нашему простому примеру
 (\myref{arrays_simple}),
можно посмотреть, как LLVM добавит проверку \q{канарейки}:

% TODO shorten the listing a bit? is full display of unrolled loop necessary?
\lstinputlisting[style=customasmARM]{patterns/13_arrays/3_BO_protection/simple_Xcode_thumb_O3_RU.asm}

\myindex{Unrolled loop}
Во-первых, LLVM \q{развернул} цикл и все значения записываются в массив по одному, 
уже вычисленные, потому что LLVM посчитал что так будет быстрее.

Кстати, инструкции режима ARM позволяют сделать это ещё быстрее и это может быть вашим 
домашним заданием.

В конце функции мы видим сравнение \q{канареек}~--- той что лежит в локальном стеке и корректной, 
на которую ссылается регистр \Reg{8}.

\myindex{ARM!\Instructions!IT}
Если они равны, срабатывает блок из четырех инструкций при помощи \INS{ITTTT EQ}.
Это запись 0 в \Reg{0}, эпилог функции и выход из нее.

Если \q{канарейки} не равны, блок не срабатывает и происходит
переход на функцию \TT{\_\_\_stack\_chk\_fail}, которая остановит работу программы.

% TODO1 illustrate this!
