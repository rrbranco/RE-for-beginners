\section{64 Bit}

\subsection{x86-64}
\myindex{x86-64}
\label{x86-64}
Es handelt sich um eine 64-Bit-Erweiterung der x86 Architektur.

Aus Sicht eines Reverse Engineers sind die wichtigsten Veränderungen die folgenden:

\myindex{\CLanguageElements!\Pointers}
\begin{itemize}

\item
Fast alle Regsiter (außer FPU und SIMD) wurden auf 64 Bit erweitert und erhielten den Präfix R-. 8 zusätzliche Register
wurden hinzugefügt.
Die \ac{GPR} sind jetzt: \RAX, \RBX, \RCX, \RDX, 
\RBP, \RSP, \RSI, \RDI, \Reg{8}, \Reg{9}, \Reg{10}, 
\Reg{11}, \Reg{12}, \Reg{13}, \Reg{14}, \Reg{15}. 
Es ist immernoch mgölich die älteren Registerteile wie gewohnt anzusprechen.
Zum Beispiel ist es möglich, auf den niederen 32-Bit-Teil von \RAX mit \EAX zuzugreifen:

\RegTableOne{RAX}{EAX}{AX}{AH}{AL}
Die neuen \GTT{R8-R15} Register besitzen ebenfalls niedere Teile: \GTT{R8D-R15D} (niedere 32 Bit),
\GTT{R8W-R15W} (niedere 16 Bit), \GTT{R8L-R15L} (niedere 8 Bit).

\RegTableFour{R8}{R8D}{R8W}{R8L}
Die Anzahl der SIMD Register wurde von 8 auf 16 erhöht:\XMM{0}-\XMM{15}.

\item
In Win64 ist die Konvention zum Aufruf von Funktionen ein wenig verändert worden und erinnert nun an
fastcall(\myref{fastcall}). Die ersten vier Argumente werden in den \RCX, \RDX, \Reg{8} und \Reg{9} Registern
gespeichert~---der Rest auf dem Stack.
Die aufrufende Funktion muss also 32 Byte reservieren, sodass der Aufgerufene die ersten 4 Argumente dort speichern 
und sie für eigene Zwecke verwenden kann.
Kurze Funktionen können Argumente verwenden, die ausschließlich aus Registern stammen, aber größere müssen ihre
Variablen auf dem Stack speichern.

System V AMD64 ABI (Linux, *BSD, \MacOSX)\SysVABI erinnert auch an
fastcall: es verwendet die 6 Register 
\RDI, \RSI, \RDX, \RCX, \Reg{8}, \Reg{9} für die ersten 6 Argumente.
Der Rest wird wie gehabt über den Stack übergeben.

Siehe dazu auch den Abschnitt über Aufrufkonventionen~(\myref{sec:callingconventions}).

\item
Der Typ \Tint in \CCpp hat aus Kompatibilitätsgründen immernoch die Größe 32 Bit.

\item
Alle Pointer haben jetzt eine Größe von 64 Bit.

\end{itemize}

\myindex{Register allocation}
Da sich die Anzahl der Register verdoppelt hat, hat der Compiler mehr Spielraum für den \glslink{register allocator}.
Für uns hat dies zur Folge, dass der erzeugte Code eine geringere Anzahl lokaler Variabeln enthält.

\myindex{DES}
Die Funktionm, die die erste S-Box im DES Verschlüsselungsalgorithmus berechnet, verarbeitet beispielsweise
32/64/128/256 Werte auf einmal (abhängig vom \GTT{DES\_type}: uint32, uint64, SSE2 oder AVX)) mithilfe der Bitslice DES
Methode (mehr zu dieser Technik unter~(\myref{bitslicedes})):

\lstinputlisting[style=customc]{patterns/20_x64/19_1.c}
Es gibt eine Menge lokaler Variablen.
Natürlich werden diese nicht alle auf dem lokalen Stack gespeichert.
Kompilieren wir mit MSVC 2008 mit der Option \GTT{Ox}:

\lstinputlisting[caption=\Optimizing MSVC 2008,style=customasmx86]{patterns/20_x64/19_2_msvc_Ox.asm}
Der Compiler legt auf dem lokalen Stack 5 Variablen an.

Untersuchen wir dasselbe Programm in der 64-Bit-Version mit MSVC 2008:

\lstinputlisting[caption=\Optimizing MSVC 2008,style=customasmx86]{patterns/20_x64/19_3_msvc_x64.asm}
Hier wurde im lokalen Stack nichts durch den Compiler angelegt, \GTT{x36} ist ein Synonym für \GTT{a5}.

\iffalse
% FIXME1 невнятно
Wir sehen hier übrigens, dass die Funktion die Register \RCX und \RDX in vom Aufrufenden bereitgestellten Speicherplatz
gesichert hat. \Reg{8} und \Reg{9} wurden nicht gesichert, aber trotzdem von Beginn an benutzt.

\fi

Es gibt übrigens CPUs mit viel mehr \ac{GPR}s, z.B. Itanium (128 Register).

\subsection{ARM}

64-Bit-Befehle erschienen in ARMv8.

\subsection{Fließkommazahlen}
Den Umgang mit Fließkommazahlen in x86-64 haben wir hier erklärt: \myref{floating_SIMD}.

\subsection{Kritik an der 64-Bit-Architektur}
Einige Leute sind manchmal am Zweifeln bezüglich der Vorteil von 64-Bit-Architekturen: man benötigt nun doppelt soviel
Platz für das Speichern von Pointern, inklusive Cachespeicher, obwohl die x64 CPUs nur 48 Bit an externem \ac{RAM}
adressieren können.

Einige leute schrieben ihre eigenen Speicherreservierungen.

\myindex{CryptoMiniSat}
Interessant ist dabei der Fall CryptoMiniSat\footnote{\url{https://github.com/msoos/cryptominisat/}}.
Dieses Programm benötigt fast nie mehr als 4GiB \ac{RAM}, macht aber starken Gebrauch von Pointern.
Es benötigt also auf einer 32-Bit-Architektur weniger Speicher als unter einer 64-Bit-Architektur.
Um dieses Problem zu lösen, hat der Autor seine eigene Speicherreservierung (in den Dateien
\IT{clauseallocator.(h|cpp)}) geschrieben, welche erlaubt, Speicher mit 32-Bit-Identifiern anstelle von 64-Bit-Pointern
zu reservieren.

