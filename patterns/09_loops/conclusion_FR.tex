% N.B.: \Conclusion{} is a macro name, do not translate
\subsection{\Conclusion{}}

Squelette grossier d'une boucle de 2 à 9 inclus:

\lstinputlisting[caption=x86,style=customasmx86]{patterns/09_loops/skeleton_x86_2_9_optimized_FR.lst}

L'opération d'incrémentation peut être représentée par 3 instructions dans du code
non optimisé:

\lstinputlisting[caption=x86,style=customasmx86]{patterns/09_loops/skeleton_x86_2_9_FR.lst}

Si le corps de la boucle est court, un registre entier peut être dédié à la variable
compteur:

\lstinputlisting[caption=x86,style=customasmx86]{patterns/09_loops/skeleton_x86_2_9_reg_FR.lst}

Certaines parties de la boucle peuvent être générées dans un ordre différent par
le compilateur:

\lstinputlisting[caption=x86,style=customasmx86]{patterns/09_loops/skeleton_x86_2_9_order_FR.lst}

En général, la condition est testée \IT{avant} le corps de la boucle, mais le compilateur
peut la réarranger afin que la condition soit testée \IT{aprés} le corps de la boucle.

Cela est fait lorsque le compilateur est certain que la condition est toujours \IT{vraie}
à la première itération, donc que le corps de la boucle doit être exécuté au moins
une fois:

\lstinputlisting[caption=x86,style=customasmx86]{patterns/09_loops/skeleton_x86_2_9_reorder_FR.lst}

\myindex{x86!\Instructions!LOOP}

En utilisant l'instruction \TT{LOOP}. Ceci est rare, les compilateurs ne l'utilisent
pas.
Lorsque vous la voyez, c'est le signe que le morceau de code a été écrit é la main:

\lstinputlisting[caption=x86,style=customasmx86]{patterns/09_loops/skeleton_x86_loop_FR.lst}

ARM. 

Le registre \Reg{4} est dédié à la variable compteur dans cet exemple:

\lstinputlisting[caption=ARM,style=customasmARM]{patterns/09_loops/skeleton_ARM_FR.lst}

% TODO MIPS

