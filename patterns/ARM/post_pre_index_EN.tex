\subsection{Addressing modes}
\myindex{ARM!Addressing modes}
\label{ARM_postindex_vs_preindex}
\myindex{\CLanguageElements!\PostIncrement}
\myindex{\CLanguageElements!\PostDecrement}
\myindex{\CLanguageElements!\PreIncrement}
\myindex{\CLanguageElements!\PreDecrement}

This instruction is possible in ARM64:

\myindex{ARM!\Instructions!LDR}
\begin{lstlisting}[style=customasmARM]
ldr	x0, [x29,24]
\end{lstlisting}

This means add 24 to the value in X29 and load the value from this address.

Please note that 24 is inside the brackets.
The meaning is different if the number is outside the brackets:

\begin{lstlisting}[style=customasmARM]
ldr	w4, [x1],28
\end{lstlisting}

This means load the value at the address in X1, then add 28 to X1.

\myindex{PDP-11}

ARM allows you to add or subtract a constant to/from the address used for loading.

And it's possible to do that both before and after loading.

There is no such addressing mode in x86, but it is present in some other processors, even on PDP-11.

There is a legend that the pre-increment, post-increment, pre-decrement and post-decrement modes in PDP-11,

were \q{guilty} for the appearance of such C language (which developed on PDP-11) constructs as
*ptr++, *++ptr, *ptr-{}-, *-{}-ptr. 

By the way, this is one of the hard to memorize C features.
This is how it is:

\small
% FIXME: add ARM assembly...
\begin{center}
\begin{tabular}{ | l | l | l | l | }
\hline
\headercolor{} C term & 
\headercolor{} ARM term & 
\headercolor{} C statement & 
\headercolor{} how it works \\
\hline
\PostIncrement & 
post-indexed addressing & 
\TT{*ptr++} & 
use \TT{*ptr} value, \\
& & & then \gls{increment} \\
& & & \TT{ptr} pointer \\
\hline
\PostDecrement & 
post-indexed addressing & 
\TT{*ptr-{}-} & 
use \TT{*ptr} value, \\
& & & then \gls{decrement} \\
& & & \TT{ptr} pointer \\
\hline
\PreIncrement & 
pre-indexed addressing & 
\TT{*++ptr} & 
\gls{increment} \TT{ptr} pointer, \\
& & & then use \\
& & & \TT{*ptr} value \\
\hline
\PreDecrement & 
pre-indexed addressing & 
\TT{*-{}-ptr} & 
\gls{decrement} \TT{ptr} pointer, \\
& & & then use \\
& & & \TT{*ptr} value \\
\hline
\end{tabular}
\end{center}
\normalsize

Pre-indexing is marked with an exclamation mark in the ARM assembly language.
For example, see line 2 in \lstref{hw_ARM64_GCC}.

Dennis Ritchie (one of the creators of the C language) mentioned that it presumably was invented by Ken Thompson
(another C creator) because this processor feature was present in PDP-7
\footnote{\url{http://yurichev.com/mirrors/C/c_dmr_postincrement.txt}}, \RitchieDevC{}.

Thus, C language compilers may use it, if it is present on the target processor.

That's very convenient for array processing.
