\subsection{Передача аргументов, сложение, вычитание}

\lstinputlisting[style=customc]{patterns/185_64bit_in_32_env/passing_add_sub/1.c}

\subsubsection{x86}

\lstinputlisting[caption=\Optimizing MSVC 2012 /Ob1,style=customasmx86]{patterns/185_64bit_in_32_env/passing_add_sub/1_MSVC.asm}

В \GTT{f\_add\_test()} видно, как каждое 64-битное число передается двумя 32-битными значениями,
сначала старшая часть, затем младшая.

Сложение и вычитание происходит также парами. 

\myindex{x86!\Instructions!ADC}
При сложении, в начале складываются младшие 32 бита.
Если при сложении был перенос, выставляется флаг CF.
Следующая инструкция \INS{ADC} складывает старшие части чисел, но также прибавляет единицу если $CF=1$.

\myindex{x86!\Instructions!SBB}
Вычитание также происходит парами.
Первый \SUB может также включить флаг переноса CF, который затем будет проверяться в \INS{SBB}:
если флаг переноса включен, то от результата отнимется единица.

Легко увидеть, как результат работы \GTT{f\_add()} затем передается в \printf{}.

\lstinputlisting[caption=GCC 4.8.1 -O1 -fno-inline,style=customasmx86]{patterns/185_64bit_in_32_env/passing_add_sub/1_GCC.asm}

Код GCC почти такой же.

\subsubsection{ARM}

\lstinputlisting[caption=\OptimizingKeilVI (\ARMMode),style=customasmARM]{patterns/185_64bit_in_32_env/passing_add_sub/Keil_ARM_O3.s}

\myindex{ARM!\Instructions!ADDS}
\myindex{ARM!\Instructions!SUBS}
\myindex{ARM!\Instructions!ADC}
\myindex{ARM!\Instructions!SBC}
Первое 64-битное значение передается в паре регистров \Reg{0} и \Reg{1}, второе --- в паре \Reg{2} и \Reg{3}.
В ARM также есть инструкция \INS{ADC} (учитывающая флаг переноса) и \INS{SBC} (\q{subtract with carry} --- вычесть с переносом).
Важная вещь: когда младшие части слагаются/вычитаются, используются инструкции \INS{ADDS} и \INS{SUBS} с суффиксом -S.
Суффикс -S означает \q{set flags} (установить флаги), а флаги (особенно флаг переноса) это то что однозначно нужно последующим инструкциями \INS{ADC}/\INS{SBC}.
А иначе инструкции без суффикса -S здесь вполне бы подошли (\ADD и \SUB).

\subsubsection{MIPS}

\lstinputlisting[caption=\Optimizing GCC 4.4.5 (IDA),style=customasmMIPS]{patterns/185_64bit_in_32_env/passing_add_sub/MIPS_O3_IDA_RU.lst}

В MIPS нет регистра флагов, так что эта информация не присутствует после исполнения арифметических операций.

Так что здесь нет инструкций как \INS{ADC} или \INS{SBB} в x86.
Чтобы получить информацию о том, был бы выставлен флаг переноса, происходит сравнение (используя инструкцию
\INS{SLTU}), которая выставляет целевой регистр в 1 или 0.

Эта 1 или 0 затем прибавляется к итоговому результату, или вычитается.

