\subsection{Übergabe von Argumenten bei Addition und Subtraktion}

\lstinputlisting[style=customc]{patterns/185_64bit_in_32_env/passing_add_sub/1.c}

\subsubsection{x86}

\lstinputlisting[caption=\Optimizing MSVC 2012 /Ob1,style=customasmx86]{patterns/185_64bit_in_32_env/passing_add_sub/1_MSVC.asm}
Wir sehen, dass in der Funktion \GTT{f\_add\_test()} jeder 64-Bit-Wert über zwei 32-Bit-Werten übergeben wird: zuerst
der höhere Teil, dann der niedere Teil.

Addition und Subtraktion werden auch mit Paaren ausgeführt.

\myindex{x86!\Instructions!ADC}
Bei einer Addition werden die niederen 32-Bit-Teile zuerst addiert.
Tritt hierbei ein Übertrag auf, wird das \CF Flag gesetzt.

Der folgende \INS{ADC} Befehl addiert die höheren Teile der Operanden und addiert 1, falls $CF=1$.

\myindex{x86!\Instructions!SBB}
Subtraktion wird auch mit den Wertepaaren durchgeführt.
Das erste \SUB setzt ggf. das \CF Flag, das von dem folgenden \INS{SBB} Befehl geprüft wird:
wenn das Carryflag gesetzt ist, wird am Ende 1 vom Ergebnis abgezogen.

Man erkennt im Code leicht, wie das Ergebnis der Funktion \GTT{f\_add()} an \printf übergeben wird.

\lstinputlisting[caption=GCC 4.8.1 -O1 -fno-inline,style=customasmx86]{patterns/185_64bit_in_32_env/passing_add_sub/1_GCC.asm}

Der Code von GCC ist identisch.

\subsubsection{ARM}

\lstinputlisting[caption=\OptimizingKeilVI (\ARMMode),style=customasmARM]{patterns/185_64bit_in_32_env/passing_add_sub/Keil_ARM_O3.s}

\myindex{ARM!\Instructions!ADDS}
\myindex{ARM!\Instructions!SUBS}
\myindex{ARM!\Instructions!ADC}
\myindex{ARM!\Instructions!SBC}
Der erste 64-Bit-Wert wird über das Registerpaar \Reg{0} und \Reg{1} übergeben, der zweite über \Reg{2} und \Reg{3}.
ARM verfügt ebenfalls über die Befehle \INS{ADC} und \INS{SBC} (die das Carryflag beachten).
Man beachte: wenn die niederen Teile addiert bzw. subtrahiert werden, werden \INS{ADDS} und \INS{SUBS} Befehle mit dem
Suffy -S verwendet. Dieser Suffix steht für \q{set flags} (dt. setze Flags) und wird von den folgenden
\INS{ADC}/\INS{SBC} Befehlen unbedingt benötigt. Würden die Flags nicht weiter beachtet werden, könnten hier \ADD und
\SUB verwendet werden.

\subsubsection{MIPS}

\lstinputlisting[caption=\Optimizing GCC 4.4.5
(IDA),style=customasmMIPS]{patterns/185_64bit_in_32_env/passing_add_sub/MIPS_O3_IDA_DE.lst}
MIPS besitzt kein Register für die Flags, sodass keine derartige Information nach der Ausführung von arithmetischen
Operationen verfügbar ist.
Es gibt also keine Befehle wie \INS{ADC} oder \INS{SBB} in x86.
Um zu prüfen, ob das Carryflag gesetzt werden muss, wird ein \INS{SLTU} Befehl verwendet, der das Zielregister auf 1
oder 0 setzt. Diese 1 oder 0 wird dann zum Ergebnis addiert bzw. davon subtrahiert.

