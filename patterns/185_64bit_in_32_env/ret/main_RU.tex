\subsection{Возврат 64-битного значения}

\lstinputlisting[style=customc]{patterns/185_64bit_in_32_env/ret/0.c}

\subsubsection{x86}

64-битные значения в 32-битной среде возвращаются из функций в паре регистров \EDX{}:\EAX{}.

\lstinputlisting[caption=\Optimizing MSVC 2010,style=customasmx86]{patterns/185_64bit_in_32_env/ret/0_MSVC_2010_Ox.asm}

\subsubsection{ARM}

64-битное значение возвращается в паре регистров \Reg{0}-\Reg{1} --- (\Reg{1} это старшая часть и \Reg{0} --- младшая часть):

\lstinputlisting[caption=\OptimizingKeilVI (\ARMMode),style=customasmARM]{patterns/185_64bit_in_32_env/ret/Keil_ARM_O3.s}

\subsubsection{MIPS}

64-битное значение возвращается в паре регистров \TT{V0}-\TT{V1} (\$2-\$3) --- (\TT{V0} (\$2) это старшая часть и \TT{V1} (\$3) --- младшая часть):

\lstinputlisting[caption=\Optimizing GCC 4.4.5 (assembly listing),style=customasmMIPS]{patterns/185_64bit_in_32_env/ret/0_MIPS.s}

\lstinputlisting[caption=\Optimizing GCC 4.4.5 (IDA),style=customasmMIPS]{patterns/185_64bit_in_32_env/ret/0_MIPS_IDA.lst}

