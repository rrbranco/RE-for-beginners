\subsection{Rückgabe von 64-Bit-Werten}

\lstinputlisting[style=customc]{patterns/185_64bit_in_32_env/ret/0.c}

\subsubsection{x86}
In einer 32-Bit-Umgebung werden 64-Bit-Werte von Funktionen über das Registerpaar \EDX:\EAX zurückgegeben:

\lstinputlisting[caption=\Optimizing MSVC 2010,style=customasmx86]{patterns/185_64bit_in_32_env/ret/0_MSVC_2010_Ox.asm}

\subsubsection{ARM}
Ein 64-Bit-Wert wird über das \Reg{0}-\Reg{1} Registerpaar zurückgegeben (\Reg{1} enthält dabei den höheren und \Reg{0}
den niederen Teil):

\lstinputlisting[caption=\OptimizingKeilVI (\ARMMode),style=customasmARM]{patterns/185_64bit_in_32_env/ret/Keil_ARM_O3.s}

\subsubsection{MIPS}

Ein 64-Bit-Wert wird über das \TT{V0}-\TT{V1} (\$2-\$3) Registerpaar zurückgegeben (\TT{V0} (\$2) enthält dabei den
höheren und \TT{V1} (\$3) den niederen Teil):

\lstinputlisting[caption=\Optimizing GCC 4.4.5 (assembly listing),style=customasmMIPS]{patterns/185_64bit_in_32_env/ret/0_MIPS.s}

\lstinputlisting[caption=\Optimizing GCC 4.4.5 (IDA),style=customasmMIPS]{patterns/185_64bit_in_32_env/ret/0_MIPS_IDA.lst}
