\section{\PrintfSeveralArgumentsSectionName}

Agora vamos extender o nosso exemplo \IT{\HelloWorldSectionName}~(\myref{sec:helloworld}),
trocando \printf no corpo da função \main() por isso:

\lstinputlisting[label=hw_c,style=customc]{patterns/03_printf/1.c}

% sections
\subsection{x86}

% subsections:
\EN{\subsubsection{x86: 3 arguments}

\myparagraph{MSVC}

When we compile it with MSVC 2010 Express we get:

\begin{lstlisting}[style=customasmx86]
$SG3830	DB	'a=%d; b=%d; c=%d', 00H

...

	push	3
	push	2
	push	1
	push	OFFSET $SG3830
	call	_printf
	add	esp, 16					; 00000010H
\end{lstlisting}

Almost the same, but now we can see the \printf arguments are pushed onto the stack in reverse order. The first argument is pushed last.

By the way, variables of \Tint type in 32-bit environment have 32-bit width, that is 4 bytes.

So, we have 4 arguments here. $4*4 = 16$~---they occupy exactly 16 bytes in the stack: a 32-bit pointer to a string and 3 numbers of type \Tint.

\myindex{x86!\Instructions!ADD}
\myindex{x86!\Registers!ESP}
\myindex{cdecl}
When the \gls{stack pointer} (\ESP register) has changed back by the\\
\INS{ADD ESP, X} instruction after a function call, often,
the number of function arguments could be deduced by simply dividing X by 4.

Of course, this is specific to the \IT{cdecl} calling convention, and only for 32-bit environment.

See also the calling conventions section ~(\myref{sec:callingconventions}).

In certain cases where several functions return right after one another, the compiler could merge multiple \q{ADD ESP, X} instructions into one, after the last call:

\begin{lstlisting}[style=customasmx86]
push a1
push a2
call ...
...
push a1
call ...
...
push a1
push a2
push a3
call ...
add esp, 24
\end{lstlisting}

Here is a real-world example:

\lstinputlisting[caption=x86,style=customasmx86]{patterns/03_printf/x86/add_example_EN.lst}

\clearpage
\myparagraph{MSVC and \olly}
\myindex{\olly}

Now let's try to load this example in \olly.
It is one of the most popular user-land win32 debuggers.
We can compile our example in MSVC 2012 with \GTT{/MD} option, which means to link with \GTT{MSVCR*.DLL},
so we can see the imported functions clearly in the debugger.

Then load the executable in \olly.
The very first breakpoint is in \GTT{ntdll.dll}, press 
F9 (run).
The second breakpoint is in \ac{CRT}-code.
Now we have to find the \main function.

Find this code by scrolling the code to the very top (MSVC allocates the \main function at the very beginning of the code section): 
\begin{figure}[H]
\centering
\myincludegraphics{patterns/03_printf/x86/olly3_1.png}
\caption{\olly: the very start of the \main function}
\label{fig:printf3_olly_1}
\end{figure}

Click on the \INS{PUSH EBP} instruction, press F2 (set breakpoint) and press F9 (run).
We have to perform these actions in order to skip \ac{CRT}-code, because we aren't really interested in it yet.

\clearpage
Press F8 (\stepover) 6 times, i.e. skip 6 instructions:

\begin{figure}[H]
\centering
\myincludegraphics{patterns/03_printf/x86/olly3_2.png}
\caption{\olly: before \printf execution}
\label{fig:printf3_olly_2}
\end{figure}

Now the \ac{PC} points to the \INS{CALL printf} instruction.
\olly, like other debuggers, highlights the value of the registers which were changed.
So each time you press F8, \EIP changes and its value is displayed in red.
\ESP changes as well, because the arguments values are pushed into the stack.\\
\\
Where are the values in the stack?
Take a look at the right bottom debugger window:

\begin{figure}[H]
\centering
\input{patterns/03_printf/x86/incl_olly3_stack}
\caption{\olly: stack after the argument values have been pushed (The red rectangular border was added by the author in a graphics editor)}
\end{figure}

We can see 3 columns there: address in the stack, value in the stack and some additional \olly comments. 
\olly understands \printf{}-like strings, so it reports the string here and the 3 values \IT{attached} to it.

It is possible to right-click on the format string, click on \q{Follow in dump},
and the format string will appear in the debugger left-bottom window, which always displays some part of the memory.
These memory values can be edited.
It is possible to change the format string, in which case the result of our example would be different.
It is not very useful in this particular case, but it could be good as an exercise so you start building a feel of how everything works here.

\clearpage
Press F8 (\stepover).

We see the following output in the console:

\lstinputlisting{patterns/03_printf/x86/console.txt}

Let's see how the registers and stack state have changed: 

\begin{figure}[H]
\centering
\myincludegraphics{patterns/03_printf/x86/olly3_3.png}
\caption{\olly after \printf{} execution}
\label{fig:printf3_olly_3}
\end{figure}

Register \EAX now contains \GTT{0xD} (13).
That is correct, since \printf returns the number of characters printed. 
The value of \EIP has changed: indeed, now it contains the address of the instruction coming after 
\INS{CALL printf}.
\ECX and \EDX values have changed as well.
Apparently, the \printf function's hidden machinery used them for its own needs.

A very important fact is that neither the \ESP value, nor the stack state have been changed!
We clearly see that the format string and corresponding 3 values are still there.
This is indeed the \IT{cdecl} calling convention behavior: \gls{callee} does not return \ESP back to its previous value.
The \gls{caller} is responsible to do so.

\clearpage
Press F8 again to execute \INS{ADD ESP, 10} instruction:

\begin{figure}[H]
\centering
\myincludegraphics{patterns/03_printf/x86/olly3_4.png}
\caption{\olly: after \INS{ADD ESP, 10} instruction execution}
\label{fig:printf3_olly_4}
\end{figure}

\ESP has changed, but the values are still in the stack!
Yes, of course; no one needs to set these values to zeros or something like that.
Everything above the stack pointer (\ac{SP}) 
is \IT{noise} or \IT{\garbage{}} and has no meaning at all.
It would be time consuming to clear the unused stack entries anyway, and no one really needs to.

\myparagraph{GCC}

Now let's compile the same program in Linux using GCC 4.4.1 and take a look at what we have got in \IDA:

\begin{lstlisting}[style=customasmx86]
main            proc near

var_10          = dword ptr -10h
var_C           = dword ptr -0Ch
var_8           = dword ptr -8
var_4           = dword ptr -4

                push    ebp
                mov     ebp, esp
                and     esp, 0FFFFFFF0h
                sub     esp, 10h
                mov     eax, offset aADBDCD ; "a=%d; b=%d; c=%d"
                mov     [esp+10h+var_4], 3
                mov     [esp+10h+var_8], 2
                mov     [esp+10h+var_C], 1
                mov     [esp+10h+var_10], eax
                call    _printf
                mov     eax, 0
                leave
                retn
main            endp
\end{lstlisting}

Its noticeable that the difference between the MSVC code and the GCC code is only in the way the arguments are stored on the stack.
Here the GCC is working directly with the stack without the use of \PUSH/\POP.

\myparagraph{GCC and GDB}
\myindex{GDB}

Let's try this example also in \ac{GDB} in Linux.

\GTT{-g} option instructs the compiler to include debug information in the executable file.

\begin{lstlisting}
$ gcc 1.c -g -o 1
\end{lstlisting}

\begin{lstlisting}
$ gdb 1
GNU gdb (GDB) 7.6.1-ubuntu
...
Reading symbols from /home/dennis/polygon/1...done.
\end{lstlisting}

\begin{lstlisting}[caption=let's set breakpoint on \printf]
(gdb) b printf
Breakpoint 1 at 0x80482f0
\end{lstlisting}

Run.
We don't have the \printf function source code here, so \ac{GDB} can't show it, but may do so.

\begin{lstlisting}
(gdb) run
Starting program: /home/dennis/polygon/1 

Breakpoint 1, __printf (format=0x80484f0 "a=%d; b=%d; c=%d") at printf.c:29
29	printf.c: No such file or directory.
\end{lstlisting}

Print 10 stack elements. The most left column contains addresses on the stack.

\begin{lstlisting}
(gdb) x/10w $esp
0xbffff11c:	0x0804844a	0x080484f0	0x00000001	0x00000002
0xbffff12c:	0x00000003	0x08048460	0x00000000	0x00000000
0xbffff13c:	0xb7e29905	0x00000001
\end{lstlisting}

The very first element is the \ac{RA} (\GTT{0x0804844a}).
We can verify this by disassembling the memory at this address:

\begin{lstlisting}[label=NOP_as_XCHG_example,style=customasmx86]
(gdb) x/5i 0x0804844a
   0x804844a <main+45>:	mov    $0x0,%eax
   0x804844f <main+50>:	leave  
   0x8048450 <main+51>:	ret    
   0x8048451:	xchg   %ax,%ax
   0x8048453:	xchg   %ax,%ax
\end{lstlisting}

The two \INS{XCHG} instructions are idle instructions, analogous to \ac{NOP}s.

The second element (\GTT{0x080484f0}) is the format string address:

\begin{lstlisting}
(gdb) x/s 0x080484f0
0x80484f0:	"a=%d; b=%d; c=%d"
\end{lstlisting}

Next 3 elements (1, 2, 3) are the \printf arguments.
The rest of the elements could be just \q{garbage} on the stack, 
but could also be values from other functions, their local variables, etc.
We can ignore them for now.

Run \q{finish}. 
The command instructs GDB to \q{execute all instructions until the end of the function}.
In this case: execute till the end of \printf.

\begin{lstlisting}
(gdb) finish
Run till exit from #0  __printf (format=0x80484f0 "a=%d; b=%d; c=%d") at printf.c:29
main () at 1.c:6
6		return 0;
Value returned is $2 = 13
\end{lstlisting}

\ac{GDB} shows what \printf returned in \EAX (13).
This is the number of characters printed out, just like in the \olly example.

We also see \q{return 0;} and the information that this expression is in the \GTT{1.c} file at the line 6.
Indeed, the \GTT{1.c} file is located in the current directory, and \ac{GDB} finds the string there.
How does \ac{GDB} know which C-code line is being currently executed?
This is due to the fact that the compiler,
while generating debugging information, also saves a table of relations between source code line
numbers and instruction addresses.
GDB is a source-level debugger, after all.

Let's examine the registers.
13 in \EAX:

\begin{lstlisting}
(gdb) info registers
eax            0xd	13
ecx            0x0	0
edx            0x0	0
ebx            0xb7fc0000	-1208221696
esp            0xbffff120	0xbffff120
ebp            0xbffff138	0xbffff138
esi            0x0	0
edi            0x0	0
eip            0x804844a	0x804844a <main+45>
...
\end{lstlisting}

Let's disassemble the current instructions.
The arrow points to the instruction to be executed next.

\begin{lstlisting}[style=customasmx86]
(gdb) disas
Dump of assembler code for function main:
   0x0804841d <+0>:	push   %ebp
   0x0804841e <+1>:	mov    %esp,%ebp
   0x08048420 <+3>:	and    $0xfffffff0,%esp
   0x08048423 <+6>:	sub    $0x10,%esp
   0x08048426 <+9>:	movl   $0x3,0xc(%esp)
   0x0804842e <+17>:	movl   $0x2,0x8(%esp)
   0x08048436 <+25>:	movl   $0x1,0x4(%esp)
   0x0804843e <+33>:	movl   $0x80484f0,(%esp)
   0x08048445 <+40>:	call   0x80482f0 <printf@plt>
=> 0x0804844a <+45>:	mov    $0x0,%eax
   0x0804844f <+50>:	leave  
   0x08048450 <+51>:	ret    
End of assembler dump.
\end{lstlisting}

\ac{GDB} uses AT\&T syntax by default.
But it is possible to switch to Intel syntax:

\begin{lstlisting}[style=customasmx86]
(gdb) set disassembly-flavor intel
(gdb) disas
Dump of assembler code for function main:
   0x0804841d <+0>:	push   ebp
   0x0804841e <+1>:	mov    ebp,esp
   0x08048420 <+3>:	and    esp,0xfffffff0
   0x08048423 <+6>:	sub    esp,0x10
   0x08048426 <+9>:	mov    DWORD PTR [esp+0xc],0x3
   0x0804842e <+17>:	mov    DWORD PTR [esp+0x8],0x2
   0x08048436 <+25>:	mov    DWORD PTR [esp+0x4],0x1
   0x0804843e <+33>:	mov    DWORD PTR [esp],0x80484f0
   0x08048445 <+40>:	call   0x80482f0 <printf@plt>
=> 0x0804844a <+45>:	mov    eax,0x0
   0x0804844f <+50>:	leave  
   0x08048450 <+51>:	ret    
End of assembler dump.
\end{lstlisting}

Execute next instruction.
\ac{GDB} shows ending bracket, meaning, it ends the block.

\begin{lstlisting}
(gdb) step
7	};
\end{lstlisting}

Let's examine the registers after the \INS{MOV EAX, 0} instruction execution.
Indeed \EAX is zero at that point.

\begin{lstlisting}
(gdb) info registers
eax            0x0	0
ecx            0x0	0
edx            0x0	0
ebx            0xb7fc0000	-1208221696
esp            0xbffff120	0xbffff120
ebp            0xbffff138	0xbffff138
esi            0x0	0
edi            0x0	0
eip            0x804844f	0x804844f <main+50>
...
\end{lstlisting}

}
\RU{\subsubsection{x86: 3 аргумента}

\myparagraph{MSVC}

Компилируем при помощи MSVC 2010 Express, и в итоге получим:

\begin{lstlisting}[style=customasmx86]
$SG3830	DB	'a=%d; b=%d; c=%d', 00H

...

	push	3
	push	2
	push	1
	push	OFFSET $SG3830
	call	_printf
	add	esp, 16
\end{lstlisting}

Всё почти то же, за исключением того, что теперь видно, что аргументы для \printf заталкиваются в стек в обратном порядке: самый первый аргумент заталкивается последним.

Кстати, вспомним, что переменные типа \Tint в 32-битной системе, как известно, имеют ширину 32 бита, это 4 байта.

Итак, у нас всего 4 аргумента. $4*4 = 16$~--- именно 16 байт занимают в стеке указатель на строку плюс ещё 3 числа типа \Tint.

\myindex{x86!\Instructions!ADD}
\myindex{x86!\Registers!ESP}
\myindex{cdecl}
Когда при помощи инструкции \INS{ADD ESP, X} корректируется \glslink{stack pointer}{указатель стека} \ESP 
после вызова какой-либо функции, зачастую можно сделать вывод о том, сколько аргументов 
у вызываемой функции было, разделив X на 4.

Конечно, это относится только к cdecl-методу передачи аргументов через стек, и только для 32-битной среды.

См. также в соответствующем разделе о способах передачи аргументов через стек ~(\myref{sec:callingconventions}).

Иногда бывает так, что подряд идут несколько вызовов разных функций, но стек корректируется только один раз, после последнего вызова:

\begin{lstlisting}[style=customasmx86]
push a1
push a2
call ...
...
push a1
call ...
...
push a1
push a2
push a3
call ...
add esp, 24
\end{lstlisting}

Вот пример из реальной жизни:

\lstinputlisting[caption=x86,style=customasmx86]{patterns/03_printf/x86/add_example_RU.lst}

\clearpage
\myparagraph{MSVC и \olly}
\myindex{\olly}

Попробуем этот же пример в \olly.
Это один из наиболее популярных win32-отладчиков пользовательского режима.
Мы можем компилировать наш пример в MSVC 2012 
с опцией \GTT{/MD} что означает линковать с библиотекой \GTT{MSVCR*.DLL},
чтобы импортируемые функции были хорошо видны в отладчике.

Затем загружаем исполняемый файл в \olly.
Самая первая точка останова в \GTT{ntdll.dll}, нажмите F9 (запустить).
Вторая точка останова в \ac{CRT}-коде.
Теперь мы должны найти функцию \main.

Найдите этот код, прокрутив окно кода до самого верха (MSVC располагает функцию \main в самом начале секции кода): 

\begin{figure}[H]
\centering
\myincludegraphics{patterns/03_printf/x86/olly3_1.png}
\caption{\olly: самое начало функции \main}
\label{fig:printf3_olly_1}
\end{figure}

Кликните на инструкции \INS{PUSH EBP}, нажмите F2 (установка точки останова) и нажмите F9 (запустить).
Нам нужно произвести все эти манипуляции, чтобы пропустить \ac{CRT}-код, потому что нам он пока
не интересен.

\clearpage
Нажмите F8 (\stepover) 6 раз, т.е. пропустить 6 инструкций:

\begin{figure}[H]
\centering
\myincludegraphics{patterns/03_printf/x86/olly3_2.png}
\caption{\olly: перед исполнением \printf}
\label{fig:printf3_olly_2}
\end{figure}

Теперь \ac{PC} указывает на инструкцию \INS{CALL printf}.
\olly, как и другие отладчики, подсвечивает регистры со значениями, которые изменились.
Поэтому каждый раз когда мы нажимаем F8, \EIP изменяется и его значение подсвечивается красным.
\ESP также меняется, потому что значения заталкиваются в стек.\\
\\
Где находятся эти значения в стеке?
Посмотрите на правое нижнее окно в отладчике:

\begin{figure}[H]
\centering
\input{patterns/03_printf/x86/incl_olly3_stack}
\caption{\olly: стек с сохраненными значениями (красная рамка добавлена в графическом редакторе)}
\end{figure}

Здесь видно 3 столбца: адрес в стеке, значение в стеке и ещё дополнительный комментарий
от \olly. 
\olly понимает \printf{}-строки, так что он показывает здесь и строку и 3 значения \IT{привязанных} к ней.

Можно кликнуть правой кнопкой мыши на строке формата, кликнуть на \q{Follow in dump}
и строка формата появится в окне слева внизу, где всегда виден какой-либо участок памяти.
Эти значения в памяти можно редактировать.
Можно изменить саму строку формата, и тогда результат работы нашего примера будет другой.
В данном случае пользы от этого немного, но для упражнения это полезно,
чтобы начать чувствовать как тут всё работает.

\clearpage
Нажмите F8 (\stepover).

В консоли мы видим вывод:

\lstinputlisting{patterns/03_printf/x86/console.txt}

Посмотрим как изменились регистры и состояние стека: 

\begin{figure}[H]
\centering
\myincludegraphics{patterns/03_printf/x86/olly3_3.png}
\caption{\olly после исполнения \printf}
\label{fig:printf3_olly_3}
\end{figure}

Регистр \EAX теперь содержит \GTT{0xD} (13).
Всё верно: \printf возвращает количество выведенных символов.
Значение \EIP изменилось. Действительно, теперь здесь адрес инструкции после \INS{CALL printf}.
Значения регистров \ECX и \EDX также изменились.
Очевидно, внутренности функции \printf используют их для каких-то своих нужд.

Очень важно то, что значение \ESP не изменилось. И аргументы-значения в стеке также!
Мы ясно видим здесь и строку формата и соответствующие ей 3 значения, они всё ещё здесь.
Действительно, по соглашению вызовов \IT{cdecl}, вызываемая функция не возвращает \ESP назад.
Это должна делать вызывающая функция (\gls{caller}).

\clearpage
Нажмите F8 снова, чтобы исполнилась инструкция \INS{ADD ESP, 10}:

\begin{figure}[H]
\centering
\myincludegraphics{patterns/03_printf/x86/olly3_4.png}
\caption{\olly: после исполнения инструкции \INS{ADD ESP, 10}}
\label{fig:printf3_olly_4}
\end{figure}

\ESP изменился, но значения всё ещё в стеке!
Конечно, никому не нужно заполнять эти значения нулями или что-то в этом роде.
Всё что выше указателя стека (\ac{SP}) 
это \IT{шум} или \IT{\garbage{}} и не имеет особой ценности.
Было бы очень затратно по времени очищать ненужные элементы стека, к тому же, никому это и не нужно.

\myparagraph{GCC}

Скомпилируем то же самое в Linux при помощи GCC 4.4.1 и посмотрим на результат в \IDA:

\begin{lstlisting}[style=customasmx86]
main      proc near

var_10    = dword ptr -10h
var_C     = dword ptr -0Ch
var_8     = dword ptr -8
var_4     = dword ptr -4

          push    ebp
          mov     ebp, esp
          and     esp, 0FFFFFFF0h
          sub     esp, 10h
          mov     eax, offset aADBDCD ; "a=%d; b=%d; c=%d"
          mov     [esp+10h+var_4], 3
          mov     [esp+10h+var_8], 2
          mov     [esp+10h+var_C], 1
          mov     [esp+10h+var_10], eax
          call    _printf
          mov     eax, 0
          leave
          retn
main      endp
\end{lstlisting}

Можно сказать, что этот короткий код, созданный GCC, отличается от кода MSVC только способом помещения 
значений в стек.
Здесь GCC снова работает со стеком напрямую без \PUSH/\POP.

\myparagraph{GCC и GDB}
\myindex{GDB}

Попробуем также этот пример и в \ac{GDB} в Linux.

\GTT{-g} означает генерировать отладочную информацию в выходном исполняемом файле.

\begin{lstlisting}
$ gcc 1.c -g -o 1
\end{lstlisting}

\begin{lstlisting}
$ gdb 1
GNU gdb (GDB) 7.6.1-ubuntu
...
Reading symbols from /home/dennis/polygon/1...done.
\end{lstlisting}

\begin{lstlisting}[caption=установим точку останова на \printf]
(gdb) b printf
Breakpoint 1 at 0x80482f0
\end{lstlisting}

Запукаем.
У нас нет исходного кода функции, поэтому \ac{GDB} не может его показать.

\begin{lstlisting}
(gdb) run
Starting program: /home/dennis/polygon/1 

Breakpoint 1, __printf (format=0x80484f0 "a=%d; b=%d; c=%d") at printf.c:29
29	printf.c: No such file or directory.
\end{lstlisting}

Выдать 10 элементов стека. Левый столбец~--- это адрес в стеке.

\begin{lstlisting}
(gdb) x/10w $esp
0xbffff11c:	0x0804844a	0x080484f0	0x00000001	0x00000002
0xbffff12c:	0x00000003	0x08048460	0x00000000	0x00000000
0xbffff13c:	0xb7e29905	0x00000001
\end{lstlisting}

Самый первый элемент это \ac{RA} (\GTT{0x0804844a}).
Мы можем удостовериться в этом, дизассемблируя память по этому адресу:

\begin{lstlisting}[label=NOP_as_XCHG_example,style=customasmx86]
(gdb) x/5i 0x0804844a
   0x804844a <main+45>:	mov    $0x0,%eax
   0x804844f <main+50>:	leave  
   0x8048450 <main+51>:	ret    
   0x8048451:	xchg   %ax,%ax
   0x8048453:	xchg   %ax,%ax
\end{lstlisting}

Две инструкции \INS{XCHG} это холостые инструкции, аналогичные \ac{NOP}.

Второй элемент (\GTT{0x080484f0}) это адрес строки формата:

\begin{lstlisting}
(gdb) x/s 0x080484f0
0x80484f0:	"a=%d; b=%d; c=%d"
\end{lstlisting}

Остальные 3 элемента (1, 2, 3) это аргументы функции \printf.
Остальные элементы это может быть и мусор в стеке, но могут быть и значения
от других функций, их локальные переменные, итд.
Пока что мы можем игнорировать их.

Исполняем \q{finish}. 
Это значит исполнять все инструкции до самого конца функции. 
В данном случае это означает исполнять до завершения \printf.

\begin{lstlisting}
(gdb) finish
Run till exit from #0  __printf (format=0x80484f0 "a=%d; b=%d; c=%d") at printf.c:29
main () at 1.c:6
6		return 0;
Value returned is $2 = 13
\end{lstlisting}

\ac{GDB} показывает, что вернула \printf в \EAX (13).
Это, так же как и в примере с \olly, количество напечатанных символов.

А ещё мы видим \q{return 0;} и что это выражение находится в файле \GTT{1.c} в строке 6.
Действительно, файл \GTT{1.c} лежит в текущем директории и \ac{GDB} находит там эту строку.
Как \ac{GDB} знает, какая строка Си-кода сейчас исполняется?
Компилятор, генерируя отладочную информацию, также сохраняет информацию о соответствии строк в исходном коде и адресов инструкций.
GDB это всё-таки отладчик уровня исходных текстов.

Посмотрим регистры.
13 в \EAX:

\begin{lstlisting}
(gdb) info registers
eax            0xd	13
ecx            0x0	0
edx            0x0	0
ebx            0xb7fc0000	-1208221696
esp            0xbffff120	0xbffff120
ebp            0xbffff138	0xbffff138
esi            0x0	0
edi            0x0	0
eip            0x804844a	0x804844a <main+45>
...
\end{lstlisting}

Попробуем дизассемблировать текущие инструкции.
Стрелка указывает на инструкцию, которая будет исполнена следующей.

\begin{lstlisting}[style=customasmx86]
(gdb) disas
Dump of assembler code for function main:
   0x0804841d <+0>:	push   %ebp
   0x0804841e <+1>:	mov    %esp,%ebp
   0x08048420 <+3>:	and    $0xfffffff0,%esp
   0x08048423 <+6>:	sub    $0x10,%esp
   0x08048426 <+9>:	movl   $0x3,0xc(%esp)
   0x0804842e <+17>:	movl   $0x2,0x8(%esp)
   0x08048436 <+25>:	movl   $0x1,0x4(%esp)
   0x0804843e <+33>:	movl   $0x80484f0,(%esp)
   0x08048445 <+40>:	call   0x80482f0 <printf@plt>
=> 0x0804844a <+45>:	mov    $0x0,%eax
   0x0804844f <+50>:	leave  
   0x08048450 <+51>:	ret    
End of assembler dump.
\end{lstlisting}

По умолчанию \ac{GDB} показывает дизассемблированный листинг в формате AT\&T.
Но можно также переключиться в формат Intel:

\begin{lstlisting}[style=customasmx86]
(gdb) set disassembly-flavor intel
(gdb) disas
Dump of assembler code for function main:
   0x0804841d <+0>:	push   ebp
   0x0804841e <+1>:	mov    ebp,esp
   0x08048420 <+3>:	and    esp,0xfffffff0
   0x08048423 <+6>:	sub    esp,0x10
   0x08048426 <+9>:	mov    DWORD PTR [esp+0xc],0x3
   0x0804842e <+17>:	mov    DWORD PTR [esp+0x8],0x2
   0x08048436 <+25>:	mov    DWORD PTR [esp+0x4],0x1
   0x0804843e <+33>:	mov    DWORD PTR [esp],0x80484f0
   0x08048445 <+40>:	call   0x80482f0 <printf@plt>
=> 0x0804844a <+45>:	mov    eax,0x0
   0x0804844f <+50>:	leave  
   0x08048450 <+51>:	ret    
End of assembler dump.
\end{lstlisting}

Исполняем следующую инструкцию.
\ac{GDB} покажет закрывающуюся скобку, означая, что это конец блока в функции.

\begin{lstlisting}
(gdb) step
7	};
\end{lstlisting}

Посмотрим регистры после исполнения инструкции \INS{MOV EAX, 0}.
\EAX здесь уже действительно ноль.

\begin{lstlisting}
(gdb) info registers
eax            0x0	0
ecx            0x0	0
edx            0x0	0
ebx            0xb7fc0000	-1208221696
esp            0xbffff120	0xbffff120
ebp            0xbffff138	0xbffff138
esi            0x0	0
edi            0x0	0
eip            0x804844f	0x804844f <main+50>
...
\end{lstlisting}
}
\PTBR{\subsubsection{x86: 3 argumentos}

\myparagraph{MSVC}

Quando compilamos esse código com o MSVC 2010 Express temos:

\begin{lstlisting}[style=customasmx86]
$SG3830	DB	'a=%d; b=%d; c=%d', 00H

...

	push	3
	push	2
	push	1
	push	OFFSET $SG3830
	call	_printf
	add	esp, 16					; 00000010H
\end{lstlisting}

Quase a mesma coisa, mas agora nós podemos ver que os argumentos da função \printf são empurrados na pilha na ordem reversa. O primeiro argumento é empurrado por último.

A propósito, variáveis do tipo \Tint em ambientes 32-bits tem 32-bits de largura, isso é 4 bytes.

Então, nós temos quatro argumentos aqui, $4*4=16$ --- eles ocupam exatamente 16 bytes na pilha: um ponteiro de 32-bits para uma string e três números do tipo \Tint.

\myindex{x86!\Instructions!ADD}
\myindex{x86!\Registers!ESP}
\myindex{cdecl}
Quando o ponteiro da pilha (registrador \ESP) volta para seu valor anterior pela instrução \INS{ADD ESP, X} depois de uma chamada de função,
geralmente o número de argumentos pode ser obtido simplismente por se dividir X, o argumento da função \ADD, por 4.

Lógico que isso é por causa da convenção de chamada do \IT{cdecl} e somente para ambientes 32-bits.

% TODO: Olly, GCC, GDB

}
\ITA{\subsubsection{x86: 3 argomenti}

\myparagraph{MSVC}

Quando compiliamo l'esempio con MSVC 2010 Express otteniamo:

\begin{lstlisting}[style=customasmx86]
$SG3830	DB	'a=%d; b=%d; c=%d', 00H

...

	push	3
	push	2
	push	1
	push	OFFSET $SG3830
	call	_printf
	add	esp, 16					; 00000010H
\end{lstlisting}

Notiamo che gli argomenti di \printf sono messi sullo stack in ordine inverso. Il primo argomento e' quello messo per ultimo.

A proposito, le variabili di tipo \Tint in ambienti a 32-bit hanno dimensione pari a 32-bit, ovvero 4 byte.

Abbiamo 4 argomenti, quindi $4*4 = 16$~---essi occupano esattamente 16 bytes nello stack: un puntatore a 32-bit alla stringa,  e 3 numeri di tipo \Tint.

\myindex{x86!\Instructions!ADD}
\myindex{x86!\Registers!ESP}
\myindex{cdecl}
Quando lo \gls{stack pointer} (registro \ESP) viene ripristinato dall'istruzione \INS{ADD ESP, X}
dopo una chiamata a funzione, in molti casi il numero di argomenti della fuznione puo' essere dedotto semplicemente dividendo X per 4.

Questa caratteristca  e' propria solo della calling convention \IT{cdecl} e in ambienti a 32-bit..

Si veda anche la sezione sulle calling conventions ~(\myref{sec:callingconventions}).

In certi casi in qui diverse funzioni ritornano una dopo l'altra, il compilatore potrebbe accorpare piu' istruzioni \q{ADD ESP, X} 
dopo l'ultima chiamata, emettendo una sola istruzione:

\begin{lstlisting}[style=customasmx86]
push a1
push a2
call ...
...
push a1
call ...
...
push a1
push a2
push a3
call ...
add esp, 24
\end{lstlisting}

Ecco un esempio reale:

% TODO translate to Italian
\lstinputlisting[caption=x86,style=customasmx86]{patterns/03_printf/x86/add_example_EN.lst}

\clearpage
\myparagraph{MSVC and \olly}
\myindex{\olly}

Proviamo ora ad esaminare l'esempio con \olly, uno dei piu' diffusi debugger win32 user-land.
Possiamo compilarlo in MSVC 2012 con l'opzione \GTT{/MD}, che indica al compilatore di linkare con \GTT{MSVCR*.DLL},
in maniera tale da poter vedere in modo chiaro, nel debugger, le funzioni importate.

Carichiamo quindi l'eseguibile in \olly.
Il primo breakpoint avviene in \GTT{ntdll.dll}, premiamo F9 (run) per continuare l'esecuzione. 
Il secondo breakpoint e' nel codice \ac{CRT}.
Ora dobbiamo trovare la funzione \main.

Per farlo scorriamo il codice verso l'alto (MSVC alloca la funzione \main proprio all'inizio della sezione code): 
\begin{figure}[H]
\centering
\myincludegraphics{patterns/03_printf/x86/olly3_1.png}
\caption{\olly: the very start of the \main function}
\label{fig:printf3_olly_1}
\end{figure}

Clickiamo sull'istruzione \INS{PUSH EBP}, premiamo F2 (set breakpoint) and quindi F9 (run).
Queste azioni ci consentono di saltare tutta la parte legata al codice \ac{CRT}, in quanto per il momento non ci interessa.

\clearpage
Premiamo F8 (\stepover) 6 volte, ovverso saltiamo (avanziamo di) 6 istruzioni:

\begin{figure}[H]
\centering
\myincludegraphics{patterns/03_printf/x86/olly3_2.png}
\caption{\olly: before \printf execution}
\label{fig:printf3_olly_2}
\end{figure}

Adesso il \ac{PC} punta all'istruzione \INS{CALL printf}.
\olly, come altri debugger, evidenzia il valore dei registri che sono stati modificati.
Quindi ogni volta che si preme F8, \EIP cambia ed il suo valore e' mostrato in rosso.
Anche \ESP cambia, poiche' i valori degli argomenti vengono messi sullo stack.

Dove sono i valori messi nello stack?
Diamo un'occhiata alla finestra del debugger in basso a destra:

\begin{figure}[H]
\centering
\input{patterns/03_printf/x86/incl_olly3_stack}
\caption{\olly: stato dello stack dopo il push dei valori degli argomenti (il rettangolo rosso e' stato aggiunto dall'autore per evidenziare la finestra)}
\end{figure}

Notiamo 3 colonne: indirizzo nello stack, valore nello stack ed alcuni commenti aggiuntivi di \olly. 
\olly riconosce le stringhe \printf{}-like , e riporta quindi la stringa insieme ai 3 valori \IT{associati}.

Facendo click destro sulla formato string, quindi click su \q{Follow in dump},
e' possibile vedere la format string nella finestra del debugger in basso a sinistra, che mostra una zona di memoria.
I valori mostrati possono anche essere modificati.
E' ad esempio possibile cambiare la format string, che renderebbe diverso il risultato dell'esempio.
Non e' molto utile in questo particolare caso, ma puo' comunque essere un esercizio utile per iniziare a prendere dimestichezza con lo strumento.

\clearpage
Premiamo F8 (\stepover).

Il seguente output viene riportato in console:

\lstinputlisting{patterns/03_printf/x86/console.txt}

Vediamo come sono cambiati i registri e lo stack:

\begin{figure}[H]
\centering
\myincludegraphics{patterns/03_printf/x86/olly3_3.png}
\caption{\olly dopo dell'esecuzione di \printf{}}
\label{fig:printf3_olly_3}
\end{figure}

Il registro \EAX adesso contiene \GTT{0xD} (13).
Questo valore e' corretto, poiche' \printf restituisce il numero di caratteri stampati. 
Il valore di \EIP e' cambiato: adesso contiene infatti l'indirizzo dell'istruzione che viene dopo \INS{CALL printf}.
Anche i valori \ECX e \EDX sono cambiati.
Apparentemente, i meccanismi interni alla funzione \printf hanno usato quei registri durante l'esecuzione di printf per le sue necessita'.

Un fatto molto importante e' che ne' il valore di ESP ne' lo stack sono stati modificati!
Vediamo chiaramente che la format string ed i suoi 3 valori si trovano ancora li'.
Questo e; infatti il comportamento della calling convention \IT{cdecl}: il \gls{callee} (la funzione chiamata) non ripristina
\ESP al suo valore precedente. Farlo e' una responsabilita' del The \gls{caller} (chiamante).

\clearpage
Premiamo F8 nuovamente per eseguire l'istruzione \INS{ADD ESP, 10}:

\begin{figure}[H]
\centering
\myincludegraphics{patterns/03_printf/x86/olly3_4.png}
\caption{\olly: after \INS{ADD ESP, 10} instruction execution}
\label{fig:printf3_olly_4}
\end{figure}

\ESP e' cambiato, ma i valori si trovano ancora sullo stack!
Ovviamente si: non c'e' necessita' di azzerare i valori o effettuare altre simili operazioni di "pulizia".
Qualunque cosa si trovi sopra lo stack pointer (\ac{SP}) e' \IT{noise} o \IT{\garbage{}} e non ha alcun significato.
Pulire i valori non utilizzati nello stack sarebbe una perdita di tempo inutile, e non vi e' alcuna necessita' di farlo.

\myparagraph{GCC}

Compiliamo adesso lo stesso programma su Linux usando GCC 4.4.1, e diamo un'occhiata al risultato con \IDA:

\begin{lstlisting}[style=customasmx86]
main            proc near

var_10          = dword ptr -10h
var_C           = dword ptr -0Ch
var_8           = dword ptr -8
var_4           = dword ptr -4

                push    ebp
                mov     ebp, esp
                and     esp, 0FFFFFFF0h
                sub     esp, 10h
                mov     eax, offset aADBDCD ; "a=%d; b=%d; c=%d"
                mov     [esp+10h+var_4], 3
                mov     [esp+10h+var_8], 2
                mov     [esp+10h+var_C], 1
                mov     [esp+10h+var_10], eax
                call    _printf
                mov     eax, 0
                leave
                retn
main            endp
\end{lstlisting}

Si nota che la differenza tra il codice prodotto da MSVC e GCC risiede soltanto nel modo in cui gli argomenti sono memorizzati sullo stack.
In questo caso GCC lavora diversamente con lo stack senza l'uso di \PUSH/\POP.

\myparagraph{GCC e GDB}
\myindex{GDB}

Proviamo l'esempio anche con \ac{GDB} su Linux.

L'opzione \GTT{-g} indica al comiplatore di includere le informazioni di debug nel file eseguibile.

\begin{lstlisting}
$ gcc 1.c -g -o 1
\end{lstlisting}

\begin{lstlisting}
$ gdb 1
GNU gdb (GDB) 7.6.1-ubuntu
...
Reading symbols from /home/dennis/polygon/1...done.
\end{lstlisting}

\begin{lstlisting}[caption=impostiamo un breakpoint su \printf]
(gdb) b printf
Breakpoint 1 at 0x80482f0
\end{lstlisting}

Avviamolo.
Non abbiamo il sorgente della funzione \printf qui, quindi \ac{GDB} non puo' mostrarlo, anche se potrebbe.

\begin{lstlisting}
(gdb) run
Starting program: /home/dennis/polygon/1 

Breakpoint 1, __printf (format=0x80484f0 "a=%d; b=%d; c=%d") at printf.c:29
29	printf.c: No such file or directory.
\end{lstlisting}

Stampiamo 10 elementi dello stack elements. La colonna piu' a sinistra contiene l'indirizzo nello stack.

\begin{lstlisting}
(gdb) x/10w $esp
0xbffff11c:	0x0804844a	0x080484f0	0x00000001	0x00000002
0xbffff12c:	0x00000003	0x08048460	0x00000000	0x00000000
0xbffff13c:	0xb7e29905	0x00000001
\end{lstlisting}

Il primo elemento e' il \ac{RA} (\GTT{0x0804844a}).
Possiamo verificarlo disassemblando la memoria a questo indirizzo:

\begin{lstlisting}[label=NOP_as_XCHG_example,style=customasmx86]
(gdb) x/5i 0x0804844a
   0x804844a <main+45>:	mov    $0x0,%eax
   0x804844f <main+50>:	leave  
   0x8048450 <main+51>:	ret    
   0x8048451:	xchg   %ax,%ax
   0x8048453:	xchg   %ax,%ax
\end{lstlisting}

Le due istruzioni \INS{XCHG} sono istruzioni \q{inutili} (idle), analoghe a \ac{NOP}.

Il secondo elemento (\GTT{0x080484f0}) e' l'indirizzo della format string:

\begin{lstlisting}
(gdb) x/s 0x080484f0
0x80484f0:	"a=%d; b=%d; c=%d"
\end{lstlisting}

I successivi 3 elementi (1, 2, 3) sono gli argomenti di \printf.
Il resto degli elementi potrebbe essere \q{immondizia} nello stack, oppure valori provenienti da altre funzioni, come le loro variabili locali, etc.
Per il momento li possiamo ignorare.

Eseguiamo il comando \q{finish}. 
Questo comando dice a GDB di \q{eseguire tutte le istruzioni fino alla fine della funzione}.
In questo caso: esegui fino alla fine di \printf.

\begin{lstlisting}
(gdb) finish
Run till exit from #0  __printf (format=0x80484f0 "a=%d; b=%d; c=%d") at printf.c:29
main () at 1.c:6
6		return 0;
Value returned is $2 = 13
\end{lstlisting}

\ac{GDB} mostr il risultato di \printf restituito in \EAX (13).
Questo e' il numero di caratteri stampati, proprio come nell'esempio in \olly.

Vediamo anche \q{return 0;} e l'informazione che questa espressione si trova nel file \GTT{1.c} a riga 6.
Infatti il file \GTT{1.c} si trova nella directory corrente, e \ac{GDB} trova la stringa li'.
Come fa \ac{GDB} a sapere quale riga del codice C viene eseguita?
Cio' e' dovuto al fatto che il compilatore, quando genera le informazioni di debug, salva anche una tabella di relazioni tra le righe del
codice sorgente e gli indirizzi delle istruzioni.
Dopotutto GDB e' un \q{source-level debugger}.

Esaminiamo i registri.
13 in \EAX:

\begin{lstlisting}
(gdb) info registers
eax            0xd	13
ecx            0x0	0
edx            0x0	0
ebx            0xb7fc0000	-1208221696
esp            0xbffff120	0xbffff120
ebp            0xbffff138	0xbffff138
esi            0x0	0
edi            0x0	0
eip            0x804844a	0x804844a <main+45>
...
\end{lstlisting}

Disassembliamo le istruzioni correnti.
La freccia punta all'a prossima istruzione da eseguire.

\begin{lstlisting}[style=customasmx86]
(gdb) disas
Dump of assembler code for function main:
   0x0804841d <+0>:	push   %ebp
   0x0804841e <+1>:	mov    %esp,%ebp
   0x08048420 <+3>:	and    $0xfffffff0,%esp
   0x08048423 <+6>:	sub    $0x10,%esp
   0x08048426 <+9>:	movl   $0x3,0xc(%esp)
   0x0804842e <+17>:	movl   $0x2,0x8(%esp)
   0x08048436 <+25>:	movl   $0x1,0x4(%esp)
   0x0804843e <+33>:	movl   $0x80484f0,(%esp)
   0x08048445 <+40>:	call   0x80482f0 <printf@plt>
=> 0x0804844a <+45>:	mov    $0x0,%eax
   0x0804844f <+50>:	leave  
   0x08048450 <+51>:	ret    
End of assembler dump.
\end{lstlisting}

\ac{GDB} usa la sintassi AT\&T di default.
Ma e' anche possibile passare alla sintassi Intel:

\begin{lstlisting}[style=customasmx86]
(gdb) set disassembly-flavor intel
(gdb) disas
Dump of assembler code for function main:
   0x0804841d <+0>:	push   ebp
   0x0804841e <+1>:	mov    ebp,esp
   0x08048420 <+3>:	and    esp,0xfffffff0
   0x08048423 <+6>:	sub    esp,0x10
   0x08048426 <+9>:	mov    DWORD PTR [esp+0xc],0x3
   0x0804842e <+17>:	mov    DWORD PTR [esp+0x8],0x2
   0x08048436 <+25>:	mov    DWORD PTR [esp+0x4],0x1
   0x0804843e <+33>:	mov    DWORD PTR [esp],0x80484f0
   0x08048445 <+40>:	call   0x80482f0 <printf@plt>
=> 0x0804844a <+45>:	mov    eax,0x0
   0x0804844f <+50>:	leave  
   0x08048450 <+51>:	ret    
End of assembler dump.
\end{lstlisting}

Eseguiamo la prossima istruzione.
\ac{GDB} mostra la parentesi graffa chiusa, che sta a significare la fine del blocco di codice.

\begin{lstlisting}
(gdb) step
7	};
\end{lstlisting}

Esaminiamo i registri dopo l'esecuzione dell'istruzione \INS{MOV EAX, 0}.
\EAX e' zero.

\begin{lstlisting}
(gdb) info registers
eax            0x0	0
ecx            0x0	0
edx            0x0	0
ebx            0xb7fc0000	-1208221696
esp            0xbffff120	0xbffff120
ebp            0xbffff138	0xbffff138
esi            0x0	0
edi            0x0	0
eip            0x804844f	0x804844f <main+50>
...
\end{lstlisting}
}
\DE{\subsubsection{x86: 3 Argumente}

\myparagraph{MSVC}

Beim Kompilieren mit MSVC 2010 Express wird folgende Ausgabe erzeugt:

\begin{lstlisting}[style=customasmx86]
$SG3830	DB	'a=%d; b=%d; c=%d', 00H

...

	push	3
	push	2
	push	1
	push	OFFSET $SG3830
	call	_printf
	add	esp, 16					; 00000010H
\end{lstlisting}

Fast die gleiche Ausgabe, allerdings sind jetzt die \printf-Argumente in umgekehrter Reihenfolge
auf dem Stack hinterlegt. Das erste Argument kommt zuerst.

Die Variablen vom Typ \Tint sind in einer 32-Bit-Umgebung 32 Bit breit, was 4 Byte entspricht.

In diesem Fall sind 4 Argumente vorhanden. $4*4 = 16$~---diese benötigen exakt 16 Byte auf dem Stack
einen 32-Bit-Zeiger auf eine Zeichenkette und 3 Zahlen des Typs \Tint.

\myindex{x86!\Instructions!ADD}
\myindex{x86!\Registers!ESP}
\myindex{cdecl}
Wenn der \gls{stack pointer} (\ESP-Register) nach einem Funktionsaufruf durch die Anweisung
\INS{ADD ESP, X} wiederhergestellt wird, wird die Anzahl Argumente oft einfach durch
die Division von X durch 4 abgeleitet.

Natürlich ist dies eine Eigenheit der \IT{cdecl}-Aufrufkonvention und nur für 32-Bit-Umgebungen gültig
(siehe auch Abschnitt über Aufrufkonventionen~(\myref{sec:callingconventions})).

In bestimmten Fällen in denen mehrere Funktionen direkt hintereinander beendet werden, kann
der Compiler mehrere \q{ADD ESP, X}-Anweisungen in einer zusammenfassen:

\begin{lstlisting}[style=customasmx86]
push a1
push a2
call ...
...
push a1
call ...
...
push a1
push a2
push a3
call ...
add esp, 24
\end{lstlisting}

Hier ist ein Beispiel aus einer realen Applikation:

\lstinputlisting[caption=x86,style=customasmx86]{patterns/03_printf/x86/add_example_DE.lst}

\clearpage
\myparagraph{MSVC und \olly}
\myindex{\olly}

% TODO User-Land is not the correct term...
Sehen wir uns das Beispiel in \olly an, der einer der populärsten User-Land-Win32-Debugger ist.
Wenn das Beispiel in MSVC 2012 mit der Option \GTT{/MD} kompiliert wird, wird gegen \GTT{MSVCR*.DLL} gelinkt.
Somit kann die importierte Funktion klar im Debugger gesehen werden.

Anschließend wird die ausführbare Datei in \olly geladen.
Der erste Breakpoint ist in \GTT{ntdll.dll}, F9 startet die Ausführung.
Der zweite Breakpoint ist im \ac{CRT}-Code.

Jetzt muss die \main-Funktion gefunden werden, die sich ganz oben im Code befindet
(MSVC allokiert die \main-Funktion ganz zu Beginn der Code-Sektion): 
\begin{figure}[H]
\centering
\myincludegraphics{patterns/03_printf/x86/olly3_1.png}
\caption{\olly: Der Start der \main-Funktion}
\label{fig:printf3_olly_1}
\end{figure}

Um den \ac{CRT}-Code zu überspringen (der hier nicht von Interesse ist) müssen
folgende Schritte ausgeführt werden: Klicken auf die \INS{PUSH EBP}-Anweisung,
Drücken von F2 (Breakpoint setzen) und Drücken von F9 (Starten

\clearpage
Sechsmaliges Drücken von F8 (\stepover) überspringt sechs Anweisungen:

\begin{figure}[H]
\centering
\myincludegraphics{patterns/03_printf/x86/olly3_2.png}
\caption{\olly: vor der Ausführung von\printf}
\label{fig:printf3_olly_2}
\end{figure}

Jetzt zeigt der \ac{PC} auf die \INS{CALL printf}-Anweisung.
\olly hebt, wie andere Debugger auch, den Wert des Registers hervor, welches sich geändert hat.
Bei jedem Drücken von F8 ändert sich also \EIP und der Wert wird in rot angezeigt.
\ESP ändert sich ebenfalls, weil die Werte der Argumente auf dem Stack gesichert werden.\\
\\
Wo befinden sich die Werte auf dem Stack? Ein Blick auf das untere, rechte Fenster
des Debuggers liefert die Antwort:

\begin{figure}[H]
\centering
\input{patterns/03_printf/x86/incl_olly3_stack}
\caption{\olly: Stack nachdem der Wert des Arguments gesichert wurde (Das rote Rechteck wurde vom Autor hinzugefügt)}
\end{figure}

Es sind hier drei Spalten sichtbar: Die Adresse im Stack, der Wert im Stack und einige weitere \olly-Kommentare.
\olly versteht \printf{}-ähnliche Zeichenketten und zeigt sie zusammen mit den drei \IT{angehangenen} Werten an.

Es ist möglich einen Rechtsklick auf den Formatstring und anschließend auf \q{Follow in dump} zu klicken.
Der Formatstring wird in dem linken-unterem Debugger erscheinen, welches immer einen Teil des Speichers zeigt.
Diese Speicherwerte können editiert werden.
Es ist möglich den Formatstring zu verändern, was die Ausgabe dieses Beispiels ebenfalls verändern würde.
In diesem Fall ist das nicht sehr nützlich, aber es könnte eine gute Übung sein um ein Gefühl dafür zu
bekommen wie hier alles funktioniert.

\clearpage
Drücken von F8 (\stepover).

Es erscheint die folgende Ausgabe auf der Konsole:

\lstinputlisting{patterns/03_printf/x86/console.txt}

Nachfolgend die Änderungen in den Registern und der Status des Stacks:

\begin{figure}[H]
\centering
\myincludegraphics{patterns/03_printf/x86/olly3_3.png}
\caption{\olly nach der Ausfühung von \printf{}}
\label{fig:printf3_olly_3}
\end{figure}

Das Register \EAX enthält nun \GTT{0xD} (13).
Dies ist auch richtig, weil \printf die Anzahl der ausgegeben Zeichen zurück gibt.
Der Wert von \EIP hat sich verändert: er enthält nun die Adresse der Anweisung, die
nach \INS{CALL printf} kommt.
Die Werte von \ECX und \EDX haben sich ebenfalls geändert.
Offensichtlich nutzt die \printf-Funktion diese für interne Zwecke.

Ein wichtiger Punkt ist, dass weder der \ESP-Wert, noch der Status des Stacks sich geändert haben!
Es ist klar erkennbar, dass der Formatstring und die drei entsprechenden Werte immer noch da sind.
Dies ist das Verhalten der \IT{cdecl}-Aufrufkonvention: \gls{callee} setzt \ESP nicht auf den vorherigen
Wert zurück. Dies ist die Aufgabe der \gls{caller}.

\clearpage
Durch erneutes Drücken von F8 wird die Anweisung \INS{ADD ESP, 10} ausgeführt:

\begin{figure}[H]
\centering
\myincludegraphics{patterns/03_printf/x86/olly3_4.png}
\caption{\olly: Nach der Ausführung von \INS{ADD ESP, 10}}
\label{fig:printf3_olly_4}
\end{figure}

\ESP hat sich geändert, aber die Werte sind immer noch auf dem Stack!
Das liegt natürlich daran, dass es keine Notwendigkeit gibt diese Werte auf Null zu setzen oder Ähnliches.
Alle oberhalb des Stack-Pointers (\ac{SP}) ist \IT{Rauschen} oder \IT{\garbage{}} und hat keine Bedeutung.
Es wäre zeitintensiv und unnötig die ungenutzten Stack-Einträge zurück zu setzen.

\myparagraph{GCC}

Nun wird das gleiche Programm unter Linux mit GCC 4.4.1 kompiliert und in \IDA untersucht:

\begin{lstlisting}[style=customasmx86]
main            proc near

var_10          = dword ptr -10h
var_C           = dword ptr -0Ch
var_8           = dword ptr -8
var_4           = dword ptr -4

                push    ebp
                mov     ebp, esp
                and     esp, 0FFFFFFF0h
                sub     esp, 10h
                mov     eax, offset aADBDCD ; "a=%d; b=%d; c=%d"
                mov     [esp+10h+var_4], 3
                mov     [esp+10h+var_8], 2
                mov     [esp+10h+var_C], 1
                mov     [esp+10h+var_10], eax
                call    _printf
                mov     eax, 0
                leave
                retn
main            endp
\end{lstlisting}

Es ist erkennbar, dass der Unterschied zwischen MSVC- und GCC-Code lediglich die Art ist,
auf der die Argumente auf dem Stack gespeichert werden.
Hier arbeitet GCC direkt mit dem Stack ohne Benutzung von \PUSH/\POP.

\myparagraph{GCC und GDB}
\myindex{GDB}

Nachfolgend das Beispiel mit \ac{GDB} unter Linux.

Die Option \GTT{-g} weist den Compiler an, Debugging-Informationen in die ausführbare Datei einzufügen.

\begin{lstlisting}
$ gcc 1.c -g -o 1
\end{lstlisting}

\begin{lstlisting}
$ gdb 1
GNU gdb (GDB) 7.6.1-ubuntu
...
Reading symbols from /home/dennis/polygon/1...done.
\end{lstlisting}

\begin{lstlisting}[caption=Setzen eines Breakpoints auf \printf]
(gdb) b printf
Breakpoint 1 at 0x80482f0
\end{lstlisting}

Ausführen.
Im Quellcode ist keine \printf-Funktion, also kann \ac{GDB} sie nicht anzeigen.

\begin{lstlisting}
(gdb) run
Starting program: /home/dennis/polygon/1 

Breakpoint 1, __printf (format=0x80484f0 "a=%d; b=%d; c=%d") at printf.c:29
29	printf.c: No such file or directory.
\end{lstlisting}

Ausgeben von 10 Stack-Einträgen. Die Spalte ganz links enthält die Adressen auf dem Stack.

\begin{lstlisting}
(gdb) x/10w $esp
0xbffff11c:	0x0804844a	0x080484f0	0x00000001	0x00000002
0xbffff12c:	0x00000003	0x08048460	0x00000000	0x00000000
0xbffff13c:	0xb7e29905	0x00000001
\end{lstlisting}

Das allererste Element ist \ac{RA} (\GTT{0x0804844a}).
Dies kann durch das Disassemblieren des Speichers an dieser Stelle überprüft werden:

\begin{lstlisting}[label=NOP_as_XCHG_example,style=customasmx86]
(gdb) x/5i 0x0804844a
   0x804844a <main+45>:	mov    $0x0,%eax
   0x804844f <main+50>:	leave  
   0x8048450 <main+51>:	ret    
   0x8048451:	xchg   %ax,%ax
   0x8048453:	xchg   %ax,%ax
\end{lstlisting}

Die beiden \INS{XCHG}-Anweisungen sind Idle-Anweisungen, analog zu \ac{NOP}.

Das zweite Element (\GTT{0x080484f0}) ist die Adresse des Formatstrings:

\begin{lstlisting}
(gdb) x/s 0x080484f0
0x80484f0:	"a=%d; b=%d; c=%d"
\end{lstlisting}

Die nächsten drei Elemente (1, 2, 3) sind die \printf-Argumente.
Der Rest der Elemente kann lediglich \q{garbage} auf dem Stack sein,
oder auch Werte von anderen Funktionen, deren lokale Variablen usw.
An dieser Stelle können sie ignoriert werden.

Ausführen von \q{finish}.
Dieses Kommando führt dazu, dass GDB \q{alle Anweisungen bis zum Ende der Funktion ausführt}.
In diesem Fall: Ausführen bis zum Ende von \printf.

\begin{lstlisting}
(gdb) finish
Run till exit from #0  __printf (format=0x80484f0 "a=%d; b=%d; c=%d") at printf.c:29
main () at 1.c:6
6		return 0;
Value returned is $2 = 13
\end{lstlisting}

\ac{GDB} zeigt was \printf in \EAX zurück gibt (13).
Die ist die Anzahl der ausgegebenen Zeichen, genau wie im \olly-Beispiel.

Sichtbar ist auch \q{return 0;} und die Information das dieser Ausdruck in der Datei \GTT{1.c} in Zeile 6 ist.
\GTT{1.c} befindet sich in aktuellen Verzeichnis und \ac{GDB} kann die Zeichenkette dort finden.
Wie erkennt \ac{GDB} welche C-Zeile gerade ausgeführt wird?
Dies kommt von der Tatsache, dass der Compiler beim Generieren der Debugging-Informationen auch eine
Tabelle mit Verweisen zwischen dem Quellcode-Zeilen und den Anweisungsadressen anlegt.
GDB ist also ein Debugger auf Quellcode-Ebene.

Schauen wir uns die Register an.
13 in \EAX:

\begin{lstlisting}
(gdb) info registers
eax            0xd	13
ecx            0x0	0
edx            0x0	0
ebx            0xb7fc0000	-1208221696
esp            0xbffff120	0xbffff120
ebp            0xbffff138	0xbffff138
esi            0x0	0
edi            0x0	0
eip            0x804844a	0x804844a <main+45>
...
\end{lstlisting}

Nachfolgend das Disassemblieren der aktuellen Anweisung.
Der Pfeil zeigt auf die nächste Anweisung die ausgeführt wird.

\begin{lstlisting}[style=customasmx86]
(gdb) disas
Dump of assembler code for function main:
   0x0804841d <+0>:	push   %ebp
   0x0804841e <+1>:	mov    %esp,%ebp
   0x08048420 <+3>:	and    $0xfffffff0,%esp
   0x08048423 <+6>:	sub    $0x10,%esp
   0x08048426 <+9>:	movl   $0x3,0xc(%esp)
   0x0804842e <+17>:	movl   $0x2,0x8(%esp)
   0x08048436 <+25>:	movl   $0x1,0x4(%esp)
   0x0804843e <+33>:	movl   $0x80484f0,(%esp)
   0x08048445 <+40>:	call   0x80482f0 <printf@plt>
=> 0x0804844a <+45>:	mov    $0x0,%eax
   0x0804844f <+50>:	leave  
   0x08048450 <+51>:	ret    
End of assembler dump.
\end{lstlisting}

\ac{GDB} nutzt standardmäßig den AT\&T-Syntax.
Es ist aber möglich auf den Intel-Syntax zu wechseln:

\begin{lstlisting}[style=customasmx86]
(gdb) set disassembly-flavor intel
(gdb) disas
Dump of assembler code for function main:
   0x0804841d <+0>:	push   ebp
   0x0804841e <+1>:	mov    ebp,esp
   0x08048420 <+3>:	and    esp,0xfffffff0
   0x08048423 <+6>:	sub    esp,0x10
   0x08048426 <+9>:	mov    DWORD PTR [esp+0xc],0x3
   0x0804842e <+17>:	mov    DWORD PTR [esp+0x8],0x2
   0x08048436 <+25>:	mov    DWORD PTR [esp+0x4],0x1
   0x0804843e <+33>:	mov    DWORD PTR [esp],0x80484f0
   0x08048445 <+40>:	call   0x80482f0 <printf@plt>
=> 0x0804844a <+45>:	mov    eax,0x0
   0x0804844f <+50>:	leave  
   0x08048450 <+51>:	ret    
End of assembler dump.
\end{lstlisting}

Ausführen der nächsten Anweisung.
\ac{GDB} zeigt Ende-Klammern, welche den Block schließt.

\begin{lstlisting}
(gdb) step
7	};
\end{lstlisting}

Das Ansehen der Register nach der \INS{MOV EAX, 0}-Anweisung zeigt. dass
das \EAX-Register an dieser Stelle Null ist.

\begin{lstlisting}
(gdb) info registers
eax            0x0	0
ecx            0x0	0
edx            0x0	0
ebx            0xb7fc0000	-1208221696
esp            0xbffff120	0xbffff120
ebp            0xbffff138	0xbffff138
esi            0x0	0
edi            0x0	0
eip            0x804844f	0x804844f <main+50>
...
\end{lstlisting}
}
\FR{\subsubsection{x86: 3 arguments}

\myparagraph{MSVC}

En le compilant avec MSVC 2010 Express nous obtenons:

\begin{lstlisting}[style=customasmx86]
$SG3830	DB	'a=%d; b=%d; c=%d', 00H

...

	push	3
	push	2
	push	1
	push	OFFSET $SG3830
	call	_printf
	add	esp, 16					; 00000010H
\end{lstlisting}

Presque la même chose, mais maintenant nous voyons que les arguments de \printf sont
poussés sur la pile en ordre inverse. Le premier argument est poussé en dernier.

A propos, dans un environnement 32-bit les variables de type \Tint ont une taille
de 32-bit. ce qui fait 4 octets.

Donc, nous avons 4 arguments ici. $4*4 = 16$~---ils occupent exactement 16 octets
dans la pile: un pointeur 32-bit sur une chaîne et 3 nombres de type \Tint.

\myindex{x86!\Instructions!ADD}
\myindex{x86!\Registers!ESP}
\myindex{cdecl}
Lorsque le \glslink{stack pointer}{pointeur de pile} (registre \ESP) est re-modifié
par l'instruction\\ \INS{ADD ESP, X} après un appel de fonction, souvent, le nombre
d'arguments de la fonction peut-être déduit en divisant simplement X par 4.

Bien sûr, cela est spécifique à la convention d'appel \IT{cdecl}, et seulement
pour un environnement 32-bit.

Voir aussi la section sur les conventions d'appel ~(\myref{sec:callingconventions}).

Dans certains cas, plusieurs fonctions se terminent les une après les autres, le
compilateur peut concaténer plusieurs instructions \q{ADD ESP, X} en une seule,
après le dernier appel:

\begin{lstlisting}[style=customasmx86]
push a1
push a2
call ...
...
push a1
call ...
...
push a1
push a2
push a3
call ...
add esp, 24
\end{lstlisting}

Voici un exemple réel:

\lstinputlisting[caption=x86,style=customasmx86]{patterns/03_printf/x86/add_example_FR.lst}

\clearpage
\myparagraph{MSVC et \olly}
\myindex{\olly}

Maintenant, essayons de charger cet exemple dans \olly.
C'est l'un des débuggers en espace-utilisateur win32 les plus populaire.
Nous pouvons compiler notre exemple avec l'option \GTT{/MD} de MSVC 2012, qui signifie
lier avec \GTT{MSVCR*.DLL}, ainsi nous verrons clairement les fonctions importées
dans le debugger.

Ensuite chargeons lexécutable dans \olly.
Le tout premier point d'arrêt est dans \GTT{ntdll.dll}, appuyez
sur F9 (run).
Le second point d'arrêt est dans le code \ac{CRT}.
Nous devons maintenant trouver la fonction \main.

Trouvez ce code en vous déplaçant au tout début du code (MSVC alloue la fonction
\main au tout début de la section de code):
\begin{figure}[H]
\centering
\myincludegraphics{patterns/03_printf/x86/olly3_1.png}
\caption{\olly: le tout début de la fonction \main}
\label{fig:printf3_olly_1}
\end{figure}

Clickez sur l'instruction \INS{PUSH EBP}, pressez F2 (mettre un point d'arrêt)
et pressez F9 (lancer le programme).
Nous devons effectuer ces actions pour éviter le code \ac{CRT}, car il ne nous intéresse
pas pour le moment.

\clearpage
Presser F8 (\stepover) 6 fois, i.e. sauter 6 instructions:

\begin{figure}[H]
\centering
\myincludegraphics{patterns/03_printf/x86/olly3_2.png}
\caption{\olly: avant l'exécution de \printf}
\label{fig:printf3_olly_2}
\end{figure}

Maintenant le \ac{PC} pointe vers l'instruction \INS{CALL printf}.
\olly, comme d'autres debuggers, souligne la valeur des registres qui ont changé.
Donc, chaque fois que vous appusez sur F8, \EIP change et sa valeur est affichée en rouge.
\ESP change aussi, car les valeurs des arguments sont poussées sur la pile.\\
\\
Où sont les valeurs dans la pile?
Regardez en bas à droite de la fenêtre du debugger:

\begin{figure}[H]
\centering
\input{patterns/03_printf/x86/incl_olly3_stack}
\caption{\olly: pile après que les valeurs des arguments aient été poussées (Le
rectangle rouge a été ajouté par l'auteur dans un éditeur graphique)}
\end{figure}

Nous pouvons voir 3 colonnes ici: adresse dans la pile, valeur dans la pile et quelques
commentaires additionnels d'\olly.
\olly reconnait les chaînes de type \printf{}, donc il signale ici la chaîne et
les 3 valeurs \IT{attachées} à elle.

Il est possible de faire un clock-droit sur la chaîne de format, cliquer sur \q{Follow in dump},
et la chaîne de format va apparaitre dans la fenêtre en bas à gauche du debugger. qui affiche
toujours des parties de la mémoire.
Ces valeurs en mémoire peuvent être modifiées.
Il est possible de changer la chaîne de format, auquel cas le résultat de notre
exemple sera différent.
Cela n'est pas très utile dans le cas présent, mais ce peut-être un bon exercice
pour commencer à comprendre comment tout fonctionne ici.

\clearpage
Appuyer sur F8 (\stepover).

Nous voyons la sortie suivante dans la console:

\lstinputlisting{patterns/03_printf/x86/console.txt}

Regardons comment les registres et la pile ont changés:

\begin{figure}[H]
\centering
\myincludegraphics{patterns/03_printf/x86/olly3_3.png}
\caption{\olly après l'exécution de \printf{}}
\label{fig:printf3_olly_3}
\end{figure}

Le registre \EAX contient maintenant \GTT{0xD} (13).
C'est correct, puisque \printf renvoie le nombre de caractères écrits.
La valeur de \EIP a changé: en effet, il contient maintenant l'adresse de l'instruction
venant après
\INS{CALL printf}.
Les valeurs de \ECX et \EDX ont également changé.
Apparemment, le mécanisme interne de la fonction \printf les a utilisés pour dans
ses propres besoins.

Un fait très important est que ni la valeur de \ESP, ni l'état de la pile n'ont
été changés!
Nous voyons clarirement que la chaîne de format et les trois valeurs corrspondantes
sont toujours là.
C'est en effet le comportement de la convention d'appel \IT{cdecl}: \glslink{callee}{l'appelée}
ne doit pas remettre \ESP à sa valeur précédente.
\glslink{caller}{L'appelant} est responsable de le faire.

\clearpage
Appuyer sur F8 à nouveau pour exécuter l'instruction \INS{ADD ESP, 10}:

\begin{figure}[H]
\centering
\myincludegraphics{patterns/03_printf/x86/olly3_4.png}
\caption{\olly: après l'exécution de l'instruction \INS{ADD ESP, 10}}
\label{fig:printf3_olly_4}
\end{figure}

\ESP a changé, mais les valeurs sont toujours dans la pile!
Oui, bien sûr; il n'y a pas besoin de mettre ces valeurs à zéro ou quelque chose
comme ça.
Tout ce qui se trouve au dessus du pointeur de pile (\ac{SP}) est du \IT{bruit}
ou des \IT{données incorrectes} et n'a pas du tout de signification.
Ca prendrait beaucoup de temps de mettre à zéro les entrées inutilisées de la
pile, et personne n'a vraiment besoin de le faire.

\myparagraph{GCC}

Maintenant, compilons la même programme sous Linux en utilisant GCC 4.4.1 et regardons
ce que nous obtenons dans \IDA:

\begin{lstlisting}[style=customasmx86]
main            proc near

var_10          = dword ptr -10h
var_C           = dword ptr -0Ch
var_8           = dword ptr -8
var_4           = dword ptr -4

                push    ebp
                mov     ebp, esp
                and     esp, 0FFFFFFF0h
                sub     esp, 10h
                mov     eax, offset aADBDCD ; "a=%d; b=%d; c=%d"
                mov     [esp+10h+var_4], 3
                mov     [esp+10h+var_8], 2
                mov     [esp+10h+var_C], 1
                mov     [esp+10h+var_10], eax
                call    _printf
                mov     eax, 0
                leave
                retn
main            endp
\end{lstlisting}

Il est visible que la différence entre le code MSVC et celui de GCC est seulement
dans la manière dont les arguments sont stockés sur la pile.
Ici GCC manipule directement la pile sans utiliser \PUSH/\POP.

\myparagraph{GCC and GDB}
\myindex{GDB}

Essayons cet exemple dans \ac{GDB} sous Linux.

L'option \GTT{-g} indique au compilateur d'inclure les informations de debug dans
le fichier exécutable.

\begin{lstlisting}
$ gcc 1.c -g -o 1
\end{lstlisting}

\begin{lstlisting}
$ gdb 1
GNU gdb (GDB) 7.6.1-ubuntu
...
Reading symbols from /home/dennis/polygon/1...done.
\end{lstlisting}

\begin{lstlisting}[caption=let's set breakpoint on \printf]
(gdb) b printf
Breakpoint 1 at 0x80482f0
\end{lstlisting}

Lançons le programme.
Nous n'avons pas la code source de la fonction \printf ici, donc \ac{GDB} ne peut
pas le montrer, mais pourrait.

\begin{lstlisting}
(gdb) run
Starting program: /home/dennis/polygon/1 

Breakpoint 1, __printf (format=0x80484f0 "a=%d; b=%d; c=%d") at printf.c:29
29	printf.c: No such file or directory.
\end{lstlisting}

Afficher 10 éléments de la pile. La colonne la plus à gauche contient les adresses
de la pile.

\begin{lstlisting}
(gdb) x/10w $esp
0xbffff11c:	0x0804844a	0x080484f0	0x00000001	0x00000002
0xbffff12c:	0x00000003	0x08048460	0x00000000	0x00000000
0xbffff13c:	0xb7e29905	0x00000001
\end{lstlisting}

Le tout premier élément est \ac{RA} (\GTT{0x0804844a}).
Nous pouvons le vérifer en désassemblant la mémoire à cette adresse:

\begin{lstlisting}[label=NOP_as_XCHG_example,style=customasmx86]
(gdb) x/5i 0x0804844a
   0x804844a <main+45>:	mov    $0x0,%eax
   0x804844f <main+50>:	leave  
   0x8048450 <main+51>:	ret    
   0x8048451:	xchg   %ax,%ax
   0x8048453:	xchg   %ax,%ax
\end{lstlisting}

Les deux instructions \INS{XCHG} sont des instructions sans effet, analogues à \ac{NOP}s.

Le second élément (\GTT{0x080484f0}) est l'adresse de la chaîne de format:

\begin{lstlisting}
(gdb) x/s 0x080484f0
0x80484f0:	"a=%d; b=%d; c=%d"
\end{lstlisting}

Les 3 éléments suivants (1, 2, 3) sont les arguments de \printf.
Le reste des éléments sont juste des \q{restes} sur la pile, mais peuvent aussi
être des valeurs d'autres fonctions, leurs variables locales, etc.
Nous pouvons les ignorer pour le moment.

Lancer la commande \q{finish}.
Cette commande ordonne à GDB d'\q{exécuter toutes les instructions jusqu'à la
fin de la fonction}.
Dans ce cas: exécuter jusqu'à la fin de \printf.

\begin{lstlisting}
(gdb) finish
Run till exit from #0  __printf (format=0x80484f0 "a=%d; b=%d; c=%d") at printf.c:29
main () at 1.c:6
6		return 0;
Value returned is $2 = 13
\end{lstlisting}

\ac{GDB} montre ce que \printf a renvoyé dans \EAX (13).
C'est le nombre de caractères écrits, exactement comme dans l'exemple avec \olly.

Nous voyons également \q{return 0;} et l'information que cette expression se trouve
à la ligne 6 du fichier \GTT{1.c}.
En effet, le fichier \GTT{1.c} se trouve dans le répertoire courant, et \ac{GDB}
y a trouvé la chaîne.
Comment est-ce que \ac{GDB} sait quelle ligne est exécutée à un instant donné?
Cela est du au fait que lorsque le compilateur génère les informations de debug,
sauve également une table contenant la relation entre le numéro des lignes du code
source et les adresses des instructions.
GDB est un debugger niveau source, après tout.

Examinons les registres.
13 in \EAX:

\begin{lstlisting}
(gdb) info registers
eax            0xd	13
ecx            0x0	0
edx            0x0	0
ebx            0xb7fc0000	-1208221696
esp            0xbffff120	0xbffff120
ebp            0xbffff138	0xbffff138
esi            0x0	0
edi            0x0	0
eip            0x804844a	0x804844a <main+45>
...
\end{lstlisting}

Désassemblons les instructions courantes.
La flèche pointe sur la prochaine instruction qui sera exécutée.

\begin{lstlisting}[style=customasmx86]
(gdb) disas
Dump of assembler code for function main:
   0x0804841d <+0>:	push   %ebp
   0x0804841e <+1>:	mov    %esp,%ebp
   0x08048420 <+3>:	and    $0xfffffff0,%esp
   0x08048423 <+6>:	sub    $0x10,%esp
   0x08048426 <+9>:	movl   $0x3,0xc(%esp)
   0x0804842e <+17>:	movl   $0x2,0x8(%esp)
   0x08048436 <+25>:	movl   $0x1,0x4(%esp)
   0x0804843e <+33>:	movl   $0x80484f0,(%esp)
   0x08048445 <+40>:	call   0x80482f0 <printf@plt>
=> 0x0804844a <+45>:	mov    $0x0,%eax
   0x0804844f <+50>:	leave  
   0x08048450 <+51>:	ret    
End of assembler dump.
\end{lstlisting}

\ac{GDB} utilise la syntaxe AT\&T par défaut.
Mais il est possible de choisir la syntaxe Intel:

\begin{lstlisting}[style=customasmx86]
(gdb) set disassembly-flavor intel
(gdb) disas
Dump of assembler code for function main:
   0x0804841d <+0>:	push   ebp
   0x0804841e <+1>:	mov    ebp,esp
   0x08048420 <+3>:	and    esp,0xfffffff0
   0x08048423 <+6>:	sub    esp,0x10
   0x08048426 <+9>:	mov    DWORD PTR [esp+0xc],0x3
   0x0804842e <+17>:	mov    DWORD PTR [esp+0x8],0x2
   0x08048436 <+25>:	mov    DWORD PTR [esp+0x4],0x1
   0x0804843e <+33>:	mov    DWORD PTR [esp],0x80484f0
   0x08048445 <+40>:	call   0x80482f0 <printf@plt>
=> 0x0804844a <+45>:	mov    eax,0x0
   0x0804844f <+50>:	leave  
   0x08048450 <+51>:	ret    
End of assembler dump.
\end{lstlisting}

Exécuter l'instruction suivante.
\ac{GDB} montre une parenthèse fermante, signifiant la fin du bloc.

\begin{lstlisting}
(gdb) step
7	};
\end{lstlisting}

Examinons les registres après l'exécution de l'instruction \INS{MOV EAX, 0}.
En effet, \EAX est à zéro à ce stade.

\begin{lstlisting}
(gdb) info registers
eax            0x0	0
ecx            0x0	0
edx            0x0	0
ebx            0xb7fc0000	-1208221696
esp            0xbffff120	0xbffff120
ebp            0xbffff138	0xbffff138
esi            0x0	0
edi            0x0	0
eip            0x804844f	0x804844f <main+50>
...
\end{lstlisting}

}


\EN{\subsubsection{x64: 8 arguments}

\myindex{x86-64}
\label{example_printf8_x64}
To see how other arguments are passed via the stack, let's change our example again 
by increasing the number of arguments to 9 (\printf format string + 8 \Tint variables):

\lstinputlisting[style=customc]{patterns/03_printf/2.c}

\myparagraph{MSVC}

As it was mentioned earlier, the first 4 arguments has to be passed through the \RCX, \RDX, \Reg{8}, \Reg{9} registers in Win64, while all the rest---via the stack.
That is exactly what we see here.
However, the \MOV instruction, instead of \PUSH, is used for preparing the stack, so the values are stored to the stack in a straightforward manner.

\lstinputlisting[caption=MSVC 2012 x64,style=customasmx86]{patterns/03_printf/x86/2_MSVC_x64_EN.asm}

The observant reader may ask why are 8 bytes allocated for \Tint values, when 4 is enough?
Yes, one has to recall: 8 bytes are allocated for any data type shorter than 64 bits.
This is established for the convenience's sake: it makes it easy to calculate the address of arbitrary argument.
Besides, they are all located at aligned memory addresses.
It is the same in the 32-bit environments: 4 bytes are reserved for all data types.

% also for local variables?

\myparagraph{GCC}

The picture is similar for x86-64 *NIX OS-es, except that the first 6 arguments are passed through the \RDI, \RSI,
\RDX, \RCX, \Reg{8}, \Reg{9} registers.
All the rest---via the stack.
GCC generates the code storing the string pointer into \EDI instead of \RDI{}---we noted that previously: 
\myref{hw_EDI_instead_of_RDI}.

We also noted earlier that the \EAX register has been cleared before a \printf call: \myref{SysVABI_input_EAX}.

\lstinputlisting[caption=\Optimizing GCC 4.4.6 x64,style=customasmx86]{patterns/03_printf/x86/2_GCC_x64_EN.s}

\myparagraph{GCC + GDB}
\myindex{GDB}

Let's try this example in \ac{GDB}.

\begin{lstlisting}
$ gcc -g 2.c -o 2
\end{lstlisting}

\begin{lstlisting}
$ gdb 2
GNU gdb (GDB) 7.6.1-ubuntu
...
Reading symbols from /home/dennis/polygon/2...done.
\end{lstlisting}

\begin{lstlisting}[caption=let's set the breakpoint to \printf{,} and run]
(gdb) b printf
Breakpoint 1 at 0x400410
(gdb) run
Starting program: /home/dennis/polygon/2 

Breakpoint 1, __printf (format=0x400628 "a=%d; b=%d; c=%d; d=%d; e=%d; f=%d; g=%d; h=%d\n") at printf.c:29
29	printf.c: No such file or directory.
\end{lstlisting}

Registers \RSI/\RDX/\RCX/\Reg{8}/\Reg{9} have the expected values.
\RIP has the address of the very first instruction of the \printf function.

\begin{lstlisting}
(gdb) info registers
rax            0x0	0
rbx            0x0	0
rcx            0x3	3
rdx            0x2	2
rsi            0x1	1
rdi            0x400628	4195880
rbp            0x7fffffffdf60	0x7fffffffdf60
rsp            0x7fffffffdf38	0x7fffffffdf38
r8             0x4	4
r9             0x5	5
r10            0x7fffffffdce0	140737488346336
r11            0x7ffff7a65f60	140737348263776
r12            0x400440	4195392
r13            0x7fffffffe040	140737488347200
r14            0x0	0
r15            0x0	0
rip            0x7ffff7a65f60	0x7ffff7a65f60 <__printf>
...
\end{lstlisting}

\begin{lstlisting}[caption=let's inspect the format string]
(gdb) x/s $rdi
0x400628:	"a=%d; b=%d; c=%d; d=%d; e=%d; f=%d; g=%d; h=%d\n"
\end{lstlisting}

Let's dump the stack with the x/g command this time---\IT{g} stands for \IT{giant words}, i.e., 64-bit words.

\begin{lstlisting}
(gdb) x/10g $rsp
0x7fffffffdf38:	0x0000000000400576	0x0000000000000006
0x7fffffffdf48:	0x0000000000000007	0x00007fff00000008
0x7fffffffdf58:	0x0000000000000000	0x0000000000000000
0x7fffffffdf68:	0x00007ffff7a33de5	0x0000000000000000
0x7fffffffdf78:	0x00007fffffffe048	0x0000000100000000
\end{lstlisting}

The very first stack element, just like in the previous case, is the \ac{RA}.
3 values are also passed through the stack: 6, 7, 8.
We also see that 8 is passed with the high 32-bits not cleared: \GTT{0x00007fff00000008}.
That's OK, because the values are of \Tint type, which is 32-bit.
So, the high register or stack element part may contain \q{random garbage}.

If you take a look at where the control will return after the \printf execution,
\ac{GDB} will show the entire \main function:

\begin{lstlisting}[style=customasmx86]
(gdb) set disassembly-flavor intel
(gdb) disas 0x0000000000400576
Dump of assembler code for function main:
   0x000000000040052d <+0>:	push   rbp
   0x000000000040052e <+1>:	mov    rbp,rsp
   0x0000000000400531 <+4>:	sub    rsp,0x20
   0x0000000000400535 <+8>:	mov    DWORD PTR [rsp+0x10],0x8
   0x000000000040053d <+16>:	mov    DWORD PTR [rsp+0x8],0x7
   0x0000000000400545 <+24>:	mov    DWORD PTR [rsp],0x6
   0x000000000040054c <+31>:	mov    r9d,0x5
   0x0000000000400552 <+37>:	mov    r8d,0x4
   0x0000000000400558 <+43>:	mov    ecx,0x3
   0x000000000040055d <+48>:	mov    edx,0x2
   0x0000000000400562 <+53>:	mov    esi,0x1
   0x0000000000400567 <+58>:	mov    edi,0x400628
   0x000000000040056c <+63>:	mov    eax,0x0
   0x0000000000400571 <+68>:	call   0x400410 <printf@plt>
   0x0000000000400576 <+73>:	mov    eax,0x0
   0x000000000040057b <+78>:	leave  
   0x000000000040057c <+79>:	ret    
End of assembler dump.
\end{lstlisting}

Let's finish executing \printf, execute the instruction
zeroing \EAX, and note that the \EAX register has a value of exactly zero.
\RIP now points to the \INS{LEAVE} instruction, i.e., the penultimate one in the \main function.

\begin{lstlisting}
(gdb) finish
Run till exit from #0  __printf (format=0x400628 "a=%d; b=%d; c=%d; d=%d; e=%d; f=%d; g=%d; h=%d\n") at printf.c:29
a=1; b=2; c=3; d=4; e=5; f=6; g=7; h=8
main () at 2.c:6
6		return 0;
Value returned is $1 = 39
(gdb) next
7	};
(gdb) info registers
rax            0x0	0
rbx            0x0	0
rcx            0x26	38
rdx            0x7ffff7dd59f0	140737351866864
rsi            0x7fffffd9	2147483609
rdi            0x0	0
rbp            0x7fffffffdf60	0x7fffffffdf60
rsp            0x7fffffffdf40	0x7fffffffdf40
r8             0x7ffff7dd26a0	140737351853728
r9             0x7ffff7a60134	140737348239668
r10            0x7fffffffd5b0	140737488344496
r11            0x7ffff7a95900	140737348458752
r12            0x400440	4195392
r13            0x7fffffffe040	140737488347200
r14            0x0	0
r15            0x0	0
rip            0x40057b	0x40057b <main+78>
...
\end{lstlisting}
}
\RU{\subsubsection{x64: 8 аргументов}

\myindex{x86-64}
\label{example_printf8_x64}
Для того чтобы посмотреть, как остальные аргументы будут передаваться через стек, 
изменим пример ещё раз, 
увеличив количество передаваемых аргументов до 9 
(строка формата \printf и 8 переменных типа \Tint):

\lstinputlisting[style=customc]{patterns/03_printf/2.c}

\myparagraph{MSVC}

Как уже было сказано ранее, первые 4 аргумента в Win64 передаются в регистрах \RCX, \RDX, \Reg{8}, \Reg{9}, а остальные~--- через стек.
Здесь мы это и видим.
Впрочем, инструкция \PUSH не используется, вместо неё при помощи \MOV значения сразу записываются в стек.

\lstinputlisting[caption=MSVC 2012 x64,style=customasmx86]{patterns/03_printf/x86/2_MSVC_x64_RU.asm}

Наблюдательный читатель может спросить, почему для значений типа \Tint отводится 8 байт, ведь нужно только 4?
Да, это нужно запомнить: для значений всех типов более коротких чем 64-бита, отводится 8 байт.
Это сделано для удобства: так всегда легко рассчитать адрес того или иного аргумента.
К тому же, все они расположены по выровненным адресам в памяти.
В 32-битных средах точно также: для всех типов резервируется 4 байта в стеке.

% also for local variables?

\myparagraph{GCC}

В *NIX-системах для x86-64 ситуация похожая, вот только первые 6 аргументов передаются через
\RDI, \RSI, \RDX, \RCX, \Reg{8}, \Reg{9}.
Остальные~--- через стек.
GCC генерирует код, записывающий указатель на строку в \EDI вместо \RDI~--- 
это мы уже рассмотрели чуть раньше: \myref{hw_EDI_instead_of_RDI}.

Почему перед вызовом \printf очищается регистр \EAX мы уже рассмотрели ранее \myref{SysVABI_input_EAX}.

\lstinputlisting[caption=\Optimizing GCC 4.4.6 x64,style=customasmx86]{patterns/03_printf/x86/2_GCC_x64_RU.s}

\myparagraph{GCC + GDB}
\myindex{GDB}

Попробуем этот пример в \ac{GDB}.

\begin{lstlisting}
$ gcc -g 2.c -o 2
\end{lstlisting}

\begin{lstlisting}
$ gdb 2
GNU gdb (GDB) 7.6.1-ubuntu
...
Reading symbols from /home/dennis/polygon/2...done.
\end{lstlisting}

\begin{lstlisting}[caption=ставим точку останова на \printf{,} запускаем]
(gdb) b printf
Breakpoint 1 at 0x400410
(gdb) run
Starting program: /home/dennis/polygon/2 

Breakpoint 1, __printf (format=0x400628 "a=%d; b=%d; c=%d; d=%d; e=%d; f=%d; g=%d; h=%d\n") at printf.c:29
29	printf.c: No such file or directory.
\end{lstlisting}

В регистрах \RSI/\RDX/\RCX/\Reg{8}/\Reg{9} 
всё предсказуемо.
А \RIP содержит адрес самой первой инструкции функции \printf{}.

\begin{lstlisting}
(gdb) info registers
rax            0x0	0
rbx            0x0	0
rcx            0x3	3
rdx            0x2	2
rsi            0x1	1
rdi            0x400628	4195880
rbp            0x7fffffffdf60	0x7fffffffdf60
rsp            0x7fffffffdf38	0x7fffffffdf38
r8             0x4	4
r9             0x5	5
r10            0x7fffffffdce0	140737488346336
r11            0x7ffff7a65f60	140737348263776
r12            0x400440	4195392
r13            0x7fffffffe040	140737488347200
r14            0x0	0
r15            0x0	0
rip            0x7ffff7a65f60	0x7ffff7a65f60 <__printf>
...
\end{lstlisting}

\begin{lstlisting}[caption=смотрим на строку формата]
(gdb) x/s $rdi
0x400628:	"a=%d; b=%d; c=%d; d=%d; e=%d; f=%d; g=%d; h=%d\n"
\end{lstlisting}

Дампим стек на этот раз с командой x/g --- \IT{g} означает \IT{giant words}, т.е. 64-битные слова.

\begin{lstlisting}
(gdb) x/10g $rsp
0x7fffffffdf38:	0x0000000000400576	0x0000000000000006
0x7fffffffdf48:	0x0000000000000007	0x00007fff00000008
0x7fffffffdf58:	0x0000000000000000	0x0000000000000000
0x7fffffffdf68:	0x00007ffff7a33de5	0x0000000000000000
0x7fffffffdf78:	0x00007fffffffe048	0x0000000100000000
\end{lstlisting}

Самый первый элемент стека, как и в прошлый раз, это \ac{RA}.
Через стек также передаются 3 значения: 6, 7, 8.
Видно, что 8 передается с неочищенной старшей 32-битной частью: \GTT{0x00007fff00000008}.
Это нормально, ведь передаются числа типа \Tint, а они 32-битные.
Так что в старшей части регистра или памяти стека остался \q{случайный мусор}.

\ac{GDB} показывает всю функцию \main, если попытаться посмотреть, куда вернется управление после исполнения \printf{}.

\begin{lstlisting}[style=customasmx86]
(gdb) set disassembly-flavor intel
(gdb) disas 0x0000000000400576
Dump of assembler code for function main:
   0x000000000040052d <+0>:	push   rbp
   0x000000000040052e <+1>:	mov    rbp,rsp
   0x0000000000400531 <+4>:	sub    rsp,0x20
   0x0000000000400535 <+8>:	mov    DWORD PTR [rsp+0x10],0x8
   0x000000000040053d <+16>:	mov    DWORD PTR [rsp+0x8],0x7
   0x0000000000400545 <+24>:	mov    DWORD PTR [rsp],0x6
   0x000000000040054c <+31>:	mov    r9d,0x5
   0x0000000000400552 <+37>:	mov    r8d,0x4
   0x0000000000400558 <+43>:	mov    ecx,0x3
   0x000000000040055d <+48>:	mov    edx,0x2
   0x0000000000400562 <+53>:	mov    esi,0x1
   0x0000000000400567 <+58>:	mov    edi,0x400628
   0x000000000040056c <+63>:	mov    eax,0x0
   0x0000000000400571 <+68>:	call   0x400410 <printf@plt>
   0x0000000000400576 <+73>:	mov    eax,0x0
   0x000000000040057b <+78>:	leave  
   0x000000000040057c <+79>:	ret    
End of assembler dump.
\end{lstlisting}

Заканчиваем исполнение \printf, исполняем инструкцию обнуляющую \EAX, удостоверяемся что в регистре \EAX именно ноль.
\RIP указывает сейчас на инструкцию \INS{LEAVE}, т.е. предпоследнюю в функции \main{}.

\begin{lstlisting}
(gdb) finish
Run till exit from #0  __printf (format=0x400628 "a=%d; b=%d; c=%d; d=%d; e=%d; f=%d; g=%d; h=%d\n") at printf.c:29
a=1; b=2; c=3; d=4; e=5; f=6; g=7; h=8
main () at 2.c:6
6		return 0;
Value returned is $1 = 39
(gdb) next
7	};
(gdb) info registers
rax            0x0	0
rbx            0x0	0
rcx            0x26	38
rdx            0x7ffff7dd59f0	140737351866864
rsi            0x7fffffd9	2147483609
rdi            0x0	0
rbp            0x7fffffffdf60	0x7fffffffdf60
rsp            0x7fffffffdf40	0x7fffffffdf40
r8             0x7ffff7dd26a0	140737351853728
r9             0x7ffff7a60134	140737348239668
r10            0x7fffffffd5b0	140737488344496
r11            0x7ffff7a95900	140737348458752
r12            0x400440	4195392
r13            0x7fffffffe040	140737488347200
r14            0x0	0
r15            0x0	0
rip            0x40057b	0x40057b <main+78>
...
\end{lstlisting}
}
\PTBR{\subsubsection{x64: 8 argumentos}

\myindex{x86-64}
\label{example_printf8_x64}
Para ver como outros argumentos são passados pela pilha,
vamos mudar nosso exemplo novamente aumentando o numero de argumentos para 9 (\printf + 8 variáveis \Tint):

\lstinputlisting[style=customc]{patterns/03_printf/2.c}

\myparagraph{MSVC}

Como mencionado anteriormente, os primeiros 4 argumentos tem de ser passados pelos registradores \RCX, \RDX, \Reg{8}, \Reg{9} no Win64, enquanto o resto pela pilha.
Isso é exatamente o que veremos aqui.
Entretanto, a instrução \MOV, ao invés de \PUSH, é usada para preparar a pilha, portanto os valores são armazenados de uma maneira direta.

% TODO translate
\lstinputlisting[caption=MSVC 2012 x64,style=customasmx86]{patterns/03_printf/x86/2_MSVC_x64_EN.asm}

Um leitor observativo pode se indagar por que são alocados 8 bytes para valores int quando 4 já é suficiente?
Sim, mas lembre-se: 8 bytes são alocados para qualquer tipo de informação menor do que 64 bits.
Isso é estabelecido com o objetivo de ser conveniente: é mais facil calcular o endereço de um argumento arbitrário.
Além do mais, eles são alocados em endereços de memórias alinhados. Da mesma maneira no 32-bits: 4 bytes são reservados para todos os tipos de informação.

% also for local variables?

\PTBRph{}

}
\ITA{\subsubsection{x64: 8 argomenti}

\myindex{x86-64}
\label{example_printf8_x64}
Per vedere come altri argomenti sono passati tramite lo stack, cambiamo nuovamente l'esempio per aumentare il numero degli argomenti a 9 (format string di \printf + 8 variabili \Tint):

\lstinputlisting[style=customc]{patterns/03_printf/2.c}

\myparagraph{MSVC}

Come gia' detto in precedenza, i primi 4 argomenti devono essere passati tramite i registri \RCX, \RDX, \Reg{8}, \Reg{9} in Win64,
mentre tutto il resto ---tramite lo stack.
E' esattamente quello che vediamo qui.
Tuttavia, l'istruzione \MOV instruction e' usata al posto di \PUSH per preparare lo stack, in modo tale che i valori siano memorizzati
nello stack in maniera diretta.

% TODO translate to Italian
\lstinputlisting[caption=MSVC 2012 x64,style=customasmx86]{patterns/03_printf/x86/2_MSVC_x64_EN.asm}

Il lettore attento potrebbe chiedere perche' per i valori \Tint sono allocati 8 byte quando ne bastano 4?
Bisogna ricordare che per ogni tipo di dato piu' piccolo di 64 bit, sono allocati 8 byte.
Questo e' stabilito per convenienza: rende piu' facile calcolare l'indirizzo di argomenti arbitrari. E inoltre fa si che tutti
siano allocati ad indirizzi di memoria allineati.
Succede lo stesso in ambienti a 32-bit environments: sono riservati 4 byte per ogni tipo di dato.

% also for local variables?

\myparagraph{GCC}

L'immagine e' simile per i sistemi operativi x86-64 e *NIX, con l'eccezzione che i primi 6 argomenti sono passati attraverso
i registri \RDI, \RSI, \RDX, \RCX, \Reg{8}, \Reg{9} registers.
Tutto il resto ---tramite lo stack.
GCC genera il codice memorizzando il puntatore alla stringa in \EDI invece che \RDI{}---lo abbiamo visto in precedenza: 
\myref{hw_EDI_instead_of_RDI}.

Abbiamo anche notato prima che il registro \EAX e' stato azzerato prima di una chiamata a \printf: \myref{SysVABI_input_EAX}.

% TODO translate to Italian
\lstinputlisting[caption=\Optimizing GCC 4.4.6 x64,style=customasmx86]{patterns/03_printf/x86/2_GCC_x64_EN.s}

\myparagraph{GCC + GDB}
\myindex{GDB}

Proviamo l'esempio in \ac{GDB}.

\begin{lstlisting}
$ gcc -g 2.c -o 2
\end{lstlisting}

\begin{lstlisting}
$ gdb 2
GNU gdb (GDB) 7.6.1-ubuntu
...
Reading symbols from /home/dennis/polygon/2...done.
\end{lstlisting}

\begin{lstlisting}[caption=impostiamo il breakpoint su \printf{,} e avviamo]
(gdb) b printf
Breakpoint 1 at 0x400410
(gdb) run
Starting program: /home/dennis/polygon/2 

Breakpoint 1, __printf (format=0x400628 "a=%d; b=%d; c=%d; d=%d; e=%d; f=%d; g=%d; h=%d\n") at printf.c:29
29	printf.c: No such file or directory.
\end{lstlisting}

I registri \RSI/\RDX/\RCX/\Reg{8}/\Reg{9} hanno i valori previsti.
\RIP ha l'indirizzo della prima istruzione della funzione \printf.

\begin{lstlisting}
(gdb) info registers
rax            0x0	0
rbx            0x0	0
rcx            0x3	3
rdx            0x2	2
rsi            0x1	1
rdi            0x400628	4195880
rbp            0x7fffffffdf60	0x7fffffffdf60
rsp            0x7fffffffdf38	0x7fffffffdf38
r8             0x4	4
r9             0x5	5
r10            0x7fffffffdce0	140737488346336
r11            0x7ffff7a65f60	140737348263776
r12            0x400440	4195392
r13            0x7fffffffe040	140737488347200
r14            0x0	0
r15            0x0	0
rip            0x7ffff7a65f60	0x7ffff7a65f60 <__printf>
...
\end{lstlisting}

\begin{lstlisting}[caption=ispezioniamo la format string]
(gdb) x/s $rdi
0x400628:	"a=%d; b=%d; c=%d; d=%d; e=%d; f=%d; g=%d; h=%d\n"
\end{lstlisting}

Effettuiamo un dump dello stack, questa volta con il comando x/g ---\IT{g} sta per \IT{giant words}, ovvero 64-bit words.

\begin{lstlisting}
(gdb) x/10g $rsp
0x7fffffffdf38:	0x0000000000400576	0x0000000000000006
0x7fffffffdf48:	0x0000000000000007	0x00007fff00000008
0x7fffffffdf58:	0x0000000000000000	0x0000000000000000
0x7fffffffdf68:	0x00007ffff7a33de5	0x0000000000000000
0x7fffffffdf78:	0x00007fffffffe048	0x0000000100000000
\end{lstlisting}

Il primo elemento dello stack, proprio come nel caso precedente, e' il {RA}.
3 valori vengono passatri trvamite lo stack : 6, 7, 8.
Notiamo anche che 8 e' passato nei 32-bit alti non azzerati: \GTT{0x00007fff00000008}.
Cio' va bene, perche' i valori hanno tipo \Tint , che e' a 32-bit.
Quindi, i registri alti o gli elementi alti dello stack potrebbero contenere \q{random garbage}.

Se andiamo a vedere dove viene restituito il controllodopo l'esecuzione di \printf, 
\ac{GDB} mostrera' l'intera funzione \main:

\begin{lstlisting}[style=customasmx86]
(gdb) set disassembly-flavor intel
(gdb) disas 0x0000000000400576
Dump of assembler code for function main:
   0x000000000040052d <+0>:	push   rbp
   0x000000000040052e <+1>:	mov    rbp,rsp
   0x0000000000400531 <+4>:	sub    rsp,0x20
   0x0000000000400535 <+8>:	mov    DWORD PTR [rsp+0x10],0x8
   0x000000000040053d <+16>:	mov    DWORD PTR [rsp+0x8],0x7
   0x0000000000400545 <+24>:	mov    DWORD PTR [rsp],0x6
   0x000000000040054c <+31>:	mov    r9d,0x5
   0x0000000000400552 <+37>:	mov    r8d,0x4
   0x0000000000400558 <+43>:	mov    ecx,0x3
   0x000000000040055d <+48>:	mov    edx,0x2
   0x0000000000400562 <+53>:	mov    esi,0x1
   0x0000000000400567 <+58>:	mov    edi,0x400628
   0x000000000040056c <+63>:	mov    eax,0x0
   0x0000000000400571 <+68>:	call   0x400410 <printf@plt>
   0x0000000000400576 <+73>:	mov    eax,0x0
   0x000000000040057b <+78>:	leave  
   0x000000000040057c <+79>:	ret    
End of assembler dump.
\end{lstlisting}

Finiamo di eseguire \printf, eseguiamo l'istruzione che azzera \EAX, e notiamo che il registro \EAX register ha esattamente il valore zero.
\RIP adesso punta all'istruzione \INS{LEAVE} , ovvero la penultima nella funzione \main.

\begin{lstlisting}
(gdb) finish
Run till exit from #0  __printf (format=0x400628 "a=%d; b=%d; c=%d; d=%d; e=%d; f=%d; g=%d; h=%d\n") at printf.c:29
a=1; b=2; c=3; d=4; e=5; f=6; g=7; h=8
main () at 2.c:6
6		return 0;
Value returned is $1 = 39
(gdb) next
7	};
(gdb) info registers
rax            0x0	0
rbx            0x0	0
rcx            0x26	38
rdx            0x7ffff7dd59f0	140737351866864
rsi            0x7fffffd9	2147483609
rdi            0x0	0
rbp            0x7fffffffdf60	0x7fffffffdf60
rsp            0x7fffffffdf40	0x7fffffffdf40
r8             0x7ffff7dd26a0	140737351853728
r9             0x7ffff7a60134	140737348239668
r10            0x7fffffffd5b0	140737488344496
r11            0x7ffff7a95900	140737348458752
r12            0x400440	4195392
r13            0x7fffffffe040	140737488347200
r14            0x0	0
r15            0x0	0
rip            0x40057b	0x40057b <main+78>
...
\end{lstlisting}
}
\FR{\subsubsection{x64: 8 arguments}

\myindex{x86-64}
\label{example_printf8_x64}
Pour voir comment les autres arguments sont passés par la pile, changeons encore
notre exemple en incrémentant le nombre d'argument à 9 (chaîne de format de
\printf + 8 variables \Tint):

\lstinputlisting[style=customc]{patterns/03_printf/2.c}

\myparagraph{MSVC}

Comme il a déjà été mentionné, les 4 premiers arguments sont passés par les registres
\RCX, \RDX, \Reg{8}, \Reg{9} sous Win64, tandis les autres le sont---par la pile.
C'est exactement de que l'on voit ici.
Toutefois, l'instruction \MOV est utilisée ici à la place de \PUSH, donc les valeurs
sont stockées sur la pile d'une manière simple.

\lstinputlisting[caption=MSVC 2012 x64,style=customasmx86]{patterns/03_printf/x86/2_MSVC_x64_FR.asm}

Le lecteur observateur pourrait demander pourquoi 8 octets sont alloués sur la
pile pour les valeurs \Tint, alors que 4 suffisent?
Oui, il faut se rappeler: 8 octets sont alloués pour tout type de données plus
petit que 64 bits.
Ceci est instauré pour des raisons de commodités: cela rend facile le calcul
de l'adresse de n'importe quel argument.
En outre, ils sont tous situés à des adresses mémoires alignées.
Il en est de même dans les environnements 32-bit: 4 octets sont réservés pour tout
types de données.

% also for local variables?

\myparagraph{GCC}

Le tableau est similaire pour les OS x86-64 *NIX, excepté que les 6 premiers arguments
sont passés par les registres \RDI, \RSI, \RDX, \RCX, \Reg{8}, \Reg{9}.
Tout les autres---par la pile.
GCC génère du code stockant le pointeur de chaîne dans \EDI au lieu de \RDI{}---nous
l'avons noté précédemment:
\myref{hw_EDI_instead_of_RDI}.

Nous avions également noté que le registre \EAX a été vidé avant l'appel à
\printf: \myref{SysVABI_input_EAX}.

\lstinputlisting[caption=GCC 4.4.6 x64 \Optimizing,style=customasmx86]{patterns/03_printf/x86/2_GCC_x64_FR.s}

\myparagraph{GCC + GDB}
\myindex{GDB}

Essayons cet exemple dans \ac{GDB}.

\begin{lstlisting}
$ gcc -g 2.c -o 2
\end{lstlisting}

\begin{lstlisting}
$ gdb 2
GNU gdb (GDB) 7.6.1-ubuntu
...
Reading symbols from /home/dennis/polygon/2...done.
\end{lstlisting}

\begin{lstlisting}[caption=mettons le point d'arrêt à \printf{,} et lançons]
(gdb) b printf
Breakpoint 1 at 0x400410
(gdb) run
Starting program: /home/dennis/polygon/2 

Breakpoint 1, __printf (format=0x400628 "a=%d; b=%d; c=%d; d=%d; e=%d; f=%d; g=%d; h=%d\n") at printf.c:29
29	printf.c: No such file or directory.
\end{lstlisting}

Les registres \RSI/\RDX/\RCX/\Reg{8}/\Reg{9} ont les valeurs attendues.
\RIP contient l'adresse de la toute première instruction de la fonction \printf.

\begin{lstlisting}
(gdb) info registers
rax            0x0	0
rbx            0x0	0
rcx            0x3	3
rdx            0x2	2
rsi            0x1	1
rdi            0x400628	4195880
rbp            0x7fffffffdf60	0x7fffffffdf60
rsp            0x7fffffffdf38	0x7fffffffdf38
r8             0x4	4
r9             0x5	5
r10            0x7fffffffdce0	140737488346336
r11            0x7ffff7a65f60	140737348263776
r12            0x400440	4195392
r13            0x7fffffffe040	140737488347200
r14            0x0	0
r15            0x0	0
rip            0x7ffff7a65f60	0x7ffff7a65f60 <__printf>
...
\end{lstlisting}

\begin{lstlisting}[caption=inspectons la chaîne de format]
(gdb) x/s $rdi
0x400628:	"a=%d; b=%d; c=%d; d=%d; e=%d; f=%d; g=%d; h=%d\n"
\end{lstlisting}

Affichons la pile avec la commande x/g cette fois---\IT{g} est l'unité pour \IT{giant words}, i.e., mots de 64-bit.

\begin{lstlisting}
(gdb) x/10g $rsp
0x7fffffffdf38:	0x0000000000400576	0x0000000000000006
0x7fffffffdf48:	0x0000000000000007	0x00007fff00000008
0x7fffffffdf58:	0x0000000000000000	0x0000000000000000
0x7fffffffdf68:	0x00007ffff7a33de5	0x0000000000000000
0x7fffffffdf78:	0x00007fffffffe048	0x0000000100000000
\end{lstlisting}

Le tout premier élément de la pile, comme dans le cas précédent, est \ac{RA}.
3 valeurs sont aussi passées par la pile: 6, 7, 8.
Nous voyons également que 8 est passé avec les 32-bits de poids fort non
éffacés: \GTT{0x00007fff00000008}.
C'est en ordre, car les valeurs sont d'un type \Tint, qui est 32-bit.
Donc, la partie haute du registre ou l'élément de la pile peuvent contenir des
\q{restes de données aléatoires}.

%%If you take a look at where the control will return after the \printf execution,
%%\ac{GDB} will show the entire \main function:
Si vous regardez où le contrôle reviendra après l'exécution de \printf,
\ac{GDB} affiche la fonction \main en entier:

\begin{lstlisting}[style=customasmx86]
(gdb) set disassembly-flavor intel
(gdb) disas 0x0000000000400576
Dump of assembler code for function main:
   0x000000000040052d <+0>:	push   rbp
   0x000000000040052e <+1>:	mov    rbp,rsp
   0x0000000000400531 <+4>:	sub    rsp,0x20
   0x0000000000400535 <+8>:	mov    DWORD PTR [rsp+0x10],0x8
   0x000000000040053d <+16>:	mov    DWORD PTR [rsp+0x8],0x7
   0x0000000000400545 <+24>:	mov    DWORD PTR [rsp],0x6
   0x000000000040054c <+31>:	mov    r9d,0x5
   0x0000000000400552 <+37>:	mov    r8d,0x4
   0x0000000000400558 <+43>:	mov    ecx,0x3
   0x000000000040055d <+48>:	mov    edx,0x2
   0x0000000000400562 <+53>:	mov    esi,0x1
   0x0000000000400567 <+58>:	mov    edi,0x400628
   0x000000000040056c <+63>:	mov    eax,0x0
   0x0000000000400571 <+68>:	call   0x400410 <printf@plt>
   0x0000000000400576 <+73>:	mov    eax,0x0
   0x000000000040057b <+78>:	leave  
   0x000000000040057c <+79>:	ret    
End of assembler dump.
\end{lstlisting}

Laissons se terminer l'exécution de \printf, exécutez l'instruction mettant \EAX
à zéro, et notez que le registre \EAX à une valeur d'exactement zéro.
\RIP pointe maintenant sur l'instruction \INS{LEAVE}, i.e, la pénultième de la
fonction \main.

\begin{lstlisting}
(gdb) finish
Run till exit from #0  __printf (format=0x400628 "a=%d; b=%d; c=%d; d=%d; e=%d; f=%d; g=%d; h=%d\n") at printf.c:29
a=1; b=2; c=3; d=4; e=5; f=6; g=7; h=8
main () at 2.c:6
6		return 0;
Value returned is $1 = 39
(gdb) next
7	};
(gdb) info registers
rax            0x0	0
rbx            0x0	0
rcx            0x26	38
rdx            0x7ffff7dd59f0	140737351866864
rsi            0x7fffffd9	2147483609
rdi            0x0	0
rbp            0x7fffffffdf60	0x7fffffffdf60
rsp            0x7fffffffdf40	0x7fffffffdf40
r8             0x7ffff7dd26a0	140737351853728
r9             0x7ffff7a60134	140737348239668
r10            0x7fffffffd5b0	140737488344496
r11            0x7ffff7a95900	140737348458752
r12            0x400440	4195392
r13            0x7fffffffe040	140737488347200
r14            0x0	0
r15            0x0	0
rip            0x40057b	0x40057b <main+78>
...
\end{lstlisting}
}



\subsection{ARM}

\EN{\subsubsection{ARM: 3 arguments}

ARM's traditional scheme for passing arguments (calling convention) behaves as follows:
the first 4 arguments are passed through the \Reg{0}-\Reg{3} registers; the remaining arguments via the stack.
This resembles the arguments passing scheme in 
fastcall~(\myref{fastcall}) or win64~(\myref{sec:callingconventions_win64}).

\myparagraph{32-bit ARM}

\mysubparagraph{\NonOptimizingKeilVI (\ARMMode)}

\begin{lstlisting}[caption=\NonOptimizingKeilVI (\ARMMode),style=customasmARM]
.text:00000000 main
.text:00000000 10 40 2D E9   STMFD   SP!, {R4,LR}
.text:00000004 03 30 A0 E3   MOV     R3, #3
.text:00000008 02 20 A0 E3   MOV     R2, #2
.text:0000000C 01 10 A0 E3   MOV     R1, #1
.text:00000010 08 00 8F E2   ADR     R0, aADBDCD     ; "a=%d; b=%d; c=%d"
.text:00000014 06 00 00 EB   BL      __2printf
.text:00000018 00 00 A0 E3   MOV     R0, #0          ; return 0
.text:0000001C 10 80 BD E8   LDMFD   SP!, {R4,PC}
\end{lstlisting}

So, the first 4 arguments are passed via the \Reg{0}-\Reg{3} registers in this order:
a pointer to the \printf format string in 
\Reg{0}, then 1 in \Reg{1}, 2 in \Reg{2} and 3 in \Reg{3}.
The instruction at \GTT{0x18} writes 0 to \Reg{0}---this is \IT{return 0} C-statement.
There is nothing unusual so far.

\OptimizingKeilVI generates the same code.

\mysubparagraph{\OptimizingKeilVI (\ThumbMode)}

\begin{lstlisting}[caption=\OptimizingKeilVI (\ThumbMode),style=customasmARM]
.text:00000000 main
.text:00000000 10 B5        PUSH    {R4,LR}
.text:00000002 03 23        MOVS    R3, #3
.text:00000004 02 22        MOVS    R2, #2
.text:00000006 01 21        MOVS    R1, #1
.text:00000008 02 A0        ADR     R0, aADBDCD     ; "a=%d; b=%d; c=%d"
.text:0000000A 00 F0 0D F8  BL      __2printf
.text:0000000E 00 20        MOVS    R0, #0
.text:00000010 10 BD        POP     {R4,PC}
\end{lstlisting}

There is no significant difference from the non-optimized code for ARM mode.

\mysubparagraph{\OptimizingKeilVI (\ARMMode) + let's remove return}
\label{ARM_B_to_printf}

Let's rework example slightly by removing \IT{return 0}:

\begin{lstlisting}[style=customc]
#include <stdio.h>

void main()
{
	printf("a=%d; b=%d; c=%d", 1, 2, 3);
};
\end{lstlisting}

The result is somewhat unusual:

\begin{lstlisting}[caption=\OptimizingKeilVI (\ARMMode),style=customasmARM]
.text:00000014 main
.text:00000014 03 30 A0 E3   MOV     R3, #3
.text:00000018 02 20 A0 E3   MOV     R2, #2
.text:0000001C 01 10 A0 E3   MOV     R1, #1
.text:00000020 1E 0E 8F E2   ADR     R0, aADBDCD     ; "a=%d; b=%d; c=%d\n"
.text:00000024 CB 18 00 EA   B       __2printf
\end{lstlisting}

\myindex{ARM!\Registers!Link Register}
\myindex{ARM!\Instructions!B}
\myindex{Function epilogue}
This is the optimized (\Othree) version for ARM mode and this time we see \INS{B} as the last instruction instead of the familiar \INS{BL}.
Another difference between this optimized version and the previous one (compiled without optimization)
is the lack of function prologue and epilogue (instructions preserving the \Reg{0} and \ac{LR} registers values).
\myindex{x86!\Instructions!JMP}
The \INS{B} instruction just jumps to another address, without any manipulation of the \ac{LR} register,
similar to \JMP in x86.
Why does it work? Because this code is, in fact, effectively equivalent to the previous.
There are two main reasons: 1) neither the stack nor \ac{SP} (the \gls{stack pointer}) is modified;
2) the call to \printf is the last instruction, so there is nothing going on afterwards.
On completion, the \printf function simply returns the control to the address 
stored in \ac{LR}.
Since the \ac{LR} currently stores the address of the point from where our function
has been called then the control from \printf will be returned to that point.
Therefore we do not have to save \ac{LR} because we do not have necessity to modify \ac{LR}.
And we do not have necessity to modify \ac{LR} because there are no other function calls except \printf. Furthermore,
after this call we do not to do anything else!
That is the reason such optimization is possible.

This optimization is often used in functions where the last statement is a call to another function.
A similar example is presented here:
\myref{jump_to_last_printf}.

\myparagraph{ARM64}

\mysubparagraph{\NonOptimizing GCC (Linaro) 4.9}

\lstinputlisting[caption=\NonOptimizing GCC (Linaro) 4.9,style=customasmARM]{patterns/03_printf/ARM/ARM3_O0_EN.lst}

\myindex{ARM!\Instructions!STP}

The first instruction \INS{STP} (\IT{Store Pair}) saves \ac{FP} (X29) and \ac{LR} (X30) in the stack.
The second \INS{ADD X29, SP, 0} instruction forms the stack frame.
It is just writing the value of \ac{SP} into X29.

\myindex{ARM!\Instructions!ADRP/ADD pair}
Next, we see the familiar \INS{ADRP}/\ADD instruction pair, which forms a pointer to the string.
\myindex{ARM64!lo12}
\IT{lo12} meaning low 12 bits, i.e., linker will write low 12 bits of LC1 address into the opcode of \ADD instruction.
\GTT{\%d} in \printf string format is a 32-bit \Tint, so the 1, 2 and 3 are loaded into 32-bit register parts.

\Optimizing GCC (Linaro) 4.9 generates the same code.

}
\RU{\subsubsection{ARM: 3 аргумента}

В ARM традиционно принята такая схема передачи аргументов в функцию: 
4 первых аргумента через регистры \Reg{0}-\Reg{3}; а остальные~--- через стек.
Это немного похоже на то, как аргументы передаются в 
fastcall~(\myref{fastcall}) или win64~(\myref{sec:callingconventions_win64}).

\myparagraph{32-битный ARM}

\mysubparagraph{\NonOptimizingKeilVI (\ARMMode)}

\begin{lstlisting}[caption=\NonOptimizingKeilVI (\ARMMode),style=customasmARM]
.text:00000000 main
.text:00000000 10 40 2D E9   STMFD   SP!, {R4,LR}
.text:00000004 03 30 A0 E3   MOV     R3, #3
.text:00000008 02 20 A0 E3   MOV     R2, #2
.text:0000000C 01 10 A0 E3   MOV     R1, #1
.text:00000010 08 00 8F E2   ADR     R0, aADBDCD     ; "a=%d; b=%d; c=%d"
.text:00000014 06 00 00 EB   BL      __2printf
.text:00000018 00 00 A0 E3   MOV     R0, #0          ; return 0
.text:0000001C 10 80 BD E8   LDMFD   SP!, {R4,PC}
\end{lstlisting}

Итак, первые 4 аргумента передаются через регистры \Reg{0}-\Reg{3}, по порядку: 
указатель на формат-строку для \printf
в \Reg{0}, затем 1 в \Reg{1}, 2 в \Reg{2} и 3 в \Reg{3}.

Инструкция на \GTT{0x18} записывает 0 в \Reg{0} --- это выражение в Си \IT{return 0}.
Пока что здесь нет ничего необычного.
\OptimizingKeilVI генерирует точно такой же код.

\mysubparagraph{\OptimizingKeilVI (\ThumbMode)}

\begin{lstlisting}[caption=\OptimizingKeilVI (\ThumbMode),style=customasmARM]
.text:00000000 main
.text:00000000 10 B5        PUSH    {R4,LR}
.text:00000002 03 23        MOVS    R3, #3
.text:00000004 02 22        MOVS    R2, #2
.text:00000006 01 21        MOVS    R1, #1
.text:00000008 02 A0        ADR     R0, aADBDCD     ; "a=%d; b=%d; c=%d"
.text:0000000A 00 F0 0D F8  BL      __2printf
.text:0000000E 00 20        MOVS    R0, #0
.text:00000010 10 BD        POP     {R4,PC}
\end{lstlisting}

Здесь нет особых отличий от неоптимизированного варианта для режима ARM.
\mysubparagraph{\OptimizingKeilVI (\ARMMode) + убираем return}
\label{ARM_B_to_printf}

Немного переделаем пример, убрав \IT{return 0}:

\begin{lstlisting}[style=customc]
#include <stdio.h>

void main()
{
	printf("a=%d; b=%d; c=%d", 1, 2, 3);
};
\end{lstlisting}

Результат получится необычным:

\begin{lstlisting}[caption=\OptimizingKeilVI (\ARMMode),style=customasmARM]
.text:00000014 main
.text:00000014 03 30 A0 E3   MOV     R3, #3
.text:00000018 02 20 A0 E3   MOV     R2, #2
.text:0000001C 01 10 A0 E3   MOV     R1, #1
.text:00000020 1E 0E 8F E2   ADR     R0, aADBDCD     ; "a=%d; b=%d; c=%d\n"
.text:00000024 CB 18 00 EA   B       __2printf
\end{lstlisting}

\myindex{ARM!\Registers!Link Register}
\myindex{ARM!\Instructions!B}
\myindex{Function epilogue}
Это оптимизированная версия (\Othree) для режима ARM, и здесь мы видим последнюю инструкцию 
\INS{B} вместо привычной нам \INS{BL}.
Отличия между этой оптимизированной версией и предыдущей, скомпилированной без оптимизации, 
ещё и в том, что здесь нет пролога и эпилога функции (инструкций, сохраняющих состояние регистров \Reg{0} и \ac{LR}).
\myindex{x86!\Instructions!JMP}
Инструкция \INS{B} просто переходит на другой адрес, без манипуляций с регистром \ac{LR}, то есть это аналог \JMP в x86.
Почему это работает нормально? Потому что этот код эквивалентен предыдущему.

Основных причин две: 1) стек не модифицируется, как и \glslink{stack pointer}{указатель стека} \ac{SP}; 2) вызов функции \printf последний, после него ничего не происходит.
Функция \printf, отработав, просто возвращает управление по адресу, записанному в \ac{LR}.

Но в \ac{LR} находится адрес места, откуда была вызвана наша функция!
А следовательно, управление из \printf вернется сразу туда.

Значит нет нужды сохранять \ac{LR}, потому что нет нужны модифицировать \ac{LR}.
А нет нужды модифицировать \ac{LR}, потому что нет иных вызовов функций, кроме \printf, к тому же, после этого вызова не нужно ничего здесь больше делать!
Поэтому такая оптимизация возможна.

Эта оптимизация часто используется в функциях, где последнее выражение~--- это вызов другой функции.

Ещё один похожий пример описан здесь:
\myref{jump_to_last_printf}.

\myparagraph{ARM64}

\mysubparagraph{\NonOptimizing GCC (Linaro) 4.9}

\lstinputlisting[caption=\NonOptimizing GCC (Linaro) 4.9,style=customasmARM]{patterns/03_printf/ARM/ARM3_O0_RU.lst}

\myindex{ARM!\Instructions!STP}
Итак, первая инструкция \INS{STP} (\IT{Store Pair}) сохраняет \ac{FP} (X29) и \ac{LR} (X30) в стеке.
Вторая инструкция \INS{ADD X29, SP, 0} формирует стековый фрейм.
Это просто запись значения \ac{SP} в X29.

\myindex{ARM!\Instructions!ADRP/ADD pair}
Далее уже знакомая пара инструкций \INS{ADRP}/\ADD формирует указатель на строку.

\myindex{ARM64!lo12}
\IT{lo12} означает младшие 12 бит, т.е., линкер запишет младшие 12 бит адреса метки LC1 в опкод инструкции \ADD.
\GTT{\%d} в формате \printf это 32-битный \Tint, так что 1, 2 и 3 заносятся в 32-битные части регистров.
\Optimizing GCC (Linaro) 4.9 генерирует почти такой же код.

}
\FR{\subsubsection{ARM: 3 arguments}

Le schéma ARM traditionnel pour passer des arguments (convention d'appel) se
comporte de cette façon:
les 4 premiers arguments sont passés par les registres \Reg{0}-\Reg{3}; les autres
par la pile.
Cela ressemble au schéma de passage des arguments dans
fastcall~(\myref{fastcall}) ou win64~(\myref{sec:callingconventions_win64}).

\myparagraph{ARM 32-bit}

\mysubparagraph{\NonOptimizingKeilVI (\ARMMode)}

\begin{lstlisting}[caption=\NonOptimizingKeilVI (\ARMMode),style=customasmARM]
.text:00000000 main
.text:00000000 10 40 2D E9   STMFD   SP!, {R4,LR}
.text:00000004 03 30 A0 E3   MOV     R3, #3
.text:00000008 02 20 A0 E3   MOV     R2, #2
.text:0000000C 01 10 A0 E3   MOV     R1, #1
.text:00000010 08 00 8F E2   ADR     R0, aADBDCD     ; "a=%d; b=%d; c=%d"
.text:00000014 06 00 00 EB   BL      __2printf
.text:00000018 00 00 A0 E3   MOV     R0, #0          ; retourner 0
.text:0000001C 10 80 BD E8   LDMFD   SP!, {R4,PC}
\end{lstlisting}

Donc, les 4 premiers arguments sont passés par les registres \Reg{0}-\Reg{3} dans
cet ordre:
un pointeur sur la chaîne de format de \printf dans \Reg{0}, puis 1 dans \Reg{1},
2 dans \Reg{2} et 3 dans \Reg{3}.
L'instruction en \GTT{0x18} écrit 0 dans \Reg{0}---c'est la déclaration C
de \IT{return 0}.

\OptimizingKeilVI génère le même code.

\mysubparagraph{\OptimizingKeilVI (\ThumbMode)}

\begin{lstlisting}[caption=\OptimizingKeilVI (\ThumbMode),style=customasmARM]
.text:00000000 main
.text:00000000 10 B5        PUSH    {R4,LR}
.text:00000002 03 23        MOVS    R3, #3
.text:00000004 02 22        MOVS    R2, #2
.text:00000006 01 21        MOVS    R1, #1
.text:00000008 02 A0        ADR     R0, aADBDCD     ; "a=%d; b=%d; c=%d"
.text:0000000A 00 F0 0D F8  BL      __2printf
.text:0000000E 00 20        MOVS    R0, #0
.text:00000010 10 BD        POP     {R4,PC}
\end{lstlisting}

Il n'y a pas de différence significative avec le code non optimisé pour le mode ARM.

\mysubparagraph{\OptimizingKeilVI (\ARMMode) + supprimons le retour}
\label{ARM_B_to_printf}

Retravaillons légèrement l'exemple en supprimant \IT{return 0}:

\begin{lstlisting}[style=customc]
#include <stdio.h>

void main()
{
	printf("a=%d; b=%d; c=%d", 1, 2, 3);
};
\end{lstlisting}

Le résultat est quelque peu inhabituel:

\begin{lstlisting}[caption=\OptimizingKeilVI (\ARMMode),style=customasmARM]
.text:00000014 main
.text:00000014 03 30 A0 E3   MOV     R3, #3
.text:00000018 02 20 A0 E3   MOV     R2, #2
.text:0000001C 01 10 A0 E3   MOV     R1, #1
.text:00000020 1E 0E 8F E2   ADR     R0, aADBDCD     ; "a=%d; b=%d; c=%d\n"
.text:00000024 CB 18 00 EA   B       __2printf
\end{lstlisting}

\myindex{ARM!\Registers!Link Register}
\myindex{ARM!\Instructions!B}
\myindex{Épilogue de la fonction}
C'est la version optimisée (\Othree) pour le mode ARM et cette fois nous voyons
\INS{B} comme dernière instruction au lieu du \INS{BL} habituel.
Une autre différence entre cette version optimisée et la précédente (compilée
sans optimisation) est l'absence de fonctions prologue et épilogue (les instructions
qui préservent les valeurs des registres \Reg{0} et \ac{LR}).
\myindex{x86!\Instructions!JMP}
L'instruction \INS{B} saute simplement à une autre adresse, sans manipuler le registre
\ac{LR}, de façon similaire au \JMP en x86.
Pourquoi est-ce que fonctionne? Parce ce code est en fait bien équivalent au précédent.
Il y a deux raisons principales: 1) Ni la pile ni \ac{SP} (\glslink{stack pointer}{pointeur de pile})
ne sont modifiés;
2) l'appel à \printf est la dernière instruction, donc il ne se passe rien après.
A la fin, la fonction \printf rend simplement le contrôle à l'adresse stockée
dans \ac{LR}.
Puisqu'\ac{LR} contient actuellement l'adresse du point depuis lequel notre fonction
a été appelée alors le contrôle après \printf sera redonné à ce point.
Par conséquent, nous n'avons pas besoin de sauver \ac{LR} car il ne nous est pas
nécessaire de le modifier.
Et il ne nous est non plus pas nécessaire de modifier \ac{LR} car il n'y a pas d'autre
appel de fonction excepté \printf. Par ailleurs, après cet appel nous ne faisons
rien d'autre!
C'est la raison pour laquelle une telle optimisation est possible.

Cette optimisation est souvent utilisée dans les fonctions où la dernière déclaration
est un appel à une autre fonction.
Un exemple similaire est présenté ici:
\myref{jump_to_last_printf}.

\myparagraph{ARM64}

\mysubparagraph{GCC (Linaro) 4.9 \NonOptimizing}

\lstinputlisting[caption=GCC (Linaro) 4.9 \NonOptimizing,style=customasmARM]{patterns/03_printf/ARM/ARM3_O0_FR.lst}

\myindex{ARM!\Instructions!STP}

La première instruction \INS{STP} (\IT{Store Pair}) sauve \ac{FP} (X29) et \ac{LR} (X30) sur la pile.

La seconde instruction, \INS{ADD X29, SP, 0} crée la pile.
Elle écrit simplement la valeur de \ac{SP} dans X29.

\myindex{ARM!\Instructions!ADRP/ADD pair}
Ensuite nous voyons la paire d'instructions habituelle \INS{ADRP}/\ADD, qui crée
le pointeur sur la chaîne.
\myindex{ARM64!lo12}
\IT{lo12} signifie les 12 bits de poids faible, i.e., le linker va écrire les 12 bits
de poids faible de l'adresse LC1 dans l'opcode de l'instruction \ADD.
\GTT{\%d} dans la chaîne de format de \printf est un \Tint 32-bit, les 1, 2 et
3 sont chargés dans les parties 32-bit des registres.

GCC (Linaro) 4.9 \Optimizing génère le même code.

}
\ITA{\subsubsection{ARM: 3 arguments}

Lo schema tradizionale per il passaggio di argomenti (calling convention) di ARM si comporta in questo modo:
i primi 4 argomenti vengono passati attraverso i registri \Reg{0}-\Reg{3} , i restanti attraverso lo stack.
Cio' ricorda molto il metodo per il passaggio di argomenti in 
fastcall~(\myref{fastcall}) o win64~(\myref{sec:callingconventions_win64}).

\myparagraph{32-bit ARM}

\mysubparagraph{\NonOptimizingKeilVI (\ARMMode)}

\begin{lstlisting}[caption=\NonOptimizingKeilVI (\ARMMode),style=customasmARM]
.text:00000000 main
.text:00000000 10 40 2D E9   STMFD   SP!, {R4,LR}
.text:00000004 03 30 A0 E3   MOV     R3, #3
.text:00000008 02 20 A0 E3   MOV     R2, #2
.text:0000000C 01 10 A0 E3   MOV     R1, #1
.text:00000010 08 00 8F E2   ADR     R0, aADBDCD     ; "a=%d; b=%d; c=%d"
.text:00000014 06 00 00 EB   BL      __2printf
.text:00000018 00 00 A0 E3   MOV     R0, #0          ; return 0
.text:0000001C 10 80 BD E8   LDMFD   SP!, {R4,PC}
\end{lstlisting}

I primi 4 argomenti sono quindi passati attraverso i registri \Reg{0}-\Reg{3} nel seguente ordine:
un puntatore alla format string di \printf in 
\Reg{0}, 1 in \Reg{1}, 2 in \Reg{2} e 3 in \Reg{3}.
L'istruzione a \GTT{0x18} scrive 0 in \Reg{0}---questo equivale allo statement C \IT{return 0}.
Niente di insolito fino a qui.

\OptimizingKeilVI genera lo stesso codice.

\mysubparagraph{\OptimizingKeilVI (\ThumbMode)}

\begin{lstlisting}[caption=\OptimizingKeilVI (\ThumbMode),style=customasmARM]
.text:00000000 main
.text:00000000 10 B5        PUSH    {R4,LR}
.text:00000002 03 23        MOVS    R3, #3
.text:00000004 02 22        MOVS    R2, #2
.text:00000006 01 21        MOVS    R1, #1
.text:00000008 02 A0        ADR     R0, aADBDCD     ; "a=%d; b=%d; c=%d"
.text:0000000A 00 F0 0D F8  BL      __2printf
.text:0000000E 00 20        MOVS    R0, #0
.text:00000010 10 BD        POP     {R4,PC}
\end{lstlisting}

Non c'e' nessuna differenza significativa nel codice non ottimizzato per modo ARM.

\mysubparagraph{\OptimizingKeilVI (\ARMMode) + rimozione di return}
\label{ARM_B_to_printf}

Modifichiamo leggermente l'esempio rimuovendo \IT{return 0}:

\begin{lstlisting}[style=customc]
#include <stdio.h>

void main()
{
	printf("a=%d; b=%d; c=%d", 1, 2, 3);
};
\end{lstlisting}

Il risultato e' alquanto insolito:

\begin{lstlisting}[caption=\OptimizingKeilVI (\ARMMode),style=customasmARM]
.text:00000014 main
.text:00000014 03 30 A0 E3   MOV     R3, #3
.text:00000018 02 20 A0 E3   MOV     R2, #2
.text:0000001C 01 10 A0 E3   MOV     R1, #1
.text:00000020 1E 0E 8F E2   ADR     R0, aADBDCD     ; "a=%d; b=%d; c=%d\n"
.text:00000024 CB 18 00 EA   B       __2printf
\end{lstlisting}

\myindex{ARM!\Registers!Link Register}
\myindex{ARM!\Instructions!B}
\myindex{Function epilogue}
Questa e' la versione ottimizzata (\Othree) per ARM mode e stavolta notiamo \INS{B} come ultima istruzione, al posto della familiare \INS{BL}.
Un'altra differenza tra questa versione ottimizzata e la precedente (compilata senza ottimizzazione)
e' la mancanza di prologo ed epilogo della funzione (le istruzioni che preservano i valori dei registri \Reg{0} e \ac{LR}).
\myindex{x86!\Instructions!JMP}
L'istruzione \INS{B} salta semplicemente ad un altro indirizzo, senza alcuna manipolazione del registro \ac{LR}, in modo simile
a \JMP in x86.
Perche' funziona? Perche' questo codice e' infatti equivalente al precedente.
Principalmente per due motivi: 1) ne' lo stack ne' \ac{SP} (lo \gls{stack pointer}) vengono modificati;
2) la chiamata a \printf e' l'ultima istruzionie, quindi non succede niente dopo di essa.
Al completamento, \printf restituisce semplicemente il controllo all'indirizzo memorizzato in \ac{LR}.
Poiche' \ac{LR} attualmente contiene l'indirizzo del punto da cui la nostra funzione era stata chiamata, 
il controllo verra' restituito da \printf a quello stesso punto.
Pertanto non c'e' alcun bisogno di salvare \ac{LR} in quanto non abbiamo necessita' di modificare \ac{LR}.
E non vogliamo affatto modificare \ac{LR} poiche' non ci sono altre chiamate a funzione ad eccezzione di \printf. Inoltre, dopo
questa chiamata non abbiamo nient'altro da fare!
Queste le ragione per cui una simile ottimizzazione e' possibile.

Questa ottimizzazione e' spesso usata in funzioni in cui l'ultimo statement e' una chiamata ad un'altra funzione.
Un esempio simile e' fornito di seguito:
\myref{jump_to_last_printf}.

\myparagraph{ARM64}

\mysubparagraph{\NonOptimizing GCC (Linaro) 4.9}

% TODO translate to Italian:
\lstinputlisting[caption=\NonOptimizing GCC (Linaro) 4.9,style=customasmARM]{patterns/03_printf/ARM/ARM3_O0_EN.lst}

\myindex{ARM!\Instructions!STP}

La prima istruzione \INS{STP} (\IT{Store Pair}) salva \ac{FP} (X29) e \ac{LR} (X30) nello stack.
La seconda istruzione \INS{ADD X29, SP, 0} forma lo stack frame.
Scrive semplicemente il valore di \ac{SP} in X29.

\myindex{ARM!\Instructions!ADRP/ADD pair}
Successivamente vediamo la familiare coppia di istruzioni \INS{ADRP}/\ADD , che forma un puntatore alla stringa.
\myindex{ARM64!lo12}
\IT{lo12} indica 12 bit bassi (low 12 bits), ovvero, il linker scrivera' i 12 bit bassi dell'indirizzo di LC1 nell'opcode dell'istruzione \ADD .
\GTT{\%d} nella format string di \printf e' un \Tint a 32-bit, quindi u valori 1, 2 e 3 sono caricati nelle parti a 32-bit dei registri.

\Optimizing GCC (Linaro) 4.9 genera lo stesso codice.
}

\EN{\subsubsection{ARM: 8 arguments}

Let's use again the example with 9 arguments from the previous section: \myref{example_printf8_x64}.

\lstinputlisting[style=customc]{patterns/03_printf/2.c}

\myparagraph{\OptimizingKeilVI: \ARMMode}

\begin{lstlisting}[style=customasmARM]
.text:00000028             main
.text:00000028
.text:00000028             var_18 = -0x18
.text:00000028             var_14 = -0x14
.text:00000028             var_4  = -4
.text:00000028
.text:00000028 04 E0 2D E5  STR    LR, [SP,#var_4]!
.text:0000002C 14 D0 4D E2  SUB    SP, SP, #0x14
.text:00000030 08 30 A0 E3  MOV    R3, #8
.text:00000034 07 20 A0 E3  MOV    R2, #7
.text:00000038 06 10 A0 E3  MOV    R1, #6
.text:0000003C 05 00 A0 E3  MOV    R0, #5
.text:00000040 04 C0 8D E2  ADD    R12, SP, #0x18+var_14
.text:00000044 0F 00 8C E8  STMIA  R12, {R0-R3}
.text:00000048 04 00 A0 E3  MOV    R0, #4
.text:0000004C 00 00 8D E5  STR    R0, [SP,#0x18+var_18]
.text:00000050 03 30 A0 E3  MOV    R3, #3
.text:00000054 02 20 A0 E3  MOV    R2, #2
.text:00000058 01 10 A0 E3  MOV    R1, #1
.text:0000005C 6E 0F 8F E2  ADR    R0, aADBDCDDDEDFDGD ; "a=%d; b=%d; c=%d; d=%d; e=%d; f=%d; g=%"...
.text:00000060 BC 18 00 EB  BL     __2printf
.text:00000064 14 D0 8D E2  ADD    SP, SP, #0x14
.text:00000068 04 F0 9D E4  LDR    PC, [SP+4+var_4],#4
\end{lstlisting}

This code can be divided into several parts:

\myindex{Function prologue}
\begin{itemize}
\item Function prologue:

\myindex{ARM!\Instructions!STR}
The very first \INS{STR LR, [SP,\#var\_4]!} instruction saves \ac{LR} on the stack, because we are going to use this register for the \printf call.
Exclamation mark at the end indicates \IT{pre-index}.

This implies that \ac{SP} is to be decreased by 4 first, and then \ac{LR} will be saved at the address stored in \ac{SP}.
This is similar to \PUSH in x86.
Read more about it at: \myref{ARM_postindex_vs_preindex}.

\myindex{ARM!\Instructions!SUB}
The second \INS{SUB SP, SP, \#0x14} instruction decreases \ac{SP} (the \gls{stack pointer}) in order to allocate \GTT{0x14} (20) bytes on the stack.
Indeed, we have to pass 5 32-bit values via the stack to the \printf function, and each one occupies 4 bytes, which is exactly $5*4=20$.
The other 4 32-bit values are to be passed through registers.

\item Passing 5, 6, 7 and 8 via the stack: they are stored in the \Reg{0}, \Reg{1}, \Reg{2} and \Reg{3} registers respectively.\\
Then, the \INS{ADD R12, SP, \#0x18+var\_14} instruction writes the stack address where these 4 variables are to be stored, into the \Reg{12} register.
\myindex{IDA!var\_?}
\IT{var\_14} is an assembly macro, equal to -0x14, created by \IDA to conveniently display the code accessing the stack.
The \IT{var\_?} macros generated by \IDA reflect local variables in the stack.

So, \GTT{SP+4} is to be stored into the \Reg{12} register. \\
\myindex{ARM!\Instructions!STMIA}
The next \INS{STMIA R12, {R0-R3}} instruction writes registers \Reg{0}-\Reg{3} contents to the memory pointed by \Reg{12}.
\INS{STMIA} abbreviates \IT{Store Multiple Increment After}. 
\IT{\q{Increment After}} implies that \Reg{12} is to be increased by 4 after each register value is written.

\item Passing 4 via the stack: 4 is stored in \Reg{0} and then this value, with the help of the \\
\INS{STR R0, [SP,\#0x18+var\_18]} instruction is saved on the stack.
\IT{var\_18} is -0x18, so the offset is to be 0, thus the value from the \Reg{0} register (4) is to be written to the address written in \ac{SP}.

\item Passing 1, 2 and 3 via registers:
The values of the first 3 numbers (a, b, c) (1, 2, 3 respectively) are passed through the 
\Reg{1}, \Reg{2} and \Reg{3}
registers right before the \printf call, and the other
5 values are passed via the stack:

\item \printf call.

\myindex{Function epilogue}
\item Function epilogue:

The \INS{ADD SP, SP, \#0x14} instruction restores the \ac{SP} pointer back to its former value,
thus annulling everything what has been stored on the stack.
Of course, what has been stored on the stack will stay there, but it will all be rewritten during the execution of subsequent functions.

\myindex{ARM!\Instructions!LDR}
The \INS{LDR PC, [SP+4+var\_4],\#4} instruction loads the saved \ac{LR} value from the stack into the \ac{PC} register, thus causing the function to exit.
There is no exclamation mark---indeed, \ac{PC} is loaded first from the address stored in \ac{SP} ($4+var\_4=4+(-4)=0$, so this instruction is analogous to \INS{LDR PC, [SP],\#4}), and then \ac{SP} is increased by 4.
This is referred as \IT{post-index}\footnote{Read more about it: \myref{ARM_postindex_vs_preindex}.}.
Why does \IDA display the instruction like that?
Because it wants to illustrate the stack layout and the fact that \GTT{var\_4} is allocated for saving the \ac{LR} value in the local stack.
This instruction is somewhat similar to \INS{POP PC} in x86\footnote{It is impossible to set \GTT{IP/EIP/RIP} value using \POP in x86, but anyway, you got the analogy right.}.

\end{itemize}

\myparagraph{\OptimizingKeilVI: \ThumbMode}

\begin{lstlisting}[style=customasmARM]
.text:0000001C             printf_main2
.text:0000001C
.text:0000001C             var_18 = -0x18
.text:0000001C             var_14 = -0x14
.text:0000001C             var_8  = -8
.text:0000001C
.text:0000001C 00 B5        PUSH    {LR}
.text:0000001E 08 23        MOVS    R3, #8
.text:00000020 85 B0        SUB     SP, SP, #0x14
.text:00000022 04 93        STR     R3, [SP,#0x18+var_8]
.text:00000024 07 22        MOVS    R2, #7
.text:00000026 06 21        MOVS    R1, #6
.text:00000028 05 20        MOVS    R0, #5
.text:0000002A 01 AB        ADD     R3, SP, #0x18+var_14
.text:0000002C 07 C3        STMIA   R3!, {R0-R2}
.text:0000002E 04 20        MOVS    R0, #4
.text:00000030 00 90        STR     R0, [SP,#0x18+var_18]
.text:00000032 03 23        MOVS    R3, #3
.text:00000034 02 22        MOVS    R2, #2
.text:00000036 01 21        MOVS    R1, #1
.text:00000038 A0 A0        ADR     R0, aADBDCDDDEDFDGD ; "a=%d; b=%d; c=%d; d=%d; e=%d; f=%d; g=%"...
.text:0000003A 06 F0 D9 F8  BL      __2printf
.text:0000003E
.text:0000003E             loc_3E   ; CODE XREF: example13_f+16
.text:0000003E 05 B0        ADD     SP, SP, #0x14
.text:00000040 00 BD        POP     {PC}
\end{lstlisting}

The output is almost like in the previous example. However, this is Thumb code and the values are packed into stack differently: 
8 goes first, then 5, 6, 7, and 4 goes third.

\myparagraph{\OptimizingXcodeIV: \ARMMode}

\begin{lstlisting}[style=customasmARM]
__text:0000290C             _printf_main2
__text:0000290C
__text:0000290C             var_1C = -0x1C
__text:0000290C             var_C  = -0xC
__text:0000290C
__text:0000290C 80 40 2D E9   STMFD  SP!, {R7,LR}
__text:00002910 0D 70 A0 E1   MOV    R7, SP
__text:00002914 14 D0 4D E2   SUB    SP, SP, #0x14
__text:00002918 70 05 01 E3   MOV    R0, #0x1570
__text:0000291C 07 C0 A0 E3   MOV    R12, #7
__text:00002920 00 00 40 E3   MOVT   R0, #0
__text:00002924 04 20 A0 E3   MOV    R2, #4
__text:00002928 00 00 8F E0   ADD    R0, PC, R0
__text:0000292C 06 30 A0 E3   MOV    R3, #6
__text:00002930 05 10 A0 E3   MOV    R1, #5
__text:00002934 00 20 8D E5   STR    R2, [SP,#0x1C+var_1C]
__text:00002938 0A 10 8D E9   STMFA  SP, {R1,R3,R12}
__text:0000293C 08 90 A0 E3   MOV    R9, #8
__text:00002940 01 10 A0 E3   MOV    R1, #1
__text:00002944 02 20 A0 E3   MOV    R2, #2
__text:00002948 03 30 A0 E3   MOV    R3, #3
__text:0000294C 10 90 8D E5   STR    R9, [SP,#0x1C+var_C]
__text:00002950 A4 05 00 EB   BL     _printf
__text:00002954 07 D0 A0 E1   MOV    SP, R7
__text:00002958 80 80 BD E8   LDMFD  SP!, {R7,PC}
\end{lstlisting}

\myindex{ARM!\Instructions!STMFA}
\myindex{ARM!\Instructions!STMIB}
Almost the same as what we have already seen, with the
exception of \INS{STMFA} (Store Multiple Full Ascending) instruction, which is a synonym of \INS{STMIB} (Store Multiple Increment Before) instruction. 
This instruction increases the value in the \ac{SP} register and only then writes the next register value into the memory, rather than performing those two actions in the opposite order.

Another thing that catches the eye is that the instructions are arranged seemingly random.
For example, the value in the \Reg{0} register is manipulated in three
places, at addresses \GTT{0x2918}, \GTT{0x2920} and \GTT{0x2928}, when it would be possible to do it in one point.

However, the optimizing compiler may have its own reasons on how to order the instructions so to achieve higher efficiency during the execution.

Usually, the processor attempts to simultaneously execute instructions located side-by-side.\\
For example, instructions like \INS{MOVT R0, \#0} and
\INS{ADD R0, PC, R0} cannot be executed simultaneously since they both modify the \Reg{0} register. 
On the other hand, \INS{MOVT R0, \#0} and \INS{MOV R2, \#4} 
instructions can be executed
simultaneously since the effects of their execution are not conflicting with each other.
Presumably, the compiler tries to generate code in such a manner (wherever it is possible).
 
\myparagraph{\OptimizingXcodeIV: \ThumbTwoMode}

\begin{lstlisting}[style=customasmARM]
__text:00002BA0               _printf_main2
__text:00002BA0
__text:00002BA0               var_1C = -0x1C
__text:00002BA0               var_18 = -0x18
__text:00002BA0               var_C  = -0xC
__text:00002BA0
__text:00002BA0 80 B5          PUSH     {R7,LR}
__text:00002BA2 6F 46          MOV      R7, SP
__text:00002BA4 85 B0          SUB      SP, SP, #0x14
__text:00002BA6 41 F2 D8 20    MOVW     R0, #0x12D8
__text:00002BAA 4F F0 07 0C    MOV.W    R12, #7
__text:00002BAE C0 F2 00 00    MOVT.W   R0, #0
__text:00002BB2 04 22          MOVS     R2, #4
__text:00002BB4 78 44          ADD      R0, PC  ; char *
__text:00002BB6 06 23          MOVS     R3, #6
__text:00002BB8 05 21          MOVS     R1, #5
__text:00002BBA 0D F1 04 0E    ADD.W    LR, SP, #0x1C+var_18
__text:00002BBE 00 92          STR      R2, [SP,#0x1C+var_1C]
__text:00002BC0 4F F0 08 09    MOV.W    R9, #8
__text:00002BC4 8E E8 0A 10    STMIA.W  LR, {R1,R3,R12}
__text:00002BC8 01 21          MOVS     R1, #1
__text:00002BCA 02 22          MOVS     R2, #2
__text:00002BCC 03 23          MOVS     R3, #3
__text:00002BCE CD F8 10 90    STR.W    R9, [SP,#0x1C+var_C]
__text:00002BD2 01 F0 0A EA    BLX      _printf
__text:00002BD6 05 B0          ADD      SP, SP, #0x14
__text:00002BD8 80 BD          POP      {R7,PC}
\end{lstlisting}

The output is almost the same as in the previous example, with the exception that Thumb-instructions are used instead.
% FIXME: also STMIA is used instead of STMIB,
% which is why it uses LR, which is 4 bytes ahead of SP

\myparagraph{ARM64}

\mysubparagraph{\NonOptimizing GCC (Linaro) 4.9}

\lstinputlisting[caption=\NonOptimizing GCC (Linaro) 4.9,style=customasmARM]{patterns/03_printf/ARM/ARM8_O0_EN.lst}

The first 8 arguments are passed in X- or W-registers: \ARMPCS.
A string pointer requires a 64-bit register, so it's passed in \RegX{0}.
All other values have a \Tint 32-bit type, so they are stored in the 32-bit part of the registers (W-).
The 9th argument (8) is passed via the stack.
Indeed: it's not possible to pass large number of arguments through registers, because the number of registers is limited.

\Optimizing GCC (Linaro) 4.9 generates the same code.
}
\RU{\subsubsection{ARM: 8 аргументов}

Снова воспользуемся примером с 9-ю аргументами из предыдущей секции: \myref{example_printf8_x64}.

\lstinputlisting[style=customc]{patterns/03_printf/2.c}

\myparagraph{\OptimizingKeilVI: \ARMMode}

\begin{lstlisting}[style=customasmARM]
.text:00000028             main
.text:00000028
.text:00000028             var_18 = -0x18
.text:00000028             var_14 = -0x14
.text:00000028             var_4  = -4
.text:00000028
.text:00000028 04 E0 2D E5  STR    LR, [SP,#var_4]!
.text:0000002C 14 D0 4D E2  SUB    SP, SP, #0x14
.text:00000030 08 30 A0 E3  MOV    R3, #8
.text:00000034 07 20 A0 E3  MOV    R2, #7
.text:00000038 06 10 A0 E3  MOV    R1, #6
.text:0000003C 05 00 A0 E3  MOV    R0, #5
.text:00000040 04 C0 8D E2  ADD    R12, SP, #0x18+var_14
.text:00000044 0F 00 8C E8  STMIA  R12, {R0-R3}
.text:00000048 04 00 A0 E3  MOV    R0, #4
.text:0000004C 00 00 8D E5  STR    R0, [SP,#0x18+var_18]
.text:00000050 03 30 A0 E3  MOV    R3, #3
.text:00000054 02 20 A0 E3  MOV    R2, #2
.text:00000058 01 10 A0 E3  MOV    R1, #1
.text:0000005C 6E 0F 8F E2  ADR    R0, aADBDCDDDEDFDGD ; "a=%d; b=%d; c=%d; d=%d; e=%d; f=%d; g=%"...
.text:00000060 BC 18 00 EB  BL     __2printf
.text:00000064 14 D0 8D E2  ADD    SP, SP, #0x14
.text:00000068 04 F0 9D E4  LDR    PC, [SP+4+var_4],#4
\end{lstlisting}

Этот код можно условно разделить на несколько частей:

\myindex{Function prologue}
\begin{itemize}
\item Пролог функции:

\myindex{ARM!\Instructions!STR}
Самая первая инструкция \INS{STR LR, [SP,\#var\_4]!} 
сохраняет в стеке \ac{LR}, ведь нам придется использовать этот регистр для вызова \printf.
Восклицательный знак в конце означает \IT{pre-index}.
Это значит, что в начале \ac{SP} должно быть уменьшено на 4, затем по адресу в \ac{SP} должно быть записано значение \ac{LR}.

Это аналог знакомой в x86 инструкции \PUSH. Читайте больше об этом: \myref{ARM_postindex_vs_preindex}.

\myindex{ARM!\Instructions!SUB}
Вторая инструкция \INS{SUB SP, SP, \#0x14} уменьшает \glslink{stack pointer}{указатель стека} \ac{SP}, но, на самом деле, эта процедура нужна для выделения в локальном стеке места размером \GTT{0x14} (20) байт.
Действительно, нам нужно передать 5 32-битных значений через стек в \printf. Каждое значение занимает 4 байта, все вместе~--- $5*4=20$.
Остальные 4 32-битных значения будут переданы через регистры.

\item Передача 5, 6, 7 и 8 через стек:
они записываются в регистры \Reg{0}, \Reg{1}, \Reg{2} и \Reg{3} соответственно.\\
Затем инструкция \INS{ADD R12, SP, \#0x18+var\_14} 
записывает в регистр \Reg{12} адрес места в стеке, куда будут помещены эти 4 значения.
\myindex{IDA!var\_?}
\IT{var\_14}~--- это макрос ассемблера, равный -0x14.
Такие макросы создает \IDA, чтобы удобнее было показывать, как код обращается к стеку.

Макросы \IT{var\_?}, создаваемые \IDA, отражают локальные переменные в стеке. Так что в \Reg{12} будет записано \GTT{SP+4}.

\myindex{ARM!\Instructions!STMIA}
Следующая инструкция \INS{STMIA R12, {R0-R3}} записывает содержимое регистров \Reg{0}-\Reg{3} по адресу в памяти, на который указывает \Reg{12}.

Инструкция \INS{STMIA} означает \IT{Store Multiple Increment After}.\\
\IT{Increment After} означает, что \Reg{12} будет увеличиваться на 4 после записи каждого значения регистра.

\item Передача 4 через стек:
4 записывается в \Reg{0}, затем инструкция \INS{STR R0, [SP,\#0x18+var\_18]} записывает его в стек.
\IT{var\_18} равен -0x18, смещение будет 0, 
так что значение из регистра \Reg{0} (4) запишется туда, куда указывает \ac{SP}.

\item Передача 1, 2 и 3 через регистры:

Значения для первых трех чисел (a, b, c) (1, 2, 3 соответственно) передаются в регистрах 
\Reg{1}, \Reg{2} и \Reg{3} перед самим вызовом \printf, а остальные 5 значений передаются через стек, и вот как:

\item Вызов \printf.

\myindex{Function epilogue}
\item Эпилог функции:

Инструкция \INS{ADD SP, SP, \#0x14} возвращает \ac{SP} на прежнее место, 
аннулируя таким образом всё, что было записано в стек.
Конечно, то что было записано в стек, там пока и останется, но всё это будет многократно 
перезаписано во время исполнения последующих функций.

\myindex{ARM!\Instructions!LDR}
Инструкция \INS{LDR PC, [SP+4+var\_4],\#4} загружает в \ac{PC} 
сохраненное значение \ac{LR} из стека, обеспечивая таким образом выход из функции.

Здесь нет восклицательного знака~--- действительно, сначала \ac{PC} загружается из места, куда указывает \ac{SP}
($4+var\_4=4+(-4)=0$, так что эта инструкция аналогична \INS{LDR PC, [SP],\#4}), затем \ac{SP} увеличивается 
на 4.
Это называется \IT{post-index}\footnote{Читайте больше об этом: \myref{ARM_postindex_vs_preindex}.}.
Почему \IDA показывает инструкцию именно так?
Потому что она хочет показать разметку стека и тот факт, что \GTT{var\_4} выделена в локальном стеке именно для сохраненного
значения \ac{LR}.
Эта инструкция в каком-то смысле аналогична \INS{POP PC} в x86
\footnote{В x86 невозможно установить значение \GTT{IP/EIP/RIP} используя \POP, но будем надеяться, вы поняли аналогию.}.

\end{itemize}

\myparagraph{\OptimizingKeilVI: \ThumbMode}

\begin{lstlisting}[style=customasmARM]
.text:0000001C             printf_main2
.text:0000001C
.text:0000001C             var_18 = -0x18
.text:0000001C             var_14 = -0x14
.text:0000001C             var_8  = -8
.text:0000001C
.text:0000001C 00 B5        PUSH    {LR}
.text:0000001E 08 23        MOVS    R3, #8
.text:00000020 85 B0        SUB     SP, SP, #0x14
.text:00000022 04 93        STR     R3, [SP,#0x18+var_8]
.text:00000024 07 22        MOVS    R2, #7
.text:00000026 06 21        MOVS    R1, #6
.text:00000028 05 20        MOVS    R0, #5
.text:0000002A 01 AB        ADD     R3, SP, #0x18+var_14
.text:0000002C 07 C3        STMIA   R3!, {R0-R2}
.text:0000002E 04 20        MOVS    R0, #4
.text:00000030 00 90        STR     R0, [SP,#0x18+var_18]
.text:00000032 03 23        MOVS    R3, #3
.text:00000034 02 22        MOVS    R2, #2
.text:00000036 01 21        MOVS    R1, #1
.text:00000038 A0 A0        ADR     R0, aADBDCDDDEDFDGD ; "a=%d; b=%d; c=%d; d=%d; e=%d; f=%d; g=%"...
.text:0000003A 06 F0 D9 F8  BL      __2printf
.text:0000003E
.text:0000003E             loc_3E   ; CODE XREF: example13_f+16
.text:0000003E 05 B0        ADD     SP, SP, #0x14
.text:00000040 00 BD        POP     {PC}
\end{lstlisting}

Это почти то же самое что и в предыдущем примере, только код для Thumb и значения помещаются в 
стек немного иначе: сначала 8 за первый раз, затем 5, 6, 7 за второй раз и 4 за третий раз.

\myparagraph{\OptimizingXcodeIV: \ARMMode}

\begin{lstlisting}[style=customasmARM]
__text:0000290C             _printf_main2
__text:0000290C
__text:0000290C             var_1C = -0x1C
__text:0000290C             var_C  = -0xC
__text:0000290C
__text:0000290C 80 40 2D E9   STMFD  SP!, {R7,LR}
__text:00002910 0D 70 A0 E1   MOV    R7, SP
__text:00002914 14 D0 4D E2   SUB    SP, SP, #0x14
__text:00002918 70 05 01 E3   MOV    R0, #0x1570
__text:0000291C 07 C0 A0 E3   MOV    R12, #7
__text:00002920 00 00 40 E3   MOVT   R0, #0
__text:00002924 04 20 A0 E3   MOV    R2, #4
__text:00002928 00 00 8F E0   ADD    R0, PC, R0
__text:0000292C 06 30 A0 E3   MOV    R3, #6
__text:00002930 05 10 A0 E3   MOV    R1, #5
__text:00002934 00 20 8D E5   STR    R2, [SP,#0x1C+var_1C]
__text:00002938 0A 10 8D E9   STMFA  SP, {R1,R3,R12}
__text:0000293C 08 90 A0 E3   MOV    R9, #8
__text:00002940 01 10 A0 E3   MOV    R1, #1
__text:00002944 02 20 A0 E3   MOV    R2, #2
__text:00002948 03 30 A0 E3   MOV    R3, #3
__text:0000294C 10 90 8D E5   STR    R9, [SP,#0x1C+var_C]
__text:00002950 A4 05 00 EB   BL     _printf
__text:00002954 07 D0 A0 E1   MOV    SP, R7
__text:00002958 80 80 BD E8   LDMFD  SP!, {R7,PC}
\end{lstlisting}

\myindex{ARM!\Instructions!STMFA}
\myindex{ARM!\Instructions!STMIB}
Почти то же самое, что мы уже видели, за исключением того, что \INS{STMFA} (Store Multiple Full Ascending)~--- 
это синоним инструкции \INS{STMIB} (Store Multiple Increment Before). 
Эта инструкция увеличивает \ac{SP} и только затем записывает в память значение очередного регистра, но не наоборот.

Далее бросается в глаза то, что инструкции как будто бы расположены случайно.
Например, значение в регистре \Reg{0} подготавливается в трех местах, по адресам \GTT{0x2918}, \GTT{0x2920} и \GTT{0x2928}, 
когда это можно было бы сделать в одном месте.
Однако, у оптимизирующего компилятора могут быть свои доводы о том, как лучше составлять инструкции 
друг с другом для лучшей эффективности исполнения.
Процессор обычно пытается исполнять одновременно идущие друг за другом инструкции.
К примеру, инструкции \INS{MOVT R0, \#0} и \INS{ADD R0, PC, R0} не могут быть исполнены одновременно,
потому что обе инструкции модифицируют регистр \Reg{0}. 
А вот инструкции \INS{MOVT R0, \#0} и \INS{MOV R2, \#4} легко можно исполнить одновременно, 
потому что эффекты от их исполнения никак не конфликтуют друг с другом.
Вероятно, компилятор старается генерировать код именно таким образом там, где это возможно.
 
\myparagraph{\OptimizingXcodeIV: \ThumbTwoMode}

\begin{lstlisting}[style=customasmARM]
__text:00002BA0               _printf_main2
__text:00002BA0
__text:00002BA0               var_1C = -0x1C
__text:00002BA0               var_18 = -0x18
__text:00002BA0               var_C  = -0xC
__text:00002BA0
__text:00002BA0 80 B5          PUSH     {R7,LR}
__text:00002BA2 6F 46          MOV      R7, SP
__text:00002BA4 85 B0          SUB      SP, SP, #0x14
__text:00002BA6 41 F2 D8 20    MOVW     R0, #0x12D8
__text:00002BAA 4F F0 07 0C    MOV.W    R12, #7
__text:00002BAE C0 F2 00 00    MOVT.W   R0, #0
__text:00002BB2 04 22          MOVS     R2, #4
__text:00002BB4 78 44          ADD      R0, PC  ; char *
__text:00002BB6 06 23          MOVS     R3, #6
__text:00002BB8 05 21          MOVS     R1, #5
__text:00002BBA 0D F1 04 0E    ADD.W    LR, SP, #0x1C+var_18
__text:00002BBE 00 92          STR      R2, [SP,#0x1C+var_1C]
__text:00002BC0 4F F0 08 09    MOV.W    R9, #8
__text:00002BC4 8E E8 0A 10    STMIA.W  LR, {R1,R3,R12}
__text:00002BC8 01 21          MOVS     R1, #1
__text:00002BCA 02 22          MOVS     R2, #2
__text:00002BCC 03 23          MOVS     R3, #3
__text:00002BCE CD F8 10 90    STR.W    R9, [SP,#0x1C+var_C]
__text:00002BD2 01 F0 0A EA    BLX      _printf
__text:00002BD6 05 B0          ADD      SP, SP, #0x14
__text:00002BD8 80 BD          POP      {R7,PC}
\end{lstlisting}

Почти то же самое, что и в предыдущем примере,
лишь за тем исключением, что здесь используются Thumb-инструкции.

% FIXME: also STMIA is used instead of STMIB,
% which is why it uses LR, which is 4 bytes ahead of SP

\myparagraph{ARM64}

\mysubparagraph{\NonOptimizing GCC (Linaro) 4.9}

\lstinputlisting[caption=\NonOptimizing GCC (Linaro) 4.9,style=customasmARM]{patterns/03_printf/ARM/ARM8_O0_RU.lst}

Первые 8 аргументов передаются в X- или W-регистрах: \ARMPCS.
Указатель на строку требует 64-битного регистра, так что он передается в \RegX{0}.
Все остальные значения имеют 32-битный тип \Tint, так что они записываются в 32-битные части регистров (W-).
Девятый аргумент (8) передается через стек.
Действительно, невозможно передать большое количество аргументов в регистрах, потому что количество регистров ограничено.

\Optimizing GCC (Linaro) 4.9 генерирует почти такой же код.
}
\FR{\subsubsection{ARM: 8 arguments}

Utilisons de nouveau l'exemple avec 9 arguments de la section précédente: \myref{example_printf8_x64}.

\lstinputlisting[style=customc]{patterns/03_printf/2.c}

\myparagraph{\OptimizingKeilVI: \ARMMode}

\begin{lstlisting}[style=customasmARM]
.text:00000028             main
.text:00000028
.text:00000028             var_18 = -0x18
.text:00000028             var_14 = -0x14
.text:00000028             var_4  = -4
.text:00000028
.text:00000028 04 E0 2D E5  STR    LR, [SP,#var_4]!
.text:0000002C 14 D0 4D E2  SUB    SP, SP, #0x14
.text:00000030 08 30 A0 E3  MOV    R3, #8
.text:00000034 07 20 A0 E3  MOV    R2, #7
.text:00000038 06 10 A0 E3  MOV    R1, #6
.text:0000003C 05 00 A0 E3  MOV    R0, #5
.text:00000040 04 C0 8D E2  ADD    R12, SP, #0x18+var_14
.text:00000044 0F 00 8C E8  STMIA  R12, {R0-R3}
.text:00000048 04 00 A0 E3  MOV    R0, #4
.text:0000004C 00 00 8D E5  STR    R0, [SP,#0x18+var_18]
.text:00000050 03 30 A0 E3  MOV    R3, #3
.text:00000054 02 20 A0 E3  MOV    R2, #2
.text:00000058 01 10 A0 E3  MOV    R1, #1
.text:0000005C 6E 0F 8F E2  ADR    R0, aADBDCDDDEDFDGD ; "a=%d; b=%d; c=%d; d=%d; e=%d; f=%d; g=%"...
.text:00000060 BC 18 00 EB  BL     __2printf
.text:00000064 14 D0 8D E2  ADD    SP, SP, #0x14
.text:00000068 04 F0 9D E4  LDR    PC, [SP+4+var_4],#4
\end{lstlisting}

Ce code peut être divisé en plusieurs parties:

\myindex{Prologue de la fonction}
\begin{itemize}
\item Prologue de la fonction:

\myindex{ARM!\Instructions!STR}
La toute première instruction \INS{STR LR, [SP,\#var\_4]!} sauve \ac{LR} sur la pile,
car nous allons utiliser ce registre pour l'appel à \printf.
Le point d'exclamation à la fin indique un \IT{pré-index}.

Cela signifie que \ac{SP} est d'abord décrémenté de 4, et qu'ensuite \ac{LR}
va être sauvé à l'adresse stockée dans \ac{SP}.
C'est similaire à \PUSH en x86.
Lire aussi à ce propos: \myref{ARM_postindex_vs_preindex}.

\myindex{ARM!\Instructions!SUB}
La seconde instruction \INS{SUB SP, SP, \#0x14} décrémente \ac{SP} (le \glslink{stack pointer}{pointeur de pile})
afin d'allouer \GTT{0x14} (20) octets sur la pile.
En effet, nous devons passer 5 valeurs de 32-bit par la pile à la fonction \printf,
et chacune occupe 4 octets, ce qui fait exactement $5*4=20$.
Les 4 autres valeurs de 32-bit sont passées par les registres.

\item Passer 5, 6, 7 et 8 par la pile: ils sont stockés dans les registres \Reg{0},
\Reg{1}, \Reg{2} et \Reg{3} respectivement.\\
Ensuite, l'instruction \INS{ADD R12, SP, \#0x18+var\_14} écrit l'adresse de la pile
où ces 4 variables doivent être stockées dans le registre \Reg{12}.
\myindex{IDA!var\_?}
\IT{var\_14} est une macro d'assemblage, égal à -0x14, créée par \IDA pour afficher
commodément le code accédant à la pile.
Les macros \IT{var\_?} générée par \IDA reflètent les variables locales dans la pile.

Donc, \GTT{SP+4} doit être stocké dans le registre \Reg{12}. \\
\myindex{ARM!\Instructions!STMIA}
L'instruction suivante \INS{STMIA R12, {R0-R3}} écrit le contenu des registres
\Reg{0}-\Reg{3} dans la mémoire pointée par \Reg{12}.
\INS{STMIA} est l'abréviation de \IT{Store Multiple Increment After} (stocker plusieur incrémenter après).
\IT{\q{Increment After}} signifie que \Reg{12} doit être incrémenté de 4 après
l'écriture de chaque valeur d'un registre.

\item Passer 4 par la pile: 4 est stocké dans \Reg{0} et ensuite, cette valeur, avec l'aide de \\
l'instruction \INS{STR R0, [SP,\#0x18+var\_18]} est sauvée dans la pile.
\IT{var\_18} est -0x18, donc l'offset est 0, donc la valeur du registre \Reg{0}
(4) est écrite à l'adresse écrite dans \ac{SP}.

\item Passer 1, 2 et 3 par des registres:
Les valeurs des 3 premiers nombres (a,b,c) (respectivement 1, 2, 3) sont passées
par les registres \Reg{1}, \Reg{2} et \Reg{3} juste avant l'appel de \printf, et
les 5 autres valeurs sont passées par la pile:

\item appel de \printf

\myindex{Épilogue de fonction}
\item Épilogue de fonction:

L'instruction \INS{ADD SP, SP, \#0x14} restaure le pointeur \ac{SP} à sa valeur
précédente, nettoyant ainsi la pile.
Bien sûr, ce qui a été stocké sur la pile y reste, mais sera récrit lors de
l'exécution ultérieure de fonctions.

\myindex{ARM!\Instructions!LDR}
L'instruction \INS{LDR PC, [SP+4+var\_4],\#4} charge la valeur sauvée de \ac{LR}
depuis la pile dans le registre \ac{PC}, provoquant ainsi la sortie de la fonction.
Il n'y a pas de point d'xclamation---effectivement, \ac{PC} est d'abord chargé
depuis l'adresse stockées dans \ac{SP} ($4+var\_4=4+(-4)=0$, donc cette instruction
est analogue à INS{LDR PC, [SP],\#4}), et ensuite \ac{SP} est incrémenté de 4.
Il s'agit de \IT{post-index}\footnote{Lire à ce propos: \myref{ARM_postindex_vs_preindex}.}.
Pourquoi est-ce qu'\IDA affiche l'instruction comme ça?
Parqu'il veut illustrer la disposition de la pile et le fait que \GTT{var\_4} est
alloué pour sauver la valeur de \ac{LR} dans la pile locale.
Cette instruction est quelque peu similaire à \INS{POP PC} en x86\footnote{Il est
impossible de définir la valeur de \GTT{IP/EIP/RIP} en utilisant \POP en x86, mais
de toutes façons, vous avez le droit de faire l'analogie.}.

\end{itemize}

\myparagraph{\OptimizingKeilVI: \ThumbMode}

\begin{lstlisting}[style=customasmARM]
.text:0000001C             printf_main2
.text:0000001C
.text:0000001C             var_18 = -0x18
.text:0000001C             var_14 = -0x14
.text:0000001C             var_8  = -8
.text:0000001C
.text:0000001C 00 B5        PUSH    {LR}
.text:0000001E 08 23        MOVS    R3, #8
.text:00000020 85 B0        SUB     SP, SP, #0x14
.text:00000022 04 93        STR     R3, [SP,#0x18+var_8]
.text:00000024 07 22        MOVS    R2, #7
.text:00000026 06 21        MOVS    R1, #6
.text:00000028 05 20        MOVS    R0, #5
.text:0000002A 01 AB        ADD     R3, SP, #0x18+var_14
.text:0000002C 07 C3        STMIA   R3!, {R0-R2}
.text:0000002E 04 20        MOVS    R0, #4
.text:00000030 00 90        STR     R0, [SP,#0x18+var_18]
.text:00000032 03 23        MOVS    R3, #3
.text:00000034 02 22        MOVS    R2, #2
.text:00000036 01 21        MOVS    R1, #1
.text:00000038 A0 A0        ADR     R0, aADBDCDDDEDFDGD ; "a=%d; b=%d; c=%d; d=%d; e=%d; f=%d; g=%"...
.text:0000003A 06 F0 D9 F8  BL      __2printf
.text:0000003E
.text:0000003E             loc_3E   ; CODE XREF: example13_f+16
.text:0000003E 05 B0        ADD     SP, SP, #0x14
.text:00000040 00 BD        POP     {PC}
\end{lstlisting}

La sortie est presque comme dans les exemples précédents. Toutefois, c'est du code
Thumb et les valeurs sont arrangées différemment dans la pile:
8 vient en premier, puis 5, 6, 7 et 4 vient en troisième premier, puis 5, 6, 7 et 4 vient en troisième..

\myparagraph{\OptimizingXcodeIV: \ARMMode}

\begin{lstlisting}[style=customasmARM]
__text:0000290C             _printf_main2
__text:0000290C
__text:0000290C             var_1C = -0x1C
__text:0000290C             var_C  = -0xC
__text:0000290C
__text:0000290C 80 40 2D E9   STMFD  SP!, {R7,LR}
__text:00002910 0D 70 A0 E1   MOV    R7, SP
__text:00002914 14 D0 4D E2   SUB    SP, SP, #0x14
__text:00002918 70 05 01 E3   MOV    R0, #0x1570
__text:0000291C 07 C0 A0 E3   MOV    R12, #7
__text:00002920 00 00 40 E3   MOVT   R0, #0
__text:00002924 04 20 A0 E3   MOV    R2, #4
__text:00002928 00 00 8F E0   ADD    R0, PC, R0
__text:0000292C 06 30 A0 E3   MOV    R3, #6
__text:00002930 05 10 A0 E3   MOV    R1, #5
__text:00002934 00 20 8D E5   STR    R2, [SP,#0x1C+var_1C]
__text:00002938 0A 10 8D E9   STMFA  SP, {R1,R3,R12}
__text:0000293C 08 90 A0 E3   MOV    R9, #8
__text:00002940 01 10 A0 E3   MOV    R1, #1
__text:00002944 02 20 A0 E3   MOV    R2, #2
__text:00002948 03 30 A0 E3   MOV    R3, #3
__text:0000294C 10 90 8D E5   STR    R9, [SP,#0x1C+var_C]
__text:00002950 A4 05 00 EB   BL     _printf
__text:00002954 07 D0 A0 E1   MOV    SP, R7
__text:00002958 80 80 BD E8   LDMFD  SP!, {R7,PC}
\end{lstlisting}

\myindex{ARM!\Instructions!STMFA}
\myindex{ARM!\Instructions!STMIB}
Presque la même chose que ce que nous avons déjà vu, avec l'exception de l'instruction
\INS{STMFA} (Store Multiple Full Ascending), qui est un synonyme de l'instruction
\INS{STMIB} (Store Multiple Increment Before).
Cette instruction incrémente la valeur du registre \ac{SP} et écrit seulement
après la valeur registre suivant dans la mémoire, plutôt que d'effectuer ces
deux actions dans l'ordre inverse.

Une autre chose qui accroche le regard est que les instructions semblent être arrangées
de manière aléatoire.
Par exemple, la valeur dans le registre \Reg{0} est manipulée en trois endroits,
aux adresses \GTT{0x2918}, \GTT{0x2920} et \GTT{0x2928}, alors qu'il serait possible
de le faire en un seul endroit.

Toutefois, le compilateur qui optimise doit avoir ses propres raisons d'ordonner
les instructions pour avoir une plus grande éfficacité à l'exécution.

D'habitude, le processeur essaye d'exécuter simultanément des instructions situées côte à côte.\\
Par exemple, des instructions comme \INS{MOVT R0, \#0} et \INS{ADD R0, PC, R0} ne
peuvent pas être exécutées simultanément puisqu'elles modifient toutes deux le
registre \Reg{0}.
D'un autre côté, les instructions \INS{MOVT R0, \#0} et \INS{MOV R2, \#4} peuvent
être exécutées simlutanément puisque leurs effets n'interfèrent pas l'un avec l'autre
lors de leurs exécution.
Probablement que le compilateur essaye de générer du code arrangé de cette façon
(lorsque c'est possible).

\myparagraph{\OptimizingXcodeIV: \ThumbTwoMode}

\begin{lstlisting}[style=customasmARM]
__text:00002BA0               _printf_main2
__text:00002BA0
__text:00002BA0               var_1C = -0x1C
__text:00002BA0               var_18 = -0x18
__text:00002BA0               var_C  = -0xC
__text:00002BA0
__text:00002BA0 80 B5          PUSH     {R7,LR}
__text:00002BA2 6F 46          MOV      R7, SP
__text:00002BA4 85 B0          SUB      SP, SP, #0x14
__text:00002BA6 41 F2 D8 20    MOVW     R0, #0x12D8
__text:00002BAA 4F F0 07 0C    MOV.W    R12, #7
__text:00002BAE C0 F2 00 00    MOVT.W   R0, #0
__text:00002BB2 04 22          MOVS     R2, #4
__text:00002BB4 78 44          ADD      R0, PC  ; char *
__text:00002BB6 06 23          MOVS     R3, #6
__text:00002BB8 05 21          MOVS     R1, #5
__text:00002BBA 0D F1 04 0E    ADD.W    LR, SP, #0x1C+var_18
__text:00002BBE 00 92          STR      R2, [SP,#0x1C+var_1C]
__text:00002BC0 4F F0 08 09    MOV.W    R9, #8
__text:00002BC4 8E E8 0A 10    STMIA.W  LR, {R1,R3,R12}
__text:00002BC8 01 21          MOVS     R1, #1
__text:00002BCA 02 22          MOVS     R2, #2
__text:00002BCC 03 23          MOVS     R3, #3
__text:00002BCE CD F8 10 90    STR.W    R9, [SP,#0x1C+var_C]
__text:00002BD2 01 F0 0A EA    BLX      _printf
__text:00002BD6 05 B0          ADD      SP, SP, #0x14
__text:00002BD8 80 BD          POP      {R7,PC}
\end{lstlisting}

La sortie est presque la même que dans l'exemple précédent, avec l'exception que
des instructions Thumb sont utilisées à la place.
% FIXME: also STMIA is used instead of STMIB,
% which is why it uses LR, which is 4 bytes ahead of SP

\myparagraph{ARM64}

\mysubparagraph{GCC (Linaro) 4.9 \NonOptimizing}

\lstinputlisting[caption=GCC (Linaro) 4.9 \NonOptimizing,style=customasmARM]{patterns/03_printf/ARM/ARM8_O0_FR.lst}

Les 8 premiers arguments sont passés dans des registres X- ou W-: \ARMPCS.
Un pointeur de chaîne nécessite un registre 64-bit, donc il est passé dans \RegX{0}.
Toutes les autres valeurs ont un type \Tint 32-bit, donc elles sont stockées dans
la partie 32-bit des registres (W-).
Le 9ème argument (8) est passé par la pile.
En effet: il m'est pas possible de passer un grand nombre d'arguments par les registres,
car le nombre de registres est limité.

GCC (Linaro) 4.9 \Optimizing génère le même code.
}
\ITA{\subsubsection{ARM: 8 arguments}

Usiamo nuovamente l'esempio con 9 argomenti della sezione precedente: \myref{example_printf8_x64}.

\lstinputlisting[style=customc]{patterns/03_printf/2.c}

\myparagraph{\OptimizingKeilVI: \ARMMode}

\begin{lstlisting}[style=customasmARM]
.text:00000028             main
.text:00000028
.text:00000028             var_18 = -0x18
.text:00000028             var_14 = -0x14
.text:00000028             var_4  = -4
.text:00000028
.text:00000028 04 E0 2D E5  STR    LR, [SP,#var_4]!
.text:0000002C 14 D0 4D E2  SUB    SP, SP, #0x14
.text:00000030 08 30 A0 E3  MOV    R3, #8
.text:00000034 07 20 A0 E3  MOV    R2, #7
.text:00000038 06 10 A0 E3  MOV    R1, #6
.text:0000003C 05 00 A0 E3  MOV    R0, #5
.text:00000040 04 C0 8D E2  ADD    R12, SP, #0x18+var_14
.text:00000044 0F 00 8C E8  STMIA  R12, {R0-R3}
.text:00000048 04 00 A0 E3  MOV    R0, #4
.text:0000004C 00 00 8D E5  STR    R0, [SP,#0x18+var_18]
.text:00000050 03 30 A0 E3  MOV    R3, #3
.text:00000054 02 20 A0 E3  MOV    R2, #2
.text:00000058 01 10 A0 E3  MOV    R1, #1
.text:0000005C 6E 0F 8F E2  ADR    R0, aADBDCDDDEDFDGD ; "a=%d; b=%d; c=%d; d=%d; e=%d; f=%d; g=%"...
.text:00000060 BC 18 00 EB  BL     __2printf
.text:00000064 14 D0 8D E2  ADD    SP, SP, #0x14
.text:00000068 04 F0 9D E4  LDR    PC, [SP+4+var_4],#4
\end{lstlisting}

Il codice puo' essre diviso in piu' parti

\myindex{Function prologue}
\begin{itemize}
\item Preambolo della funzione:

\myindex{ARM!\Instructions!STR}
La prima istruzione \INS{STR LR, [SP,\#var\_4]!} salva \ac{LR} sullo stack, poiche' questo registro sara' usato per la chiamata a \printf.
Il punto esclamativo all fine indica il \IT{pre-index}.

Questo implica che \ac{SP} deve essere prima decrementato di 4, e successivamente \ac{LR} sara' salvato all'indirizzo memorizzato in \ac{SP}.
Tutto cio' e' simile a \PUSH in x86.
Maggiori informazioni qui: \myref{ARM_postindex_vs_preindex}.

\myindex{ARM!\Instructions!SUB}
La seconda istruzione \INS{SUB SP, SP, \#0x14} decrementa \ac{SP} (lo \gls{stack pointer}) per allocare \GTT{0x14} (20) byte sullo stack.
Infatti dobbiamo passare 5 valori a 32-bit tramite lo stack per la funzione \printf, e ciascuno di essi occupa 4 byte, che e' esattamente $5*4=20$.
Gli altri 4 valori a 32-bit saranno passati tramite registri.

\item Passaggio di 5, 6, 7 e 8 tramite lo stack: sono memorizzati nei registri \Reg{0}, \Reg{1}, \Reg{2} e \Reg{3}, rispettivamente.\\
Successivamente l'istruzione \INS{ADD R12, SP, \#0x18+var\_14} scrive l'indirizzo dello stack, dove queste 4 variabili saranno memorizzate,
nel registri \Reg{12}.
\myindex{IDA!var\_?}
\IT{var\_14} e' una macro assembly, uguale a -0x14, creata da \IDA per visualizzare in maniera conveniente il codice che accede allo stack.
Le macro \IT{var\_?} generate da \IDA riflettono le variabili locali nello stack.

Quindi, \GTT{SP+4} sara' memorizzato nel registro \Reg{12}.
\myindex{ARM!\Instructions!STMIA}
L'istruzione successiva \INS{STMIA R12, {R0-R3}} scrive il contenuto dei registri \Reg{0}-\Reg{3} alla memoria puntata da \Reg{12}.
\INS{STMIA} e' abbreviazione per \IT{Store Multiple Increment After}. 
\IT{\q{Increment After}} (incrementa dopo) implica che \Reg{12} deve essere incrementato di 4 dopo ciascuna scrittura di un valore nei registri.

\item Passaggio di 4 tramite lo stack: 4 e' memorizzato in \Reg{0} e questo valore, con l'aiuto dell'istruzione \INS{STR R0, [SP,\#0x18+var\_18]}, viene salvato sullo stack.
\IT{var\_18} e' -0x18, quindi l'offset deve essere 0, da cui il valore dal registro \Reg{0} (4) sara' scritto all'indirizzo memorizzato in \ac{SP}.

\item Passaggio di 1, 2 e 3 tramite registri:
I valori dei primi 3 numeri (a, b, c) (1, 2, 3 rispettivamente) sono passatti attraverso i registri
\Reg{1}, \Reg{2} e \Reg{3}
poco prima della chiamata a \printf call, e gli altri 5 valori sono passati tramite lo stack:

\item chiamata a \printf.

\myindex{Function epilogue}
\item epilogo della funzione:

% TODO to be synced with EN version:
L'istruzione \INS{ADD SP, SP, \#0x14} ripristina il puntatore \ac{SP} al suo valore precedente, pulendo quindi lo stack.
Ovviamente quello che era stato memorizzato nello stack rimarra li', e sara' probabilmente riscritto interamente durante l'e
secuzione delle funzioni seguenti.

\myindex{ARM!\Instructions!LDR}
L'istruzione \INS{LDR PC, [SP+4+var\_4],\#4} carica il valore di \ac{LR} salvato dallo stack nel nel registro \ac{PC}, causando quindi
l'uscita dalla funzione.
Non c'e' il punto esclamativo ---infatti \ac{PC} e' caricato prima dall'indirizzo memorizzato in \ac{SP} ($4+var\_4=4+(-4)=0$, questa istruzione e' quindi analoga a \INS{LDR PC, [SP],\#4}), e successivamente \ac{SP} e' incrementato di 4.
Questo e' detto \IT{post-index}\footnote{Maggiori dettagli: \myref{ARM_postindex_vs_preindex}.}.
Perche' \IDA mostra l'istruzione in quel modo?
Perche' vuole illustrare il layout dello stack ed il fatto che \GTT{var\_4} e' allocata per salvare il valore di \ac{LR} nello stack locale.
Questa istruzione e' piu' o meno simile a \INS{POP PC} in x86\footnote{E' impossibile settare il valore di \GTT{IP/EIP/RIP} usando \POP in x86, ma in ogni caso hai capito l'analogia.}.

\end{itemize}

\myparagraph{\OptimizingKeilVI: \ThumbMode}

\begin{lstlisting}[style=customasmARM]
.text:0000001C             printf_main2
.text:0000001C
.text:0000001C             var_18 = -0x18
.text:0000001C             var_14 = -0x14
.text:0000001C             var_8  = -8
.text:0000001C
.text:0000001C 00 B5        PUSH    {LR}
.text:0000001E 08 23        MOVS    R3, #8
.text:00000020 85 B0        SUB     SP, SP, #0x14
.text:00000022 04 93        STR     R3, [SP,#0x18+var_8]
.text:00000024 07 22        MOVS    R2, #7
.text:00000026 06 21        MOVS    R1, #6
.text:00000028 05 20        MOVS    R0, #5
.text:0000002A 01 AB        ADD     R3, SP, #0x18+var_14
.text:0000002C 07 C3        STMIA   R3!, {R0-R2}
.text:0000002E 04 20        MOVS    R0, #4
.text:00000030 00 90        STR     R0, [SP,#0x18+var_18]
.text:00000032 03 23        MOVS    R3, #3
.text:00000034 02 22        MOVS    R2, #2
.text:00000036 01 21        MOVS    R1, #1
.text:00000038 A0 A0        ADR     R0, aADBDCDDDEDFDGD ; "a=%d; b=%d; c=%d; d=%d; e=%d; f=%d; g=%"...
.text:0000003A 06 F0 D9 F8  BL      __2printf
.text:0000003E
.text:0000003E             loc_3E   ; CODE XREF: example13_f+16
.text:0000003E 05 B0        ADD     SP, SP, #0x14
.text:00000040 00 BD        POP     {PC}
\end{lstlisting}

L'output e' quasi identico al precedente esempio. Tuttavia questo e' codice Thumb e i valori sono disposti nello stack in modo differente:
8 per primo, quindi 5, 6, 7, e infine 4.

\myparagraph{\OptimizingXcodeIV: \ARMMode}

\begin{lstlisting}[style=customasmARM]
__text:0000290C             _printf_main2
__text:0000290C
__text:0000290C             var_1C = -0x1C
__text:0000290C             var_C  = -0xC
__text:0000290C
__text:0000290C 80 40 2D E9   STMFD  SP!, {R7,LR}
__text:00002910 0D 70 A0 E1   MOV    R7, SP
__text:00002914 14 D0 4D E2   SUB    SP, SP, #0x14
__text:00002918 70 05 01 E3   MOV    R0, #0x1570
__text:0000291C 07 C0 A0 E3   MOV    R12, #7
__text:00002920 00 00 40 E3   MOVT   R0, #0
__text:00002924 04 20 A0 E3   MOV    R2, #4
__text:00002928 00 00 8F E0   ADD    R0, PC, R0
__text:0000292C 06 30 A0 E3   MOV    R3, #6
__text:00002930 05 10 A0 E3   MOV    R1, #5
__text:00002934 00 20 8D E5   STR    R2, [SP,#0x1C+var_1C]
__text:00002938 0A 10 8D E9   STMFA  SP, {R1,R3,R12}
__text:0000293C 08 90 A0 E3   MOV    R9, #8
__text:00002940 01 10 A0 E3   MOV    R1, #1
__text:00002944 02 20 A0 E3   MOV    R2, #2
__text:00002948 03 30 A0 E3   MOV    R3, #3
__text:0000294C 10 90 8D E5   STR    R9, [SP,#0x1C+var_C]
__text:00002950 A4 05 00 EB   BL     _printf
__text:00002954 07 D0 A0 E1   MOV    SP, R7
__text:00002958 80 80 BD E8   LDMFD  SP!, {R7,PC}
\end{lstlisting}

\myindex{ARM!\Instructions!STMFA}
\myindex{ARM!\Instructions!STMIB}
Quasi lo stesso codice visto prima, ad eccezione dell'istruzione \INS{STMFA} (Store Multiple Full Ascending),
che e' sinonimo di \INS{STMIB} (Store Multiple Increment Before). 
Questa istruzione incrementa il valore nel registro \ac{SP} e solo successivamente scrive il prossimo valore del registro in memoria, invece che operare le due azioni in ordine inverso.

Un'altra cosa che salta all'occhio e' che le istruzioni sono disponste in maniera apparentemente casuale.
Ad esempio, il valore nel registro \Reg{0} e' manipolato in tre posti diversi
agli indirizzi \GTT{0x2918}, \GTT{0x2920} e \GTT{0x2928}, quando invece sarebbe stato possibile farlo in un punto solo.

Ad ogni modo, il compilatore ottimizzante avra' avuto le sue ragioni per ordinare le istruzioni in questa maniera ed ottenere una maggiore efficacia durante l'esecuzione del codice.

Solitamente il processero prova ad eseguire simultaneamentele istruzioni vicine.
Ad esempio, istruzioni come \INS{MOVT R0, \#0} e
\INS{ADD R0, PC, R0} non possono essere eseguite simultaneamente poiche' entrambe modificano il registro \Reg{0}. 
D'altra parte, \INS{MOVT R0, \#0} e \INS{MOV R2, \#4} 
possono invece essere eseguite simultaneamente poiche' l'effetto della loro esecuzione non genera conflitti tra loro.
Presumibilmente, il compilatore prova a generare codice in questo modo (quando possibile).
 
\myparagraph{\OptimizingXcodeIV: \ThumbTwoMode}

\begin{lstlisting}[style=customasmARM]
__text:00002BA0               _printf_main2
__text:00002BA0
__text:00002BA0               var_1C = -0x1C
__text:00002BA0               var_18 = -0x18
__text:00002BA0               var_C  = -0xC
__text:00002BA0
__text:00002BA0 80 B5          PUSH     {R7,LR}
__text:00002BA2 6F 46          MOV      R7, SP
__text:00002BA4 85 B0          SUB      SP, SP, #0x14
__text:00002BA6 41 F2 D8 20    MOVW     R0, #0x12D8
__text:00002BAA 4F F0 07 0C    MOV.W    R12, #7
__text:00002BAE C0 F2 00 00    MOVT.W   R0, #0
__text:00002BB2 04 22          MOVS     R2, #4
__text:00002BB4 78 44          ADD      R0, PC  ; char *
__text:00002BB6 06 23          MOVS     R3, #6
__text:00002BB8 05 21          MOVS     R1, #5
__text:00002BBA 0D F1 04 0E    ADD.W    LR, SP, #0x1C+var_18
__text:00002BBE 00 92          STR      R2, [SP,#0x1C+var_1C]
__text:00002BC0 4F F0 08 09    MOV.W    R9, #8
__text:00002BC4 8E E8 0A 10    STMIA.W  LR, {R1,R3,R12}
__text:00002BC8 01 21          MOVS     R1, #1
__text:00002BCA 02 22          MOVS     R2, #2
__text:00002BCC 03 23          MOVS     R3, #3
__text:00002BCE CD F8 10 90    STR.W    R9, [SP,#0x1C+var_C]
__text:00002BD2 01 F0 0A EA    BLX      _printf
__text:00002BD6 05 B0          ADD      SP, SP, #0x14
__text:00002BD8 80 BD          POP      {R7,PC}
\end{lstlisting}

L'output e' quasi lo stesso dell'esempio precedente, ad eccezione dell'uso di istruzioni Thumb. 
% FIXME: also STMIA is used instead of STMIB,
% which is why it uses LR, which is 4 bytes ahead of SP

\myparagraph{ARM64}

\mysubparagraph{\NonOptimizing GCC (Linaro) 4.9}

% TODO translate to Italian:
\lstinputlisting[caption=\NonOptimizing GCC (Linaro) 4.9,style=customasmARM]{patterns/03_printf/ARM/ARM8_O0_EN.lst}

I primi 8 argomenti sono passati nei registri X- o W-: \ARMPCS.
Un puntatore ad una string richiede un registro a 64-bit, quindi e' passato in \RegX{0}.
Tutti gli altri valori hanno timpo \Tint a 32-bit, quindi sono memorizzati nella parte a 32-bit dei registri (W-).
Il nono argomento (8) e' passato tramite lo stack.
Infatti non e' possibile passare un grande numero di argomenti tramite registri, in quanto il loro numero e' limitato.

\Optimizing GCC (Linaro) 4.9 genera lo stesso codice.
}

\EN{\subsection{MIPS}

\subsubsection{3 arguments}

\myparagraph{\Optimizing GCC 4.4.5}

The main difference with the \q{\HelloWorldSectionName} example is that in this case \printf is called
instead of \puts and 3 more arguments are passed through the registers \$5\dots \$7 (or \$A0\dots \$A2).
That is why these registers are prefixed with A-, which implies they are used for function arguments passing.

\lstinputlisting[caption=\Optimizing GCC 4.4.5 (\assemblyOutput),style=customasmMIPS]{patterns/03_printf/MIPS/printf3.O3_EN.s}

\lstinputlisting[caption=\Optimizing GCC 4.4.5 (IDA),style=customasmMIPS]{patterns/03_printf/MIPS/printf3.O3.IDA_EN.lst}

\IDA has coalesced pair of \INS{LUI} and \INS{ADDIU} instructions into one \INS{LA} pseudo instruction.
That's why there are no instruction at address 0x1C: because \INS{LA} \IT{occupies} 8 bytes.

\myparagraph{\NonOptimizing GCC 4.4.5}

\NonOptimizing GCC is more verbose:

\lstinputlisting[caption=\NonOptimizing GCC 4.4.5 (\assemblyOutput),style=customasmMIPS]{patterns/03_printf/MIPS/printf3.O0_EN.s}

\lstinputlisting[caption=\NonOptimizing GCC 4.4.5 (IDA),style=customasmMIPS]{patterns/03_printf/MIPS/printf3.O0.IDA_EN.lst}

\subsubsection{8 arguments}

Let's use again the example with 9 arguments from the previous section: \myref{example_printf8_x64}.

\lstinputlisting[style=customc]{patterns/03_printf/2.c}

\myparagraph{\Optimizing GCC 4.4.5}

Only the first 4 arguments are passed in the \$A0 \dots \$A3 registers, the rest are passed via the stack.
\myindex{MIPS!O32}

This is the O32 calling convention (which is the most common one in the MIPS world).
Other calling conventions (like N32) may use the registers for different purposes.

\myindex{MIPS!\Instructions!SW}

\INS{SW} abbreviates \q{Store Word} (from register to memory).
MIPS lacks instructions for storing a value into memory, so an instruction pair has to be used instead (\INS{LI}/\INS{SW}).

\lstinputlisting[caption=\Optimizing GCC 4.4.5 (\assemblyOutput),style=customasmMIPS]{patterns/03_printf/MIPS/printf8.O3_EN.s}

\lstinputlisting[caption=\Optimizing GCC 4.4.5 (IDA),style=customasmMIPS]{patterns/03_printf/MIPS/printf8.O3.IDA_EN.lst}

\myparagraph{\NonOptimizing GCC 4.4.5}

\NonOptimizing GCC is more verbose:

\lstinputlisting[caption=\NonOptimizing GCC 4.4.5 (\assemblyOutput),style=customasmMIPS]{patterns/03_printf/MIPS/printf8.O0_EN.s}

\lstinputlisting[caption=\NonOptimizing GCC 4.4.5 (IDA),style=customasmMIPS]{patterns/03_printf/MIPS/printf8.O0.IDA_EN.lst}

}
\RU{\subsection{MIPS}

\subsubsection{3 аргумента}

\myparagraph{\Optimizing GCC 4.4.5}

Главное отличие от примера \q{\HelloWorldSectionName} в том, что здесь на самом деле
вызывается \printf вместо \puts и ещё три аргумента передаются в регистрах  \$5\dots \$7 (или \$A0\dots \$A2).
Вот почему эти регистры имеют префикс A-. Это значит, что они используются для передачи аргументов.

\lstinputlisting[caption=\Optimizing GCC 4.4.5 (\assemblyOutput),style=customasmMIPS]{patterns/03_printf/MIPS/printf3.O3_RU.s}

\lstinputlisting[caption=\Optimizing GCC 4.4.5 (IDA),style=customasmMIPS]{patterns/03_printf/MIPS/printf3.O3.IDA_RU.lst}

\IDA объединила пару инструкций \INS{LUI} и \INS{ADDIU} в одну псевдоинструкцию \INS{LA}.
Вот почему здесь нет инструкции по адресу 0x1C: потому что \INS{LA} \IT{занимает} 8 байт.

\myparagraph{\NonOptimizing GCC 4.4.5}

\NonOptimizing GCC более многословен:

\lstinputlisting[caption=\NonOptimizing GCC 4.4.5 (\assemblyOutput),style=customasmMIPS]{patterns/03_printf/MIPS/printf3.O0_RU.s}

\lstinputlisting[caption=\NonOptimizing GCC 4.4.5 (IDA),style=customasmMIPS]{patterns/03_printf/MIPS/printf3.O0.IDA_RU.lst}

\subsubsection{8 аргументов}

Снова воспользуемся примером с 9-ю аргументами из предыдущей секции: \myref{example_printf8_x64}.

\lstinputlisting[style=customc]{patterns/03_printf/2.c}

\myparagraph{\Optimizing GCC 4.4.5}

Только 4 первых аргумента передаются в регистрах \$A0 \dots \$A3, так что остальные передаются через стек.

\myindex{MIPS!O32}
Это соглашение о вызовах O32 (самое популярное в мире MIPS).
Другие соглашения о вызовах (например N32) могут наделять регистры другими функциями.

\myindex{MIPS!\Instructions!SW}
\INS{SW} означает \q{Store Word} (записать слово из регистра в память).
В MIPS нет инструкции для записи значения в память, так что для этого используется пара инструкций (\INS{LI}/\INS{SW}).

\lstinputlisting[caption=\Optimizing GCC 4.4.5 (\assemblyOutput),style=customasmMIPS]{patterns/03_printf/MIPS/printf8.O3_RU.s}

\lstinputlisting[caption=\Optimizing GCC 4.4.5 (IDA),style=customasmMIPS]{patterns/03_printf/MIPS/printf8.O3.IDA_RU.lst}

\myparagraph{\NonOptimizing GCC 4.4.5}

\NonOptimizing GCC более многословен:

\lstinputlisting[caption=\NonOptimizing GCC 4.4.5 (\assemblyOutput),style=customasmMIPS]{patterns/03_printf/MIPS/printf8.O0_RU.s}

\lstinputlisting[caption=\NonOptimizing GCC 4.4.5 (IDA),style=customasmMIPS]{patterns/03_printf/MIPS/printf8.O0.IDA_RU.lst}

}
\ITA{\subsection{MIPS}

\subsubsection{3 argomenti}

\myparagraph{\Optimizing GCC 4.4.5}

La differenza principale con l'esempio \q{\HelloWorldSectionName} e' che in questo caso \printf e' chiamata 
al posto di \puts, e 3 argomenti aggiuntivi sono passati attraverso i registri \$5\dots \$7 (o \$A0\dots \$A2).
Questo e' il motivo per cui questi registri hanno il prefisso A-, che implica il loro uso per il passaggio di argomenti di funzioni.

% TODO translate to Italian:
\lstinputlisting[caption=\Optimizing GCC 4.4.5 (\assemblyOutput),style=customasmMIPS]{patterns/03_printf/MIPS/printf3.O3_EN.s}

% TODO translate to Italian:
\lstinputlisting[caption=\Optimizing GCC 4.4.5 (IDA),style=customasmMIPS]{patterns/03_printf/MIPS/printf3.O3.IDA_EN.lst}

\IDA ha fuso le coppie di istruzioni \INS{LUI} e \INS{ADDIU} in una unica pseudoistruzione \INS{LA}.
Questo e' il motivo per cui non c'e' nessuna istruzione all'indirizzo 0x1C: perche' \INS{LA} \IT{occupa} 8 byte.%

\myparagraph{\NonOptimizing GCC 4.4.5}

\NonOptimizing GCC e' piu' verboso:

% TODO translate to Italian:
\lstinputlisting[caption=\NonOptimizing GCC 4.4.5 (\assemblyOutput),style=customasmMIPS]{patterns/03_printf/MIPS/printf3.O0_EN.s}

% TODO translate to Italian:
\lstinputlisting[caption=\NonOptimizing GCC 4.4.5 (IDA),style=customasmMIPS]{patterns/03_printf/MIPS/printf3.O0.IDA_EN.lst}

\subsubsection{8 argomenti}

Usiamo nuovamente l'esempio con 9 argomenti dalla sezione prcedente: \myref{example_printf8_x64}.

\lstinputlisting[style=customc]{patterns/03_printf/2.c}

\myparagraph{\Optimizing GCC 4.4.5}

Solo i primi 4 argomenti sono passati nei registri \$A0 \dots \$A3, gli altri sono passati tramite lo stack.
\myindex{MIPS!O32}

Questa e' la calling convention O32 (che e' la piu' comune nel mondo MIPS).
Altre calling conventions (come N32) possono usare i registri per scopi diversi.

\myindex{MIPS!\Instructions!SW}

\INS{SW} e' l'abbreviazione di \q{Store Word} (da un registro alla memoria).
MIPS manca di istruzioni per memorizzare un valore in memoria, e' quindi necessario usare una coppia di istruzioni (LI/SW).

% TODO translate to Italian:
\lstinputlisting[caption=\Optimizing GCC 4.4.5 (\assemblyOutput),style=customasmMIPS]{patterns/03_printf/MIPS/printf8.O3_EN.s}

% TODO translate to Italian:
\lstinputlisting[caption=\Optimizing GCC 4.4.5 (IDA),style=customasmMIPS]{patterns/03_printf/MIPS/printf8.O3.IDA_EN.lst}

\myparagraph{\NonOptimizing GCC 4.4.5}

\NonOptimizing GCC e' piu' verboso:

% TODO translate to Italian:
\lstinputlisting[caption=\NonOptimizing GCC 4.4.5 (\assemblyOutput),style=customasmMIPS]{patterns/03_printf/MIPS/printf8.O0_EN.s}

% TODO translate to Italian:
\lstinputlisting[caption=\NonOptimizing GCC 4.4.5 (IDA),style=customasmMIPS]{patterns/03_printf/MIPS/printf8.O0.IDA_EN.lst}
}
\FR{\subsection{MIPS}

\subsubsection{3 arguments}

\myparagraph{GCC 4.4.5 \Optimizing}

La différence principale avec l'exemple \q{\HelloWorldSectionName} est que dans ce cas, \printf est appelée
à la place de \puts et 3 arguments de plus sont passés à travers les registres \$5\dots \$7 (ou \$A0\dots \$A2).
C'est pourquoi ces registres sont préfixés avec A-, ceci sous-entend qu'ils
sont utilisés pour le passage des arguments aux fonctions.

\lstinputlisting[caption=GCC 4.4.5 \Optimizing (\assemblyOutput),style=customasmMIPS]{patterns/03_printf/MIPS/printf3.O3_FR.s}

\lstinputlisting[caption=GCC 4.4.5 \Optimizing (IDA),style=customasmMIPS]{patterns/03_printf/MIPS/printf3.O3.IDA_FR.lst}

\IDA a concaténé la paire d'instructions \INS{LUI} et \INS{ADDIU} en une pseudo instruction \INS{LA}.
C'est pourquoi il n'y a pas d'instruction à l'adresse 0x1C: car \INS{LA} \IT{occupe} 8 octets.

\myparagraph{GCC 4.4.5 \NonOptimizing}

GCC \NonOptimizing est plus verbeux:

\lstinputlisting[caption=GCC 4.4.5 \NonOptimizing (\assemblyOutput),style=customasmMIPS]{patterns/03_printf/MIPS/printf3.O0_FR.s}

\lstinputlisting[caption=GCC 4.4.5 \NonOptimizing (IDA),style=customasmMIPS]{patterns/03_printf/MIPS/printf3.O0.IDA_FR.lst}

\subsubsection{8 arguments}

Utilisons encore l'exemple de la section précédente avec 9 arguments: \myref{example_printf8_x64}.

\lstinputlisting[style=customc]{patterns/03_printf/2.c}

\myparagraph{GCC 4.4.5 \Optimizing}

Seul les 4 premiers arguments sont passés dans les registres \$A0 \dots \$A3,
les autres sont passés par la pile.
\myindex{MIPS!O32}

C'est la convention d'appel O32 (qui est la plus commune dans le monde MIPS).
D'autres conventions d'appel (comme N32) peuvent utiliser les registres à d'autres fins.

\myindex{MIPS!\Instructions!SW}

\INS{SW} est l'abbréviation de \q{Store Word} (depuis un registre vers la mémoire).
En MIPS, il manque une instructions pour stocker une valeur dans la mémoire, donc
une paire d'instruction doit être utilisée à la place (\INS{LI}/\INS{SW}).

\lstinputlisting[caption=GCC 4.4.5 \Optimizing (\assemblyOutput),style=customasmMIPS]{patterns/03_printf/MIPS/printf8.O3_FR.s}

\lstinputlisting[caption=GCC 4.4.5 \Optimizing (IDA),style=customasmMIPS]{patterns/03_printf/MIPS/printf8.O3.IDA_FR.lst}

\myparagraph{GCC 4.4.5 \NonOptimizing}

GCC \NonOptimizing est plus verbeux:

\lstinputlisting[caption=\NonOptimizing GCC 4.4.5 (\assemblyOutput),style=customasmMIPS]{patterns/03_printf/MIPS/printf8.O0_FR.s}

\lstinputlisting[caption=\NonOptimizing GCC 4.4.5 (IDA),style=customasmMIPS]{patterns/03_printf/MIPS/printf8.O0.IDA_FR.lst}

}


\subsection{\Conclusion{}}

Aqui está uma estrutura bem rústica da chamada da função

% TODO to be translated to PTBR:
\begin{lstlisting}[caption=x86,style=customasmx86]
...
PUSH 3rd argument
PUSH 2nd argument
PUSH 1st argument
CALL function
; modify stack pointer (if needed)
\end{lstlisting}

\begin{lstlisting}[caption=x64 (MSVC),style=customasmx86]
MOV RCX, 1st argument
MOV RDX, 2nd argument
MOV R8, 3rd argument
MOV R9, 4th argument
...
PUSH 5th, 6th argument, etc. (if needed)
CALL function
; modify stack pointer (if needed)
\end{lstlisting}

\begin{lstlisting}[caption=x64 (GCC),style=customasmx86]
MOV RDI, 1st argument
MOV RSI, 2nd argument
MOV RDX, 3rd argument
MOV RCX, 4th argument
MOV R8, 5th argument
MOV R9, 6th argument
...
PUSH 7th, 8th argument, etc. (if needed)
CALL function
; modify stack pointer (if needed)
\end{lstlisting}

\begin{lstlisting}[caption=ARM,style=customasmARM]
MOV R0, 1st argument
MOV R1, 2nd argument
MOV R2, 3rd argument
MOV R3, 4th argument
; pass 5th, 6th argument, etc., in stack (if needed)
BL function
; modify stack pointer (if needed)
\end{lstlisting}

\begin{lstlisting}[caption=ARM64,style=customasmARM]
MOV X0, 1st argument
MOV X1, 2nd argument
MOV X2, 3rd argument
MOV X3, 4th argument
MOV X4, 5th argument
MOV X5, 6th argument
MOV X6, 7th argument
MOV X7, 8th argument
; pass 9th, 10th argument, etc., in stack (if needed)
BL function
; modify stack pointer (if needed)
\end{lstlisting}

\myindex{MIPS!O32}
\begin{lstlisting}[caption=MIPS (O32 calling convention),style=customasmMIPS]
LI $4, 1st argument ; AKA $A0
LI $5, 2nd argument ; AKA $A1
LI $6, 3rd argument ; AKA $A2
LI $7, 4th argument ; AKA $A3
; pass 5th, 6th argument, etc., in stack (if needed)
LW temp_reg, address of function
JALR temp_reg
\end{lstlisting}

\subsection{A propósito}

\myindex{fastcall}
A propósito, a diferença entre os argumentos passados em x86, x64, fastcall, ARM e MIPS  é uma boa demonstração do fato de como a CPU é indiferente sobre como os argumentos são passados para as funções.
Também é possível criar um compilador hipotético capaz de passar argumentos por alguma outra estrutura especial sem usar a pilha de nenhuma maneira.

\myindex{MIPS!O32}
\PTBRph{}

A \ac{CPU} não está ciente de convenções de chamada de funções.

Agora nós podemos também relembrar de dos programadores novatos de assembly passando argumentos para outras funções:
geralmente via registradores, sem nenhuma sequência explícita, ou mesmo por variáveis globais. Logicamente, também funciona.

