\subsubsection{ARM: 3 arguments}

Lo schema tradizionale per il passaggio di argomenti (calling convention) di ARM si comporta in questo modo:
i primi 4 argomenti vengono passati attraverso i registri \Reg{0}-\Reg{3} , i restanti attraverso lo stack.
Cio' ricorda molto il metodo per il passaggio di argomenti in 
fastcall~(\myref{fastcall}) o win64~(\myref{sec:callingconventions_win64}).

\myparagraph{32-bit ARM}

\mysubparagraph{\NonOptimizingKeilVI (\ARMMode)}

\begin{lstlisting}[caption=\NonOptimizingKeilVI (\ARMMode),style=customasmARM]
.text:00000000 main
.text:00000000 10 40 2D E9   STMFD   SP!, {R4,LR}
.text:00000004 03 30 A0 E3   MOV     R3, #3
.text:00000008 02 20 A0 E3   MOV     R2, #2
.text:0000000C 01 10 A0 E3   MOV     R1, #1
.text:00000010 08 00 8F E2   ADR     R0, aADBDCD     ; "a=%d; b=%d; c=%d"
.text:00000014 06 00 00 EB   BL      __2printf
.text:00000018 00 00 A0 E3   MOV     R0, #0          ; return 0
.text:0000001C 10 80 BD E8   LDMFD   SP!, {R4,PC}
\end{lstlisting}

I primi 4 argomenti sono quindi passati attraverso i registri \Reg{0}-\Reg{3} nel seguente ordine:
un puntatore alla format string di \printf in 
\Reg{0}, 1 in \Reg{1}, 2 in \Reg{2} e 3 in \Reg{3}.
L'istruzione a \GTT{0x18} scrive 0 in \Reg{0}---questo equivale allo statement C \IT{return 0}.
Niente di insolito fino a qui.

\OptimizingKeilVI genera lo stesso codice.

\mysubparagraph{\OptimizingKeilVI (\ThumbMode)}

\begin{lstlisting}[caption=\OptimizingKeilVI (\ThumbMode),style=customasmARM]
.text:00000000 main
.text:00000000 10 B5        PUSH    {R4,LR}
.text:00000002 03 23        MOVS    R3, #3
.text:00000004 02 22        MOVS    R2, #2
.text:00000006 01 21        MOVS    R1, #1
.text:00000008 02 A0        ADR     R0, aADBDCD     ; "a=%d; b=%d; c=%d"
.text:0000000A 00 F0 0D F8  BL      __2printf
.text:0000000E 00 20        MOVS    R0, #0
.text:00000010 10 BD        POP     {R4,PC}
\end{lstlisting}

Non c'e' nessuna differenza significativa nel codice non ottimizzato per modo ARM.

\mysubparagraph{\OptimizingKeilVI (\ARMMode) + rimozione di return}
\label{ARM_B_to_printf}

Modifichiamo leggermente l'esempio rimuovendo \IT{return 0}:

\begin{lstlisting}[style=customc]
#include <stdio.h>

void main()
{
	printf("a=%d; b=%d; c=%d", 1, 2, 3);
};
\end{lstlisting}

Il risultato e' alquanto insolito:

\begin{lstlisting}[caption=\OptimizingKeilVI (\ARMMode),style=customasmARM]
.text:00000014 main
.text:00000014 03 30 A0 E3   MOV     R3, #3
.text:00000018 02 20 A0 E3   MOV     R2, #2
.text:0000001C 01 10 A0 E3   MOV     R1, #1
.text:00000020 1E 0E 8F E2   ADR     R0, aADBDCD     ; "a=%d; b=%d; c=%d\n"
.text:00000024 CB 18 00 EA   B       __2printf
\end{lstlisting}

\myindex{ARM!\Registers!Link Register}
\myindex{ARM!\Instructions!B}
\myindex{Function epilogue}
Questa e' la versione ottimizzata (\Othree) per ARM mode e stavolta notiamo \INS{B} come ultima istruzione, al posto della familiare \INS{BL}.
Un'altra differenza tra questa versione ottimizzata e la precedente (compilata senza ottimizzazione)
e' la mancanza di prologo ed epilogo della funzione (le istruzioni che preservano i valori dei registri \Reg{0} e \ac{LR}).
\myindex{x86!\Instructions!JMP}
L'istruzione \INS{B} salta semplicemente ad un altro indirizzo, senza alcuna manipolazione del registro \ac{LR}, in modo simile
a \JMP in x86.
Perche' funziona? Perche' questo codice e' infatti equivalente al precedente.
Principalmente per due motivi: 1) ne' lo stack ne' \ac{SP} (lo \gls{stack pointer}) vengono modificati;
2) la chiamata a \printf e' l'ultima istruzionie, quindi non succede niente dopo di essa.
Al completamento, \printf restituisce semplicemente il controllo all'indirizzo memorizzato in \ac{LR}.
Poiche' \ac{LR} attualmente contiene l'indirizzo del punto da cui la nostra funzione era stata chiamata, 
il controllo verra' restituito da \printf a quello stesso punto.
Pertanto non c'e' alcun bisogno di salvare \ac{LR} in quanto non abbiamo necessita' di modificare \ac{LR}.
E non vogliamo affatto modificare \ac{LR} poiche' non ci sono altre chiamate a funzione ad eccezzione di \printf. Inoltre, dopo
questa chiamata non abbiamo nient'altro da fare!
Queste le ragione per cui una simile ottimizzazione e' possibile.

Questa ottimizzazione e' spesso usata in funzioni in cui l'ultimo statement e' una chiamata ad un'altra funzione.
Un esempio simile e' fornito di seguito:
\myref{jump_to_last_printf}.

\myparagraph{ARM64}

\mysubparagraph{\NonOptimizing GCC (Linaro) 4.9}

% TODO translate to Italian:
\lstinputlisting[caption=\NonOptimizing GCC (Linaro) 4.9,style=customasmARM]{patterns/03_printf/ARM/ARM3_O0_EN.lst}

\myindex{ARM!\Instructions!STP}

La prima istruzione \INS{STP} (\IT{Store Pair}) salva \ac{FP} (X29) e \ac{LR} (X30) nello stack.
La seconda istruzione \INS{ADD X29, SP, 0} forma lo stack frame.
Scrive semplicemente il valore di \ac{SP} in X29.

\myindex{ARM!\Instructions!ADRP/ADD pair}
Successivamente vediamo la familiare coppia di istruzioni \INS{ADRP}/\ADD , che forma un puntatore alla stringa.
\myindex{ARM64!lo12}
\IT{lo12} indica 12 bit bassi (low 12 bits), ovvero, il linker scrivera' i 12 bit bassi dell'indirizzo di LC1 nell'opcode dell'istruzione \ADD .
\GTT{\%d} nella format string di \printf e' un \Tint a 32-bit, quindi u valori 1, 2 e 3 sono caricati nelle parti a 32-bit dei registri.

\Optimizing GCC (Linaro) 4.9 genera lo stesso codice.
