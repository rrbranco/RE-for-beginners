\section{Mehr zu Rückgabewerten}

\myindex{x86!\Registers!EAX}
In x86 wird das Ergebnis einer Funktionsausführung\footnote{\Seealso: MSDN: Return Values (C++):
\href{http://go.yurichev.com/17258}{MSDN}} normalerweise über das \EAX Register zurückgegeben.
Wenn es sich um ein Byte oder einen \Tchar handelt, dann wird der niedere Teil des Registers \EAX (\AL) verwendet.
Wenn eine Funktion eine \Tfloat Zahl zurückgibt, wird das FPU Register \ST{0} stattdessen verwendet.

\myindex{ARM!\Registers!R0}
In ARM wird das Ergebnis üblicherweise über das \Reg{0} Register zurückgegeben.

\subsection{Versuch einen Rückgabewert vom Typ \Tvoid zu verwenden}
Was würde passieren, wenn der Rückgabewert der Funktion \main vom Typ \Tvoid und nicht \Tint wäre?
Der sogenannte Startup-Code ruft \main in etwa wie folgt auf:

\begin{lstlisting}[style=customasmx86]
push envp
push argv
push argc
call main
push eax
call exit
\end{lstlisting}

Mit anderen Worten:

\begin{lstlisting}[style=customc]
exit(main(argc,argv,envp));
\end{lstlisting}
Wenn man \main als \Tvoid deklariert muss nichts explizit zurückgegeben werden (mit dem \IT{return} Ausdruck). In diesem
Fall wird etwas Zufälliges, das am Ende der Ausführung von \main in \EAX steht, das einzige Argument der exit()
Funktion.
Höchstwahrscheinlich wird es sich um einen Zufallswert handelt, der von der Ausführung der Funktion dort belassen wurde,
sodass der Exitcode des Programms pseudozufällig ist.
\par
Wir veranschaulichen diese Tatsache. 
Beachten Sie, dass die Funktion \main hier den Rückgabetyp \Tvoid hat: 

\begin{lstlisting}[style=customc]
#include <stdio.h>

void main()
{
	printf ("Hello, world!\n");
};
\end{lstlisting}

Kompilieren wir es in Linux.

\myindex{puts() instead of printf()}
GCC 4.8.1 hat \printf durch \puts ersetzt. 
(wir haben dies vorher gesehen: \myref{puts}), aber das ist in Ordnung,
denn \puts liefert die Anzahl der ausgegebenen Zeichen zurück, genau wie \printf.
Man beachte, dass \EAX vor dem Ende von \main nicht geleert wird.

Das bedeutet, dass \EAX am Ende von \main den Wert enthält, den \puts dort hinterlassen hat.

\begin{lstlisting}[caption=GCC 4.8.1,style=customasmx86]
.LC0:
	.string	"Hello, world!"
main:
	push	ebp
	mov	ebp, esp
	and	esp, -16
	sub	esp, 16
	mov	DWORD PTR [esp], OFFSET FLAT:.LC0
	call	puts
	leave
	ret
\end{lstlisting}

\myindex{bash}

Schreiben wir ein Bash Skript, das den Exitcode anzeigt:

\begin{lstlisting}[caption=tst.sh]
#!/bin/sh
./hello_world
echo $?
\end{lstlisting}

Führen wir es aus:

\begin{lstlisting}
$ tst.sh 
Hello, world!
14
\end{lstlisting}

14 ist Anzahl der ausgegebenen Zeichen.
\ac{TBT}

\myindex{Hex-Rays}
Wenn wir übrigens C++ in Hex-Rays dekompilieren, stoßen wir häufig auf eine Funktion, die mit dem Destruktor einer
Klasse endet:

\begin{lstlisting}[style=customasmx86]
...

call    ??1CString@@QAE@XZ ; CString::~CString(void)
mov     ecx, [esp+30h+var_C]
pop     edi
pop     ebx
mov     large fs:0, ecx
add     esp, 28h
retn
\end{lstlisting}
Gemäß C++ Standard gibt der Destruktor nichts zurück, aber wenn Hex-Rays dies nicht erkennt, und davon ausgeht, dass
sowohl Destruktor und diese Funktion \Tint zurückgeben, finden wir so etwas wie das Folgende im Output:

\begin{lstlisting}[style=customc]
...

	return CString::~CString(&Str);
}
\end{lstlisting}

\subsection{Was, wenn wir das Funktionsergebnis nicht verwenden?}
\printf gibt die Anzahl der erfolgreich ausgegebenen Zeichen zurück, aber das Ergebnis dieser Funktion wird in der
Praxis kaum verwendet.

Es ist also möglich eine Funktion aufzurufen, die einen Wert zurückgibt, diesen Wert aber nicht zu verwenden:


\begin{lstlisting}[style=customc]
int f()
{
    // skip first 3 random values:
    rand();
    rand();
    rand();
    // and use 4th:
    return rand();
};
\end{lstlisting}
Das Ergebnis der Funktion rand() bleibt in allen vier Fällen in \EAX.

In den ersten drei Fällen wird der Wert in \EAX aber nicht verwendet.

\subsection{Eine Struktur zurückgeben}

\myindex{\CLanguageElements!return}
Gehen wir nochmals darauf ein, dass der Rückgabewert im \EAX Register verbleibt.

Dies ist der Grund dafür, dass alte C Kompiler keine Funktion erzeugen konnten, die einen Wert zurückgeben, der nicht in
ein Register passt (normalerweise \Tint), aber falls es erforderlich ist, kann man Informationen über Pointer, die der
Funktion als Argumente übergeben wurden, zurückgeben.

Wenn also eine Funktion mehrere Werte zurückgeben soll, gibt sie normalerweise nur einen zurück und den Rest über
Pointer.

Nun ist es möglich geworden beispielsweise eine komplette Struktur zurückzugeben, aber die Möglichkeit wird nicht
besonders häufig genutzt. Wenn eine Funktion eine große Struktur zurückgeben muss, muss der Aufrufer Speicher hierfür
reservieren und einen Pointer darauf als erstes Argument übergeben.
Dies ist das fast das gleiche wie einen Pointer manuell in das erste Argument zu übergeben, aber der Kompiler verbirgt
es.

Ein kleines Beispiel:

\lstinputlisting[style=customc]{patterns/06_return_results/6_1.c}

\dots was wir erhalten, ist (MSVC 2010 \Ox):

\lstinputlisting[style=customasmx86]{patterns/06_return_results/6_1.asm}

Der Name des Makros um intern einen Pointer auf eine Struktur zu übergeben, ist hier \GTT{\$T3853}.

\myindex{\CLanguageElements!C99}
Dieses Beispiel kann mithilfe der C99 Spracherweiterungen umgeschrieben werden:

\lstinputlisting[style=customc]{patterns/06_return_results/6_1_C99.c}

\lstinputlisting[caption=GCC 4.8.1,style=customasmx86]{patterns/06_return_results/6_1_C99.asm}
Wie wir sehen, füllt die Funktion nur die Felder der Struktur, die durch die aufrufende Funktion angelegt wurden als ob
ein Pointer auf die Struktur übergeben worden wäre. Es gibt an dieser Stelle also keine Nachteile bezüglich Performance.
