\subsection{Ternärer Vergleichsoperator}
\label{chap:cond}
Der ternäre Vergleichsoperator in \CCpp hat die folgende Syntax:

\begin{lstlisting}
expression ? expression : expression
\end{lstlisting}

Hier ist ein Beispiel:

\lstinputlisting[style=customc]{patterns/07_jcc/cond_operator/cond.c}

\subsubsection{x86}
Alte und nicht optimerende Compiler erzeugen Assemblercode als wenn ein \TT{if/else} Ausdruck verwendet wurde:

\lstinputlisting[caption=\NonOptimizing MSVC 2008,style=customasmx86]{patterns/07_jcc/cond_operator/MSVC2008_DE.asm}

\lstinputlisting[caption=\Optimizing MSVC 2008,style=customasmx86]{patterns/07_jcc/cond_operator/MSVC2008_Ox_DE.asm}

Neuere Compiler sind ein wenig präziser:

\lstinputlisting[caption=\Optimizing MSVC 2012
x64,style=customasmx86]{patterns/07_jcc/cond_operator/MSVC2012_Ox_x64_DE.asm}

\myindex{x86!\Instructions!CMOVcc}
\Optimizing GCC 4.8 für x86 verwendet ebenfalls den \TT{CMOVcc} Befehl, wohingegen der nicht optimierende GCC 4.8
bedingte Sprünge verwendet.

\subsubsection{ARM}

\myindex{x86!\Instructions!ADRcc}
\Optimizing Keil im ARM mode verwendet ebenfalls bedingte Sprungbefehle \TT{ADRcc}:

\lstinputlisting[label=cond_Keil_ARM_O3,caption=\OptimizingKeilVI
(\ARMMode),style=customasmARM]{patterns/07_jcc/cond_operator/Keil_ARM_O3_DE.s}

Ohne manuellen Eingriff können die beiden Befehle \TT{ADREQ} und \TT{ADRNE} nicht in einem Durchlauf ausgeführt werden.

\Optimizing Keil für Thumb mode muss bedingte Sprungbefehle verwenden, da es keine Ladebefehle gibt, die
Bedingungsflags unterstützen.

\lstinputlisting[caption=\OptimizingKeilVI
(\ThumbMode),style=customasmARM]{patterns/07_jcc/cond_operator/Keil_thumb_O3_DE.s}

\subsubsection{ARM64}

\Optimizing GCC (Linaro) 4.9 für ARM64 verwendet auch bedingte Sprünge:

\lstinputlisting[label=cond_ARM64,caption=\Optimizing GCC (Linaro)
4.9,style=customasmARM]{patterns/07_jcc/cond_operator/ARM64_GCC_O3_DE.s} Das liegt daran, dass ARM64 über keinen
einfachen Ladebefehl mit Bedingungsflags verfügt wie z.B. \TT{ADRcc} im 32-Bit-ARM-Modus oder \INS{CMOVcc} in x86.

\myindex{ARM!\Instructions!CSEL}
Es gibt dafür den \q{Conditional SELect} Befehl (\TT{CSEL})\InSqBrackets{\ARMSixFourRef p390, C5.5}, aber GCC 4.9
scheint nicht ausgereift genug zu sein um ihn in einem solchen Codestück zu verwenden.

\subsubsection{MIPS}

Leider ist GCC 4.4.5 für MIPS auch nicht besser:

\lstinputlisting[caption=\Optimizing GCC 4.4.5
(\assemblyOutput),style=customasmMIPS]{patterns/07_jcc/cond_operator/MIPS_O3_DE.s}

\subsubsection{Schreiben wir es mit \TT{if/else}}

\lstinputlisting[style=customc]{patterns/07_jcc/cond_operator/cond2.c}

\myindex{x86!\Instructions!CMOVcc}
Interessanterweise war der optimierende GCC 4.8 für x86 ebenfalls in der Lage \TT{CMOVcc} hier zu verwenden:

\lstinputlisting[caption=\Optimizing GCC 4.8,style=customasmx86]{patterns/07_jcc/cond_operator/cond2_GCC_O3_DE.s}

\Optimizing Keil im ARM mode erzeugt identischen Code zu \lstref{cond_Keil_ARM_O3}.
Der optimierende MSVC 2012 ist hingegen (noch) nicht so gut.


\subsubsection{\Conclusion{}}
Warum versuchen optimierende Compiler bedingte Sprünge zu entfernen? Mehr dazu finden Sie hier:
\myref{branch_predictors}.
