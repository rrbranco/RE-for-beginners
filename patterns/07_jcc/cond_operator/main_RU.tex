\subsection{Тернарный условный оператор}
\label{chap:cond}

Тернарный условный оператор (ternary conditional operator) в \CCpp это:

\begin{lstlisting}
expression ? expression : expression
\end{lstlisting}

И вот пример:

\lstinputlisting[style=customc]{patterns/07_jcc/cond_operator/cond.c}

\subsubsection{x86}

Старые и неоптимизирующие компиляторы генерируют код так, как если бы выражение \TT{if/else} было использовано
вместо него:

\lstinputlisting[caption=\NonOptimizing MSVC 2008,style=customasmx86]{patterns/07_jcc/cond_operator/MSVC2008_RU.asm}

\lstinputlisting[caption=\Optimizing MSVC 2008,style=customasmx86]{patterns/07_jcc/cond_operator/MSVC2008_Ox_RU.asm}

Новые компиляторы могут быть более краткими:

\lstinputlisting[caption=\Optimizing MSVC 2012 x64,style=customasmx86]{patterns/07_jcc/cond_operator/MSVC2012_Ox_x64_RU.asm}

\myindex{x86!\Instructions!CMOVcc}
\Optimizing GCC 4.8 для x86 также использует инструкцию \TT{CMOVcc},
тогда как неоптимизирующий GCC 4.8 использует условные переходы.

\subsubsection{ARM}

\myindex{x86!\Instructions!ADRcc}
\Optimizing Keil для режима ARM также использует инструкцию \INS{ADRcc}, срабатывающую при некотором
условии:

\lstinputlisting[label=cond_Keil_ARM_O3,caption=\OptimizingKeilVI (\ARMMode),style=customasmARM]{patterns/07_jcc/cond_operator/Keil_ARM_O3_RU.s}

Без внешнего вмешательства инструкции \TT{ADREQ} и \TT{ADRNE} никогда не исполнятся одновременно.
\Optimizing Keil для режима Thumb вынужден использовать инструкции условного перехода, потому
что тут нет инструкции загрузки значения, поддерживающей флаги условия:

\lstinputlisting[caption=\OptimizingKeilVI (\ThumbMode),style=customasmARM]{patterns/07_jcc/cond_operator/Keil_thumb_O3_RU.s}

\subsubsection{ARM64}

\Optimizing GCC (Linaro) 4.9 для ARM64 также использует условные переходы:

\lstinputlisting[label=cond_ARM64,caption=\Optimizing GCC (Linaro) 4.9,style=customasmARM]{patterns/07_jcc/cond_operator/ARM64_GCC_O3_RU.s}

Это потому что в ARM64 нет простой инструкции загрузки с флагами условия, как \TT{ADRcc} в 32-битном 
режиме ARM или \TT{CMOVcc} в x86.

\myindex{ARM!\Instructions!CSEL}
Но с другой стороны, там есть инструкция \TT{CSEL} (\q{Conditional SELect})
\InSqBrackets{\ARMSixFourRef p390, C5.5},
но GCC 4.9 наверное, пока не так
хорош, чтобы генерировать её в таком фрагменте кода

\subsubsection{MIPS}

GCC 4.4.5 для MIPS тоже не так хорош, к сожалению:

\lstinputlisting[caption=\Optimizing GCC 4.4.5 (\assemblyOutput),style=customasmMIPS]{patterns/07_jcc/cond_operator/MIPS_O3_RU.s}

\subsubsection{Перепишем, используя обычный \TT{if/else}}

\lstinputlisting[style=customc]{patterns/07_jcc/cond_operator/cond2.c}

\myindex{x86!\Instructions!CMOVcc}
Интересно, оптимизирующий GCC 4.8 для x86 также может генерировать \TT{CMOVcc} в этом случае:

\lstinputlisting[caption=\Optimizing GCC 4.8,style=customasmx86]{patterns/07_jcc/cond_operator/cond2_GCC_O3_RU.s}

\Optimizing Keil в режиме ARM генерирует код идентичный этому: \lstref{cond_Keil_ARM_O3}.

Но оптимизирующий MSVC 2012 пока не так хорош.

\subsubsection{\Conclusion{}}

Почему оптимизирующие компиляторы стараются избавиться от условных переходов? Читайте больше об этом здесь:
 \myref{branch_predictors}.
