\subsection{Betrag berechnen}
\label{sec:abs}

Eine einfache Funktion:

\lstinputlisting[style=customc]{abs.c}

\subsubsection{\Optimizing MSVC}

Normalerweise wird folgender Code erzeugt:

\lstinputlisting[caption=\Optimizing MSVC 2012 x64,style=customasmx86]{patterns/07_jcc/abs/abs_MSVC2012_Ox_x64_DE.asm}

GCC 4.9 macht ungefähr das gleiche.

\subsubsection{\OptimizingKeilVI: \ThumbMode}

\lstinputlisting[caption=\OptimizingKeilVI: \ThumbMode,style=customasmARM]{patterns/07_jcc/abs/abs_Keil_thumb_O3_DE.s}

\myindex{ARM!\Instructions!RSB}
ARM fehlt ein Befehl zur Negation, sodass der Keil Compiler den \q{Reverse
Subtract} Befehl verwendet, der mit umgekehrten Operanden subtrahiert.

\subsubsection{\OptimizingKeilVI: \ARMMode}
Es ist im ARM mode möglich, einigen Befehlen condition codes hinzuzufügen und genau das tut der Keil Compiler:


\lstinputlisting[caption=\OptimizingKeilVI: \ARMMode,style=customasmARM]{patterns/07_jcc/abs/abs_Keil_ARM_O3_DE.s}
Jetzt sind keine bedingten Sprünge mehr übrig und das ist vorteilhaft: \myref{branch_predictors}.


\subsubsection{\NonOptimizing GCC 4.9 (ARM64)}

\myindex{ARM!\Instructions!XOR}

ARM64 kennt den Befehl \INS{NEG} zum Negieren:

\lstinputlisting[caption=\Optimizing GCC 4.9 (ARM64),style=customasmARM]{patterns/07_jcc/abs/abs_GCC49_ARM64_O0_DE.s}

\subsubsection{MIPS}

\lstinputlisting[caption=\Optimizing GCC 4.4.5 (IDA),style=customasmMIPS]{patterns/07_jcc/abs/MIPS_O3_IDA_DE.lst}

\myindex{MIPS!\Instructions!BLTZ}
Hier finden wir einen neuen Befehl: \INS{BLTZ} (\q{Branch if Less Than Zero}).
\myindex{MIPS!\Instructions!SUBU}
\myindex{MIPS!\Pseudoinstructions!NEGU}
Es gibt zusätzlich noch den \INS{NEGU} Pseudo-Befehl, der eine Subtraktion von Null durchführt. Der Suffix \q{U} bei
\INS{SUBU} und \INS{NEGU} zeigt an, dass keine Exception für den Fall eines Integer Overflows geworfen wird.


\subsubsection{Verzweigungslose Version?}
Man kann auch eine verzweigungslose Version dieses Codes erzeugen. Dies werden wir später betrachten:
\myref{chap:branchless_abs}. 
