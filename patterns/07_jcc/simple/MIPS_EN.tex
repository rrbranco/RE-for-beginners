\subsubsection{MIPS}

One distinctive MIPS feature is the absence of flags.
Apparently, it was done to simplify the analysis of data dependencies.

\myindex{x86!\Instructions!SETcc}
\myindex{MIPS!\Instructions!SLT}
\myindex{MIPS!\Instructions!SLTU}

There are instructions similar to \INS{SETcc} in x86: \INS{SLT} (\q{Set on Less Than}: signed version) and 
\INS{SLTU} (unsigned version).
These instructions sets destination register value to 1 if the condition is true or to 0 if otherwise.

\myindex{MIPS!\Instructions!BEQ}
\myindex{MIPS!\Instructions!BNE}

The destination register is then checked using \INS{BEQ} (\q{Branch on Equal}) or \INS{BNE} (\q{Branch on Not Equal}) 
and a jump may occur.
So, this instruction pair has to be used in MIPS for comparison and branch.
Let's first start with the signed version of our function:

\lstinputlisting[caption=\NonOptimizing GCC 4.4.5 (IDA),style=customasmMIPS]{patterns/07_jcc/simple/O0_MIPS_signed_IDA_EN.lst}

\INS{SLT REG0, REG0, REG1} is reduced by IDA to its 
shorter form:\\
\INS{SLT REG0, REG1}.
\myindex{MIPS!\Pseudoinstructions!BEQZ}

We also see there \INS{BEQZ} pseudo instruction (\q{Branch if Equal to Zero}),\\
which are in fact \INS{BEQ REG, \$ZERO, LABEL}.

\myindex{MIPS!\Instructions!SLTU}

The unsigned version is just the same, but \INS{SLTU} (unsigned version, hence \q{U} in name) is used instead of \INS{SLT}:

\lstinputlisting[caption=\NonOptimizing GCC 4.4.5 (IDA),style=customasmMIPS]{patterns/07_jcc/simple/O0_MIPS_unsigned_IDA.lst}

