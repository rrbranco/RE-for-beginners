\myparagraph{32-bit ARM}
\label{subsec:jcc_ARM}

\mysubparagraph{\OptimizingKeilVI (\ARMMode)}

\lstinputlisting[caption=\OptimizingKeilVI (\ARMMode),style=customasmARM]{patterns/07_jcc/simple/ARM/ARM_O3_signed.asm}

\myindex{ARM!Condition codes}
% FIXME \ref -> which instructions?

Many instructions in ARM mode could be executed only when specific flags are set.
E.g. this is often used when comparing numbers.

\myindex{ARM!\Instructions!ADD}
\myindex{ARM!\Instructions!ADDAL}

For instance, the \ADD instruction is in fact named \TT{ADDAL} internally, where \TT{AL} stands for
\IT{Always}, i.e., execute always.
The predicates are encoded in 4 high bits of the 32-bit ARM instructions (\IT{condition field}).
\myindex{ARM!\Instructions!B}
The \TT{B} instruction for unconditional jumping is in fact conditional and encoded just like any other
conditional jump, but has \TT{AL} in the \IT{condition field}, and it implies \IT{execute ALways}, 
ignoring flags.

\myindex{ARM!\Instructions!ADR}
\myindex{ARM!\Instructions!ADRcc}
\myindex{ARM!\Instructions!CMP}

The \TT{ADRGT} instruction works just like \TT{ADR} but executes only in case the previous \CMP
instruction founds one of the numbers greater than the another, while comparing the two (\IT{Greater Than}).

\myindex{ARM!\Instructions!BL}
\myindex{ARM!\Instructions!BLcc}

The next \TT{BLGT} instruction behaves exactly as \TT{BL} 
and is triggered only if the result of the comparison has been (\IT{Greater Than}). 
\TT{ADRGT} writes a pointer to the string \TT{a>b\textbackslash{}n} into \Reg{0} and \TT{BLGT} calls \printf.
Therefore, instructions suffixed with \TT{-GT} are to execute only in case the value in \Reg{0} (which is $a$) is bigger than the value in \Reg{4} (which is $b$).

\myindex{ARM!\Instructions!ADRcc}
\myindex{ARM!\Instructions!BLcc}

Moving forward we see the \TT{ADREQ} and \TT{BLEQ} instructions.
They behave just like \TT{ADR} and \TT{BL}, but are to be executed only if operands were equal to each
other during the last comparison.
Another \CMP is located before them (because the \printf execution may have tampered the flags).

\myindex{ARM!\Instructions!LDMccFD}
\myindex{ARM!\Instructions!LDMFD}

Then we see \TT{LDMGEFD}, this instruction works just like \TT{LDMFD}\footnote{\ac{LDMFD}},
but is triggered only when one of the values is greater or equal than the other (\IT{Greater or Equal}).
The \TT{LDMGEFD SP!, \{R4-R6,PC\}} instruction acts like a function epilogue, but it will be triggered only if $a>=b$, and only then the function execution will finish.

\myindex{Function epilogue}

But if that condition is not satisfied, i.e., $a<b$, then the control flow will continue to the next \\
\TT{\q{LDMFD SP!, \{R4-R6,LR\}}} instruction, which is one more function epilogue. This instruction restores not only the \TT{R4-R6} registers state, but also \ac{LR} instead of \ac{PC}, thus, it does not return from the function.
The last two instructions call \printf with the string <<a<b\textbackslash{}n>> as a sole argument.
We already examined an unconditional jump to the \printf function instead of function return in <<\PrintfSeveralArgumentsSectionName>> section~(\myref{ARM_B_to_printf}).

\myindex{ARM!\Instructions!ADRcc}
\myindex{ARM!\Instructions!BLcc}
\myindex{ARM!\Instructions!LDMccFD}
\TT{f\_unsigned} is similar, only the \TT{ADRHI}, \TT{BLHI}, and \TT{LDMCSFD} instructions are used there, these predicates (\IT{HI = Unsigned higher, CS = Carry Set (greater than or equal)}) are analogous to those examined before, but for unsigned values.

There is not much new in the \main function for us:

\lstinputlisting[caption=\main,style=customasmARM]{patterns/07_jcc/simple/ARM/ARM_O3_main.asm}

That is how you can get rid of conditional jumps in ARM mode.

\myindex{RISC pipeline}
Why is this so good? Read here: \myref{branch_predictors}.

\myindex{x86!\Instructions!CMOVcc}

There is no such feature in x86, except the \TT{CMOVcc} instruction, it is the same as \MOV,
but triggered only when specific flags are set, usually set by \CMP.

\mysubparagraph{\OptimizingKeilVI (\ThumbMode)}

\lstinputlisting[caption=\OptimizingKeilVI (\ThumbMode),style=customasmARM]{patterns/07_jcc/simple/ARM/ARM_thumb_signed.asm}

\myindex{ARM!\Instructions!BLE}
\myindex{ARM!\Instructions!BNE}
\myindex{ARM!\Instructions!BGE}
\myindex{ARM!\Instructions!BLS}
\myindex{ARM!\Instructions!BCS}
\myindex{ARM!\Instructions!B}
\myindex{ARM!\ThumbMode}

Only \TT{B} instructions in Thumb mode may be supplemented by \IT{condition codes}, so the Thumb code 
looks more ordinary.

\TT{BLE} is a normal conditional jump \IT{Less than or Equal}, 
\TT{BNE}---\IT{Not Equal}, 
\TT{BGE}---\IT{Greater than or Equal}.

\TT{f\_unsigned} is similar, only other instructions are used while dealing 
with unsigned values: \TT{BLS} 
(\IT{Unsigned lower or same}) and \TT{BCS} (\IT{Carry Set (Greater than or equal)}).
