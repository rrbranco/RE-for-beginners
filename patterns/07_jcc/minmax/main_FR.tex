\subsection{Trouver les valeurs minimale et maximale}

\subsubsection{32-bit}

\lstinputlisting[style=customc]{patterns/07_jcc/minmax/minmax.c}

\lstinputlisting[caption=MSVC 2013 \NonOptimizing,style=customasmx86]{patterns/07_jcc/minmax/minmax_MSVC_2013_FR.asm}

\myindex{x86!\Instructions!Jcc}

Ces deux fonctions ne diffèrent que de l'instruction de saut conditionnel:
\INS{JGE} (\q{Jump if Greater or Equal} saut si supérieur ou égal) est utilisée
dans la première et \INS{JLE} (\q{Jump if Less or Equal} saut si inférieur ou égal)
dans la seconde.

\myindex{\CompilerAnomaly}
\label{MSVC_double_JMP_anomaly}

Il y a une instruction \JMP en trop dans chaque fonction, que MSVC a probablement
mise par erreur.

\myparagraph{Sans branchement}

Le mode Thumb d'ARM nous rappelle le code x86:

\lstinputlisting[caption=\OptimizingKeilVI (\ThumbMode),style=customasmARM]{patterns/07_jcc/minmax/minmax_Keil_Thumb_O3_FR.s}

\myindex{ARM!\Instructions!Bcc}

Les fonctions diffèrent au niveau de l'instruction de branchement: \INS{BGT} et \INS{BLT}.
Il est possible d'utiliser le suffixe conditionnel en mode ARM, donc le code est plus
court.

\myindex{ARM!\Instructions!MOVcc}
\INS{MOVcc} n'est exécutée que si la condition est remplie:

\lstinputlisting[caption=\OptimizingKeilVI (\ARMMode),style=customasmARM]{patterns/07_jcc/minmax/minmax_Keil_ARM_O3_FR.s}

\myindex{x86!\Instructions!CMOVcc}
GCC 4.8.1 \Optimizing et MSVC 2013 \Optimizing peuvent utiliser l'instruction \INS{CMOVcc},
qui est analogue à \INS{MOVcc} en ARM:

\lstinputlisting[caption=MSVC 2013 \Optimizing,style=customasmx86]{patterns/07_jcc/minmax/minmax_GCC481_O3_FR.s}

\subsubsection{64-bit}

\lstinputlisting[style=customc]{patterns/07_jcc/minmax/minmax64.c}

Il y a beaucoup de code inutile qui embrouille, mais il est compréhensible:

\lstinputlisting[caption=GCC 4.9.1 ARM64 \NonOptimizing,style=customasmARM]{patterns/07_jcc/minmax/minmax64_GCC_49_ARM64_O0.s}

\myparagraph{Sans branchement}

Il n'y a pas besoin de lire les arguments dans la pile, puisqu'ils sont déjà dans
les registres:

\lstinputlisting[caption=GCC 4.9.1 x64 \Optimizing,style=customasmx86]{patterns/07_jcc/minmax/minmax64_GCC_49_x64_O3_FR.s}

MSVC 2013 fait presque la même chose.

\myindex{ARM!\Instructions!CSEL}

ARM64 possède l'instruction \INS{CSEL}, qui fonctionne comme \INS{MOVcc} en ARM ou
\INS{CMOVcc} en x86, seul le nom diffère:
\q{Conditional SELect}.

\lstinputlisting[caption=GCC 4.9.1 ARM64 \Optimizing,style=customasmARM]{patterns/07_jcc/minmax/minmax64_GCC_49_ARM64_O3_FR.s}

\subsubsection{MIPS}

Malheureusement, GCC 4.4.5 pour MIPS n'est pas si performant:

\lstinputlisting[caption=GCC 4.4.5 \Optimizing (IDA),style=customasmMIPS]{patterns/07_jcc/minmax/minmax_MIPS_O3_IDA_FR.lst}

N'oubliez pas le slot de délai de branchement (\IT{branch delay slots}): le premier
\INS{MOVE} est exécuté \IT{avant} \INS{BEQZ}, le second \INS{MOVE} n'est exécuté
que si la branche n'a pas été prise.

