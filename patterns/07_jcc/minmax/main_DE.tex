\subsection{Minimale und maximale Werte berechnen}

\subsubsection{32-bit}

\lstinputlisting[style=customc]{patterns/07_jcc/minmax/minmax.c}

\lstinputlisting[caption=\NonOptimizing MSVC 2013,style=customasmx86]{patterns/07_jcc/minmax/minmax_MSVC_2013_DE.asm}

\myindex{x86!\Instructions!Jcc}
Diese beiden Funktionen unterscheiden sich nur hinsichtliche der bedingten Sprungbefehle:
\INS{JGE} (\q{Jump if Greater or Equal}) wird in der ersten verwendet
und \INS{JLE} (\q{Jump if Less or Equal}) in der zweiten.

\myindex{\CompilerAnomaly}
\label{MSVC_double_JMP_anomaly}
Hier gibt es jeweils einen unnötigen \JMP Befehl pro Funtion, den MSVC wahrscheinlich fehlerhafterweise dort belassen
hat.

\myparagraph{Verzweigungslos}

ARM im Thumb mode erinnert uns an den x86 Code:

\lstinputlisting[caption=\OptimizingKeilVI
(\ThumbMode),style=customasmARM]{patterns/07_jcc/minmax/minmax_Keil_Thumb_O3_DE.s}

\myindex{ARM!\Instructions!Bcc}
Die Funktionen unterscheiden sich in den Verzweigebefehlen: \INS{BGT} und \INS{BLT}.
Es ist möglich im ARM mode conditional codes zu verwenden, sodass der Code kürzer ist.

\myindex{ARM!\Instructions!MOVcc}
\INS{MOVcc} wird nur ausgeführt, wenn die Bedingung erfüllt (d.h. wahr) ist:

\lstinputlisting[caption=\OptimizingKeilVI
(\ARMMode),style=customasmARM]{patterns/07_jcc/minmax/minmax_Keil_ARM_O3_DE.s}

\myindex{x86!\Instructions!CMOVcc}
\Optimizing GCC 4.8.1 und der optimierende MSVC 2013 können den \INS{CMOVcc} Befehl verwenden, der analog zu
\INS{MOVcc} in ARM funktioniert:

\lstinputlisting[caption=\Optimizing MSVC 2013,style=customasmx86]{patterns/07_jcc/minmax/minmax_GCC481_O3_DE.s}

\subsubsection{64-bit}

\lstinputlisting[style=customc]{patterns/07_jcc/minmax/minmax64.c}
Hier findet ein unnötiges Verschieben von Variablen statt, aber der Code ist verständlich:

\lstinputlisting[caption=\NonOptimizing GCC 4.9.1 ARM64,style=customasmARM]{patterns/07_jcc/minmax/minmax64_GCC_49_ARM64_O0.s}

\myparagraph{Verzweigungslos}
Die Funtionsargumente müssen nicht vom Stack geladen werden, da sie sich bereits in den Registern befinden:

\lstinputlisting[caption=\Optimizing GCC 4.9.1
x64,style=customasmx86]{patterns/07_jcc/minmax/minmax64_GCC_49_x64_O3_DE.s}

MSVC 2013 tut beinahe das gleiche:

\myindex{ARM!\Instructions!CSEL}

ARM64 verfügt über den \INS{CSEL} Befehl, der genau wie \INS{MOVcc} in ARM oder \INS{CMOVcc} in x86 arbeitet; er hat
lediglich einen anderen Namen:
\q{Conditional SELect}.

\lstinputlisting[caption=\Optimizing GCC 4.9.1
ARM64,style=customasmARM]{patterns/07_jcc/minmax/minmax64_GCC_49_ARM64_O3_DE.s}

\subsubsection{MIPS}

Leider ist GCC 4.4.5 für MIPS nicht so gut:

\lstinputlisting[caption=\Optimizing GCC 4.4.5
(IDA),style=customasmMIPS]{patterns/07_jcc/minmax/minmax_MIPS_O3_IDA_DE.lst} 
Vergessen Sie nicht die \IT{branch delay slots}: der erste \INS{MOVE} wird \IT{vor} \INS{BEQZ} ausgeführt, der zweite
\INS{MOVE} wird nur dann ausgeführt, wenn die Verzweigung nicht genommen wird.


