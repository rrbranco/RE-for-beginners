\section{Saut conditionnels}
\label{sec:Jcc}
\myindex{\CLanguageElements!if}

% sections
\subsection{\RU{Простой пример}\EN{Simple example}\DEph{}
\FR{Exemple simple}
}

\lstinputlisting[style=customc]{patterns/07_jcc/simple/ex.c}

% subsections
\EN{\subsubsection{x86}

\myparagraph{x86 + MSVC}

Here is how the \TT{f\_signed()} function looks like:

\lstinputlisting[caption=\NonOptimizing MSVC 2010,style=customasmx86]{patterns/07_jcc/simple/signed_MSVC.asm}

\myindex{x86!\Instructions!JLE}

The first instruction, \JLE, stands for \IT{Jump if Less or Equal}. 
In other words, if the second operand is 
larger or equal to the first one, the control flow will be passed to the address or label specified in the instruction.
If this condition does not trigger because the second operand is smaller than the first one, the control flow would not be altered and the first \printf would be executed.
\myindex{x86!\Instructions!JNE}
The second check is \JNE: \IT{Jump if Not Equal}.
The control flow will not change if the operands are equal.

\myindex{x86!\Instructions!JGE}
The third check is \JGE: \IT{Jump if Greater or Equal}---jump if the first operand is larger than 
the second or if they are equal.
So, if all three conditional jumps are triggered, none of the \printf calls would be executed whatsoever. 
This is impossible without special intervention.
Now let's take a look at the \TT{f\_unsigned()} function.
The \TT{f\_unsigned()} function is the same as \TT{f\_signed()}, with the exception that the \JBE and \JAE instructions
are used instead of \JLE and \JGE, as follows:

\lstinputlisting[caption=GCC,style=customasmx86]{patterns/07_jcc/simple/unsigned_MSVC.asm}

\myindex{x86!\Instructions!JBE}
\myindex{x86!\Instructions!JAE}

As already mentioned, the branch instructions are different:
\JBE---\IT{Jump if Below or Equal} and \JAE---\IT{Jump if Above or Equal}.
These instructions (\JA/\JAE/\JB/\JBE) differ from \JG/\JGE/\JL/\JLE in the fact that they work with unsigned numbers.

\myindex{x86!\Instructions!JA}
\myindex{x86!\Instructions!JB}
\myindex{x86!\Instructions!JG}
\myindex{x86!\Instructions!JL}
\myindex{Signed numbers}

See also the section about signed number representations~(\myref{sec:signednumbers}).
That is why if we see \JG/\JL in use instead of \JA/\JB or vice-versa, 
we can be almost sure that the variables are signed or unsigned, respectively.
Here is also the \main function, where there is nothing much new to us:

\lstinputlisting[caption=\main,style=customasmx86]{patterns/07_jcc/simple/main_MSVC.asm}

\input{patterns/07_jcc/simple/olly_EN.tex}

\clearpage
\myparagraph{x86 + MSVC + Hiew}
\myindex{Hiew}

We can try to patch the executable file in a way 
that the \TT{f\_unsigned()} function would always print \q{a==b}, 
no matter the input values.
Here is how it looks in Hiew:

\begin{figure}[H]
\centering
\myincludegraphics{patterns/07_jcc/simple/hiew_unsigned1.png}
\caption{Hiew: \TT{f\_unsigned()} function}
\label{fig:jcc_hiew_1}
\end{figure}

Essentially, we have to accomplish three tasks:
\begin{itemize}
\item force the first jump to always trigger;
\item force the second jump to never trigger;
\item force the third jump to always trigger.
\end{itemize}

Thus we can direct the code flow to always pass through the second \printf, and output \q{a==b}.

Three instructions (or bytes) has to be patched:

\begin{itemize}
\item The first jump becomes \JMP, but the \gls{jump offset} would remain the same.

\item 
The second jump might be triggered sometimes, but in any case it will jump to the next
instruction, because, we set the \gls{jump offset} to 0.

In these instructions the \gls{jump offset} is added to the address for the next instruction.
So if the offset is 0,
the jump will transfer the control to the next instruction.

\item 
The third jump we replace with \JMP just as we do with the first one, so it will always trigger.

\end{itemize}

\clearpage
Here is the modified code:

\begin{figure}[H]
\centering
\myincludegraphics{patterns/07_jcc/simple/hiew_unsigned2.png}
\caption{Hiew: let's modify the \TT{f\_unsigned()} function}
\label{fig:jcc_hiew_2}
\end{figure}

If we miss to change any of these jumps, then several \printf calls may execute, while we want to execute only one.

\myparagraph{\NonOptimizing GCC}

\myindex{puts() instead of printf()}
\NonOptimizing GCC 4.4.1 
produces almost the same code, but with \puts~(\myref{puts}) instead of \printf.

\myparagraph{\Optimizing GCC}

An observant reader may ask, why execute \CMP several times, 
if the flags has the same values after each execution?

Perhaps optimizing MSVC cannot do this, but optimizing GCC 4.8.1 can go deeper:

\lstinputlisting[caption=GCC 4.8.1 f\_signed(),style=customasmx86]{patterns/07_jcc/simple/GCC_O3_signed.asm}

% should be here instead of 'switch' section?
We also see \TT{JMP puts} here instead of \TT{CALL puts / RETN}.

This kind of trick will have explained later: \myref{JMP_instead_of_RET}.

This type of x86 code 
is somewhat rare.
MSVC 2012 as it seems, can't generate such code.
On the other hand, assembly language programmers are fully aware of the fact that \TT{Jcc} 
instructions can be stacked.

So if you see such stacking somewhere, it is highly probable that the code was hand-written.

The \TT{f\_unsigned()} function is not that 
\ae{}sthetically short:

\lstinputlisting[caption=GCC 4.8.1 f\_unsigned(),style=customasmx86]{patterns/07_jcc/simple/GCC_O3_unsigned_EN.asm}

Nevertheless, there are two \TT{CMP} instructions instead of three.

So optimization algorithms of GCC 4.8.1 are probably not perfect yet. 
}
\RU{\subsubsection{x86}

\myparagraph{x86 + MSVC}

Имеем в итоге функцию \TT{f\_signed()}:

\lstinputlisting[caption=\NonOptimizing MSVC 2010,style=customasmx86]{patterns/07_jcc/simple/signed_MSVC.asm}

\myindex{x86!\Instructions!JLE}
Первая инструкция \JLE значит \IT{Jump if Less or Equal}. 
Если второй операнд больше первого или равен ему, произойдет переход туда, где будет следующая проверка.

А если это условие не срабатывает (то есть второй операнд меньше первого), то перехода не будет, 
и сработает первый \printf.

\myindex{x86!\Instructions!JNE}
Вторая проверка это \JNE: \IT{Jump if Not Equal}.
Переход не произойдет, если операнды равны.

\myindex{x86!\Instructions!JGE}
Третья проверка \JGE: \IT{Jump if Greater or Equal} --- переход 
если первый операнд больше второго или равен ему.
Кстати, если все три условных перехода сработают, ни один \printf не вызовется. 
Но без внешнего вмешательства это невозможно.

Функция \TT{f\_unsigned()} точно такая же, за тем исключением, что используются инструкции 
\JBE и \JAE вместо \JLE и \JGE:

\lstinputlisting[caption=GCC,style=customasmx86]{patterns/07_jcc/simple/unsigned_MSVC.asm}

\myindex{x86!\Instructions!JBE}
\myindex{x86!\Instructions!JAE}
Здесь всё то же самое, только инструкции условных переходов немного другие:

\JBE --- \IT{Jump if Below or Equal} и \JAE\EMDASH{}\IT{Jump if Above or Equal}.
Эти инструкции (\JA/\JAE/\JB/\JBE) 
отличаются от \JG/\JGE/\JL/\JLE тем, что работают с беззнаковыми переменными.

\myindex{x86!\Instructions!JA}
\myindex{x86!\Instructions!JB}
\myindex{x86!\Instructions!JG}
\myindex{x86!\Instructions!JL}
\myindex{Signed numbers}
Отступление: смотрите также секцию о представлении знака в числах ~(\myref{sec:signednumbers}).
Таким образом, увидев где используется \JG/\JL вместо \JA/\JB и наоборот, 
можно сказать почти уверенно насчет того, 
является ли тип переменной знаковым (signed) или беззнаковым (unsigned).

Далее функция \main, где ничего нового для нас нет:

\lstinputlisting[caption=\main,style=customasmx86]{patterns/07_jcc/simple/main_MSVC.asm}

\input{patterns/07_jcc/simple/olly_RU.tex}

\clearpage
\myparagraph{x86 + MSVC + Hiew}
\myindex{Hiew}

Можем попробовать модифицировать исполняемый файл так, чтобы функция \TT{f\_unsigned()} всегда показывала \q{a==b},
при любых входящих значениях.
Вот как она выглядит в Hiew:

\begin{figure}[H]
\centering
\myincludegraphics{patterns/07_jcc/simple/hiew_unsigned1.png}
\caption{Hiew: функция \TT{f\_unsigned()}}
\label{fig:jcc_hiew_1}
\end{figure}

Собственно, задач три:
\begin{itemize}
\item заставить первый переход срабатывать всегда;
\item заставить второй переход не срабатывать никогда;
\item заставить третий переход срабатывать всегда.
\end{itemize}

Так мы направим путь исполнения кода (code flow) во второй \printf,
и он всегда будет срабатывать и выводить на консоль \q{a==b}.

Для этого нужно изменить три инструкции (или байта):

\begin{itemize}
\item Первый переход теперь будет \JMP, но смещение перехода 
(\gls{jump offset}) останется прежним.

\item Второй переход может быть и будет срабатывать иногда, но в любом случае он будет совершать переход
только на следующую инструкцию, потому что мы выставляем смещение перехода (\gls{jump offset}) в 0.

В этих инструкциях смещение перехода просто прибавляется к адресу следующей инструкции.

Когда смещение 0, переход будет на следующую инструкцию.

\item Третий переход конвертируем в \JMP точно так же, как и первый, он будет срабатывать всегда.

\end{itemize}

\clearpage
Что и делаем:

\begin{figure}[H]
\centering
\myincludegraphics{patterns/07_jcc/simple/hiew_unsigned2.png}
\caption{Hiew: модифицируем функцию \TT{f\_unsigned()}}
\label{fig:jcc_hiew_2}
\end{figure}

Если забыть про какой-то из переходов, то тогда будет срабатывать несколько вызовов \printf, 
а нам ведь нужно чтобы исполнялся только один.

\myparagraph{\NonOptimizing GCC}

\myindex{puts() вместо printf()}
\NonOptimizing GCC 4.4.1 производит почти такой же код, за исключением \puts~(\myref{puts}) вместо \printf.

\myparagraph{\Optimizing GCC}

Наблюдательный читатель может спросить, зачем исполнять \CMP так много раз,
если флаги всегда одни и те же?
По видимому, оптимизирующий MSVC не может этого делать, но GCC 4.8.1 делает больше оптимизаций:

\lstinputlisting[caption=GCC 4.8.1 f\_signed(),style=customasmx86]{patterns/07_jcc/simple/GCC_O3_signed.asm}

% should be here instead of 'switch' section?
Мы здесь также видим \TT{JMP puts} вместо \TT{CALL puts / RETN}.
Этот прием описан немного позже: \myref{JMP_instead_of_RET}.

Нужно сказать, что x86-код такого типа редок.
MSVC 2012, как видно, не может генерировать подобное.
С другой стороны, программисты на ассемблере прекрасно осведомлены о том,
что инструкции \TT{Jcc} можно располагать последовательно.

Так что если вы видите это где-то, имеется немалая вероятность, что этот фрагмент кода был написан вручную.

Функция \TT{f\_unsigned()} получилась не настолько эстетически короткой:

\lstinputlisting[caption=GCC 4.8.1 f\_unsigned(),style=customasmx86]{patterns/07_jcc/simple/GCC_O3_unsigned_RU.asm}

Тем не менее, здесь 2 инструкции \TT{CMP} вместо трех.

Так что, алгоритмы оптимизации GCC 4.8.1, наверное, ещё пока не идеальны.
}
\DE{\subsubsection{x86}

\myparagraph{x86 + MSVC}

Die Funktions \TT{f\_signed()} sieht folgendermaßen aus:

\lstinputlisting[caption=\NonOptimizing MSVC 2010,style=customasmx86]{patterns/07_jcc/simple/signed_MSVC.asm}

\myindex{x86!\Instructions!JLE}
Der erste Befehl, \JLE steht für \IT{Jump if Less or Equal}.
Mit anderen Worten, wenn der zweite Operand größer gleich dem ersten ist, wird der Control Flow an die angegebene
Adresse bzw. das angegebene Label übergeben.
Wenn die Bedingung falsch ist, weil der zweite Operand kleiner ist als der erste, wird der Control Flow nicht verändert
und das erste \printf wird ausgeführt.

Tmyindex{x86!\Instructions!JNE}
Der zweite Check ist \JNE: \IT{Jump if Not Equal}.
Der Control Flow wird nicht verändert, wenn die Operanden gleich sind.

\myindex{x86!\Instructions!JGE}
Der dritte Check ist \JGE: \IT{Jump if Greater or Equal}---springe, falls der erste Operand größer gleich dem zweiten
ist.
Wenn also alle drei bedingten Sprünge ausgeführt werden, wird also kein Aufruf von \printf ausgeführt.
Dies ist ohne manuellen Eingriff unmöglich.
Werfen wir nun einen Blick auf die Funktion \TT{f\_unsigned()}.
Diese Funktion macht prinzipiell das gleiche wie \TT{f\_signed()} mit der Ausnahme, dass die Befehle \JBE und \JAE
anstelle von \JLE und \JGE wie folgt verwendet werden:

\lstinputlisting[caption=GCC,style=customasmx86]{patterns/07_jcc/simple/unsigned_MSVC.asm}

\myindex{x86!\Instructions!JBE}
\myindex{x86!\Instructions!JAE}
Wie bereits erwähnt unterscheiden sich die Verzweigungsbefehle:
\JBE---\IT{Jump if Below or Equal} und \JAE---\IT{Jump if Above or Equal}.
Diese Befehle (\JA/\JAE/\JB/\JBE) unterscheiden sich von \JG/\JGE/\JL/\JLE dadurch, dass sie mit vorzeichenlosen Zahlen
arbeiten.

\myindex{x86!\Instructions!JA}
\myindex{x86!\Instructions!JB}
\myindex{x86!\Instructions!JG}
\myindex{x86!\Instructions!JL}
\myindex{Signed numbers}
Siehe hierzu auch den Abschnitt über Darstellung vorzeichenbehafteter Zahlen~(\myref{sec:signednumbers}).
Das ist der Grund warum wir, wenn wir \JG/\JL anstelle von \JA/\JB und umgekehrt finden, fast mit Gewissheit sagen
können, dass die Variablen vorzeichenbehaftet bzw. vorzeichenlos sind.
Hier befindet sich auch die Funktion \main, welche für uns nichts Neues bereithält:

\lstinputlisting[caption=\main,style=customasmx86]{patterns/07_jcc/simple/main_MSVC.asm}

\input{patterns/07_jcc/simple/olly_EN.tex}

\clearpage
\myparagraph{x86 + MSVC + Hiew}
\myindex{Hiew}
Wir können versuchen, die Executable so zu verändern, dass die Funktion \TT{f\_unsigned()} stets \q{a==b} ausgibt, egal
was wir eingben.
So sieht das ganze in Hiew aus:
\begin{figure}[H]
\centering
\myincludegraphics{patterns/07_jcc/simple/hiew_unsigned1.png}
\caption{Hiew: Funktion \TT{f\_unsigned()}}
\label{fig:jcc_hiew_1}
\end{figure}
Grundsätzlich haben wir drei Dinge zu erzwingen:
\begin{itemize}
  \item den ersten Sprung stets ausführen;
  \item den zweiten Sprung nie ausführen;
  \item den dritten Sprung stets ausführen.
\end{itemize}

Dadurch können wir den Code Flow so manupulieren, dass das zweite \printf immer ausgeführt wird und \q{a==b} ausgibt.
Drei Befehle (oder Bytes) müssen verändert werden:
\begin{itemize}
\item Der erste Sprung wird zu \JMP verändert, aber der \gls{jump offset} bleibt gleich.

 
\item Der zweite Sprung könnte manchmal ausgeführt werden, wird aber in jedem Fall zum nächsten Befehl springen, denn
wir setzen den \gls{jump offset} auf 0.
Bei diesen Befehlen wird der \gls{jump offset} zu der Adresse der nächsten Befehls addiert. Wenn der Offset 0 ist, wird
die Ausführung also beim nächsten Befehl fortgesetzt.
\item 
Den dritten Sprung ersetzen wie genau wie den ersten durch \JMP, damit er stets ausgeführt wird.

\end{itemize}

\clearpage
Hier ist der veränderte Code:

\begin{figure}[H]
\centering
\myincludegraphics{patterns/07_jcc/simple/hiew_unsigned2.png}
\caption{Hiew: Veränderte Funktion \TT{f\_unsigned()}}
\label{fig:jcc_hiew_2}
\end{figure}
Wenn wir es verpassen, einen dieser Sprünge zu verändern, könnten mehrere Aufrufe von \printf ausgeführt werden; wir
wollen aber nur genau einen Aufruf ausführen.

\myparagraph{\NonOptimizing GCC}

\myindex{puts() anstelle von printf()}
\NonOptimizing GCC 4.4.1 
erzeugt fast identischen Code, aber mit \puts~(\myref{puts}) anstelle von \printf.

\myparagraph{\Optimizing GCC}
Der aufmerksame Leser fragt sich vielleicht, warum \CMP mehrmals ausgeführt wird, wenn doch die Flags nach jeder
Ausführung dieselben Werte haben. 

Vielleicht kann der optimierende MSVC dies nicht leisten, aber der optimierende GCC 4.8.1 gräbt tiefer:

\lstinputlisting[caption=GCC 4.8.1 f\_signed(),style=customasmx86]{patterns/07_jcc/simple/GCC_O3_signed.asm}

% should be here instead of 'switch' section?
Wir finden auch hier \TT{JMP puts} anstelle von \TT{CALL puts / RETN}.

Dieser Trick wird später erklärt:\myref{JMP_instead_of_RET}.

Diese Sorte x86 Code ist trotzdem selten. MSVC 2012 kann wie es scheint solchen Code nicht erzeugen.
Andererseits sind Assemblerprogrammierer sich natürlich der Tatsache bewusst, dass \TT{Jcc} Befehle gestackt werden
können.
Wenn man solche gestackten Befehle findet, ist es sehr wahrscheinlich, dass der entsprechende Code von Hand geschrieben
wurde. 
Die Funktion \TT{f\_unsigned()} ist nicht so ästhetisch:


\lstinputlisting[caption=GCC 4.8.1 f\_unsigned(),style=customasmx86]{patterns/07_jcc/simple/GCC_O3_unsigned_DE.asm}
Trotzdem werden hier immerhin nur zwei statt drei \TT{CMP} Befehle verwendet.

Die Optimierungsalgorithmen von GCC 4.8.1 sind möglicherweise noch nicht so ausgereift.
}
\FR{\subsubsection{x86}

\myparagraph{x86 + MSVC}

Voici à quoi ressemble la fonction  \TT{f\_signed()}:

\lstinputlisting[caption=MSVC 2010 \NonOptimizing,style=customasmx86]{patterns/07_jcc/simple/signed_MSVC.asm}

\myindex{x86!\Instructions!JLE}

La première instruction, \JLE, représente \IT{Jump if Less or Equal} (saut si inférieur ou égal).
En d'autres mots, si la deuxième opérande est plus grande ou égale à la première,
le flux d'exécution est passé à l'adresse ou au label spécifié dans l'instruction.
Si la condition ne déclenche pas le saut car seconde opérande est plus petite que
la première, le flux d'exécution ne sera pas altéré et le premier \printf sera
exécuté.
\myindex{x86!\Instructions!JNE}
Le second test est \JNE: \IT{Jump if Not Equal} (saut si non égal).
Le flux d'exécution ne changera pas si les opérandes sont égales.

\myindex{x86!\Instructions!JGE}
Le troisième test est \JGE: \IT{Jump if Greater or Equal}---saute si la première
opérande est supérieure ou égale à la deuxième.
Donc, si les trois sauts conditionnels sont effectués, aucun des appels à \printf
ne sera exécuté.
Ceci est impossible sans intervention spéciale.
Regardons maintenant la fonction \TT{f\_unsigned()}.
La fonction \TT{f\_unsigned()} est la même que \TT{f\_signed()}, à la différence
que les instructions \JBE et \JAE sont utilisées à la place de \JLE et \JGE, comme
suit:

\lstinputlisting[caption=GCC,style=customasmx86]{patterns/07_jcc/simple/unsigned_MSVC.asm}

\myindex{x86!\Instructions!JBE}
\myindex{x86!\Instructions!JAE}

Comme déjà mentionné, les intructions de branchement sont différentes:
\JBE---\IT{Jump if Below or Equal} (saut si inférieur ou égal) et \JAE---\IT{Jump if Above or Equal}
(saut si supérieur ou égal).
Ces instructions (\JA/\JAE/\JB/\JBE) diffèrent de \JG/\JGE/\JL/\JLE par le fait qu'elles
travaillent avec des nombres non signés.

\myindex{x86!\Instructions!JA}
\myindex{x86!\Instructions!JB}
\myindex{x86!\Instructions!JG}
\myindex{x86!\Instructions!JL}
\myindex{Signed numbers}

Voir aussi la section sur la représentation des nombres signés~(\myref{sec:signednumbers}).
C'est pourquoi si nous voyons que \JG/\JL sont utilisés à la place de \JA/\JB ou
vice-versa, nous pouvons être presque sûr que les variables sont signées ou non
signées, respectivement.
Voici la fonction \main, où presque rien n'est nouveau pour nous:

\lstinputlisting[caption=\main,style=customasmx86]{patterns/07_jcc/simple/main_MSVC.asm}

\input{patterns/07_jcc/simple/olly_FR.tex}

\clearpage
\myparagraph{x86 + MSVC + Hiew}
\myindex{Hiew}

Nous pouvons essayer de patcher l'exécutable afin que la fonction \TT{f\_unsigned()}
affiche toujours \q{a==b}, quelque soient les valeurs en entrée.
Voici à quoi ça ressemble dans Hiew:

\begin{figure}[H]
\centering
\myincludegraphics{patterns/07_jcc/simple/hiew_unsigned1.png}
\caption{Hiew: fonction \TT{f\_unsigned()}}
\label{fig:jcc_hiew_1}
\end{figure}

Essentielleemnt, nous devons accomplir ces trois choses:
\begin{itemize}
\item forcer le premier saut à toujours être effectué;
\item forcer le second saut à n'être jamais effectué;
\item forcer le troisième saut à toujours être effectué.
\end{itemize}

Nous devons donc diriger le déroulement du code pour toujours efectuer le second \printf,
et afficher \q{a==b}.

Trois instructions (ou octets) doivent être modifiées:

\begin{itemize}
\item Le premier saut devient un \JMP, mais l'\glslink{jump offset}{offset} reste
le même.

\item
Le second saut peut être parfois suivi, mais dans chaque cas il sautera à l'instruction
suivante, car nous avons mis l'\glslink{jump offset}{offset} à 0.

Dans cette instruction, l'\glslink{jump offset}{offset} est ajouté à l'adresse
de l'instruction suivante. Donc si l'offset est 0, le saut va transfèrer l'exécution
à l'instruction suivante.

\item
Le troisième saut est remplacé par \JMP comme nous l'avons fait pour le premier,
il sera donc toujours effectué.

\end{itemize}

\clearpage
Voici le code modifié:

\begin{figure}[H]
\centering
\myincludegraphics{patterns/07_jcc/simple/hiew_unsigned2.png}
\caption{Hiew: modifions la fonction \TT{f\_unsigned()}}
\label{fig:jcc_hiew_2}
\end{figure}

Si nous oublions de modifier une de ces sauts condtionnels, plusieurs appels à \printf
pourraient être faits, alors que nous voulons qu'un seul soit exécuté.

\myparagraph{GCC \NonOptimizing}

\myindex{puts() instead of printf()}
GCC 4.4.1 \NonOptimizing produit presque le même code, mais avec \puts~(\myref{puts})
à la place de \printf.

\myparagraph{GCC \Optimizing}

Le lecteur attentif pourrait demander pourquoi exécuter \CMP plusieurs fois, si
les flags ont les mêmes valeurs après l'exécution ?

Peut-être que l'optimiseur de de MSVC ne peut pas faire cela, mais celui de GCC
4.8.1 peut aller plus loin:

\lstinputlisting[caption=GCC 4.8.1 f\_signed(),style=customasmx86]{patterns/07_jcc/simple/GCC_O3_signed.asm}

% should be here instead of 'switch' section?
Nous voyons ici \TT{JMP puts} au lieu de \TT{CALL puts / RETN}.

Ce genre de truc sera expliqué plus loin: \myref{JMP_instead_of_RET}.

Ce genre de code x86 est plutôt rare.
Il semble que MSVC 2012 ne puisse pas générer un tel code.
D'un autre côté, les programmeurs en langage d'assemblage sont parfaitement conscients
du fait que les instructions \TT{Jcc} peuvent être empilées.

Donc si vous voyez ce genre d'empilement, il est très probable que le code a été
écrit à la main.

La fonction \TT{f\_unsigned()} n'est \ae{}sthétiquement courte:

\lstinputlisting[caption=GCC 4.8.1 f\_unsigned(),style=customasmx86]{patterns/07_jcc/simple/GCC_O3_unsigned_FR.asm}

Néanmoins, il y a deux instructions \TT{CMP} au lieu de trois.

Donc les algorithmes d'optimisation de GCC 4.8.1 ne sont probablement pas encore parfaits.

}

\subsubsection{ARM}

% subsubsections here
\EN{\myparagraph{32-bit ARM}
\label{subsec:jcc_ARM}

\mysubparagraph{\OptimizingKeilVI (\ARMMode)}

\lstinputlisting[caption=\OptimizingKeilVI (\ARMMode),style=customasmARM]{patterns/07_jcc/simple/ARM/ARM_O3_signed.asm}

\myindex{ARM!Condition codes}
% FIXME \ref -> which instructions?

Many instructions in ARM mode could be executed only when specific flags are set.
E.g. this is often used when comparing numbers.

\myindex{ARM!\Instructions!ADD}
\myindex{ARM!\Instructions!ADDAL}

For instance, the \ADD instruction is in fact named \TT{ADDAL} internally, where \TT{AL} stands for
\IT{Always}, i.e., execute always.
The predicates are encoded in 4 high bits of the 32-bit ARM instructions (\IT{condition field}).
\myindex{ARM!\Instructions!B}
The \TT{B} instruction for unconditional jumping is in fact conditional and encoded just like any other
conditional jump, but has \TT{AL} in the \IT{condition field}, and it implies \IT{execute ALways}, 
ignoring flags.

\myindex{ARM!\Instructions!ADR}
\myindex{ARM!\Instructions!ADRcc}
\myindex{ARM!\Instructions!CMP}

The \TT{ADRGT} instruction works just like \TT{ADR} but executes only in case the previous \CMP
instruction founds one of the numbers greater than the another, while comparing the two (\IT{Greater Than}).

\myindex{ARM!\Instructions!BL}
\myindex{ARM!\Instructions!BLcc}

The next \TT{BLGT} instruction behaves exactly as \TT{BL} 
and is triggered only if the result of the comparison has been (\IT{Greater Than}). 
\TT{ADRGT} writes a pointer to the string \TT{a>b\textbackslash{}n} into \Reg{0} and \TT{BLGT} calls \printf.
Therefore, instructions suffixed with \TT{-GT} are to execute only in case the value in \Reg{0} (which is $a$) is bigger than the value in \Reg{4} (which is $b$).

\myindex{ARM!\Instructions!ADRcc}
\myindex{ARM!\Instructions!BLcc}

Moving forward we see the \TT{ADREQ} and \TT{BLEQ} instructions.
They behave just like \TT{ADR} and \TT{BL}, but are to be executed only if operands were equal to each
other during the last comparison.
Another \CMP is located before them (because the \printf execution may have tampered the flags).

\myindex{ARM!\Instructions!LDMccFD}
\myindex{ARM!\Instructions!LDMFD}

Then we see \TT{LDMGEFD}, this instruction works just like \TT{LDMFD}\footnote{\ac{LDMFD}},
but is triggered only when one of the values is greater or equal than the other (\IT{Greater or Equal}).
The \TT{LDMGEFD SP!, \{R4-R6,PC\}} instruction acts like a function epilogue, but it will be triggered only if $a>=b$, and only then the function execution will finish.

\myindex{Function epilogue}

But if that condition is not satisfied, i.e., $a<b$, then the control flow will continue to the next \\
\TT{\q{LDMFD SP!, \{R4-R6,LR\}}} instruction, which is one more function epilogue. This instruction restores not only the \TT{R4-R6} registers state, but also \ac{LR} instead of \ac{PC}, thus, it does not return from the function.
The last two instructions call \printf with the string <<a<b\textbackslash{}n>> as a sole argument.
We already examined an unconditional jump to the \printf function instead of function return in <<\PrintfSeveralArgumentsSectionName>> section~(\myref{ARM_B_to_printf}).

\myindex{ARM!\Instructions!ADRcc}
\myindex{ARM!\Instructions!BLcc}
\myindex{ARM!\Instructions!LDMccFD}
\TT{f\_unsigned} is similar, only the \TT{ADRHI}, \TT{BLHI}, and \TT{LDMCSFD} instructions are used there, these predicates (\IT{HI = Unsigned higher, CS = Carry Set (greater than or equal)}) are analogous to those examined before, but for unsigned values.

There is not much new in the \main function for us:

\lstinputlisting[caption=\main,style=customasmARM]{patterns/07_jcc/simple/ARM/ARM_O3_main.asm}

That is how you can get rid of conditional jumps in ARM mode.

\myindex{RISC pipeline}
Why is this so good? Read here: \myref{branch_predictors}.

\myindex{x86!\Instructions!CMOVcc}

There is no such feature in x86, except the \TT{CMOVcc} instruction, it is the same as \MOV,
but triggered only when specific flags are set, usually set by \CMP.

\mysubparagraph{\OptimizingKeilVI (\ThumbMode)}

\lstinputlisting[caption=\OptimizingKeilVI (\ThumbMode),style=customasmARM]{patterns/07_jcc/simple/ARM/ARM_thumb_signed.asm}

\myindex{ARM!\Instructions!BLE}
\myindex{ARM!\Instructions!BNE}
\myindex{ARM!\Instructions!BGE}
\myindex{ARM!\Instructions!BLS}
\myindex{ARM!\Instructions!BCS}
\myindex{ARM!\Instructions!B}
\myindex{ARM!\ThumbMode}

Only \TT{B} instructions in Thumb mode may be supplemented by \IT{condition codes}, so the Thumb code 
looks more ordinary.

\TT{BLE} is a normal conditional jump \IT{Less than or Equal}, 
\TT{BNE}---\IT{Not Equal}, 
\TT{BGE}---\IT{Greater than or Equal}.

\TT{f\_unsigned} is similar, only other instructions are used while dealing 
with unsigned values: \TT{BLS} 
(\IT{Unsigned lower or same}) and \TT{BCS} (\IT{Carry Set (Greater than or equal)}).
}
\RU{\myparagraph{32-битный ARM}
\label{subsec:jcc_ARM}

\mysubparagraph{\OptimizingKeilVI (\ARMMode)}

\lstinputlisting[caption=\OptimizingKeilVI (\ARMMode),style=customasmARM]{patterns/07_jcc/simple/ARM/ARM_O3_signed.asm}

\myindex{ARM!Condition codes}
% FIXME \ref -> which instructions?
Многие инструкции в режиме ARM могут быть исполнены только при некоторых выставленных флагах.

Это нередко используется для сравнения чисел.

\myindex{ARM!\Instructions!ADD}
\myindex{ARM!\Instructions!ADDAL}
К примеру, инструкция \ADD на самом деле называется \TT{ADDAL} внутри, \TT{AL} означает \IT{Always}, то есть, исполнять всегда.
Предикаты кодируются в 4-х старших битах инструкции 32-битных ARM-инструкций (\IT{condition field}).
\myindex{ARM!\Instructions!B}
Инструкция безусловного перехода \TT{B} на самом деле условная и кодируется так же, 
как и прочие инструкции условных переходов, но имеет \TT{AL} в \IT{condition field}, 
то есть исполняется всегда (\IT{execute ALways}), игнорируя флаги.

\myindex{ARM!\Instructions!ADR}
\myindex{ARM!\Instructions!ADRcc}
\myindex{ARM!\Instructions!CMP}
Инструкция \TT{ADRGT} работает так же, как и \TT{ADR}, но исполняется только в случае,
если предыдущая инструкция \CMP,
сравнивая два числа, обнаруживает, что одно из них больше второго (\IT{Greater Than}).

\myindex{ARM!\Instructions!BL}
\myindex{ARM!\Instructions!BLcc}
Следующая инструкция \TT{BLGT} ведет себя так же, как и \TT{BL} и сработает, только если 
результат сравнения ``больше чем'' (\IT{Greater Than}).
\TT{ADRGT} записывает в \Reg{0} указатель на строку \TT{a>b\textbackslash{}n}, а \TT{BLGT} вызывает \printf.
Следовательно, эти инструкции с суффиксом \TT{-GT} исполнятся только в том случае, если значение
в \Reg{0} (там $a$) было больше, чем значение в \Reg{4} (там $b$).

\myindex{ARM!\Instructions!ADRcc}
\myindex{ARM!\Instructions!BLcc}
Далее мы увидим инструкции \TT{ADREQ} и \TT{BLEQ}.
Они работают так же, как и \TT{ADR} и \TT{BL}, но исполнятся только если значения при последнем сравнении были равны.
Перед ними расположен ещё один \CMP, потому что вызов \printf мог испортить состояние флагов.

\myindex{ARM!\Instructions!LDMccFD}
\myindex{ARM!\Instructions!LDMFD}
Далее мы увидим \TT{LDMGEFD}. Эта инструкция работает так же, как и \TT{LDMFD}\footnote{\ac{LDMFD}}, 
но сработает только если в результате сравнения одно из значений было больше или равно второму (\IT{Greater or Equal}).
Смысл инструкции \TT{LDMGEFD SP!, \{R4-R6,PC\}} 
в том, что это как бы эпилог функции, но он сработает только если $a>=b$, только тогда работа 
функции закончится.

\myindex{Function epilogue}
Но если это не так, то есть $a<b$, то исполнение дойдет до следующей инструкции 
\TT{LDMFD SP!, \{R4-R6,LR\}}. Это ещё один эпилог функции. Эта инструкция восстанавливает состояние регистров
\TT{R4-R6}, но и \ac{LR} вместо \ac{PC}, таким образом, пока что, не делая возврата из функции.

Последние две инструкции вызывают \printf 
со строкой <<a<b\textbackslash{}n>> в качестве единственного аргумента.
Безусловный переход на \printf вместо возврата из функции мы уже рассматривали в секции
 <<\PrintfSeveralArgumentsSectionName>>~(\myref{ARM_B_to_printf}).

\myindex{ARM!\Instructions!ADRcc}
\myindex{ARM!\Instructions!BLcc}
\myindex{ARM!\Instructions!LDMccFD}
Функция \TT{f\_unsigned} точно такая же, но там используются инструкции \TT{ADRHI}, \TT{BLHI}, и \TT{LDMCSFD}. Эти предикаты
(\IT{HI = Unsigned higher, CS = Carry Set (greater than or equal)})
аналогичны рассмотренным, но служат для работы с беззнаковыми значениями.

В функции \main ничего нового для нас нет:

\lstinputlisting[caption=\main,style=customasmARM]{patterns/07_jcc/simple/ARM/ARM_O3_main.asm}

Так, в режиме ARM можно обойтись без условных переходов.

\myindex{Конвейер RISC}
Почему это хорошо? Читайте здесь: \myref{branch_predictors}.

\myindex{x86!\Instructions!CMOVcc}
В x86 нет аналогичной возможности, если не считать инструкцию \TT{CMOVcc}, это то же что и \MOV, 
но она срабатывает только при определенных выставленных флагах, обычно выставленных при помощи \CMP во время сравнения.

\mysubparagraph{\OptimizingKeilVI (\ThumbMode)}

\lstinputlisting[caption=\OptimizingKeilVI (\ThumbMode),style=customasmARM]{patterns/07_jcc/simple/ARM/ARM_thumb_signed.asm}

\myindex{ARM!\Instructions!BLE}
\myindex{ARM!\Instructions!BNE}
\myindex{ARM!\Instructions!BGE}
\myindex{ARM!\Instructions!BLS}
\myindex{ARM!\Instructions!BCS}
\myindex{ARM!\Instructions!B}
\myindex{ARM!\ThumbMode}
В режиме Thumb только инструкции \TT{B} могут быть дополнены условием исполнения (\IT{condition code}), 
так что код для режима Thumb выглядит привычнее.

\TT{BLE} это обычный переход с условием \IT{Less than or Equal}, 
\TT{BNE} --- \IT{Not Equal}, 
\TT{BGE} --- \IT{Greater than or Equal}.

Функция \TT{f\_unsigned} точно такая же, но для работы с беззнаковыми величинами 
там используются инструкции \TT{BLS} 
(\IT{Unsigned lower or same}) и \TT{BCS} (\IT{Carry Set (Greater than or equal)}).
}
\DE{\myparagraph{32-bit ARM}
\label{subsec:jcc_ARM}

\mysubparagraph{\OptimizingKeilVI (\ARMMode)}

\lstinputlisting[caption=\OptimizingKeilVI (\ARMMode),style=customasmARM]{patterns/07_jcc/simple/ARM/ARM_O3_signed.asm}

\myindex{ARM!Condition codes}
% FIXME \ref -> which instructions?
Viele Befehle im ARM mode können nur ausgeführt werden, wenn spezielle Flags gesetzt sind.
Dies ist beispielsweise oft beim Vergleich von Zahlen der Fall.

\myindex{ARM!\Instructions!ADD}
\myindex{ARM!\Instructions!ADDAL}
Der \ADD Befehl zum Beispiel heißt hier intern \TT{ADDAL}, wobei \TT{AL} für \IT{Always} (dt. immer) steht, d.h. er wird
immer ausgeführt.
Die Prädikate werden in den 4 höchstwertigsten Bits des 32-Bit-ARM-Befehls kodiert, dem \IT{condition field}.

\myindex{ARM!\Instructions!B}
Der Befehl \TT{B} für einen unbedingten Sprung ist tatsächlich doch bedingt und genau wie jeder andere bedingte Sprung
kodiert, nut dass er \TT{AL} im \IT{condition field} hat und dadurch die Flags ignoriert und immer ausgeführt wird.

\myindex{ARM!\Instructions!ADR}
\myindex{ARM!\Instructions!ADRcc}
\myindex{ARM!\Instructions!CMP}
Der Befehl \TT{ADRGT} arbeitet wie \TT{ADR}, wird aber nur ausgeführt, wenn das vorangehende \CMP ergeben hat, dass eine
der beiden Eingabezahlen größer war als die andere. 

\myindex{ARM!\Instructions!BL}
\myindex{ARM!\Instructions!BLcc}
% ToBeUpdated
Der folgende \TT{BLGT} Befehl verhält sich genau wie \TT{BL} und wird nur dann ausgeführt, wenn das Ergebnis des
Vergleichs das gleiche war (d.h. größer als).
\TT{ADRGT} schreibt einen Pointer auf den String \TT{a>b\textbackslash{}n} nach \Reg{0} und \TT{BLGT} ruft \printf auf.
Das heißt, Befehl mit dem Suffix \TT{-GT} werden nur ausgeführt, wenn der Wert in \Reg{0} (das ist $a$) größer ist als
der Wert in \Reg{4} (das ist $b$).

\myindex{ARM!\Instructions!ADRcc}
\myindex{ARM!\Instructions!BLcc}
Im Folgenden finden wir die Befehle \TT{ADREQ} und \TT{BLEQ}.
Sie verhalten sich wie \TT{ADR} und \TT{BL}, werden aber nur ausgeführt, wenn die beiden Operanden des letzten
Vergleichs gleich waren.
Ein weiteres \CMP befindet sich davor (denn die Ausführung von \printf könnte die Flags verändert haben).

\myindex{ARM!\Instructions!LDMccFD}
\myindex{ARM!\Instructions!LDMFD}
Dann finden wir \TT{LDMGEFD}; dieser Befehl arbeitet genau wie \TT{LDMFD}\footnote{\ac{LDMFD}}, wird aber nur
ausgeführt, wenn einer der Werte größer gleich dem anderen ist. 
Der Befehl \TT{LDMGEFD SP!, \{R4-R6,PC\}} fungiert als Funktionsepilog, wird aber nur ausgeführt, ewnn $a>=b$ und nur
dann wird die Funktionsausführung beendet.
\myindex{Function epilogue}
Wenn aber diese Bedingung nicht erfüllt ist, d.h. $a<b$, wird der Control Flow zum nächsten \\
\TT{\q{LDMFD SP!, \{R4-R6,LR\}}} springen, der ebenfalls einen Funktionsepilog darstellt. Dieser Befehl stellt nicht nur
den Zustand der \TT{R4-R6} Register wieder her, sondern auch \ac{LR} anstatt \ac{PC}, dadurch gibt er nichts aus der
Funktion zurück.
Die letzten beiden Befehle rufen \printf mit dem String <<a<b\textbackslash{}n>> als einzigem Argument auf.
Wir haben bereits einen unbedingten Sprung zur Funktion \printf anstelle einer Funktionsrückgabe im Abschnitt
<<\PrintfSeveralArgumentsSectionName>>~(\myref{ARM_B_to_printf}) untersucht.

\myindex{ARM!\Instructions!ADRcc}
\myindex{ARM!\Instructions!BLcc}
\myindex{ARM!\Instructions!LDMccFD}
\TT{f\_unsigned} ist ähnlich, nur die Befehle \TT{ADRHI}, \TT{BLHI} und \TT{LDMCSFD} werden hier verwendet.
Deren Prädikaten (\IT{HI = Unsigned higher, CS = Carry Set (greater than or equal)}) sind analog zu den eben
betrachteten, nur eben für vorzeichenlose Werte. 

In der Funktion \main finden wir nicht viel Neues:

\lstinputlisting[caption=\main,style=customasmARM]{patterns/07_jcc/simple/ARM/ARM_O3_main.asm}
Auf diese Weise kann man bedingte Sprünge im ARM mode entfernen.


\myindex{RISC pipeline}
Für eine Begründung warum dies vorteilhaft ist, siehe: \myref{branch_predictors}.

\myindex{x86!\Instructions!CMOVcc}
In x86 gibt es kein solches Feature, außer dem \TT{CMOVcc} Befehl, der genau wie \MOV funktioniert, aber nur ausgeführt
wird, wenn spezielle Flags - normalerweise durch \CMP - gesetzt sind.


\mysubparagraph{\OptimizingKeilVI (\ThumbMode)}

\lstinputlisting[caption=\OptimizingKeilVI (\ThumbMode),style=customasmARM]{patterns/07_jcc/simple/ARM/ARM_thumb_signed.asm}

\myindex{ARM!\Instructions!BLE}
\myindex{ARM!\Instructions!BNE}
\myindex{ARM!\Instructions!BGE}
\myindex{ARM!\Instructions!BLS}
\myindex{ARM!\Instructions!BCS}
\myindex{ARM!\Instructions!B}
\myindex{ARM!\ThumbMode}
Nur der \TT{B} Befehl im Thumb mode kann mit condition codes versehen werden, sodass der Thumb Code gewöhnlicher
aussieht.


\TT{BLE} ist ein normaler bedingter Sprung \IT{Less than or Equal}, 
\TT{BNE}---\IT{Not Equal}, 
\TT{BGE}---\IT{Greater than or Equal}.

\TT{f\_unsigned} ist ähnlich, nur dass andere Befehle verwendet werden, wenn mit vorzeichenlosen Werten umgegangen wird:
\TT{BLS} (\IT{Unsigned lower or same}) und \TT{BCS} (\IT{Carry Set (Greater than or equal)}).
}
\FR{\myparagraph{ARM 32-bit}
\label{subsec:jcc_ARM}

\mysubparagraph{\OptimizingKeilVI (\ARMMode)}

\lstinputlisting[caption=\OptimizingKeilVI (\ARMMode),style=customasmARM]{patterns/07_jcc/simple/ARM/ARM_O3_signed.asm}

\myindex{ARM!Condition codes}
% FIXME \ref -> which instructions?

Beaucoup d'instructions en mode ARM ne peuvent être exécutées que lorsque certains
flags sont mis.
E.g, ceci est souvent utilisé lorsque l'on compare les nombres

\myindex{ARM!\Instructions!ADD}
\myindex{ARM!\Instructions!ADDAL}

Par exemple, l'instruction \ADD est en fait appelée \TT{ADDAL} en interne, oú \TT{AL}
signifie \IT{Always}, i.e., toujours exécuter.
Les préficats sont encodés dans les 4 bits du haut des instructions ARM 32-bit. (\IT{condition field}).
\myindex{ARM!\Instructions!B}
L'instruction de saut inconditionnel \TT{B} est en fait conditionnelle et encodée
comme toutes les autres instructions de saut conditionnel, mais a \TT{AL} dans le
\IT{champ de condition}, et \IT{s'exécute toujours} (ALways), ignorants les flags.

\myindex{ARM!\Instructions!ADR}
\myindex{ARM!\Instructions!ADRcc}
\myindex{ARM!\Instructions!CMP}

L'instruction \TT{ADRGT} fonctionne comme \TT{ADR} mais ne s'exécute que dans le
cas oú l'instruction \CMP précédente a trouvé un des nombres plus grand que l'autre,
en comparant les deux (\IT{Greater Than}).

\myindex{ARM!\Instructions!BL}
\myindex{ARM!\Instructions!BLcc}

L'instruction \TT{BLGT} se comporte exactement comme \TT{BL} et n'est effectuée
que si le résultat de la comparaison était \IT{Greater Than} (plus grand).
\TT{ADRGT} écrit un pointeur sur la chaîne \TT{a>b\textbackslash{}n} dans \Reg{0}
et \TT{BLGT} appelle \printf.
Donc, les instructions suffixées par \TT{-GT} ne sont exécutées que si la valeur
dans \Reg{0} (qui est $a$) est plus grande que la valeur dans \Reg{4} (qui est $b$).

\myindex{ARM!\Instructions!ADRcc}
\myindex{ARM!\Instructions!BLcc}

En avançant, nous voyons les instructions \TT{ADREQ} et \TT{BLEQ}.
Elles se comportent comme \TT{ADR} et \TT{BL}, mais ne sont exécutées que si les
opérandes étaient égales lors de la dernière comparaison.
Un autre \CMP se trouve avant elles (car l'exécution de \printf pourrait avoir
modifiée les flags).

\myindex{ARM!\Instructions!LDMccFD}
\myindex{ARM!\Instructions!LDMFD}

Ensuite nous voyons \TT{LDMGEFD}, cette instruction fonctionne comme \TT{LDMFD}\footnote{\ac{LDMFD}},
mais n'est exécutée que si l'une des valeurs est supérieure ou égale à l'autre
(\IT{Greater or Equal}).
L'instruction \TT{LDMGEFD SP!, \{R4-R6,PC\}} se comporte comme une fonction épilogue,
mais elle ne sera exécutée que si $a>=b$, et seulement lorsque l'exécution de la
fonction se terminera.

\myindex{Function epilogue}

Mais si cette condition n'est pas satisfaite, i.e., $a<b$, alors le flux d'exécution
continue à l'instruction suivante, \TT{\q{LDMFD SP!, \{R4-R6,LR\}}}, qui est aussi
un épilogue de la fonction. Cette instruction ne restaure pas seulement l'état des
registres \TT{R4-R6}, mais aussi \ac{LR} au lieu de \ac{PC}, donc il ne retourne
pas de la fonction.
Les deux dernières instructions appellent \printf avec la chaîne <<a<b\textbackslash{}n>>
comme unique argument.
Nous avons déjà examiné un saut inconditionnel à la fonction \printf au lieu
d'un appel avec retour dans <<\PrintfSeveralArgumentsSectionName>> section~(\myref{ARM_B_to_printf}).

\myindex{ARM!\Instructions!ADRcc}
\myindex{ARM!\Instructions!BLcc}
\myindex{ARM!\Instructions!LDMccFD}
\TT{f\_unsigned} est très similaire, à part les instructions \TT{ADRHI}, \TT{BLHI},
et \TT{LDMCSFD} utilisées ici, ces prédicats (\IT{HI = Unsigned higher, CS = Carry
Set (greater than or equal)}) sont analogues à ceux examinés avant, mais pour des
valeurs non signées.

Il n'y a pas grand chose de nouveau pour nous dans la fonction \main:

\lstinputlisting[caption=\main,style=customasmARM]{patterns/07_jcc/simple/ARM/ARM_O3_main.asm}

C'est ainsi que vous pouvez vous débarrasser des sauts conditionnels en mode ARM.

\myindex{RISC pipeline}
Pourquoi est-ce que c'est si utile? Lire ici: \myref{branch_predictors}.

\myindex{x86!\Instructions!CMOVcc}

Il n'y a pas de telle caractéristique en x86, exceptée l'instruction \TT{CMOVcc},
qui est comme un \MOV, mais effectuée seulement lorsque certains flags sont mis,
en général mis par \CMP.

\mysubparagraph{\OptimizingKeilVI (\ThumbMode)}

\lstinputlisting[caption=\OptimizingKeilVI (\ThumbMode),style=customasmARM]{patterns/07_jcc/simple/ARM/ARM_thumb_signed.asm}

\myindex{ARM!\Instructions!BLE}
\myindex{ARM!\Instructions!BNE}
\myindex{ARM!\Instructions!BGE}
\myindex{ARM!\Instructions!BLS}
\myindex{ARM!\Instructions!BCS}
\myindex{ARM!\Instructions!B}
\myindex{ARM!\ThumbMode}

En mode Thumb, seules les instructions \TT{B} peuvent être complètées par un
\IT{condition codes}, (code de condition) donc le code Thumb paraît plus ordinaire.

\TT{BLE} est un saut conditionnel normal \IT{Less than or Equal} (inférieur ou égal),
\TT{BNE}---\IT{Not Equal} (non égal),
\TT{BGE}---\IT{Greater than or Equal} (supérieur ou égal).

\TT{f\_unsigned} est similaire, seules d'autres instructions sont utilisées
pour travailler avec des valeurs non-signées: \TT{BLS}
(\IT{Unsigned lower or same} non signée, inférieur ou égal) et \TT{BCS} (\IT{Carry
Set (Greater than or equal)} supérieur ou égal).
}

\EN{\myparagraph{ARM64: \Optimizing GCC (Linaro) 4.9}

\lstinputlisting[caption=f\_signed(),style=customasmARM]{patterns/07_jcc/simple/ARM/ARM64_GCC_O3_signed_EN.lst}

\lstinputlisting[caption=f\_unsigned(),style=customasmARM]{patterns/07_jcc/simple/ARM/ARM64_GCC_O3_unsigned_EN.lst}

The comments were added by the author of this book.
What is striking is that the compiler is not aware that some conditions are not possible at all,
so there is dead code at some places, which can never be executed.

\mysubparagraph{\Exercise}

Try to optimize these functions manually for size, removing redundant instructions, without adding new ones.

}
\RU{\myparagraph{ARM64: \Optimizing GCC (Linaro) 4.9}

\lstinputlisting[caption=f\_signed(),style=customasmARM]{patterns/07_jcc/simple/ARM/ARM64_GCC_O3_signed_RU.lst}

\lstinputlisting[caption=f\_unsigned(),style=customasmARM]{patterns/07_jcc/simple/ARM/ARM64_GCC_O3_unsigned_RU.lst}

Комментарии добавлены автором этой книги.
В глаза бросается то, что компилятор не в курсе, что некоторые ситуации невозможны,
поэтому кое-где в функциях остается код, который никогда не исполнится.

\mysubparagraph{\Exercise}

Попробуйте вручную оптимизировать функции по размеру, убрав избыточные инструкции и не добавляя новых.
}
\DE{\myparagraph{ARM64: \Optimizing GCC (Linaro) 4.9}

\lstinputlisting[caption=f\_signed(),style=customasmARM]{patterns/07_jcc/simple/ARM/ARM64_GCC_O3_signed_DE.lst}

\lstinputlisting[caption=f\_unsigned(),style=customasmARM]{patterns/07_jcc/simple/ARM/ARM64_GCC_O3_unsigned_DE.lst}
Die Kommentare stammen vom Autor. 
Erstaunlich ist hier, dass der Compiler nicht bemerkt, dass einige Bedingungen unmöglich zu erfüllen sind, sodass Dead
Code vorliegt, der nie ausgeführt werden kann.

\mysubparagraph{\Exercise}
Versuchen Sie die Funktionen manuell hinsichtlich Größe und Entfernen redundanter Befehle zu optimieren.
}
\FR{\myparagraph{ARM64: GCC (Linaro) 4.9 \Optimizing}

\lstinputlisting[caption=f\_signed(),style=customasmARM]{patterns/07_jcc/simple/ARM/ARM64_GCC_O3_signed_FR.lst}

\lstinputlisting[caption=f\_unsigned(),style=customasmARM]{patterns/07_jcc/simple/ARM/ARM64_GCC_O3_unsigned_FR.lst}

Les commentaires ont été ajoutés par l'auteur de ce livre.
Ce qui frappe ici, c'est que le compilateur n'est pas au courant que certaines conditions
ne sont pas possible du tout, donc il y a du code mort par endroit, qui ne sera jamais
exécuté.

\mysubparagraph{\Exercise}

Essayez d'optimiser manuellement la taille de ces fonctions, en supprimant les instructions
redondantes, sans en ajouter de nouvelles.

}


\EN{\subsubsection{MIPS}

One distinctive MIPS feature is the absence of flags.
Apparently, it was done to simplify the analysis of data dependencies.

\myindex{x86!\Instructions!SETcc}
\myindex{MIPS!\Instructions!SLT}
\myindex{MIPS!\Instructions!SLTU}

There are instructions similar to \INS{SETcc} in x86: \INS{SLT} (\q{Set on Less Than}: signed version) and 
\INS{SLTU} (unsigned version).
These instructions sets destination register value to 1 if the condition is true or to 0 if otherwise.

\myindex{MIPS!\Instructions!BEQ}
\myindex{MIPS!\Instructions!BNE}

The destination register is then checked using \INS{BEQ} (\q{Branch on Equal}) or \INS{BNE} (\q{Branch on Not Equal}) 
and a jump may occur.
So, this instruction pair has to be used in MIPS for comparison and branch.
Let's first start with the signed version of our function:

\lstinputlisting[caption=\NonOptimizing GCC 4.4.5 (IDA),style=customasmMIPS]{patterns/07_jcc/simple/O0_MIPS_signed_IDA_EN.lst}

\INS{SLT REG0, REG0, REG1} is reduced by IDA to its 
shorter form:\\
\INS{SLT REG0, REG1}.
\myindex{MIPS!\Pseudoinstructions!BEQZ}

We also see there \INS{BEQZ} pseudo instruction (\q{Branch if Equal to Zero}),\\
which are in fact \INS{BEQ REG, \$ZERO, LABEL}.

\myindex{MIPS!\Instructions!SLTU}

The unsigned version is just the same, but \INS{SLTU} (unsigned version, hence \q{U} in name) is used instead of \INS{SLT}:

\lstinputlisting[caption=\NonOptimizing GCC 4.4.5 (IDA),style=customasmMIPS]{patterns/07_jcc/simple/O0_MIPS_unsigned_IDA.lst}

}
\RU{\subsubsection{MIPS}

Одна отличительная особенность MIPS это отсутствие регистра флагов.
Очевидно, так было сделано для упрощения анализа зависимости данных (data dependency).

\myindex{x86!\Instructions!SETcc}
\myindex{MIPS!\Instructions!SLT}
\myindex{MIPS!\Instructions!SLTU}
Так что здесь есть инструкция, похожая на \INS{SETcc} в x86: \INS{SLT} (\q{Set on Less Than}~--- установить если
меньше чем, знаковая версия) и \INS{SLTU} (беззнаковая версия).
Эта инструкция устанавливает регистр-получатель в 1 если условие верно или в 0 в противном случае.

\myindex{MIPS!\Instructions!BEQ}
\myindex{MIPS!\Instructions!BNE}
Затем регистр-получатель проверяется, используя инструкцию 
\INS{BEQ} (\q{Branch on Equal} --- переход если равно) или \INS{BNE} (\q{Branch on Not Equal} --- переход если не равно) 
и может произойти переход.
Так что эта пара инструкций должна использоваться в MIPS для сравнения и перехода.
Начнем с знаковой версии нашей функции:

\lstinputlisting[caption=\NonOptimizing GCC 4.4.5 (IDA),style=customasmMIPS]{patterns/07_jcc/simple/O0_MIPS_signed_IDA_RU.lst}

\INS{SLT REG0, REG0, REG1} сокращается в IDA до более короткой формы \INS{SLT REG0, REG1}.
\myindex{MIPS!\Pseudoinstructions!BEQZ}
Мы также видим здесь псевдоинструкцию \INS{BEQZ} (\q{Branch if Equal to Zero}~--- переход если равно нулю), 
которая, на самом деле, \INS{BEQ REG, \$ZERO, LABEL}.

\myindex{MIPS!\Instructions!SLTU}
Беззнаковая версия точно такая же, только здесь используется \INS{SLTU} (беззнаковая версия, 
отсюда \q{U} в названии) вместо \INS{SLT}:

\lstinputlisting[caption=\NonOptimizing GCC 4.4.5 (IDA),style=customasmMIPS]{patterns/07_jcc/simple/O0_MIPS_unsigned_IDA.lst}

}
\DE{\subsubsection{MIPS}
Ein wesentliches Feature von MIPS ist das Fehlen von Flags.
Der Grund dafür ist offenbar, dass die Analyse von Datenabhängigeiten dadurch vereinfacht wird.


\myindex{x86!\Instructions!SETcc}
\myindex{MIPS!\Instructions!SLT}
\myindex{MIPS!\Instructions!SLTU}
Es gibt Befehle, die Ähnlichkeit mit \INS{SETcc} in x86 haben:\INS{SLT} (\q{Set on Less Than}: vorzeichenbehaftete
Version) und \INS{SLTU} (Version ohne Vorzeichen).
Diese Befehle setzen das Zielregister auf den Wert 1, falls die Bedingung wahr ist und ansonsten auf 0.


\myindex{MIPS!\Instructions!BEQ}
\myindex{MIPS!\Instructions!BNE}
Das Zielregister wird dann mit \INS{BEQ} (\q{Branch on Equal}) oder \INS{BNE} (\q{Branch on Not Equal}) überprüft und
gegebenenfalls ein Sprung ausgeführt. 
Dieses Befehlspaar muss also in MIPS für Vergleiche und Verzweigungen verwendet werden.
Beginnen wir mit der vorzeichenbehafteten Version unserer Funtion:

\lstinputlisting[caption=\NonOptimizing GCC 4.4.5
(IDA),style=customasmMIPS]{patterns/07_jcc/simple/O0_MIPS_signed_IDA_DE.lst}

\INS{SLT REG0, REG0, REG1} wird von IDA auf seine kürzere Form reduziert:\\
\INS{SLT REG0, REG1}.
\myindex{MIPS!\Pseudoinstructions!BEQZ}
Wir finden dort auch den Pseudo-Befehl \INS{BEQZ} (\q{Branch if Equal to Zero}), die \INS{BEQ REG, \$ZERO, LABEL}
entspricht.

\myindex{MIPS!\Instructions!SLTU}
Die vorzeichenlose Version ist identisch, aber \INS{SLTU} (vorzeichenlose Version, daher das \q{U} im Namen) wird
anstelle von \INS{SLT} verwendet:


\lstinputlisting[caption=\NonOptimizing GCC 4.4.5 (IDA),style=customasmMIPS]{patterns/07_jcc/simple/O0_MIPS_unsigned_IDA.lst}

}
\FR{\subsubsection{MIPS}

Une des caractéristiques distinctives de MIPS est l'absence de flag. 
Apparemment, cela a été fait pour simplifier l'analyse des dépendances de données.

\myindex{x86!\Instructions!SETcc}
\myindex{MIPS!\Instructions!SLT}
\myindex{MIPS!\Instructions!SLTU}

Il y a des instructions similaires à \INS{SETcc} en x86: \INS{SLT} (\q{Set on Less Than}:
mettre si plus petit que, version signée) et \INS{SLTU} (version non signée).
Ces instructions mettent le registre de destination à 1 si la condition est vraie
ou à 0 autrement.

\myindex{MIPS!\Instructions!BEQ}
\myindex{MIPS!\Instructions!BNE}

Le registre de destination est ensuite testé avec \INS{BEQ} (\q{Branch on Equal}
branchement si égal) ou \INS{BNE} (\q{Branch on Not Equal} branchement si non égal)
et un saut peut survenir.
Donc, cette paire d'instructions doit être utilisée en MIPS pour comparer et effectuer
un branchement.
Essayons avec la version signée de notre fonction:

\lstinputlisting[caption=GCC 4.4.5 \NonOptimizing (IDA),style=customasmMIPS]{patterns/07_jcc/simple/O0_MIPS_signed_IDA_FR.lst}

\INS{SLT REG0, REG0, REG1} est réduit par IDA à sa forme plus courte:\\
\INS{SLT REG0, REG1}. 
\myindex{MIPS!\Pseudoinstructions!BEQZ}

Nous voyons également ici la pseudo instruction \INS{BEQZ} (\q{Branch if Equal to Zero}
branchement si égal à zéro),\\
qui est en fait \INS{BEQ REG, \$ZERO, LABEL}.

\myindex{MIPS!\Instructions!SLTU}

La version non signée est la même, mais \INS{SLTU} (version non signée, d'où
le \q{U} de unsigned) est utilisée au lieu de \INS{SLT}:

\lstinputlisting[caption=GCC 4.4.5 \NonOptimizing (IDA),style=customasmMIPS]{patterns/07_jcc/simple/O0_MIPS_unsigned_IDA.lst}

}


\EN{\subsection{Calculating absolute value}
\label{sec:abs}

A simple function:

\lstinputlisting[style=customc]{abs.c}

\subsubsection{\Optimizing MSVC}

This is how the code is usually generated:

\lstinputlisting[caption=\Optimizing MSVC 2012 x64,style=customasmx86]{patterns/07_jcc/abs/abs_MSVC2012_Ox_x64_EN.asm}

GCC 4.9 does mostly the same.

\subsubsection{\OptimizingKeilVI: \ThumbMode}

\lstinputlisting[caption=\OptimizingKeilVI: \ThumbMode,style=customasmARM]{patterns/07_jcc/abs/abs_Keil_thumb_O3_EN.s}

\myindex{ARM!\Instructions!RSB}

ARM lacks a negate instruction, so the Keil compiler uses the \q{Reverse Subtract} instruction, which just subtracts with reversed operands.

\subsubsection{\OptimizingKeilVI: \ARMMode}

It is possible to add condition codes to some instructions in ARM mode, so that is what the Keil compiler does:

\lstinputlisting[caption=\OptimizingKeilVI: \ARMMode,style=customasmARM]{patterns/07_jcc/abs/abs_Keil_ARM_O3_EN.s}

Now there are no conditional jumps and this is good: \myref{branch_predictors}.

\subsubsection{\NonOptimizing GCC 4.9 (ARM64)}

\myindex{ARM!\Instructions!XOR}

ARM64 has instruction \INS{NEG} for negating:

\lstinputlisting[caption=\Optimizing GCC 4.9 (ARM64),style=customasmARM]{patterns/07_jcc/abs/abs_GCC49_ARM64_O0_EN.s}

\subsubsection{MIPS}

\lstinputlisting[caption=\Optimizing GCC 4.4.5 (IDA),style=customasmMIPS]{patterns/07_jcc/abs/MIPS_O3_IDA_EN.lst}

\myindex{MIPS!\Instructions!BLTZ}
Here we see a new instruction: \INS{BLTZ} (\q{Branch if Less Than Zero}).
\myindex{MIPS!\Instructions!SUBU}
\myindex{MIPS!\Pseudoinstructions!NEGU}

There is also the \INS{NEGU} pseudo instruction, which just does subtraction from zero.
The \q{U} suffix in both \INS{SUBU} and \INS{NEGU} implies that no exception to be raised in case of integer overflow.

\subsubsection{Branchless version?}

You could have also a branchless version of this code. This we will review later: \myref{chap:branchless_abs}.
}
\RU{\subsection{Вычисление абсолютной величины}
\label{sec:abs}

Это простая функция:

\lstinputlisting[style=customc]{abs.c}

\subsubsection{\Optimizing MSVC}

Обычный способ генерации кода:

\lstinputlisting[caption=\Optimizing MSVC 2012 x64,style=customasmx86]{patterns/07_jcc/abs/abs_MSVC2012_Ox_x64_RU.asm}

GCC 4.9 делает почти то же самое.

\subsubsection{\OptimizingKeilVI: \ThumbMode}

\lstinputlisting[caption=\OptimizingKeilVI: \ThumbMode,style=customasmARM]{patterns/07_jcc/abs/abs_Keil_thumb_O3_RU.s}

\myindex{ARM!\Instructions!RSB}
В ARM нет инструкции для изменения знака, так что компилятор Keil использует инструкцию \q{Reverse Subtract},
которая просто вычитает, но с операндами, переставленными наоборот.

\subsubsection{\OptimizingKeilVI: \ARMMode}

В режиме ARM можно добавлять коды условий к некоторым инструкций, что компилятор Keil и сделал:

\lstinputlisting[caption=\OptimizingKeilVI: \ARMMode,style=customasmARM]{patterns/07_jcc/abs/abs_Keil_ARM_O3_RU.s}

Теперь здесь нет условных переходов и это хорошо:
 
\myref{branch_predictors}.

\subsubsection{\NonOptimizing GCC 4.9 (ARM64)}

\myindex{ARM!\Instructions!XOR}
В ARM64 есть инструкция \INS{NEG} для смены знака:

\lstinputlisting[caption=\Optimizing GCC 4.9 (ARM64),style=customasmARM]{patterns/07_jcc/abs/abs_GCC49_ARM64_O0_RU.s}

\subsubsection{MIPS}

\lstinputlisting[caption=\Optimizing GCC 4.4.5 (IDA),style=customasmMIPS]{patterns/07_jcc/abs/MIPS_O3_IDA_RU.lst}

\myindex{MIPS!\Instructions!BLTZ}
Видим здесь новую инструкцию: \INS{BLTZ} (\q{Branch if Less Than Zero}).
\myindex{MIPS!\Instructions!SUBU}
\myindex{MIPS!\Pseudoinstructions!NEGU}
Тут есть также псевдоинструкция \INS{NEGU}, которая на самом деле вычитает из нуля.
Суффикс \q{U} в обоих инструкциях \INS{SUBU} и \INS{NEGU} означает, что при целочисленном переполнении исключение не
сработает.

\subsubsection{Версия без переходов?}

Возможна также версия и без переходов, мы рассмотрим её позже: \myref{chap:branchless_abs}.
}
\DE{\subsection{Betrag berechnen}
\label{sec:abs}

Eine einfache Funktion:

\lstinputlisting[style=customc]{abs.c}

\subsubsection{\Optimizing MSVC}

Normalerweise wird folgender Code erzeugt:

\lstinputlisting[caption=\Optimizing MSVC 2012 x64,style=customasmx86]{patterns/07_jcc/abs/abs_MSVC2012_Ox_x64_DE.asm}

GCC 4.9 macht ungefähr das gleiche.

\subsubsection{\OptimizingKeilVI: \ThumbMode}

\lstinputlisting[caption=\OptimizingKeilVI: \ThumbMode,style=customasmARM]{patterns/07_jcc/abs/abs_Keil_thumb_O3_DE.s}

\myindex{ARM!\Instructions!RSB}
ARM fehlt ein Befehl zur Negation, sodass der Keil Compiler den \q{Reverse
Subtract} Befehl verwendet, der mit umgekehrten Operanden subtrahiert.

\subsubsection{\OptimizingKeilVI: \ARMMode}
Es ist im ARM mode möglich, einigen Befehlen condition codes hinzuzufügen und genau das tut der Keil Compiler:


\lstinputlisting[caption=\OptimizingKeilVI: \ARMMode,style=customasmARM]{patterns/07_jcc/abs/abs_Keil_ARM_O3_DE.s}
Jetzt sind keine bedingten Sprünge mehr übrig und das ist vorteilhaft: \myref{branch_predictors}.


\subsubsection{\NonOptimizing GCC 4.9 (ARM64)}

\myindex{ARM!\Instructions!XOR}

ARM64 kennt den Befehl \INS{NEG} zum Negieren:

\lstinputlisting[caption=\Optimizing GCC 4.9 (ARM64),style=customasmARM]{patterns/07_jcc/abs/abs_GCC49_ARM64_O0_DE.s}

\subsubsection{MIPS}

\lstinputlisting[caption=\Optimizing GCC 4.4.5 (IDA),style=customasmMIPS]{patterns/07_jcc/abs/MIPS_O3_IDA_DE.lst}

\myindex{MIPS!\Instructions!BLTZ}
Hier finden wir einen neuen Befehl: \INS{BLTZ} (\q{Branch if Less Than Zero}).
\myindex{MIPS!\Instructions!SUBU}
\myindex{MIPS!\Pseudoinstructions!NEGU}
Es gibt zusätzlich noch den \INS{NEGU} Pseudo-Befehl, der eine Subtraktion von Null durchführt. Der Suffix \q{U} bei
\INS{SUBU} und \INS{NEGU} zeigt an, dass keine Exception für den Fall eines Integer Overflows geworfen wird.


\subsubsection{Verzweigungslose Version?}
Man kann auch eine verzweigungslose Version dieses Codes erzeugen. Dies werden wir später betrachten:
\myref{chap:branchless_abs}. 
}
\FR{\subsection{Calcul de valeur absolue}
\label{sec:abs}

Une fonction simple:

\lstinputlisting[style=customc]{abs.c}

\subsubsection{MSVC \Optimizing}

Ceci est le code généré habituellement:

\lstinputlisting[caption=MSVC 2012 x64 \Optimizing,style=customasmx86]{patterns/07_jcc/abs/abs_MSVC2012_Ox_x64_FR.asm}

GCC 4.9 génère en gros le même code:

\subsubsection{\OptimizingKeilVI: \ThumbMode}

\lstinputlisting[caption=\OptimizingKeilVI: \ThumbMode,style=customasmARM]{patterns/07_jcc/abs/abs_Keil_thumb_O3_FR.s}

\myindex{ARM!\Instructions!RSB}

Il manque une instruction de négation en ARM, donc le compilateur Keil utilise l'instruction
\q{Reverse Subtract}, qui soustrait la valeur du registre de l'opérande.

\subsubsection{\OptimizingKeilVI: \ARMMode}

Il est possible d'ajouter un code de condition à certaines instructions en mode
ARM, c'est donc ce que fait le compilateur Keil:

\lstinputlisting[caption=\OptimizingKeilVI: \ARMMode,style=customasmARM]{patterns/07_jcc/abs/abs_Keil_ARM_O3_FR.s}

Maintenant, il n'y a plus de saut conditionnel et c'est mieux: \myref{branch_predictors}.

\subsubsection{GCC 4.9 \NonOptimizing (ARM64)}

\myindex{ARM!\Instructions!XOR}

ARM64 possède l'instruction \INS{NEG} pour effectuer la négation:

\lstinputlisting[caption=GCC 4.9 \Optimizing (ARM64),style=customasmARM]{patterns/07_jcc/abs/abs_GCC49_ARM64_O0_FR.s}

\subsubsection{MIPS}

\lstinputlisting[caption=GCC 4.4.5 \Optimizing (IDA),style=customasmMIPS]{patterns/07_jcc/abs/MIPS_O3_IDA_FR.lst}

\myindex{MIPS!\Instructions!BLTZ}
Nous voyons ici une nouvelle instruction: \INS{BLTZ} (\q{Branch if Less Than Zero}
branchement si plus petit que zéro).
\myindex{MIPS!\Instructions!SUBU}
\myindex{MIPS!\Pseudoinstructions!NEGU}

Il y a aussi la pseudo-instruction \INS{NEGU}, qui effectue une soustraction à zéro.
Le suffixe \q{U} dans les deux instructions \INS{SUBU} et \INS{NEGU} indique qu'aucune
exception ne sera levée en cas de débordement de la taille d'un entier.

\subsubsection{Version sans branchement?}

Vous pouvez aussi avoir une version sans branchement de ce code. Ceci sera revu plus
tard: \myref{chap:branchless_abs}.

}

\EN{\subsection{Ternary conditional operator}
\label{chap:cond}

The ternary conditional operator in \CCpp has the following syntax:

\begin{lstlisting}
expression ? expression : expression
\end{lstlisting}

Here is an example:

\lstinputlisting[style=customc]{patterns/07_jcc/cond_operator/cond.c}

\subsubsection{x86}

Old and non-optimizing compilers generate assembly code just as if an \TT{if/else} statement was used:

\lstinputlisting[caption=\NonOptimizing MSVC 2008,style=customasmx86]{patterns/07_jcc/cond_operator/MSVC2008_EN.asm}

\lstinputlisting[caption=\Optimizing MSVC 2008,style=customasmx86]{patterns/07_jcc/cond_operator/MSVC2008_Ox_EN.asm}

Newer compilers are more concise:

\lstinputlisting[caption=\Optimizing MSVC 2012 x64,style=customasmx86]{patterns/07_jcc/cond_operator/MSVC2012_Ox_x64_EN.asm}

\myindex{x86!\Instructions!CMOVcc}
\Optimizing GCC 4.8 for x86 also uses the \TT{CMOVcc} instruction, while the non-optimizing GCC 4.8 uses conditional jumps.

\subsubsection{ARM}

\myindex{x86!\Instructions!ADRcc}
\Optimizing Keil for ARM mode also uses the conditional instructions \TT{ADRcc}:

\lstinputlisting[label=cond_Keil_ARM_O3,caption=\OptimizingKeilVI (\ARMMode),style=customasmARM]{patterns/07_jcc/cond_operator/Keil_ARM_O3_EN.s}

Without manual intervention, the two instructions \TT{ADREQ} and \TT{ADRNE} cannot be executed in the same run.

\Optimizing Keil for Thumb mode needs to use conditional jump instructions, since there are no load instructions
that support conditional flags:

\lstinputlisting[caption=\OptimizingKeilVI (\ThumbMode),style=customasmARM]{patterns/07_jcc/cond_operator/Keil_thumb_O3_EN.s}

\subsubsection{ARM64}

\Optimizing GCC (Linaro) 4.9 for ARM64 also uses conditional jumps:

\lstinputlisting[label=cond_ARM64,caption=\Optimizing GCC (Linaro) 4.9,style=customasmARM]{patterns/07_jcc/cond_operator/ARM64_GCC_O3_EN.s}

That is because ARM64 does not have a simple load instruction with conditional flags,
like \TT{ADRcc} in 32-bit ARM mode or \INS{CMOVcc} in x86.

\myindex{ARM!\Instructions!CSEL}
It has, however, \q{Conditional SELect} instruction (\TT{CSEL})\InSqBrackets{\ARMSixFourRef p390, C5.5},
but GCC 4.9 does not seem to be smart enough to use it in such piece of code.

\subsubsection{MIPS}

Unfortunately, GCC 4.4.5 for MIPS is not very smart, either:

\lstinputlisting[caption=\Optimizing GCC 4.4.5 (\assemblyOutput),style=customasmMIPS]{patterns/07_jcc/cond_operator/MIPS_O3_EN.s}

\subsubsection{Let's rewrite it in an \TT{if/else} way}

\lstinputlisting[style=customc]{patterns/07_jcc/cond_operator/cond2.c}

\myindex{x86!\Instructions!CMOVcc}

Interestingly, optimizing GCC 4.8 for x86 was also able to use \TT{CMOVcc} in this case:

\lstinputlisting[caption=\Optimizing GCC 4.8,style=customasmx86]{patterns/07_jcc/cond_operator/cond2_GCC_O3_EN.s}

\Optimizing Keil in ARM mode generates code identical to \lstref{cond_Keil_ARM_O3}.

But the optimizing MSVC 2012 is not that good (yet).

\subsubsection{\Conclusion{}}

Why optimizing compilers try to get rid of conditional jumps? Read here about it: \myref{branch_predictors}.
}
\RU{\subsection{Тернарный условный оператор}
\label{chap:cond}

Тернарный условный оператор (ternary conditional operator) в \CCpp это:

\begin{lstlisting}
expression ? expression : expression
\end{lstlisting}

И вот пример:

\lstinputlisting[style=customc]{patterns/07_jcc/cond_operator/cond.c}

\subsubsection{x86}

Старые и неоптимизирующие компиляторы генерируют код так, как если бы выражение \TT{if/else} было использовано
вместо него:

\lstinputlisting[caption=\NonOptimizing MSVC 2008,style=customasmx86]{patterns/07_jcc/cond_operator/MSVC2008_RU.asm}

\lstinputlisting[caption=\Optimizing MSVC 2008,style=customasmx86]{patterns/07_jcc/cond_operator/MSVC2008_Ox_RU.asm}

Новые компиляторы могут быть более краткими:

\lstinputlisting[caption=\Optimizing MSVC 2012 x64,style=customasmx86]{patterns/07_jcc/cond_operator/MSVC2012_Ox_x64_RU.asm}

\myindex{x86!\Instructions!CMOVcc}
\Optimizing GCC 4.8 для x86 также использует инструкцию \TT{CMOVcc},
тогда как неоптимизирующий GCC 4.8 использует условные переходы.

\subsubsection{ARM}

\myindex{x86!\Instructions!ADRcc}
\Optimizing Keil для режима ARM также использует инструкцию \INS{ADRcc}, срабатывающую при некотором
условии:

\lstinputlisting[label=cond_Keil_ARM_O3,caption=\OptimizingKeilVI (\ARMMode),style=customasmARM]{patterns/07_jcc/cond_operator/Keil_ARM_O3_RU.s}

Без внешнего вмешательства инструкции \TT{ADREQ} и \TT{ADRNE} никогда не исполнятся одновременно.
\Optimizing Keil для режима Thumb вынужден использовать инструкции условного перехода, потому
что тут нет инструкции загрузки значения, поддерживающей флаги условия:

\lstinputlisting[caption=\OptimizingKeilVI (\ThumbMode),style=customasmARM]{patterns/07_jcc/cond_operator/Keil_thumb_O3_RU.s}

\subsubsection{ARM64}

\Optimizing GCC (Linaro) 4.9 для ARM64 также использует условные переходы:

\lstinputlisting[label=cond_ARM64,caption=\Optimizing GCC (Linaro) 4.9,style=customasmARM]{patterns/07_jcc/cond_operator/ARM64_GCC_O3_RU.s}

Это потому что в ARM64 нет простой инструкции загрузки с флагами условия, как \TT{ADRcc} в 32-битном 
режиме ARM или \TT{CMOVcc} в x86.

\myindex{ARM!\Instructions!CSEL}
Но с другой стороны, там есть инструкция \TT{CSEL} (\q{Conditional SELect})
\InSqBrackets{\ARMSixFourRef p390, C5.5},
но GCC 4.9 наверное, пока не так
хорош, чтобы генерировать её в таком фрагменте кода

\subsubsection{MIPS}

GCC 4.4.5 для MIPS тоже не так хорош, к сожалению:

\lstinputlisting[caption=\Optimizing GCC 4.4.5 (\assemblyOutput),style=customasmMIPS]{patterns/07_jcc/cond_operator/MIPS_O3_RU.s}

\subsubsection{Перепишем, используя обычный \TT{if/else}}

\lstinputlisting[style=customc]{patterns/07_jcc/cond_operator/cond2.c}

\myindex{x86!\Instructions!CMOVcc}
Интересно, оптимизирующий GCC 4.8 для x86 также может генерировать \TT{CMOVcc} в этом случае:

\lstinputlisting[caption=\Optimizing GCC 4.8,style=customasmx86]{patterns/07_jcc/cond_operator/cond2_GCC_O3_RU.s}

\Optimizing Keil в режиме ARM генерирует код идентичный этому: \lstref{cond_Keil_ARM_O3}.

Но оптимизирующий MSVC 2012 пока не так хорош.

\subsubsection{\Conclusion{}}

Почему оптимизирующие компиляторы стараются избавиться от условных переходов? Читайте больше об этом здесь:
 \myref{branch_predictors}.
}
\DE{\subsection{Ternärer Vergleichsoperator}
\label{chap:cond}
Der ternäre Vergleichsoperator in \CCpp hat die folgende Syntax:

\begin{lstlisting}
expression ? expression : expression
\end{lstlisting}

Hier ist ein Beispiel:

\lstinputlisting[style=customc]{patterns/07_jcc/cond_operator/cond.c}

\subsubsection{x86}
Alte und nicht optimerende Compiler erzeugen Assemblercode als wenn ein \TT{if/else} Ausdruck verwendet wurde:

\lstinputlisting[caption=\NonOptimizing MSVC 2008,style=customasmx86]{patterns/07_jcc/cond_operator/MSVC2008_DE.asm}

\lstinputlisting[caption=\Optimizing MSVC 2008,style=customasmx86]{patterns/07_jcc/cond_operator/MSVC2008_Ox_DE.asm}

Neuere Compiler sind ein wenig präziser:

\lstinputlisting[caption=\Optimizing MSVC 2012
x64,style=customasmx86]{patterns/07_jcc/cond_operator/MSVC2012_Ox_x64_DE.asm}

\myindex{x86!\Instructions!CMOVcc}
\Optimizing GCC 4.8 für x86 verwendet ebenfalls den \TT{CMOVcc} Befehl, wohingegen der nicht optimierende GCC 4.8
bedingte Sprünge verwendet.

\subsubsection{ARM}

\myindex{x86!\Instructions!ADRcc}
\Optimizing Keil im ARM mode verwendet ebenfalls bedingte Sprungbefehle \TT{ADRcc}:

\lstinputlisting[label=cond_Keil_ARM_O3,caption=\OptimizingKeilVI
(\ARMMode),style=customasmARM]{patterns/07_jcc/cond_operator/Keil_ARM_O3_DE.s}

Ohne manuellen Eingriff können die beiden Befehle \TT{ADREQ} und \TT{ADRNE} nicht in einem Durchlauf ausgeführt werden.

\Optimizing Keil für Thumb mode muss bedingte Sprungbefehle verwenden, da es keine Ladebefehle gibt, die
Bedingungsflags unterstützen.

\lstinputlisting[caption=\OptimizingKeilVI
(\ThumbMode),style=customasmARM]{patterns/07_jcc/cond_operator/Keil_thumb_O3_DE.s}

\subsubsection{ARM64}

\Optimizing GCC (Linaro) 4.9 für ARM64 verwendet auch bedingte Sprünge:

\lstinputlisting[label=cond_ARM64,caption=\Optimizing GCC (Linaro)
4.9,style=customasmARM]{patterns/07_jcc/cond_operator/ARM64_GCC_O3_DE.s} Das liegt daran, dass ARM64 über keinen
einfachen Ladebefehl mit Bedingungsflags verfügt wie z.B. \TT{ADRcc} im 32-Bit-ARM-Modus oder \INS{CMOVcc} in x86.

\myindex{ARM!\Instructions!CSEL}
Es gibt dafür den \q{Conditional SELect} Befehl (\TT{CSEL})\InSqBrackets{\ARMSixFourRef p390, C5.5}, aber GCC 4.9
scheint nicht ausgereift genug zu sein um ihn in einem solchen Codestück zu verwenden.

\subsubsection{MIPS}

Leider ist GCC 4.4.5 für MIPS auch nicht besser:

\lstinputlisting[caption=\Optimizing GCC 4.4.5
(\assemblyOutput),style=customasmMIPS]{patterns/07_jcc/cond_operator/MIPS_O3_DE.s}

\subsubsection{Schreiben wir es mit \TT{if/else}}

\lstinputlisting[style=customc]{patterns/07_jcc/cond_operator/cond2.c}

\myindex{x86!\Instructions!CMOVcc}
Interessanterweise war der optimierende GCC 4.8 für x86 ebenfalls in der Lage \TT{CMOVcc} hier zu verwenden:

\lstinputlisting[caption=\Optimizing GCC 4.8,style=customasmx86]{patterns/07_jcc/cond_operator/cond2_GCC_O3_DE.s}

\Optimizing Keil im ARM mode erzeugt identischen Code zu \lstref{cond_Keil_ARM_O3}.
Der optimierende MSVC 2012 ist hingegen (noch) nicht so gut.


\subsubsection{\Conclusion{}}
Warum versuchen optimierende Compiler bedingte Sprünge zu entfernen? Mehr dazu finden Sie hier:
\myref{branch_predictors}.
}
\FR{\subsection{Opérateur conditionnel ternaire}
\label{chap:cond}

L'opérateur conditionnel ternaire en \CCpp a la syntaxe suivante:

\begin{lstlisting}
expression ? expression : expression
\end{lstlisting}

Voici un exemple:

\lstinputlisting[style=customc]{patterns/07_jcc/cond_operator/cond.c}

\subsubsection{x86}

Les vieux compilateurs et ceux sans optimisation génèrent du code assembleur comme
si des instructions \TT{if/else} avaient été utilisées:

\lstinputlisting[caption=MSVC 2008 \NonOptimizing,style=customasmx86]{patterns/07_jcc/cond_operator/MSVC2008_FR.asm}

\lstinputlisting[caption=MSVC 2008 \Optimizing ,style=customasmx86]{patterns/07_jcc/cond_operator/MSVC2008_Ox_FR.asm}

Les nouveaux compilateurs sont plus concis:

\lstinputlisting[caption=MSVC 2012 x64 \Optimizing,style=customasmx86]{patterns/07_jcc/cond_operator/MSVC2012_Ox_x64_FR.asm}

\myindex{x86!\Instructions!CMOVcc}
GCC 4.8 \Optimizing pour x86 utilise également l'instruction \TT{CMOVcc}, tandis
que GCC 4.8 \NonOptimizing utilise des sauts conditionnels.

\subsubsection{ARM}

\myindex{x86!\Instructions!ADRcc}
Keil \Optimizing pour le mode ARM utilise les instructions conditionnelles \TT{ADRcc}:

\lstinputlisting[label=cond_Keil_ARM_O3,caption=\OptimizingKeilVI (\ARMMode),style=customasmARM]{patterns/07_jcc/cond_operator/Keil_ARM_O3_FR.s}

Sans intervention manuelle, les deux instructions \TT{ADREQ} et \TT{ADRNE} ne peuvent
être exécutées lors de la même exécution.

Keil \Optimizing pour le mode Thumb à besoin d'utiliser des instructions de saut
conditionnel, puisqu'il n'y a pas d'instruction qui supporte le flag conditionnel.

\lstinputlisting[caption=\OptimizingKeilVI (\ThumbMode),style=customasmARM]{patterns/07_jcc/cond_operator/Keil_thumb_O3_FR.s}

\subsubsection{ARM64}

GCC (Linaro) 4.9 \Optimizing pour ARM64 utilise aussi des sauts conditionnels:

\lstinputlisting[label=cond_ARM64,caption=GCC (Linaro) 4.9 \Optimizing,style=customasmARM]{patterns/07_jcc/cond_operator/ARM64_GCC_O3_FR.s}

C'est parce qu'ARM64 n'a pas d'instruction de chargement simple avec le flag conditionnel
comme \TT{ADRcc} en ARM 32-bit ou \INS{CMOVcc} en x86.

\myindex{ARM!\Instructions!CSEL}
Il a toutefois l'instruction \q{Conditional SELect} (\TT{CSEL})\InSqBrackets{\ARMSixFourRef p390, C5.5},
mais GCC 4.9 ne semble pas assez malin pour l'utiliser dans un tel morceau de code.

\subsubsection{MIPS}

Malheureusement, GCC 4.4.5 pour MIPS n'est pas très malin non plus:

\lstinputlisting[caption=GCC 4.4.5 \Optimizing (\assemblyOutput),style=customasmMIPS]{patterns/07_jcc/cond_operator/MIPS_O3_FR.s}

\subsubsection{Récrivons-le à l'aide d'un\TT{if/else}}

\lstinputlisting[style=customc]{patterns/07_jcc/cond_operator/cond2.c}

\myindex{x86!\Instructions!CMOVcc}

Curieusement, GCC 4,8 avec l'optimisation a pû utiliser \TT{CMOVcc} dans ce cas:

\lstinputlisting[caption=GCC 4.8 \Optimizing,style=customasmx86]{patterns/07_jcc/cond_operator/cond2_GCC_O3_FR.s}

Keil avec optimisation génère un code identique à \lstref{cond_Keil_ARM_O3}.

Mais MSVC 2012 avec optimisation n'est pas (encore) si bon.

\subsubsection{\Conclusion{}}

Pourquoi est-ce que les compilateurs qui optimisent essayent de se débarrasser des
sauts conditionnels? Voir à ce propos: \myref{branch_predictors}.
}

\EN{\subsection{Getting minimal and maximal values}

\subsubsection{32-bit}

\lstinputlisting[style=customc]{patterns/07_jcc/minmax/minmax.c}

\lstinputlisting[caption=\NonOptimizing MSVC 2013,style=customasmx86]{patterns/07_jcc/minmax/minmax_MSVC_2013_EN.asm}

\myindex{x86!\Instructions!Jcc}

These two functions differ only in the conditional jump instruction: 
\INS{JGE} (\q{Jump if Greater or Equal}) is used in the first one
and \INS{JLE} (\q{Jump if Less or Equal}) in the second.

\myindex{\CompilerAnomaly}
\label{MSVC_double_JMP_anomaly}

There is one unneeded \JMP instruction in each function, which MSVC presumably left by mistake.

\myparagraph{Branchless}

ARM for Thumb mode reminds us of x86 code:

\lstinputlisting[caption=\OptimizingKeilVI (\ThumbMode),style=customasmARM]{patterns/07_jcc/minmax/minmax_Keil_Thumb_O3_EN.s}

\myindex{ARM!\Instructions!Bcc}

The functions differ in the branching instruction: \INS{BGT} and \INS{BLT}.
It's possible to use conditional suffixes in ARM mode, so the code is shorter.

\myindex{ARM!\Instructions!MOVcc}
\INS{MOVcc} is to be executed only if the condition is met:

\lstinputlisting[caption=\OptimizingKeilVI (\ARMMode),style=customasmARM]{patterns/07_jcc/minmax/minmax_Keil_ARM_O3_EN.s}

\myindex{x86!\Instructions!CMOVcc}
\Optimizing GCC 4.8.1 and optimizing MSVC 2013 can use \INS{CMOVcc} instruction, which is analogous to \INS{MOVcc} in ARM:

\lstinputlisting[caption=\Optimizing MSVC 2013,style=customasmx86]{patterns/07_jcc/minmax/minmax_GCC481_O3_EN.s}

\subsubsection{64-bit}

\lstinputlisting[style=customc]{patterns/07_jcc/minmax/minmax64.c}

There is some unneeded value shuffling, but the code is comprehensible:

\lstinputlisting[caption=\NonOptimizing GCC 4.9.1 ARM64,style=customasmARM]{patterns/07_jcc/minmax/minmax64_GCC_49_ARM64_O0.s}

\myparagraph{Branchless}

No need to load function arguments from the stack, as they are already in the registers:

\lstinputlisting[caption=\Optimizing GCC 4.9.1 x64,style=customasmx86]{patterns/07_jcc/minmax/minmax64_GCC_49_x64_O3_EN.s}

MSVC 2013 does almost the same.

\myindex{ARM!\Instructions!CSEL}

ARM64 has the \INS{CSEL} instruction, which works just as \INS{MOVcc} in ARM or \INS{CMOVcc} in x86, just the name is different:
\q{Conditional SELect}.

\lstinputlisting[caption=\Optimizing GCC 4.9.1 ARM64,style=customasmARM]{patterns/07_jcc/minmax/minmax64_GCC_49_ARM64_O3_EN.s}

\subsubsection{MIPS}

Unfortunately, GCC 4.4.5 for MIPS is not that good:

\lstinputlisting[caption=\Optimizing GCC 4.4.5 (IDA),style=customasmMIPS]{patterns/07_jcc/minmax/minmax_MIPS_O3_IDA_EN.lst}

Do not forget about the \IT{branch delay slots}: the first \INS{MOVE} is executed \IT{before} \INS{BEQZ}, 
the second \INS{MOVE} is executed only if the branch hasn't been taken.

}
\RU{\subsection{Поиск минимального и максимального значения}

\subsubsection{32-bit}

\lstinputlisting[style=customc]{patterns/07_jcc/minmax/minmax.c}

\lstinputlisting[caption=\NonOptimizing MSVC 2013,style=customasmx86]{patterns/07_jcc/minmax/minmax_MSVC_2013_RU.asm}

\myindex{x86!\Instructions!Jcc}
Эти две функции отличаются друг от друга только инструкцией условного перехода:
\INS{JGE} (\q{Jump if Greater or Equal}~--- переход если больше или равно) используется в первой
и \INS{JLE} (\q{Jump if Less or Equal}~--- переход если меньше или равно) во второй.

\myindex{\CompilerAnomaly}
\label{MSVC_double_JMP_anomaly}
Здесь есть ненужная инструкция \JMP в каждой функции, которую MSVC, наверное, оставил по ошибке.

\myparagraph{Без переходов}

ARM в режиме Thumb напоминает нам x86-код:

\lstinputlisting[caption=\OptimizingKeilVI (\ThumbMode),style=customasmARM]{patterns/07_jcc/minmax/minmax_Keil_Thumb_O3_RU.s}

\myindex{ARM!\Instructions!Bcc}
Функции отличаются только инструкцией перехода: \INS{BGT} и \INS{BLT}.
А в режиме ARM можно использовать условные суффиксы, так что код более плотный.
\INS{MOVcc} будет исполнена только если условие верно:

\myindex{ARM!\Instructions!MOVcc}

\lstinputlisting[caption=\OptimizingKeilVI (\ARMMode),style=customasmARM]{patterns/07_jcc/minmax/minmax_Keil_ARM_O3_RU.s}

\myindex{x86!\Instructions!CMOVcc}
\Optimizing GCC 4.8.1 и оптимизирующий MSVC 2013 
могут использовать инструкцию \INS{CMOVcc}, которая аналогична \INS{MOVcc} в ARM:

\lstinputlisting[caption=\Optimizing MSVC 2013,style=customasmx86]{patterns/07_jcc/minmax/minmax_GCC481_O3_RU.s}

\subsubsection{64-bit}

\lstinputlisting[style=customc]{patterns/07_jcc/minmax/minmax64.c}

Тут есть ненужные перетасовки значений, но код в целом понятен:

\lstinputlisting[caption=\NonOptimizing GCC 4.9.1 ARM64,style=customasmARM]{patterns/07_jcc/minmax/minmax64_GCC_49_ARM64_O0.s}

\myparagraph{Без переходов}

Нет нужды загружать аргументы функции из стека, они уже в регистрах:

\lstinputlisting[caption=\Optimizing GCC 4.9.1 x64,style=customasmx86]{patterns/07_jcc/minmax/minmax64_GCC_49_x64_O3_RU.s}

MSVC 2013 делает то же самое.

\myindex{ARM!\Instructions!CSEL}
В ARM64 есть инструкция \INS{CSEL}, которая работает точно также, как и \INS{MOVcc} в ARM и \INS{CMOVcc} в x86,
но название другое: \q{Conditional SELect}.

\lstinputlisting[caption=\Optimizing GCC 4.9.1 ARM64,style=customasmARM]{patterns/07_jcc/minmax/minmax64_GCC_49_ARM64_O3_RU.s}

\subsubsection{MIPS}

А GCC 4.4.5 для MIPS не так хорош, к сожалению:

\lstinputlisting[caption=\Optimizing GCC 4.4.5 (IDA),style=customasmMIPS]{patterns/07_jcc/minmax/minmax_MIPS_O3_IDA_RU.lst}

Не забывайте о \IT{branch delay slots}: первая \INS{MOVE} исполняется \IT{перед} \INS{BEQZ},
вторая \INS{MOVE} исполняется только если переход не произошел.

}
\DE{\subsection{Minimale und maximale Werte berechnen}

\subsubsection{32-bit}

\lstinputlisting[style=customc]{patterns/07_jcc/minmax/minmax.c}

\lstinputlisting[caption=\NonOptimizing MSVC 2013,style=customasmx86]{patterns/07_jcc/minmax/minmax_MSVC_2013_DE.asm}

\myindex{x86!\Instructions!Jcc}
Diese beiden Funktionen unterscheiden sich nur hinsichtliche der bedingten Sprungbefehle:
\INS{JGE} (\q{Jump if Greater or Equal}) wird in der ersten verwendet
und \INS{JLE} (\q{Jump if Less or Equal}) in der zweiten.

\myindex{\CompilerAnomaly}
\label{MSVC_double_JMP_anomaly}
Hier gibt es jeweils einen unnötigen \JMP Befehl pro Funtion, den MSVC wahrscheinlich fehlerhafterweise dort belassen
hat.

\myparagraph{Verzweigungslos}

ARM im Thumb mode erinnert uns an den x86 Code:

\lstinputlisting[caption=\OptimizingKeilVI
(\ThumbMode),style=customasmARM]{patterns/07_jcc/minmax/minmax_Keil_Thumb_O3_DE.s}

\myindex{ARM!\Instructions!Bcc}
Die Funktionen unterscheiden sich in den Verzweigebefehlen: \INS{BGT} und \INS{BLT}.
Es ist möglich im ARM mode conditional codes zu verwenden, sodass der Code kürzer ist.

\myindex{ARM!\Instructions!MOVcc}
\INS{MOVcc} wird nur ausgeführt, wenn die Bedingung erfüllt (d.h. wahr) ist:

\lstinputlisting[caption=\OptimizingKeilVI
(\ARMMode),style=customasmARM]{patterns/07_jcc/minmax/minmax_Keil_ARM_O3_DE.s}

\myindex{x86!\Instructions!CMOVcc}
\Optimizing GCC 4.8.1 und der optimierende MSVC 2013 können den \INS{CMOVcc} Befehl verwenden, der analog zu
\INS{MOVcc} in ARM funktioniert:

\lstinputlisting[caption=\Optimizing MSVC 2013,style=customasmx86]{patterns/07_jcc/minmax/minmax_GCC481_O3_DE.s}

\subsubsection{64-bit}

\lstinputlisting[style=customc]{patterns/07_jcc/minmax/minmax64.c}
Hier findet ein unnötiges Verschieben von Variablen statt, aber der Code ist verständlich:

\lstinputlisting[caption=\NonOptimizing GCC 4.9.1 ARM64,style=customasmARM]{patterns/07_jcc/minmax/minmax64_GCC_49_ARM64_O0.s}

\myparagraph{Verzweigungslos}
Die Funtionsargumente müssen nicht vom Stack geladen werden, da sie sich bereits in den Registern befinden:

\lstinputlisting[caption=\Optimizing GCC 4.9.1
x64,style=customasmx86]{patterns/07_jcc/minmax/minmax64_GCC_49_x64_O3_DE.s}

MSVC 2013 tut beinahe das gleiche:

\myindex{ARM!\Instructions!CSEL}

ARM64 verfügt über den \INS{CSEL} Befehl, der genau wie \INS{MOVcc} in ARM oder \INS{CMOVcc} in x86 arbeitet; er hat
lediglich einen anderen Namen:
\q{Conditional SELect}.

\lstinputlisting[caption=\Optimizing GCC 4.9.1
ARM64,style=customasmARM]{patterns/07_jcc/minmax/minmax64_GCC_49_ARM64_O3_DE.s}

\subsubsection{MIPS}

Leider ist GCC 4.4.5 für MIPS nicht so gut:

\lstinputlisting[caption=\Optimizing GCC 4.4.5
(IDA),style=customasmMIPS]{patterns/07_jcc/minmax/minmax_MIPS_O3_IDA_DE.lst} 
Vergessen Sie nicht die \IT{branch delay slots}: der erste \INS{MOVE} wird \IT{vor} \INS{BEQZ} ausgeführt, der zweite
\INS{MOVE} wird nur dann ausgeführt, wenn die Verzweigung nicht genommen wird.


}
\FR{\subsection{Trouver les valeurs minimale et maximale}

\subsubsection{32-bit}

\lstinputlisting[style=customc]{patterns/07_jcc/minmax/minmax.c}

\lstinputlisting[caption=MSVC 2013 \NonOptimizing,style=customasmx86]{patterns/07_jcc/minmax/minmax_MSVC_2013_FR.asm}

\myindex{x86!\Instructions!Jcc}

Ces deux fonctions ne diffèrent que de l'instruction de saut conditionnel:
\INS{JGE} (\q{Jump if Greater or Equal} saut si supérieur ou égal) est utilisée
dans la première et \INS{JLE} (\q{Jump if Less or Equal} saut si inférieur ou égal)
dans la seconde.

\myindex{\CompilerAnomaly}
\label{MSVC_double_JMP_anomaly}

Il y a une instruction \JMP en trop dans chaque fonction, que MSVC a probablement
mise par erreur.

\myparagraph{Sans branchement}

Le mode Thumb d'ARM nous rappelle le code x86:

\lstinputlisting[caption=\OptimizingKeilVI (\ThumbMode),style=customasmARM]{patterns/07_jcc/minmax/minmax_Keil_Thumb_O3_FR.s}

\myindex{ARM!\Instructions!Bcc}

Les fonctions diffèrent au niveau de l'instruction de branchement: \INS{BGT} et \INS{BLT}.
Il est possible d'utiliser le suffixe conditionnel en mode ARM, donc le code est plus
court.

\myindex{ARM!\Instructions!MOVcc}
\INS{MOVcc} n'est exécutée que si la condition est remplie:

\lstinputlisting[caption=\OptimizingKeilVI (\ARMMode),style=customasmARM]{patterns/07_jcc/minmax/minmax_Keil_ARM_O3_FR.s}

\myindex{x86!\Instructions!CMOVcc}
GCC 4.8.1 \Optimizing et MSVC 2013 \Optimizing peuvent utiliser l'instruction \INS{CMOVcc},
qui est analogue à \INS{MOVcc} en ARM:

\lstinputlisting[caption=MSVC 2013 \Optimizing,style=customasmx86]{patterns/07_jcc/minmax/minmax_GCC481_O3_FR.s}

\subsubsection{64-bit}

\lstinputlisting[style=customc]{patterns/07_jcc/minmax/minmax64.c}

Il y a beaucoup de code inutile qui embrouille, mais il est compréhensible:

\lstinputlisting[caption=GCC 4.9.1 ARM64 \NonOptimizing,style=customasmARM]{patterns/07_jcc/minmax/minmax64_GCC_49_ARM64_O0.s}

\myparagraph{Sans branchement}

Il n'y a pas besoin de lire les arguments dans la pile, puisqu'ils sont déjà dans
les registres:

\lstinputlisting[caption=GCC 4.9.1 x64 \Optimizing,style=customasmx86]{patterns/07_jcc/minmax/minmax64_GCC_49_x64_O3_FR.s}

MSVC 2013 fait presque la même chose.

\myindex{ARM!\Instructions!CSEL}

ARM64 possède l'instruction \INS{CSEL}, qui fonctionne comme \INS{MOVcc} en ARM ou
\INS{CMOVcc} en x86, seul le nom diffère:
\q{Conditional SELect}.

\lstinputlisting[caption=GCC 4.9.1 ARM64 \Optimizing,style=customasmARM]{patterns/07_jcc/minmax/minmax64_GCC_49_ARM64_O3_FR.s}

\subsubsection{MIPS}

Malheureusement, GCC 4.4.5 pour MIPS n'est pas si performant:

\lstinputlisting[caption=GCC 4.4.5 \Optimizing (IDA),style=customasmMIPS]{patterns/07_jcc/minmax/minmax_MIPS_O3_IDA_FR.lst}

N'oubliez pas le slot de délai de branchement (\IT{branch delay slots}): le premier
\INS{MOVE} est exécuté \IT{avant} \INS{BEQZ}, le second \INS{MOVE} n'est exécuté
que si la branche n'a pas été prise.

}


\subsection{\Conclusion{}}

\subsubsection{x86}

Voici le squelette générique d'un saut conditionnel:

\begin{lstlisting}[caption=x86,style=customasmx86]
CMP registre, registre/valeur
Jcc true ; cc=condition code, code de condition
false:
... §le code qui sera exécuté si le résultat de la comparaison est faux (false)§ ...
JMP exit 
true:
... §le code qui sera exécuté si le résultat de la comparaison est vrai (true)§ ...
exit:
\end{lstlisting}

\subsubsection{ARM}

\begin{lstlisting}[caption=ARM,style=customasmARM]
CMP registre, registre/valeur
Bcc true ; cc=condition code
false:
... §le code qui sera exécuté si le résultat de la comparaison est faux (false)§ ...
JMP exit 
true:
... §le code qui sera exécuté si le résultat de la comparaison est vrai (true)§ ...
exit:
\end{lstlisting}

\subsubsection{MIPS}

\begin{lstlisting}[caption=Check si zéro (Branch if EQual Zero),style=customasmMIPS]
BEQZ REG, label
...
\end{lstlisting}

\begin{lstlisting}[caption=Check si plus petit que zéro (Branch if Less Than Zero) en utilisant une pseudo instruction,style=customasmMIPS]
BLTZ REG, label
...
\end{lstlisting}

\begin{lstlisting}[caption=Check si les valeurs sont égales (Branch if EQual),style=customasmMIPS]
BEQ REG1, REG2, label
...
\end{lstlisting}

\begin{lstlisting}[caption=Check si les valeurs ne sont pas égales (Branch if Not Equal),style=customasmMIPS]
BNE REG1, REG2, label
...
\end{lstlisting}

\begin{lstlisting}[caption=Check REG2 plus petit que REG3 (signé),style=customasmMIPS]
SLT REG1, REG2, REG3
BEQ REG1, label
...
\end{lstlisting}

\begin{lstlisting}[caption=Check REG2 plus petit que REG3 (non signé),style=customasmMIPS]
SLTU REG1, REG2, REG3
BEQ REG1, label
...
\end{lstlisting}

\subsubsection{Sans branchement}

\myindex{ARM!\Instructions!MOVcc}
\myindex{x86!\Instructions!CMOVcc}
\myindex{ARM!\Instructions!CSEL}
Si le corps d'instruction conditionnelle est très petit, l'instruction de déplacement
conditionnel peut être utilisée:
\INS{MOVcc} en ARM (en mode ARM), \INS{CSEL} en ARM64, \INS{CMOVcc} en x86.

\myparagraph{ARM}

Il est possible d'utiliser les suffixes conditionnels pour certaines instructions
ARM:

\begin{lstlisting}[caption=ARM (\ARMMode),style=customasmARM]
CMP registre, registre/valeur
instr1_cc ; §cette instruction sera exécutée si le code conditionnel est vrai§ (true)
instr2_cc ; §cette autre instruction sera exécutée si cet autre code conditionnel est vrai§ (true)
... etc.
\end{lstlisting}

Bien sûr, il n'y a pas de limite au nombre d'instructions avec un suffixe de code
conditionnel, tant que les flags du CPU ne sont pas modifiés par l'une d'entre elles.
% FIXME: list of such instructions or \myref{} to it

\myindex{ARM!\Instructions!IT}

Le mode Thumb possède l'instruction \INS{IT}, permettant d'ajouter le suffixe conditionnel
pour les quatres instructions suivantes.
Lire à ce propos: \myref{ARM_Thumb_IT}.

\begin{lstlisting}[caption=ARM (\ThumbMode),style=customasmARM]
CMP registre, registre/valeur
ITEEE EQ ; met ces suffixes: if-then-else-else-else
instr1   ; §instruction exécutée si la condition est vraie§
instr2   ; §instruction exécutée si la condition est fausse§
instr3   ; §instruction exécutée si la condition est fausse§
instr4   ; §instruction exécutée si la condition est fausse§
\end{lstlisting}

\subsection{\Exercise}

(ARM64) Essayez de récrire le code pour \lstref{cond_ARM64} en supprimant toutes
les instructions de saut conditionnel et en utilisant l'instruction \TT{CSEL}.

