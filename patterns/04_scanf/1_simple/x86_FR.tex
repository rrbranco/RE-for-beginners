\subsubsection{x86}

\myparagraph{MSVC}

Voici ce que l'on obtient après avoir compilé avec MSVC 2010:

\lstinputlisting[style=customasmx86]{patterns/04_scanf/1_simple/ex1_MSVC_FR.asm}

\TT{x} est une variable locale.

D'après le standard \CCpp elle ne doit être visible que dans cette fonction et dans
aucune autre portée.
Traditionnellement, les variables locales sont stockées dans la pile.
Il y a probablement d'autres moyens de les allouer, mais en x86, c'est la façon de faire.

\myindex{x86!\Instructions!PUSH}
Le but de l'instruction suivant le prologue de la fonction, \TT{PUSH ECX}, n'est
pas de sauver l'état de \ECX (noter l'absence d'un \TT{POP ECX} à la fin de la
fonction).

En fait, cela alloue 4 octets sur la pile pour stocker la variable \TT{x}.

\label{stack_frame}
\myindex{\Stack!Stack frame}
\myindex{x86!\Registers!EBP}
\TT{x} est accèdée à l'aide de la macro \TT{\_x\$} (qui vaut -4) et du registre \EBP
qui pointe sur la structure de pile courante.

Pendant la durée de l'exécution de la fonction, \EBP pointe sur la \glslink{stack frame}{structure locale de pile}
courante, rendant possible l'accès aux variables locales et aux arguments de la
fonction via \TT{EBP+offset}.

\myindex{x86!\Registers!ESP}
Il est aussi possible d'utiliser \ESP dans le même but, bien que ça ne soit pas
très commode, car il change fréquemment.
La valeur de \EBP peut être perçue comme un \IT{état figé} de la valeur de \ESP
au début de l'exécution de la fonction.

% FIXME1 это уже было в 02_stack?
Voici une \glslink{stack frame}{strucutre de pile} typique dans un environnement 32-bit:

\begin{center}
\begin{tabular}{ | l | l | }
\hline
\dots & \dots \\
\hline
EBP-8 & variable locale \#2, \MarkedInIDAAs{} \TT{var\_8} \\
\hline
EBP-4 & variable locale \#1, \MarkedInIDAAs{} \TT{var\_4} \\
\hline
EBP & valeur sauvée de \EBP \\
\hline
EBP+4 & adresse de retour \\
\hline
EBP+8 & \argument \#1, \MarkedInIDAAs{} \TT{arg\_0} \\
\hline
EBP+0xC & \argument \#2, \MarkedInIDAAs{} \TT{arg\_4} \\
\hline
EBP+0x10 & \argument \#3, \MarkedInIDAAs{} \TT{arg\_8} \\
\hline
\dots & \dots \\
\hline
\end{tabular}
\end{center}

La fonction \scanf de notre exemple a deux arguments.

Le premier est un pointeur sur la chaîne contenant \TT{\%d} et le second est l'adresse
de la variable \TT{x}.

\myindex{x86!\Instructions!LEA}
Tout d'abord, l'adresse de la variable \TT{x} est chargée dans le registre \EAX
par l'instruction \\ \TT{lea eax, DWORD PTR \_x\$[ebp]}.

\LEA signifie \IT{load effective address} (charger l'adresse effective) et est souvent
utilisée pour composer une adresse ~(\myref{sec:LEA}).

Nous pouvons dire que dans ce cas, \LEA stocke simplement la somme de la valeur du
registre \EBP et de la macro \TT{\_x\$} dans le registre \EAX.

C'est la même chose que \INS{lea eax, [ebp-4]}.

Donc, 4 est soustrait de la valeur du registre \EBP et le résultat est chargé dans
le registre \EAX.
Ensuite, la valeur du registre \EAX est poussée sur la pile et \scanf est appelée.

\printf est appelée ensuite avec son premier argument --- un pointeur sur la chaîne:
\TT{You entered \%d...\textbackslash{}n}.

Le second argument est préparé avec: \TT{mov ecx, [ebp-4]}.
L'instruction stocke la valeur de la variable \TT{x} et non son adresse, dans le
registre \ECX.

Puis, la valeur de \ECX est stockée sur la pile et le dernier appel à \printf
est effectué.

\EN{\clearpage
\subsubsection{MSVC + \olly}
\myindex{\olly}

Let's try this example in \olly.
Let's load it and keep pressing F8 (\stepover) until we reach our executable file instead of \TT{ntdll.dll}.
Scroll up until \main appears.

Click on the first instruction (\TT{PUSH EBP}), press F2 (\IT{set a breakpoint}), then F9 (\IT{Run}).
The breakpoint will be triggered when \main begins.

Let's trace to the point where the address of the variable $x$ is calculated:

\begin{figure}[H]
\centering
\myincludegraphics{patterns/04_scanf/1_simple/ex1_olly_1.png}
\caption{\olly: The address of the local variable is calculated}
\label{fig:scanf_ex1_olly_1}
\end{figure}

Right-click the \EAX in the registers window and then select \q{Follow in stack}.

This address will appear in the stack window.
The red arrow has been added, pointing to the variable in the local stack.
At that moment this location contains some garbage (\TT{0x6E494714}).
Now with the help of \PUSH instruction the address of this stack element is going to be stored to the same stack on the next position.
Let's trace with F8 until the \scanf execution completes.
During the \scanf execution, we input, for example, 123, in the console window:

\lstinputlisting{patterns/04_scanf/1_simple/console.txt}

\clearpage
\scanf completed its execution already:

\begin{figure}[H]
\centering
\myincludegraphics{patterns/04_scanf/1_simple/ex1_olly_3.png}
\caption{\olly: \scanf executed}
\label{fig:scanf_ex1_olly_3}
\end{figure}

\scanf returns 1 in \EAX, which implies that it has read successfully one value.
If we look again at the stack element corresponding to the local variable it now contains \TT{0x7B} (123).

\clearpage

Later this value is copied from the stack to the \ECX register and passed to \printf:

\begin{figure}[H]
\centering
\myincludegraphics{patterns/04_scanf/1_simple/ex1_olly_4.png}
\caption{\olly: preparing the value for passing to \printf}
\label{fig:scanf_ex1_olly_4}
\end{figure}
}
\RU{\clearpage
\subsubsection{MSVC + \olly}
\myindex{\olly}

Попробуем этот же пример в \olly.
Загружаем, нажимаем F8 (\stepover) до тех пор, пока не окажемся в своем исполняемом файле,
а не в \TT{ntdll.dll}.
Прокручиваем вверх до тех пор, пока не найдем \main.
Щелкаем на первой инструкции (\TT{PUSH EBP}), нажимаем F2 (\IT{set a breakpoint}), 
затем F9 (\IT{Run}) и точка останова срабатывает на начале \main.

Трассируем до того места, где готовится адрес переменной $x$:

\begin{figure}[H]
\centering
\myincludegraphics{patterns/04_scanf/1_simple/ex1_olly_1.png}
\caption{\olly: вычисляется адрес локальной переменной}
\label{fig:scanf_ex1_olly_1}
\end{figure}

На \EAX в окне регистров можно нажать правой кнопкой и далее выбрать \q{Follow in stack}.
Этот адрес покажется в окне стека.

Смотрите, это переменная в локальном стеке. Там дорисована красная стрелка.
И там сейчас какой-то мусор (\TT{0x6E494714}).
Адрес этого элемента стека сейчас, при помощи \PUSH запишется в этот же стек рядом.
Трассируем при помощи F8 вплоть до конца исполнения \scanf.
А пока \scanf исполняется, в консольном окне, вводим, например, 123:

\lstinputlisting{patterns/04_scanf/1_simple/console.txt}

\clearpage
Вот тут \scanf отработал:

\begin{figure}[H]
\centering
\myincludegraphics{patterns/04_scanf/1_simple/ex1_olly_3.png}
\caption{\olly: \scanf исполнилась}
\label{fig:scanf_ex1_olly_3}
\end{figure}

\scanf вернул 1 в \EAX, что означает, что он успешно прочитал одно значение.
В наблюдаемом нами элементе стека теперь \TT{0x7B} (123).

\clearpage
Чуть позже это значение копируется из стека в регистр \ECX и передается в \printf{}:

\begin{figure}[H]
\centering
\myincludegraphics{patterns/04_scanf/1_simple/ex1_olly_4.png}
\caption{\olly: готовим значение для передачи в \printf}
\label{fig:scanf_ex1_olly_4}
\end{figure}
}
\ITA{\clearpage
\subsubsection{MSVC + \olly}
\myindex{\olly}

Proviamo ad analizzare l'esempio con \olly.
Carichiamo l'eseguibile e premiamo F8 (\stepover) fino a raggiungere il nostro eseguibile invece che \TT{ntdll.dll}.
Scorriamo verso l'alto finche' appare \main .

Clicchiamo sulla prima istruzione (\TT{PUSH EBP}), premiamo F2 (\IT{set a breakpoint}), e quindi F9 (\IT{Run}).
Il breakpoint sara' scatenato all'inizio della funzione \main .

Tracciamo adesso fino al punto in cui viene calcolato l'indirizzo della variabile $x$:

\begin{figure}[H]
\centering
\myincludegraphics{patterns/04_scanf/1_simple/ex1_olly_1.png}
\caption{\olly: The address of the local variable is calculated}
\label{fig:scanf_ex1_olly_1}
\end{figure}


Click di destra su \EAX nella finestra dei registri e selezioniamo \q{Follow in stack}.

Questo indirizzo apparira' nella finestra dello stack.
La freccia rossa aggiunta punta alla variabile nello stack locale.
Al momento questa locazione contiene un po' di immondizia (garbage) (\TT{0x6E494714}).
Con l'aiuto dell'istruzione \PUSH l'indirizzo di questo elemento dello stack sara' memorizzato nello stesso stack alla posizione successiva.
Tracciamo con F8 finche' non viene completata l'esecuzione della funzione \scanf.
Durante l'esecuzione di \scanf, diamo in input un valore nella console. Ad esempio 123:

\lstinputlisting{patterns/04_scanf/1_simple/console.txt}

\clearpage
\scanf ha gia' completato la sua esecuzione:

\begin{figure}[H]
\centering
\myincludegraphics{patterns/04_scanf/1_simple/ex1_olly_3.png}
\caption{\olly: \scanf executed}
\label{fig:scanf_ex1_olly_3}
\end{figure}

\scanf restituisce 1 in \EAX, e cio' implica che ha letto con successo un valore.
Se guardiamo nuovamente l'elemento nello stack corrispondente alla variabile locale, adesso contiente \TT{0x7B} (123).

\clearpage

Successivamente questo valore viene copiato dallo stack al registro \ECX e passato a \printf:

\begin{figure}[H]
\centering
\myincludegraphics{patterns/04_scanf/1_simple/ex1_olly_4.png}
\caption{\olly: preparing the value for passing to \printf}
\label{fig:scanf_ex1_olly_4}
\end{figure}
}
\DE{\clearpage
\subsubsection{MSVC + \olly}
\myindex{\olly}
Schauen wir uns diese Beispiel in \olly an.
Wir laden es und drücken F8 (\stepover) bis wir unsere ausführbare Datei anstelle von \TT{ntdll.dll} erreicht haben. Wir scrollen nach oben bis \main erscheint.

Wir klicken auf den ersten Befehl (\TT{PUSH EBO}), drücken F2 (\IT{set a breakpoint}), dann F9 (\IT{Run}). Der Breakpoint wird ausgelöst, wenn die Funktion \main beginnt.

Verfolgen wir den Ablauf bis zu der Stelle, an der die Adresse der Variablen $x$ berechnet wird:

\begin{figure}[H]
\centering
\myincludegraphics{patterns/04_scanf/1_simple/ex1_olly_1.png}
\caption{\olly: Die Adresse der lokalen Variable wird berechnet.}
\label{fig:scanf_ex1_olly_1}
\end{figure}

Wir machen einen Rechtsklick auf \EAX in Registerfenster und wählen \q{Follow in stack}. 

Diese Adresse wird im Stackfenster erscheinen. Der rote Pfeil wurde nachträglich hinzugefügt; er zeigt auf die Variable im lokalen Stack. Im Moment enthält diese Speicherstelle Zufallswerte (\TT{0x6E494714}). Jetzt wird mithilfe des \PUSH Befehls die Adresse dieses Stackelements auf demselben Stack an der folgenden Position gespeichert. 
Verfolgen wir den Ablauf mit F8 bis die Ausführung von \scanf abgeschlossen ist. Während der Ausführung von \scanf geben wir beispielsweise 123 in der Konsole ein:

\lstinputlisting{patterns/04_scanf/1_simple/console.txt}

\clearpage
\scanf ist bereits beendet:

\begin{figure}[H]
\centering
\myincludegraphics{patterns/04_scanf/1_simple/ex1_olly_3.png}
\caption{\olly: \scanf wurde ausgeführt}
\label{fig:scanf_ex1_olly_3}
\end{figure}

\scanf liefert 1 im \EAX Register zurück, was aussagt, dass die Funktion einen Wert erfolgreich eingelesen hat. Wenn wir wiederum auf das zugehörige Stackelement für die lokale Variable schauen, enthält diese nun den Wert \TT{0x7B} (dez. 123).

\clearpage
Im weiteren Verlauf wird dieser Wert vom Stack in das \ECX Register kopiert und an \printf übergeben:

\begin{figure}[H]
\centering
\myincludegraphics{patterns/04_scanf/1_simple/ex1_olly_4.png}
\caption{\olly: Wert für Übergabe an \printf vorbereiten.}
\label{fig:scanf_ex1_olly_4}
\end{figure}

}
\FR{\clearpage
\subsubsection{MSVC + \olly}
\myindex{\olly}

% TODO look in French olly for text translation, if exists?
Essayons cet exemple dans \olly.
Chargeons-le et appuyons sur F8 (\stepover) jusqu'à ce que nous atteignons notre
exécutable au lieu de \TT{ntdll.dll}.
Défiler vers le haut jusqu'à ce que \main apparaisse.

Cliquer sur la première instruction  (\TT{PUSH EBP}), appuyer sur F2 (\IT{set a
breakpoint}), puis F9 (\IT{Run}).
Le point d'arrêt sera déclenché lorsque \main commencera.

Continuons jusqu'au point où la variable $x$ est calculée:

\begin{figure}[H]
\centering
\myincludegraphics{patterns/04_scanf/1_simple/ex1_olly_1.png}
\caption{\olly: L'adresse de la variable locale est calculée}
\label{fig:scanf_ex1_olly_1}
\end{figure}

Cliquer droit sur \EAX dans la fenêtre des registres et choisir \q{Follow in stack}.

Cette adresse va apparaître dans la fenêtre de la pile.
La flèche rouge a été ajoutée, pointant la variable dans la pile locale.
A ce point, cet espace contient des restes de données (\TT{0x6E494714}).
Maintenant. avec l'aide de l'instruction \PUSH, l'adresse de cet élément de pile
va être stockée sur la même pile à la position suivante.
Appuyons sur F8 jusqu'à la fin de l'exécution de \scanf.
Pendant l'exécution de \scanf, entrons, par exemple, 123, dans la fenêtre de la console:

\lstinputlisting{patterns/04_scanf/1_simple/console.txt}

\clearpage
\scanf a déjà fini de s'exécuter:

\begin{figure}[H]
\centering
\myincludegraphics{patterns/04_scanf/1_simple/ex1_olly_3.png}
\caption{\olly: \scanf s'est exécutée}
\label{fig:scanf_ex1_olly_3}
\end{figure}

\scanf renvoie 1 dans \EAX, ce qui indique qu'elle a lu avec succès une valeur.
Si nous regardons de nouveau l'élément de la pile correspondant à la variable
locale, il contient maintenant \TT{0x7B} (123).

\clearpage

Plus tard, cette valeur est copiée de la pile vers le registre \ECX et passée à \printf:

\begin{figure}[H]
\centering
\myincludegraphics{patterns/04_scanf/1_simple/ex1_olly_4.png}
\caption{\olly: préparation de la valeur pour la passer à \printf}
\label{fig:scanf_ex1_olly_4}
\end{figure}
}
\PTBR{\clearpage
\subsubsection{MSVC + \olly}
\myindex{\olly}

Vamos fazer este exemplo no \olly.
Carregue o programa e pressione F8 (\stepover) até que cheguemos no nosso executável ao invés do \TT{ntdll.dll}.
Role o scroll para cima até chegar no \main.

Click na primeira instrução (\TT{PUSH EBP}), pressione F2 (\IT{para setar um breakpoint}), e pressione F9 (\IT{Executar}).
A execução irá parar quando atingir o breakpoint no começo da \main.

Vamos até o trecho onde o endereço da variável $x$ é calculado:


\begin{figure}[H]
\centering
\myincludegraphics{patterns/04_scanf/1_simple/ex1_olly_1.png}
\caption{\olly: O endereço da variável local é calculado}
\label{fig:scanf_ex1_olly_1}
\end{figure}

Na janela de registradores, clique com o botão direito do mouse em \EAX e selecione \q{Seguir na pilha}.

O endereço irá aparecer na janela de pilha.
A seta vermelha está apontando para a variável na pilha local.

Neste momento, esse local contém apenas lixo(\TT{0x6E494714}).Agora,  a instrução \PUSH irá armazenar na pilha o endereço deste elemento na pilha será armazendo na próxima posição.

Vamos rastrear com F8 até terminar a execução do \scanf.

Durante a execução do \scanf, digite, por exemplo, 123, na janela do console:

\lstinputlisting{patterns/04_scanf/1_simple/console.txt}

\clearpage
\scanf terminou sua execução:

\begin{figure}[H]
\centering
\myincludegraphics{patterns/04_scanf/1_simple/ex1_olly_3.png}
\caption{\olly: \scanf executado}
\label{fig:scanf_ex1_olly_3}
\end{figure}

\scanf retorna 1 no registrador \EAX, o que significa que leu com sucesso o valor.
Se olharmos novamente no endereço da pilha correspondente a variável local, ela agora contém \TT{0x7B} (123).

\clearpage

Depois este valor é copiado da pilha para o registrador \ECX e então passado como argumento para \printf:


\begin{figure}[H]
\centering
\myincludegraphics{patterns/04_scanf/1_simple/ex1_olly_4.png}
\caption{\olly: Preparando o valor para o \printf}
\label{fig:scanf_ex1_olly_4}
\end{figure}
}


\myparagraph{GCC}

Compilons ce code avec GCC 4.4.1 sous Linux:

\lstinputlisting[style=customasmx86]{patterns/04_scanf/1_simple/ex1_GCC.asm}

\myindex{puts() instead of printf()}
GCC a remplacé l'appel à \printf avec un appel à \puts. La raison de cela a été
expliquée dans ~(\myref{puts}).

% TODO: rewrite
%\RU{Почему \scanf переименовали в \TT{\_\_\_isoc99\_scanf}, я честно говоря, пока не знаю.}
%\EN{Why \scanf is renamed to \TT{\_\_\_isoc99\_scanf}, I do not know yet.}
% 
% Apparently it has to do with the ISO c99 standard compliance. By default GCC allows specifying a standard to adhere to.
% For example if you compile with -std=c89 the outputted assmebly file will contain scanf and not __isoc99__scanf. I guess current GCC version adhares to c99 by default.
% According to my understanding the two implementations differ in the set of suported modifyers (See printf man page)

Comme dans l'exemple avec MSVC---les arguments sont placés dans la pile avec l'instruction
\MOV.

\myparagraph{By the way}

Á propos, ce simple exemple est la démonstration du fait que le compilateur traduit
une liste d'expression en bloc-\CCpp en une liste séquentielle d'instructions.
% TODO FIXME: better translation / clarify ?
Il n'y a rien entre les expressions en \CCpp, et le résultat en code machine,
il n'y a rien entre le déroulement du flux de contrôle d'une expression à la suivante.
