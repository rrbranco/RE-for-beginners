\clearpage
\subsubsection{MSVC + \olly}
\myindex{\olly}

Vamos fazer este exemplo no \olly.
Carregue o programa e pressione F8 (\stepover) até que cheguemos no nosso executável ao invés do \TT{ntdll.dll}.
Role o scroll para cima até chegar no \main.

Click na primeira instrução (\TT{PUSH EBP}), pressione F2 (\IT{para setar um breakpoint}), e pressione F9 (\IT{Executar}).
A execução irá parar quando atingir o breakpoint no começo da \main.

Vamos até o trecho onde o endereço da variável $x$ é calculado:


\begin{figure}[H]
\centering
\myincludegraphics{patterns/04_scanf/1_simple/ex1_olly_1.png}
\caption{\olly: O endereço da variável local é calculado}
\label{fig:scanf_ex1_olly_1}
\end{figure}

Na janela de registradores, clique com o botão direito do mouse em \EAX e selecione \q{Seguir na pilha}.

O endereço irá aparecer na janela de pilha.
A seta vermelha está apontando para a variável na pilha local.

Neste momento, esse local contém apenas lixo(\TT{0x6E494714}).Agora,  a instrução \PUSH irá armazenar na pilha o endereço deste elemento na pilha será armazendo na próxima posição.

Vamos rastrear com F8 até terminar a execução do \scanf.

Durante a execução do \scanf, digite, por exemplo, 123, na janela do console:

\lstinputlisting{patterns/04_scanf/1_simple/console.txt}

\clearpage
\scanf terminou sua execução:

\begin{figure}[H]
\centering
\myincludegraphics{patterns/04_scanf/1_simple/ex1_olly_3.png}
\caption{\olly: \scanf executado}
\label{fig:scanf_ex1_olly_3}
\end{figure}

\scanf retorna 1 no registrador \EAX, o que significa que leu com sucesso o valor.
Se olharmos novamente no endereço da pilha correspondente a variável local, ela agora contém \TT{0x7B} (123).

\clearpage

Depois este valor é copiado da pilha para o registrador \ECX e então passado como argumento para \printf:


\begin{figure}[H]
\centering
\myincludegraphics{patterns/04_scanf/1_simple/ex1_olly_4.png}
\caption{\olly: Preparando o valor para o \printf}
\label{fig:scanf_ex1_olly_4}
\end{figure}
