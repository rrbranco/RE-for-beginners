\subsection{Ein einfaches Beispiel}

\lstinputlisting[style=customc]{patterns/04_scanf/1_simple/ex1.c}

Es ist nicht ratsam \scanf heutzutage noch für User Interaktionen zu verwenden. Aber dennoch können wir hier die Übergabe eines Pointers an eine Variable vom Typ \Tint betrachten.

\subsubsection{Pointer}
\myindex{\CLanguageElements!\Pointers}
Pointer sind eines der fundamentalen Konzepte in der Informatik. Oft ist das Übergeben eines großen Array, eines Structs oder Objekts als Funktionsargument zu teuer, während die Übergabe der Adresse wesentlich billiger ist. 
Wenn man zum Beispiel einen Textstring auf der Konsole ausgeben möchte, ist es deutlich einfacher, nur dessen Adresse in den Kernel des \ac{OS} zu übergeben.

Wenn die aufgerufene Funktion außerdem das große Array oder Struct verändern muss und das gesamte Object zurückgeben muss, ist die Situation beinahe absurd. 
Das einfachste ist also die Adresse eines Arrays oder Structs an die aufgerufene Funktion zu übergeben und sie dann die notwendigen Veränderungen durchführen zu lassen.

Ein Pointer ist in \CCpp nichts anderes als die Adresse einer Speicherstelle.


\myindex{x86-64}
In x86 wird die Adresse als 32-Bit-Zahl dargestellt, d.h. sie benötigt 4 Byte, während in x86-64 eine Darstellung durch 64 Bit (d.h. 8 Byte) erfolgt.
Dies ist übrigens der Grund dafür, dass einige Leute den Wechsel zu x86-64 ablehnen--alle Pointer in der x64-Architektur erfordern doppelt soviel Speicherplatz, inklusive Speicher in Cache, der ein sehr teurer Speicher ist.

\myindex{\CStandardLibrary!memcpy()}
Es ist möglich lediglich mit untypisierten Pointern zu arbeiten, wenn man ein wenig zusätzlichen Aufwand betreibt; z.B. in der Standard-C-Funktion \TT{memcpy()}, die einen Datenblock von einer Speicherstelle zu einer anderen kopiert, werden zwei Pointer vom Typ \TT{void*} als Argumente verwendet, da es nicht vorhersagbar ist, welchen Datentyp die Funktion kopieren soll. Datentypen sind hier nicht wichtig, entscheidend ist hier nur die Größe des Speicherblocks.

Pointer werden außerdem häufig verwendet, wenn eine Funktion mehr als einen Wert zurückgeben muss. (Darauf kommen wir später in ~(\myref{label_pointers}) zurück.)

Die Funktion \IT{scanf()} ist solch ein Fall: Neben der Tatsache, dass die Funktion angeben muss wie viele Werte erfolgreich gelesen wurden, muss sie auch alle diese Werte zurückliefern.

In \CCpp wird der Pointertyp nur für Typüberprüfungen zur Compilezeit benötigt.

Intern steckt im kompilierten Code keinerlei Information über die Typen der enthaltenen Pointer.

\EN{\subsubsection{x86}

\myparagraph{MSVC}

Here is what we get after compiling with MSVC 2010:

\lstinputlisting[style=customasmx86]{patterns/04_scanf/1_simple/ex1_MSVC_EN.asm}

\TT{x} is a local variable.

According to the \CCpp standard it must be visible only in this function and not from any other external scope. 
Traditionally, local variables are stored on the stack. 
There are probably other ways to allocate them, but in x86 that is the way it is.

\myindex{x86!\Instructions!PUSH}
The goal of the instruction following the function prologue, \TT{PUSH ECX}, is not to save the \ECX state 
(notice the absence of corresponding \TT{POP ECX} at the function's end).

In fact it allocates 4 bytes on the stack for storing the \TT{x} variable.

\label{stack_frame}
\myindex{\Stack!Stack frame}
\myindex{x86!\Registers!EBP}
\TT{x} is to be accessed with the assistance of the \TT{\_x\$} macro (it equals to -4) and the \EBP register pointing to the current frame.

Over the span of the function's execution, \EBP is pointing to the current \gls{stack frame}
making it possible to access local variables and function arguments via \TT{EBP+offset}.

\myindex{x86!\Registers!ESP}
It is also possible to use \ESP for the same purpose, although that is not very convenient since it changes frequently.
The value of the \EBP could be perceived as a \IT{frozen state} of the value in \ESP at the start of the function's execution.

% FIXME1 это уже было в 02_stack?
Here is a typical \gls{stack frame} layout in 32-bit environment:

\begin{center}
\begin{tabular}{ | l | l | }
\hline
\dots & \dots \\
\hline
EBP-8 & local variable \#2, \MarkedInIDAAs{} \TT{var\_8} \\
\hline
EBP-4 & local variable \#1, \MarkedInIDAAs{} \TT{var\_4} \\
\hline
EBP & saved value of \EBP \\
\hline
EBP+4 & return address \\
\hline
EBP+8 & \argument \#1, \MarkedInIDAAs{} \TT{arg\_0} \\
\hline
EBP+0xC & \argument \#2, \MarkedInIDAAs{} \TT{arg\_4} \\
\hline
EBP+0x10 & \argument \#3, \MarkedInIDAAs{} \TT{arg\_8} \\
\hline
\dots & \dots \\
\hline
\end{tabular}
\end{center}

The \scanf function in our example has two arguments.

The first one is a pointer to the string containing \TT{\%d} and the second is the address of the \TT{x} variable.

\myindex{x86!\Instructions!LEA}
First, the \TT{x} variable's address is loaded into the \EAX register by the \\
\TT{lea eax, DWORD PTR \_x\$[ebp]} instruction.

\LEA stands for \IT{load effective address}, and is often used for forming an address ~(\myref{sec:LEA}).

We could say that in this case \LEA simply stores the sum of the \EBP register value and the \TT{\_x\$} macro in the \EAX register.

This is the same as \INS{lea eax, [ebp-4]}.

So, 4 is being subtracted from the \EBP register value and the result is loaded in the \EAX register.
Next the \EAX register value is pushed into the stack and \scanf is being called.

\printf is being called after that with its first argument --- a pointer to the string:
\TT{You entered \%d...\textbackslash{}n}.

The second argument is prepared with: \TT{mov ecx, [ebp-4]}.
The instruction stores the \TT{x} variable value and not its address, in the \ECX register.

Next the value in the \ECX is stored on the stack and the last \printf is being called.

\EN{\input{patterns/04_scanf/1_simple/olly_EN}}
\RU{\input{patterns/04_scanf/1_simple/olly_RU}}
\ITA{\input{patterns/04_scanf/1_simple/olly_ITA}}
\DE{\input{patterns/04_scanf/1_simple/olly_DE}}
\FR{\input{patterns/04_scanf/1_simple/olly_FR}}
\PTBR{\input{patterns/04_scanf/1_simple/olly_PTBR}}


\myparagraph{GCC}

Let's try to compile this code in GCC 4.4.1 under Linux:

\lstinputlisting[style=customasmx86]{patterns/04_scanf/1_simple/ex1_GCC.asm}

\myindex{puts() instead of printf()}
GCC replaced the \printf call with call to \puts. The reason for this was explained in ~(\myref{puts}).

% TODO: rewrite
%\RU{Почему \scanf переименовали в \TT{\_\_\_isoc99\_scanf}, я честно говоря, пока не знаю.}
%\EN{Why \scanf is renamed to \TT{\_\_\_isoc99\_scanf}, I do not know yet.}
% 
% Apparently it has to do with the ISO c99 standard compliance. By default GCC allows specifying a standard to adhere to.
% For example if you compile with -std=c89 the outputted assmebly file will contain scanf and not __isoc99__scanf. I guess current GCC version adhares to c99 by default.
% According to my understanding the two implementations differ in the set of suported modifyers (See printf man page)

As in the MSVC example---the arguments are placed on the stack using the \MOV instruction.

\myparagraph{By the way}

By the way, this simple example is a demonstration of the fact that compiler translates
list of expressions in \CCpp-block into sequential list of instructions.
There are nothing between expressions in \CCpp, and so in resulting machine code, 
there are nothing between, control flow slips from one expression to the next one.

}
\RU{\subsubsection{x86}

\myparagraph{MSVC}

Что получаем на ассемблере, компилируя в MSVC 2010:

\lstinputlisting[style=customasmx86]{patterns/04_scanf/1_simple/ex1_MSVC_RU.asm}

Переменная \TT{x} является локальной.

По стандарту \CCpp она доступна только из этой же функции и нигде более. 
Так получилось, что локальные переменные располагаются в стеке. 
Может быть, можно было бы использовать и другие варианты, но в x86 это традиционно так.

\myindex{x86!\Instructions!PUSH}
Следующая после пролога инструкция \TT{PUSH ECX} не ставит своей целью сохранить 
значение регистра \ECX. 
(Заметьте отсутствие соответствующей инструкции \TT{POP ECX} в конце функции).

Она на самом деле выделяет в стеке 4 байта для хранения \TT{x} в будущем.

\label{stack_frame}
\myindex{\Stack!Стековый фрейм}
\myindex{x86!\Registers!EBP}
Доступ к \TT{x} будет осуществляться при помощи объявленного макроса \TT{\_x\$} (он равен -4) и регистра \EBP указывающего на текущий фрейм.

Во всё время исполнения функции \EBP указывает на текущий \glslink{stack frame}{фрейм} и через \TT{EBP+смещение}
можно получить доступ как к локальным переменным функции, так и аргументам функции.

\myindex{x86!\Registers!ESP}
Можно было бы использовать \ESP, но он во время исполнения функции часто меняется, а это не удобно. 
Так что можно сказать, что \EBP это \IT{замороженное состояние} \ESP на момент начала исполнения функции.

% FIXME1 это уже было в 02_stack?
Разметка типичного стекового \glslink{stack frame}{фрейма} в 32-битной среде:

\begin{center}
\begin{tabular}{ | l | l | }
\hline
\dots & \dots \\
\hline
EBP-8 & локальная переменная \#2, \MarkedInIDAAs{} \TT{var\_8} \\
\hline
EBP-4 & локальная переменная \#1, \MarkedInIDAAs{} \TT{var\_4} \\
\hline
EBP & сохраненное значение \EBP \\
\hline
EBP+4 & адрес возврата \\
\hline
EBP+8 & \argument \#1, \MarkedInIDAAs{} \TT{arg\_0} \\
\hline
EBP+0xC & \argument \#2, \MarkedInIDAAs{} \TT{arg\_4} \\
\hline
EBP+0x10 & \argument \#3, \MarkedInIDAAs{} \TT{arg\_8} \\
\hline
\dots & \dots \\
\hline
\end{tabular}
\end{center}

У функции \scanf в нашем примере два аргумента.

Первый~--- указатель на строку, содержащую \TT{\%d} и второй~--- адрес переменной \TT{x}.

\myindex{x86!\Instructions!LEA}
Вначале адрес \TT{x} помещается в регистр \EAX при помощи инструкции \TT{lea eax, DWORD PTR \_x\$[ebp]}.

Инструкция \LEA означает \IT{load effective address}, и часто используется для формирования адреса чего-либо ~(\myref{sec:LEA}).

Можно сказать, что в данном случае \LEA просто помещает в \EAX результат суммы значения в регистре \EBP и макроса \TT{\_x\$}.

Это тоже что и \INS{lea eax, [ebp-4]}.

Итак, от значения \EBP отнимается 4 и помещается в \EAX.
Далее значение \EAX заталкивается в стек и вызывается \scanf.

После этого вызывается \printf. Первый аргумент вызова строка:
\TT{You entered \%d...\textbackslash{}n}.

Второй аргумент: \INS{mov ecx, [ebp-4]}.
Эта инструкция помещает в \ECX не адрес переменной \TT{x}, а её значение.

Далее значение \ECX заталкивается в стек и вызывается \printf.

\EN{\input{patterns/04_scanf/1_simple/olly_EN}}
\RU{\input{patterns/04_scanf/1_simple/olly_RU}}
\ITA{\input{patterns/04_scanf/1_simple/olly_ITA}}
\DE{\input{patterns/04_scanf/1_simple/olly_DE}}
\FR{\input{patterns/04_scanf/1_simple/olly_FR}}
\PTBR{\input{patterns/04_scanf/1_simple/olly_PTBR}}


\myparagraph{GCC}

Попробуем тоже самое скомпилировать в Linux при помощи GCC 4.4.1:

\lstinputlisting[style=customasmx86]{patterns/04_scanf/1_simple/ex1_GCC.asm}

\myindex{puts() вместо printf()}
GCC заменил первый вызов \printf на \puts. Почему это было сделано, 
уже было описано ранее~(\myref{puts}).

% TODO: rewrite
%\RU{Почему \scanf переименовали в \TT{\_\_\_isoc99\_scanf}, я честно говоря, пока не знаю.}
%\EN{Why \scanf is renamed to \TT{\_\_\_isoc99\_scanf}, I do not know yet.}
% 
% Apparently it has to do with the ISO c99 standard compliance. By default GCC allows specifying a standard to adhere to.
% For example if you compile with -std=c89 the outputted assmebly file will contain scanf and not __isoc99__scanf. I guess current GCC version adhares to c99 by default.
% According to my understanding the two implementations differ in the set of suported modifyers (See printf man page)


Далее всё как и прежде~--- параметры заталкиваются через стек при помощи \MOV.

\myparagraph{Кстати}

Кстати, этот простой пример иллюстрирует то обстоятельство, что компилятор преобразует
список выражений в \CCpp-блоке просто в последовательный набор инструкций.
Между выражениями в \CCpp ничего нет, и в итоговом машинном коде между ними тоже ничего нет, 
управление переходит от одной инструкции к следующей за ней.

}
\PTBR{\subsubsection{x86}

\myparagraph{MSVC}

Aqui está o que o resultado depois de se compilar com o MSVC 2010:

% TODO to translate
\lstinputlisting[style=customasmx86]{patterns/04_scanf/1_simple/ex1_MSVC_EN.asm}

\TT{x} é uma variável local.

De acordo com os padrões de \CCpp ela só deve ser visível nessa função e não além dela.
Tradicionalmente, variáveis locais são guardadas na pilha.
Provavelmente há outras maneiras de alocá-las, mas no x86 é assim que é feito.

\myindex{x86!\Instructions!PUSH}
O objetivo da intrução que se segue após o cabeçalho da função, \INS{PUSH ECX},
não é para salvar o valor de \ECX
(perceba que não há a instrução \INS{POP ECX} no fim da função).

Na verdade, esse PUSH aloca 4 bytes na pilha para guardar a variável \TT{x}.

\label{stack_frame}
\myindex{\Stack!\PTBRph{}}
\myindex{x86!\Registers!EBP}
\TT{x} é para ser acessada com a ajuda do macro \TT{\_x\$} (que é igual a -4) e o registrador \EBP apontando para a posição atual.

Conforme a execução da função avança, \EBP está apontando para a posição atual da pilha,
sendo possível acessar variáveis locais e argumentos da função via \TT{EBP+offset}.

\myindex{x86!\Registers!ESP}
Também é possível usar \ESP para o mesmo objetivo, mas não é muito conveniente pois ele se altera com frequência.
O valor de \EBP pode ser visto como uma cópia do valor de \ESP no começo da execução da função.

Aqui está a aparência típica de uma pilha em um ambiente de 32-bits:

\begin{center}
\begin{tabular}{ | l | l | }
\hline
\dots & \dots \\
\hline
EBP-8 & variável local \#2, \MarkedInIDAAs{} \TT{var\_8} \\
\hline
EBP-4 & variável local \#1, \MarkedInIDAAs{} \TT{var\_4} \\
\hline
EBP & valor salvo de \EBP \\
\hline
EBP+4 & Endereço de retorno \\
\hline
EBP+8 & \argument \#1, \MarkedInIDAAs{} \TT{arg\_0} \\
\hline
EBP+0xC & \argument \#2, \MarkedInIDAAs{} \TT{arg\_4} \\
\hline
EBP+0x10 & \argument \#3, \MarkedInIDAAs{} \TT{arg\_8} \\
\hline
\dots & \dots \\
\hline
\end{tabular}
\end{center}

A função \scanf no nosso exemplo tem dois argumentos.

O primeiro é um ponteiro para a string contendo \TT{\%d} e a segunda é o endereço da variável \TT{x}.

\myindex{x86!\Instructions!LEA}
Primeiro, o endereço da variável \TT{x} é carregado no registrador \EAX pela instrução \TT{lea eax, DWORD PTR \_x\$[ebp].}.

\PTBRph{}

Nós podemos dizer que nesse caso, \LEA simplesmente armazena a soma do valor em \EBP e o macro \TT{\_x\$} no registrador \EAX.

É a mesma coisa que \INS{lea eax, [ebp-4]}.

Então, 4 está sendo subtraido do registrador \EBP e o resultado é carregado no registrador \EAX.
Depois, o valor em \EAX é empurrado para dentro da pilha e o \scanf é chamado.

\printf é chamado depois disso com seu primeiro argumento --- um ponteiro para a string:
\TT{You entered \%d...\textbackslash{}n}.

O segundo argumento é preparado com: \TT{mov ecx, [ebp-4]}.
A instrução armazena o valor de x e não o seu endereço, no registrador \ECX.

Depois, o valor em \ECX é armazenado na pilha e o último \printf é chamado.

% TODO: olly, 

\myparagraph{\PTBRph{}}

A propósito, esse simples exemplo é a demonstração do fato de que o compilador traduz a lista de expressões em \CCpp em uma lista sequêncial de instruções.
Não há nada entre as expressões em \CCpp, como também no código de máquina resultante, não há nada, o controle de fluxo passa de uma expressão para a outra diretamente.

}
\ITA{\subsubsection{x86}

\myparagraph{MSVC}

Questo e' cio' che si ottiene dopo la compilazione con MSVC 2010:

\lstinputlisting[style=customasmx86]{patterns/04_scanf/1_simple/ex1_MSVC_EN.asm}

\TT{x} e' una variabile locale.

In base allo standard \CCpp deve essere visibile soltanto in questa funzione e non in altri ambiti (esterni alla funzione).
Tradizionalmente, le variabili locali sono memorizzate sullo stack. 
Ci sono probabilmente altri modi per allocarle, ma in x86 e' cosi'.

\myindex{x86!\Instructions!PUSH}
Lo scopo dell'istruzione che segue il prologo della funzione, \TT{PUSH ECX}, non e' quello di salvare lo stato di \ECX  
(si noti infatti l'assenza della corrispondente istruzione \TT{POP ECX} alla fine della funzione).

Infatti alloca 4 byte sullo stack per memorizzare la variabile \TT{x}.

\label{stack_frame}
\myindex{\Stack!Stack frame}
\myindex{x86!\Registers!EBP}
\TT{x} sara' acceduta con l'aiuto della macro \TT{\_x\$} (che e' uguale a -4) ed il registro \EBP che punta al frame corrente.

Durante l'esecuzione delle funziona, \EBP punta allo \gls{stack frame} corrente 
rendendo possibile accedere alle variabili locali ed agli argomenti della funzione attraverso \TT{EBP+offset}.

\myindex{x86!\Registers!ESP}
E' anche possibile usare \ESP per lo stesso scopo, tuttavia non e' molto conveniente poiche' cambia di frequente.
Il valore di \EBP puo' essere pensato come uno \IT{stato congelato} del valore in \ESP all'inizio dell'esecuzione della funzione.

% FIXME1 это уже было в 02_stack?
Questo e' un tipico layout di uno \gls{stack frame} in un ambiente a 32-bit:

\begin{center}
\begin{tabular}{ | l | l | }
\hline
\dots & \dots \\
\hline
EBP-8 & local variable \#2, \MarkedInIDAAs{} \TT{var\_8} \\
\hline
EBP-4 & local variable \#1, \MarkedInIDAAs{} \TT{var\_4} \\
\hline
EBP & saved value of \EBP \\
\hline
EBP+4 & return address \\
\hline
EBP+8 & \argument \#1, \MarkedInIDAAs{} \TT{arg\_0} \\
\hline
EBP+0xC & \argument \#2, \MarkedInIDAAs{} \TT{arg\_4} \\
\hline
EBP+0x10 & \argument \#3, \MarkedInIDAAs{} \TT{arg\_8} \\
\hline
\dots & \dots \\
\hline
\end{tabular}
\end{center}

La funzione \scanf nel nostro esempio ha due argomenti.
Il primo e' un puntatore alla stringa contenente \TT{\%d} e il secondo e' l'indirizzo della variabile \TT{x}.

\myindex{x86!\Instructions!LEA}
Per prima cosa l'indirizzo della variabile \TT{x} e' caricato nel registro \EAX dall'istruzione \TT{lea eax, DWORD PTR \_x\$[ebp]}.

\LEA sta per \IT{load effective address}, ed e' spesso usata per formare un indirizzo ~(\myref{sec:LEA}).

Potremmo dire che in questo caso \LEA memorizza semplicemente la somma del valore nel registro \EBP e della macro \TT{\_x\$} nel registro \EAX.

E' l'equivalente di \INS{lea eax, [ebp-4]}.

Quindi, 4 viene sottratto dal valore del registro \EBP ed il risultato e' memorizzato nel registro \EAX.
Successivamente il registro \EAX e' messo sullo stack (push) e \scanf viene chiamata.

\printf viene chiamata subito dopo con il suo primo argomento --- un puntatore alla stringa:
\TT{You entered \%d...\textbackslash{}n}.

Il secondo argomento e' preparato con: \TT{mov ecx, [ebp-4]}.
L'istruzione memorizza il valore della variabile \TT{x},  non il suo indirizzo, nel registro \ECX.

Successivamente il valore in \ECX e' memorizzato sullo stack e l'ultima \printf viene chiamata.

\EN{\input{patterns/04_scanf/1_simple/olly_EN}}
\RU{\input{patterns/04_scanf/1_simple/olly_RU}}
\ITA{\input{patterns/04_scanf/1_simple/olly_ITA}}
\DE{\input{patterns/04_scanf/1_simple/olly_DE}}
\FR{\input{patterns/04_scanf/1_simple/olly_FR}}
\PTBR{\input{patterns/04_scanf/1_simple/olly_PTBR}}


\myparagraph{GCC}

Proviamo a compilare questo codice con GCC 4.4.1 su Linux:

\lstinputlisting[style=customasmx86]{patterns/04_scanf/1_simple/ex1_GCC.asm}

\myindex{puts() instead of printf()}
GCC ha sostituito la chiamata a \printf con \puts. La ragione per cui cio' avviene e' stata spiegata in ~(\myref{puts}).

% TODO: rewrite
%\RU{Почему \scanf переименовали в \TT{\_\_\_isoc99\_scanf}, я честно говоря, пока не знаю.}
%\EN{Why \scanf is renamed to \TT{\_\_\_isoc99\_scanf}, I do not know yet.}
% 
% Apparently it has to do with the ISO c99 standard compliance. By default GCC allows specifying a standard to adhere to.
% For example if you compile with -std=c89 the outputted assmebly file will contain scanf and not __isoc99__scanf. I guess current GCC version adhares to c99 by default.
% According to my understanding the two implementations differ in the set of suported modifyers (See printf man page)

Come nell'esempio compilato con MSVC ---gli argomenti sono messi sullo stack utilizzando l'istruzione \MOV.


}
\DE{\subsubsection{x86}

\myparagraph{MSVC}
Den folgenden Code erhalten wie nach dem Kompilieren mit MSVC 2010:

\lstinputlisting[style=customasmx86]{patterns/04_scanf/1_simple/ex1_MSVC_DE.asm}

\TT{x} ist eine lokale Variable.

Gemäß dem \CCpp-Standard darf diese nur innerhalb dieser Funktion sichtbar sein und nicht aus einem anderen, äußeren Scope.
Traditionell werden lokale Variablen auf dem Stack gespeichert.
Es gibt möglicherweise andere Wege sie anzulegen, aber in x86 geschieht es auf diese Weise.


\myindex{x86!\Instructions!PUSH}
Das Ziel des Befehls direkt nach dem Funktionsprolog, \TT{PUSH ECX}), ist es nicht, den Status von \ECX zu sichern
(man beachte, dass Fehlen eines entsprechenden \TT{POP ECX} im Funktionsepilog).
Tatsächlich reserviert der Befehl 4 Byte auf dem Stack, um die Variable $x$ speichern zu können.

\label{stack_frame}
\myindex{\Stack!Stack frame}
\myindex{x86!\Registers!EBP}
Auf \TT{x} wird mithilfe des \TT{\_x\$} Makros (es entspricht -4) und des \EBP Registers, das auf den aktuellen Stack Frame zeigt, zugegriffen. 
Während der Dauer der Funktionsausführung zeigt \EBP auf den aktuellen \glslink{stack frame}{Stack Frame}, wodurch mittels \TT{EBP+offset} auf lokalen Variablen und Funktionsargumente zugegriffen werden kann.

\TT{x} is to be accessed with the assistance of the \TT{\_x\$} macro (it equals to -4) and the \EBP register pointing to the current frame.

\myindex{x86!\Registers!ESP}
Es ist auch möglich, das \ESP Register zu diesem Zweck zu verwenden, aber dies ist ungebräuchlich, da es sich häufig verändert.
Der Wert von \EBP kann als eingefrorener Wert des Wertes von \ESP zu Beginn der Funktionsausführung verstanden werden.

It is also possible to use \ESP for the same purpose, although that is not very convenient since it changes frequently.
The value of the \EBP could be perceived as a \IT{frozen state} of the value in \ESP at the start of the function's execution.

% FIXME1 это уже было в 02_stack?
Hier ist ein typisches Layour eines Stack Frames in einer 32-Bit-Umgebung:

\begin{center}
\begin{tabular}{ | l | l | }
\hline
\dots & \dots \\
\hline
EBP-8 & local variable \#2, \MarkedInIDAAs{} \TT{var\_8} \\
\hline
EBP-4 & local variable \#1, \MarkedInIDAAs{} \TT{var\_4} \\
\hline
EBP & saved value of \EBP \\
\hline
EBP+4 & return address \\
\hline
EBP+8 & \argument \#1, \MarkedInIDAAs{} \TT{arg\_0} \\
\hline
EBP+0xC & \argument \#2, \MarkedInIDAAs{} \TT{arg\_4} \\
\hline
EBP+0x10 & \argument \#3, \MarkedInIDAAs{} \TT{arg\_8} \\
\hline
\dots & \dots \\
\hline
\end{tabular}
\end{center}
Die Funktion \scanf in unserem Beispiel hat zwei Argumente.

Das erste ist ein Pointer auf den String \TT{\%d} und das zweite ist die Adresse der Variablen \TT{x}.

\myindex{x86!\Instructions!LEA}
Zunächst wird die Adresse der Variablen $x$ durch den Befehl \\
\TT{lea eax, DWORD PTR \_x\$[ebp]} in das \EAX Register geladen.

\LEA steht für \IT{load effective address} und wird häufig benutzt, um eine Adresse zu erstellen ~(\myref{sec:LEA}).
In diesem Fall speichert \LEA einfach die Summe des \EBP Registers und des \TT{\_\$} Makros im Register \EAX.
Dies entspricht dem Befehl \INS{lea eax, [ebp-4]}.

Es wird also 4 von Wert in \EBP abgezogen und das Ergebnis in das Register \EAX geladen.
Danach wird der Wert in \EAX auf dem Stack abgelegt und \scanf wird aufgerufen.

Anschließend wird \printf mit einem Argument aufgerufen--einen Pointer auf den String:
\TT{You entered \%d...\textbackslash{}n}.

Das zweite Argument wird mit \TT{mov ecx, [ebp-4]} vorbereitet.
Dieser Befehl speichert den Wert der Variablen $x$ (nicht seine Adresse) im Register \ECX.

Schließlich wird der Wert in \ECX auf dem Stack gespeichert und das letzte \printf wird aufgerufen.

\EN{\input{patterns/04_scanf/1_simple/olly_EN}}
\RU{\input{patterns/04_scanf/1_simple/olly_RU}}
\ITA{\input{patterns/04_scanf/1_simple/olly_ITA}}
\DE{\input{patterns/04_scanf/1_simple/olly_DE}}
\FR{\input{patterns/04_scanf/1_simple/olly_FR}}
\PTBR{\input{patterns/04_scanf/1_simple/olly_PTBR}}


\myparagraph{GCC}

Kompilieren wir diesen Code mit GCC 4.4.1 unter Linux:

\lstinputlisting[style=customasmx86]{patterns/04_scanf/1_simple/ex1_GCC.asm}

\myindex{puts() instead of printf()}
GCC ersetzt den Aufruf von \printf durch einen Aufruf von \puts. Der Grund hierfür wurde bereits in ~(\myref{puts}) erklärt.

% TODO: rewrite
%\RU{Почему \scanf переименовали в \TT{\_\_\_isoc99\_scanf}, я честно говоря, пока не знаю.}
%\EN{Why \scanf is renamed to \TT{\_\_\_isoc99\_scanf}, I do not know yet.}
% 
% Apparently it has to do with the ISO c99 standard compliance. By default GCC allows specifying a standard to adhere to.
% For example if you compile with -std=c89 the outputted assmebly file will contain scanf and not __isoc99__scanf. I guess current GCC version adhares to c99 by default.
% According to my understanding the two implementations differ in the set of suported modifyers (See printf man page)
Genau wie im MSVC Beispiel werden die Argumente mithilfe des Befehls \MOV auf dem Stack abgelegt.

\myparagraph{By the way}
Dieses einfache Beispiel ist übrigens eine Demonstration der Tatsache, dass der Compiler eine Liste von Ausdrücken in einem
\CCpp-Block in eine sequentielle Liste von Befehlen übersetzt.
Es gibt nichts zwischen zwei \CCpp-Anweisungen und genauso verhält es sich auch im Maschinencode.
Der Control Flow geht von einem Ausdruck direkt an den folgenden über.
}
\FR{\subsubsection{x86}

\myparagraph{MSVC}

Voici ce que l'on obtient après avoir compilé avec MSVC 2010:

\lstinputlisting[style=customasmx86]{patterns/04_scanf/1_simple/ex1_MSVC_FR.asm}

\TT{x} est une variable locale.

D'après le standard \CCpp elle ne doit être visible que dans cette fonction et dans
aucune autre portée.
Traditionnellement, les variables locales sont stockées dans la pile.
Il y a probablement d'autres moyens de les allouer, mais en x86, c'est la façon de faire.

\myindex{x86!\Instructions!PUSH}
Le but de l'instruction suivant le prologue de la fonction, \TT{PUSH ECX}, n'est
pas de sauver l'état de \ECX (noter l'absence d'un \TT{POP ECX} à la fin de la
fonction).

En fait, cela alloue 4 octets sur la pile pour stocker la variable \TT{x}.

\label{stack_frame}
\myindex{\Stack!Stack frame}
\myindex{x86!\Registers!EBP}
\TT{x} est accèdée à l'aide de la macro \TT{\_x\$} (qui vaut -4) et du registre \EBP
qui pointe sur la structure de pile courante.

Pendant la durée de l'exécution de la fonction, \EBP pointe sur la \glslink{stack frame}{structure locale de pile}
courante, rendant possible l'accès aux variables locales et aux arguments de la
fonction via \TT{EBP+offset}.

\myindex{x86!\Registers!ESP}
Il est aussi possible d'utiliser \ESP dans le même but, bien que ça ne soit pas
très commode, car il change fréquemment.
La valeur de \EBP peut être perçue comme un \IT{état figé} de la valeur de \ESP
au début de l'exécution de la fonction.

% FIXME1 это уже было в 02_stack?
Voici une \glslink{stack frame}{strucutre de pile} typique dans un environnement 32-bit:

\begin{center}
\begin{tabular}{ | l | l | }
\hline
\dots & \dots \\
\hline
EBP-8 & variable locale \#2, \MarkedInIDAAs{} \TT{var\_8} \\
\hline
EBP-4 & variable locale \#1, \MarkedInIDAAs{} \TT{var\_4} \\
\hline
EBP & valeur sauvée de \EBP \\
\hline
EBP+4 & adresse de retour \\
\hline
EBP+8 & \argument \#1, \MarkedInIDAAs{} \TT{arg\_0} \\
\hline
EBP+0xC & \argument \#2, \MarkedInIDAAs{} \TT{arg\_4} \\
\hline
EBP+0x10 & \argument \#3, \MarkedInIDAAs{} \TT{arg\_8} \\
\hline
\dots & \dots \\
\hline
\end{tabular}
\end{center}

La fonction \scanf de notre exemple a deux arguments.

Le premier est un pointeur sur la chaîne contenant \TT{\%d} et le second est l'adresse
de la variable \TT{x}.

\myindex{x86!\Instructions!LEA}
Tout d'abord, l'adresse de la variable \TT{x} est chargée dans le registre \EAX
par l'instruction \\ \TT{lea eax, DWORD PTR \_x\$[ebp]}.

\LEA signifie \IT{load effective address} (charger l'adresse effective) et est souvent
utilisée pour composer une adresse ~(\myref{sec:LEA}).

Nous pouvons dire que dans ce cas, \LEA stocke simplement la somme de la valeur du
registre \EBP et de la macro \TT{\_x\$} dans le registre \EAX.

C'est la même chose que \INS{lea eax, [ebp-4]}.

Donc, 4 est soustrait de la valeur du registre \EBP et le résultat est chargé dans
le registre \EAX.
Ensuite, la valeur du registre \EAX est poussée sur la pile et \scanf est appelée.

\printf est appelée ensuite avec son premier argument --- un pointeur sur la chaîne:
\TT{You entered \%d...\textbackslash{}n}.

Le second argument est préparé avec: \TT{mov ecx, [ebp-4]}.
L'instruction stocke la valeur de la variable \TT{x} et non son adresse, dans le
registre \ECX.

Puis, la valeur de \ECX est stockée sur la pile et le dernier appel à \printf
est effectué.

\EN{\input{patterns/04_scanf/1_simple/olly_EN}}
\RU{\input{patterns/04_scanf/1_simple/olly_RU}}
\ITA{\input{patterns/04_scanf/1_simple/olly_ITA}}
\DE{\input{patterns/04_scanf/1_simple/olly_DE}}
\FR{\input{patterns/04_scanf/1_simple/olly_FR}}
\PTBR{\input{patterns/04_scanf/1_simple/olly_PTBR}}


\myparagraph{GCC}

Compilons ce code avec GCC 4.4.1 sous Linux:

\lstinputlisting[style=customasmx86]{patterns/04_scanf/1_simple/ex1_GCC.asm}

\myindex{puts() instead of printf()}
GCC a remplacé l'appel à \printf avec un appel à \puts. La raison de cela a été
expliquée dans ~(\myref{puts}).

% TODO: rewrite
%\RU{Почему \scanf переименовали в \TT{\_\_\_isoc99\_scanf}, я честно говоря, пока не знаю.}
%\EN{Why \scanf is renamed to \TT{\_\_\_isoc99\_scanf}, I do not know yet.}
% 
% Apparently it has to do with the ISO c99 standard compliance. By default GCC allows specifying a standard to adhere to.
% For example if you compile with -std=c89 the outputted assmebly file will contain scanf and not __isoc99__scanf. I guess current GCC version adhares to c99 by default.
% According to my understanding the two implementations differ in the set of suported modifyers (See printf man page)

Comme dans l'exemple avec MSVC---les arguments sont placés dans la pile avec l'instruction
\MOV.

\myparagraph{By the way}

Á propos, ce simple exemple est la démonstration du fait que le compilateur traduit
une liste d'expression en bloc-\CCpp en une liste séquentielle d'instructions.
% TODO FIXME: better translation / clarify ?
Il n'y a rien entre les expressions en \CCpp, et le résultat en code machine,
il n'y a rien entre le déroulement du flux de contrôle d'une expression à la suivante.
}

\EN{\subsubsection{x64}

\myindex{x86-64}
The picture here is similar with the difference that the registers, rather than the stack, are used for arguments passing.

\myparagraph{MSVC}

\lstinputlisting[caption=MSVC 2012 x64,style=customasmx86]{patterns/04_scanf/1_simple/ex1_MSVC_x64_EN.asm}

\myparagraph{GCC}

\lstinputlisting[caption=\Optimizing GCC 4.4.6 x64,style=customasmx86]{patterns/04_scanf/1_simple/ex1_GCC_x64_EN.s}

}
\RU{\subsubsection{x64}

\myindex{x86-64}
Всё то же самое, только используются регистры вместо стека для передачи аргументов функций.

\myparagraph{MSVC}

\lstinputlisting[caption=MSVC 2012 x64,style=customasmx86]{patterns/04_scanf/1_simple/ex1_MSVC_x64_RU.asm}

\myparagraph{GCC}

\lstinputlisting[caption=\Optimizing GCC 4.4.6 x64,style=customasmx86]{patterns/04_scanf/1_simple/ex1_GCC_x64_RU.s}
}
\PTBR{\subsubsection{x64}

\myindex{x86-64}
A situação aqui é parecida, mas com a diferença de que os registradores, ao invés da pilha, são usados para passar argumentos.

\myparagraph{MSVC}

% TODO to translate
\lstinputlisting[caption=MSVC 2012 x64,style=customasmx86]{patterns/04_scanf/1_simple/ex1_MSVC_x64_EN.asm}

\myparagraph{GCC}

% TODO to translate
\lstinputlisting[caption=\Optimizing GCC 4.4.6 x64,style=customasmx86]{patterns/04_scanf/1_simple/ex1_GCC_x64_EN.s}
}
\ITA{\subsubsection{x64}

\myindex{x86-64}
La situazione e' simile, con l'unica differenza che, per il passaggio degli argomenti, i registri sono usati al posto dello stack.

\myparagraph{MSVC}

\lstinputlisting[caption=MSVC 2012 x64,style=customasmx86]{patterns/04_scanf/1_simple/ex1_MSVC_x64_EN.asm}

\myparagraph{GCC}

\lstinputlisting[caption=\Optimizing GCC 4.4.6 x64,style=customasmx86]{patterns/04_scanf/1_simple/ex1_GCC_x64_EN.s}

}
\DE{\subsubsection{x64}

\myindex{x86-64}
Hier zeigt sich ein ähnliches Bild mit dem Unterschied, dass die Register anstelle des Stacks für die Übergabe der Funktionsargumente verwendet werden.

\myparagraph{MSVC}

\lstinputlisting[caption=MSVC 2012 x64,style=customasmx86]{patterns/04_scanf/1_simple/ex1_MSVC_x64_DE.asm}

\myparagraph{GCC}

\lstinputlisting[caption=\Optimizing GCC 4.4.6 x64,style=customasmx86]{patterns/04_scanf/1_simple/ex1_GCC_x64_DE.s}

}
\FR{\subsubsection{x64}

\myindex{x86-64}
Le schéma est ici similaire, avec la différence que les registres, plutôt que la
pile, sont utilisés pour le passage des arguments.

\myparagraph{MSVC}

\lstinputlisting[caption=MSVC 2012 x64,style=customasmx86]{patterns/04_scanf/1_simple/ex1_MSVC_x64_FR.asm}

\myparagraph{GCC}

\lstinputlisting[caption=GCC 4.4.6 x64 \Optimizing,style=customasmx86]{patterns/04_scanf/1_simple/ex1_GCC_x64_FR.s}

}

\EN{\subsubsection{ARM}

\myparagraph{\OptimizingKeilVI (\ThumbMode)}

\lstinputlisting[style=customasmARM]{patterns/04_scanf/1_simple/ARM_IDA.lst}

\myindex{\CLanguageElements!\Pointers}

In order for \scanf to be able to read item it needs a parameter---pointer to an \Tint.
\Tint is 32-bit, so we need 4 bytes to store it somewhere in memory, and it fits exactly in a 32-bit register.
\myindex{IDA!var\_?}
A place for the local variable \GTT{x} is allocated in the stack and \IDA
has named it \IT{var\_8}. It is not necessary, however, to allocate a such since \ac{SP} (\gls{stack pointer}) is already pointing to that space and it can be used directly.

So, \ac{SP}'s value is copied to the \Reg{1} register and, together with the format-string, passed to \scanf.
\myindex{ARM!\Instructions!LDR}
Later, with the help of the \INS{LDR} instruction, this value is moved from the stack to the \Reg{1} register in order to be passed to \printf.

\myparagraph{ARM64}

\lstinputlisting[caption=\NonOptimizing GCC 4.9.1 ARM64,numbers=left,style=customasmARM]{patterns/04_scanf/1_simple/ARM64_GCC491_O0_EN.s}

There is 32 bytes are allocated for stack frame, which is bigger than it needed. Perhaps some memory aligning issue?
The most interesting part is finding space for the $x$ variable in the stack frame (line 22).
Why 28? Somehow, compiler decided to place this variable at the end of stack frame instead of beginning.
The address is passed to \scanf, which just stores the user input value in the memory at that address.
This is 32-bit value of type \Tint.
The value is fetched at line 27 and then passed to \printf.

}
\RU{\subsubsection{ARM}

\myparagraph{\OptimizingKeilVI (\ThumbMode)}

\lstinputlisting[style=customasmARM]{patterns/04_scanf/1_simple/ARM_IDA.lst}

\myindex{\CLanguageElements!\Pointers}
Чтобы \scanf мог вернуть значение, ему нужно передать указатель на переменную типа \Tint.
\Tint~--- 32-битное значение, для его хранения нужно только 4 байта, и оно помещается в 32-битный регистр.

\myindex{IDA!var\_?}
Место для локальной переменной \GTT{x} выделяется в стеке, \IDA наименовала её \IT{var\_8}. 
Впрочем, место для неё выделять не обязательно, т.к. \glslink{stack pointer}{указатель стека} \ac{SP} уже указывает на место, 
свободное для использования.
Так что значение указателя \ac{SP} копируется в регистр \Reg{1}, и вместе с format-строкой, 
передается в \scanf.

\myindex{ARM!\Instructions!LDR}
Позже, при помощи инструкции \INS{LDR}, это значение перемещается из стека в регистр \Reg{1}, чтобы быть переданным в \printf.

\myparagraph{ARM64}

\lstinputlisting[caption=\NonOptimizing GCC 4.9.1 ARM64,numbers=left,style=customasmARM]{patterns/04_scanf/1_simple/ARM64_GCC491_O0_RU.s}

Под стековый фрейм выделяется 32 байта, что больше чем нужно. Может быть, это связано с выравниваем по границе памяти?
Самая интересная часть~--- это поиск места под переменную $x$ в стековом фрейме (строка 22).
Почему 28? Почему-то, компилятор решил расположить эту переменную в конце стекового фрейма, а не в начале.
Адрес потом передается в \scanf, которая просто сохраняет значение, введенное пользователем, в памяти по этому адресу.
Это 32-битное значение типа \Tint.
Значение загружается в строке 27 и затем передается в \printf.

}
\ITA{\subsubsection{ARM}

\myparagraph{\OptimizingKeilVI (\ThumbMode)}

\lstinputlisting[style=customasmARM]{patterns/04_scanf/1_simple/ARM_IDA.lst}

\myindex{\CLanguageElements!\Pointers}

Affinche' \scanf possa leggere l'input, necessita di un parametro ---puntatore ad un \Tint.
\Tint e' 32-bit, quindi servono 4 byte per memorizzarlo da qualche parte in memoria, e entra perfettamente in un registro a 32-bit.
\myindex{IDA!var\_?}
Uno spazio per la variabile locale \GTT{x} e' allocato nello stack e \IDA
lo ha chiamato \IT{var\_8}. Non e' comunque necessario allocarlo in questo modo poiche' \ac{SP} (\gls{stack pointer}) punta gia' a quella posizione e puo' essere usato direttamente.

Successivamente il valore di \ac{SP} e' copiato nel registro \Reg{1} e sono passati, insieme alla format-string, a \scanf.
\myindex{ARM!\Instructions!LDR}
Questo valore, con l'aiuto dell'istruzione \INS{LDR} , viene poi spostato dallo stakc al registro \Reg{1} per essere passato a \printf.

\myparagraph{ARM64}

\lstinputlisting[caption=\NonOptimizing GCC 4.9.1 ARM64,numbers=left,style=customasmARM]{patterns/04_scanf/1_simple/ARM64_GCC491_O0_EN.s}

Ci sono 32 byte allocati per lo stack frame, che e' piu' grande del necessario. Forse a causa di meccanismi di allineamento della memoria?
La parte piu' interessante e' quella in cui trova spazio per la variabile $x$ nello stack frame (riga 22).
Perche' 28? Il compilatore ha in qualche modo deciso di piazzare questa variabile alla fine dello stack frame anziche' all'inizio.
L'indirizzo e' passato a \scanf, che memorizzera' il valore immesso dall'utente nella memoria a quell'indirizzo.
Si tratta di un valore a 32-bit di tipo \Tint.
Il valore e' recuperato successivamente a riga 27 e passato a \printf.

}
\DE{\subsubsection{ARM}

\myparagraph{\OptimizingKeilVI (\ThumbMode)}

\lstinputlisting[style=customasmARM]{patterns/04_scanf/1_simple/ARM_IDA.lst}

\myindex{\CLanguageElements!\Pointers}
Damit \scanf Elemente einlesen kann, benötigt die Funktion einen Paramter--einen Pointer vom Typ \Tint.
\Tint hat die Größe 32 Bit, wir benötigen also 4 Byte, um den Wert im Speicher abzulegen, und passt daher genau in ein 32-Bit-Register.
\myindex{IDA!var\_?}
Auf dem Stack wird Platz für die lokalen Variable \GTT{x} reserviert und IDA bezeichnet diese Variable mit \IT{var\_8}. 
Eigentlich ist aber an dieser Stelle gar nicht notwendig, Platz auf dem Stack zu reservieren, da \ac{SP} (\gls{stack pointer} 
bereits auf die Adresse zeigt und auch direkt verwendet werden kann.

Der Wert von \ac{SP} wird also in das \Reg{1} Register kopiert und zusammen mit dem Formatierungsstring an \scanf übergeben.
\myindex{ARM!\Instructions!LDR}
Später wird mithilfe des \INS{LDR} Befehls dieser Wert vom Stack in das \Reg{1} Register verschoben um an \printf übergeben werden zu können.

\myparagraph{ARM64}

\lstinputlisting[caption=\NonOptimizing GCC 4.9.1 ARM64,numbers=left,style=customasmARM]{patterns/04_scanf/1_simple/ARM64_GCC491_O0_DE.s}

Im Stack Frame werden 32 Byte reserviert, was deutlich mehr als benötigt ist. Vielleicht handelt es sich um eine Frage des Aligning (dt. Angleichens) von Speicheradressen.
Der interessanteste Teil ist, im Stack Frame einen Platz für die Variable $x$ zu finden (Zeile 22).
Warum 28? Irgendwie hat der Compiler entschieden die Variable am Ende des Stack Frames anstatt an dessen Beginn abzulegen.
Die Adresse wird an \scanf übergeben; diese Funktion speichert den Userinput an der genannten Adresse im Speicher.
Es handelt sich hier um einen 32-Bit-Wert vom Typ \Tint. 
Der Wert wird in Zeile 27 abgeholt und dann an \printf übergeben.


}
\FR{\subsubsection{ARM}

\myparagraph{\OptimizingKeilVI (\ThumbMode)}

\lstinputlisting[style=customasmARM]{patterns/04_scanf/1_simple/ARM_IDA.lst}

\myindex{\CLanguageElements!\Pointers}

Afin que \scanf puisse lire l'item, elle a besoin d'un paramètre---un pointeur sur un \Tint.
Le type \Tint est 32-bit, donc nous avons besoin de 4 octets pour le stocker quelque
part en mémoire, et il tient exactement dans un registre 32-bit.
\myindex{IDA!var\_?}
De l'espace pour la variable locale \GTT{x} est allouée sur la pile et \IDA l'a
nommée \IT{var\_8}. Il n'est toutefois pas nécessaire de définir cette macro, puisque
le \ac{SP} (\glslink{stack pointer}{pointeur de pile}) pointe déjà sur cet espace et
peut être utilisé directement.

Doc, la valeur de \ac{SP} est copiée dans la registre \Reg{1} et, avec la chaîne
de format, passée à \scanf.
\myindex{ARM!\Instructions!LDR}
Plus tard, avec l'aide de l'instruction \INS{LDR}, cette valeur est copiée depuis
la pile vers le registre \Reg{1} afin de la passer à \printf.

\myparagraph{ARM64}

\lstinputlisting[caption=GCC 4.9.1 ARM64 \NonOptimizing,numbers=left,style=customasmARM]{patterns/04_scanf/1_simple/ARM64_GCC491_O0_FR.s}

Il y a 32 octets alloués pour la structure de pile, qui est plus que nécessaire.
Peut-être dans un soucis d'alignement de mémoire?
La partie la plus intéressante est de trouver de l'espace pour la variable $x$ dans
la structure de pile (ligne 22).
Pourquoi 28? Pour une certaine raison, le compilateur a décidé de stocker cette
variable à la fin de la structure de pile locale au lieu du début.
L'adresse est passée à \scanf, qui stocke l'entrée de l'utilisateur en mémoire à
cette adresse.
Il s'agit d'une valeur sur 32-bit de type \Tint.
La valeur est prise à la ligne 27 puis passée à \printf.

}

\EN{\subsubsection{MIPS}

A place in the local stack is allocated for the $x$ variable, and it is to be referred as $\$sp+24$.
\myindex{MIPS!\Instructions!LW}

Its address is passed to \scanf, and the user input values is loaded using the \INS{LW} (\q{Load Word}) instruction
and then passed to \printf.

\lstinputlisting[caption=\Optimizing GCC 4.4.5 (\assemblyOutput),style=customasmMIPS]{patterns/04_scanf/1_simple/MIPS/ex1.O3_EN.s}

IDA displays the stack layout as follows:

\lstinputlisting[caption=\Optimizing GCC 4.4.5 (IDA),style=customasmMIPS]{patterns/04_scanf/1_simple/MIPS/ex1.O3.IDA_EN.lst}

% TODO non-optimized version?
}
\RU{\subsubsection{MIPS}

Для переменной $x$ выделено место в стеке, и к нему будут производиться обращения как $\$sp+24$.

\myindex{MIPS!\Instructions!LW}
Её адрес передается в \scanf, а значение прочитанное от пользователя загружается используя 
инструкцию \INS{LW} (\q{Load Word}~--- загрузить слово) и затем оно передается в \printf.

\lstinputlisting[caption=\Optimizing GCC 4.4.5 (\assemblyOutput),style=customasmMIPS]{patterns/04_scanf/1_simple/MIPS/ex1.O3_RU.s}

IDA показывает разметку стека следующим образом:

\lstinputlisting[caption=\Optimizing GCC 4.4.5 (IDA),style=customasmMIPS]{patterns/04_scanf/1_simple/MIPS/ex1.O3.IDA_RU.lst}

% TODO non-optimized version?
}
\ITA{\subsubsection{MIPS}

Nello stack locale viene allocato spazio per la variabile $x$ , a cui viene fatto riferimento come $\$sp+24$.
\myindex{MIPS!\Instructions!LW}

Il suo indirizzo è passato a \scanf, il valore immesso dall'utente è caricato usand l'istruzione \INS{LW} (\q{Load Word}) ed è infine passato a \printf.

\lstinputlisting[caption=\Optimizing GCC 4.4.5 (\assemblyOutput),style=customasmMIPS]{patterns/04_scanf/1_simple/MIPS/ex1.O3_EN.s}

IDA mostra il layout dello stack nel modo seguente:

\lstinputlisting[caption=\Optimizing GCC 4.4.5 (IDA),style=customasmMIPS]{patterns/04_scanf/1_simple/MIPS/ex1.O3.IDA_EN.lst}

% TODO non-optimized version?
}
\DE{\subsubsection{MIPS}
Auf dem lokalen Stack wird Platz für die Variable $x$ reserviert und als $\$sp+24$ referenziert.

\myindex{MIPS!\Instructions!LW}
Die Adresse wird an \scanf übergeben und der Userinput wird mithilfe des Befehls \INS{LW} (\q{Load Word}) geladen und dann an \printf übergeben.

\lstinputlisting[caption=\Optimizing GCC 4.4.5 (\assemblyOutput),style=customasmMIPS]{patterns/04_scanf/1_simple/MIPS/ex1.O3_DE.s}

IDA stellt das Stack Layout wie folgt dar:

\lstinputlisting[caption=\Optimizing GCC 4.4.5 (IDA),style=customasmMIPS]{patterns/04_scanf/1_simple/MIPS/ex1.O3.IDA_DE.lst}

% TODO non-optimized version?
}
\FR{\subsubsection{MIPS}

%%A place in the local stack is allocated for the $x$ variable, and it is to be referred as $\$sp+24$.
Une place est allouée sur la pile locale pour la variable $x$, et elle doit être appelée par $\$sp+24$.
\myindex{MIPS!\Instructions!LW}

%%Its address is passed to \scanf, and the user input values is loaded using the \INS{LW} (\q{Load Word}) instruction
%%and then passed to \printf.
Son adresse est passée à \scanf, et l'entrée de l'utilisateur est chargée en utilisant
l'instruction \INS{LW} (\q{Load Word}).

%%\lstinputlisting[caption=\Optimizing GCC 4.4.5 (\assemblyOutput),style=customasmMIPS]{patterns/04_scanf/1_simple/MIPS/ex1.O3_EN.s}
\lstinputlisting[caption=GCC 4.4.5 \Optimizing (\assemblyOutput),style=customasmMIPS]{patterns/04_scanf/1_simple/MIPS/ex1.O3_EN.s}

%%IDA displays the stack layout as follows:
IDA affiche la disposition de la pile comme suit:

%%\lstinputlisting[caption=\Optimizing GCC 4.4.5 (IDA),style=customasmMIPS]{patterns/04_scanf/1_simple/MIPS/ex1.O3.IDA_EN.lst}
\lstinputlisting[caption=GCC 4.4.5 \Optimizing (IDA),style=customasmMIPS]{patterns/04_scanf/1_simple/MIPS/ex1.O3.IDA_EN.lst}

% TODO non-optimized version?
}



