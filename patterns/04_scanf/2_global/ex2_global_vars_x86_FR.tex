\subsubsection{MSVC: x86}

\lstinputlisting[style=customasmx86]{patterns/04_scanf/2_global/ex2_MSVC.asm}

Dans ce cas, la variable \TT{x} est définie dans la section \TT{\_DATA} et il n'y
a pas de mémoire allouée sur la pile locale. Elle est accédée directement, pas par
la pile.
Les variables globales non initialisées ne prennent pas de place dans le fichier
exécutable (en effet, pourquoi aurait-on besoin d'allouer de l'espace pour des variables
initialement mises à zéro ?), mais lorsque quelqu'un accède à leur adresse, l'\ac{OS}
va y allouer un bloc de zéros\footnote{cwC'est comme ça que secomporte les \ac{VM}}.

Maintenant, assignons explicitement une valeur à la variable:

\lstinputlisting[style=customc]{patterns/04_scanf/2_global/default_value_FR.c}

Nous obtenons:

\begin{lstlisting}[style=customasmx86]
_DATA	SEGMENT
_x	DD	0aH

...
\end{lstlisting}

Ici nous voyons une valeur \TT{0xA} de type DWORD (DD signifie DWORD = 32 bit) pour
cette variable.

Si vous ouvrez le .exe compilé dans \IDA, vous pouvez voir la variable \IT{x} placée
au début du segment \TT{\_DATA}, et après elle vous pouvez voir la chaîne de texte.

Si vous ouvrez le .exe compilé de l'exemple précédent dans \IDA, oú la valeur de
\IT{x} n'était pas mise, vous verrez quelque chose comme ça:

\lstinputlisting[caption=\IDA,style=customasmx86]{patterns/04_scanf/2_global/IDA.lst}

\TT{\_x} est marquée avec \TT{?} avec le reste des variables qui ne doivent pas être
initialisées.
Ceci implique qu'après avoir chargé le .exe en mémoire, de l'espace pour toutes ces
variables doit être alloué et rempli avec des zéros \InSqBrackets{\CNineNineStd 6.7.8p10}.
Mais dans le fichier .exe, ces variables non initialisées n'occupent rien du tout.
C'est pratique pour les gros tableaux, par exemple.

\EN{\clearpage
\subsubsection{MSVC: x86 + \olly}
\myindex{\olly}

Things are even simpler here:

\begin{figure}[H]
\centering
\myincludegraphics{patterns/04_scanf/2_global/ex2_olly_1.png}
\caption{\olly: after \scanf execution}
\label{fig:scanf_ex2_olly_1}
\end{figure}

The variable is located in the data segment.
After the \PUSH instruction (pushing the address of $x$) gets executed, 
the address appears in the stack window. Right-click on that row and select \q{Follow in dump}.
The variable will appear in the memory window on the left.
After we have entered 123 in the console, 
\TT{0x7B} appears in the memory window (see the highlighted screenshot regions).

But why is the first byte \TT{7B}?
Thinking logically, \TT{00 00 00 7B} must be there.
The cause for this is referred as  \gls{endianness}, and x86 uses \IT{little-endian}.
This implies that the lowest byte is written first, and the highest written last.
Read more about it at: \myref{sec:endianness}.
Back to the example, the 32-bit value is loaded from this memory address into \EAX and passed to \printf.

The memory address of $x$ is \TT{0x00C53394}.

\clearpage
In \olly we can review the process memory map (Alt-M)
and we can see that this address is inside the \TT{.data} PE-segment of our program:

\label{olly_memory_map_example}
\begin{figure}[H]
\centering
\myincludegraphics{patterns/04_scanf/2_global/ex2_olly_2.png}
\caption{\olly: process memory map}
\label{fig:scanf_ex2_olly_2}
\end{figure}

}
\RU{\clearpage
\subsubsection{MSVC: x86 + \olly}
\myindex{\olly}

Тут даже проще:

\begin{figure}[H]
\centering
\myincludegraphics{patterns/04_scanf/2_global/ex2_olly_1.png}
\caption{\olly: после исполнения \scanf}
\label{fig:scanf_ex2_olly_1}
\end{figure}

Переменная хранится в сегменте данных.
Кстати, после исполнения инструкции \PUSH (заталкивающей адрес $x$) адрес появится в стеке, 
и на этом элементе можно нажать правой кнопкой, выбрать \q{Follow in dump}.
И в окне памяти слева появится эта переменная.

После того как в консоли введем 123, здесь появится \TT{0x7B}.

Почему самый первый байт это \TT{7B}?
По логике вещей, здесь должно было бы быть \TT{00 00 00 7B}.
Это называется \gls{endianness}, и в x86 принят формат \IT{little-endian}.
Это означает, что в начале записывается самый младший байт, а заканчивается самым старшим байтом.
Больше об этом: \myref{sec:endianness}.

Позже из этого места в памяти 32-битное значение загружается в \EAX и передается в \printf.

Адрес переменной $x$ в памяти \TT{0x00C53394}.

\clearpage
\label{olly_memory_map_example}

В \olly{} мы можем посмотреть карту памяти процесса (Alt-M) и увидим, что этот адрес
внутри PE-сегмента \TT{.data} нашей программы:

\begin{figure}[H]
\centering
\myincludegraphics{patterns/04_scanf/2_global/ex2_olly_2.png}
\caption{\olly: карта памяти процесса}
\label{fig:scanf_ex2_olly_2}
\end{figure}
}
\ITA{\clearpage
\subsubsection{MSVC: x86 + \olly}
\myindex{\olly}

Il quadro qui è ancora più semplice:

\begin{figure}[H]
\centering
\myincludegraphics{patterns/04_scanf/2_global/ex2_olly_1.png}
\caption{\olly: after \scanf execution}
\label{fig:scanf_ex2_olly_1}
\end{figure}

La variabile è collocata nel data segment.
Dopo che l'istruzione \PUSH (che fa il push dell'indirizzo di $x$) viene eseguita, 
l'indirizzo appare nella finestra dello stack. Facciamo click destro su quella riga e selezioniamo \q{Follow in dump}.
La variabile apparirà nella finestra di memoria a sinistra.
Dopo aver inserito il valore 123 in console, 
\TT{0x7B} apparirà nella finestra della memoria (vedere regioni evidenziate nello screenshot).

Ma perchè il primo byte è \TT{7B}?
A rigor di logica, dovremmo trovare \TT{00 00 00 7B}.
La causa per cui troviamo invece \TT{7B} è detta \gls{endianness}, e x86 usa la convenzione \IT{little-endian}.
Ciò significa che il byte piu basso è scritto per primo, e quello più alto per ultimo.
Maggiori informazioni sono disponibili nella sezione: \myref{sec:endianness}.
Tornando all'esempio, il valore a 32-bit è caricato da questo indirizzo di memoria in \EAX e passato a \printf.

L'indirizzo in memoria di $x$ è \TT{0x00C53394}.

\clearpage
\label{olly_memory_map_example}

In \olly possiamo osservare la mappa di memoria di un processo  (process memory map, Alt-M)
e notare che questo indirizzo è dentro il segmento PE \TT{.data} del nostro programma:

\begin{figure}[H]
\centering
\myincludegraphics{patterns/04_scanf/2_global/ex2_olly_2.png}
\caption{\olly: process memory map}
\label{fig:scanf_ex2_olly_2}
\end{figure}

}
\DE{\clearpage
\subsubsection{MSVC: x86 + \olly}
\myindex{\olly}

Hier sehen die Dinge noch einfacher aus:

\begin{figure}[H]
\centering
\myincludegraphics{patterns/04_scanf/2_global/ex2_olly_1.png}
\caption{\olly: nach Ausführung von \scanf}
\label{fig:scanf_ex2_olly_1}
\end{figure}

Die Variable befindet sich im Datensegment.
Nachdem der \PUSH Befehl (der die Adresse von $x$ speichert) ausgeführt worden ist,
erscheint die Adresse im Stackfenster. Wir machen einen Rechtsklick auf die Zeile und wählen \q{Follow in dump}.
Die Variable erscheint nun im Speicherfenster auf der linken Seite. 
Nachdem wir in der Konsole 123 eingegeben haben, erscheint \TT{0x7B} im Speicherfenster (siehe markiertes Feld im
Screenshot).

Warum ist das erste Byte \TT{7B}?
Logisch gedacht müsste dort \TT{00 00 00 7B} sein. 
Der Grund dafür ist die sogenannte \glslink{endianness}{Endianess} und x86 verwendet \IT{litte Endian}. 
Dies bedeutet, dass das niederwertigste Byte zuerst und das höchstwertigste zuletzt geschrieben werden.
Für mehr Informationen dazu siehe: \myref{sec:endianness}.
Zurück zu Beispiel: der 32-Bit-Wert wird von dieser Speicheradresse nach \EAX geladen und an \printf übergeben. 

Die Speicheradresse von $x$ ist \TT{0x00C53394}.

\clearpage
In \olly können wir die Speicherzuordnung des Prozesses nachvollziehen (Alt-M) und wir erkennen, dass sich diese Adresse
innerhalb des \TT{.data} PE-Segments von unserem Programm befindet:

\label{olly_memory_map_example}
\begin{figure}[H]
\centering
\myincludegraphics{patterns/04_scanf/2_global/ex2_olly_2.png}
\caption{\olly: Speicherzuordnung}
\label{fig:scanf_ex2_olly_2}
\end{figure}

}
\FR{\clearpage
\subsubsection{MSVC: x86 + \olly}
\myindex{\olly}

Les choses sont encore plus simple ici:

\begin{figure}[H]
\centering
\myincludegraphics{patterns/04_scanf/2_global/ex2_olly_1.png}
\caption{\olly: après l'exécution de \scanf}
\label{fig:scanf_ex2_olly_1}
\end{figure}

La variable se trouve dans le segment de données.
Après que l'instruction \PUSH (pousser l'adresse de $x$) ait été exécutée,
l'adresse apparaît dans la fenêtre de la pile. Cliquer droit sur cette ligne
et choisir \q{Follow in dump}. % TODO olly French ?
La variable va apparaître dans la fenêtre de la mémoire sur la gauche.
Après que nous ayons entré 123 dans la console, \TT{0x7B} apparaît dans la fenêtre
de la mémoire (voir les régions surlignées dans la copie d'écran).

Mais pourquoi est-ce que le premier octet est \TT{7B}?
Logiquement, Il devrait y avoir \TT{00 00 00 7B} ici.
La cause de ceci est référé comme \gls{endianness}, et x86 utilise \IT{little-endian}.
Cela implique que l'octet le plus faible poids est écrit en premier, et le plus fort
en dernier.
Voir à ce propos: \myref{sec:endianness}.
Revenons à l'exemple, la valeur 32-bit est chargée depuis son adresse mémoire
dans \EAX et passée à \printf.

L'adresse mémoire de $x$ est \TT{0x00C53394}.

\clearpage
Dans \olly nous pouvons examiner l'espace mémoire du processus  (Alt-M) et nous
pouvons voir que cette adresse se trouve dans le PE-segment \TT{.data} de notre
programme:

\label{olly_memory_map_example}
\begin{figure}[H]
\centering
\myincludegraphics{patterns/04_scanf/2_global/ex2_olly_2.png}
\caption{\olly: espace mémoire du processus}
\label{fig:scanf_ex2_olly_2}
\end{figure}

}
\PTBR{\clearpage
\subsubsection{MSVC: x86 + \olly}
\myindex{\olly}

Um pouco mais simples aqui:

\begin{figure}[H]
\centering
\myincludegraphics{patterns/04_scanf/2_global/ex2_olly_1.png}
\caption{\olly: depois da execução do \scanf}
\label{fig:scanf_ex2_olly_1}
\end{figure}

A variável está localizada na seção de data.

Depois da instrução \PUSH (colocando o endereço de $x$) é executada, o endereço aparece no topo da pilha. Clique com o botão direito naquela linha e selecione \q{Follow in dump}.

A variável aparecerá na janela de memória na esquerda.
Após digitarmos 123 no console, \TT{0x7B}  aparece na janela de memória(veja a região marcada no screenshot).

Mas, porque o primeiro byte é \TT{7B}?
Pela lógica deveria ser \TT{00 00 00 7B}.
O motivo para isso é conhecido como \gls{endianness},e a plataforma x86 usa o que é chamado \IT{little-endian}.
Isso implica que a parte de baixa ordem do byte é escrita primeiro e a parte alta é escrita por último.

Leia mais sobre em: \myref{sec:endianness}.
De volta ao exemplo, o valor de 32-bits é carregado do endereço da memória pro \EAX e passado para o \printf.

O endereço de memória de $x$ é \TT{0x00C53394}.

\clearpage
No \olly podemos rever o estado da memória do processo (Alt+M) e podemos ver que este endereço está dentro do da seção \TT{.data} do nosso arquivo PE:

\label{olly_memory_map_example}
\begin{figure}[H]
\centering
\myincludegraphics{patterns/04_scanf/2_global/ex2_olly_2.png}
\caption{\olly: Estado da memória}
\label{fig:scanf_ex2_olly_2}
\end{figure}

}

\subsubsection{GCC: x86}

\myindex{ELF}
Le schéma sous Linux est presque le même, avec la différence que les variables
non initialisées se trouvent dans le segment \TT{\_bss}.
Dans un fichier \ac{ELF} ce segment possède les attributs suivants:

\begin{lstlisting}
; Segment type: Uninitialized
; Segment permissions: Read/Write
\end{lstlisting}

Si toutefois vous initialisez la variable avec une velauer quelconque, e.g. 10,
elle sera placée dans le segment \TT{\_data}, qui possède les attributs suivants:

\begin{lstlisting}
; Segment type: Pure data
; Segment permissions: Read/Write
\end{lstlisting}

\subsubsection{MSVC: x64}

\lstinputlisting[caption=MSVC 2012 x64,style=customasmx86]{patterns/04_scanf/2_global/ex2_MSVC_x64_FR.asm}

Le code est presque le même qu'en x86.
Notez toutefois que l'adresse de la variable $x$ est passée à \TT{scanf()} en utilisant
une instruction \LEA, tandis que la valeur de la variable est passée au second \printf
en utilisant une instruction \MOV.
\TT{DWORD PTR}---fait partie du langage d'assemblage (aucune relation avec le code machine),
indique que la taille de la variable est 32-bit et que l'instruction \MOV doit être
encodée en conséquence.

