\subsection{Erreur courante}

C'est une erreur très courante (et/ou une typo) de passer la valeur de \IT{x} au
lieu d'un pointeur sur \IT{x}:

\lstinputlisting[style=customc]{patterns/04_scanf/error.c}

Donc que se passe-t-il ici?
\IT{x} n'est pas non-initialisée et contient des données aléatoires de la pile
locale.
Lorsque \scanf est appelée, elle prend la chaîne de l'utilisateur, la convertit
en nombre et essaye de l'écrire dans \IT{x}, la considérant comme une adresse en
mémoire.
Mais il s'agit de bruit aléatoire, donc \scanf va essayer d'écrire à une adresse
aléatoire.
Très probablement, le processus va planter.

\myindex{0xCCCCCCCC}
\myindex{0x0BADF00D}
Assez intéressant, certaines bibliothèques \ac{CRT} compilées en debug, mettent
un signe distinctif lors de l'allocation de la mémoire, comme 0xCCCCCCCC ou
0x0BADF00D etc.
Dans ce cas, \IT{x} peut contenir 0xCCCCCCCC, et \scanf va essayer d'écrire à
l'adresse 0xCCCCCCCC.
Et si vous remarquez que quelque chose dans votre processus essaye d'écrire à
l'adresse 0xCCCCCCCC, vous saurez qu'une variable non initialisée (ou un pointeur)
a été utilisée sans initialisation préalable.
C'est mieux que si la mémoire nouvellement allouée est juste mise à zéro.

