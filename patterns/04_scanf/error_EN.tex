\subsection{Popular mistake}

It's a very popular mistake (and/or typo) to pass value of \IT{x} instead of pointer to \IT{x}:

\lstinputlisting[style=customc]{patterns/04_scanf/error.c}

So what happens here?
\IT{x} is not uninitialized and contains some random noise from local stack.
When \scanf called, it takes string from user, parses it into number and tries to write it into \IT{x}, treating
it as an address in memory.
But there is a random noise, so \scanf will try to write at random address.
Most likely, the process will crash.

\myindex{0xCCCCCCCC}
\myindex{0x0BADF00D}
Interestingly enough, some \ac{CRT} libraries in debug build, put visually distinctive patterns
into memory just allocated, like 0xCCCCCCCC or 0x0BADF00D and so on.
In this case, \IT{x} may contain 0xCCCCCCCC, and \scanf would try to write at address 0xCCCCCCCC.
And if you'll notice that something in your process tries to write at address 0xCCCCCCCC, you'll know
that uninitialized variable (or pointer) gets used without prior initialization.
This is better than as if newly allocated memory is just cleared.

