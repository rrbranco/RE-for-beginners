\clearpage
\subsubsection{MSVC: x86 + \olly}

Vamos hackear nosso programa no \olly, forçando-o a acreditar que o \scanf sempre funciona sem erro.

Quando o endereço de uma variável local é passado para o \scanf, a variável geralmente contém algum lixo de memória, neste caso \TT{0x6E494714}:

\begin{figure}[H]
\centering
\myincludegraphics{patterns/04_scanf/3_checking_retval/olly_1.png}
\caption{\olly: Passando o endereço da variável pro \scanf}
\label{fig:scanf_ex3_olly_1}
\end{figure}

\clearpage
Enquanto \scanf o está em execução, ja janela do console entre com um valor que não é um número, por exemplo \q{asdasd}.

\scanf termina com 0 no \EAX,  o que indica que houve um erro:

\begin{figure}[H]
\centering
\myincludegraphics{patterns/04_scanf/3_checking_retval/olly_2.png}
\caption{\olly: \scanf retornando erro}
\label{fig:scanf_ex3_olly_2}
\end{figure}

Também podemos checar o valor da variável local na pilha e perceba que o valor não mudou.
O que levaria o \scanf escrever lá?
Simplesmente não fez nada, exceto retornar zero.

Vamos tentar \q{hack} nosso programa.
Clique com o botão direito do mouse em \EAX, entre outras opções selecione \q{Set to 1}. É o que precisamos.

Agora temos 1 em \EAX, a próxima e o \printf irá imprimir o valor da variável na pilha.

Quando executamos o programa (F9) podemos ver o seguinte no console:

\lstinputlisting[caption=console window]{patterns/04_scanf/3_checking_retval/console.txt}

De fato, 1850296084 é a representação decimal do número na pilha (\TT{0x6E494714})!