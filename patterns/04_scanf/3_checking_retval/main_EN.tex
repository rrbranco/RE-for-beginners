\subsection{scanf()}

As was noted before, it is slightly old-fashioned to use \scanf today. 
But if we have to, we have to check if \scanf finishes correctly without an error.

\lstinputlisting[style=customc]{patterns/04_scanf/3_checking_retval/ex3.c}

By standard, the \scanf\footnote{scanf, wscanf: \href{http://go.yurichev.com/17255}{MSDN}} function returns the number of fields it has successfully read.

In our case, if everything goes fine and the user enters a number \scanf returns 1, or in case of error (or \ac{EOF}) --- 0.

Let's add some C code to check the \scanf return value and print error message in case of an error.

This works as expected:

\begin{lstlisting}
C:\...>ex3.exe
Enter X:
123
You entered 123...

C:\...>ex3.exe
Enter X:
ouch
What you entered? Huh?
\end{lstlisting}

% subsections
\EN{\subsubsection{MSVC: x86}

Here is what we get in the assembly output (MSVC 2010):

\lstinputlisting[style=customasmx86]{patterns/04_scanf/3_checking_retval/ex3_MSVC_x86.asm}

\myindex{x86!\Registers!EAX}
The \gls{caller} function (\main) needs the \gls{callee} function (\scanf) result, 
so the \gls{callee} returns it in the \EAX register.

\myindex{x86!\Instructions!CMP}
We check it with the help of the instruction \TT{CMP EAX, 1} (\IT{CoMPare}). In other words, we compare the value in the \EAX register with 1.

\myindex{x86!\Instructions!JNE}
A \JNE conditional jump follows the \CMP instruction. \JNE stands for \IT{Jump if Not Equal}.

So, if the value in the \EAX register is not equal to 1, the \ac{CPU} will pass the execution to the address mentioned in the \JNE operand, in our case \TT{\$LN2@main}.
Passing the control to this address results in the \ac{CPU} executing \printf with the argument \TT{What you entered? Huh?}.
But if everything is fine, the conditional jump is not be taken, and another \printf call is to be executed, with two arguments:\\
\TT{'You entered \%d...'} and the value of \TT{x}.

\myindex{x86!\Instructions!XOR}
\myindex{\CLanguageElements!return}
Since in this case the second \printf has not to be executed, there is a \JMP preceding it (unconditional jump). 
It passes the control to the point after the second \printf and just before the \TT{XOR EAX, EAX} instruction, which implements \TT{return 0}.

% FIXME internal \ref{} to x86 flags instead of wikipedia
\myindex{x86!\Registers!\Flags}
So, it could be said that comparing a value with another is \IT{usually} implemented by \CMP/\Jcc instruction pair, where \IT{cc} is \IT{condition code}.
\CMP compares two values and sets processor flags\footnote{x86 flags, see also: \href{http://go.yurichev.com/17120}{wikipedia}.}.
\Jcc checks those flags and decides to either pass the control to the specified address or not.

\myindex{x86!\Instructions!CMP}
\myindex{x86!\Instructions!SUB}
\myindex{x86!\Instructions!JNE}
\myindex{x86!\Registers!ZF}
\label{CMPandSUB}
This could sound paradoxical, but the \CMP instruction is in fact \SUB (subtract).
All arithmetic instructions set processor flags, not just \CMP.
If we compare 1 and 1, $1-1$ is 0 so the \ZF flag would be set (meaning that the last result is 0).
In no other circumstances \ZF can be set, except when the operands are equal.
\JNE checks only the \ZF flag and jumps only if it is not set.  \JNE is in fact a synonym for \JNZ (\IT{Jump if Not Zero}).
Assembler translates both \JNE and \JNZ instructions into the same opcode.
So, the \CMP instruction can be replaced with a \SUB instruction and almost everything will be fine, with the difference that \SUB alters the value of the first operand.
\CMP is \IT{SUB without saving the result, but affecting flags}.

\subsubsection{MSVC: x86: IDA}

\myindex{IDA}
It is time to run \IDA and try to do something in it.
By the way, for beginners it is good idea to use \TT{/MD} option in MSVC, which means that all these
standard functions are not be linked with the executable file, 
but are to be imported from the \TT{MSVCR*.DLL} file instead.
Thus it will be easier to see which standard function are used and where.

While analyzing code in \IDA, it is very helpful to leave notes for oneself (and others).
In instance, analyzing this example, 
we see that \TT{JNZ} is to be triggered in case of an error.
So it is possible to move the cursor to the label, press \q{n} and rename it to \q{error}.
Create another label---into \q{exit}.
Here is my result:

\lstinputlisting[style=customasmx86]{patterns/04_scanf/3_checking_retval/ex3.lst}

Now it is slightly easier to understand the code.
However, it is not a good idea to comment on every instruction.

% FIXME draw button?
You could also hide(collapse) parts of a function in \IDA.
To do that mark the block, then press \q{--} on the numerical pad and enter the text to be displayed instead.

Let's hide two blocks and give them names:

\lstinputlisting[style=customasmx86]{patterns/04_scanf/3_checking_retval/ex3_2.lst}

% FIXME draw button?
To expand previously collapsed parts of the code, use \q{+} on the numerical pad.

\clearpage
By pressing \q{space}, we can see how \IDA represents a function as a graph:

\begin{figure}[H]
\centering
\myincludegraphics{patterns/04_scanf/3_checking_retval/IDA.png}
\caption{Graph mode in IDA}
\label{fig:ex3_IDA_1}
\end{figure}

There are two arrows after each conditional jump: green and red.
The green arrow points to the block which executes if the jump is triggered, and red if otherwise.

\clearpage
It is possible to fold nodes in this mode and give them names as well (\q{group nodes}).
Let's do it for 3 blocks:

\begin{figure}[H]
\centering
\myincludegraphics{patterns/04_scanf/3_checking_retval/IDA2.png}
\caption{Graph mode in IDA with 3 nodes folded}
\label{fig:ex3_IDA_2}
\end{figure}

That is very useful.
It could be said that a very important part of the reverse engineers' job (and any other researcher as well) is to reduce the amount of information they deal with.

\clearpage
\subsubsection{MSVC: x86 + \olly}

Let's try to hack our program in \olly, forcing it to think \scanf always works without error.
When an address of a local variable is passed into \scanf,
the variable initially contains some random garbage, in this case \TT{0x6E494714}:

\begin{figure}[H]
\centering
\myincludegraphics{patterns/04_scanf/3_checking_retval/olly_1.png}
\caption{\olly: passing variable address into \scanf}
\label{fig:scanf_ex3_olly_1}
\end{figure}

\clearpage
While \scanf executes, in the console we enter something that is definitely not a number, like \q{asdasd}.
\scanf finishes with 0 in \EAX, which indicates that an error has occurred:

\begin{figure}[H]
\centering
\myincludegraphics{patterns/04_scanf/3_checking_retval/olly_2.png}
\caption{\olly: \scanf returning error}
\label{fig:scanf_ex3_olly_2}
\end{figure}

We can also check the local variable in the stack and note that it has not changed.
Indeed, what would \scanf write there?
It simply did nothing except returning zero.

Let's try to \q{hack} our program.
Right-click on \EAX, 
Among the options there is \q{Set to 1}.
This is what we need.

We now have 1 in \EAX, so the following check is to be executed as intended, 
and \printf will print the value of the variable in the stack.

When we run the program (F9) we can see the following in the console window:

\lstinputlisting[caption=console window]{patterns/04_scanf/3_checking_retval/console.txt}

Indeed, 1850296084 is a decimal representation of the number in the stack (\TT{0x6E494714})!


\clearpage
\subsubsection{MSVC: x86 + Hiew}
\myindex{Hiew}

This can also be used as a simple example of executable file patching.
We may try to patch the executable so the program would always print the input, no matter what we enter.

Assuming that the executable is compiled against external \TT{MSVCR*.DLL} (i.e., with \TT{/MD} option)
\footnote{that's what also called \q{dynamic linking}}, 
we see the \main function at the beginning of the \TT{.text} section.
Let's open the executable in Hiew and find the beginning of the \TT{.text} section (Enter, F8, F6, Enter, Enter).

We can see this:

\begin{figure}[H]
\centering
\myincludegraphics{patterns/04_scanf/3_checking_retval/hiew_1.png}
\caption{Hiew: \main function}
\label{fig:scanf_ex3_hiew_1}
\end{figure}

Hiew finds \ac{ASCIIZ} strings and displays them, as it does with the imported functions' names.

\clearpage
Move the cursor to address \TT{.00401027} (where the \TT{JNZ} instruction, we have to bypass, is located), press F3, and then type \q{9090} (meaning two \ac{NOP}s):

\begin{figure}[H]
\centering
\myincludegraphics{patterns/04_scanf/3_checking_retval/hiew_2.png}
\caption{Hiew: replacing \TT{JNZ} with two \ac{NOP}s}
\label{fig:scanf_ex3_hiew_2}
\end{figure}

Then press F9 (update). Now the executable is saved to the disk. It will behave as we wanted.

Two \ac{NOP}s are probably not the most \ae{}sthetic approach.
Another way to patch this instruction is to write just 0 to the second opcode byte (\gls{jump offset}), 
so that \TT{JNZ} will always jump to the next instruction.

We could also do the opposite: replace first byte with \TT{EB} while not touching the second byte (\gls{jump offset}).
We would get an unconditional jump that is always triggered.
In this case the error message would be printed every time, no matter the input.

}
\RU{\subsubsection{MSVC: x86}

Вот что выходит на ассемблере (MSVC 2010):

\lstinputlisting[style=customasmx86]{patterns/04_scanf/3_checking_retval/ex3_MSVC_x86.asm}

\myindex{x86!\Registers!EAX}
Для того чтобы вызывающая функция имела доступ к результату вызываемой функции, 
вызываемая функция (в нашем случае \scanf) оставляет это значение в регистре \EAX.

\myindex{x86!\Instructions!CMP}
Мы проверяем его инструкцией \TT{CMP EAX, 1} (\IT{CoMPare}), то есть сравниваем значение в \EAX с 1.

\myindex{x86!\Instructions!JNE}
Следующий за инструкцией \CMP: условный переход \JNE. Это означает \IT{Jump if Not Equal}, то есть условный переход \IT{если не равно}.

Итак, если \EAX не равен 1, то \JNE заставит \ac{CPU} перейти по адресу указанном в операнде \JNE, у нас это \TT{\$LN2@main}.
Передав управление по этому адресу, \ac{CPU} начнет исполнять вызов \printf с аргументом \TT{What you entered? Huh?}.
Но если всё нормально, перехода не случится и исполнится другой \printf с двумя аргументами:\\
\TT{'You entered \%d...'} и значением переменной \TT{x}.

\myindex{x86!\Instructions!XOR}
\myindex{\CLanguageElements!return}
Для того чтобы после этого вызова не исполнился сразу второй вызов \printf, 
после него есть инструкция \JMP, безусловный переход, который отправит процессор на место 
после второго \printf и перед инструкцией \TT{XOR EAX, EAX}, которая реализует \TT{return 0}.

% FIXME internal \ref{} to x86 flags instead of wikipedia
\myindex{x86!\Registers!\Flags}
Итак, можно сказать, что в подавляющих случаях сравнение какой-либо переменной с чем-то другим происходит при помощи пары инструкций \CMP и \Jcc, где \IT{cc} это \IT{condition code}.
\CMP сравнивает два значения и выставляет  флаги процессора\footnote{См. также о флагах x86-процессора: \href{http://go.yurichev.com/17120}{wikipedia}.}.
\Jcc проверяет нужные ему флаги и выполняет переход по указанному адресу (или не выполняет).

\myindex{x86!\Instructions!CMP}
\myindex{x86!\Instructions!SUB}
\myindex{x86!\Instructions!JNE}
\myindex{x86!\Registers!ZF}
\label{CMPandSUB}
Но на самом деле, как это не парадоксально поначалу звучит, \CMP это почти то же самое что и инструкция \SUB, которая отнимает числа одно от другого.
Все арифметические инструкции также выставляют флаги в соответствии с результатом, не только \CMP.
Если мы сравним 1 и 1, от единицы отнимется единица, получится 0, и выставится флаг \ZF (\IT{zero flag}), означающий, что последний полученный результат был 0.
Ни при каких других значениях \EAX, флаг \ZF не может быть выставлен, кроме тех, когда операнды равны друг другу.
Инструкция \JNE проверяет только флаг \ZF, и совершает переход только если флаг не поднят. Фактически, \JNE это синоним инструкции \JNZ (\IT{Jump if Not Zero}).
Ассемблер транслирует обе инструкции в один и тот же опкод.
Таким образом, можно \CMP заменить на \SUB и всё будет работать также, но разница в том, что \SUB всё-таки испортит значение в первом операнде.
\CMP это \IT{SUB без сохранения результата, но изменяющая флаги}.

\subsubsection{MSVC: x86: IDA}

\myindex{IDA}
Наверное, уже пора делать первые попытки анализа кода в \IDA.
Кстати, начинающим полезно компилировать в MSVC с ключом \TT{/MD}, что означает, что все эти стандартные
функции не будут скомпонованы с исполняемым файлом, а будут импортироваться из файла \TT{MSVCR*.DLL}.
Так будет легче увидеть, где какая стандартная функция используется.

Анализируя код в \IDA, очень полезно делать пометки для себя (и других).
Например, разбирая этот пример, мы сразу видим, что \TT{JNZ} срабатывает в случае ошибки.
Можно навести курсор на эту метку, нажать \q{n} и переименовать метку в \q{error}.
Ещё одну метку --- в \q{exit}.
Вот как у меня получилось в итоге:

\lstinputlisting[style=customasmx86]{patterns/04_scanf/3_checking_retval/ex3.lst}

Так понимать код становится чуть легче.
Впрочем, меру нужно знать во всем и комментировать каждую инструкцию не стоит.

% FIXME draw button?
В \IDA также можно скрывать части функций: нужно выделить скрываемую часть, нажать \q{--} на цифровой клавиатуре и ввести текст.

Скроем две части и придумаем им названия:

\lstinputlisting[style=customasmx86]{patterns/04_scanf/3_checking_retval/ex3_2.lst}

% FIXME draw button?
Раскрывать скрытые части функций можно при помощи \q{+} на цифровой клавиатуре.

\clearpage
Нажав \q{пробел}, мы увидим, как \IDA может представить функцию в виде графа:

\begin{figure}[H]
\centering
\myincludegraphics{patterns/04_scanf/3_checking_retval/IDA.png}
\caption{Отображение функции в IDA в виде графа}
\label{fig:ex3_IDA_1}
\end{figure}

После каждого условного перехода видны две стрелки: зеленая и красная.
Зеленая ведет к тому блоку, который исполнится если переход сработает, 
а красная~--- если не сработает.

\clearpage
В этом режиме также можно сворачивать узлы и давать им названия (\q{group nodes}).
Сделаем это для трех блоков:

\begin{figure}[H]
\centering
\myincludegraphics{patterns/04_scanf/3_checking_retval/IDA2.png}
\caption{Отображение в IDA в виде графа с тремя свернутыми блоками}
\label{fig:ex3_IDA_2}
\end{figure}

Всё это очень полезно делать.
Вообще, очень важная часть работы реверсера (да и любого исследователя) состоит в том, чтобы уменьшать количество имеющейся информации.

\clearpage
\subsubsection{MSVC: x86 + \olly}

Попробуем в \olly немного хакнуть программу и сделать вид, что \scanf срабатывает всегда без ошибок.
Когда в \scanf передается адрес локальной переменной, изначально в этой переменной
находится некий мусор. В данном случае это \TT{0x6E494714}:

\begin{figure}[H]
\centering
\myincludegraphics{patterns/04_scanf/3_checking_retval/olly_1.png}
\caption{\olly: передача адреса переменной в \scanf}
\label{fig:scanf_ex3_olly_1}
\end{figure}

\clearpage
Когда \scanf запускается, вводим в консоли что-то непохожее на число, например \q{asdasd}.
\scanf заканчивается с 0 в \EAX, что означает, что произошла ошибка:

\begin{figure}[H]
\centering
\myincludegraphics{patterns/04_scanf/3_checking_retval/olly_2.png}
\caption{\olly: \scanf закончился с ошибкой}
\label{fig:scanf_ex3_olly_2}
\end{figure}

Вместе с этим мы можем посмотреть на локальную переменную в стеке~--- она не изменилась.
Действительно, ведь что туда записала бы функция \scanf?
Она не делала ничего кроме возвращения нуля.
Попробуем ещё немного \q{хакнуть} нашу программу.
Щелкнем правой кнопкой на \EAX, там, в числе опций, будет также \q{Set to 1}.
Это нам и нужно.

В \EAX теперь 1, последующая проверка пройдет как надо, и \printf выведет значение переменной из стека.

Запускаем (F9) и видим в консоли следующее:

\lstinputlisting[caption=консоль]{patterns/04_scanf/3_checking_retval/console.txt}

Действительно, 1850296084 это десятичное представление числа в стеке (\TT{0x6E494714})!


\clearpage
\subsubsection{MSVC: x86 + Hiew}
\myindex{Hiew}

Это ещё может быть и простым примером исправления исполняемого файла.
Мы можем попробовать исправить его таким образом, что программа всегда будет выводить числа, вне зависимости от ввода.

Исполняемый файл скомпилирован с импортированием функций из
\TT{MSVCR*.DLL} (т.е. с опцией \TT{/MD})\footnote{то, что ещё называют \q{dynamic linking}}, 
поэтому мы можем отыскать функцию \main в самом начале секции \TT{.text}.
Откроем исполняемый файл в Hiew, найдем самое начало секции \TT{.text} (Enter, F8, F6, Enter, Enter).

Мы увидим следующее:

\begin{figure}[H]
\centering
\myincludegraphics{patterns/04_scanf/3_checking_retval/hiew_1.png}
\caption{Hiew: функция \main}
\label{fig:scanf_ex3_hiew_1}
\end{figure}

Hiew находит \ac{ASCIIZ}-строки и показывает их, также как и имена импортируемых функций.

\clearpage
Переведите курсор на адрес \TT{.00401027} (с инструкцией \TT{JNZ}, которую мы хотим заблокировать), нажмите F3, затем наберите \q{9090} (что означает два \ac{NOP}-а):

\begin{figure}[H]
\centering
\myincludegraphics{patterns/04_scanf/3_checking_retval/hiew_2.png}
\caption{Hiew: замена \TT{JNZ} на два \ac{NOP}-а}
\label{fig:scanf_ex3_hiew_2}
\end{figure}

Затем F9 (update). Теперь исполняемый файл записан на диск. Он будет вести себя так, как нам надо.

Два \ac{NOP}-а, возможно, не так эстетично, как могло бы быть.
Другой способ изменить инструкцию это записать 0 во второй байт опкода (смещение перехода),
так что \TT{JNZ} всегда будет переходить на следующую инструкцию.

Можно изменить и наоборот: первый байт заменить на \TT{EB}, второй байт (смещение перехода) не трогать.
Получится всегда срабатывающий безусловный переход.
Теперь сообщение об ошибке будет выдаваться всегда, даже если мы ввели число.

}
\PTBR{\subsubsection{MSVC: x86}

Aqui está o a saída em assembly (MSVC 2010):

\lstinputlisting[style=customasmx86]{patterns/04_scanf/3_checking_retval/ex3_MSVC_x86.asm}

\myindex{x86!\Registers!EAX}
A função que chamou (\main) precisa do resultado da função chamada (\scanf),
então a função chamada retorna esse valor no registrador \EAX.

\myindex{x86!\Instructions!CMP}
Nós verificamos com a ajuda da instrução \TT{CMP EAX, 1} (\IT{CoMParar}). Em outras palavras, comparamos o valor em \EAX com 1.

\myindex{x86!\Instructions!JNE}
O jump condicional \JNE está logo depois da instrução \CMP. \JNE significa \IT{Jump if Not Equal} ou seja, ela desvia se o valor não for igual ao comparado.

Então, se o valor em \EAX não é 1, a \ac{CPU} vai passar a execução para o endereço contido no operando de \JNE, no nosso caso \TT{\$LN2@main}.
Passando a execução para esse endereço resulta na \ac{CPU} executando \printf com o argumento \TT{What you entered? Huh?}.
Mas se tudo estiver correto, o jump condicional não será efetuado e outra chamada do \printf é executada, com dois argumentos: \TT{`You entered \%d...'} e o valor de \TT{x}.

\myindex{x86!\Instructions!XOR}
\myindex{\CLanguageElements!return}
Como nesse caso o segundo \printf() não tem que ser executado, tem um \JMP precedendo ele (jump incondicional).
Ele passa a execução para o ponto depois do segundo \printf e logo antes de \TT{XOR EAX, EAX}, que implementa \TT{return 0}.

% FIXME internal \ref{} to x86 flags instead of wikipedia
\myindex{x86!\Registers!\Flags}
Então, podemos dizer que comparar um valor com outro é geralmente realizado através do par de instruções \CMP/\Jcc, onde \IT{cc} é código condicional.
\CMP compara dois valores e altera os registros da \ac{CPU} (flags)\footnote{\ac{TBT}: x86 flags, see also: \href{http://go.yurichev.com/17120}{wikipedia}.}.
\Jcc checa esses registro e decide passar a execução para o endereço específico contido no operando ou não.

\myindex{x86!\Instructions!CMP}
\myindex{x86!\Instructions!SUB}
\myindex{x86!\Instructions!JNE}
\myindex{x86!\Registers!ZF}
\label{CMPandSUB}
Isso pode parecer meio paradoxal, mas a instrução \CMP é na verdade \SUB (subtrair).
Todo o conjunto de instruções aritiméticas alteram os registros da \ac{CPU}, não só \CMP.
Se compararmos 1 e 1, $1-1$ é 0 então \ZF (zero flag) será acionado (significando que o último resultado foi zero).
Em nenhuma outra circunstância \ZF pode ser acionado, exceto quando os operandos forem iguais.
\JNE verifica somente o ZF e desvia só não estiver acionado.
\JNE é na verdade um sinônimo para \JNZ (jump se não zero).
\JNE e \JNZ são traduzidos no mesmo código de operação.
Então, a instrução CMP pode ser substituida com a instrução \SUB e quase tudo estará certo, com a diferença de que \SUB altera o valor do primeiro operando.
\CMP é \SUB sem salvar o resultado, mas afetando os registros da \ac{CPU}.

\subsubsection{MSVC: x86: IDA}

\PTBRph{}

% TODO translate: \clearpage
\subsubsection{MSVC: x86 + \olly}

Vamos hackear nosso programa no \olly, forçando-o a acreditar que o \scanf sempre funciona sem erro.

Quando o endereço de uma variável local é passado para o \scanf, a variável geralmente contém algum lixo de memória, neste caso \TT{0x6E494714}:

\begin{figure}[H]
\centering
\myincludegraphics{patterns/04_scanf/3_checking_retval/olly_1.png}
\caption{\olly: Passando o endereço da variável pro \scanf}
\label{fig:scanf_ex3_olly_1}
\end{figure}

\clearpage
Enquanto \scanf o está em execução, ja janela do console entre com um valor que não é um número, por exemplo \q{asdasd}.

\scanf termina com 0 no \EAX,  o que indica que houve um erro:

\begin{figure}[H]
\centering
\myincludegraphics{patterns/04_scanf/3_checking_retval/olly_2.png}
\caption{\olly: \scanf retornando erro}
\label{fig:scanf_ex3_olly_2}
\end{figure}

Também podemos checar o valor da variável local na pilha e perceba que o valor não mudou.
O que levaria o \scanf escrever lá?
Simplesmente não fez nada, exceto retornar zero.

Vamos tentar \q{hack} nosso programa.
Clique com o botão direito do mouse em \EAX, entre outras opções selecione \q{Set to 1}. É o que precisamos.

Agora temos 1 em \EAX, a próxima e o \printf irá imprimir o valor da variável na pilha.

Quando executamos o programa (F9) podemos ver o seguinte no console:

\lstinputlisting[caption=console window]{patterns/04_scanf/3_checking_retval/console.txt}

De fato, 1850296084 é a representação decimal do número na pilha (\TT{0x6E494714})!

\clearpage
\subsubsection{MSVC: x86 + Hiew}
\myindex{Hiew}

Esse exemplo também pode ser usado como uma maneira simples de exemplificar o patch de arquivos executáveis.
Nós podemos tentar rearranjar o executável de forma que o programa sempre imprima a saída, não importando o que inserirmos.

Assumindo que o executavel está compilado com a opção \TT{/MD}\footnote{isso também é chamada ``linkagem dinâmica''}
(\TT{MSVCR*.DLL}), nós vemos a função main no começo da seção \TT{.text}.
Vamos abrir o executável no Hiew e procurar o começo da seção \TT{.text} (Enter, F8, F6, Enter, Enter).

Nós chegamos a isso:

\begin{figure}[H]
\centering
\myincludegraphics{patterns/04_scanf/3_checking_retval/hiew_1.png}
\caption{\PTBRph{}}
\label{fig:scanf_ex3_hiew_1}
\end{figure}

Hiew encontra strings em \ac{ASCIIZ} e as exibe, como faz com os nomes de funções importadas.

\clearpage
Mova o cursor para o endereço \TT{.00401027} (onde a instrução \TT{JNZ}, que temos de evitar, está localizada), aperte F3 e então digite \q{9090} (que significa dois \ac{NOP}s):

\begin{figure}[H]
\centering
\myincludegraphics{patterns/04_scanf/3_checking_retval/hiew_2.png}
\caption{PTBRph{}}
\label{fig:scanf_ex3_hiew_2}
\end{figure}

Então aperte F9 (atualizar). Agora o executável está salvo no disco. Ele executará da maneira que nós desejávamos.

Duas instruções \ac{NOP} não é a abordagem mais estética.
Outra maneira de rearranjar essa instrução é somente escrever um 0 no operando da instrução jump,
então \INS{JNZ} só avançará para a próxima instrução.

Nós poderíamos também ter feito o oposto: mudado o primeiro byte com \TT{EB} e deixa o segundo byte como está.
Nós teriamos um jump incondicional que é sempre deviado.
Nesse caso, a mensagem de erro seria mostrada todas as vezes, não importando a entrada.

}
\ITA{\subsubsection{MSVC: x86}

Il seguente e' l'output assembly ottenuto con MSVC 2010:

\lstinputlisting[style=customasmx86]{patterns/04_scanf/3_checking_retval/ex3_MSVC_x86.asm}

\myindex{x86!\Registers!EAX}
La funzione chiamante (\gls{caller}) \main necessita di ottenere il risultato della funzione chiamata (\gls{callee}), 
e pertanto quest'ultima lo restituisce nel registro \EAX register.

\myindex{x86!\Instructions!CMP}
Il controllo viene eseguito con l'aiuto dell'istruzione \TT{CMP EAX, 1} (\IT{CoMPare}). In altre parole, confrontiamo il valore nel registro \EAX con 1.

\myindex{x86!\Instructions!JNE}
U jump condizionale \JNE segue l'istruzione \CMP. \JNE sta per \IT{Jump if Not Equal}.

Quindi, se il valore nel registro \EAX non e' uguale a 1, la \ac{CPU} passera' l'esecuzione all'indirizzo specificato nell'operando di \JNE, nel nostro caso \TT{\$LN2@main}.
Passare il controllo a questo indirizzo risulta nel fatto che la \ac{CPU} eseguira' la funzione \printf con l'argomento \TT{What you entered? Huh?}.
Ma se tutto va bene, il salto condizionale non viene effettuato, e viene eseguita un'altra chiamata a \printf con due argomenti: \TT{'You entered \%d...'} e il valore di \TT{x}.

\myindex{x86!\Instructions!XOR}
\myindex{\CLanguageElements!return}
Poiche' in questo caso la saconda \printf non deve essere eseguita, c'e' un jump non condizionale (unconditional jump) \JMP che la precede. 
Questo passa il controllo al punto dopo la seconda \printf e prima dell'istruzione \TT{XOR EAX, EAX}, che implementa \TT{return 0}.

% FIXME internal \ref{} to x86 flags instead of wikipedia
\myindex{x86!\Registers!\Flags}
Possiamo quindi dire che il confronto di valori e' \IT{solitamente} implementato con una coppia di istruzioni \CMP/\Jcc, dove \IT{cc} e' un \IT{condition code}.
\CMP confronta due valori e imposta i flag del processore \footnote{x86 flags, vedere anche: \href{http://go.yurichev.com/17120}{wikipedia}.}.
\Jcc controlla questi flag e decide se passare o meno il controllo all'indirizzo specificato.

\myindex{x86!\Instructions!CMP}
\myindex{x86!\Instructions!SUB}
\myindex{x86!\Instructions!JNE}
\myindex{x86!\Registers!ZF}
\label{CMPandSUB}
Puo' sembrare un paradosso, ma l'istruzione \CMP e' in effetti una \SUB (subtract).
Tutte le istruzioni aritmetiche settano i flag del processore, non solo \CMP.
Se confrontiamo 1 e 1, $1-1$ e' 0 e quindi il flag \ZF sarebbe impostato a 1 (significando che l'ultimo risultato era 0).
In nessun'altra circostanza il flag \ZF puo' essere impostato, eccetto il caso in cui gli operandi sono uguali.
\JNE controlla soltanto il flag \ZF e salta se e solo se il flag non e' settato.  \JNE e' infatti un sinonimo di \JNZ (\IT{Jump if Not Zero}).
L'assembler traduce entrambe le istruzioni \JNE e \JNZ nello stesso opcode.
Quindi l'istruzione \CMP puo' essere sostituita dall'istruzione \SUB e quasi tutto funzionera', con la differenza che \SUB altera il valore del primo operando.
\CMP e' uguale a \IT{SUB senza salvare il risultato, ma settando i flag}.

\subsubsection{MSVC: x86: IDA}

\myindex{IDA}
E' arrivato il momento di avviare \IDA. A proposito, per i principianti e' buona norma usare l'opzione \TT{/MD} in MSVC, che significa che tutte le funzioni
standard non saranno linkate dentro il file eseguibile, ma importate dal file \TT{MSVCR*.DLL}.
In questo modo sara' piu' facile vedere quali funzioni standard sono usate, e dove.

Quando si analizza il codice con \IDA, e' sempre molto utile lasciare note per se stessi (e per gli altri, nel caso in cui si lavori in gruppo).
Per esempio, analizzando questo esempio, notiamo che 
\TT{JNZ} sara' innescato in caso di errore.
E' possibile muovere il cursore fino alla label, premere \q{n} e rinominarla in \q{errore}.
Creare un'altra label ---in \q{exit}.
Ecco il mio risultato:

\lstinputlisting[style=customasmx86]{patterns/04_scanf/3_checking_retval/ex3.lst}

Adesso e' leggermente piu' facile capire il codice.
Non e' comunque una buona idea commentare ogni istruzione!

% FIXME draw button?
Si possono anche nascondere (collapse) parti di una funzione in \IDA.
Per farlo, selezionare il blocco e premere \q{--} sul tastierino numerico, inserendo il testo da visualizzare al posto del blocco di codice.

Nascondiamo due blocchi e diamogli un nome:

\lstinputlisting[style=customasmx86]{patterns/04_scanf/3_checking_retval/ex3_2.lst}

% FIXME draw button?
Per espandere dei blocchi nascosti, premere \q{+} sul tastierino numerico.

\clearpage
Premendo \q{spazio}, possiamo vedere come \IDA rappresenta una funzione in forma di grafo:

\begin{figure}[H]
\centering
\myincludegraphics{patterns/04_scanf/3_checking_retval/IDA.png}
\caption{Graph mode in IDA}
\label{fig:ex3_IDA_1}
\end{figure}

Ci sono due frecce dopo ogni jump condizionale: verde e rossa.
La freccia verde punta al blocco che viene eseguito se il jump e' innescato, la rossa nel caso opposto.

\clearpage
Anche in questa modalita' e' possibile "chiudere" i nodi e dargli un'etichetta (\q{group nodes}).
Facciamolo per 3 blocchi:

\begin{figure}[H]
\centering
\myincludegraphics{patterns/04_scanf/3_checking_retval/IDA2.png}
\caption{Graph mode in IDA with 3 nodes folded}
\label{fig:ex3_IDA_2}
\end{figure}

Come si puo' vedere questa funzione e' molto utile.
Si puo' dire che una buona parte del lavoro di un reverse engineer (cosi' come di altri tipi di ricercatori) e' rappresentata dalla riduzione della quantita' di informazioni da trattare.

\clearpage
\subsubsection{MSVC: x86 + \olly}

Proviamo ad hackerare il nostro programma in \olly, forzandolo a pensare che \scanf funzioni sempre senza errori.
Quando l'indirizzo di una variabile locale e' passato a \scanf, la variabile inizialmente contiene un valore random inutile, in questo caso \TT{0x6E494714}:

\begin{figure}[H]
\centering
\myincludegraphics{patterns/04_scanf/3_checking_retval/olly_1.png}
\caption{\olly: passing variable address into \scanf}
\label{fig:scanf_ex3_olly_1}
\end{figure}

\clearpage
Quando \scanf viene eseguita, immettiamo nella console qualcosa di diverso da un numero, come \q{asdasd}.
\scanf finisce con 0 in \EAX, indicante che un errore si e' verificato:

\begin{figure}[H]
\centering
\myincludegraphics{patterns/04_scanf/3_checking_retval/olly_2.png}
\caption{\olly: \scanf returning error}
\label{fig:scanf_ex3_olly_2}
\end{figure}

Possiamo anche controllare la variabile locale nello stack e notare che non e' stata modificata.
Infatti cosa avrebbe potuto scrivere \scanf in essa? Non ha fatto niente oltre che restituire zero. 


Proviamo ad \q{hackerare} il nostro programma.
Click destro su \EAX, 
Tra le opzioni vediamo \q{Set to 1}.
Esattamente cio' che ci serve.

Adesso abbiamo 1 in \EAX, il controllo successivo sta per essere eseguito come previsto,
e \printf stampera' il valore della variabile nello stack.

Quando avviamo il programma (F9) vediamo il seguente output nella finestra della console:

\lstinputlisting[caption=console window]{patterns/04_scanf/3_checking_retval/console.txt}

1850296084 e' infatti la rappresentazione decimale del numero nello stack (\TT{0x6E494714})!


\clearpage
\subsubsection{MSVC: x86 + Hiew}
\myindex{Hiew}

Quanto detto puo' essere anche usato come semplice esempio di patching di un eseguibile.
Possiamo provare a modificare l'eseguibile in modo che il programma stampi sempre l'input, a prescindere da cosa si inserisce.

Assumendo che l'eseguibile sia compilato rispetto \TT{MSVCR*.DLL} esterna (ovvero con l'opzione \TT{/MD})
\footnote{detta anche \q{dynamic linking}}, 
vediamo la funzione \main all'inizio della sezione \TT{.text}.
Apriamo l'eseguibile con Hiew e troviamo l'inizio della sezione \TT{.text} (Enter, F8, F6, Enter, Enter).

Vedremo questo:

\begin{figure}[H]
\centering
\myincludegraphics{patterns/04_scanf/3_checking_retval/hiew_1.png}
\caption{Hiew: \main function}
\label{fig:scanf_ex3_hiew_1}
\end{figure}

Hiew trova le stringhe \ac{ASCIIZ} e le visualizza, cosi' come i nomi delle funzioni importate.

\clearpage
Spostiamo il cursore all'indirizzo \TT{.00401027} (dove si trova l'istruzione \TT{JNZ} che vogliamo bypassare), premiamo F3, e scriviamo \q{9090} (cioe' due \ac{NOP}):

\begin{figure}[H]
\centering
\myincludegraphics{patterns/04_scanf/3_checking_retval/hiew_2.png}
\caption{Hiew: replacing \TT{JNZ} with two \ac{NOP}s}
\label{fig:scanf_ex3_hiew_2}
\end{figure}

Premiamo quindi F9 (update). L'eseguibile viene quindi salvato su disco, e si comportera' come vogliamo.

Utilizzare due \ac{NOP} non rappresenta l'approccio esteticamente migliore.
Un altro modo di patchare questa istruzione e' scrivere 0 al secondo byte dell'opcode (\gls{jump offset}), 
in modo che \TT{JNZ} salti sempre alla prossima istruzione.

Potremmo anche fare l'opposto: sostituire il primo byte con \TT{EB} senza toccare il secondo byte (\gls{jump offset}).
Otterremmo un jump non condizionale che e' sempre eseguito.
In questo caso il messaggio di errore sarebbe stampato sempre, a prescindere dall'input.

}
\FR{\subsubsection{MSVC: x86}

Voici ce que nous obtenons dans la sortie assembleur (MSVC 2010):

\lstinputlisting[style=customasmx86]{patterns/04_scanf/3_checking_retval/ex3_MSVC_x86.asm}

\myindex{x86!\Registers!EAX}
La fonction \glslink{caller}{appelante} (\main) à besoin du résultat de la fonction
\glslink{callee}{appelée}, donc la fonction \glslink{callee}{appelée} le renvoie
dans la registre \EAX.

\myindex{x86!\Instructions!CMP}
Nous le vérifions avec l'aide de l'instruction \TT{CMP EAX, 1} (\IT{CoMPare}).
En d'autres mots, nous comparons la valeur dans le registre \EAX avec 1.

\myindex{x86!\Instructions!JNE}
Une instruction de saut conditonnelle \JNE suit l'instruction \CMP. \JNE signifie
\IT{Jump if Not Equal} (saut si non égal).

Donc, si la valeur dans le registre \EAX n'est pas égale à 1, le \ac{CPU} va poursuivre
l'exécution à l'adresse mentionnée dans l'opérande \JNE, dans notre cas \TT{\$LN2@main}.
Passez le contrôle à cette adresse résulte en l'exécution par le \ac{CPU} de
\printf avec l'argument \TT{What you entered? Huh?}.
Mais si tout est bon, le saut conditionnel n'est pas pris, et un autre appel à \printf
est exécuté, avec deux arguments:\\
\TT{'You entered \%d...'} et la valeur de \TT{x}.

\myindex{x86!\Instructions!XOR}
\myindex{\CLanguageElements!return}
Puisque dans ce cas le second \printf n'a pas été exécuté, il y a un \JMP qui le précède (saut inconditionnel).
Il passe le contrôle au point après le second \printf et juste avant l'instruction \TT{XOR EAX, EAX}, qui implémente \TT{return 0}.

% FIXME internal \ref{} to x86 flags instead of wikipedia
\myindex{x86!\Registers!\Flags}
Donc, on peut dire que comparer une valeur avec une autre est \IT{usuellement} implémenté
par la paire d'instructions \CMP/\Jcc, oú \IT{cc} est un \IT{code de condition}.
\CMP compare deux valeurs et met les flags\footnote{flags x86, voir aussi: \href{http://go.yurichev.com/17120}{wikipedia}.}
du processeur.
\Jcc vérifie ces flags et décide de passer le contrôle à l'adresse spécifiée ou non.

\myindex{x86!\Instructions!CMP}
\myindex{x86!\Instructions!SUB}
\myindex{x86!\Instructions!JNE}
\myindex{x86!\Registers!ZF}
\label{CMPandSUB} 
Cela peut sembler paradoxal, mais l'instruction \CMP est en fait un \SUB (soustraction).
Toutes les instructions arithmétiques mettent les flags du processeur, pas seulement \CMP.
Si nous comparons 1 et 1, $1-1$ donne 0 donc le flag \ZF va être mis (signifiant
que le dernier résultat est 0).
Dans aucune autre circonstance \ZF ne sera mis, à l'exception que les opérandes
ne soient égales.
\JNE vérifie seulement le flag \ZF et saute seulement si il n'est pas mis. \JNE
est un synonyme pour \JNZ (\IT{Jump if Not Zero} (saut si non zéro)).
L'assembleur génère le même opcode pour les instructions \JNE et \JNZ.
Donc, l'instruction \CMP peut être remplacée par une instruction \SUB et presque
tout ira bien, à la différence que \SUB altère la valeur de la première opérande.
\CMP est un \IT{SUB sans sauver le résultat, mais modifiant les flags}.

\subsubsection{MSVC: x86: IDA}

\myindex{IDA}
C'est le moment de lancer \IDA et d'essayer de faire quelque chose avec.
Á propos, pour les débutants, c'est une bonne idée d'utiliser l'option \TT{/MD}
de MSVC, qui signifie que toutes les fonctions standards ne vont pas être liées
avec le fichier exécutable, mais vont à la place être importées depuis le fichier
\TT{MSVCR*.DLL}.
Ainsi il est plus facile de voir quelles fonctions standards sont utilisées et oú.

En analysant du code dans \IDA, il est très utile de laisser des notes pour soi-même
(et les autres).
En la circonstance, analysons cet exemple, nous voyons que \TT{JNZ} sera déclenché
en cas d'erreur.
Donc il est possible de déplacer le curseur sur le label, de presser \q{n} et de
lui donner le nom \q{error}.
Créons un autre label---dans \q{exit}.
Voici mon résultat:

\lstinputlisting[style=customasmx86]{patterns/04_scanf/3_checking_retval/ex3.lst}

Maintenant, il est légèrement plus facile de comprendre le code.
Toutefois, ce n'est pas une bonne idée de commenter chaque instruction.

% FIXME draw button?
Vous pouvez aussi cacher(replier) des parties d'une fonction dans \IDA.
Pour faire cela, marquez le bloc, puis appueaz sur \q{--} sur le pavé numérique et
entrez le texte qui doit être affiché à la place.

Cachons deux blocs et donnons leurs un nom:

\lstinputlisting[style=customasmx86]{patterns/04_scanf/3_checking_retval/ex3_2.lst}

% FIXME draw button?
Pour étendre les parties de code précédemment cachées. utilisez \q{+} sur le
pavé numérique.

\clearpage
En appuyant sur \q{space}, nous voyons comment \IDA représente une fonction sous
forme de graphe:

\begin{figure}[H]
\centering
\myincludegraphics{patterns/04_scanf/3_checking_retval/IDA.png}
\caption{IDA en mode graphe}
\label{fig:ex3_IDA_1}
\end{figure}

Il y a deux flèches après chaque saut conditionnel: une verte et une rouge.
La flèche verte pointe vers le bloc qui sera exécuté si le saut est déclenché,
et la rouge sinon.

\clearpage
Il est possible de replier des noeuds dans ce mode et de leurs donner aussi un nom (\q{group nodes}).
Essayons avec 3 blocs:

\begin{figure}[H]
\centering
\myincludegraphics{patterns/04_scanf/3_checking_retval/IDA2.png}
\caption{IDA en mode graphe avec 3 noeuds repliés}
\label{fig:ex3_IDA_2}
\end{figure}

C'est très pratique.
%%It could be said that a very important part of the reverse engineers' job (and any other researcher as well) is to reduce the amount of information they deal with.
On peut dire qu'une part importante du travail des rétro-ingénieurs (et de tout
autre chercheur également) est de réduire la quantité d'information avec laquelle
travailler.

\clearpage
\subsubsection{MSVC: x86 + \olly}

Essayons de hacker notre programme dans \olly, pour le forcer à penser que \scanf
fonctionne toujours sans erreur.
Lorsque l'adresse d'une variable locale est passée à \scanf, la variable contient
initiallement toujours des restes de données aléatoires, dans ce cas \TT{0x6E494714}:

\begin{figure}[H]
\centering
\myincludegraphics{patterns/04_scanf/3_checking_retval/olly_1.png}
\caption{\olly: passer l'adressed de la variable à \scanf}
\label{fig:scanf_ex3_olly_1}
\end{figure}

\clearpage
Lorsque \scanf s'exécute dans la console, entrons quelque chose qui n'est pas du
tout un nombre, comme \q{asdasd}.
\scanf termine avec 0 dans \EAX, ce qui indique qu'une erreur s'est produite:

\begin{figure}[H]
\centering
\myincludegraphics{patterns/04_scanf/3_checking_retval/olly_2.png}
\caption{\olly: \scanf renvoyant une erreur}
\label{fig:scanf_ex3_olly_2}
\end{figure}

Nous pouvons vérifier la variable locale dans le pile et noter qu'elle n'a pas changé.
En effet, qu'aurait écrit \scanf ici?
Elle n'a simplement rien fait à part renvoyer zéro.

Essayons de \q{hacker} notre programme.
Clique-droit sur \EAX,
Parmis les options il y a \q{Set to 1} (mettre à 1).
C'est ce dont nous avons besoin.

Nous avons maintenant 1 dans \EAX, donc la vérification suivante va s'exécuter comme
souhaiter et \printf va afficher la valeur de la variable dans la pile.

Lorsque nous lançons le programme (F9) nous pouvons voir ceci dans la fenêtre
de la console:

\lstinputlisting[caption=fenêtre console]{patterns/04_scanf/3_checking_retval/console.txt}

En effet, 1850296084 est la représentation en décimal du nombre dans la pile (\TT{0x6E494714})!


\clearpage
\subsubsection{MSVC: x86 + Hiew}
\myindex{Hiew}

Cela peut également être utilisé comme un exemple simple de modification de fichier
exécutable.
Nous pouvons essayer de modifier l'exécutable de telle sorte que le programme va
toujours afficher notre entrée, quelle quelle soit.

En supposant que l'exécutable est compilé avec la bibliothèque externe \TT{MSVCR*.DLL}
(i.e., avec l'option \TT{/MD}) \footnote{c'est aussi appelé \q{dynamic linking}},
nous voyons la fonction \main au début de la section \TT{.text}.
Ouvrons l'exécutable dans Hiew et cherchons le début de la section \TT{.text} (Enter,
F8, F6, Enter, Enter).

Nous pouvons voir cela:

\begin{figure}[H]
\centering
\myincludegraphics{patterns/04_scanf/3_checking_retval/hiew_1.png}
\caption{Hiew: fonction \main}
\label{fig:scanf_ex3_hiew_1}
\end{figure}

Hiew trouve les chaîne \ac{ASCIIZ} et les affiche, comme il le fait avec le nom
des fonctions importées.

\clearpage
Déplacez le curseur à l'adresse \TT{.00401027} (oú se trouve l'instruction \TT{JNZ},
que l'on doit sauter), appuyez sur F3, et ensuite tapez \q{9090} (qui signifie deux
\ac{NOP}s):

\begin{figure}[H]
\centering
\myincludegraphics{patterns/04_scanf/3_checking_retval/hiew_2.png}
\caption{Hiew: remplacement de \TT{JNZ} par deux \ac{NOP}s}
\label{fig:scanf_ex3_hiew_2}
\end{figure}

Appuyez sur F9 (update). Maintenant, l'exécutable est sauvé sur le disque. Il va
se comporter comme nous le voulions.

Deux \ac{NOP}s ne constitue probablement pas l'approche la plus \ae{}sthétique.
Une autre façon de modifier cette instruction est d'écrire simplement 0 dans le
second octet de l'opcode ((\gls{jump offset}), donc ce \TT{JNZ} va toujours sauter
à l'instruction suivante.

Nous pouvons également faire le contraire: remplacer le premier octet avec \TT{EB}
sans modifier le second octet (\gls{jump offset}).
Nous obtiendrions un saut inconditionnel qui est toujours déclenché.
Dans ce cas le message d'erreur sera affiché à chaque fois, peu importe l'entrée.

}

\subsubsection{MSVC: x64}

\myindex{x86-64}

\EN{\subsubsection{MSVC: x64}

\myindex{x86-64}

Since we work here with \Tint{}-typed variables, which are still 32-bit in x86-64, we see how the 32-bit part of the registers (prefixed with \TT{E-}) are used here as well.
While working with pointers, however, 64-bit register parts are used, prefixed with \TT{R-}.

\lstinputlisting[caption=MSVC 2012 x64,style=customasmx86]{patterns/04_scanf/3_checking_retval/ex3_MSVC_x64_EN.asm}

}
\RU{\subsubsection{MSVC: x64}

\myindex{x86-64}

Так как здесь мы работаем с переменными типа \Tint, а они в x86-64 остались 32-битными, то мы здесь видим, как продолжают использоваться регистры с префиксом \TT{E-}.
Но для работы с указателями, конечно, используются 64-битные части регистров с префиксом \TT{R-}.

\lstinputlisting[caption=MSVC 2012 x64,style=customasmx86]{patterns/04_scanf/3_checking_retval/ex3_MSVC_x64_RU.asm}

}
\PTBR{\subsubsection{MSVC: x64}

\myindex{x86-64}

Como trabalhamos aqui com variáveis do tipo \Tint, que ainda são 32-bits no x86-64, nós vemos como a parte de 32-bits dos registradores (com o prefixo \TT{E-}) sao usadas aqui da mesma maneira.
No entanto, quando trabalhamos com ponteiros, as partes dos registradores de 64-bits são usadas, prefixadas com \TT{R-}.

% TODO translate
\lstinputlisting[caption=MSVC 2012 x64,style=customasmx86]{patterns/04_scanf/3_checking_retval/ex3_MSVC_x64_EN.asm}

}
\ITA{\subsubsection{MSVC: x64}

\myindex{x86-64}

Poiche' qui lavoriamo con variabili di tipo \Tint{}, che sono sempre a 32-bit in x86-64, vediamo che viene usata la parte a 32-bit dei registri (con il prefisso \TT{E-}).
Lavorando invece con i puntatori, sono usate la parti a 64-bit dei registri (con il prefisso \TT{R-}).

% TODO translate
\lstinputlisting[caption=MSVC 2012 x64,style=customasmx86]{patterns/04_scanf/3_checking_retval/ex3_MSVC_x64_EN.asm}

}


\EN{\subsubsection{ARM}

\myparagraph{ARM: \OptimizingKeilVI (\ThumbMode)}

\lstinputlisting[caption=\OptimizingKeilVI (\ThumbMode),style=customasmARM]{patterns/04_scanf/3_checking_retval/ex3_ARM_Keil_thumb_O3.asm}

\myindex{ARM!\Instructions!CMP}
\myindex{ARM!\Instructions!BEQ}

The new instructions here are \CMP and \ac{BEQ}.

\CMP is analogous to the x86 instruction with the same name, it subtracts one of the arguments from the other and updates the conditional flags if needed.
% TODO: в мануале ARM $op1 + NOT(op2) + 1$ вместо вычитания

\myindex{ARM!\Registers!Z}
\myindex{x86!\Instructions!JZ}
\ac{BEQ} jumps to another address if the operands were equal to each other, or,
if the result of the last computation has been 0, or if the Z flag is 1.
It behaves as \JZ in x86.

Everything else is simple: the execution flow forks in two branches, then the branches
converge at the point where 0 is written into the \Reg{0} as a function return value, and then the function ends.

\myparagraph{ARM64}

\lstinputlisting[caption=\NonOptimizing GCC 4.9.1 ARM64,numbers=left,style=customasmARM]{patterns/04_scanf/3_checking_retval/ARM64_GCC491_O0_EN.s}

\myindex{ARM!\Instructions!CMP}
\myindex{ARM!\Instructions!Bcc}
Code flow in this case forks with the use of \INS{CMP}/\INS{BNE} (Branch if Not Equal) instructions pair.

}
\RU{\subsubsection{ARM}

\myparagraph{ARM: \OptimizingKeilVI (\ThumbMode)}

\lstinputlisting[caption=\OptimizingKeilVI (\ThumbMode)style=customasmARM]{patterns/04_scanf/3_checking_retval/ex3_ARM_Keil_thumb_O3.asm}

\myindex{ARM!\Instructions!CMP}
\myindex{ARM!\Instructions!BEQ}
Здесь для нас есть новые инструкции: \CMP и \ac{BEQ}.

\CMP аналогична той что в x86: она отнимает один аргумент от второго и сохраняет флаги.

% TODO: в мануале ARM $op1 + NOT(op2) + 1$ вместо вычитания

\myindex{ARM!\Registers!Z}
\myindex{x86!\Instructions!JZ}
\ac{BEQ} совершает переход по другому адресу, 
если операнды при сравнении были равны, 
либо если результат последнего вычисления был 0, либо если флаг Z равен 1.
То же что и \JZ в x86.

Всё остальное просто: исполнение разветвляется на две ветки, затем они сходятся там, 
где в \Reg{0} записывается 0 как возвращаемое из функции значение и происходит выход из функции.

\myparagraph{ARM64}

\lstinputlisting[caption=\NonOptimizing GCC 4.9.1 ARM64,numbers=left,style=customasmARM]{patterns/04_scanf/3_checking_retval/ARM64_GCC491_O0_RU.s}

\myindex{ARM!\Instructions!CMP}
\myindex{ARM!\Instructions!Bcc}

Исполнение здесь разветвляется, используя пару инструкций \INS{CMP}/\INS{BNE} (Branch if Not Equal: переход если не равно).

}
\ITA{\subsubsection{ARM}

\myparagraph{ARM: \OptimizingKeilVI (\ThumbMode)}

\lstinputlisting[caption=\OptimizingKeilVI (\ThumbMode),style=customasmARM]{patterns/04_scanf/3_checking_retval/ex3_ARM_Keil_thumb_O3.asm}

\myindex{ARM!\Instructions!CMP}
\myindex{ARM!\Instructions!BEQ}

Le due nuove istruzioni qui sono \CMP e \ac{BEQ}.

\CMP e' analoga all'istruzione omonima in x86, sottrae uno degli argomenti dall'altro e aggiorna il conditional flags (se necessario).
% TODO: в мануале ARM $op1 + NOT(op2) + 1$ вместо вычитания

\myindex{ARM!\Registers!Z}
\myindex{x86!\Instructions!JZ}
\ac{BEQ} salta ad un altro indirizzo se gli operandi sono uguali, o se il risultato dell'ultima operazione era 0, oppure ancora se il flag Z e' 1.
Si comporta come \JZ in x86.

Tutto il resto e' semplice: il flusso di esecuzione si divide in due rami, e successivamente i due rami convergono al punto in cui 0 viene scritto in 
\Reg{0} come valore di ritorno di una funzione, infine la funzione termina.

\myparagraph{ARM64}

\lstinputlisting[caption=\NonOptimizing GCC 4.9.1 ARM64,numbers=left,style=customasmARM]{patterns/04_scanf/3_checking_retval/ARM64_GCC491_O0_EN.s}

\myindex{ARM!\Instructions!CMP}
\myindex{ARM!\Instructions!Bcc}
Il flusso di codice in questo caso si divide con l'uso della coppia di istruzioni \INS{CMP}/\INS{BNE} (Branch if Not Equal).

}
\FR{\subsubsection{ARM}

\myparagraph{ARM: \OptimizingKeilVI (\ThumbMode)}

\lstinputlisting[caption=\OptimizingKeilVI (\ThumbMode),style=customasmARM]{patterns/04_scanf/3_checking_retval/ex3_ARM_Keil_thumb_O3.asm}

\myindex{ARM!\Instructions!CMP}
\myindex{ARM!\Instructions!BEQ}

Les nouvelles instructions sont \CMP et \ac{BEQ}.

\CMP est similaire à l'instruction x86 du même nom, elle soustrait l'un des arguments
à l'autre et met à jour les flags si nécessaire.

\myindex{ARM!\Registers!Z}
\myindex{x86!\Instructions!JZ}
\ac{BEQ} saute à une autre adresse si les opérandes étaient égales l'une à l'autre,
ou, si le résultat du dernier calcul était 0, ou si le flag Z est à 1.
Elle se comporte comme \JZ en x86.

Tout le reste est simple: le flux d'exécution se sépare en deux branches, puis les
branches convergent vers le point où 0 est écrit dans le registre \Reg{0} comme
valeur de retour de la fonction, et cette dernière se termine.

\myparagraph{ARM64}

\lstinputlisting[caption=GCC 4.9.1 ARM64 \NonOptimizing,numbers=left,style=customasmARM]{patterns/04_scanf/3_checking_retval/ARM64_GCC491_O0_FR.s}

\myindex{ARM!\Instructions!CMP}
\myindex{ARM!\Instructions!Bcc}
Dans ce cas, le flux de code se sépare avec l'utilisation de la paire d'instructions
\INS{CMP}/\INS{BNE} (Branch if Not Equal) (branchement si non égal).

}

\subsubsection{MIPS}

\lstinputlisting[caption=\Optimizing GCC 4.4.5 (IDA),style=customasmMIPS]{patterns/04_scanf/3_checking_retval/MIPS_O3_IDA.lst}

\myindex{MIPS!\Instructions!BEQ}

\EN{\scanf returns the result of its work in register \$V0. It is checked at address 0x004006E4
by comparing the values in \$V0 with \$V1 (1 has been stored in \$V1 earlier, at 0x004006DC).
\INS{BEQ} stands for \q{Branch Equal}.
If the two values are equal (i.e., success), the execution jumps to address 0x0040070C.

}
\RU{\scanf возвращает результат своей работы в регистре \$V0 и он проверяется по адресу 0x004006E4
сравнивая значения в \$V0 и \$V1 (1 записан в \$V1 ранее, на 0x004006DC).
\INS{BEQ} означает \q{Branch Equal} (переход если равно).
Если значения равны (т.е. в случае успеха), произойдет переход по адресу 0x0040070C.

}
\ITA{\scanf restituisce il risultato del suo lavoro nel registro \$V0. Cio' viene controllato all'indirizzo 0x004006E4
confrontando il valore in \$V0 con quello in \$V1 (1 era stato memorizzato in \$V1 precedentemente, a 0x004006DC).
\INS{BEQ} sta per \q{Branch Equal}.
Se i due valori sono uguali (cioe' \scanf e' terminata con successo), l'esecuzione salta all'indirizzo 0x0040070C.

}



\subsubsection{\Exercise}

\myindex{x86!\Instructions!Jcc}
\myindex{ARM!\Instructions!Bcc}
As we can see, the \INS{JNE}/\INS{JNZ} instruction can be easily replaced by the \INS{JE}/\INS{JZ} and vice versa 
(or \INS{BNE} by \INS{BEQ} and vice versa).
But then the basic blocks must also be swapped.
Try to do this in some of the examples.

