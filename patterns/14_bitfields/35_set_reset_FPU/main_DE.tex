\subsection{Setzen und Löschen einzelner Bits: \ac{FPU} Beispiele}

\myindex{IEEE 754}
In der folgenden Form werden die Bits in einem \Tfloat gemäß IEEE 754 Format
abgelegt:

\bigskip
% a hack used here! http://tex.stackexchange.com/questions/73524/bytefield-package
\begin{center}
\begin{bytefield}{32}
	\bitheader[endianness=big]{0,22,23,30,31} \\
	\bitbox{1}{S} & 
	\bitbox{8}{%
		\RU{экспонента}%
		\EN{exponent}%
		\ES{exponente}%
		\PTBRph{}%
		\DEph{}\PLph{}%
		\ITAph{}%
		\FR{exposant}
	} & 
	\bitbox{23}{%
		\RU{мантисса}%
		\EN{mantissa or fraction}%
		\ES{mantisa o fracci\'on}%
		\PTBRph{}%
		\DEph{}\PLph{}%
		\ITAph{}%
		\FR{mantisse ou fraction}
	}
\end{bytefield}
\end{center}

\begin{center}
( S\EMDASH{}%
	\RU{знак}%
	\EN{sign}%
	\ES{signo}%
	\PTBRph{}%
	\DEph{}\PLph{}%
	\ITAph{}%
	\FR{signe}
)
\end{center}

Das Vorzeichen der Zahl ist im \ac{MSB} kodiert.
Wir fragen uns, ob es möglich ist, das Vorzeichen einer Fließkommazahl ohne FPU
Befehle zu ändern.

\lstinputlisting[style=customc]{patterns/14_bitfields/35_set_reset_FPU/abs.c}
Wir brauchen dieser Trickserei in \CCpp um von oder in \Tfloat Werte ohne
tatsächliche Konvertierung zu kopieren.
Hier gibt es also drei Funktionen: my\_abs() resettet \ac{MSB}; \TT{set\_sign()}
setzt das \ac{MSB} und negate() kippt es.

\XOR kann verwendet werden, um ein Bit zu kippen:\myref{XOR_property}.

\subsubsection{x86}

Der Code ist einfach:

\lstinputlisting[caption=\Optimizing MSVC 2012,style=customasmx86]{patterns/14_bitfields/35_set_reset_FPU/abs_MSVC2012_Ox.asm}
Ein Eingabewert von Typ \Tfloat wird vom Stack gelesen, aber wie ein Integerwert
behandelt.

\AND und \OR resetten und setzen das gewünschte Bit und \XOR kippt es.
Schließlich wird der modifizierte Wert nach \TT{ST0} geladen, das
Fließkommazahlen über dieses Register zurückgegeben werden.

Betrachten wir den optimierenden MSVC 2012 für x64:

\lstinputlisting[caption=\Optimizing MSVC 2012 x64,style=customasmx86]{patterns/14_bitfields/35_set_reset_FPU/abs_MSVC2012_x64_Ox.asm}

\myindex{x86!\Instructions!BTR}
\myindex{x86!\Instructions!BTS}
\myindex{x86!\Instructions!BTC}
Der Eingabewert wird nach \TT{XMM0} übergeben und dann auf den lokalen Stack
kopiert. Wir finden hier einige für uns neue Befehle: \BTR, \BTS und \BTC.

Diese Befehle werden zum Resetten (\BTR), Setzen (\BTS) und Invertieren (oder
Komplementieren: \BTC) einzelner Bits verwendet.
Das 31. Bit von 0 gezählt ist das \ac{MSB}.

Schließlich wird das Ergebnis nach \TT{XMM0} kopiert, da Fließkommazahlen in
einer Win64 Umgebung über das Register \TT{XMM0} zurückgegeben werden.

\subsubsection{MIPS}

GCC 4.4.5 für MIPS erzeugt im Großen und Ganzen den gleichen Code:

\lstinputlisting[caption=\Optimizing GCC 4.4.5
(IDA),style=customasmMIPS]{patterns/14_bitfields/35_set_reset_FPU/MIPS_O3_IDA_DE.lst}

\myindex{MIPS!\Instructions!LUI}
Ein einzelner \LUI Befehl wird verwendet, um 0x80000000 in ein Register zu
laden, den \LUI löscht die niederen 16 Bits und da diese ohnehin Nullen in der
Konstanten entsprechen genügt hier ein \LUI ohne nachfolgendes \ORI.

\subsubsection{ARM}

\myparagraph{\OptimizingKeilVI (\ARMMode)}

\lstinputlisting[caption=\OptimizingKeilVI
(\ARMMode),style=customasmARM]{patterns/14_bitfields/35_set_reset_FPU/abs_Keil_ARM_O3_DE.s}

So weit, so gut.
\myindex{ARM!\Instructions!BIC}
\myindex{ARM!\Instructions!EOR}
ARM verfügt über den \BIC Befehl, der ausdrücklich spezifizierte Bits löscht.
\EOR ist in ARM der Name des Befehls für \XOR (\q{exklusives OR}).

\myparagraph{\OptimizingKeilVI (\ThumbMode)}

\lstinputlisting[caption=\OptimizingKeilVI (\ThumbMode),style=customasmx86]{patterns/14_bitfields/35_set_reset_FPU/abs_Keil_thumb_O3.s}
Thumb mode im ARM verwendet 16-Bit-Befehle und da in diesen nicht viele Daten
kodiert werden können, wird hier ein \INS{MOVS}/\INS{LSLS} Befehlspaar benötigt,
um die Konstante 0x80000000 zu konstruieren. 
Das Befehlspaar funktioniert wie folgt: $1<<31 = 0x80000000$.

\myindex{ARM!\Instructions!LSLS}
\myindex{ARM!\Instructions!LSRS}
Der Code von \TT{my\_abs} ist seltsam und entspricht tatsächlich dem folgenden
Ausdruck: $(i<<1)>>1$. Dieser Ausdruck scheint zunächst bedeutungslos. 
Wenn aber $input<<1$ ausgeführt wird, befinden sich alle Bits an ihren korrekten
Plätzen, nur dass das \ac{MSB} null ist, da alle neuen Bits aus, die durch den
Schiebebefehl eingefügt werden, stets Nullen sind.
Auf diese Weise löscht das Befehlspaar \LSLS/\LSRS das \ac{MSB}.

\myparagraph{\Optimizing GCC 4.6.3 (Raspberry Pi, \ARMMode)}

\lstinputlisting[caption=\Optimizing GCC 4.6.3 for Raspberry Pi
(\ARMMode),style=customasmARM]{patterns/14_bitfields/35_set_reset_FPU/raspberry_GCC_O3_ARM_mode_DE.lst}
Lassen wir den Raspberry Pi Linux in QEMU laufen und emulieren eine ARM FPU,
dann werden hier S-Register anstelle der R-Register für den Umgang mit
Fließkommazahlen verwendet.

\myindex{ARM!\Instructions!FMRS}
Der Befehl \FMRS kopiert Daten von \ac{GPR} zur FPU und zurück
\TT{my\_abs()} und \TT{set\_sign()} sehen wie erwartet aus, aber was ist mit
\TT{negate()}? Warum wird hier \ADD anstelle von \XOR verwendet?

\myindex{ARM!\Instructions!XOR}
\myindex{ARM!\Instructions!ADD}
Es ist auf den ersten Blick schwer zu glauben, aber der Befehl \INS{ADD
register, 0x80000000} entspricht\\
\INS{XOR register, 0x80000000}.
Erinnern wir uns an das Ziel des Befehls: Das Ziel ist es, das \ac{MSB} zu
invertieren, also kümmern wir uns zunächst nicht um den \XOR Befehl.
Aus der Schulmathematik wissen wir, dass die Addition von Werten wie z.B. 1000
die letzten drei Stellen einer Zahl nicht verändert.
Zum Beispiel gilt: $1234567 + 10000 = 1244567$ (die letzten vier Stellen können
sich nicht verändern).

Hier arbeiten wir mit Binärzahlen und \\
0x80000000 ist 0b100000000000000000000000000000000, d.h. hier sind nur das
höchste Bit gesetzt. 

Eine Addition von 0x800000000 zu einem anderen Werte kann also nie die niederen
31 Bit verändern, sondern nur das \ac{MSB}.
Addieren wir 1 zu 0, erhalten wir 1.

Addieren wir 1 zu 1, erhalten wir 0b10 in binär, aber das 32. Bit (von 0
gezählt) wird fallengelassen, da unsere Register eine Breite von 32 Bit haben,
sodass das Ergebnis 0 ist.
Deshalb kann in diesem Fall \XOR durch \ADD ersetzt werden.

Es ist schwer nachzuvollziehen, warum GCC diese Ersetzung vorgenommen hat, aber
sie funktioniert tadellos.
