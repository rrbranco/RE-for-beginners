\subsubsection{x64}
\label{subsec:popcnt}

Let's modify the example slightly to extend it to 64-bit:

\lstinputlisting[label=popcnt_x64_example,style=customc]{patterns/14_bitfields/4_popcnt/shifts64.c}

\myparagraph{\NonOptimizing GCC 4.8.2}

So far so easy.

\lstinputlisting[caption=\NonOptimizing GCC 4.8.2,style=customasmx86]{patterns/14_bitfields/4_popcnt/shifts64_GCC_O0_EN.s}

\myparagraph{\Optimizing GCC 4.8.2}

\lstinputlisting[caption=\Optimizing GCC 4.8.2,numbers=left,label=shifts64_GCC_O3,style=customasmx86]{patterns/14_bitfields/4_popcnt/shifts64_GCC_O3_EN.s}

This code is terser, but has a quirk.

In all examples that we see so far, we were incrementing the \q{rt} value after comparing a specific bit,
but the code here increments \q{rt} before (line 6), writing the new value into register \EDX .
Thus, if the last bit is 1, the \CMOVNE\footnote{Conditional MOVe if Not Equal} instruction
(which is a synonym for \CMOVNZ\footnote{Conditional MOVe if Not Zero}) \IT{commits} 
the new value of \q{rt}
by moving \EDX (\q{proposed rt value}) into \EAX (\q{current rt} to be returned at the end).

Hence, the incrementing is performed at each step of loop, i.e., 64 times, without any relation to the input value.

The advantage of this code is that it contain only one conditional jump (at the end of the loop) instead of 
two jumps (skipping the \q{rt} value increment and at the end of loop).
And that might work faster on the modern CPUs with branch predictors: \myref{branch_predictors}.

\label{FATRET}
\myindex{x86!\Instructions!FATRET}
The last instruction is \INS{REP RET} (opcode \TT{F3 C3}) 
which is also called \INS{FATRET} by MSVC.
This is somewhat optimized version of \RET, 
which is recommended by AMD to be placed at the end of function, if \RET goes right after conditional jump: 
\InSqBrackets{\AMDOptimization p.15}
\footnote{More information on it: \url{http://go.yurichev.com/17328}}.

\myparagraph{\Optimizing MSVC 2010}

\lstinputlisting[caption=MSVC 2010,style=customasmx86]{patterns/14_bitfields/4_popcnt/MSVC_2010_x64_Ox_EN.asm}

\myindex{x86!\Instructions!ROL}
Here the \ROL instruction is used instead of 
\SHL, which is in fact \q{rotate left} 
instead of \q{shift left},
but in this example it works just as \TT{SHL}.

You can read more about the rotate instruction here: \myref{ROL_ROR}.

\Reg{8} here is counting from 64 to 0.
It's just like an inverted $i$.

Here is a table of some registers during the execution:

\begin{center}
\begin{tabular}{ | l | l | }
\hline
\HeaderColor RDX & \HeaderColor R8 \\
\hline
0x0000000000000001 & 64 \\
\hline
0x0000000000000002 & 63 \\
\hline
0x0000000000000004 & 62 \\
\hline
0x0000000000000008 & 61 \\
\hline
... & ... \\
\hline
0x4000000000000000 & 2 \\
\hline
0x8000000000000000 & 1 \\
\hline
\end{tabular}
\end{center}

\myindex{x86!\Instructions!FATRET}
At the end we see the \INS{FATRET} instruction, which was explained here: \myref{FATRET}.

\myparagraph{\Optimizing MSVC 2012}

\lstinputlisting[caption=MSVC 2012,style=customasmx86]{patterns/14_bitfields/4_popcnt/MSVC_2012_x64_Ox_EN.asm}

\myindex{\CompilerAnomaly}
\label{MSVC2012_anomaly}
\Optimizing MSVC 2012 does almost the same job as 
optimizing MSVC 2010, but somehow, it generates two identical loop bodies and the loop count is now 32 instead of 64.

To be honest, it's not possible to say why. Some optimization trick? Maybe it's better for the loop body to be slightly 
longer?

Anyway, such code is relevant here to show that sometimes the compiler output may be really weird and 
illogical, but perfectly working.

