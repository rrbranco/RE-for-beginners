\clearpage
\mysubparagraph{\olly}
\myindex{\olly}

Betrachten wir dieses Beispiel in \olly. 
Sei der Eingabewert dabei \TT{0x12345678}.

Für $i=1$ sehen wir, wie $i$ nach \ECX geladen wird: 

\begin{figure}[H]
\centering
\myincludegraphics{patterns/14_bitfields/4_popcnt/olly1_1.png}
\caption{\olly: $i=1$, $i$ wird nach \ECX geladen}
\label{fig:shifts_olly1_1}
\end{figure}

\EDX ist 1. \SHL wird jetzt ausgeführt.

\clearpage
\SHL wurde ausgeführt:

\begin{figure}[H]
\centering
\myincludegraphics{patterns/14_bitfields/4_popcnt/olly1_2.png}
\caption{\olly: $i=1$, \EDX=$1 \ll 1=2$}
\label{fig:shifts_olly1_2}
\end{figure}

\EDX enthält $1 \ll 1$ (oder 2). Hierbei handelt es sich um eine Bitmaske.

\clearpage
\AND setzt \ZF auf 1, was bedeutet, dass der Eingabewert (\TT{0x12345678}) 
mit 2 verUNDet wird. Das Ergebnis ist 0:

\begin{figure}[H]
\centering
\myincludegraphics{patterns/14_bitfields/4_popcnt/olly1_3.png}
\caption{\olly: $i=1$, 
ist hier das Bit im Eingabewert gesetzt? Nein. (\ZF=1)}
\label{fig:shifts_olly1_3}
\end{figure}
Es gibt hier also kein entsprechendes Bit im Eingabewert.

Das Codestück, welches den Zähler erhöht, wird also nicht ausgeführt:
Der \JZ Befehl überspringt es.

\clearpage
Verfolgen wir den Ablauf ein bisschen weiter bis $i$ den Wert 4 hat.
\SHL wird jetzt ausgeführt:

\begin{figure}[H]
\centering
\myincludegraphics{patterns/14_bitfields/4_popcnt/olly4_1.png}
\caption{\olly: $i=4$, $i$ wird nach \ECX geladen}
\label{fig:shifts_olly4_1}
\end{figure}

\clearpage
\EDX=$1 \ll 4$ (oder \TT{0x10} oder 16): 

\begin{figure}[H]
\centering
\myincludegraphics{patterns/14_bitfields/4_popcnt/olly4_2.png}
\caption{\olly: $i=4$, \EDX=$1 \ll 4=0x10$}
\label{fig:shifts_olly4_2}
\end{figure}

Hierbei handelt es sich um eine weitere Bitmaske.

\clearpage
\AND wird ausgeführt:

\begin{figure}[H]
\centering
\myincludegraphics{patterns/14_bitfields/4_popcnt/olly4_3.png}
\caption{\olly: $i=4$, 
ist hier das Bit im Eingabewert gesetzt? Ja. (\ZF=0)}
\label{fig:shifts_olly4_3}
\end{figure}

\ZF ist 0 , da das Bit im Eingabewert gesetzt ist.\\
Tatsächlich gilt \TT{0x12345678 \& 0x10 = 0x10}. 

Das Bit wird gezählt: der Sprung wird nicht ausgeführt und der Zähler wird
erhöht.

Die Funktion liefert den Wert 13 zurück.
Dies entspricht der Anzahl der in der binären Darstellung von \TT{0x12345678}
gesetzten Bits.

