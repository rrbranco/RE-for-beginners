\subsubsection{x64}
\label{subsec:popcnt}
Verändern wir das Beispiel ein wenig um es auf 64 Bit zu erweitern:

\lstinputlisting[label=popcnt_x64_example,style=customc]{patterns/14_bitfields/4_popcnt/shifts64.c}

\myparagraph{\NonOptimizing GCC 4.8.2}

So weit, so einfach.

\lstinputlisting[caption=\NonOptimizing GCC
4.8.2,style=customasmx86]{patterns/14_bitfields/4_popcnt/shifts64_GCC_O0_DE.s}

\myparagraph{\Optimizing GCC 4.8.2}

\lstinputlisting[caption=\Optimizing GCC
4.8.2,numbers=left,label=shifts64_GCC_O3,style=customasmx86]{patterns/14_bitfields/4_popcnt/shifts64_GCC_O3_DE.s}
Dieser Code ist kürzer, birgt aber eine Besonderheit.

In allen bisher betrachteten Beispieln haben wir den Wert von \q{rt} nach dem
Vergleich mit einem speziellen Bit erhöht, aber dieser Code erhöht \q{rt} vorher
(Zeile 6) und schreibt den neuen Wert in das Register \EDX.
Dadurch überträgt der Befehl \CMOVNE\footnote{Conditional MOVe if Not Equal}
(der ein Synonym für \CMOVNZ\footnote{Conditional MOVe if Not Zero} ist) den
neuen Wert von \q{rt} durch Verschieben des Wertes in \EDX (vorgeschlagener
Wert von \q{rt}) nach \EAX (\q{aktueller Wert von rt}). Der in \EAX befindliche
Wert wird schließlich zurückgegeben.

Deshalb wird die Erhöhung des Zählers in jedem Durchlauf der Schleife
durchgeführt, d.h. 64 mal, ohne dass eine Abhängigkeit vom Eingabewert
besteht.

Der Vorteil dieses Code ist, dass er nur einen bedingten Sprung enthält (am
Ende der Schleife) anstatt zwei Sprüngen (Überspringen des Erhöhens von \q{rt}
und Ende der Schleife). 
Der Code ist somit auf modernen CPUs mit Branch Pedictors möglicherweise
schneller:\myref{branch_predictors}.

\label{FATRET}
\myindex{x86!\Instructions!FATRET}
Der letzte Befehl hier ist \INS{REP RET} (Opcode \TT{F3 C3}), der von MSVC auch
\INS{FATRET} genannt wird.
Hierbei handelt es sich um eine optimierte Version von \RET, die von ARM
bevorzugt am Ende der Funktion verwendet wird, wenn \RET direkt nach einem
bedingten Sprung folg:.
\InSqBrackets{\AMDOptimization p.15}
\footnote{Mehr Informationen dazu: \url{http://go.yurichev.com/17328}}.

\myparagraph{\Optimizing MSVC 2010}

\lstinputlisting[caption=MSVC
2010,style=customasmx86]{patterns/14_bitfields/4_popcnt/MSVC_2010_x64_Ox_DE.asm}

\myindex{x86!\Instructions!ROL}
Hier wird der Befehl \ROL anstelle von \SHL verwendet, welches einer
\q{Linksrotation} anstatt einer \q{Linksverschiebung} entspricht.
In diesem Beispiel entspricht \ROL einem \TT{SHL}.

Für mehr Informationen zu Rotationsbefehlen siehe: \myref{ROL_ROR}. 

\Reg{8} zählt hier von 64 auf 0 herunter. 
Dies entspricht dem invertierten $i$.

Hier ist eine Tabelle einiger Register während der Ausführung des Programms:

\begin{center}
\begin{tabular}{ | l | l | }
\hline
\HeaderColor RDX & \HeaderColor R8 \\
\hline
0x0000000000000001 & 64 \\
\hline
0x0000000000000002 & 63 \\
\hline
0x0000000000000004 & 62 \\
\hline
0x0000000000000008 & 61 \\
\hline
... & ... \\
\hline
0x4000000000000000 & 2 \\
\hline
0x8000000000000000 & 1 \\
\hline
\end{tabular}
\end{center}

\myindex{x86!\Instructions!FATRET}
Am Ende finden wir den Befehl \INS{FATRET}, der hier schon erklärt
wurde:\myref{FATRET}.

\myparagraph{\Optimizing MSVC 2012}

\lstinputlisting[caption=MSVC
2012,style=customasmx86]{patterns/14_bitfields/4_popcnt/MSVC_2012_x64_Ox_DE.asm}

\myindex{\CompilerAnomaly}
\label{MSVC2012_anomaly}
Der optimierende MSVC 2012 erzeugt fast den gleichen Code wie MSVC 2012,
generiert aber aus irgendeinem Grund zwei identischen Rümpfe für die Schleifen
und die Schleife zählt nun bis 32 anstatt 64.

Ehrlich gesagt, kann man nicht genau erklären warum. Es könnte sich um einen
Optimierungstrick handeln. Vielleicht ist es für den Rumpf der Schleife besser
ein wenig länger zu sein.

Trotzdem ist solcher Code relevant um zu zeigen, dass der Output des Compilers
manchmal sehr merkwürdig und unlogisch sein kann und dennoch tadellos
funktioniert.
