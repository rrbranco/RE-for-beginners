\subsubsection{MIPS}

\myparagraph{\NonOptimizing GCC}

\lstinputlisting[caption=\NonOptimizing GCC 4.4.5 (IDA),style=customasmMIPS]{patterns/14_bitfields/4_popcnt/MIPS_O0_IDA_EN.lst}

\myindex{MIPS!\Instructions!SLL}
\myindex{MIPS!\Instructions!SLLV}

That is verbose: all local variables are located in the local stack and reloaded each time they're needed.

The \SLLV instruction is \q{Shift Word Left Logical Variable}, it differs from \SLL only in that
the shift amount is encoded in the \SLL instruction (and is fixed, as a consequence), 
but \SLLV takes shift amount from a register.

\myparagraph{\Optimizing GCC}

That is terser.
There are two shift instructions instead of one.
Why?

It's possible to replace the first \SLLV instruction with an unconditional branch instruction that jumps right to the second \SLLV.
But this is another branching instruction in the function, and it's always favorable to get rid of them: \myref{branch_predictors}.

\lstinputlisting[caption=\Optimizing GCC 4.4.5 (IDA),style=customasmMIPS]{patterns/14_bitfields/4_popcnt/MIPS_O3_IDA_EN.lst}

