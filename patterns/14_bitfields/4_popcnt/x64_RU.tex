\subsubsection{x64}
\label{subsec:popcnt}

Немного изменим пример, расширив его до 64-х бит:

\lstinputlisting[label=popcnt_x64_example,style=customc]{patterns/14_bitfields/4_popcnt/shifts64.c}

\myparagraph{\NonOptimizing GCC 4.8.2}

Пока всё просто.

\lstinputlisting[caption=\NonOptimizing GCC 4.8.2,style=customasmx86]{patterns/14_bitfields/4_popcnt/shifts64_GCC_O0_RU.s}

\myparagraph{\Optimizing GCC 4.8.2}

\lstinputlisting[caption=\Optimizing GCC 4.8.2,numbers=left,label=shifts64_GCC_O3,style=customasmx86]{patterns/14_bitfields/4_popcnt/shifts64_GCC_O3_RU.s}

Код более лаконичный, но содержит одну необычную вещь.
Во всех примерах, что мы пока видели, инкремент значения переменной \q{rt} происходит после сравнения 
определенного бита с единицей, но здесь \q{rt} увеличивается на 1 до этого (строка 6), записывая новое значение
в регистр \EDX.

Затем, если последний бит был 1, инструкция \CMOVNE\footnote{Conditional MOVe if Not Equal (\MOV если не равно)}
(которая синонимична \CMOVNZ\footnote{Conditional MOVe if Not Zero (\MOV если не ноль)}) \IT{фиксирует} 
новое значение \q{rt}
копируя значение из \EDX (\q{предполагаемое значение rt}) 
в \EAX (\q{текущее rt} которое будет возвращено в конце функции).
Следовательно, инкремент происходит на каждом шаге цикла, т.е. 64 раза, вне всякой связи с входным
значением.

Преимущество этого кода в том, что он содержит только один условный переход (в конце цикла) вместо
двух (пропускающий инкремент \q{rt} и ещё одного в конце цикла).

И это может работать быстрее на современных CPU с предсказателем переходов: \myref{branch_predictors}.

\label{FATRET}
\myindex{x86!\Instructions!FATRET}
Последняя инструкция это \INS{REP RET} (опкод \TT{F3 C3}) 
которая также называется \INS{FATRET} в MSVC.
Это оптимизированная версия \RET, рекомендуемая AMD для вставки в конце функции, если \RET идет
сразу после условного перехода: 
\InSqBrackets{\AMDOptimization p.15}
\footnote{Больше об этом: \url{http://go.yurichev.com/17328}}.

\myparagraph{\Optimizing MSVC 2010}

\lstinputlisting[caption=MSVC 2010,style=customasmx86]{patterns/14_bitfields/4_popcnt/MSVC_2010_x64_Ox_RU.asm}

\myindex{x86!\Instructions!ROL}
Здесь используется инструкция \ROL вместо 
\SHL, которая на самом деле \q{rotate left} (прокручивать влево) 
вместо \q{shift left} (сдвиг влево),
но здесь, в этом примере, она работает так же как и  \TT{SHL}.

Об этих \q{прокручивающих} инструкциях больше читайте здесь: \myref{ROL_ROR}.

\Reg{8} здесь считает от 64 до 0. 
Это как бы инвертированная переменная $i$.

Вот таблица некоторых регистров в процессе исполнения:

\begin{center}
\begin{tabular}{ | l | l | }
\hline
\HeaderColor RDX & \HeaderColor R8 \\
\hline
0x0000000000000001 & 64 \\
\hline
0x0000000000000002 & 63 \\
\hline
0x0000000000000004 & 62 \\
\hline
0x0000000000000008 & 61 \\
\hline
... & ... \\
\hline
0x4000000000000000 & 2 \\
\hline
0x8000000000000000 & 1 \\
\hline
\end{tabular}
\end{center}

\myindex{x86!\Instructions!FATRET}
В конце видим инструкцию \INS{FATRET}, которая была описана здесь: \myref{FATRET}.

\myparagraph{\Optimizing MSVC 2012}

\lstinputlisting[caption=MSVC 2012,style=customasmx86]{patterns/14_bitfields/4_popcnt/MSVC_2012_x64_Ox_RU.asm}

\myindex{\CompilerAnomaly}
\label{MSVC2012_anomaly}
\Optimizing MSVC 2012 делает почти то же самое что и оптимизирующий MSVC 2010, но почему-то он генерирует 2 идентичных тела цикла и счетчик цикла теперь 32
вместо 64.
Честно говоря, нельзя сказать, почему. Какой-то трюк с оптимизацией? Может быть, телу цикла лучше быть
немного длиннее?

Так или иначе, такой код здесь уместен, чтобы показать, что результат компилятора
иногда может быть очень странный и нелогичный, но прекрасно работающий, конечно же.

