\subsubsection{ARM}

\myindex{Linux}
В ядре Linux 3.8.0 бит \TT{O\_CREAT} проверяется немного иначе.

\begin{lstlisting}[caption=linux kernel 3.8.0,style=customc]
struct file *do_filp_open(int dfd, struct filename *pathname,
		const struct open_flags *op)
{
...
	filp = path_openat(dfd, pathname, &nd, op, flags | LOOKUP_RCU);
...
}

static struct file *path_openat(int dfd, struct filename *pathname,
		struct nameidata *nd, const struct open_flags *op, int flags)
{
...
	error = do_last(nd, &path, file, op, &opened, pathname);
...
}


static int do_last(struct nameidata *nd, struct path *path,
		   struct file *file, const struct open_flags *op,
		   int *opened, struct filename *name)
{
...
	if (!(open_flag & O_CREAT)) {
    ...
		error = lookup_fast(nd, path, &inode);
    ...
	} else {
    ...
		error = complete_walk(nd);
        }
...
}
\end{lstlisting}

Вот как это выглядит в \IDA, ядро скомпилированное для режима ARM:

\lstinputlisting[caption=do\_last() из vmlinux (IDA),style=customasmARM]{patterns/14_bitfields/1_check/IDA.lst}

\myindex{ARM!\Instructions!TST}
\myindex{x86!\Instructions!TEST}
\TT{TST} это аналог инструкции \TEST в x86.
Мы можем \q{узнать} визуально этот фрагмент кода по тому что в одном случае исполнится 
функция \TT{lookup\_fast()}, а в другом \TT{complete\_walk()}.
Это соответствует исходному коду функции \TT{do\_last()}.
Макрос \TT{O\_CREAT} здесь так же равен \TT{0x40}.

