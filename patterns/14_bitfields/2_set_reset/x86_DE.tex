\subsubsection{x86}

\myparagraph{\NonOptimizing MSVC}

Wir erhalten folgenden Code: (MSVC 2010):

\lstinputlisting[caption=MSVC 2010,style=customasmx86]{patterns/14_bitfields/2_set_reset/set_reset_msvc.asm}

\myindex{x86!\Instructions!OR}
Der \OR Befehl setzt ein Bit auf einen Wert, während die übrigen ignoriert
werden.

\myindex{x86!\Instructions!AND}
\AND resettet ein Bit. Man kann sagen, dass \AND einfach alle Bits bis auf eines
kopiert. Tatsächlich werden im zweiten Operanden von \AND nur die Bits gesetzt,
die auch gespeichert werden müssen, lediglich das eine, das nicht kopiert werden
soll (die 0 in der Bitmaske), wird nicht gesetzt.
Auf diese Weise kann man sich die Logik des Befehls leichter merken.

\clearpage
\mysubparagraph{\olly}

Untersuchen wir dieses Beispiel in \olly.
Schauen wir zunächst die binäre Form der zu verwendenden Konstanten an:

\TT{0x200} (0b0000000000000000000{\color{red}1}000000000) (d.h. das 10. Bit
(vom ersten aus gezählt)).

\TT{0x200} ist invertiert \TT{0xFFFFFDFF}
(0b1111111111111111111{\color{red}0}111111111).

\TT{0x4000} (0b00000000000000{\color{red}1}00000000000000) (d.h. das 15. Bit).

Der Eingabewert ist: \TT{0x12340678} (0b10010001101000000011001111000).
Wir beobachten, wie der Wert geladen wird:

\begin{figure}[H]
\centering
\myincludegraphics{patterns/14_bitfields/2_set_reset/olly1.png}
\caption{\olly: der Wert wird nach \ECX geladen}
\label{fig:set_reset_olly1}
\end{figure}

\clearpage
\OR wurde ausgeführt:

\begin{figure}[H]
\centering
\myincludegraphics{patterns/14_bitfields/2_set_reset/olly2.png}
\caption{\olly: \OR wurde ausgeführt}
\label{fig:set_reset_olly2}
\end{figure}

Das 15. Bit ist gesetzt: \TT{0x1234{\color{red}4}678} 
(0b10010001101000{\color{red}1}00011001111000).

\clearpage
Der Wert wird erneut geladen (da der Compiler nicht optimiert hat):

\begin{figure}[H]
\centering
\myincludegraphics{patterns/14_bitfields/2_set_reset/olly3.png}
\caption{\olly: der Wert wird erneut nach \EDX geladen}
\label{fig:set_reset_olly3}
\end{figure}

\clearpage
\AND wurde ausgeführt:

\begin{figure}[H]
\centering
\myincludegraphics{patterns/14_bitfields/2_set_reset/olly4.png}
\caption{\olly: \AND wurde ausgeführt}
\label{fig:set_reset_olly4}
\end{figure}
Das 10. Bit wurde gelöscht (oder, mit anderen Worten: alle Bite außer dem 10.
wurden stehengelassen) und das Endergebnis ist\\
\TT{0x12344{\color{red}4}78} (0b1001000110100010001{\color{red}0}001111000).


\myparagraph{\Optimizing MSVC}
Wenn wir das Beispiel mit MSVC mit Optimierung (\Ox) kompilieren, ist der Code
noch kürzer:

\lstinputlisting[caption=\Optimizing MSVC,style=customasmx86]{patterns/14_bitfields/2_set_reset/set_reset_msvc_Ox.asm}

\myparagraph{\NonOptimizing GCC}

Untersuchen wir GCC 4.4.1 ohne Optimierung:

\lstinputlisting[caption=\NonOptimizing GCC,style=customasmx86]{patterns/14_bitfields/2_set_reset/set_reset_gcc.asm}
Obwohl es hier redundanten Code gibt, ist das Ergebnis kürzer als die MSVC
Version ohne Optimierung.

Jetzt aktivieren wir die Optimierung von GCC mit \Othree: 

\myparagraph{\Optimizing GCC}

\lstinputlisting[caption=\Optimizing GCC,style=customasmx86]{patterns/14_bitfields/2_set_reset/set_reset_gcc_O3.asm}

Das Ergebnis ist noch kürzer.
Man beachte, dass der Compiler mit dem \EAX Regsiter über das Teilregister \AH
arbeitet---das ist der Teil vom 8. bis zum 15. Bit (jeweils einschließlich).

\RegTableOne{RAX}{EAX}{AX}{AH}{AL}

\myindex{Intel!8086}
\myindex{Intel!80386}
Der 16-Bit CPU 8086 Akkumulator wurde \AX getauft und bestand aus zwei
8-Bit-Hälften---\AL (niederes Byte) und \AH (höheres Byte).
In 80386 wurden fast alle Register auf 32 Bit erweitert und der Akkumulator
wurde fortan \EAX genannt, aber aus Kompatibilitätsgründen ist es immernoch
möglich gezielt \AX/\AH/\AL anzusprechen.

Da alle x86 CPUs Nachfolger der 16-Bit 8086 CPU sind, sind die älteren 16-Bit
Opcodes kürzer als die neueren 32-Bit Opcodes.
Aus diesem Gründ benötigt der Befehl \INS{or ah, 40h} nur 3 Bytes. Logischer
wäre es zwar, hier \INS{or eax, 04000h} zu verwenden, aber dieser Befehl würde 5
oder sogar 6 Byte (falls das Register im ersten Operanden nicht \EAX ist)
verbrauchen.

\myparagraph{\Optimizing GCC und regparm}
Es wäre noch kürzer, wenn man die Optimierung mit \Othree anschaltet und
\TT{regparm=3} setzt.

\lstinputlisting[caption=\Optimizing GCC,style=customasmx86]{patterns/14_bitfields/2_set_reset/set_reset_gcc_O3_regparm3.asm}

\myindex{Inline code}
Das erste Argument ist schon nach \EAX geladen worden, sodass es möglich ist,
damit direkt weiterzuarbeiten. Bemerkenswert ist, dass sowohl der
Funktionsprolog (\INS{push ebp / mov ebp,esp}) als auch der Funktionsepilog
(\INS{pop ebp}) hier wegfallen können. GCC ist aber möglicherweise nicht gut
genug um eine solche Code Optimierung hier durchzuführen. Auf jeden Fall sind
solche kurzen Funktion am besten als \IT{inline functions}
(\myref{inline_code}) zu kennzeichnen.
