\section{Шутка с task manager (Windows Vista)}
\myindex{Windows!Windows Vista}

Посмотрим, сможем ли мы немного хакнуть Task Manager, чтобы он находил больше ядер в \ac{CPU}, чем присутствует.

\myindex{Windows!NTAPI}
В начале задумаемся, откуда Task Manager знает количество ядер?

В win32 имеется функция \TT{GetSystemInfo()}, при помощи которой можно узнать.

Но она не импортируется в \TT{taskmgr.exe}.
Есть еще одна в \gls{NTAPI}, \TT{NtQuerySystemInformation()}, которая используется в 
\TT{taskmgr.exe} в ряде мест.

Чтобы узнать количество ядер, нужно вызвать эту функцию с константной \\
\TT{SystemBasicInformation} в первом аргументе (а это ноль

\footnote{\href{http://go.yurichev.com/17251}{MSDN}}).

Второй аргумент должен указывать на буфер, который примет всю информацию.

Так что нам нужно найти все вызовы функции \\
\TT{NtQuerySystemInformation(0, ?, ?, ?)}.

Откроем \TT{taskmgr.exe} в IDA. 
\myindex{Windows!PDB}
Что всегда хорошо с исполняемыми файлами от Microsoft, это то что IDA может скачать соответствующий 
\gls{PDB}-файл именно для этого файла и добавить все имена функций.

Видимо, Task Manager написан на \Cpp и некоторые функции и классы имеют говорящие за себя имена.

Тут есть классы CAdapter, CNetPage, CPerfPage, CProcInfo, CProcPage, CSvcPage, 
CTaskPage, CUserPage.
Должно быть, каждый класс соответствует каждой вкладке в Task Manager.

Пройдемся по всем вызовам и добавим комментарий с числом, передающимся как первый аргумент.

В некоторых местах напишем \q{not zero}, потому что значение в тех местах однозначно не ноль, 
но что-то другое (больше об этом во второй части главы).%

А мы все-таки ищем ноль передаваемый как аргумент.

\begin{figure}[H]
\centering
\myincludegraphics{examples/taskmgr/IDA_xrefs.png}
\caption{IDA: вызовы функции NtQuerySystemInformation()}
\end{figure}

Да, имена действительно говорящие сами за себя.

Когда мы внимательно изучим каждое место, где вызывается\\
\TT{NtQuerySystemInformation(0, ?, ?, ?)},
то быстро найдем то что нужно в функции \TT{InitPerfInfo()}:%

\lstinputlisting[caption=taskmgr.exe (Windows Vista),style=customasmx86]{examples/taskmgr/taskmgr.lst}

\TT{g\_cProcessors} это глобальная переменная и это имя присвоено IDA в соответствии с \gls{PDB}-файлом,
скачанным с сервера символов Microsoft.

Байт берется из \TT{var\_C20}. 
И \TT{var\_C58} передается в\\
\TT{NtQuerySystemInformation()} 
как указатель на принимающий буфер.
Разница между 0xC20 и 0xC58 это 0x38 (56).
Посмотрим на формат структуры, который можно найти в MSDN:

\begin{lstlisting}[style=customc]
typedef struct _SYSTEM_BASIC_INFORMATION {
    BYTE Reserved1[24];
    PVOID Reserved2[4];
    CCHAR NumberOfProcessors;
} SYSTEM_BASIC_INFORMATION;
\end{lstlisting}

Это система x64, так что каждый PVOID занимает здесь 8 байт.

Так что все \IT{reserved}-поля занимают $24+4*8=56$.

О да, это значит, что \TT{var\_C20} в локальном стеке это именно поле
\TT{NumberOfProcessors} структуры \TT{SYSTEM\_BASIC\_INFORMATION}.

Проверим нашу догадку.
Скопируем \TT{taskmgr.exe} из \TT{C:\textbackslash{}Windows\textbackslash{}System32} 
в какую-нибудь другую папку 
(чтобы \IT{Windows Resource Protection} не пыталась восстанавливать измененный \TT{taskmgr.exe}).

Откроем его в Hiew и найдем это место:

\begin{figure}[H]
\centering
\myincludegraphics{examples/taskmgr/hiew2.png}
\caption{Hiew: найдем это место}
\end{figure}

Заменим инструкцию \TT{MOVZX} на нашу.

Сделаем вид что у нас 64 ядра процессора.
Добавим дополнительную инструкцию \ac{NOP} (потому что наша инструкция короче чем та что там сейчас):

\begin{figure}[H]
\centering
\myincludegraphics{examples/taskmgr/hiew1.png}
\caption{Hiew: меняем инструкцию}
\end{figure}

И это работает!
Конечно же, данные в графиках неправильные.
Иногда, Task Manager даже показывает общую загрузку CPU более 100\%.

\begin{figure}[H]
\centering
\myincludegraphics{examples/taskmgr/taskmgr_64cpu_crop.png}
\caption{Обманутый Windows Task Manager}
\end{figure}

Самое большое число, при котором Task Manager не падает, это 64.

Должно быть, Task Manager в Windows Vista не тестировался на компьютерах с большим количеством ядер.

И, наверное, там есть внутри какие-то статичные структуры данных, ограниченные до 64-х ядер.

\subsection{Использование LEA для загрузки значений}
\label{TaskMgr_LEA}

Иногда, \TT{LEA} используется в \TT{taskmgr.exe} вместо \TT{MOV} для установки первого аргумента \\
\TT{NtQuerySystemInformation()}:

\lstinputlisting[caption=taskmgr.exe (Windows Vista),style=customasmx86]{examples/taskmgr/taskmgr2.lst}

\myindex{x86!\Instructions!LEA}
Должно быть, \ac{MSVC} сделал так, потому что код инструкции \INS{LEA} короче чем \INS{MOV REG, 5} (было бы 5 байт вместо 4).

\INS{LEA} со смещением в пределах $-128..127$ (смещение будет занимать 1 байт в опкоде) с 32-битными регистрами даже еще короче (из-за отсутствия REX-префикса) --- 3 байта.

Еще один пример подобного: \myref{using_MOV_and_pack_of_LEA_to_load_values}.
