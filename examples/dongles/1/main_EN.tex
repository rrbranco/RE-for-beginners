\newcommand{\PPC}{\InSqBrackets{\IT{PowerPC(tm) Microprocessor Family: The Programming Environments for 32-Bit Microprocessors}, (2000)}\footnote{\AlsoAvailableAs \url{http://yurichev.com/mirrors/PowerPC/6xx_pem.pdf}}}

\newcommand{\PPCABI}{[Steve Zucker, SunSoft and Kari Karhi, IBM, \IT{SYSTEM V APPLICATION BINARY INTERFACE: PowerPC Processor Supplement}, (1995)]\footnote{\AlsoAvailableAs \url{http://yurichev.com/mirrors/PowerPC/elfspec_ppc.pdf}}}

\subsection{Example \#1: MacOS Classic and PowerPC}

\myindex{PowerPC}
\myindex{Mac OS Classic}
Here is an example of a program for MacOS Classic
\footnote{pre-UNIX MacOS}, for PowerPC.
The company who developed the software product
has disappeared a long time ago, so the (legal) customer was afraid of physical dongle damage.

While running without a dongle connected, a message box with the text
"Invalid Security Device" appeared.

Luckily, this text string could easily be found in the executable binary file.

Let's pretend we are not very familiar both with Mac OS Classic and PowerPC, but will try anyway.

\ac{IDA} 
opened the executable file smoothly, reported its type as 
"PEF (Mac OS or Be OS executable)" (indeed, it is a standard Mac OS Classic file format).

By searching for the text string with the error message, we've got into this code fragment:

\lstinputlisting[style=customasmPPC]{examples/dongles/1/1.lst}

\myindex{ARM}
\myindex{MIPS}
Yes, this is PowerPC code.

The CPU is a very typical 32-bit \ac{RISC} of 1990s era.

Each instruction occupies 4 bytes (just as in MIPS and ARM) and the names somewhat resemble
MIPS instruction names.

\TT{check1()} is a function name we'll give to it later.
\TT{BL} is \IT{Branch Link} 
instruction, e.g., intended for calling subroutines.

The crucial point is the \ac{BNE} instruction which jumps if the dongle protection check passes 
or not if an error occurs: 
then the address of the text string gets loaded into the r3 register for the subsequent passing into a message box routine.

From the \PPCABI we will found out that the r3 register is used for return values (and r4, in case of 64-bit values).

\myindex{x86!\Instructions!MOVZX}
Another yet unknown instruction is \TT{CLRLWI}. 
From \PPC we'll learn that this instruction does both clearing and loading. 
In our case, it clears the 24 high bits from the value in r3
and puts them in r0, so it is analogical to \MOVZX in x86 (\myref{movzx}),
but it also sets the flags, so \ac{BNE} 
can check them afterwards.

Let's take a look into the \TT{check1()} function:

\lstinputlisting[style=customasmPPC]{examples/dongles/1/check1.lst}

As you can see in \ac{IDA}, that function is called from many places in the program, but only the r3 register's value
is checked after each call.
\myindex{thunk-functions}

All this function does is to call the other function, so it is a \gls{thunk function}: 
there are function prologue and epilogue, but the r3 register is not touched, so \TT{checkl()} 
returns what \TT{check2()} returns.

\ac{BLR} looks like the return from the function, but since \ac{IDA} does the function layout, we probably do not need
to care about this.

Since it is a typical \ac{RISC}, it seems that subroutines are called using a \gls{link register},
just like in ARM.

The \TT{check2()} function is more complex:

\lstinputlisting[style=customasmPPC]{examples/dongles/1/check2.lst}

\myindex{USB}

We are lucky again: some function names are left in the executable 
(debug symbols section? \\
Hard to say while we are not very familiar with the file format, maybe it is
some kind of PE exports? (\myref{PE_exports_imports})),\\
like \TT{.RBEFINDNEXT()} and \TT{.RBEFINDFIRST()}.

Eventually these functions call other functions with names like \TT{.GetNextDeviceViaUSB()}, 
\TT{.USBSendPKT()},
so these are clearly dealing with an USB device.

There is even a function named 
\TT{.GetNextEve3Device()}---sounds familiar, there was a Sentinel Eve3 
dongle for ADB port (present on Macs) in 1990s.

Let's first take a look on how the r3 register is set before return, while ignoring everything else.

We know that a \q{good} r3 value has to be non-zero, zero r3 leads the execution
flow to the message box with an error message.

There are two \TT{li \%r3, 1} 
instructions present in the function and one \TT{li \%r3, 0} 
(\IT{Load Immediate}, i.e., loading a value into a register).
The first instruction is at 
\TT{0x001186B0}---and frankly speaking, it's hard to say what it means.

What we see next is, however, easier to understand: 
\TT{.RBEFINDFIRST()} is called:
if it fails, 0 is written into r3 and we jump to \IT{exit}, otherwise another
function is called (\TT{check3()})---if it fails too, 
\TT{.RBEFINDNEXT()} 
is called, probably in order to look for another USB device.

N.B.: \TT{clrlwi. \%r0, \%r3, 16} it is analogical to what we already saw, but it clears
16 bits, i.e., \\
\TT{.RBEFINDFIRST()} 
probably returns a 16-bit value.

\TT{B} (stands for \IT{branch}) unconditional jump.

\ac{BEQ} is the inverse instruction of \ac{BNE}.

Let's see \TT{check3()}:

\lstinputlisting[style=customasmPPC]{examples/dongles/1/check3.lst}

There are a lot of calls to \TT{.RBEREAD()}. 

Perhaps, the function returns some values from the dongle,
so they are compared here with some hard-coded variables using \TT{CMPLWI}.

We also see that the r3 register is also filled before each call to \TT{.RBEREAD()} 
with one of these values: 0, 1, 8, 0xA, 0xB, 0xC, 0xD, 4, 5.
Probably a memory address or something like that?

Yes, indeed, by googling these function names it is easy to find the Sentinel Eve3 dongle manual!

Perhaps we don't even have to learn any other PowerPC instructions: all this function does is just
call \TT{.RBEREAD()}, compare its results with the constants and returns 1 if the comparisons
are fine or 0 otherwise.

OK, all we've got is that \TT{check1()} has always to return 1 or any other non-zero value.

But since we are not very confident in our knowledge of PowerPC instructions, we are going to be careful: we will patch the jumps in \TT{check2()} at
\TT{0x001186FC} and \TT{0x00118718}.

At \TT{0x001186FC} we'll write bytes 0x48 and 0 thus converting the \ac{BEQ} instruction in an \TT{B} (unconditional jump):
we can spot its opcode in the code without even referring to \PPC.

At \TT{0x00118718} we'll write 0x60 and 3 zero bytes, thus converting it to a
\ac{NOP} instruction:
Its opcode we could spot in the code too.

And now it all works without a dongle connected.

In summary, such small modifications can be done with \ac{IDA} and minimal assembly language knowledge.

