\section{\q{QR9}: Любительская криптосистема, вдохновленная кубиком Рубика}

Любительские криптосистемы иногда встречаются довольно странные.

Однажды автора сих строк попросили разобраться с одним таким любительским криптоалгоритмом встроенным в 
утилиту для шифрования, исходный код которой был утерян\footnote{Он также получил разрешение от 
клиента на публикацию деталей алгоритма}.

Вот листинг этой утилиты для шифрования, полученный при помощи \IDA:

% TODO translate:
\lstinputlisting[style=customasmx86]{examples/qr9/qr9_original.lst}

Все имена функций и меток даны мною в процессе анализа.

Начнем с самого верха. Вот функция, берущая на вход два имени файла и пароль.


\begin{lstlisting}[style=customasmx86]
.text:00541320 ; int __cdecl crypt_file(int Str, char *Filename, int password)
.text:00541320 crypt_file      proc near
.text:00541320
.text:00541320 Str             = dword ptr  4
.text:00541320 Filename        = dword ptr  8
.text:00541320 password        = dword ptr  0Ch
.text:00541320
\end{lstlisting}

Открыть файл и сообщить об ошибке в случае ошибки:

\begin{lstlisting}[style=customasmx86]
.text:00541320                 mov     eax, [esp+Str]
.text:00541324                 push    ebp
.text:00541325                 push    offset Mode     ; "rb"
.text:0054132A                 push    eax             ; Filename
.text:0054132B                 call    _fopen          ; open file
.text:00541330                 mov     ebp, eax
.text:00541332                 add     esp, 8
.text:00541335                 test    ebp, ebp
.text:00541337                 jnz     short loc_541348
.text:00541339                 push    offset Format   ; "Cannot open input file!\n"
.text:0054133E                 call    _printf
.text:00541343                 add     esp, 4
.text:00541346                 pop     ebp
.text:00541347                 retn
.text:00541348
.text:00541348 loc_541348:
\end{lstlisting}

\myindex{\CStandardLibrary!fseek()}
\myindex{\CStandardLibrary!ftell()}
Узнать размер файла используя \TT{fseek()}/\TT{ftell()}:

\lstinputlisting[style=customasmx86]{examples/qr9/1_RU}

Этот фрагмент кода вычисляет длину файла, выровненную по 64-байтной границе.
Это потому что этот алгоритм шифрования работает только с блоками размерами 64 байта.
Работает очень просто: разделить длину файла на 64, забыть об остатке, прибавить 1,
умножить на 64.
Следующий код удаляет остаток от деления, как если бы это значение уже было разделено 
на 64 и добавляет 64. Это почти то же самое.

\lstinputlisting[style=customasmx86]{examples/qr9/2_RU}

Выделить буфер с выровненным размером:

\begin{lstlisting}[style=customasmx86]
.text:00541373                 push    esi             ; Size
.text:00541374                 call    _malloc
\end{lstlisting}

\myindex{\CStandardLibrary!calloc()}
Вызвать memset(), т.е. очистить выделенный буфер\footnote{malloc() + memset() можно было бы 
заменить на calloc()}.

\lstinputlisting[style=customasmx86]{examples/qr9/3_RU}

Чтение файла используя стандартную функцию Си \TT{fread()}.

\begin{lstlisting}[style=customasmx86]
.text:00541392                 mov     eax, [esp+38h+Str]
.text:00541396                 push    eax             ; ElementSize
.text:00541397                 push    ebx             ; DstBuf
.text:00541398                 call    _fread          ; read file
.text:0054139D                 push    ebp             ; File
.text:0054139E                 call    _fclose
\end{lstlisting}

Вызов \TT{crypt()}. Эта функция берет на вход буфер, длину буфера (выровненную) и строку пароля.

\begin{lstlisting}[style=customasmx86]
.text:005413A3                 mov     ecx, [esp+44h+password]
.text:005413A7                 push    ecx             ; password
.text:005413A8                 push    esi             ; aligned size
.text:005413A9                 push    ebx             ; buffer
.text:005413AA                 call    crypt           ; do crypt
\end{lstlisting}

Создать выходной файл. Кстати, разработчик забыл вставить проверку, создался ли файл успешно!
Результат открытия файла, впрочем, проверяется.

\begin{lstlisting}[style=customasmx86]
.text:005413AF                 mov     edx, [esp+50h+Filename]
.text:005413B3                 add     esp, 40h
.text:005413B6                 push    offset aWb      ; "wb"
.text:005413BB                 push    edx             ; Filename
.text:005413BC                 call    _fopen
.text:005413C1                 mov     edi, eax
\end{lstlisting}

Теперь хэндл созданного файла в регистре \EDI. Записываем сигнатуру \q{QR9}.

\begin{lstlisting}[style=customasmx86]
.text:005413C3                 push    edi             ; File
.text:005413C4                 push    1               ; Count
.text:005413C6                 push    3               ; Size
.text:005413C8                 push    offset aQr9     ; "QR9"
.text:005413CD                 call    _fwrite         ; write file signature
\end{lstlisting}

Записываем настоящую длину файла (не выровненную):

\begin{lstlisting}[style=customasmx86]
.text:005413D2                 push    edi             ; File
.text:005413D3                 push    1               ; Count
.text:005413D5                 lea     eax, [esp+30h+Str]
.text:005413D9                 push    4               ; Size
.text:005413DB                 push    eax             ; Str
.text:005413DC                 call    _fwrite         ; write original file size
\end{lstlisting}

Записываем шифрованный буфер:

\begin{lstlisting}[style=customasmx86]
.text:005413E1                 push    edi             ; File
.text:005413E2                 push    1               ; Count
.text:005413E4                 push    esi             ; Size
.text:005413E5                 push    ebx             ; Str
.text:005413E6                 call    _fwrite         ; write encrypted file
\end{lstlisting}

Закрыть файл и освободить выделенный буфер:

\begin{lstlisting}[style=customasmx86]
.text:005413EB                 push    edi             ; File
.text:005413EC                 call    _fclose
.text:005413F1                 push    ebx             ; Memory
.text:005413F2                 call    _free
.text:005413F7                 add     esp, 40h
.text:005413FA                 pop     edi
.text:005413FB                 pop     esi
.text:005413FC                 pop     ebx
.text:005413FD                 pop     ebp
.text:005413FE                 retn
.text:005413FE crypt_file      endp
\end{lstlisting}

Переписанный на Си код:

\begin{lstlisting}[style=customc]
void crypt_file(char *fin, char* fout, char *pw)
{
	FILE *f;
	int flen, flen_aligned;
	BYTE *buf;

	f=fopen(fin, "rb");
	
	if (f==NULL)
	{
		printf ("Cannot open input file!\n");
		return;
	};

	fseek (f, 0, SEEK_END);
	flen=ftell (f);
	fseek (f, 0, SEEK_SET);

	flen_aligned=(flen&0xFFFFFFC0)+0x40;

	buf=(BYTE*)malloc (flen_aligned);
	memset (buf, 0, flen_aligned);

	fread (buf, flen, 1, f);

	fclose (f);

	crypt (buf, flen_aligned, pw);
	
	f=fopen(fout, "wb");

	fwrite ("QR9", 3, 1, f);
	fwrite (&flen, 4, 1, f);
	fwrite (buf, flen_aligned, 1, f);

	fclose (f);

	free (buf);
};
\end{lstlisting}

Процедура дешифрования почти такая же:

\begin{lstlisting}[style=customasmx86]
.text:00541400 ; int __cdecl decrypt_file(char *Filename, int, void *Src)
.text:00541400 decrypt_file    proc near
.text:00541400
.text:00541400 Filename        = dword ptr  4
.text:00541400 arg_4           = dword ptr  8
.text:00541400 Src             = dword ptr  0Ch
.text:00541400
.text:00541400                 mov     eax, [esp+Filename]
.text:00541404                 push    ebx
.text:00541405                 push    ebp
.text:00541406                 push    esi
.text:00541407                 push    edi
.text:00541408                 push    offset aRb      ; "rb"
.text:0054140D                 push    eax             ; Filename
.text:0054140E                 call    _fopen
.text:00541413                 mov     esi, eax
.text:00541415                 add     esp, 8
.text:00541418                 test    esi, esi
.text:0054141A                 jnz     short loc_54142E
.text:0054141C                 push    offset aCannotOpenIn_0 ; "Cannot open input file!\n"
.text:00541421                 call    _printf
.text:00541426                 add     esp, 4
.text:00541429                 pop     edi
.text:0054142A                 pop     esi
.text:0054142B                 pop     ebp
.text:0054142C                 pop     ebx
.text:0054142D                 retn
.text:0054142E
.text:0054142E loc_54142E:
.text:0054142E                 push    2               ; Origin
.text:00541430                 push    0               ; Offset
.text:00541432                 push    esi             ; File
.text:00541433                 call    _fseek
.text:00541438                 push    esi             ; File
.text:00541439                 call    _ftell
.text:0054143E                 push    0               ; Origin
.text:00541440                 push    0               ; Offset
.text:00541442                 push    esi             ; File
.text:00541443                 mov     ebp, eax
.text:00541445                 call    _fseek
.text:0054144A                 push    ebp             ; Size
.text:0054144B                 call    _malloc
.text:00541450                 push    esi             ; File
.text:00541451                 mov     ebx, eax
.text:00541453                 push    1               ; Count
.text:00541455                 push    ebp             ; ElementSize
.text:00541456                 push    ebx             ; DstBuf
.text:00541457                 call    _fread
.text:0054145C                 push    esi             ; File
.text:0054145D                 call    _fclose
\end{lstlisting}

Проверяем сигнатуру (первые 3 байта):

\begin{lstlisting}[style=customasmx86]
.text:00541462                 add     esp, 34h
.text:00541465                 mov     ecx, 3
.text:0054146A                 mov     edi, offset aQr9_0 ; "QR9"
.text:0054146F                 mov     esi, ebx
.text:00541471                 xor     edx, edx
.text:00541473                 repe cmpsb
.text:00541475                 jz      short loc_541489
\end{lstlisting}

Сообщить об ошибке если сигнатура отсутствует:

\begin{lstlisting}[style=customasmx86]
.text:00541477                 push    offset aFileIsNotCrypt ; "File is not encrypted!\n"
.text:0054147C                 call    _printf
.text:00541481                 add     esp, 4
.text:00541484                 pop     edi
.text:00541485                 pop     esi
.text:00541486                 pop     ebp
.text:00541487                 pop     ebx
.text:00541488                 retn
.text:00541489
.text:00541489 loc_541489:
\end{lstlisting}

Вызвать \TT{decrypt()}.

\begin{lstlisting}[style=customasmx86]
.text:00541489                 mov     eax, [esp+10h+Src]
.text:0054148D                 mov     edi, [ebx+3]
.text:00541490                 add     ebp, 0FFFFFFF9h
.text:00541493                 lea     esi, [ebx+7]
.text:00541496                 push    eax             ; Src
.text:00541497                 push    ebp             ; int
.text:00541498                 push    esi             ; int
.text:00541499                 call    decrypt
.text:0054149E                 mov     ecx, [esp+1Ch+arg_4]
.text:005414A2                 push    offset aWb_0    ; "wb"
.text:005414A7                 push    ecx             ; Filename
.text:005414A8                 call    _fopen
.text:005414AD                 mov     ebp, eax
.text:005414AF                 push    ebp             ; File
.text:005414B0                 push    1               ; Count
.text:005414B2                 push    edi             ; Size
.text:005414B3                 push    esi             ; Str
.text:005414B4                 call    _fwrite
.text:005414B9                 push    ebp             ; File
.text:005414BA                 call    _fclose
.text:005414BF                 push    ebx             ; Memory
.text:005414C0                 call    _free
.text:005414C5                 add     esp, 2Ch
.text:005414C8                 pop     edi
.text:005414C9                 pop     esi
.text:005414CA                 pop     ebp
.text:005414CB                 pop     ebx
.text:005414CC                 retn
.text:005414CC decrypt_file    endp
\end{lstlisting}

Переписанный на Си код:

\begin{lstlisting}[style=customc]
void decrypt_file(char *fin, char* fout, char *pw)
{
	FILE *f;
	int real_flen, flen;
	BYTE *buf;

	f=fopen(fin, "rb");
	
	if (f==NULL)
	{
		printf ("Cannot open input file!\n");
		return;
	};

	fseek (f, 0, SEEK_END);
	flen=ftell (f);
	fseek (f, 0, SEEK_SET);

	buf=(BYTE*)malloc (flen);

	fread (buf, flen, 1, f);

	fclose (f);

	if (memcmp (buf, "QR9", 3)!=0)
	{
		printf ("File is not encrypted!\n");
		return;
	};

	memcpy (&real_flen, buf+3, 4);

	decrypt (buf+(3+4), flen-(3+4), pw);
	
	f=fopen(fout, "wb");

	fwrite (buf+(3+4), real_flen, 1, f);

	fclose (f);

	free (buf);
};
\end{lstlisting}

OK, посмотрим глубже.

Функция \TT{crypt()}:

\begin{lstlisting}[style=customasmx86]
.text:00541260 crypt           proc near
.text:00541260
.text:00541260 arg_0           = dword ptr  4
.text:00541260 arg_4           = dword ptr  8
.text:00541260 arg_8           = dword ptr  0Ch
.text:00541260
.text:00541260                 push    ebx
.text:00541261                 mov     ebx, [esp+4+arg_0]
.text:00541265                 push    ebp
.text:00541266                 push    esi
.text:00541267                 push    edi
.text:00541268                 xor     ebp, ebp
.text:0054126A
.text:0054126A loc_54126A:
\end{lstlisting}

\myindex{x86!\Instructions!MOVSD}
Этот фрагмент кода копирует часть входного буфера во внутренний буфер, который мы позже назовем \q{cube64}.%

Длина в регистре \ECX. \TT{MOVSD} означает \IT{скопировать 32-битное слово}, так что, 16 32-битных слов
это как раз 64 байта.

\begin{lstlisting}[style=customasmx86]
.text:0054126A                 mov     eax, [esp+10h+arg_8]
.text:0054126E                 mov     ecx, 10h
.text:00541273                 mov     esi, ebx   ; EBX is pointer within input buffer
.text:00541275                 mov     edi, offset cube64
.text:0054127A                 push    1
.text:0054127C                 push    eax
.text:0054127D                 rep movsd
\end{lstlisting}

Вызвать \TT{rotate\_all\_with\_password()}:

\begin{lstlisting}[style=customasmx86]
.text:0054127F                 call    rotate_all_with_password
\end{lstlisting}

Скопировать зашифрованное содержимое из \q{cube64} назад в буфер:

\begin{lstlisting}[style=customasmx86]
.text:00541284                 mov     eax, [esp+18h+arg_4]
.text:00541288                 mov     edi, ebx
.text:0054128A                 add     ebp, 40h
.text:0054128D                 add     esp, 8
.text:00541290                 mov     ecx, 10h
.text:00541295                 mov     esi, offset cube64
.text:0054129A                 add     ebx, 40h  ; add 64 to input buffer pointer
.text:0054129D                 cmp     ebp, eax  ; EBP = amount of encrypted data.
.text:0054129F                 rep movsd
\end{lstlisting}

Если \EBP не больше чем длина во входном аргументе, тогда переходим к следующему блоку.

\begin{lstlisting}[style=customasmx86]
.text:005412A1                 jl      short loc_54126A
.text:005412A3                 pop     edi
.text:005412A4                 pop     esi
.text:005412A5                 pop     ebp
.text:005412A6                 pop     ebx
.text:005412A7                 retn
.text:005412A7 crypt           endp
\end{lstlisting}

Реконструированная функция \TT{crypt()}:

\begin{lstlisting}[style=customc]
void crypt (BYTE *buf, int sz, char *pw)
{
	int i=0;
	
	do
	{
		memcpy (cube, buf+i, 8*8);
		rotate_all (pw, 1);
		memcpy (buf+i, cube, 8*8);
		i+=64;
	}
	while (i<sz);
};
\end{lstlisting}

OK, углубимся в функцию \TT{rotate\_all\_with\_password()}. Она берет на вход два аргумента: 
строку пароля и число.
В функции \TT{crypt()}, число 1 используется и в \TT{decrypt()} \\
(где \TT{rotate\_all\_with\_password()} функция вызывается также), число 3.

\begin{lstlisting}[style=customasmx86]
.text:005411B0 rotate_all_with_password proc near
.text:005411B0
.text:005411B0 arg_0           = dword ptr  4
.text:005411B0 arg_4           = dword ptr  8
.text:005411B0
.text:005411B0                 mov     eax, [esp+arg_0]
.text:005411B4                 push    ebp
.text:005411B5                 mov     ebp, eax
\end{lstlisting}

Проверяем символы в пароле. Если это ноль, выходим:

\begin{lstlisting}[style=customasmx86]
.text:005411B7                 cmp     byte ptr [eax], 0
.text:005411BA                 jz      exit
.text:005411C0                 push    ebx
.text:005411C1                 mov     ebx, [esp+8+arg_4]
.text:005411C5                 push    esi
.text:005411C6                 push    edi
.text:005411C7
.text:005411C7 loop_begin:
\end{lstlisting}

\myindex{\CStandardLibrary!tolower()}
Вызываем \TT{tolower()}, стандартную функцию Си.

\begin{lstlisting}[style=customasmx86]
.text:005411C7                 movsx   eax, byte ptr [ebp+0]
.text:005411CB                 push    eax             ; C
.text:005411CC                 call    _tolower
.text:005411D1                 add     esp, 4
\end{lstlisting}

Хмм, если пароль содержит символ не из латинского алфавита, он пропускается!
Действительно, если мы запускаем утилиту для шифрования используя символы не латинского алфавита, 
похоже, они просто игнорируются.

\begin{lstlisting}[style=customasmx86]
.text:005411D4                 cmp     al, 'a'
.text:005411D6                 jl      short next_character_in_password
.text:005411D8                 cmp     al, 'z'
.text:005411DA                 jg      short next_character_in_password
.text:005411DC                 movsx   ecx, al
\end{lstlisting}

Отнимем значение \q{a} (97) от символа.

\begin{lstlisting}[style=customasmx86]
.text:005411DF                 sub     ecx, 'a'  ; 97
\end{lstlisting}

После вычитания, тут будет 0 для \q{a}, 1 для \q{b}, и так далее. И 25 для \q{z}.

\begin{lstlisting}[style=customasmx86]
.text:005411E2                 cmp     ecx, 24
.text:005411E5                 jle     short skip_subtracting
.text:005411E7                 sub     ecx, 24
\end{lstlisting}

Похоже, символы \q{y} и \q{z} также исключительные.
После этого фрагмента кода, \q{y} становится 0, а \q{z} ~--- 1.
Это значит, что 26 латинских букв становятся значениями в интервале 0..23, (всего 24).

\begin{lstlisting}[style=customasmx86]
.text:005411EA
.text:005411EA skip_subtracting:                       ; CODE XREF: rotate_all_with_password+35
\end{lstlisting}

Это, на самом деле, деление через умножение.
Читайте об этом больше в секции \q{\DivisionByMultSectionName}~(\myref{sec:divisionbymult}).

Это код, на самом деле, делит значение символа пароля на 3.

% TODO1: add Mathematica calculations
\begin{lstlisting}[style=customasmx86]
.text:005411EA                 mov     eax, 55555556h
.text:005411EF                 imul    ecx
.text:005411F1                 mov     eax, edx
.text:005411F3                 shr     eax, 1Fh
.text:005411F6                 add     edx, eax
.text:005411F8                 mov     eax, ecx
.text:005411FA                 mov     esi, edx
.text:005411FC                 mov     ecx, 3
.text:00541201                 cdq
.text:00541202                 idiv    ecx
\end{lstlisting}

\EDX --- остаток от деления.

\lstinputlisting[style=customasmx86]{examples/qr9/4_RU}

Если остаток 2, вызываем \TT{rotate3()}. 
\EDX это второй аргумент функции \TT{rotate\_all\_with\_password()}. 
Как мы уже заметили, 1 это для шифрования, 3 для дешифрования.
Так что здесь цикл, функции rotate1/2/3 будут вызываться столько же раз, сколько значение переменной
в первом аргументе.

\begin{lstlisting}[style=customasmx86]
.text:00541215 call_rotate3:
.text:00541215                 push    esi
.text:00541216                 call    rotate3
.text:0054121B                 add     esp, 4
.text:0054121E                 dec     edi
.text:0054121F                 jnz     short call_rotate3
.text:00541221                 jmp     short next_character_in_password
.text:00541223
.text:00541223 call_rotate2:
.text:00541223                 test    ebx, ebx
.text:00541225                 jle     short next_character_in_password
.text:00541227                 mov     edi, ebx
.text:00541229
.text:00541229 loc_541229:
.text:00541229                 push    esi
.text:0054122A                 call    rotate2
.text:0054122F                 add     esp, 4
.text:00541232                 dec     edi
.text:00541233                 jnz     short loc_541229
.text:00541235                 jmp     short next_character_in_password
.text:00541237
.text:00541237 call_rotate1:
.text:00541237                 test    ebx, ebx
.text:00541239                 jle     short next_character_in_password
.text:0054123B                 mov     edi, ebx
.text:0054123D
.text:0054123D loc_54123D:
.text:0054123D                 push    esi
.text:0054123E                 call    rotate1
.text:00541243                 add     esp, 4
.text:00541246                 dec     edi
.text:00541247                 jnz     short loc_54123D
.text:00541249
\end{lstlisting}

Достать следующий символ из строки пароля.

\begin{lstlisting}[style=customasmx86]
.text:00541249 next_character_in_password:
.text:00541249                 mov     al, [ebp+1]
\end{lstlisting}

\glslink{increment}{Инкремент} указателя на символ в строке пароля:

\begin{lstlisting}[style=customasmx86]
.text:0054124C                 inc     ebp
.text:0054124D                 test    al, al
.text:0054124F                 jnz     loop_begin
.text:00541255                 pop     edi
.text:00541256                 pop     esi
.text:00541257                 pop     ebx
.text:00541258
.text:00541258 exit:
.text:00541258                 pop     ebp
.text:00541259                 retn
.text:00541259 rotate_all_with_password endp
\end{lstlisting}

Реконструированный код на Си:

\begin{lstlisting}[style=customc]
void rotate_all (char *pwd, int v)
{
	char *p=pwd;

	while (*p)
	{
		char c=*p;
		int q;

		c=tolower (c);

		if (c>='a' && c<='z')
		{
			q=c-'a';
			if (q>24)
				q-=24;

			int quotient=q/3;
			int remainder=q % 3;

			switch (remainder)
			{
			case 0: for (int i=0; i<v; i++) rotate1 (quotient); break;
			case 1: for (int i=0; i<v; i++) rotate2 (quotient); break;
			case 2: for (int i=0; i<v; i++) rotate3 (quotient); break;
			};
		};

		p++;
	};
};
\end{lstlisting}

Углубимся еще дальше и исследуем функции rotate1/2/3.
Каждая функция вызывает еще две.
В итоге мы назовем их \TT{set\_bit()} и \TT{get\_bit()}.

Начнем с \TT{get\_bit()}:

\begin{lstlisting}[style=customasmx86]
.text:00541050 get_bit         proc near
.text:00541050
.text:00541050 arg_0           = dword ptr  4
.text:00541050 arg_4           = dword ptr  8
.text:00541050 arg_8           = byte ptr  0Ch
.text:00541050
.text:00541050                 mov     eax, [esp+arg_4]
.text:00541054                 mov     ecx, [esp+arg_0]
.text:00541058                 mov     al, cube64[eax+ecx*8]
.text:0054105F                 mov     cl, [esp+arg_8]
.text:00541063                 shr     al, cl
.text:00541065                 and     al, 1
.text:00541067                 retn
.text:00541067 get_bit         endp
\end{lstlisting}

\dots иными словами: подсчитать индекс в массиве cube64: \IT{arg\_4 + arg\_0 * 8}.
Затем сдвинуть байт из массива вправо на количество бит заданных в arg\_8. 
Изолировать самый младший бит и вернуть его

Посмотрим другую функцию, \TT{set\_bit()}:

\begin{lstlisting}[style=customasmx86]
.text:00541000 set_bit         proc near
.text:00541000
.text:00541000 arg_0           = dword ptr  4
.text:00541000 arg_4           = dword ptr  8
.text:00541000 arg_8           = dword ptr  0Ch
.text:00541000 arg_C           = byte ptr  10h
.text:00541000
.text:00541000                 mov     al, [esp+arg_C]
.text:00541004                 mov     ecx, [esp+arg_8]
.text:00541008                 push    esi
.text:00541009                 mov     esi, [esp+4+arg_0]
.text:0054100D                 test    al, al
.text:0054100F                 mov     eax, [esp+4+arg_4]
.text:00541013                 mov     dl, 1
.text:00541015                 jz      short loc_54102B
\end{lstlisting}

\TT{DL} тут равно 1. Сдвигаем эту единицу на количество, указанное в arg\_8. Например, если в arg\_8 число 4,
тогда значение в \TT{DL} станет 0x10 или 1000b в двоичной системе счисления.

\begin{lstlisting}[style=customasmx86]
.text:00541017                 shl     dl, cl
.text:00541019                 mov     cl, cube64[eax+esi*8]
\end{lstlisting}

Вытащить бит из массива и явно выставить его. % TODO1: rewrite

\begin{lstlisting}[style=customasmx86]
.text:00541020                 or      cl, dl
\end{lstlisting}

Сохранить его назад: % TODO1: rewrite

\begin{lstlisting}[style=customasmx86]
.text:00541022                 mov     cube64[eax+esi*8], cl
.text:00541029                 pop     esi
.text:0054102A                 retn
.text:0054102B
.text:0054102B loc_54102B:
.text:0054102B                 shl     dl, cl
\end{lstlisting}

Если arg\_C не ноль\dots

\begin{lstlisting}[style=customasmx86]
.text:0054102D                 mov     cl, cube64[eax+esi*8]
\end{lstlisting}

\myindex{x86!\Instructions!NOT}
\dots инвертировать DL. Например, если состояние DL после сдвига 0x10 или 1000b в двоичной системе,
здесь будет 0xEF после инструкции \NOT или 11101111b в двоичной системе.

\begin{lstlisting}[style=customasmx86]
.text:00541034                 not     dl
\end{lstlisting}

Эта инструкция сбрасывает бит, иными словами, она сохраняет все биты в \TT{CL} которые также
выставлены в \TT{DL} кроме тех в \TT{DL}, что были сброшены. Это значит, что если в \TT{DL}, например,
11101111b в двоичной системе, все биты будут сохранены кроме пятого (считая с младшего бита).

\begin{lstlisting}[style=customasmx86]
.text:00541036                 and     cl, dl
\end{lstlisting}

Сохранить его назад

\begin{lstlisting}[style=customasmx86]
.text:00541038                 mov     cube64[eax+esi*8], cl
.text:0054103F                 pop     esi
.text:00541040                 retn
.text:00541040 set_bit         endp
\end{lstlisting}

Это почти то же самое что и \TT{get\_bit()}, кроме того, что если arg\_C ноль, тогда функция сбрасывает
указанный бит в массиве, либо же, в противном случае, выставляет его в 1.

Мы также знаем что размер массива 64. Первые два аргумента и у \TT{set\_bit()} и у \TT{get\_bit()}
могут быть представлены как двумерные координаты. Таким образом, массив ~--- это матрица 8*8.

Представление на Си всего того, что мы уже знаем:

\begin{lstlisting}[style=customc]
#define IS_SET(flag, bit)       ((flag) & (bit))
#define SET_BIT(var, bit)       ((var) |= (bit))
#define REMOVE_BIT(var, bit)    ((var) &= ~(bit))

static BYTE cube[8][8];

void set_bit (int x, int y, int shift, int bit)
{
	if (bit)
		SET_BIT (cube[x][y], 1<<shift);
	else
		REMOVE_BIT (cube[x][y], 1<<shift);
};

bool get_bit (int x, int y, int shift)
{
	if ((cube[x][y]>>shift)&1==1)
		return 1;
	return 0;
};
\end{lstlisting}

Теперь вернемся к функциям rotate1/2/3.

\begin{lstlisting}[style=customasmx86]
.text:00541070 rotate1         proc near
.text:00541070
\end{lstlisting}

Выделение внутреннего массива размером 64 байта в локальном стеке:

\begin{lstlisting}[style=customasmx86]
.text:00541070 internal_array_64= byte ptr -40h
.text:00541070 arg_0           = dword ptr  4
.text:00541070
.text:00541070                 sub     esp, 40h
.text:00541073                 push    ebx
.text:00541074                 push    ebp
.text:00541075                 mov     ebp, [esp+48h+arg_0]
.text:00541079                 push    esi
.text:0054107A                 push    edi
.text:0054107B                 xor     edi, edi        ; EDI is loop1 counter
\end{lstlisting}

\EBX указывает на внутренний массив

\begin{lstlisting}[style=customasmx86]
.text:0054107D                 lea     ebx, [esp+50h+internal_array_64]
.text:00541081
\end{lstlisting}

Здесь два вложенных цикла:

\lstinputlisting[style=customasmx86]{examples/qr9/5_RU}

Мы видим, что оба счетчика циклов в интервале 0..7. 
Также, они используются как первый и второй аргумент \TT{get\_bit()}.
Третий аргумент \TT{get\_bit()} это единственный аргумент \TT{rotate1()}. 
То что возвращает \TT{get\_bit()} будет сохранено во внутреннем массиве.

Снова приготовить указатель на внутренний массив:

\lstinputlisting[style=customasmx86]{examples/qr9/6_RU}

\dots этот код помещает содержимое из внутреннего массива в глобальный массив cube используя функцию 
\TT{set\_bit()}, \IT{но}, в обратном порядке!
Теперь счетчик первого цикла в интервале 7 до 0, уменьшается на 1 на каждой итерации!

Представление кода на Си выглядит так:

\begin{lstlisting}[style=customc]
void rotate1 (int v)
{
	bool tmp[8][8]; // internal array
	int i, j;

	for (i=0; i<8; i++)
		for (j=0; j<8; j++)
			tmp[i][j]=get_bit (i, j, v);

	for (i=0; i<8; i++)
		for (j=0; j<8; j++)
			set_bit (j, 7-i, v, tmp[x][y]);
};
\end{lstlisting}

Не очень понятно, но если мы посмотрим в функцию \TT{rotate2()}:

\lstinputlisting[style=customasmx86]{examples/qr9/7_RU}

\IT{Почти} то же самое, за исключением иного порядка аргументов в \TT{get\_bit()} и \TT{set\_bit()}.
Перепишем это на Си-подобный код:

\begin{lstlisting}[style=customc]
void rotate2 (int v)
{
	bool tmp[8][8]; // internal array
	int i, j;

	for (i=0; i<8; i++)
		for (j=0; j<8; j++)
			tmp[i][j]=get_bit (v, i, j);

	for (i=0; i<8; i++)
		for (j=0; j<8; j++)
			set_bit (v, j, 7-i, tmp[i][j]);
};
\end{lstlisting}

Перепишем так же функцию \TT{rotate3()}:

\begin{lstlisting}[style=customasmx86]
void rotate3 (int v)
{
	bool tmp[8][8];
	int i, j;

	for (i=0; i<8; i++)
		for (j=0; j<8; j++)
			tmp[i][j]=get_bit (i, v, j);

	for (i=0; i<8; i++)
		for (j=0; j<8; j++)
			set_bit (7-j, v, i, tmp[i][j]);
};
\end{lstlisting}

Теперь всё проще. Если мы представим cube64 как трехмерный куб 8*8*8, где каждый элемент это бит,
то \TT{get\_bit()} и \TT{set\_bit()} просто берут на вход координаты бита.

Функции rotate1/2/3 просто поворачивают все биты на определенной плоскости.
Три функции, каждая на каждую сторону куба и аргумент \TT{v} выставляет плоскость в интервале 0..7

Может быть, автор алгоритма думал о кубике Рубика 8*8*8

\footnote{\href{http://go.yurichev.com/17115}{wikipedia}}?!

Да, действительно.

Рассмотрим функцию \TT{decrypt()}, вот её переписанная версия:%

\begin{lstlisting}[style=customc]
void decrypt (BYTE *buf, int sz, char *pw)
{
	char *p=strdup (pw);
	strrev (p);
	int i=0;

	do
	{
		memcpy (cube, buf+i, 8*8);
		rotate_all (p, 3);
		memcpy (buf+i, cube, 8*8);
		i+=64;
	}
	while (i<sz);
	
	free (p);
};
\end{lstlisting}

Почти то же самое что и crypt(), \IT{но} строка пароля разворачивается стандартной функцией Си
strrev() \footnote{\href{http://go.yurichev.com/17249}{MSDN}}
и \TT{rotate\_all()} вызывается с аргументом 3.

Это значит, что, в случае дешифровки, rotate1/2/3 будут вызываться трижды.

Это почти кубик Рубика!
Если вы хотите вернуть его состояние назад, делайте то же самое в обратном порядке и направлении!
Чтобы вернуть эффект от поворота плоскости по часовой стрелке, нужно повернуть её же против 
часовой стрелки, либо же трижды по часовой стрелке.

\TT{rotate1()}, вероятно, поворот \q{лицевой} плоскости. 
\TT{rotate2()}, вероятно, поворот \q{верхней} плоскости.
\TT{rotate3()}, вероятно, поворот \q{левой} плоскости.

Вернемся к ядру функции \TT{rotate\_all()}

\begin{lstlisting}[style=customc]
q=c-'a';
if (q>24)
	q-=24;

int quotient=q/3; // in range 0..7
int remainder=q % 3;

switch (remainder)
{
    case 0: for (int i=0; i<v; i++) rotate1 (quotient); break; // front
    case 1: for (int i=0; i<v; i++) rotate2 (quotient); break; // top
    case 2: for (int i=0; i<v; i++) rotate3 (quotient); break; // left
};
\end{lstlisting}

Так понять проще: каждый символ пароля определяет сторону (одну из трех) и плоскость (одну из восьми).
3*8 = 24, вот почему два последних символа латинского алфавита переопределяются так чтобы алфавит состоял
из 24-х элементов.

Алгоритм очевидно слаб: в случае коротких паролей, в бинарном редакторе файлов можно будет увидеть, 
что в зашифрованных файлах остались незашифрованные символы.

Весь исходный код в реконструированном виде:

\lstinputlisting[style=customc]{examples/qr9/qr9.cpp}

