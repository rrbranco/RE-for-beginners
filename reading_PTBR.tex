\chapter{Livros/blogs que valem a leitura}

\section{Livros e outros materiais}

\subsection{Engenharia Reversa}

\input{RE_books}

\subsection{Windows}

\input{Win_reading}

\subsection{\CCpp}

\input{CCppBooks}

\subsection{x86 / x86-64}

\label{x86_manuals}
\begin{itemize}
\item Manuais da Intel\footnote{\AlsoAvailableAs \url{http://www.intel.com/content/www/us/en/processors/architectures-software-developer-manuals.html}}

\item Manuais da AMD\footnote{\AlsoAvailableAs \url{http://developer.amd.com/resources/developer-guides-manuals/}}

\item \AgnerFog{}\footnote{\AlsoAvailableAs \url{http://agner.org/optimize/microarchitecture.pdf}}

\item \AgnerFogCC{}\footnote{\AlsoAvailableAs \url{http://www.agner.org/optimize/calling_conventions.pdf}}

\item \IntelOptimization

\item \AMDOptimization
\end{itemize}

Alguns materiais desatualizados, mas valem a pena ler:

\MAbrash\footnote{\AlsoAvailableAs \url{https://github.com/jagregory/abrash-black-book}}
(Ele é famoso pelo trabalho de optimização em baixo nível para projetos como Windows NT 3.1 e id Quake).

\subsection{ARM}

\begin{itemize}
\item Manuais ARM\footnote{\AlsoAvailableAs \url{http://infocenter.arm.com/help/index.jsp?topic=/com.arm.doc.subset.architecture.reference/index.html}}

\item \ARMSevenRef

\item \ARMSixFourRefURL

\item \ARMCookBook\footnote{\AlsoAvailableAs \url{http://go.yurichev.com/17273}}
\end{itemize}

\subsection{Java}

\JavaBook.

\subsection{UNIX}

\TAOUP

\subsection{Programação em Geral}

\begin{itemize}

\item \RobPikePractice

\item \HenryWarren.

\item (For hard-core geeks with computer science and mathematical background) Donald E. Knuth, \IT{The Art of Computer Programming}.

\end{itemize}

% subsection:
\input{crypto_reading}

