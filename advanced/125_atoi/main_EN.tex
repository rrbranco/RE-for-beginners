\section{String to number conversion (atoi())}

\myindex{\CStandardLibrary!atoi()}

Let's try to reimplement the standard atoi() C function.

\subsection{Simple example}

Here is the simplest possible way to read a number represented in \ac{ASCII} encoding.

It's not error-prone: a character other than a digit leads to incorrect result.

\lstinputlisting[style=customc]{\CURPATH/atoi.c}

So what the algorithm does is just reading digits from left to right.

The zero \ac{ASCII} character is subtracted from each digit. 

The digits from \q{0} to \q{9} are consecutive in the \ac{ASCII} table, so 
we do not even need to know the exact value of the \q{0} character.

All we have to know is that \q{0} minus \q{0} is 0, \q{9} minus \q{0}'is 9 and so on.

Subtracting \q{0} from each character results in a number from 0 to 9 inclusive.

Any other character leads to an incorrect result, of course!

Each digit has to be added to the final result (in variable \q{rt}), but the final result
is also multiplied by 10 at each digit.

In other words, the result is shifted left by one position in decimal form on each iteration.

The last digit is added, but there is no shift.

\subsubsection{\Optimizing MSVC 2013 x64}

\lstinputlisting[caption=\Optimizing MSVC 2013 x64,style=customasmx86]{\CURPATH/atoi.asm.MSVC2013.x64.Ox_EN}

A character can be loaded in two places: the first character and all subsequent characters.
This is arranged so for loop regrouping.
\myindex{x86!\Instructions!LEA}

There is no instruction for multiplication by 10, two LEA instruction do this instead.
\myindex{x86!\Instructions!ADD}
\myindex{x86!\Instructions!SUB}

MSVC sometimes uses the ADD instruction with a negative constant instead of SUB.
This is the case.

It's very hard to say why this is better then SUB.
But MSVC does this often.

\subsubsection{\Optimizing GCC 4.9.1 x64}

\Optimizing GCC 4.9.1 is more concise, but there is one redundant RET instruction at the end.
One would be enough.

\lstinputlisting[caption=\Optimizing GCC 4.9.1 x64,style=customasmx86]{\CURPATH/atoi.s.GCC491.O3.x64_EN}

\subsubsection{\OptimizingKeilVI (\ARMMode)}

\lstinputlisting[caption=\OptimizingKeilVI (\ARMMode),style=customasmARM]{\CURPATH/atoi.s.ARM.O3_EN}

\subsubsection{\OptimizingKeilVI (\ThumbMode)}

\lstinputlisting[caption=\OptimizingKeilVI (\ThumbMode),style=customasmARM]{\CURPATH/atoi.s.thumb.O3_EN}

Interestingly, from school mathematics we may recall that the order of addition and 
subtraction operations doesn't matter.

That's our case: first, the $rt*10 - '0'$ expression is computed, then the input character value 
is added to it.

Indeed, the result is the same, but the compiler did some regrouping.

\subsubsection{\Optimizing GCC 4.9.1 ARM64}

The ARM64 compiler can use the pre-increment instruction suffix:

\lstinputlisting[caption=\Optimizing GCC 4.9.1 ARM64,style=customasmARM]{\CURPATH/atoi.s.GCC49.ARM64.O3_EN}

\subsection{A slightly advanced example}

My new code snippet is more advanced, now it checks for the \q{minus} sign at the first character
and reports an error if a non-digit has been found in the input string:

\lstinputlisting[style=customc]{\CURPATH/atoi2.c}

\subsubsection{\Optimizing GCC 4.9.1 x64}

\lstinputlisting[caption=\Optimizing GCC 4.9.1 x64,style=customasmx86]{\CURPATH/atoi2.s.GCC491.O3.x64_EN}

\myindex{x86!\Instructions!NEG}

If the \q{minus} sign has been encountered at the string start, the \INS{NEG} instruction is to be executed at the end.
It just negates the number.

\label{one_comparison_instead_of_two}
There is one more thing that needs mentioning.

How would a common programmer check if the character is not a digit?
Just how we have it in the source code:

\begin{lstlisting}[style=customc]
if (*s<'0' || *s>'9')
    ...
\end{lstlisting}

There are two comparison operations.

What is interesting is that we can replace both operations by single one:
just subtract \q{0} from character value,

treat result as unsigned value (this is important) and check if it's greater than 9.

For example, let's say that the user input contains the dot character (\q{.}) which has \ac{ASCII} code 46.
$46-48=-2$ if we treat the result as a signed number.

Indeed, the dot character is located two places earlier than the \q{0} character in the \ac{ASCII} table.
But it is \TT{0xFFFFFFFE} (4294967294) 
if we treat the result as an unsigned value, and that's definitely bigger than 9!

The compilers do this often, so it's important to recognize these tricks.

Another example of it in this book: 
\myref{toupper_one_comparison}.

\Optimizing MSVC 2013 x64 does the same tricks.

\subsubsection{\OptimizingKeilVI (\ARMMode)}

\lstinputlisting[caption=\OptimizingKeilVI (\ARMMode),numbers=left,style=customasmARM]{\CURPATH/atoi2.s.ARM.O3_EN}

There is no NEG instruction in 32-bit ARM, so the \q{Reverse Subtraction} operation (line 31) 
is used here.

It is triggered if the result of the \INS{CMP} instruction (at line 29) has been \q{Not Equal} (hence \TT{-NE} suffix).
\myindex{ARM!\Instructions!RSB}

So what \INS{RSBNE} does is to subtract the resulting value from 0.

It works just like the regular subtraction operation, but swaps operands.

Subtracting any number from 0 results in negation: $0-x=-x$.

Thumb mode code is mostly the same.

\myindex{ARM!\Instructions!NEG}
GCC 4.9 for ARM64 can use the NEG instruction, which is available in ARM64.

\subsection{\Exercise{}}

\myindex{Fuzzing}
Oh, by the way, security researchers deals often with unpredictable behavior of program while handling of incorrect data.

For example, while fuzzing.
As an exercise, you may try to enter non-digit characters and see what happens.

Try to explain, what happened and why.



