\section{More about pointers}
\myindex{\CLanguageElements!\Pointers}
\label{label_pointers}

\epigraph{The way C handles pointers, for example, was a brilliant innovation;
it solved a lot of problems that we had before in data structuring and
made the programs look good afterwards.}{Donald Knuth, interview (1993)}

For those, who still have hard time understanding \CCpp pointers, here are more examples.
Some of them are weird and serves only demonstration purpose:
use them in production code only if you really know what you're doing.

\subsection{Working with addresses instead of pointers}

Pointer is just an address in memory. But why we write \TT{char* string} instead of something like \TT{address string}?
Pointer variable is supplied with a type of the value to which pointer points.
So then compiler will able to catch data typization bugs during compilation.

To be pedantic, data typing in programming languages is all about preventing bugs and self-documentation.
It's possible to use maybe two of data types like \IT{int} (or \IT{int64\_t}) and byte---these are the only types which are available to assembly language programmers.
But it's just very hard task to write big and practical assembly programs without nasty bugs.
Any small typo can lead to hard-to-find bug.

Data type information is absent in a compiled code (and this is one of the main problems for decompilers),
and I can demonstrate this.

This is what sane C/C++ programmer can write:

\begin{lstlisting}[style=customc]
#include <stdio.h>
#include <stdint.h>

void print_string (char *s)
{
	printf ("(address: 0x%llx)\n", s);
	printf ("%s\n", s);
};

int main()
{
	char *s="Hello, world!";

	print_string (s);
};
\end{lstlisting}

This is what I can write:

\begin{lstlisting}[style=customc]
#include <stdio.h>
#include <stdint.h>

void print_string (uint64_t address)
{
	printf ("(address: 0x%llx)\n", address);
	puts ((char*)address);
};

int main()
{
	char *s="Hello, world!";

	print_string ((uint64_t)s);
};
\end{lstlisting}

I use \IT{uint64\_t} because I run this example on Linux x64. \IT{int} would work for 32-bit \ac{OS}-es.
First, a pointer to character (the very first in the greeting string) is casted to \IT{uint64\_t}, then it's passed further.
\TT{print\_string()} function casts back incoming \IT{uint64\_t} value into pointer to a character.

What is interesting is that GCC 4.8.4 produces identical assembly output for both versions:

\begin{lstlisting}
gcc 1.c -S -masm=intel -O3 -fno-inline
\end{lstlisting}

\begin{lstlisting}[style=customasmx86]
.LC0:
	.string	"(address: 0x%llx)\n"
print_string:
	push	rbx
	mov	rdx, rdi
	mov	rbx, rdi
	mov	esi, OFFSET FLAT:.LC0
	mov	edi, 1
	xor	eax, eax
	call	__printf_chk
	mov	rdi, rbx
	pop	rbx
	jmp	puts
.LC1:
	.string	"Hello, world!"
main:
	sub	rsp, 8
	mov	edi, OFFSET FLAT:.LC1
	call	print_string
	add	rsp, 8
	ret
\end{lstlisting}

(I've removed all insignificant GCC directives.)

I also tried UNIX \IT{diff} utility and it shows no differences at all.

Let's continue to abuse C/C++ programming traditions heavily.
Someone may write this:

\begin{lstlisting}[style=customc]
#include <stdio.h>
#include <stdint.h>

uint8_t load_byte_at_address (uint8_t* address)
{
	return *address;
	//this is also possible: return address[0]; 
};

void print_string (char *s)
{
	char* current_address=s;
	while (1)
	{
		char current_char=load_byte_at_address(current_address);
		if (current_char==0)
			break;
		printf ("%c", current_char);
		current_address++;
	};
};

int main()
{
	char *s="Hello, world!";

	print_string (s);
};
\end{lstlisting}

It can be rewritten like this:

\begin{lstlisting}[style=customc]
#include <stdio.h>
#include <stdint.h>

uint8_t load_byte_at_address (uint64_t address)
{
	return *(uint8_t*)address;
	//this is also possible: return address[0]; 
};

void print_string (uint64_t address)
{
	uint64_t current_address=address;
	while (1)
	{
		char current_char=load_byte_at_address(current_address);
		if (current_char==0)
			break;
		printf ("%c", current_char);
		current_address++;
	};
};

int main()
{
	char *s="Hello, world!";

	print_string ((uint64_t)s);
};
\end{lstlisting}

Both source codes resulting in the same assembly output:

\begin{lstlisting}
gcc 1.c -S -masm=intel -O3 -fno-inline
\end{lstlisting}

\begin{lstlisting}[style=customasmx86]
load_byte_at_address:
	movzx	eax, BYTE PTR [rdi]
	ret
print_string:
.LFB15:
	push	rbx
	mov	rbx, rdi
	jmp	.L4
.L7:
	movsx	edi, al
	add	rbx, 1
	call	putchar
.L4:
	mov	rdi, rbx
	call	load_byte_at_address
	test	al, al
	jne	.L7
	pop	rbx
	ret
.LC0:
	.string	"Hello, world!"
main:
	sub	rsp, 8
	mov	edi, OFFSET FLAT:.LC0
	call	print_string
	add	rsp, 8
	ret
\end{lstlisting}

(I have also removed all insignificant GCC directives.)

No difference: C/C++ pointers are essentially addresses, but supplied with type information, in order to prevent possible mistakes at the time of compilation.
Types are not checked during runtime---it would be huge (and unneeded) overhead.


\subsection{Passing values as pointers; tagged unions}

Here is an example on how to pass values in pointers:

\begin{lstlisting}[label=unsigned_multiply_C,style=customc]
#include <stdio.h>
#include <stdint.h>

uint64_t multiply1 (uint64_t a, uint64_t b)
{
	return a*b;
};

uint64_t* multiply2 (uint64_t *a, uint64_t *b)
{
	return (uint64_t*)((uint64_t)a*(uint64_t)b);
};

int main()
{
	printf ("%d\n", multiply1(123, 456));
	printf ("%d\n", (uint64_t)multiply2((uint64_t*)123, (uint64_t*)456));
};
\end{lstlisting}

It works smoothly and GCC 4.8.4 compiles both multiply1() and multiply2() functions identically!

\begin{lstlisting}[label=unsigned_multiply_lst,style=customasmx86]
multiply1:
	mov	rax, rdi
	imul	rax, rsi
	ret

multiply2:
	mov	rax, rdi
	imul	rax, rsi
	ret
\end{lstlisting}

As long as you do not dereference pointer (in other words, you don't read any data from the address stored in pointer), everything will work fine.
Pointer is a variable which can store anything, like usual variable.

\myindex{x86!\Instructions!MUL}
\myindex{x86!\Instructions!IMUL}
Signed multiplication instruction (\IMUL) is used here instead of unsigned one (\MUL), read more about it here:
\ref{IMUL_over_MUL}.

\myindex{Tagged pointers}
By the way, it's well-known hack to abuse pointers a little called \IT{tagged pointers}.
In short, if all your pointers points to blocks of memory with size of, let's say, 16 bytes (or it is always aligned on 16-byte boundary), 4 lowest bits of pointer is always zero bits and this space
can be used somehow.
\myindex{LISP}
It's very popular in LISP compilers and interpreters.
They store cell/object type in these unused bits, this can save some memory.
Even more, you can judge about cell/object type using just pointer, with no additional memory access.
Read more about it: \InSqBrackets{\CNotes 1.3}.

% TODO Example of tagged ptrs here


\subsection{Pointers abuse in Windows kernel}

The resource section of PE executable file in Windows OS is a section containing pictures, icons, strings, etc.
Early Windows versions allowed to address resources only by IDs, but then Microsoft added a way to address them using strings.

So then it would be possible to pass ID or string to 
\href{https://msdn.microsoft.com/en-us/library/windows/desktop/ms648042%28v=vs.85%29.aspx}{FindResource()} function.
Which is declared like this:

\myindex{win32!FindResource()}

\begin{lstlisting}[style=customc]
HRSRC WINAPI FindResource(
  _In_opt_ HMODULE hModule,
  _In_     LPCTSTR lpName,
  _In_     LPCTSTR lpType
);
\end{lstlisting}

\IT{lpName} and \IT{lpType} has \IT{char*} or \IT{wchar*} types, and when someone still wants to pass ID,
he/she have to use
\href{https://msdn.microsoft.com/en-us/library/windows/desktop/ms648029%28v=vs.85%29.aspx}{MAKEINTRESOURCE} macro, like this:

\myindex{win32!MAKEINTRESOURCE()}

\begin{lstlisting}[style=customc]
result = FindResource(..., MAKEINTRESOURCE(1234), ...);
\end{lstlisting}

It's interesting fact that MAKEINTRESOURCE is merely casting integer to pointer.
In MSVC 2013, in the file\\
\IT{Microsoft SDKs\textbackslash{}Windows\textbackslash{}v7.1A\textbackslash{}Include\textbackslash{}Ks.h} we can find this:

\begin{lstlisting}[style=customc]
...

#if (!defined( MAKEINTRESOURCE )) 
#define MAKEINTRESOURCE( res ) ((ULONG_PTR) (USHORT) res)
#endif

...
\end{lstlisting}

Sounds insane. Let's peek into ancient leaked Windows NT4 source code.
In \IT{private/windows/base/client/module.c} we can find \IT{FindResource()} source code:

\begin{lstlisting}[style=customc]
HRSRC
FindResourceA(
    HMODULE hModule,
    LPCSTR lpName,
    LPCSTR lpType
    )

...

{
    NTSTATUS Status;
    ULONG IdPath[ 3 ];
    PVOID p;

    IdPath[ 0 ] = 0;
    IdPath[ 1 ] = 0;
    try {
        if ((IdPath[ 0 ] = BaseDllMapResourceIdA( lpType )) == -1) {
            Status = STATUS_INVALID_PARAMETER;
            }
        else
        if ((IdPath[ 1 ] = BaseDllMapResourceIdA( lpName )) == -1) {
            Status = STATUS_INVALID_PARAMETER;
...
\end{lstlisting}

Let's proceed to \IT{BaseDllMapResourceIdA()} in the same source file:

\begin{lstlisting}[style=customc]
ULONG
BaseDllMapResourceIdA(
    LPCSTR lpId
    )
{
    NTSTATUS Status;
    ULONG Id;
    UNICODE_STRING UnicodeString;
    ANSI_STRING AnsiString;
    PWSTR s;

    try {
        if ((ULONG)lpId & LDR_RESOURCE_ID_NAME_MASK) {
            if (*lpId == '#') {
                Status = RtlCharToInteger( lpId+1, 10, &Id );
                if (!NT_SUCCESS( Status ) || Id & LDR_RESOURCE_ID_NAME_MASK) {
                    if (NT_SUCCESS( Status )) {
                        Status = STATUS_INVALID_PARAMETER;
                        }
                    BaseSetLastNTError( Status );
                    Id = (ULONG)-1;
                    }
                }
            else {
                RtlInitAnsiString( &AnsiString, lpId );
                Status = RtlAnsiStringToUnicodeString( &UnicodeString,
                                                       &AnsiString,
                                                       TRUE
                                                     );
                if (!NT_SUCCESS( Status )){
                    BaseSetLastNTError( Status );
                    Id = (ULONG)-1;
                    }
                else {
                    s = UnicodeString.Buffer;
                    while (*s != UNICODE_NULL) {
                        *s = RtlUpcaseUnicodeChar( *s );
                        s++;
                        }

                    Id = (ULONG)UnicodeString.Buffer;
                    }
                }
            }
        else {
            Id = (ULONG)lpId;
            }
        }
    except (EXCEPTION_EXECUTE_HANDLER) {
        BaseSetLastNTError( GetExceptionCode() );
        Id =  (ULONG)-1;
        }
    return Id;
}
\end{lstlisting}

\IT{lpId} is ANDed with \IT{LDR\_RESOURCE\_ID\_NAME\_MASK}. Which we can find in \IT{public/sdk/inc/ntldr.h}:

\begin{lstlisting}[style=customc]
...

#define LDR_RESOURCE_ID_NAME_MASK 0xFFFF0000

...
\end{lstlisting}

So \IT{lpId} is ANDed with \IT{0xFFFF0000} and if some bits beyond lowest 16 bits are still present,
first half of function is executed (\IT{lpId} is treated as an address of string).
Otherwise---second half (\IT{lpId} is treated as 16-bit value).

Still, this code can be found in Windows 7 kernel32.dll file:

\begin{lstlisting}[style=customasmx86]
....

.text:0000000078D24510 ; __int64 __fastcall BaseDllMapResourceIdA(PCSZ SourceString)
.text:0000000078D24510 BaseDllMapResourceIdA proc near         ; CODE XREF: FindResourceExA+34
.text:0000000078D24510                                         ; FindResourceExA+4B
.text:0000000078D24510
.text:0000000078D24510 var_38          = qword ptr -38h
.text:0000000078D24510 var_30          = qword ptr -30h
.text:0000000078D24510 var_28          = _UNICODE_STRING ptr -28h
.text:0000000078D24510 DestinationString= _STRING ptr -18h
.text:0000000078D24510 arg_8           = dword ptr  10h
.text:0000000078D24510
.text:0000000078D24510 ; FUNCTION CHUNK AT .text:0000000078D42FB4 SIZE 000000D5 BYTES
.text:0000000078D24510
.text:0000000078D24510                 push    rbx
.text:0000000078D24512                 sub     rsp, 50h
.text:0000000078D24516                 cmp     rcx, 10000h
.text:0000000078D2451D                 jnb     loc_78D42FB4
.text:0000000078D24523                 mov     [rsp+58h+var_38], rcx
.text:0000000078D24528                 jmp     short $+2
.text:0000000078D2452A ; ---------------------------------------------------------------------------
.text:0000000078D2452A
.text:0000000078D2452A loc_78D2452A:                           ; CODE XREF: BaseDllMapResourceIdA+18
.text:0000000078D2452A                                         ; BaseDllMapResourceIdA+1EAD0
.text:0000000078D2452A                 jmp     short $+2
.text:0000000078D2452C ; ---------------------------------------------------------------------------
.text:0000000078D2452C
.text:0000000078D2452C loc_78D2452C:                           ; CODE XREF: BaseDllMapResourceIdA:loc_78D2452A
.text:0000000078D2452C                                         ; BaseDllMapResourceIdA+1EB74
.text:0000000078D2452C                 mov     rax, rcx
.text:0000000078D2452F                 add     rsp, 50h
.text:0000000078D24533                 pop     rbx
.text:0000000078D24534                 retn
.text:0000000078D24534 ; ---------------------------------------------------------------------------
.text:0000000078D24535                 align 20h
.text:0000000078D24535 BaseDllMapResourceIdA endp

....

.text:0000000078D42FB4 loc_78D42FB4:                           ; CODE XREF: BaseDllMapResourceIdA+D
.text:0000000078D42FB4                 cmp     byte ptr [rcx], '#'
.text:0000000078D42FB7                 jnz     short loc_78D43005
.text:0000000078D42FB9                 inc     rcx
.text:0000000078D42FBC                 lea     r8, [rsp+58h+arg_8]
.text:0000000078D42FC1                 mov     edx, 0Ah
.text:0000000078D42FC6                 call    cs:__imp_RtlCharToInteger
.text:0000000078D42FCC                 mov     ecx, [rsp+58h+arg_8]
.text:0000000078D42FD0                 mov     [rsp+58h+var_38], rcx
.text:0000000078D42FD5                 test    eax, eax
.text:0000000078D42FD7                 js      short loc_78D42FE6
.text:0000000078D42FD9                 test    rcx, 0FFFFFFFFFFFF0000h
.text:0000000078D42FE0                 jz      loc_78D2452A

....

\end{lstlisting}

If value in input pointer is greater than 0x10000, jump to string processing is occurred.
Otherwise, input value of \IT{lpId} is returned as is.
\IT{0xFFFF0000} mask is not used here any more, because this is 64-bit code after all, but still, \IT{0xFFFFFFFFFFFF0000} could work here.

Attentive reader may ask, what if address of input string is lower than 0x10000?
This code relied on the fact that in Windows there are nothing on addresses below 0x10000, at least in Win32 realm.

Raymond Chen \href{https://blogs.msdn.microsoft.com/oldnewthing/20130925-00/?p=3123}{writes} about this:

\begin{framed}
\begin{quotation}
How does MAKE­INT­RESOURCE work? It just stashes the integer in the bottom 16 bits of a pointer, leaving the upper bits zero. This relies on the convention that the first 64KB of address space is never mapped to valid memory, a convention that is enforced starting in Windows 7.
\end{quotation}
\end{framed}

In short words, this is dirty hack and probably one should use it only if there is a real necessity.
Perhaps, \IT{FindResource()} function in past had \IT{SHORT} type for its arguments, and then Microsoft has added a way to pass strings there,
but older code must also be supported.

Now here is my short distilled example:

\begin{lstlisting}[style=customc]
#include <stdio.h>
#include <stdint.h>

void f(char* a)
{
	if (((uint64_t)a)>0x10000)
		printf ("Pointer to string has been passed: %s\n", a);
	else
		printf ("16-bit value has been passed: %d\n", (uint64_t)a);
};

int main()
{
	f("Hello!"); // pass string
	f((char*)1234); // pass 16-bit value
};
\end{lstlisting}

It works!

\subsubsection{Pointers abuse in Linux kernel}

As it has been noted in \href{https://news.ycombinator.com/item?id=11823647}{comments on Hacker News}, Linux kernel also has something like that.

For example, this function can return both error code and pointer:

\begin{lstlisting}[style=customc]
struct kernfs_node *kernfs_create_link(struct kernfs_node *parent,
				       const char *name,
				       struct kernfs_node *target)
{
	struct kernfs_node *kn;
	int error;

	kn = kernfs_new_node(parent, name, S_IFLNK|S_IRWXUGO, KERNFS_LINK);
	if (!kn)
		return ERR_PTR(-ENOMEM);

	if (kernfs_ns_enabled(parent))
		kn->ns = target->ns;
	kn->symlink.target_kn = target;
	kernfs_get(target);	/* ref owned by symlink */

	error = kernfs_add_one(kn);
	if (!error)
		return kn;

	kernfs_put(kn);
	return ERR_PTR(error);
}
\end{lstlisting}

( \url{https://github.com/torvalds/linux/blob/fceef393a538134f03b778c5d2519e670269342f/fs/kernfs/symlink.c#L25} )

\IT{ERR\_PTR} is a macro to cast integer to pointer:

\begin{lstlisting}[style=customc]
static inline void * __must_check ERR_PTR(long error)
{
	return (void *) error;
}
\end{lstlisting}

( \url{https://github.com/torvalds/linux/blob/61d0b5a4b2777dcf5daef245e212b3c1fa8091ca/tools/virtio/linux/err.h} )

This header file also has a macro helper to distinguish error code from pointer:

\begin{lstlisting}[style=customc]
#define IS_ERR_VALUE(x) unlikely((x) >= (unsigned long)-MAX_ERRNO)
\end{lstlisting}

This means, error codes are the ``pointers'' which are very close to -1 and, hopefully, there are nothing in kernel memory
on the addresses like 0xFFFFFFFFFFFFFFFF, 0xFFFFFFFFFFFFFFFE, 0xFFFFFFFFFFFFFFFD, etc.

Much more popular solution is to return \IT{NULL} in case of error and to pass error code via additional argument.
Linux kernel authors don't do that, but everyone who use these functions must always keep in mind that returning pointer
must always be checked with \IT{IS\_ERR\_VALUE} before dereferencing.

For example:

\begin{lstlisting}[style=customc]
	fman->cam_offset = fman_muram_alloc(fman->muram, fman->cam_size);
	if (IS_ERR_VALUE(fman->cam_offset)) {
		dev_err(fman->dev, "%s: MURAM alloc for DMA CAM failed\n",
			__func__);
		return -ENOMEM;
	}
\end{lstlisting}

( \url{https://github.com/torvalds/linux/blob/aa00edc1287a693eadc7bc67a3d73555d969b35d/drivers/net/ethernet/freescale/fman/fman.c#L826} )

\subsubsection{Pointers abuse in UNIX userland}

\myindex{UNIX!mmap()}
mmap() function returns -1 in case of error (or \TT{MAP\_FAILED}, which equals to -1).
Some people say, mmap() can map a memory at zeroth address in rare situations, so it can't use 0 or NULL as error code.


\subsection{Null pointers}

\subsubsection{``Null pointer assignment'' error of MS-DOS era}

\myindex{MS-DOS}
Some oldschoolers may recall a weird error message of MS-DOS era: ``Null pointer assignment''.
What does it mean?

It's not possible to write a memory at zero address in *NIX and Windows OSes, but it was possible to do so in MS-DOS due to absence of memory protection whatsoever.

\myindex{Turbo C++}
\myindex{Borland C++}
So I've pulled my ancient Turbo C++ 3.0 (later it was renamed to Borland C++) from early 1990s and tried to compile this:

\begin{lstlisting}[style=customc]
#include <stdio.h>

int main()
{
	int *ptr=NULL;
	*ptr=1234;
	printf ("Now let's read at NULL\n");
	printf ("%d\n", *ptr);
};
\end{lstlisting}

Hard to believe, but it works, with error upon exit, though:

\begin{lstlisting}[caption=Ancient Turbo C 3.0]
C:\TC30\BIN\1
Now let's read at NULL
1234
Null pointer assignment

C:\TC30\BIN>_
\end{lstlisting}

Let's dig deeper into the source code of \ac{CRT} of Borland C++ 3.1, file \IT{c0.asm}:

\begin{lstlisting}[style=customasmx86]
;       _checknull()    check for null pointer zapping copyright message

...

;       Check for null pointers before exit

__checknull     PROC    DIST
                PUBLIC  __checknull

IF      LDATA  EQ  false
  IFNDEF  __TINY__
                push    si
                push    di
                mov     es, cs:DGROUP@@
                xor     ax, ax
                mov     si, ax
                mov     cx, lgth_CopyRight
ComputeChecksum label   near
                add     al, es:[si]
                adc     ah, 0
                inc     si
                loop    ComputeChecksum
                sub     ax, CheckSum
                jz      @@SumOK
                mov     cx, lgth_NullCheck
                mov     dx, offset DGROUP: NullCheck
                call    ErrorDisplay
@@SumOK:        pop     di
                pop     si
  ENDIF
ENDIF

_DATA           SEGMENT

;       Magic symbol used by the debug info to locate the data segment
                public DATASEG@
DATASEG@        label   byte

;       The CopyRight string must NOT be moved or changed without
;       changing the null pointer check logic

CopyRight       db      4 dup(0)
                db      'Borland C++ - Copyright 1991 Borland Intl.',0
lgth_CopyRight  equ     $ - CopyRight

IF      LDATA  EQ  false
IFNDEF  __TINY__
CheckSum        equ     00D5Ch
NullCheck       db      'Null pointer assignment', 13, 10
lgth_NullCheck  equ     $ - NullCheck
ENDIF
ENDIF

...

\end{lstlisting}

The MS-DOS memory model was really weird (\ref{8086_memory_model}) and probably not worth looking into it unless you're fan of retrocomputing or retrogaming.
One thing we have to keep in mind is that memory segment (included data segment) in MS-DOS is a memory segment in which code or data is stored,
but unlike ``serious'' \ac{OS}es, it's started at address 0.

And in Borland C++ \ac{CRT}, the data segment is started with 4 zero bytes and the copyright string ``Borland C++ - Copyright 1991 Borland Intl.''.
The integrity of the 4 zero bytes and text string is checked upon exit, and if it's corrupted, the error message is displayed.

But why? Writing at null pointer is common mistake in \CCpp, and if you do so in *NIX or Windows, your application will crash.
MS-DOS has no memory protection, so \ac{CRT} has to check this post-factum and warn about it upon exit.
If you see this message, this means, your program at some point has written at address 0.

Our program did so. And this is why 1234 number has been read correctly: because it was written at the place of the first 4 zero bytes.
Checksum is incorrect upon exit (because the number has left there), so error message has been displayed.

Am I right?
I've rewritten the program to check my assumptions:

\begin{lstlisting}[style=customc]
#include <stdio.h>

int main()
{
	int *ptr=NULL;
	*ptr=1234;
	printf ("Now let's read at NULL\n");
	printf ("%d\n", *ptr);
	*ptr=0; // psst, cover our tracks!
};
\end{lstlisting}

This program executes without error message upon exit.

Though method to warn about null pointer assignment is relevant for MS-DOS,
perhaps, it can still be used today in low-cost \ac{MCU}s with no memory protection and/or \ac{MMU}.

\subsubsection{Why would anyone write at address 0?}

But why would sane programmer write a code which writes something at address 0?
It can be done accidentally: for example, a pointer must be initialized to newly allocated memory block and then passed to some function which returns data through pointer.

\begin{lstlisting}[style=customc]
int *ptr=NULL;

... we forgot to allocate memory and initialize ptr

strcpy (ptr, buf); // strcpy() terminates silently because MS-DOS has no memory protection
\end{lstlisting}

Even worse:

\begin{lstlisting}[style=customc]
int *ptr=malloc(1000);

... we forgot to check if memory has been really allocated: this is MS-DOS after all and computers had small amount of RAM,
... and RAM shortage was very common.
... if malloc() returned NULL, the ptr will also be NULL.

strcpy (ptr, buf); // strcpy() terminates silently because MS-DOS has no memory protection
\end{lstlisting}

\subsubsection{NULL in \CCpp}

NULL in C/C++ is just a macro which is often defined like this:

\begin{lstlisting}[style=customc]
#define NULL  ((void*)0)
\end{lstlisting}
( \href{https://github.com/wzhy90/linaro_toolchains/blob/8ff8ae680bac04558d10cc9626e12c4c2f6c1348/arm-cortex_a15-linux-gnueabihf/libc/usr/include/libio.h#L70}{libio.h file} )

\IT{void*} is a data type reflecting the fact it's the pointer, but to a value of unknown data type (\IT{void}).

NULL is usually used to show absence of an object.
For example, you have a single-linked list, and each node has a value (or pointer to a value) and \IT{next} pointer.
To show that there are no next node, 0 is stored to \IT{next} field.
Other solutions are just worse.
Perhaps, you may have some crazy environment where you need to allocate memory blocks at zero address. How would you indicate absence of the next node?
Some kind of \IT{magic number}? Maybe -1? Or maybe using additional bit?

In Wikipedia we may find this:

\begin{framed}
\begin{quotation}
In fact, quite contrary to the zero page's original preferential use, some modern operating systems such as FreeBSD, Linux and Microsoft Windows[2] actually make the zero page inaccessible to trap uses of NULL pointers. 
\end{quotation}
\end{framed}
( \url{https://en.wikipedia.org/wiki/Zero_page} )

\subsubsection{Null pointer to function}

It's possible to call function by its address.
For example, I compile this by MSVC 2010 and run it in Windows 7:

\begin{lstlisting}[style=customc]
#include <windows.h>
#include <stdio.h>

int main()
{
	printf ("0x%x\n", &MessageBoxA);
};
\end{lstlisting}

The result is \IT{0x7578feae} and doesn't changing after several times I run it,
because user32.dll (where MessageBoxA function resides) is always loads at the same address.
And also because \ac{ASLR} is not enabled (result would be different each time in that case).

Let's call \IT{MessageBoxA()} by address:

\begin{lstlisting}[style=customc]
#include <windows.h>
#include <stdio.h>

typedef int (*msgboxtype)(HWND hWnd, LPCTSTR lpText, LPCTSTR lpCaption,  UINT uType);

int main()
{
	msgboxtype msgboxaddr=0x7578feae;

	// force to load DLL into process memory, 
	// since our code doesn't use any function from user32.dll, 
	// and DLL is not imported
	LoadLibrary ("user32.dll");

	msgboxaddr(NULL, "Hello, world!", "hello", MB_OK);
};
\end{lstlisting}

Weird, but works in Windows 7 x86.

This is commonly used in shellcodes, because it's hard to call DLL functions by name from there.
And \ac{ASLR} is a countermeasure.

Now what is really weird, some embedded C programmers may be familiar with a code like that:

\begin{lstlisting}[style=customc]
int reset()
{
	void (*foo)(void) = 0;
	foo();
};
\end{lstlisting}

Who will want to call a function at address 0?
This is portable way to jump at zero address.
Many low-cost cheap microcontrollers also have no memory protection or \ac{MMU} and after reset, they start to execute code at address 0, where some kind of initialization code is stored.
So jumping to address 0 is a way to reset itself.
One could use inline assembly, but if it's not possible, this portable method can be used.

It even compiles correctly by my GCC 4.8.4 on Linux x64:

\begin{lstlisting}[style=customasmx86]
reset:
        sub     rsp, 8
        xor     eax, eax
        call    rax
        add     rsp, 8
        ret
\end{lstlisting}

The fact that stack pointer is shifted is not a problem: initialization code in microcontrollers usually completely ignores registers and \ac{RAM} state and boots from scratch.

And of course, this code will crash on *NIX or Windows because of memory protection and even in absence of protection, there are no code at address 0.

GCC even has non-standard extension, allowing to jump to a specific address rather than call a function there:
\url{http://gcc.gnu.org/onlinedocs/gcc/Labels-as-Values.html}.


\subsection{Array as function argument}

Someone may ask, what is the difference between declaring function argument type as array and as pointer?

As it seems, there are no difference at all:

\begin{lstlisting}[style=customc]
void write_something1(int a[16])
{
	a[5]=0;
};

void write_something2(int *a)
{
	a[5]=0;
};

int f()
{
	int a[16];
	write_something1(a);
	write_something2(a);
};
\end{lstlisting}

Optimizing GCC 4.8.4:

\begin{lstlisting}[style=customasmx86]
write_something1:
        mov     DWORD PTR [rdi+20], 0
        ret

write_something2:
        mov     DWORD PTR [rdi+20], 0
        ret
\end{lstlisting}

But you may still declare array instead of pointer for self-documenting purposes, if the size of array is always fixed.
And maybe, some static analysis tool will be able to warn you about possible buffer overflow.
Or is it possible with some tools today?

Some people, including Linus Torvalds, criticizes this \CCpp feature: \url{https://lkml.org/lkml/2015/9/3/428}.

C99 standard also have \IT{static} keyword \InSqBrackets{\CNineNineStd{} 6.7.5.3}:

\begin{framed}
\begin{quotation}
If the keyword static also appears  within the [ and ] of the array type derivation, then for each call to the function, the value of the corresponding actual argument shall provide access to the first element of an array with at least as many elements as specified by the size expression.
\end{quotation}
\end{framed}


\subsection{Pointer to function}

A function name in \CCpp without brackets, like ``printf'' is a pointer to function of \IT{void (*)()} type.
Let's try to read function's contents and patch it:

\lstinputlisting[style=customc]{advanced/450_more_ptrs/6.c}

It tells, that the first 3 bytes of functions are \TT{55 89 e5}.
Indeed, these are opcodes of \INS{PUSH EBP} and \INS{MOV EBP, ESP} instructions (these are x86 opcodes).
But then our program crashes, because \IT{text} section is readonly.

We can recompile our example and make \IT{text} section writable
\footnote{\url{http://stackoverflow.com/questions/27581279/make-text-segment-writable-elf}}:

\begin{lstlisting}
gcc --static -g -Wl,--omagic -o example example.c
\end{lstlisting}

That works!

\begin{lstlisting}
we are in print_something()
first 3 bytes: 55 89 e5...
going to call patched print_something():
it must exit at this point
\end{lstlisting}



\subsection{Pointer as object identificator}

Both assembly language and C has no \ac{OOP} features, but it's possible to write a code in \ac{OOP} style
(just treat structure as an object).

It's interesting, that sometimes, pointer to an object (or its address) is called as ID
(in sense of data hiding/encapsulation).

\myindex{win32!LoadLibrary()}
\myindex{win32!GetProcAddress()}
For example, LoadLibrary(), according to \ac{MSDN}, returns ``handle to the module''
\footnote{\url{https://msdn.microsoft.com/ru-ru/library/windows/desktop/ms684175(v=vs.85).aspx}}.
Then you pass this ``handle'' to other functions like GetProcAddress().
But in fact, LoadLibrary() returns pointer to DLL file mapped into memory
\footnote{\url{https://blogs.msdn.microsoft.com/oldnewthing/20041025-00/?p=37483}}.
You can read two bytes from the address LoadLibrary() returns, and that would be ``MZ'' (first two bytes of any
.EXE/.DLL file in Windows).

\myindex{win32!HMODULE}
\myindex{win32!HINSTANCE}
Apparently, Microsoft ``hides'' that fact to provide better forward compatibility.
Also, HMODULE and HINSTANCE data types had another meaning in 16-bit Windows.

Probably, this is reason why \printf has ``\%p'' modifier, which is used for printing pointers (32-bit integers
on 32-bit architectures, 64-bit on 64-bit, etc) in hexadecimal form.
Address of a structure dumped into debug log may help in finding it in another place of log.

\myindex{SQLite}
Here is also from SQLite source code:

\begin{lstlisting}

...

struct Pager {
  sqlite3_vfs *pVfs;          /* OS functions to use for IO */
  u8 exclusiveMode;           /* Boolean. True if locking_mode==EXCLUSIVE */
  u8 journalMode;             /* One of the PAGER_JOURNALMODE_* values */
  u8 useJournal;              /* Use a rollback journal on this file */
  u8 noSync;                  /* Do not sync the journal if true */

....

static int pagerLockDb(Pager *pPager, int eLock){
  int rc = SQLITE_OK;

  assert( eLock==SHARED_LOCK || eLock==RESERVED_LOCK || eLock==EXCLUSIVE_LOCK );
  if( pPager->eLock<eLock || pPager->eLock==UNKNOWN_LOCK ){
    rc = sqlite3OsLock(pPager->fd, eLock);
    if( rc==SQLITE_OK && (pPager->eLock!=UNKNOWN_LOCK||eLock==EXCLUSIVE_LOCK) ){
      pPager->eLock = (u8)eLock;
      IOTRACE(("LOCK %p %d\n", pPager, eLock))
    }
  }
  return rc;
}

...

  PAGER_INCR(sqlite3_pager_readdb_count);
  PAGER_INCR(pPager->nRead);
  IOTRACE(("PGIN %p %d\n", pPager, pgno));
  PAGERTRACE(("FETCH %d page %d hash(%08x)\n",
               PAGERID(pPager), pgno, pager_pagehash(pPg)));

...

\end{lstlisting}

