\subsection{Нулевые указатели}

\subsubsection{Ошибка ``Null pointer assignment'' во времена MS-DOS}

\myindex{MS-DOS}
Некоторые люди постарше могут помнить очень странную ошибку эпохи MS-DOS: ``Null pointer assignment''.
Что она означает?

В *NIX и Windows нельзя записывать в память по нулевому адресу, но это было возможно в MS-DOS, из-за отсутствия
защиты памяти как таковой.

\myindex{Turbo C++}
\myindex{Borland C++}
Так что я могу найти древний Turbo C++ 3.0 (позже он был переименован в C++) из начала 1990-х и попытаться
скомпилировать это:

\begin{lstlisting}[style=customc]
#include <stdio.h>

int main()
{
	int *ptr=NULL;
	*ptr=1234;
	printf ("Now let's read at NULL\n");
	printf ("%d\n", *ptr);
};
\end{lstlisting}

Трудно поверить, но это работает, но с ошибкой при выходе:

\begin{lstlisting}[caption=Древний Turbo C++ 3.0]
C:\TC30\BIN\1
Now let's read at NULL
1234
Null pointer assignment

C:\TC30\BIN>_
\end{lstlisting}

Посмотрим внутри исходного кода \ac{CRT} компилятора Borland C++ 3.1, файл \IT{c0.asm}:

\begin{lstlisting}[style=customasmx86]
;       _checknull()    check for null pointer zapping copyright message

...

;       Check for null pointers before exit

__checknull     PROC    DIST
                PUBLIC  __checknull

IF      LDATA  EQ  false
  IFNDEF  __TINY__
                push    si
                push    di
                mov     es, cs:DGROUP@@
                xor     ax, ax
                mov     si, ax
                mov     cx, lgth_CopyRight
ComputeChecksum label   near
                add     al, es:[si]
                adc     ah, 0
                inc     si
                loop    ComputeChecksum
                sub     ax, CheckSum
                jz      @@SumOK
                mov     cx, lgth_NullCheck
                mov     dx, offset DGROUP: NullCheck
                call    ErrorDisplay
@@SumOK:        pop     di
                pop     si
  ENDIF
ENDIF

_DATA           SEGMENT

;       Magic symbol used by the debug info to locate the data segment
                public DATASEG@
DATASEG@        label   byte

;       The CopyRight string must NOT be moved or changed without
;       changing the null pointer check logic

CopyRight       db      4 dup(0)
                db      'Borland C++ - Copyright 1991 Borland Intl.',0
lgth_CopyRight  equ     $ - CopyRight

IF      LDATA  EQ  false
IFNDEF  __TINY__
CheckSum        equ     00D5Ch
NullCheck       db      'Null pointer assignment', 13, 10
lgth_NullCheck  equ     $ - NullCheck
ENDIF
ENDIF

...

\end{lstlisting}

Модель памяти в MS-DOS крайне странная (\ref{8086_memory_model}), и, вероятно, её и не нужно изучать, если только вы не фанат ретрокомпьютинга
или ретрогейминга.
Одну только вещь можно держать в памяти, это то, что сегмент памяти (включая сегмент данных) в MS-DOS это место где
хранится код или данные, но в отличие от ``серьезных'' \ac{OS}, он начинается с нулевого адреса.

И в Borland C++ \ac{CRT}, сегмент данных начинается с 4-х нулевых байт и строки копирайта
``Borland C++ - Copyright 1991 Borland Intl.''.
Целостность 4-х нулевых байт и текстовой строки проверяется в конце, и если что-то нарушено, выводится сообщение об ошибке.

Но зачем? Запись по нулевому указателю это распространенная ошибка в \CCpp, и если вы делаете это в *NIX или Windows,
ваше приложение упадет.
В MS-DOS нет защиты памяти, так что это приходится проверять в \ac{CRT} во время выхода, пост-фактум.
Если вы видите это сообщение, значит ваша программа в каком-то месте что-то записала по нулевому адресу.

Наша программа это сделала. И вот почему число 1234 было прочитано корректно: потому что оно было записано на месте
первых 4-х байт.
Контрольная сумма во время выхода неверна (потому что наше число там осталось), так что сообщение было выведено.

Прав ли я?
Я переписал программу для проверки моих предположений:

\begin{lstlisting}[style=customc]
#include <stdio.h>

int main()
{
	int *ptr=NULL;
	*ptr=1234;
	printf ("Now let's read at NULL\n");
	printf ("%d\n", *ptr);
	*ptr=0; // psst, cover our tracks!
};
\end{lstlisting}

Программа исполняется без ошибки во время выхода.

Хотя и метод предупреждать о записи по нулевому указателю имел смысл в MS-DOS,
вероятно, это всё может использоваться и сегодня, на маломощных \ac{MCU} без защиты памяти и/или \ac{MMU}.

\subsubsection{Почему кому-то может понадобиться писать по нулевому адресу?}

Но почему трезвомыслящему программисту может понадобиться записывать что-то по нулевому адресу?
Это может быть сделано случайно, например, указатель должен быть инициализирован и указывать на только что выделенный
блок в памяти, а затем должен быть передан в какую-то ф-цию, возвращающую данные через указатель.

\begin{lstlisting}[style=customc]
int *ptr=NULL;

... мы забыли выделить память и инициализировать ptr

strcpy (ptr, buf); // strcpy() завершает работу молча, потому что в MS-DOS нет защиты памяти
\end{lstlisting}

И даже хуже:

\begin{lstlisting}[style=customc]
int *ptr=malloc(1000);

... мы забыли проверить, действительно ли память была выделена: это же MS-DOS и у тогдащних компьютеров было мало памяти,
... и нехватка памяти была обычной ситуацией.
... если malloc() вернул NULL, тогда ptr будет тоже NULL.

strcpy (ptr, buf); // strcpy() завершает работу молча, потому что в MS-DOS нет защиты памяти
\end{lstlisting}

\subsubsection{NULL в \CCpp}

NULL в C/C++ это просто макрос, который часто определяют так:

\begin{lstlisting}[style=customc]
#define NULL  ((void*)0)
\end{lstlisting}
( \href{https://github.com/wzhy90/linaro_toolchains/blob/8ff8ae680bac04558d10cc9626e12c4c2f6c1348/arm-cortex_a15-linux-gnueabihf/libc/usr/include/libio.h#L70}{libio.h file} )

\IT{void*} это тип данных, отражающий тот факт, что это указатель, но на значение неизвестного типа (\IT{void}).

NULL обычно используется чтобы показать отсутствие объекта.
Например, у вас есть односвязный список, и каждый узел имеет значение (или указатель на значение) и указатель вроде \IT{next}.
Чтобы показать, что следующего узла нет, в поле \IT{next} записывается 0.
Другие решения просто хуже.
Вероятно, вы можете использовать какую-то крайне экзотическую среду, где можно выделить память по нулевому адресу.
Как вы будете показывать отсутствие следующего узла?
Какой-нибудь \IT{magic number}? Может быть -1? Или дополнительным битом?

В Википедии мы можем найти это:

\begin{framed}
\begin{quotation}
In fact, quite contrary to the zero page's original preferential use, some modern operating systems such as FreeBSD, Linux and Microsoft Windows[2] actually make the zero page inaccessible to trap uses of NULL pointers. 
\end{quotation}
\end{framed}
( \url{https://en.wikipedia.org/wiki/Zero_page} )

\subsubsection{Нулевой указатель на ф-цию}

Можно вызывать ф-ции по их адресу.
Например, я компилирую это при помощи MSVC 2010 и запускаю в Windows 7:

\begin{lstlisting}[style=customc]
#include <windows.h>
#include <stdio.h>

int main()
{
	printf ("0x%x\n", &MessageBoxA);
};
\end{lstlisting}

Результат \IT{0x7578feae}, и он не меняется и после того, как я запустил это несколько раз, потому что
user32.dll (где находится ф-ция MessageBoxA) всегда загружается по одному и тому же адресу.
И потому что \ac{ASLR} не включено (тогда результат был бы всё время разным).

Вызовем ф-цию \IT{MessageBoxA()} по адресу:

\begin{lstlisting}[style=customc]
#include <windows.h>
#include <stdio.h>

typedef int (*msgboxtype)(HWND hWnd, LPCTSTR lpText, LPCTSTR lpCaption,  UINT uType);

int main()
{
	msgboxtype msgboxaddr=0x7578feae;

	// заставить загрузиться DLL в память процесса,
	// т.к., наш код не использует никакую ф-цию из user32.dll, 
	// и DLL не импортируется
	LoadLibrary ("user32.dll");

	msgboxaddr(NULL, "Hello, world!", "hello", MB_OK);
};
\end{lstlisting}

Странно выглядит, но работает в Windows 7 x86.

Это часто используется в шеллкода, потому что оттуда трудно вызывать ф-ции из DLL по их именам.
А \ac{ASLR} это контрмера.

И вот теперь что по-настоящему странно выглядит, некоторые программисты на Си для встраиваемых (embedded) систем, могут быть
знакомы с таким кодом:

\begin{lstlisting}[style=customc]
int reset()
{
	void (*foo)(void) = 0;
	foo();
};
\end{lstlisting}

Кому понадобится вызывать ф-цию по адресу 0?
Это портабельный способ перейти на нулевой адрес.
Множество маломощных микроконтроллеров не имеют защиты памяти или \ac{MMU}, и после сброса, они просто начинают
исполнять код по нулевому адресу, где может быть записан инициализирующий код.
Так что переход по нулевому адресу это способ сброса.
Можно использовать и inline-ассемблер, но если это неудобно, тогда можно использовать этот портабельный метод.

Это даже корректно компилируется при помощи GCC 4.8.4 на Linux x64:

\begin{lstlisting}[style=customasmx86]
reset:
        sub     rsp, 8
        xor     eax, eax
        call    rax
        add     rsp, 8
        ret
\end{lstlisting}

То обстоятельство, что указатель стека сдвинут, это не проблема: инициализирующий код в микроконтроллерах обычно
полностью игнорирует состояние регистров и памяти и загружает всё ``с чистого листа''.

И конечно, этот код упадет в *NIX или Windows, из-за защиты памяти, и даже если бы её не было, по нулевому адресу
нет никакого кода.

В GCC даже есть нестандартное расширение, позволяющее перейти по определенному адресу, вместо того чтобы вызывать ф-цию:
\url{http://gcc.gnu.org/onlinedocs/gcc/Labels-as-Values.html}.

