\section{Strings trimming}
\newcommand{\CRLF}{\ac{CR}/\ac{LF}}

A very common string processing task is to remove some characters at the start and/or at the end.

In this example, we are going to work with a function which removes all newline characters 
(\CRLF{}) from the end of the input string:

\lstinputlisting[style=customc]{\CURPATH/strtrim_EN.c}

The input argument is always returned on exit, this is convenient when you want to chain 
string processing functions, like it has done here in the \main function.

The second part of for() (\TT{str\_len>0 \&\& (c=s[str\_len-1])}) is the so called \q{short-circuit} 
in \CCpp and is very convenient \InSqBrackets{\CNotes 1.3.8}.

The \CCpp compilers guarantee an evaluation sequence from left to right.

So if the first clause is false after evaluation, the second one is never to be evaluated.

% subsections
\subsection{x64: \Optimizing MSVC 2013}

\lstinputlisting[caption=\Optimizing MSVC 2013 x64,style=customasmx86]{\CURPATH/MSVC2013_x64_Ox_EN.asm}

First, MSVC inlined the \strlen{} function code, because it concluded this 
is to be faster than the usual \strlen{} work + the cost of calling it and returning from it.
This is called inlining: \myref{inline_code}.

\myindex{x86!\Instructions!OR}
\myindex{\CStandardLibrary!strlen()}
\label{using_OR_instead_of_MOV}
The first instruction of the inlined \strlen{} is\\
\TT{OR RAX, 0xFFFFFFFFFFFFFFFF}. 

MSVC often uses \TT{OR} instead of \TT{MOV RAX, 0xFFFFFFFFFFFFFFFF}, because resulting opcode is shorter.

And of course, it is equivalent: all bits are set, and a number with all bits set is $-1$ 
in two's complement arithmetic: \myref{sec:signednumbers}.

Why would the $-1$ number be used in \strlen{}, one might ask.
Due to optimizations, of course.
Here is the code that MSVC generated:

\lstinputlisting[caption=Inlined \strlen{} by MSVC 2013 x64,style=customasmx86]{\CURPATH/strlen_MSVC_EN.asm}

Try to write shorter if you want to initialize the counter at 0!
OK, let' try:

\lstinputlisting[caption=Our version of \strlen{},style=customasmx86]{\CURPATH/my_strlen_EN.asm}

We failed. We have to use additional \INS{JMP} instruction!

So what the MSVC 2013 compiler did is to move the \TT{INC} instruction to the place before 
the actual character loading.

If the first character is 0, that's OK, \RAX is 0 at this moment, 
so the resulting string length is 0.

The rest in this function seems easy to understand.

\subsection{x64: \NonOptimizing GCC 4.9.1}

\lstinputlisting[style=customasmx86]{\CURPATH/GCC491_x64_O0_EN.asm}

Comments are added by the author of the book.

After the execution of \strlen{}, the control is passed to the L2 label, 
and there two clauses are checked, one after another.

The second will never be checked, if the first one (\IT{str\_len==0}) is false 
(this is \q{short-circuit}).

Now let's see this function in short form:

\begin{itemize}
\item First for() part (call to \strlen{})
\item goto L2
\item L5: for() body. goto exit, if needed
\item for() third part (decrement of str\_len)
\item L2: 
for() second part: check first clause, then second. goto loop body begin or exit.
\item L4: // exit
\item return s
\end{itemize}

\subsection{x64: \Optimizing GCC 4.9.1}
\label{string_trim_GCC_x64_O3}

\lstinputlisting[style=customasmx86]{\CURPATH/GCC491_x64_O3_EN.asm}

Now this is more complex.

The code before the loop's body start is executed only once, but it has the \CRLF{} 
characters check too!
What is this code duplication for?

The common way to implement the main loop is probably this:

\begin{itemize}
\item (loop start) check for 
\CRLF{} characters, make decisions
\item store zero character
\end{itemize}

But GCC has decided to reverse these two steps. 

Of course, \IT{store zero character} cannot be first step, so another check is needed:

\begin{itemize}
\item workout first character. match it to \CRLF{}, exit if character is not \CRLF{}

\item (loop begin) store zero character

\item check for \CRLF{} characters, make decisions
\end{itemize}

Now the main loop is very short, which is good for latest \ac{CPU}s.

The code doesn't use the str\_len variable, but str\_len-1.
So this is more like an index in a buffer.

Apparently, GCC notices that the str\_len-1 statement is used twice.

So it's better to allocate a variable which always holds a value that's smaller than 
the current string length by one, 
and decrement it (this is the same effect as decrementing the str\_len variable).

\subsubsection{ARM64}

\myparagraph{\Optimizing GCC (Linaro) 4.9}

\lstinputlisting[style=customasmARM]{patterns/12_FPU/3_comparison/ARM/ARM64_GCC_O3_EN.lst}

The ARM64 \ac{ISA} has FPU-instructions 
which set \ac{APSR} the CPU flags instead of \ac{FPSCR} for convenience.
The\ac{FPU} is not a separate device here anymore (at least, logically).
\myindex{ARM!\Instructions!FCMPE}
Here we see \INS{FCMPE}. It compares the two values passed in \RegD{0} and \RegD{1} (which are the first and second arguments of the function)
and sets \ac{APSR} flags (N, Z, C, V).

\myindex{ARM!\Instructions!FCSEL}
\INS{FCSEL} (\IT{Floating Conditional Select}) copies the value of \RegD{0} or \RegD{1} into \RegD{0} depending on the condition (\GTT{GT}---Greater Than),
and again, it uses flags in \ac{APSR} register instead of \ac{FPSCR}.

This is much more convenient, compared to the instruction set in older CPUs.

If the condition is true (\GTT{GT}), then the value of \RegD{0} 
is copied into \RegD{0} (i.e., nothing happens).
If the condition is not true, the value of \RegD{1} 
is copied into \RegD{0}.

\myparagraph{\NonOptimizing GCC (Linaro) 4.9}

\lstinputlisting[style=customasmARM]{patterns/12_FPU/3_comparison/ARM/ARM64_GCC_EN.lst}

Non-optimizing GCC is more verbose.

First, the function saves its input argument values in the local stack (\IT{Register Save Area}).
Then the code reloads these values into registers
\RegX{0}/\RegX{1} and finally copies them to 
\RegD{0}/\RegD{1} to be compared using \INS{FCMPE}. 
A lot of redundant code, 
but that is how non-optimizing compilers work.
\INS{FCMPE} compares the values and sets the \ac{APSR} flags.
At this moment, 
the compiler is not thinking yet about the more convenient \INS{FCSEL} instruction, so it proceed using old methods: 
using the \INS{BLE} instruction (\IT{Branch if Less than or Equal}).
In the first case ($a>b$), the value of $a$ gets loaded 
into \RegX{0}.
In the other case ($a<=b$), the value of $b$ gets loaded into 
\RegX{0}.
Finally, the value from \RegX{0} gets copied into \RegD{0}, 
because the return value needs to be in this 
register.

\mysubparagraph{\Exercise}

As an exercise, you can try optimizing this piece of code 
manually by removing redundant instructions and not introducing new ones (including \INS{FCSEL}).

\myparagraph{\Optimizing GCC (Linaro) 4.9---float}

Let's also rewrite this example to use \Tfloat instead of \Tdouble.

\begin{lstlisting}[style=customc]
float f_max (float a, float b)
{
	if (a>b)
		return a;

	return b;
};
\end{lstlisting}

\lstinputlisting[style=customasmARM]{patterns/12_FPU/3_comparison/ARM/ARM64_GCC_O3_float_EN.lst}

It is the same code, but the S-registers are used instead of D- ones.
It's because numbers of type \Tfloat are passed in 32-bit S-registers (which are in fact the lower parts of the 64-bit D-registers).


\subsubsection{ARM + \OptimizingKeilVI (\ARMMode)}

\begin{lstlisting}[caption=\OptimizingKeilVI (\ARMMode),style=customasmARM]
02 0C C0 E3          BIC     R0, R0, #0x200
01 09 80 E3          ORR     R0, R0, #0x4000
1E FF 2F E1          BX      LR
\end{lstlisting}

\myindex{ARM!\Instructions!BIC}
\INS{BIC} (\IT{BItwise bit Clear}) is an instruction for clearing 
specific bits. This is just like the \AND instruction, but with inverted operand.
I.e., it's analogous to a \NOT+\AND instruction pair.

\myindex{ARM!\Instructions!ORR}
\INS{ORR} is \q{logical or}, analogous to \OR in x86.

So far it's easy.

\subsubsection{ARM + \OptimizingKeilVI (\ThumbMode)}

\begin{lstlisting}[caption=\OptimizingKeilVI (\ThumbMode),style=customasmARM]
01 21 89 03          MOVS    R1, 0x4000
08 43                ORRS    R0, R1
49 11                ASRS    R1, R1, #5   ; generate 0x200 and place to R1
88 43                BICS    R0, R1
70 47                BX      LR
\end{lstlisting}

Seems like Keil decided that the code in Thumb mode,
making \TT{0x200} from \TT{0x4000}, 
is more compact than the code 
for writing \TT{0x200} to an arbitrary register.
% TODO1 пример, как компилятор при помощи сдвигов оптизирует такое: a=0x1000; b=0x2000; c=0x4000, etc

\myindex{ARM!\Instructions!ASRS}

So that is why, with the help of \INS{ASRS} (\ASRdesc), this value is calculated as $\TT{0x4000} \gg 5$.

\subsubsection{ARM + \OptimizingXcodeIV (\ARMMode)}
\label{anomaly:LLVM}
\myindex{\CompilerAnomaly}

\begin{lstlisting}[caption=\OptimizingXcodeIV (\ARMMode),label=ARM_leaf_example3,style=customasmARM]
42 0C C0 E3          BIC             R0, R0, #0x4200
01 09 80 E3          ORR             R0, R0, #0x4000
1E FF 2F E1          BX              LR
\end{lstlisting}

The code that was generated by LLVM, in source code form could be something like this:

\begin{lstlisting}[style=customc]
    REMOVE_BIT (rt, 0x4200);
    SET_BIT (rt, 0x4000);
\end{lstlisting}

And it does exactly what we need. 
But why \TT{0x4200}? 
Perhaps that an artifact from LLVM's optimizer
\footnote{It was LLVM build 2410.2.00 bundled with Apple Xcode 4.6.3}.

Probably a compiler's optimizer error, but the generated code works correctly anyway.

You can read more about compiler anomalies here~(\myref{anomaly:Intel}).

\OptimizingXcodeIV for Thumb mode generates the same code.

\subsubsection{ARM: more about the \INS{BIC} instruction}
\myindex{ARM!\Instructions!BIC}

Let's rework the example slightly:

\begin{lstlisting}[style=customc]
int f(int a)
{
    int rt=a;

    REMOVE_BIT (rt, 0x1234);

    return rt;
};
\end{lstlisting}

Then the optimizing Keil 5.03 
in ARM mode does:

\begin{lstlisting}[style=customasmARM]
f PROC
        BIC      r0,r0,#0x1000
        BIC      r0,r0,#0x234
        BX       lr
        ENDP
\end{lstlisting}

There are two \INS{BIC} instructions, i.e., bits \TT{0x1234} are cleared in two passes.

This is because it's not possible to encode \TT{0x1234} in a \INS{BIC} instruction, 
but it's possible to encode \TT{0x1000} and \TT{0x234}.

\subsubsection{ARM64: \Optimizing GCC (Linaro) 4.9}

\Optimizing GCCcompiling for ARM64 can use the \AND instruction instead of \INS{BIC}:

\begin{lstlisting}[caption=\Optimizing GCC (Linaro) 4.9,style=customasmARM]
f:
	and	w0, w0, -513	; 0xFFFFFFFFFFFFFDFF
	orr	w0, w0, 16384	; 0x4000
	ret
\end{lstlisting}

\subsubsection{ARM64: \NonOptimizing GCC (Linaro) 4.9}

\NonOptimizing GCC generates more redundant code, but works just like optimized:

\begin{lstlisting}[caption=\NonOptimizing GCC (Linaro) 4.9,style=customasmARM]
f:
	sub	sp, sp, #32
	str	w0, [sp,12]
	ldr	w0, [sp,12]
	str	w0, [sp,28]
	ldr	w0, [sp,28]
	orr	w0, w0, 16384	; 0x4000
	str	w0, [sp,28]
	ldr	w0, [sp,28]
	and	w0, w0, -513	; 0xFFFFFFFFFFFFFDFF
	str	w0, [sp,28]
	ldr	w0, [sp,28]
	add	sp, sp, 32
	ret
\end{lstlisting}

\subsubsection{MIPS}

\lstinputlisting[caption=\Optimizing GCC 4.4.5 (IDA),style=customasmMIPS]{patterns/14_bitfields/2_set_reset/MIPS_O3_IDA_EN.lst}

\myindex{MIPS!\Instructions!ORI}

\INS{ORI} is, of course, the OR operation. \q{I} in the instruction name means that the value is embedded in the machine code.

\myindex{MIPS!\Instructions!AND}

But after that we have \AND. There is no way to use \INS{ANDI} because it's not possible to embed the 0xFFFFFDFF number
in a single instruction, so the compiler has to load 0xFFFFFDFF into register \$V0 first and then generates
\AND which takes all its values from registers.


