\section{Обрезка строк}
\newcommand{\CRLF}{\ac{CR}/\ac{LF}}

Весьма востребованная операция со строками --- это удаление некоторых символов в начале и/или конце
строки.

В этом примере, мы будем работать с функцией, удаляющей все символы перевода строки 
(\CRLF{}) в конце входной строки:

\lstinputlisting[style=customc]{\CURPATH/strtrim_RU.c}

Входной аргумент всегда возвращается на выходе, это удобно, когда вам нужно объединять
функции обработки строк в цепочки, как это сделано здесь в функции \main.

Вторая часть for() (\TT{str\_len>0 \&\& (c=s[str\_len-1])}) называется в \CCpp \q{short-circuit} 
(короткое замыкание) и это очень удобно: \InSqBrackets{\CNotes 1.3.8}.

Компиляторы \CCpp гарантируют последовательное вычисление слева направо.

Так что если первое условие не истинно после вычисления, второе никогда не будет
вычисляться.

% subsections
\subsubsection{x64}
\label{subsec:popcnt}

Немного изменим пример, расширив его до 64-х бит:

\lstinputlisting[label=popcnt_x64_example,style=customc]{patterns/14_bitfields/4_popcnt/shifts64.c}

\myparagraph{\NonOptimizing GCC 4.8.2}

Пока всё просто.

\lstinputlisting[caption=\NonOptimizing GCC 4.8.2,style=customasmx86]{patterns/14_bitfields/4_popcnt/shifts64_GCC_O0_RU.s}

\myparagraph{\Optimizing GCC 4.8.2}

\lstinputlisting[caption=\Optimizing GCC 4.8.2,numbers=left,label=shifts64_GCC_O3,style=customasmx86]{patterns/14_bitfields/4_popcnt/shifts64_GCC_O3_RU.s}

Код более лаконичный, но содержит одну необычную вещь.
Во всех примерах, что мы пока видели, инкремент значения переменной \q{rt} происходит после сравнения 
определенного бита с единицей, но здесь \q{rt} увеличивается на 1 до этого (строка 6), записывая новое значение
в регистр \EDX.

Затем, если последний бит был 1, инструкция \CMOVNE\footnote{Conditional MOVe if Not Equal (\MOV если не равно)}
(которая синонимична \CMOVNZ\footnote{Conditional MOVe if Not Zero (\MOV если не ноль)}) \IT{фиксирует} 
новое значение \q{rt}
копируя значение из \EDX (\q{предполагаемое значение rt}) 
в \EAX (\q{текущее rt} которое будет возвращено в конце функции).
Следовательно, инкремент происходит на каждом шаге цикла, т.е. 64 раза, вне всякой связи с входным
значением.

Преимущество этого кода в том, что он содержит только один условный переход (в конце цикла) вместо
двух (пропускающий инкремент \q{rt} и ещё одного в конце цикла).

И это может работать быстрее на современных CPU с предсказателем переходов: \myref{branch_predictors}.

\label{FATRET}
\myindex{x86!\Instructions!FATRET}
Последняя инструкция это \INS{REP RET} (опкод \TT{F3 C3}) 
которая также называется \INS{FATRET} в MSVC.
Это оптимизированная версия \RET, рекомендуемая AMD для вставки в конце функции, если \RET идет
сразу после условного перехода: 
\InSqBrackets{\AMDOptimization p.15}
\footnote{Больше об этом: \url{http://go.yurichev.com/17328}}.

\myparagraph{\Optimizing MSVC 2010}

\lstinputlisting[caption=MSVC 2010,style=customasmx86]{patterns/14_bitfields/4_popcnt/MSVC_2010_x64_Ox_RU.asm}

\myindex{x86!\Instructions!ROL}
Здесь используется инструкция \ROL вместо 
\SHL, которая на самом деле \q{rotate left} (прокручивать влево) 
вместо \q{shift left} (сдвиг влево),
но здесь, в этом примере, она работает так же как и  \TT{SHL}.

Об этих \q{прокручивающих} инструкциях больше читайте здесь: \myref{ROL_ROR}.

\Reg{8} здесь считает от 64 до 0. 
Это как бы инвертированная переменная $i$.

Вот таблица некоторых регистров в процессе исполнения:

\begin{center}
\begin{tabular}{ | l | l | }
\hline
\HeaderColor RDX & \HeaderColor R8 \\
\hline
0x0000000000000001 & 64 \\
\hline
0x0000000000000002 & 63 \\
\hline
0x0000000000000004 & 62 \\
\hline
0x0000000000000008 & 61 \\
\hline
... & ... \\
\hline
0x4000000000000000 & 2 \\
\hline
0x8000000000000000 & 1 \\
\hline
\end{tabular}
\end{center}

\myindex{x86!\Instructions!FATRET}
В конце видим инструкцию \INS{FATRET}, которая была описана здесь: \myref{FATRET}.

\myparagraph{\Optimizing MSVC 2012}

\lstinputlisting[caption=MSVC 2012,style=customasmx86]{patterns/14_bitfields/4_popcnt/MSVC_2012_x64_Ox_RU.asm}

\myindex{\CompilerAnomaly}
\label{MSVC2012_anomaly}
\Optimizing MSVC 2012 делает почти то же самое что и оптимизирующий MSVC 2010, но почему-то он генерирует 2 идентичных тела цикла и счетчик цикла теперь 32
вместо 64.
Честно говоря, нельзя сказать, почему. Какой-то трюк с оптимизацией? Может быть, телу цикла лучше быть
немного длиннее?

Так или иначе, такой код здесь уместен, чтобы показать, что результат компилятора
иногда может быть очень странный и нелогичный, но прекрасно работающий, конечно же.


\subsubsection{ARM64}

\myparagraph{\Optimizing GCC (Linaro) 4.9}

\myindex{Fused multiply–add}
\myindex{ARM!\Instructions!MADD}
Тут всё просто.
\TT{MADD} это просто инструкция, производящая умножение и сложение одновременно (как \TT{MLA}, 
которую мы уже видели).
Все 3 аргумента передаются в 32-битных частях X-регистров.
Действительно, типы аргументов это 32-битные \IT{int}'ы.
Результат возвращается в \TT{W0}.

\lstinputlisting[caption=\Optimizing GCC (Linaro) 4.9,style=customasmARM]{patterns/05_passing_arguments/ARM/ARM64_O3_RU.s}

Также расширим все типы данных до 64-битных \TT{uint64\_t} и попробуем:

\lstinputlisting[style=customc]{patterns/05_passing_arguments/ex64.c}

\begin{lstlisting}[style=customasmARM]
f:
	madd	x0, x0, x1, x2
	ret
main:
	mov	x1, 13396
	adrp	x0, .LC8
	stp	x29, x30, [sp, -16]!
	movk	x1, 0x27d0, lsl 16
	add	x0, x0, :lo12:.LC8
	movk	x1, 0x122, lsl 32
	add	x29, sp, 0
	movk	x1, 0x58be, lsl 48
	bl	printf
	mov	w0, 0
	ldp	x29, x30, [sp], 16
	ret

.LC8:
	.string	"%lld\n"
\end{lstlisting}

Функция \ttf{} точно такая же, только теперь используются полные части 64-битных X-регистров.
Длинные 64-битные значения загружаются в регистры по частям, это описано здесь: \myref{ARM_big_constants_loading}.

\myparagraph{\NonOptimizing GCC (Linaro) 4.9}

Неоптимизирующий компилятор выдает немного лишнего кода:

\begin{lstlisting}[style=customasmARM]
f:
	sub	sp, sp, #16
	str	w0, [sp,12]
	str	w1, [sp,8]
	str	w2, [sp,4]
	ldr	w1, [sp,12]
	ldr	w0, [sp,8]
	mul	w1, w1, w0
	ldr	w0, [sp,4]
	add	w0, w1, w0
	add	sp, sp, 16
	ret
\end{lstlisting}

Код сохраняет входные аргументы в локальном стеке на случай если кому-то (или чему-то) в этой функции
понадобится использовать регистры \TT{W0...W2}, перезаписывая оригинальные аргументы функции, которые
могут понадобится в будущем.
Это называется \IT{Register Save Area.} (\ARMPCS)
Вызываемая функция не обязана сохранять их.
Это то же что и \q{Shadow Space}: \myref{shadow_space}.

Почему оптимизирующий GCC 4.9 убрал этот, сохраняющий аргументы, код?

Потому что он провел дополнительную работу по оптимизации и сделал вывод, 
что аргументы функции не понадобятся в будущем и регистры \TT{W0...W2} также не будут использоваться.

\myindex{ARM!\Instructions!MUL}
\myindex{ARM!\Instructions!ADD}
Также мы видим пару инструкций \TT{MUL}/\TT{ADD} вместо одной \TT{MADD}.


\subsubsection{ARM + \OptimizingXcodeIV (\ARMMode)}

\lstinputlisting[caption=\OptimizingXcodeIV (\ARMMode),label=ARM_leaf_example4,style=customasmARM]{patterns/14_bitfields/4_popcnt/ARM_Xcode_O3_RU.lst}

\myindex{ARM!\Instructions!TST}
\TST это то же что и \TEST в x86.

\myindex{ARM!Optional operators!LSL}
\myindex{ARM!Optional operators!LSR}
\myindex{ARM!Optional operators!ASR}
\myindex{ARM!Optional operators!ROR}
\myindex{ARM!Optional operators!RRX}
\myindex{ARM!\Instructions!MOV}
\myindex{ARM!\Instructions!TST}
\myindex{ARM!\Instructions!CMP}
\myindex{ARM!\Instructions!ADD}
\myindex{ARM!\Instructions!SUB}
\myindex{ARM!\Instructions!RSB}
Как уже было указано~(\myref{shifts_in_ARM_mode}),
в режиме ARM нет отдельной инструкции для сдвигов.

Однако, модификаторами 
LSL (\IT{Logical Shift Left}), 
LSR (\IT{Logical Shift Right}), 
ASR (\IT{Arithmetic Shift Right}), 
ROR (\IT{Rotate Right}) и
RRX (\IT{Rotate Right with Extend}) можно дополнять некоторые инструкции, такие как \MOV, \TST,
\CMP, \ADD, \SUB, \RSB\footnote{\DataProcessingInstructionsFootNote}.

Эти модификаторы указывают, как сдвигать второй операнд, и на сколько.

\myindex{ARM!\Instructions!TST}
\myindex{ARM!Optional operators!LSL}
Таким образом, инструкция  \TT{\q{TST R1, R2,LSL R3}} здесь работает как $R1 \land (R2 \ll R3)$.

\subsubsection{ARM + \OptimizingXcodeIV (\ThumbTwoMode)}

\myindex{ARM!\Instructions!LSL.W}
\myindex{ARM!\Instructions!LSL}
Почти такое же, только здесь применяется пара инструкций \INS{LSL.W}/\TST вместо одной \TST,
ведь в режиме Thumb нельзя добавлять модификатор \LSL прямо в \TST.

\begin{lstlisting}[label=ARM_leaf_example5,style=customasmARM]
                MOV             R1, R0
                MOVS            R0, #0
                MOV.W           R9, #1
                MOVS            R3, #0
loc_2F7A
                LSL.W           R2, R9, R3
                TST             R2, R1
                ADD.W           R3, R3, #1
                IT NE
                ADDNE           R0, #1
                CMP             R3, #32
                BNE             loc_2F7A
                BX              LR
\end{lstlisting}

\subsubsection{ARM64 + \Optimizing GCC 4.9}

Возьмем 64-битный пример, который уже был здесь использован: \myref{popcnt_x64_example}.

\lstinputlisting[caption=\Optimizing GCC (Linaro) 4.8,style=customasmARM]{patterns/14_bitfields/4_popcnt/ARM64_GCC_O3_RU.s}
Результат очень похож на тот, что GCC сгенерировал для x64: \myref{shifts64_GCC_O3}.

\myindex{ARM!\Instructions!CSEL}
Инструкция \CSEL это \q{Conditional SELect} (выбор при условии). 
Она просто выбирает одну из переменных, в зависимости от флагов выставленных \TST и копирует значение в регистр \RegW{2}, содержащий переменную \q{rt}.

\subsubsection{ARM64 + \NonOptimizing GCC 4.9}

И снова будем использовать 64-битный пример, который мы использовали ранее: \myref{popcnt_x64_example}.
Код более многословный, как обычно.

\lstinputlisting[caption=\NonOptimizing GCC (Linaro) 4.8,style=customasmARM]{patterns/14_bitfields/4_popcnt/ARM64_GCC_O0_RU.s}


\subsubsection{MIPS}

\myindex{MIPS!\Registers!FCCR}

В сопроцессоре MIPS есть бит результата, который устанавливается в FPU и проверяется в CPU.

Ранние MIPS имели только один бит (с названием FCC0), а у поздних их 8 (с названием FCC7-FCC0).
Этот бит (или биты) находятся в регистре с названием FCCR.

\lstinputlisting[caption=\Optimizing GCC 4.4.5 (IDA),style=customasmMIPS]{patterns/12_FPU/3_comparison/MIPS_O3_IDA_RU.lst}

\myindex{MIPS!\Instructions!C.LT.D}
\INS{C.LT.D} сравнивает два значения. 
\GTT{LT} это условие \q{Less Than} (меньше чем).
\GTT{D} означает переменные типа \Tdouble.

В зависимости от результата сравнения, бит FCC0 устанавливается или очищается.

\myindex{MIPS!\Instructions!BC1T}
\myindex{MIPS!\Instructions!BC1F}
\INS{BC1T} проверяет бит FCC0 и делает переход, если бит выставлен.
\GTT{T} означает, что переход произойдет если бит выставлен (\q{True}).
Имеется также инструкция \INS{BC1F} которая сработает, если бит сброшен (\q{False}).

В зависимости от перехода один из аргументов функции помещается в регистр \$F0.



