\subsubsection{Encapsulation}

Encapsulation is hiding the data in the \IT{private} sections of the class, 
e.g. to allow access to them only from this class methods.

However, are there any marks in code the about the fact that some field is private and
some other---not?

No, there are no such marks.

Let's try this simple example:

\lstinputlisting[style=customc]{\CURPATH/classes/classes2_0.cpp}

Let's compile it again in MSVC 2008 with \Ox and \Obzero options and see the \TT{box::dump()} method code:

\lstinputlisting[style=customasmx86]{\CURPATH/classes/classes2_1.asm}

Here is a memory layout of the class:

\begin{center}
\begin{tabular}{ | l | l | }
\hline
  \tableheader{} \\
\hline
  +0x0 & int color \\
\hline
  +0x4 & int width \\
\hline
  +0x8 & int height \\
\hline
  +0xC & int depth \\
\hline
\end{tabular}
\end{center}

All fields are private and not allowed to be accessed from any other
function, but knowing this layout, can we create code that modifies these fields? 

To do this we'll add the \TT{hack\_oop\_encapsulation()} function, 
which is not going to compile if it looked like this:

\lstinputlisting[style=customc]{\CURPATH/classes/classes2_2_EN.cpp}

Nevertheless, if we cast the \IT{box} type to a \IT{pointer to an int array},
and we modify the array of \Tint{}-s that we have, we can succeed.

\lstinputlisting[style=customc]{\CURPATH/classes/classes2_3.cpp}

This function's code is very simple---it can be said that the function takes a pointer to an array of \Tint{}-s for input
and writes \IT{123} to the second \Tint{}:

\lstinputlisting[style=customasmx86]{\CURPATH/classes/classes2_5.asm}

Let's check how it works:

\lstinputlisting[style=customc]{\CURPATH/classes/classes2_4.cpp}

Let's run:

\begin{lstlisting}
this is box. color=1, width=10, height=20, depth=30
this is box. color=1, width=123, height=20, depth=30
\end{lstlisting}

We see that the encapsulation is just protection of class fields only in the compilation stage.

The \Cpp compiler is not allowing the generation of code that modifies protected 
fields straightforwardly, nevertheless,
it is possible with the help of \IT{dirty hacks}.

