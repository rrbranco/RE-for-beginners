\subsubsection{std::list}
\myindex{\Cpp!STL!std::list}
\label{std_list}
% TODO translate:
\index{Doubly linked list}

This is the well-known doubly-linked list: each element has two pointers, to the previous and next
elements.

This implies that the memory footprint is enlarged by 2 \glslink{word}{words} for each element (8 bytes in 32-bit environment or 16 bytes in 64-bit).

\Cpp STL just adds the \q{next} and \q{previous} pointers to the existing structure of the type that you want to 
unite in a list.

Let's work out an example with a simple 2-variable structure that we want to store in a list.

Although the \Cpp standard does not say how to implement it, 
both MSVC's and GCC's implementations are straightforward and similar, so here is only one
source code for both:

\lstinputlisting[style=customc]{\CURPATH/STL/list/2_EN.cpp}

\myparagraph{GCC}

Let's start with GCC.

When we run the example, we'll see a long dump, let's work with it in pieces.

\begin{lstlisting}
* empty list:
ptr=0x0028fe90 _Next=0x0028fe90 _Prev=0x0028fe90 x=3 y=0
\end{lstlisting}

Here we see an empty list.

Despite the fact it is empty, it has one element with garbage (\ac{AKA} \IT{dummy node})
in $x$ and $y$.
Both the \q{next} and \q{prev} pointers are pointing to the self node:

\begin{center}
\begin{tikzpicture}[thick,scale=0.85, every node/.style={scale=0.85}]
	\tikzstyle{every path}=[thick]
	\tikzstyle{node1}=[draw,rectangle,minimum height=1cm, minimum width=2.5cm, text width=2.5cm, fill=gray!20]

	\node [node1] (n1next) {Next};
	\node [node1] (n1prev) [below of=n1next] {Prev};
	\node [node1] (n1x) [below of=n1prev] {X=\garbage{}};
	\node [node1] (n1y) [below of=n1x] {Y=\garbage{}};

	\node [node1] (variable) [above=1.5cm of n1next] {\RU{Переменная}\EN{Variable} std::list};
	
	\node [node1] (itbegin) [above left=1.5cm and 0.2cm of n1next] {list.begin()};
	\node [node1] (itend) [above right=1.5cm and 0.2cm of n1next] {list.end()};
	
	\draw [->] (itbegin.south) -- (n1next.north);
	
	\draw [->] (variable.south) -- (n1next.north);
	
	\draw [->] (itend.south) -- (n1next.north);
	
	\node (ia1) [inner sep=0pt, above left=0.5cm and 0.5cm of n1next] {};
	\node (ia2) [inner sep=0pt, left=0.5cm of n1next] {};
	\node (ib1) [inner sep=0pt, above right=0.5cm and 0.5cm of n1next] {};
	\node (ib2) [inner sep=0pt, right=0.5cm of n1next] {};
	
	\node (oa1) [inner sep=0pt, above left=1cm and 1cm of n1next] {};
	\node (oa2) [inner sep=0pt, left=1cm of n1prev] {};
	\node (ob1) [inner sep=0pt, above right=1cm and 1cm of n1next] {};
	\node (ob2) [inner sep=0pt, right=1cm of n1prev] {};

	\draw [->] (n1next.east) -- (ib2) -- (ib1) -- (ia1) -- (ia2) -- (n1next.west);
	\draw [->] (n1prev.west) -- (oa2) -- (oa1) -- (ob1) -- (ob2) -- (n1prev.east);
\end{tikzpicture}
\end{center}


At this moment, the .begin and .end iterators are equal to each other.

If we push 3 elements, the list internally will be:

\begin{lstlisting}
* 3-elements list:
ptr=0x000349a0 _Next=0x00034988 _Prev=0x0028fe90 x=3 y=4
ptr=0x00034988 _Next=0x00034b40 _Prev=0x000349a0 x=1 y=2
ptr=0x00034b40 _Next=0x0028fe90 _Prev=0x00034988 x=5 y=6
ptr=0x0028fe90 _Next=0x000349a0 _Prev=0x00034b40 x=5 y=6
\end{lstlisting}

The last element is still at 0x0028fe90,
it not to be moved until the list's disposal.

It still contain random garbage in $x$ and $y$ (5 and 6). 
By coincidence, these values are the same as in the last element, but it doesn't mean that they are meaningful.

Here is how these 3 elements are stored in memory:

\begin{center}
\begin{tikzpicture}[thick,scale=0.85, every node/.style={scale=0.85}]
	\tikzstyle{every path}=[thick]
	\tikzstyle{node1}=[draw,rectangle,minimum height=1cm, minimum width=2.5cm, text width=2.5cm, fill=gray!20]

	\node [node1] (n1next) {Next};
	\node [node1] (n1prev) [below of=n1next] {Prev};
	\node [node1] (n1x) [below of=n1prev] {X=\RU{1-й элемент}\EN{1st element}};
	\node [node1] (n1y) [below of=n1x] {Y=\RU{1-й элемент}\EN{1st element}};

	\node [node1] (n2next) [right=0.5cm of n1next] {Next};
	\node [node1] (n2prev) [below of=n2next] {Prev};
	\node [node1] (n2x) [below of=n2prev] {X=\RU{2-й элемент}\EN{2nd element}};
	\node [node1] (n2y) [below of=n2x] {Y=\RU{2-й элемент}\EN{2nd element}};
	
	\node [node1] (n3next) [right=0.5cm of n2next] {Next};
	\node [node1] (n3prev) [below of=n3next] {Prev};
	\node [node1] (n3x) [below of=n3prev] {X=\RU{3-й элемент}\EN{3rd element}};
	\node [node1] (n3y) [below of=n3x] {Y=\RU{3-й элемент}\EN{3rd element}};
	
	\node [node1] (n4next) [right=0.5cm of n3next] {Next};
	\node [node1] (n4prev) [below of=n4next] {Prev};
	\node [node1] (n4x) [below of=n4prev] {X=\garbage{}};
	\node [node1] (n4y) [below of=n4x] {Y=\garbage{}};
	
	\node [node1] (variable) [above=3cm of n1next] {\RU{Переменная}\EN{Variable} std::list};
	
	\node [node1] (itbegin) [above=2cm of n1next] {list.begin()};
	\draw [->] (itbegin.south) -- (n1next.north);
	
	\draw [->] (variable.south) -- (itbegin.north);

	\node [node1] (itend) [above=2cm of n4next] {list.end()};
	\draw [->] (itend.south) -- (n4next.north);
	
	\node (ia1) [inner sep=0pt, above left=0.5cm and 0.5cm of n1next] {};
	\node (ia2) [inner sep=0pt, left=0.5cm of n1next] {};
	\node (ib1) [inner sep=0pt, above right=0.5cm and 0.5cm of n4next] {};
	\node (ib2) [inner sep=0pt, right=0.5cm of n4next] {};
	
	\node (oa1) [inner sep=0pt, above left=1cm and 1cm of n1next] {};
	\node (oa2) [inner sep=0pt, left=1cm of n1prev] {};
	\node (ob1) [inner sep=0pt, above right=1cm and 1cm of n4next] {};
	\node (ob2) [inner sep=0pt, right=1cm of n4prev] {};

	\draw [->] (n1next.east) -- (n2next.west);
	\draw [->] (n2next.east) -- (n3next.west);
	\draw [->] (n3next.east) -- (n4next.west);

	\draw [<-] (n1prev.east) -- (n2prev.west);
	\draw [<-] (n2prev.east) -- (n3prev.west);
	\draw [<-] (n3prev.east) -- (n4prev.west);
	
	\draw [->] (n4next.east) -- (ib2) -- (ib1) -- (ia1) -- (ia2) -- (n1next.west);
	\draw [->] (n1prev.west) -- (oa2) -- (oa1) -- (ob1) -- (ob2) -- (n4prev.east);
\end{tikzpicture}
\end{center}


The $l$ variable always points to the first node.

The .begin() and .end() iterators are not variables, but functions, 
which when called return pointers to the corresponding nodes.

Having a dummy element (\ac{AKA} \IT{sentinel node}) is a very popular practice in implementing doubly-linked lists.

Without it, a lot of operations may become slightly more complex and, hence, slower.

The iterator is in fact just a pointer to a node.
list.begin() and list.end() just return pointers.

\begin{lstlisting}
node at .begin:
ptr=0x000349a0 _Next=0x00034988 _Prev=0x0028fe90 x=3 y=4
node at .end:
ptr=0x0028fe90 _Next=0x000349a0 _Prev=0x00034b40 x=5 y=6
\end{lstlisting}

The fact that the last element has a pointer to the first and the first element has a pointer 
to the last one remind us of circular lists.

This is very helpful here: having a pointer to the first list element,
i.e., that is in the $l$ variable,
it is easy to get a pointer to the last one quickly, without the necessity to traverse the whole list.

Inserting an element at the end of the list is also quick, thanks to this feature.

\TT{operator--} and \TT{operator++} just set the current iterator's value to the \\
\TT{current\_node->prev} or \TT{current\_node->next} values.

The reverse iterators (.rbegin, .rend) work just as the same, but in reverse.

\TT{operator*} just returns a pointer to the point 
in the node structure, where the user's structure starts, i.e., a pointer to the 
first element of the structure ($x$).

The list insertion and deletion are trivial: just allocate a new node (or deallocate) and update all pointers
to be valid.

That's why an iterator may become invalid after element deletion: 
it may still point to the node that has been already deallocated.
This is also called a \IT{dangling pointer}.

And of course, the information from the freed node (to which iterator still points) 
cannot be used anymore.

The GCC implementation (as of 4.8.1) doesn't store the current size of the list: this implies a slow .size() method:
it has to traverse the whole list to count the elements, because it doesn't have any other way to get the information.

This means that this operation is $O(n)$, i.e., it steadily gets slower as the list grows.

\lstinputlisting[caption=\Optimizing GCC 4.8.1 -fno-inline-small-functions,style=customasmx86]{\CURPATH/STL/list/GCC_EN.asm}

\lstinputlisting[caption=The whole output]{\CURPATH/STL/list/GCC.txt}

\myparagraph{MSVC}
\label{MSVC_std_list}

MSVC's implementation (2012) is just the same, but it also stores the current size of the list.

This implies that the .size() method is very fast ($O(1)$): it just reads one value from memory.

On the other hand, the size variable must be updated at each insertion/deletion.

MSVC's implementation is also slightly different in the way it arranges the nodes:

\begin{center}
\begin{tikzpicture}[thick,scale=0.85, every node/.style={scale=0.85}]
	\tikzstyle{every path}=[thick]
	\tikzstyle{node1}=[draw,rectangle,minimum height=1cm, minimum width=2.5cm, text width=2.5cm, fill=gray!20]

	\node [node1] (n1next) {Next};
	\node [node1] (n1prev) [below of=n1next] {Prev};
	\node [node1] (n1x) [below of=n1prev] {X=\garbage{}};
	\node [node1] (n1y) [below of=n1x] {Y=\garbage{}};

	\node [node1] (n2next) [right=0.5cm of n1next] {Next};
	\node [node1] (n2prev) [below of=n2next] {Prev};
	\node [node1] (n2x) [below of=n2prev] {X=\RU{1-й элемент}\EN{1st element}};
	\node [node1] (n2y) [below of=n2x] {Y=\RU{1-й элемент}\EN{1st element}};
	
	\node [node1] (n3next) [right=0.5cm of n2next] {Next};
	\node [node1] (n3prev) [below of=n3next] {Prev};
	\node [node1] (n3x) [below of=n3prev] {X=\RU{2-й элемент}\EN{2nd element}};
	\node [node1] (n3y) [below of=n3x] {Y=\RU{2-й элемент}\EN{2nd element}};
	
	\node [node1] (n4next) [right=0.5cm of n3next] {Next};
	\node [node1] (n4prev) [below of=n4next] {Prev};
	\node [node1] (n4x) [below of=n4prev] {X=\RU{3-й элемент}\EN{3rd element}};
	\node [node1] (n4y) [below of=n4x] {Y=\RU{3-й элемент}\EN{3rd element}};
	
	\node [node1] (variable) [above=3cm of n1next] {\RU{Переменная}\EN{Variable} std::list};

	\node [node1] (itend) [above=2cm of n1next] {list.end()};
	\draw [->] (itend.south) -- (n1next.north);
	
	\draw [->] (variable.south) -- (itend.north);
	
	\node [node1] (itbegin) [above=2cm of n2next] {list.begin()};
	\draw [->] (itbegin.south) -- (n2next.north);

	\node (ia1) [inner sep=0pt, above left=0.5cm and 0.5cm of n1next] {};
	\node (ia2) [inner sep=0pt, left=0.5cm of n1next] {};
	\node (ib1) [inner sep=0pt, above right=0.5cm and 0.5cm of n4next] {};
	\node (ib2) [inner sep=0pt, right=0.5cm of n4next] {};
	
	\node (oa1) [inner sep=0pt, above left=1cm and 1cm of n1next] {};
	\node (oa2) [inner sep=0pt, left=1cm of n1prev] {};
	\node (ob1) [inner sep=0pt, above right=1cm and 1cm of n4next] {};
	\node (ob2) [inner sep=0pt, right=1cm of n4prev] {};

	\draw [->] (n1next.east) -- (n2next.west);
	\draw [->] (n2next.east) -- (n3next.west);
	\draw [->] (n3next.east) -- (n4next.west);

	\draw [<-] (n1prev.east) -- (n2prev.west);
	\draw [<-] (n2prev.east) -- (n3prev.west);
	\draw [<-] (n3prev.east) -- (n4prev.west);
	
	\draw [->] (n4next.east) -- (ib2) -- (ib1) -- (ia1) -- (ia2) -- (n1next.west);
	\draw [->] (n1prev.west) -- (oa2) -- (oa1) -- (ob1) -- (ob2) -- (n4prev.east);
\end{tikzpicture}
\end{center}


GCC has its dummy element at the end of the list, while MSVC's is at the beginning.

\lstinputlisting[caption=\Optimizing MSVC 2012 /Fa2.asm /GS- /Ob1,style=customasmx86]{\CURPATH/STL/list/MSVC_EN.asm}

Unlike GCC, MSVC's code allocates the dummy element at the start of the function with the help of the \q{Buynode} function,
it is also used to allocate the rest of the nodes (
GCC's code allocates the first element in the local stack).

\lstinputlisting[caption=The whole output]{\CURPATH/STL/list/MSVC.txt}

\myparagraph{C++11 std::forward\_list}
\myindex{\Cpp!STL!std::forward\_list}

The same thing as std::list, but singly-linked one, i.e., having only the \q{next} field at each node.

It has a smaller memory footprint, but also don't offer the ability to traverse list backwards.

