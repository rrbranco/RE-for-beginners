\subsubsection{std::vector}
\myindex{\Cpp!STL!std::vector}

We would call \TT{std::vector} a safe wrapper of the \ac{PODT} C array.
Internally it is somewhat similar to \TT{std::string} (\myref{std_string}):
it has a pointer to the allocated buffer, a pointer to the end of the array, and a pointer to the end of the allocated buffer.

The array's elements  lie in memory adjacently to each other, just like in a normal array (\myref{arrays}).
\myindex{\Cpp!C++11}
In C++11 there is a new method called \TT{.data()} , that returns a pointer to the buffer, like \TT{.c\_str()} in \TT{std::string}.

The buffer allocated in the \gls{heap} can be larger than the array itself.

Both MSVC's and GCC's implementations are similar, just the names of the structure's fields are slightly different\footnote
{GCC internals: \url{http://go.yurichev.com/17086}}, so here is one source
code that works for both compilers.
Here is again the C-like code for dumping the structure of \TT{std::vector}:

\lstinputlisting[style=customc]{\CURPATH/STL/vector/2_EN.cpp}

Here is the output of this program when compiled in MSVC:

\lstinputlisting{\CURPATH/STL/vector/MSVC.txt}

As it can be seen, there is no allocated buffer when \main starts.
After the first \TT{push\_back()} call, a buffer is allocated.
And then, after each \TT{push\_back()} 
call, both array size and buffer size (\IT{capacity}) are increased.
But the buffer address changes as well, because \TT{push\_back()} 
reallocates the buffer in the \gls{heap} each time.
It is costly operation, that's why it is very important to predict the size of the array in the future and reserve 
enough space for it with the \TT{.reserve()} method.

The last number is garbage: there are no array elements at this point, so a random number is printed.
This illustrates the fact that \TT{operator[]} of 
\TT{std::vector} does not check of the index is in the array's bounds.
The slower \TT{.at()} 
method, however, does this checking and throws an \TT{std::out\_of\_range} 
exception in case of error.

Let's see the code:

\lstinputlisting[caption=MSVC 2012 /GS- /Ob1,style=customasmx86]{\CURPATH/STL/vector/MSVC.asm}

We see how the \TT{.at()} 
method checks the bounds and throws an exception in case of error.
The number that the last \printf call prints is just taken from the memory, without any checks.

One may ask, why not use the variables like \q{size} and \q{capacity}, 
like it was done in \TT{std::string}.
Supposedly, this was done for faster bounds checking.

\myindex{Inline code}

The code GCC generates is in general almost the same, but the \TT{.at()} method is inlined:

\lstinputlisting[caption=GCC 4.8.1 -fno-inline-small-functions -O1,style=customasmx86]{\CURPATH/STL/vector/GCC.asm}

\TT{.reserve()} is inlined as well.
It calls \TT{new()} if the buffer is too small for the new size, calls \TT{memmove()} 
to copy the contents of the buffer, and calls \TT{delete()} to free the old buffer.

Let's also see what the compiled program outputs if compiled with GCC:

\lstinputlisting{\CURPATH/STL/vector/GCC.txt}

We can spot that the buffer size grows in a different way that in MSVC.

Simple experimentation shows that in MSVC's implementation the buffer grows by \textasciitilde{}50\% each time it needs to be enlarged,
while GCC's code enlarges it by 100\% each time, i.e., doubles it.

