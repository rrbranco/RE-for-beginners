\subsection{Версия структуры в Си}

Многие программисты под Windows видели это в MSDN:

\begin{lstlisting}
SizeOfStruct
    The size of the structure, in bytes. This member must be set to sizeof(SYMBOL_INFO).
\end{lstlisting}

( \url{https://msdn.microsoft.com/en-us/library/windows/desktop/ms680686(v=vs.85).aspx} )

Некоторые структуры вроде \IT{SYMBOL\_INFO} действительно начинаются с такого поля. Почему?
Это что-то вроде версии структуры.

Представьте, что у вас есть ф-ция, рисующая круги.
Она берет один аргумент - указатель на структуру с двумя полями: X и Y.
И затем цветные дисплеи наводнили рынок, где-то в 80-х. И вы хотите добавить аргумент \IT{цвет} в ф-цию.
Но, скажем так, вы не можете добавить еще один аргумент в нее (множество ПО используют ваше \ac{API} и его нельзя
перекомпилировать).
И если какое-то старое ПО использует ваше \ac{API} с цветным дисплеем,
пусть ваша ф-ция рисует круг в цветах по умолчанию (черный и белый).

Позже вы добавляете еще одну возможность: круг может быть закрашен, и можно выбирать тип заливки.

Вот одно из решений проблемы:

\begin{lstlisting}[style=customc]
#include <stdio.h>

struct ver1
{
	size_t SizeOfStruct;
	int coord_X;
	int coord_Y;
};

struct ver2
{
	size_t SizeOfStruct;
	int coord_X;
	int coord_Y;
	int color;
};

struct ver3
{
	size_t SizeOfStruct;
	int coord_X;
	int coord_Y;
	int color;
	int fill_brush_type; // §0 - не заливать круг§
};

void draw_circle(struct ver3 *s) // §здесь используется самая последняя версия структуры§
{
	// §мы полагаем что в SizeOfStruct всегда присутствуют поля coord\_X и coord\_Y§
	printf ("§Собираемся рисовать круг на§ %d:%d\n", s->coord_X, s->coord_Y);

	if (s->SizeOfStruct>=sizeof(int)*4)
	{
		// §это минимум ver2, поле цвета присутствует§
		printf ("§Собираемся установить цвет§ %d\n", s->color);
	}

	if (s->SizeOfStruct>=sizeof(int)*5)
	{
		// §это минимум ver3, присутствует поле с типом заливки§
		printf ("§Мы собираемся залить его используя тип заливки§ %d\n", s->fill_brush_type);
	}
};

// ранее ПО
void call_as_ver1()
{
	struct ver1 s;
	s.SizeOfStruct=sizeof(s);
	s.coord_X=123;
	s.coord_Y=456;
	printf ("** %s()\n", __FUNCTION__);
	draw_circle(&s);
};

// следующая версия
void call_as_ver2()
{
	struct ver2 s;
	s.SizeOfStruct=sizeof(s);
	s.coord_X=123;
	s.coord_Y=456;
	s.color=1;
	printf ("** %s()\n", __FUNCTION__);
	draw_circle(&s);
};

// самая поздняя, наиболее продвинутая версия
void call_as_ver3()
{
	struct ver3 s;
	s.SizeOfStruct=sizeof(s);
	s.coord_X=123;
	s.coord_Y=456;
	s.color=1;
	s.fill_brush_type=3;
	printf ("** %s()\n", __FUNCTION__);
	draw_circle(&s);
};

int main()
{
	call_as_ver1();
	call_as_ver2();
	call_as_ver3();
};
\end{lstlisting}

Другими словами, поле \IT{SizeOfStruct} берет на себя роль поля \IT{версия структуры}.
Это может быть перечисляемый тип (1, 2, 3, итд.), но установка поля \IT{SizeOfStruct} равным \IT{sizeof(struct...)},
это лучше защищено от ошибок.

В Си++ эта проблема решается \IT{наследованием} (\myref{cpp_inheritance}).
Просто расширяете ваш базовый класс (назовем его \IT{Circle}),
и затем вам нужен \IT{ColoredCircle}, а потом \IT{FilledColoredCircle}, и так далее.
Текущая \IT{версия} объекта (или более точно, текущий \IT{тип}) будет определяться при помощи \ac{RTTI} в Си++.

Так что если вы где-то в \ac{MSDN} видите \IT{SizeOfStruct} --- вероятно, эта структура уже расширялась в прошлом,
как минимум один раз.

