\section{OpenMP}
\label{openmp}
\myindex{OpenMP}

OpenMP это один из простейших способов распараллелить работу простого алгоритма.

\myindex{Bitcoin}
В качестве примера, попытаемся написать программу для вычисления криптографического \IT{nonce}.
В моем простейшем примере, \IT{nonce} это число, добавляемое к нешифрованному тексту, чтобы получить
хэш с какой-то особенностью.
Например, на одной из стадии, протокол Bitcoin требует найти такую \IT{nonce}, чтобы в результате
хэширования подряд шли определенное количество нулей.

Это еще называется \q{proof of work}
\footnote{\href{http://go.yurichev.com/17100}{wikipedia}} 
(т.е. система доказывает, что она произвела какие-то очень ресурсоёмкие вычисления и затратила
время на это).

\myindex{SHA512}
Мой пример не связан с Bitcoin, 
он будет пытаться добавлять числа к строке \q{hello, world!\_}
чтобы найти такое число, при котором строка вида 
\q{hello, world!\_<number>} после хеширования алгоритмом SHA512 будет содержать как минимум 3 нулевых
байта в начале.

Ограничимся перебором всех чисел в интервале
0..INT32\_MAX-1 (т.е., \TT{0x7FFFFFFE} или 2147483646).

Алгоритм очень простой:

\lstinputlisting[style=customc]{advanced/700_openmp/openmp_example.c}

\TT{check\_nonce()} просто добавляет число к строке, хеширует алгоритмом SHA512 и проверяет 
3 нулевых байта в начале.

Очень важная часть кода --- это:

\begin{lstlisting}[style=customc]
	#pragma omp parallel for
	for (i=0; i<INT32_MAX; i++)
		check_nonce (i);
\end{lstlisting}

Да, вот настолько просто, без \TT{\#pragma} мы просто вызываем
 \TT{check\_nonce()} для каждого числа от 0 до 
\TT{INT32\_MAX} (\TT{0x7fffffff} или 2147483647).
С \TT{\#pragma}, компилятор добавляет специальный код, который разрежет интервал цикла
на меньшие интервалы, чтобы запустить их на доступных ядрах \ac{CPU}
\footnote{N.B.: Это намеренно упрощенный пример, но на практике, 
применение OpenMP может быть труднее и сложнее}.

Пример может быть скомпилирован
\footnote{файлы sha512.(c|h) и u64.h можно взять из библиотеки OpenSSL:
\url{http://go.yurichev.com/17324}}
в MSVC 2012:
% FIXME1: \footnote{other source code files can be downloaded here: ...} 

\begin{lstlisting}
cl openmp_example.c sha512.obj /openmp /O1 /Zi /Faopenmp_example.asm
\end{lstlisting}

Или в GCC:

\begin{lstlisting}
gcc -fopenmp 2.c sha512.c -S -masm=intel
\end{lstlisting}

\subsection{MSVC}

Вот как MSVC 2012 генерирует главный цикл:

\lstinputlisting[caption=MSVC 2012,style=customasmx86]{advanced/700_openmp/MSVC_loop.asm}

Функции с префиксом \TT{vcomp} связаны с OpenMP и находятся в файле vcomp*.dll.
Так что тут запускается группа тредов.

Посмотрим на \TT{\_main\$omp\$1}:

\lstinputlisting[caption=MSVC 2012,style=customasmx86]{advanced/700_openmp/MSVC_1.asm}

Эта функция будет запущена $n$ раз параллельно, где $n$ это число ядер \ac{CPU}.\\
\TT{vcomp\_for\_static\_simple\_init()} вычисляет интервал для конструкта
for() для текущего треда, в зависимости от текущего номера треда.
Значения начала и конца цикла записаны в локальных переменных \TT{\$T1} и \TT{\$T2}.
Вы также можете заметить \TT{7ffffffeh} (или 2147483646) как аргумент для функции 
\TT{vcomp\_for\_static\_simple\_init()}
это количество итераций всего цикла, оно будет поделено на равные части.

Потом мы видим новый цикл с вызовом функции \TT{check\_nonce()} делающей всю работу.

Добавил также немного кода в начале функции \TT{check\_nonce()} для сбора статистики,
с какими аргументами эта функция вызывалась.

Вот что мы видим если запустим:

\begin{lstlisting}
threads=4
...
checked=2800000
checked=3000000
checked=3200000
checked=3300000
found (thread 3): [hello, world!_1611446522]. seconds spent=3
__min[0]=0x00000000 __max[0]=0x1fffffff
__min[1]=0x20000000 __max[1]=0x3fffffff
__min[2]=0x40000000 __max[2]=0x5fffffff
__min[3]=0x60000000 __max[3]=0x7ffffffe
\end{lstlisting}

Да, результат правильный, первые 3 байта это нули:

\begin{lstlisting}
C:\...\sha512sum test
000000f4a8fac5a4ed38794da4c1e39f54279ad5d9bb3c5465cdf57adaf60403
df6e3fe6019f5764fc9975e505a7395fed780fee50eb38dd4c0279cb114672e2 *test
\end{lstlisting}

Оно требует $\approx2..3$ секунды на 4-х ядерном Intel Xeon E3-1220 3.10 GHz.

В task manager мы видим 5 тредов: один главный тред + 4 запущенных.
Никаких оптимизаций не было сделано, чтобы оставить этот пример в как можно более простом виде.
Но, наверное, этот алгоритм может работать быстрее.
У моего \ac{CPU} 4 ядра, вот почему OpenMP запустил именно 4 треда.

Глядя на таблицу статистики, можно легко увидеть, что цикл был разделен очень точно на 4 равных части.
Ну хорошо, почти равных, если не учитывать последний бит.

Имеются также прагмы и для \glslink{atomic operation}{атомарных операций}.

Посмотрим, как вот этот код будет скомпилирован:

\begin{lstlisting}[style=customc]
	#pragma omp atomic
	checked++;

	#pragma omp critical
	if ((checked % 100000)==0)
		printf ("checked=%d\n", checked);
\end{lstlisting}

\lstinputlisting[caption=MSVC 2012,style=customasmx86]{advanced/700_openmp/MSVC_2.asm}

Как выясняется, функция \TT{vcomp\_atomic\_add\_i4()} 
в vcomp*.dll это просто крохотная функция имеющая инструкцию
 \TT{LOCK XADD}\footnote{О префиксе LOCK читайте больше: \myref{x86_lock}}.

\TT{vcomp\_enter\_critsect()} в конце концов вызывает функцию win32 \ac{API} \\
\TT{EnterCriticalSection()}
\footnote{О критических секциях читайте больше тут: \myref{critical_sections}}.

\subsection{GCC}

GCC 4.8.1 выдает программу показывающую точно такую же таблицу со статистикой, 
так что, реализация GCC делит цикл на части точно так же.

\begin{lstlisting}[caption=GCC 4.8.1,style=customasmx86]
	mov	edi, OFFSET FLAT:main._omp_fn.0
	call	GOMP_parallel_start
	mov	edi, 0
	call	main._omp_fn.0
	call	GOMP_parallel_end
\end{lstlisting}

В отличие от реализации MSVC, то, что делает код GCC, это запускает 3 треда, но также запускает 
четвертый прямо в текущем треде. Так что здесь всего 4 треда а не 5 как в случае
с MSVC.

Вот функция \TT{main.\_omp\_fn.0}:
 
\lstinputlisting[caption=GCC 4.8.1,style=customasmx86]{advanced/700_openmp/GCC_1.asm}

Здесь мы видим это деление явно: вызывая 
\TT{omp\_get\_num\_threads()} и \TT{omp\_get\_thread\_num()}
мы получаем количество запущенных тредов, а также номер текущего треда, и затем определяем интервал цикла.
И затем запускаем \TT{check\_nonce()}.

GCC также вставляет инструкцию \TT{LOCK ADD} 
прямо в том месте кода, где MSVC сгенерировал вызов отдельной функции в DLL:

\lstinputlisting[caption=GCC 4.8.1,style=customasmx86]{advanced/700_openmp/GCC_2.asm}

Функции с префиксом GOMP это часть библиотеки 
GNU OpenMP.
В отличие от vcomp*.dll, её исходный код свободно доступен: 
\href{http://go.yurichev.com/17102}{GitHub}.

