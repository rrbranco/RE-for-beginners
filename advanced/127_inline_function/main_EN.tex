\section{Inline functions}
\myindex{Inline code}
\label{inline_code}

Inlined code is when the compiler, instead of placing a call instruction to a small or tiny function,
just places its body right in-place.

\lstinputlisting[caption=A simple example,style=customc]{\CURPATH/1.c}

\dots is compiled in very predictable way, however, if we turn on GCC optimizations (\Othree), we'll see:

\lstinputlisting[caption=\Optimizing GCC 4.8.1,style=customasmx86]{\CURPATH/1.s}

(Here the division is performed by multiplication(\myref{sec:divisionbymult}).)

Yes, our small function \TT{celsius\_to\_fahrenheit()} has just been placed before the \printf call.

Why? It can be faster than executing this function's code plus the overhead of calling/returning.

Modern optimizing compilers are choosing small functions for inlining automatically.
But it's possible to force compiler additionally to inline some function, if to mark
it with the \q{inline} keyword in its declaration.

% sections
\section{Java}
\myindex{Java}

% sections:
\EN{% TODO proof-reading
\subsection{Introduction}

\newcommand{\JADURL}{\url{http://varaneckas.com/jad/}}

There are some well-known decompilers for Java (or \ac{JVM} bytecode in general)
\footnote{For example, JAD: \JADURL}.

The reason is the decompilation of \ac{JVM}-bytecode is somewhat easier 
than for lower level x86 code:

\begin{itemize}
\item There is much more information about the data types.
\item The \ac{JVM} memory model is much more rigorous and outlined.
\item The Java compiler don't do any optimizations (the \ac{JVM} \ac{JIT} does them at runtime),
      so the bytecode in the class files is usually pretty readable.
      
\end{itemize}

When can the knowledge of \ac{JVM} be useful?

\newcommand{\URLListOfJVMLangs}{\url{http://en.wikipedia.org/wiki/List_of_JVM_languages}}

\begin{itemize}
\item Quick-and-dirty patching tasks of class files without the need to recompile the decompiler's results.
\item Analyzing obfuscated code.
\item Building your own obfuscator.
\item Building a compiler codegenerator (back-end) targeting \ac{JVM} (like Scala, Clojure, etc.
      \footnote{Full list: \URLListOfJVMLangs}).
      
\end{itemize}

Let's start with some simple pieces of code.
JDK 1.7 is used everywhere, unless mentioned otherwise.

This is the command used to decompile class files everywhere:\\
\GTT{javap -c -verbose}.

This is the book I used while preparing all examples: \JavaBook.

}
\RU{% TODO proof-reading
\subsection{Введение}

\newcommand{\JADURL}{\url{http://varaneckas.com/jad/}}

Есть немало известных декомпиляторов для Java (или для \ac{JVM}-байткода вообще)
\footnote{Например, JAD: \JADURL}.

Причина в том что декомпиляция \ac{JVM}-байткода проще чем низкоуровневого x86-кода:

\begin{itemize}
\item Здесь намного больше информации о типах.
      
\item Модель памяти в \ac{JVM} более строгая и очерченная.
\item Java-компилятор не делает никаких оптимизаций (это делает \ac{JVM} \ac{JIT} во время 
       исполнения), так что байткод в class-файлах легко читаем.
\end{itemize}

Когда знания \ac{JVM}-байткода могут быть полезны?

\newcommand{\URLListOfJVMLangs}{\url{http://en.wikipedia.org/wiki/List_of_JVM_languages}}

\begin{itemize}
\item Мелкая/несложная работа по патчингу class-файлов без необходимости снова компилировать результаты 
      декомпилятора.
\item Анализ обфусцированного кода.
\item Создание вашего собственного обфускатора.
\item Создание кодегенератора компилятора (back-end), создающего код для \ac{JVM} (как Scala, Clojure, итд
      \footnote{Полный список: \URLListOfJVMLangs}).
\end{itemize}

Начнем с простых фрагментов кода.

Если не указано иное, везде используется JDK 1.7.

Эта команда использовалась везде для декомпиляции class-файлов:\\
\GTT{javap -c -verbose}.

Эта книга использовалась мною для подготовки всех примеров: \JavaBook.

}
\ES{\subsection{Introducción}

Hay algunos descompiladores conocidos para Java (o bytecode JVM en general).
La razón es porque la descompilación de bytecode JVM es algo mas fácil que código x86 de bajo nivel:

\begin{itemize}
\item Hay mucha mas información sobre los tipos de datos.

\item El modelo de memoria de JVM es mucho mas riguroso y delineado.

\item El compilador Java no hace ninguna optimización (la JVM las hace en tiempo de ejecución), así que el byte
		code de archivos class es usualmente bastante legible.

\end{itemize}

Cuándo el conocimiento de JVM es útil?

\item Tareas rápidas y sucias de parcheo de archivos class

}
\DE{% TODO proof-reading
\subsection{Einführung}

\newcommand{\JADURL}{\url{http://varaneckas.com/jad/}}

Es gibt einige bekannte Decompiler für Java (oder \ac{JVM}-Bytecode
allgemein)\footnote{Beispielsweise JAD: \JADURL}.

Der Grund ist, dass das Dekompilieren von \ac{JVM}-Bytecode einfacher ist als
von Low-Level x86-Code:

\begin{itemize}
\item Es gibt sehr viel mehr Informationen über die Datentypen.
\item Das \ac{JVM}-Speichermodell ist sehr viel strenger und genauer beschrieben.
\item Der Java-Compiler führt keine Optimierungen durch (dies macht der \ac{JVM}
\ac{JIT} während der Laufzeit, do dass der Bytecode in der Klassendatei normalerweise
gut lesbar ist).
\end{itemize}

Wann kann das Wissen über \ac{JVM} nützlich sein?

\newcommand{\URLListOfJVMLangs}{\url{http://en.wikipedia.org/wiki/List_of_JVM_languages}}

\begin{itemize}
\item Quick-and-dirty-Patchen von Klassendateien ohne das neukompilieren der Decompiler-Ergebnisse.
\item Analysieren von obfuskierten Code.
\item Erstellen eines eigenen Obfuscators.
\item Erstellen eines Compiler Code-Generators (back-end) mit dem Ziek \ac{JVM}
(wie Scala, Clojure, usw.) \footnote{vollständige Liste: \URLListOfJVMLangs}).
\end{itemize}

Starten wir mit einigen einfachen Code-Beispielen.
Wenn nicht anders erwähnt wird JDK 1.7 verwendet.

Das folgende Kommando wird verwendet um Klassen zu decompilieren:


\GTT{javap -c -verbose}.

Das folgende Buch habe ich während der Vorbereitung der Beispiele genutzt
\JavaBook.
}

\EN{% TODO proof-reading
\subsection{Returning a value}

Probably the simplest Java function is the one which returns some value.

Oh, and we must keep in mind that there are no \q{free} functions in Java in common sense,
they are \q{methods}. 

Each method is related to some class, so it's not possible to define
a method outside of a class.

But we'll call them \q{functions} anyway, for simplicity.

\begin{lstlisting}[style=customjava]
public class ret
{
	public static int main(String[] args) 
	{
		return 0;
	}
}
\end{lstlisting}

Let's compile it:

\begin{lstlisting}
javac ret.java
\end{lstlisting}

\dots and decompile it using the standard Java utility:

\begin{lstlisting}
javap -c -verbose ret.class
\end{lstlisting}

And we get:

\begin{lstlisting}[caption=JDK 1.7 (excerpt)]
  public static int main(java.lang.String[]);
    flags: ACC_PUBLIC, ACC_STATIC
    Code:
      stack=1, locals=1, args_size=1
         0: iconst_0      
         1: ireturn       
\end{lstlisting}

The Java developers decided that 0 is one of the busiest constants in programming, 
so there is a separate short one-byte \GTT{iconst\_0} instruction which pushes 0

\footnote{Just like in MIPS, where a separate register for zero constant exists: \myref{MIPS_zero_register}.}.
There are also \GTT{iconst\_1} (which pushes 1), \GTT{iconst\_2}, etc., up to \GTT{iconst\_5}.

There is also \GTT{iconst\_m1} which pushes -1.

The stack is used in JVM for passing data to called functions and also for returning values.
So \GTT{iconst\_0} pushes 0 into the stack.
\GTT{ireturn} returns an integer value (\IT{i} in name means \IT{integer}) from the \ac{TOS}.

Let's rewrite our example slightly, now we return 1234:

\begin{lstlisting}[style=customjava]
public class ret
{
	public static int main(String[] args)
	{
		return 1234;
	}
}
\end{lstlisting}

\dots we get:

\begin{lstlisting}[caption=JDK 1.7 (excerpt)]
  public static int main(java.lang.String[]);
    flags: ACC_PUBLIC, ACC_STATIC
    Code:
      stack=1, locals=1, args_size=1
         0: sipush        1234
         3: ireturn       
\end{lstlisting}

\GTT{sipush} (\IT{short integer}) pushes 1234 into the stack.
\IT{short} in name implies a 16-bit value is to be pushed. 
The number 1234 indeed fits well in a 16-bit value.

What about larger values?

\begin{lstlisting}[style=customjava]
public class ret
{
	public static int main(String[] args) 
	{
		return 12345678;
	}
}
\end{lstlisting}

\begin{lstlisting}[caption=Constant pool]
...
   #2 = Integer            12345678
...
\end{lstlisting}

\begin{lstlisting}
  public static int main(java.lang.String[]);
    flags: ACC_PUBLIC, ACC_STATIC
    Code:
      stack=1, locals=1, args_size=1
         0: ldc           #2                  // int 12345678
         2: ireturn       
\end{lstlisting}

It's not possible to encode a 32-bit number in a JVM instruction opcode, 
the developers didn't leave such possibility.

So the 32-bit number 12345678 is stored in so called \q{constant pool} which is, let's say,
the library of most used constants (including strings, objects, etc.).

This way of passing constants is not unique to JVM.

MIPS, ARM and other RISC CPUs also can't encode a 32-bit number
in a 32-bit opcode, so the RISC CPU code (including MIPS and ARM) has to construct the value 
in several steps, or to keep it in the data segment:
\myref{ARM_big_constants}, \myref{MIPS_big_constants}.

MIPS code also traditionally has a constant pool, named \q{literal pool}, the segments
are called \q{.lit4} (for 32-bit single precision floating point number constants) and \q{.lit8}
(for 64-bit double precision floating point number constants).

Let's try some other data types!

Boolean:

\begin{lstlisting}[style=customjava]
public class ret
{
	public static boolean main(String[] args) 
	{
		return true;
	}
}
\end{lstlisting}

\begin{lstlisting}
  public static boolean main(java.lang.String[]);
    flags: ACC_PUBLIC, ACC_STATIC
    Code:
      stack=1, locals=1, args_size=1
         0: iconst_1      
         1: ireturn       
\end{lstlisting}

This JVM bytecode is no different from one returning integer 1.

32-bit data slots in the stack are also used here for boolean values, like in \CCpp.

But one could not use returned boolean value as integer or vice versa --- type information is stored in the class file and checked at runtime.

It's the same story with a 16-bit \IT{short}:

\begin{lstlisting}[style=customjava]
public class ret
{
	public static short main(String[] args) 
	{
		return 1234;
	}
}
\end{lstlisting}

\begin{lstlisting}
  public static short main(java.lang.String[]);
    flags: ACC_PUBLIC, ACC_STATIC
    Code:
      stack=1, locals=1, args_size=1
         0: sipush        1234
         3: ireturn       
\end{lstlisting}

\dots and \IT{char}!

\begin{lstlisting}[style=customjava]
public class ret
{
	public static char main(String[] args) 
	{
		return 'A';
	}
}
\end{lstlisting}

\begin{lstlisting}
  public static char main(java.lang.String[]);
    flags: ACC_PUBLIC, ACC_STATIC
    Code:
      stack=1, locals=1, args_size=1
         0: bipush        65
         2: ireturn       
\end{lstlisting}

\INS{bipush} means \q{push byte}.
Needless to say that a \IT{char} in Java is 16-bit UTF-16 character, 
and it's equivalent to \IT{short}, but the ASCII code of the \q{A} character is 65, and it's possible
to use the instruction for pushing a byte in the stack.

Let's also try a \IT{byte}:

\begin{lstlisting}[style=customjava]
public class retc
{
	public static byte main(String[] args) 
	{
		return 123;
	}
}
\end{lstlisting}

\begin{lstlisting}
  public static byte main(java.lang.String[]);
    flags: ACC_PUBLIC, ACC_STATIC
    Code:
      stack=1, locals=1, args_size=1
         0: bipush        123
         2: ireturn       
\end{lstlisting}

One may ask, why bother with a 16-bit \IT{short} data type which internally works
as a 32-bit integer?

Why use a \IT{char} data type if it is the same as a \IT{short} data type?

The answer is simple: for data type control and source code readability.

A \IT{char} may essentially be the same as a \IT{short}, but we quickly grasp that it's a placeholder for
an UTF-16 character, and not for some other integer value.

When using \IT{short}, we show everyone that the variable's range is limited by 16 bits.

It's a very good idea to use the \IT{boolean} type where needed to, 
instead of the C-style \IT{int}.

There is also a 64-bit integer data type in Java:

\begin{lstlisting}[style=customjava]
public class ret3
{
	public static long main(String[] args)
	{
		return 1234567890123456789L;
	}
}
\end{lstlisting}

\begin{lstlisting}[caption=Constant pool]
...
   #2 = Long               1234567890123456789l
...
\end{lstlisting}

\begin{lstlisting}
  public static long main(java.lang.String[]);
    flags: ACC_PUBLIC, ACC_STATIC
    Code:
      stack=2, locals=1, args_size=1
         0: ldc2_w        #2                  // long 1234567890123456789l
         3: lreturn       
\end{lstlisting}

The 64-bit number is also stored in a constant pool, \INS{ldc2\_w} loads it and \INS{lreturn} 
(\IT{long return}) returns it.

The \INS{ldc2\_w} instruction is also used to load double precision floating point numbers 
(which also occupy 64 bits) from a constant pool:

\begin{lstlisting}[style=customjava]
public class ret
{
	public static double main(String[] args)
	{
		return 123.456d;
	}
}
\end{lstlisting}

\begin{lstlisting}[caption=Constant pool]
...
   #2 = Double             123.456d
...
\end{lstlisting}

\begin{lstlisting}
  public static double main(java.lang.String[]);
    flags: ACC_PUBLIC, ACC_STATIC
    Code:
      stack=2, locals=1, args_size=1
         0: ldc2_w        #2                  // double 123.456d
         3: dreturn       
\end{lstlisting}

\INS{dreturn} stands for \q{return double}.

And finally, a single precision floating point number:

\begin{lstlisting}[style=customjava]
public class ret
{
	public static float main(String[] args)
	{
		return 123.456f;
	}
}
\end{lstlisting}

\begin{lstlisting}[caption=Constant pool]
...
   #2 = Float              123.456f
...
\end{lstlisting}

\begin{lstlisting}
  public static float main(java.lang.String[]);
    flags: ACC_PUBLIC, ACC_STATIC
    Code:
      stack=1, locals=1, args_size=1
         0: ldc           #2                  // float 123.456f
         2: freturn       
\end{lstlisting}

The \INS{ldc} instruction used here is the same one as for loading 32-bit integer numbers
from a constant pool.

\INS{freturn} stands for \q{return float}.

Now what about function that return nothing?

\begin{lstlisting}[style=customjava]
public class ret
{
	public static void main(String[] args) 
	{
		return;
	}
}
\end{lstlisting}

\begin{lstlisting}
  public static void main(java.lang.String[]);
    flags: ACC_PUBLIC, ACC_STATIC
    Code:
      stack=0, locals=1, args_size=1
         0: return        
\end{lstlisting}

This means that the \INS{return} instruction is used to return control without returning 
an actual value.

Knowing all this, it's very easy to deduce the function's (or method's) returning type 
from the last instruction.

}
\RU{% TODO proof-reading
\subsection{Возврат значения}

Наверное, самая простая из всех возможных функций на Java это та, что возвращает 
некоторое значение.

О, и мы не должны забывать, что в Java нет \q{свободных} функций в общем смысле,
это \q{методы}.

Каждый метод принадлежит какому-то классу, так что невозможно объявить метод 
вне какого-либо класса.

Но мы все равно будем называть их \q{функциями}, для простоты.

\begin{lstlisting}[style=customjava]
public class ret
{
	public static int main(String[] args) 
	{
		return 0;
	}
}
\end{lstlisting}

Компилируем это:

\begin{lstlisting}
javac ret.java
\end{lstlisting}

\dots и декомпилирую используя стандартную утилиту в Java:

\begin{lstlisting}
javap -c -verbose ret.class
\end{lstlisting}

И получаем:

\begin{lstlisting}[caption=JDK 1.7 (excerpt)]
  public static int main(java.lang.String[]);
    flags: ACC_PUBLIC, ACC_STATIC
    Code:
      stack=1, locals=1, args_size=1
         0: iconst_0      
         1: ireturn       
\end{lstlisting}

Разработчики Java решили, что 0 это самая используемая константа в программировании,
так что здесь есть отдельная однобайтная инструкция \GTT{iconst\_0}, заталкивающая 0 в стек
\footnote{Так же как и в MIPS, где для нулевой константы имеется отдельный регистр: \myref{MIPS_zero_register}.}.

Здесь есть также \GTT{iconst\_1} (заталкивающая 1), \GTT{iconst\_2}, итд, 
вплоть до \GTT{iconst\_5}.
Есть также \GTT{iconst\_m1} заталкивающая -1.

The stack is used in JVM for passing data to called functions and also for returning values.
So \GTT{iconst\_0} pushes 0 into the stack.
\GTT{ireturn} returns an integer value (\IT{i} in name means \IT{integer}) from the \ac{TOS}.

Немного перепишем наш пример, теперь возвращаем 1234:

\begin{lstlisting}[style=customjava]
public class ret
{
	public static int main(String[] args)
	{
		return 1234;
	}
}
\end{lstlisting}

\dots получаем:

\begin{lstlisting}[caption=JDK 1.7 (excerpt)]
  public static int main(java.lang.String[]);
    flags: ACC_PUBLIC, ACC_STATIC
    Code:
      stack=1, locals=1, args_size=1
         0: sipush        1234
         3: ireturn       
\end{lstlisting}

\GTT{sipush} (\IT{short integer}) заталкивает значение 1234 в стек.
\IT{short} в имени означает, что 16-битное значение будет заталкиваться в стек. 

Число 1234 действительно помещается в 16-битное значение.

Как насчет б\'{о}льших значений?

\begin{lstlisting}[style=customjava]
public class ret
{
	public static int main(String[] args) 
	{
		return 12345678;
	}
}
\end{lstlisting}

\begin{lstlisting}[caption=Constant pool]
...
   #2 = Integer            12345678
...
\end{lstlisting}

\begin{lstlisting}
  public static int main(java.lang.String[]);
    flags: ACC_PUBLIC, ACC_STATIC
    Code:
      stack=1, locals=1, args_size=1
         0: ldc           #2                  // int 12345678
         2: ireturn       
\end{lstlisting}

Невозможно закодировать 32-битное число в опкоде какой-либо JVM-инструкции, 
разработчики не оставили такой возможности.

Так что 32-битное число 12345678 сохранено в так называемом \q{constant pool} (пул констант),
который, так скажем, является библиотекой наиболее используемых констант (включая строки, объекты,
итд).

Этот способ передачи констант не уникален для JVM.

MIPS, ARM и прочие RISC-процессоры не могут кодировать 32-битные числа в 32-битных опкодах,
так что код для RISC-процессоров (включая MIPS и ARM) должен конструировать значения 
в несколько шагов, или держать их в сегменте данных:
\myref{ARM_big_constants}, \myref{MIPS_big_constants}.

Код для MIPS также традиционно имеет пул констант, называемый \q{literal pool}, это сегменты
с названиями \q{.lit4} (для хранения 32-битных чисел с плавающей точкой одинарной точности) и
\q{.lit8}(для хранения 64-битных чисел с плавающей точкой двойной точности).

Попробуем некоторые другие типы данных!

Boolean:

\begin{lstlisting}[style=customjava]
public class ret
{
	public static boolean main(String[] args) 
	{
		return true;
	}
}
\end{lstlisting}

\begin{lstlisting}
  public static boolean main(java.lang.String[]);
    flags: ACC_PUBLIC, ACC_STATIC
    Code:
      stack=1, locals=1, args_size=1
         0: iconst_1      
         1: ireturn       
\end{lstlisting}

Этот JVM-байткод не отличается от того, что возвращает целочисленную 1.

32-битные слоты данных в стеке также используются для булевых значений, как в \CCpp.

Но нельзя использовать возвращаемое значение булевого типа как целочисленное и наоборот --- информация 
о типах сохраняется в class-файлах и проверяется при запуске.

Та же история с 16-битным \IT{short}:

\begin{lstlisting}[style=customjava]
public class ret
{
	public static short main(String[] args) 
	{
		return 1234;
	}
}
\end{lstlisting}

\begin{lstlisting}
  public static short main(java.lang.String[]);
    flags: ACC_PUBLIC, ACC_STATIC
    Code:
      stack=1, locals=1, args_size=1
         0: sipush        1234
         3: ireturn       
\end{lstlisting}

\dots и \IT{char}!

\begin{lstlisting}[style=customjava]
public class ret
{
	public static char main(String[] args) 
	{
		return 'A';
	}
}
\end{lstlisting}

\begin{lstlisting}
  public static char main(java.lang.String[]);
    flags: ACC_PUBLIC, ACC_STATIC
    Code:
      stack=1, locals=1, args_size=1
         0: bipush        65
         2: ireturn       
\end{lstlisting}

\INS{bipush} означает \q{push byte}.

Нужно сказать, что \IT{char} в Java, это 16-битный символ в кодировке UTF-16,
и он эквивалентен \IT{short}, но ASCII-код символа \q{A} это 65, и можно воспользоваться
инструкцией для передачи байта в стек.

Попробуем также \IT{byte}:

\begin{lstlisting}[style=customjava]
public class retc
{
	public static byte main(String[] args) 
	{
		return 123;
	}
}
\end{lstlisting}

\begin{lstlisting}
  public static byte main(java.lang.String[]);
    flags: ACC_PUBLIC, ACC_STATIC
    Code:
      stack=1, locals=1, args_size=1
         0: bipush        123
         2: ireturn       
\end{lstlisting}

Кто-то может спросить, зачем заморачиваться использованием 16-битного типа \IT{short}, который
внутри все равно 32-битный integer?

Зачем использовать тип данных \IT{char}, если это то же самое что и тип \IT{short}?

Ответ прост: для контроля типов данных и читабельности исходников.

\IT{char} может быть эквивалентом \IT{short}, но мы быстро понимаем, что это ячейка
для символа в кодировке UTF-16, а не для какого-то другого целочисленного значения.

Когда используем \IT{short}, мы можем показать всем, что диапазон этой переменной 
ограничен 16-ю битами.

Очень хорошая идея использовать тип \IT{boolean} где нужно, 
вместо \IT{int} для тех же целей, как это было в Си.

В Java есть также 64-битный целочисленный тип:

\begin{lstlisting}[style=customjava]
public class ret3
{
	public static long main(String[] args)
	{
		return 1234567890123456789L;
	}
}
\end{lstlisting}

\begin{lstlisting}[caption=Constant pool]
...
   #2 = Long               1234567890123456789l
...
\end{lstlisting}

\begin{lstlisting}
  public static long main(java.lang.String[]);
    flags: ACC_PUBLIC, ACC_STATIC
    Code:
      stack=2, locals=1, args_size=1
         0: ldc2_w        #2                  // long 1234567890123456789l
         3: lreturn       
\end{lstlisting}

64-битное число также хранится в пуле констант, \INS{ldc2\_w} загружает его и \INS{lreturn} 
(\IT{long return}) возвращает его.

Инструкция \INS{ldc2\_w} также используется для загрузки чисел с плавающей точкой двойной 
точности (которые также занимают 64 бита) из пула констант:

\begin{lstlisting}[style=customjava]
public class ret
{
	public static double main(String[] args)
	{
		return 123.456d;
	}
}
\end{lstlisting}

\begin{lstlisting}[caption=Constant pool]
...
   #2 = Double             123.456d
...
\end{lstlisting}

\begin{lstlisting}
  public static double main(java.lang.String[]);
    flags: ACC_PUBLIC, ACC_STATIC
    Code:
      stack=2, locals=1, args_size=1
         0: ldc2_w        #2                  // double 123.456d
         3: dreturn       
\end{lstlisting}

\INS{dreturn} означает \q{return double}.

И наконец, числа с плавающей точкой одинарной точности:

\begin{lstlisting}[style=customjava]
public class ret
{
	public static float main(String[] args)
	{
		return 123.456f;
	}
}
\end{lstlisting}

\begin{lstlisting}[caption=Constant pool]
...
   #2 = Float              123.456f
...
\end{lstlisting}

\begin{lstlisting}
  public static float main(java.lang.String[]);
    flags: ACC_PUBLIC, ACC_STATIC
    Code:
      stack=1, locals=1, args_size=1
         0: ldc           #2                  // float 123.456f
         2: freturn       
\end{lstlisting}

Используемая здесь инструкция \INS{ldc} та же, что и для загрузки 32-битных целочисленных чисел
из пула констант.

\INS{freturn} означает \q{return float}.

А что насчет тех случаев, когда функция ничего не возвращает?

\begin{lstlisting}[style=customjava]
public class ret
{
	public static void main(String[] args) 
	{
		return;
	}
}
\end{lstlisting}

\begin{lstlisting}
  public static void main(java.lang.String[]);
    flags: ACC_PUBLIC, ACC_STATIC
    Code:
      stack=0, locals=1, args_size=1
         0: return        
\end{lstlisting}

Это означает, что инструкция \INS{return} используется для возврата управления 
без возврата какого-либо значения.

Зная все это, по последней инструкции очень легко определить тип возвращаемого 
значения функции (или метода).
}
\DE{% TODO proof-reading
\subsection{Rückgabe eines Wertes}

Die vermutlich einfachste Java-Funktion ist eine, die einen Wert zurück gibt.

Es sollte hier noch beachtet werden, dass es keine \q{freien} Funktionen im
allgemeinen Sinne in Java gibt sondern \q{Methoden}. 

Jede Methode gehört zu einer Klasse, somit ist es nicht möglich eine Methode
außerhalb einer Klasse zu definieren.

Wir werden die Methoden hier trotzdem der Einfachheit halber \q{Funktionen} nennen.

\begin{lstlisting}[style=customjava]
public class ret
{
	public static int main(String[] args) 
	{
		return 0;
	}
}
\end{lstlisting}

Kompilieren wir diesen Code:

\begin{lstlisting}
javac ret.java
\end{lstlisting}

\dots und dekompilieren ihn mit dem Standard-Java-Tool:

\begin{lstlisting}
javap -c -verbose ret.class
\end{lstlisting}

Und wir bekommen:

\begin{lstlisting}[caption=JDK 1.7 (excerpt)]
  public static int main(java.lang.String[]);
    flags: ACC_PUBLIC, ACC_STATIC
    Code:
      stack=1, locals=1, args_size=1
         0: iconst_0      
         1: ireturn       
\end{lstlisting}

Die Java-Entwickler entschieden, dass 0 eine der beliebteste Konstanten in der
Programmierung ist, also gibt es eine separate, kurze Ein-Byte-Anweisung die
0 pushed.

%\footnote{Just like in MIPS, where a separate register for zero constant exists: \myref{MIPS_zero_register}.}.
%There are also \GTT{iconst\_1} (which pushes 1), \GTT{iconst\_2}, etc., up to \GTT{iconst\_5}.
%
%There is also \GTT{iconst\_m1} which pushes -1.
%
%The stack is used in JVM for passing data to called functions and also for returning values.
%So \GTT{iconst\_0} pushes 0 into the stack.
%\GTT{ireturn} returns an integer value (\IT{i} in name means \IT{integer}) from the \ac{TOS}.
%
%Let's rewrite our example slightly, now we return 1234:
%
%\begin{lstlisting}[style=customjava]
%public class ret
%{
%	public static int main(String[] args)
%	{
%		return 1234;
%	}
%}
%\end{lstlisting}
%
%\dots we get:
%
%\begin{lstlisting}[caption=JDK 1.7 (excerpt)]
%  public static int main(java.lang.String[]);
%    flags: ACC_PUBLIC, ACC_STATIC
%    Code:
%      stack=1, locals=1, args_size=1
%         0: sipush        1234
%         3: ireturn       
%\end{lstlisting}
%
%\GTT{sipush} (\IT{short integer}) pushes 1234 into the stack.
%\IT{short} in name implies a 16-bit value is to be pushed. 
%The number 1234 indeed fits well in a 16-bit value.
%
%What about larger values?
%
%\begin{lstlisting}[style=customjava]
%public class ret
%{
%	public static int main(String[] args) 
%	{
%		return 12345678;
%	}
%}
%\end{lstlisting}
%
%\begin{lstlisting}[caption=Constant pool]
%...
%   #2 = Integer            12345678
%...
%\end{lstlisting}
%
%\begin{lstlisting}
%  public static int main(java.lang.String[]);
%    flags: ACC_PUBLIC, ACC_STATIC
%    Code:
%      stack=1, locals=1, args_size=1
%         0: ldc           #2                  // int 12345678
%         2: ireturn       
%\end{lstlisting}
%
%It's not possible to encode a 32-bit number in a JVM instruction opcode, 
%the developers didn't leave such possibility.
%
%So the 32-bit number 12345678 is stored in so called \q{constant pool} which is, let's say,
%the library of most used constants (including strings, objects, etc.).
%
%This way of passing constants is not unique to JVM.
%
%MIPS, ARM and other RISC CPUs also can't encode a 32-bit number
%in a 32-bit opcode, so the RISC CPU code (including MIPS and ARM) has to construct the value 
%in several steps, or to keep it in the data segment:
%\myref{ARM_big_constants}, \myref{MIPS_big_constants}.
%
%MIPS code also traditionally has a constant pool, named \q{literal pool}, the segments
%are called \q{.lit4} (for 32-bit single precision floating point number constants) and \q{.lit8}
%(for 64-bit double precision floating point number constants).
%
%Let's try some other data types!
%
%Boolean:
%
%\begin{lstlisting}[style=customjava]
%public class ret
%{
%	public static boolean main(String[] args) 
%	{
%		return true;
%	}
%}
%\end{lstlisting}
%
%\begin{lstlisting}
%  public static boolean main(java.lang.String[]);
%    flags: ACC_PUBLIC, ACC_STATIC
%    Code:
%      stack=1, locals=1, args_size=1
%         0: iconst_1      
%         1: ireturn       
%\end{lstlisting}
%
%This JVM bytecode is no different from one returning integer 1.
%
%32-bit data slots in the stack are also used here for boolean values, like in \CCpp.
%
%But one could not use returned boolean value as integer or vice versa --- type information is stored in the class file and checked at runtime.
%
%It's the same story with a 16-bit \IT{short}:
%
%\begin{lstlisting}[style=customjava]
%public class ret
%{
%	public static short main(String[] args) 
%	{
%		return 1234;
%	}
%}
%\end{lstlisting}
%
%\begin{lstlisting}
%  public static short main(java.lang.String[]);
%    flags: ACC_PUBLIC, ACC_STATIC
%    Code:
%      stack=1, locals=1, args_size=1
%         0: sipush        1234
%         3: ireturn       
%\end{lstlisting}
%
%\dots and \IT{char}!
%
%\begin{lstlisting}[style=customjava]
%public class ret
%{
%	public static char main(String[] args) 
%	{
%		return 'A';
%	}
%}
%\end{lstlisting}
%
%\begin{lstlisting}
%  public static char main(java.lang.String[]);
%    flags: ACC_PUBLIC, ACC_STATIC
%    Code:
%      stack=1, locals=1, args_size=1
%         0: bipush        65
%         2: ireturn       
%\end{lstlisting}
%
%\INS{bipush} means \q{push byte}.
%Needless to say that a \IT{char} in Java is 16-bit UTF-16 character, 
%and it's equivalent to \IT{short}, but the ASCII code of the \q{A} character is 65, and it's possible
%to use the instruction for pushing a byte in the stack.
%
%Let's also try a \IT{byte}:
%
%\begin{lstlisting}[style=customjava]
%public class retc
%{
%	public static byte main(String[] args) 
%	{
%		return 123;
%	}
%}
%\end{lstlisting}
%
%\begin{lstlisting}
%  public static byte main(java.lang.String[]);
%    flags: ACC_PUBLIC, ACC_STATIC
%    Code:
%      stack=1, locals=1, args_size=1
%         0: bipush        123
%         2: ireturn       
%\end{lstlisting}
%
%One may ask, why bother with a 16-bit \IT{short} data type which internally works
%as a 32-bit integer?
%
%Why use a \IT{char} data type if it is the same as a \IT{short} data type?
%
%The answer is simple: for data type control and source code readability.
%
%A \IT{char} may essentially be the same as a \IT{short}, but we quickly grasp that it's a placeholder for
%an UTF-16 character, and not for some other integer value.
%
%When using \IT{short}, we show everyone that the variable's range is limited by 16 bits.
%
%It's a very good idea to use the \IT{boolean} type where needed to, 
%instead of the C-style \IT{int}.
%
%There is also a 64-bit integer data type in Java:
%
%\begin{lstlisting}[style=customjava]
%public class ret3
%{
%	public static long main(String[] args)
%	{
%		return 1234567890123456789L;
%	}
%}
%\end{lstlisting}
%
%\begin{lstlisting}[caption=Constant pool]
%...
%   #2 = Long               1234567890123456789l
%...
%\end{lstlisting}
%
%\begin{lstlisting}
%  public static long main(java.lang.String[]);
%    flags: ACC_PUBLIC, ACC_STATIC
%    Code:
%      stack=2, locals=1, args_size=1
%         0: ldc2_w        #2                  // long 1234567890123456789l
%         3: lreturn       
%\end{lstlisting}
%
%The 64-bit number is also stored in a constant pool, \INS{ldc2\_w} loads it and \INS{lreturn} 
%(\IT{long return}) returns it.
%
%The \INS{ldc2\_w} instruction is also used to load double precision floating point numbers 
%(which also occupy 64 bits) from a constant pool:
%
%\begin{lstlisting}[style=customjava]
%public class ret
%{
%	public static double main(String[] args)
%	{
%		return 123.456d;
%	}
%}
%\end{lstlisting}
%
%\begin{lstlisting}[caption=Constant pool]
%...
%   #2 = Double             123.456d
%...
%\end{lstlisting}
%
%\begin{lstlisting}
%  public static double main(java.lang.String[]);
%    flags: ACC_PUBLIC, ACC_STATIC
%    Code:
%      stack=2, locals=1, args_size=1
%         0: ldc2_w        #2                  // double 123.456d
%         3: dreturn       
%\end{lstlisting}
%
%\INS{dreturn} stands for \q{return double}.
%
%And finally, a single precision floating point number:
%
%\begin{lstlisting}[style=customjava]
%public class ret
%{
%	public static float main(String[] args)
%	{
%		return 123.456f;
%	}
%}
%\end{lstlisting}
%
%\begin{lstlisting}[caption=Constant pool]
%...
%   #2 = Float              123.456f
%...
%\end{lstlisting}
%
%\begin{lstlisting}
%  public static float main(java.lang.String[]);
%    flags: ACC_PUBLIC, ACC_STATIC
%    Code:
%      stack=1, locals=1, args_size=1
%         0: ldc           #2                  // float 123.456f
%         2: freturn       
%\end{lstlisting}
%
%The \INS{ldc} instruction used here is the same one as for loading 32-bit integer numbers
%from a constant pool.
%
%\INS{freturn} stands for \q{return float}.
%
%Now what about function that return nothing?
%
%\begin{lstlisting}[style=customjava]
%public class ret
%{
%	public static void main(String[] args) 
%	{
%		return;
%	}
%}
%\end{lstlisting}
%
%\begin{lstlisting}
%  public static void main(java.lang.String[]);
%    flags: ACC_PUBLIC, ACC_STATIC
%    Code:
%      stack=0, locals=1, args_size=1
%         0: return        
%\end{lstlisting}
%
%This means that the \INS{return} instruction is used to return control without returning 
%an actual value.
%
%Knowing all this, it's very easy to deduce the function's (or method's) returning type 
%from the last instruction.
%
}

\EN{% TODO proof-reading
\subsection{Simple calculating functions}

Let's continue with a simple calculating functions.

\begin{lstlisting}[style=customjava]
public class calc
{
	public static int half(int a)
	{
		return a/2;
	}
}
\end{lstlisting}

Here's the output when the \INS{iconst\_2} instruction is used:


\begin{lstlisting}
  public static int half(int);
    flags: ACC_PUBLIC, ACC_STATIC
    Code:
      stack=2, locals=1, args_size=1
         0: iload_0       
         1: iconst_2      
         2: idiv          
         3: ireturn       
\end{lstlisting}
         
\INS{iload\_0} takes the zeroth function argument and pushes it to the stack.

\INS{iconst\_2} pushes 2 in the stack.
After the execution of these two instructions, this is how stack looks like:


% FIXME: TikZ
\begin{lstlisting}
      +---+
TOS ->| 2 |
      +---+
      | a |
      +---+
\end{lstlisting}

\INS{idiv} just takes the two values at the \ac{TOS}, divides one by the other and leaves
the result at \ac{TOS}:


% FIXME: TikZ
\begin{lstlisting}
      +--------+
TOS ->| result |
      +--------+
\end{lstlisting}

\INS{ireturn} takes it and returns.

Let's proceed with double precision floating point numbers:


\begin{lstlisting}[style=customjava]
public class calc
{
	public static double half_double(double a)
	{
		return a/2.0;
	}
}
\end{lstlisting}

\begin{lstlisting}[caption=Constant pool]
...
   #2 = Double             2.0d
...
\end{lstlisting}

\begin{lstlisting}
  public static double half_double(double);
    flags: ACC_PUBLIC, ACC_STATIC
    Code:
      stack=4, locals=2, args_size=1
         0: dload_0       
         1: ldc2_w        #2                  // double 2.0d
         4: ddiv          
         5: dreturn       
\end{lstlisting}

It's the same, but the \INS{ldc2\_w} instruction is used to load the constant 
2.0 from the constant pool.

Also, the other three instructions have the \IT{d} prefix, 
meaning they work with \IT{double} data type values.


Let's now use a function with two arguments:


\begin{lstlisting}[style=customjava]
public class calc
{
	public static int sum(int a, int b)
	{
		return a+b;
	}
}
\end{lstlisting}

\begin{lstlisting}
  public static int sum(int, int);
    flags: ACC_PUBLIC, ACC_STATIC
    Code:
      stack=2, locals=2, args_size=2
         0: iload_0       
         1: iload_1       
         2: iadd          
         3: ireturn       
\end{lstlisting}

\INS{iload\_0} loads the first function argument (a), \INS{iload\_1}---second (b).

Here is the stack after the execution of both instructions:


\begin{lstlisting}
      +---+
TOS ->| b |
      +---+
      | a |
      +---+
\end{lstlisting}

\INS{iadd} adds the two values and leaves the result at \ac{TOS}:


\begin{lstlisting}
      +--------+
TOS ->| result |
      +--------+
\end{lstlisting}

Let's extend this example to the \IT{long} data type:


\begin{lstlisting}[style=customjava]
	public static long lsum(long a, long b)
	{
		return a+b;
	}
\end{lstlisting}

\dots we got:

\begin{lstlisting}
  public static long lsum(long, long);
    flags: ACC_PUBLIC, ACC_STATIC
    Code:
      stack=4, locals=4, args_size=2
         0: lload_0       
         1: lload_2       
         2: ladd          
         3: lreturn       
\end{lstlisting}

The second \TT{lload} instruction takes the second argument from the 2nd slot.

That's because a 64-bit \IT{long} value occupies exactly two 32-bit slots.


Slightly more advanced example:


\begin{lstlisting}[style=customjava]
public class calc
{
	public static int mult_add(int a, int b, int c)
	{
		return a*b+c;
	}
}
\end{lstlisting}

\begin{lstlisting}
  public static int mult_add(int, int, int);
    flags: ACC_PUBLIC, ACC_STATIC
    Code:
      stack=2, locals=3, args_size=3
         0: iload_0       
         1: iload_1       
         2: imul          
         3: iload_2       
         4: iadd          
         5: ireturn       
\end{lstlisting}

The first step is multiplication. The product is left at the \ac{TOS}:


\begin{lstlisting}
      +---------+
TOS ->| product |
      +---------+
\end{lstlisting}

\TT{iload\_2} loads the third argument (c) in the stack:

\begin{lstlisting}
      +---------+
TOS ->|    c    |
      +---------+
      | product |
      +---------+
\end{lstlisting}

Now the \TT{iadd} instruction can add the two values.

}
\RU{% TODO proof-reading
\subsection{Простая вычисляющая функция}

Продолжим с простой вычисляющей функцией.

\begin{lstlisting}[style=customjava]
public class calc
{
	public static int half(int a)
	{
		return a/2;
	}
}
\end{lstlisting}


Это тот случай, когда используется инструкция \INS{iconst\_2}:

\begin{lstlisting}
  public static int half(int);
    flags: ACC_PUBLIC, ACC_STATIC
    Code:
      stack=2, locals=1, args_size=1
         0: iload_0       
         1: iconst_2      
         2: idiv          
         3: ireturn       
\end{lstlisting}
         
\INS{iload\_0} 
Берет нулевой аргумент функции и заталкивает его в стек.
\INS{iconst\_2} заталкивает в стек 2.

Вот как выглядит стек после исполнения этих двух инструкций:

% FIXME: TikZ
\begin{lstlisting}
      +---+
TOS ->| 2 |
      +---+
      | a |
      +---+
\end{lstlisting}


\INS{idiv} просто берет два значения на вершине стека (\ac{TOS}), 
делит одно на другое и оставляет результат на вершине (\ac{TOS}):

% FIXME: TikZ
\begin{lstlisting}
      +--------+
TOS ->| result |
      +--------+
\end{lstlisting}

\INS{ireturn} берет его и возвращает.


Продолжим с числами с плавающей запятой, двойной точности:

\begin{lstlisting}[style=customjava]
public class calc
{
	public static double half_double(double a)
	{
		return a/2.0;
	}
}
\end{lstlisting}

\begin{lstlisting}[caption=Constant pool]
...
   #2 = Double             2.0d
...
\end{lstlisting}

\begin{lstlisting}
  public static double half_double(double);
    flags: ACC_PUBLIC, ACC_STATIC
    Code:
      stack=4, locals=2, args_size=1
         0: dload_0       
         1: ldc2_w        #2                  // double 2.0d
         4: ddiv          
         5: dreturn       
\end{lstlisting}


Почти то же самое, но инструкция \INS{ldc2\_w} используется для загрузки константы 
2.0 из пула констант.

Также, все три инструкции имеют префикс \IT{d}, что означает, что они работают с переменными
типа \IT{double}.


Теперь перейдем к функции с двумя аргументами:

\begin{lstlisting}[style=customjava]
public class calc
{
	public static int sum(int a, int b)
	{
		return a+b;
	}
}
\end{lstlisting}

\begin{lstlisting}
  public static int sum(int, int);
    flags: ACC_PUBLIC, ACC_STATIC
    Code:
      stack=2, locals=2, args_size=2
         0: iload_0       
         1: iload_1       
         2: iadd          
         3: ireturn       
\end{lstlisting}


\INS{iload\_0} загружает первый аргумент функции (a), \INS{iload\_1} --- второй (b).

Вот так выглядит стек после исполнения обоих инструкций:

\begin{lstlisting}
      +---+
TOS ->| b |
      +---+
      | a |
      +---+
\end{lstlisting}


\INS{iadd} складывает два значения и оставляет результат на \ac{TOS}:

\begin{lstlisting}
      +--------+
TOS ->| result |
      +--------+
\end{lstlisting}


Расширим этот пример до типа данных \IT{long}:

\begin{lstlisting}[style=customjava]
	public static long lsum(long a, long b)
	{
		return a+b;
	}
\end{lstlisting}

\dots получим:

\begin{lstlisting}
  public static long lsum(long, long);
    flags: ACC_PUBLIC, ACC_STATIC
    Code:
      stack=4, locals=4, args_size=2
         0: lload_0       
         1: lload_2       
         2: ladd          
         3: lreturn       
\end{lstlisting}


Вторая инструкция \TT{lload} берет второй аргумент из второго слота.

Это потому что 64-битное значение \IT{long} занимает ровно два 32-битных слота.


Немного более сложный пример:

\begin{lstlisting}[style=customjava]
public class calc
{
	public static int mult_add(int a, int b, int c)
	{
		return a*b+c;
	}
}
\end{lstlisting}

\begin{lstlisting}
  public static int mult_add(int, int, int);
    flags: ACC_PUBLIC, ACC_STATIC
    Code:
      stack=2, locals=3, args_size=3
         0: iload_0       
         1: iload_1       
         2: imul          
         3: iload_2       
         4: iadd          
         5: ireturn       
\end{lstlisting}


Первый шаг это умножение. Произведение остается на \ac{TOS}:

\begin{lstlisting}
      +---------+
TOS ->| product |
      +---------+
\end{lstlisting}

\TT{iload\_2} загружает третий аргумент (c) в стек:

\begin{lstlisting}
      +---------+
TOS ->|    c    |
      +---------+
      | product |
      +---------+
\end{lstlisting}


Теперь инструкция \TT{iadd} может сложить два значения.
}
\DE{% TODO proof-reading
\subsection{Einfache Berechnungsfunktionen}

%Let's continue with a simple calculating functions.
%
%\begin{lstlisting}[style=customjava]
%public class calc
%{
%	public static int half(int a)
%	{
%		return a/2;
%	}
%}
%\end{lstlisting}
%
%Here's the output when the \INS{iconst\_2} instruction is used:
%
%
%\begin{lstlisting}
%  public static int half(int);
%    flags: ACC_PUBLIC, ACC_STATIC
%    Code:
%      stack=2, locals=1, args_size=1
%         0: iload_0       
%         1: iconst_2      
%         2: idiv          
%         3: ireturn       
%\end{lstlisting}
%         
%\INS{iload\_0} takes the zeroth function argument and pushes it to the stack.
%
%\INS{iconst\_2} pushes 2 in the stack.
%After the execution of these two instructions, this is how stack looks like:
%
%
%% FIXME: TikZ
%\begin{lstlisting}
%      +---+
%TOS ->| 2 |
%      +---+
%      | a |
%      +---+
%\end{lstlisting}
%
%\INS{idiv} just takes the two values at the \ac{TOS}, divides one by the other and leaves
%the result at \ac{TOS}:
%
%
%% FIXME: TikZ
%\begin{lstlisting}
%      +--------+
%TOS ->| result |
%      +--------+
%\end{lstlisting}
%
%\INS{ireturn} takes it and returns.
%
%Let's proceed with double precision floating point numbers:
%
%
%\begin{lstlisting}[style=customjava]
%public class calc
%{
%	public static double half_double(double a)
%	{
%		return a/2.0;
%	}
%}
%\end{lstlisting}
%
%\begin{lstlisting}[caption=Constant pool]
%...
%   #2 = Double             2.0d
%...
%\end{lstlisting}
%
%\begin{lstlisting}
%  public static double half_double(double);
%    flags: ACC_PUBLIC, ACC_STATIC
%    Code:
%      stack=4, locals=2, args_size=1
%         0: dload_0       
%         1: ldc2_w        #2                  // double 2.0d
%         4: ddiv          
%         5: dreturn       
%\end{lstlisting}
%
%It's the same, but the \INS{ldc2\_w} instruction is used to load the constant 
%2.0 from the constant pool.
%
%Also, the other three instructions have the \IT{d} prefix, 
%meaning they work with \IT{double} data type values.
%
%
%Let's now use a function with two arguments:
%
%
%\begin{lstlisting}[style=customjava]
%public class calc
%{
%	public static int sum(int a, int b)
%	{
%		return a+b;
%	}
%}
%\end{lstlisting}
%
%\begin{lstlisting}
%  public static int sum(int, int);
%    flags: ACC_PUBLIC, ACC_STATIC
%    Code:
%      stack=2, locals=2, args_size=2
%         0: iload_0       
%         1: iload_1       
%         2: iadd          
%         3: ireturn       
%\end{lstlisting}
%
%\INS{iload\_0} loads the first function argument (a), \INS{iload\_1}---second (b).
%
%Here is the stack after the execution of both instructions:
%
%
%\begin{lstlisting}
%      +---+
%TOS ->| b |
%      +---+
%      | a |
%      +---+
%\end{lstlisting}
%
%\INS{iadd} adds the two values and leaves the result at \ac{TOS}:
%
%
%\begin{lstlisting}
%      +--------+
%TOS ->| result |
%      +--------+
%\end{lstlisting}
%
%Let's extend this example to the \IT{long} data type:
%
%
%\begin{lstlisting}[style=customjava]
%	public static long lsum(long a, long b)
%	{
%		return a+b;
%	}
%\end{lstlisting}
%
%\dots we got:
%
%\begin{lstlisting}
%  public static long lsum(long, long);
%    flags: ACC_PUBLIC, ACC_STATIC
%    Code:
%      stack=4, locals=4, args_size=2
%         0: lload_0       
%         1: lload_2       
%         2: ladd          
%         3: lreturn       
%\end{lstlisting}
%
%The second \TT{lload} instruction takes the second argument from the 2nd slot.
%
%That's because a 64-bit \IT{long} value occupies exactly two 32-bit slots.
%
%
%Slightly more advanced example:
%
%
%\begin{lstlisting}[style=customjava]
%public class calc
%{
%	public static int mult_add(int a, int b, int c)
%	{
%		return a*b+c;
%	}
%}
%\end{lstlisting}
%
%\begin{lstlisting}
%  public static int mult_add(int, int, int);
%    flags: ACC_PUBLIC, ACC_STATIC
%    Code:
%      stack=2, locals=3, args_size=3
%         0: iload_0       
%         1: iload_1       
%         2: imul          
%         3: iload_2       
%         4: iadd          
%         5: ireturn       
%\end{lstlisting}
%
%The first step is multiplication. The product is left at the \ac{TOS}:
%
%
%\begin{lstlisting}
%      +---------+
%TOS ->| product |
%      +---------+
%\end{lstlisting}
%
%\TT{iload\_2} loads the third argument (c) in the stack:
%
%\begin{lstlisting}
%      +---------+
%TOS ->|    c    |
%      +---------+
%      | product |
%      +---------+
%\end{lstlisting}
%
%Now the \TT{iadd} instruction can add the two values.
%
}

\EN{% TODO proof-reading
\subsection{\ac{JVM} memory model}

x86 and other low-level environments use the stack for argument passing and 
as a local variables storage.

\ac{JVM} is slightly different.

It has:

\begin{itemize}
\item Local variable array (\ac{LVA}).
Used as storage for incoming function arguments and local variables.

Instructions like \INS{iload\_0} load values from it.

\INS{istore} stores values in it.
At the beginning the function arguments are stored: starting at 0 or at 1 
(if the zeroth argument is occupied by \IT{this} pointer).

Then the local variables are allocated.


Each slot has size of 32-bit.

Hence, values of \IT{long} and \IT{double} data types occupy two slots.


\item Operand stack (or just \q{stack}).
It's used for computations and passing arguments while calling other functions.

Unlike low-level environments like x86, it's not possible to access the stack without using
instructions which explicitly pushes or pops values to/from it.


\item Heap. It is used as storage for objects and arrays.

\end{itemize}

These 3 areas are isolated from each other.

}
\RU{% TODO proof-reading
\subsection{Модель памяти в \ac{JVM}}


x86 и другие низкоуровневые среды используют стек для передачи аргументов и как
хранилище локальных переменных.
\ac{JVM} устроена немного иначе.

Тут есть:

\begin{itemize}
\item Массив локальных переменных (\ac{LVA}).

Используется как хранилище для аргументов функций и локальных переменных.

Инструкции вроде \INS{iload\_0} загружают значения оттуда.
\INS{istore} записывает значения туда.

В начале идут аргументы функции: начиная с 0, или с 1 
(если нулевой аргумент занят указателем \IT{this}.

Затем располагаются локальные переменные.


Каждый слот имеет размер 32 бита.

Следовательно, значения типов \IT{long} и \IT{double} занимают два слота.

\item Стек операндов (или просто \q{стек}).

Используется для вычислений и для передачи аргументов во время вызова других функций.

В отличие от низкоуровневых сред вроде x86, здесь невозможно работать со стеком
без использования инструкций, которые явно заталкивают или выталкивают значения туда/оттуда.

\item 
Куча (heap). Используется как хранилище для объектов и массивов.
\end{itemize}


Эти 3 области изолированы друг от друга.
}
\DE{% TODO proof-reading
\subsection{\ac{JVM}-Speichermodell}

%x86 and other low-level environments use the stack for argument passing and 
%as a local variables storage.
%
%\ac{JVM} is slightly different.
%
%It has:
%
%\begin{itemize}
%\item Local variable array (\ac{LVA}).
%Used as storage for incoming function arguments and local variables.
%
%Instructions like \INS{iload\_0} load values from it.
%
%\INS{istore} stores values in it.
%At the beginning the function arguments are stored: starting at 0 or at 1 
%(if the zeroth argument is occupied by \IT{this} pointer).
%
%Then the local variables are allocated.
%
%
%Each slot has size of 32-bit.
%
%Hence, values of \IT{long} and \IT{double} data types occupy two slots.
%
%
%\item Operand stack (or just \q{stack}).
%It's used for computations and passing arguments while calling other functions.
%
%Unlike low-level environments like x86, it's not possible to access the stack without using
%instructions which explicitly pushes or pops values to/from it.
%
%
%\item Heap. It is used as storage for objects and arrays.
%
%\end{itemize}
%
%These 3 areas are isolated from each other.
%
}

\EN{% TODO proof-reading
\subsection{Simple function calling}

\TT{Math.random()} returns a pseudorandom number in range of [0.0 \dots 1.0), but let's say that
for some reason we need to devise a function that returns a number in range of [0.0 \dots 0.5):


\begin{lstlisting}[style=customjava]
public class HalfRandom
{ 
	public static double f()
	{
		return Math.random()/2;
	}
}
\end{lstlisting}

\begin{lstlisting}[caption=Constant pool]
...
   #2 = Methodref          #18.#19    //  java/lang/Math.random:()D
   #3 = Double             2.0d
...
  #12 = Utf8               ()D
...
  #18 = Class              #22        //  java/lang/Math
  #19 = NameAndType        #23:#12    //  random:()D
  #22 = Utf8               java/lang/Math
  #23 = Utf8               random
\end{lstlisting}

\begin{lstlisting}
  public static double f();
    flags: ACC_PUBLIC, ACC_STATIC
    Code:
      stack=4, locals=0, args_size=0
         0: invokestatic  #2          // Method java/lang/Math.random:()D
         3: ldc2_w        #3          // double 2.0d
         6: ddiv          
         7: dreturn       
\end{lstlisting}

\TT{invokestatic} calls the \TT{Math.random()} function and leaves the result at the \ac{TOS}.

Then the result is divided by 2.0 and returned.

But how is the function name encoded?

It's encoded in the constant pool using a \TT{Methodref} expression.

It defines the class and method names.

The first field of \TT{Methodref} points to a \TT{Class} expression which, in turn, points to
the usual text string (\q{java/lang/Math}).

The second \TT{Methodref} expression points to a \TT{NameAndType} expression which also 
has two links to the strings.

The first string is \q{random}, which is the name of the method.

The second string is \q{()D}, which encodes the function's type.
It means that it returns a \IT{double} value (hence the \IT{D} in the string).

This is the way 
1) JVM can check data for type correctness; 
2) Java decompilers can restore data types from a compiled class file.


Now let's try the \q{Hello, world!} example:


\begin{lstlisting}[style=customjava]
public class HelloWorld
{
	public static void main(String[] args)
	{
		System.out.println("Hello, World");
	}
}
\end{lstlisting}

\begin{lstlisting}[caption=Constant pool]
...
   #2 = Fieldref           #16.#17        //  java/lang/System.out:Ljava/io/PrintStream;
   #3 = String             #18            //  Hello, World
   #4 = Methodref          #19.#20        //  java/io/PrintStream.println:(Ljava/lang/String;)V
...
  #16 = Class              #23            //  java/lang/System
  #17 = NameAndType        #24:#25        //  out:Ljava/io/PrintStream;
  #18 = Utf8               Hello, World
  #19 = Class              #26            //  java/io/PrintStream
  #20 = NameAndType        #27:#28        //  println:(Ljava/lang/String;)V
...
  #23 = Utf8               java/lang/System
  #24 = Utf8               out
  #25 = Utf8               Ljava/io/PrintStream;
  #26 = Utf8               java/io/PrintStream
  #27 = Utf8               println
  #28 = Utf8               (Ljava/lang/String;)V
...
\end{lstlisting}

\begin{lstlisting}
  public static void main(java.lang.String[]);
    flags: ACC_PUBLIC, ACC_STATIC
    Code:
      stack=2, locals=1, args_size=1
         0: getstatic     #2        // Field java/lang/System.out:Ljava/io/PrintStream;
         3: ldc           #3        // String Hello, World
         5: invokevirtual #4        // Method java/io/PrintStream.println:(Ljava/lang/String;)V
         8: return        
\end{lstlisting}

\TT{ldc} at offset 3 takes a pointer to the \q{Hello, World} string in the constant pool
and pushes in the stack.

It's called a \IT{reference} in the Java world, but it's rather a pointer, or an address

\footnote{About difference in pointers and \IT{reference}'s in C++ see: \myref{cpp_references}.}.

The familiar \TT{invokevirtual} instruction takes the information about the \TT{println} 
function (or method) from the constant pool and calls it.

As we may know, there are several \TT{println} methods, one for each data type.

Our case is the version of \TT{println} intended for the \IT{String} data type.


But what about the first \TT{getstatic} instruction?

This instruction takes a \IT{reference} (or address of) a field of the object \TT{System.out} 
and pushes it in the stack.

This value is acts like the \IT{this} pointer for the \TT{println} method.

Thus, internally, the \TT{println} method takes two arguments for input:
1) \IT{this}, i.e., a pointer to an object; 
2) the address of the \q{Hello, World} string.


Indeed, \TT{println()} is called as a method within an initialized \TT{System.out} object.


For convenience, the \TT{javap} utility writes all this information in the comments.

}
\RU{% TODO proof-reading
\subsection{Простой вызов функций}


\TT{Math.random()} возвращает псевдослучайное число в пределах [0.0 \dots 1.0), но представим,
по какой-то причине, нам нужна функция, возвращающая число в пределах [0.0 \dots 0.5):

\begin{lstlisting}[style=customjava]
public class HalfRandom
{ 
	public static double f()
	{
		return Math.random()/2;
	}
}
\end{lstlisting}

\begin{lstlisting}[caption=Constant pool]
...
   #2 = Methodref          #18.#19    //  java/lang/Math.random:()D
   #3 = Double             2.0d
...
  #12 = Utf8               ()D
...
  #18 = Class              #22        //  java/lang/Math
  #19 = NameAndType        #23:#12    //  random:()D
  #22 = Utf8               java/lang/Math
  #23 = Utf8               random
\end{lstlisting}

\begin{lstlisting}
  public static double f();
    flags: ACC_PUBLIC, ACC_STATIC
    Code:
      stack=4, locals=0, args_size=0
         0: invokestatic  #2          // Method java/lang/Math.random:()D
         3: ldc2_w        #3          // double 2.0d
         6: ddiv          
         7: dreturn       
\end{lstlisting}


\TT{invokestatic} вызывает функцию \TT{Math.random()} и оставляет результат на \ac{TOS}.

Затем результат делится на 2.0 и возвращается.

Но как закодировано имя функции?

Оно закодировано в пуле констант используя выражение \TT{Methodref}.

Оно определяет имена класса и метода.

Первое поле \TT{Methodref} указывает на выражение \TT{Class}, которое, в свою очередь,
указывает на обычную текстовую строку (\q{java/lang/Math}).

Второе выражение \TT{Methodref} указывает на выражение \TT{NameAndType}, которое также
имеет две ссылки на строки.

Первая строка это \q{random}, это имя метода.

Вторая строка это \q{()D}, которая кодирует тип функции. Это означает, что возвращаемый тип --- \IT{double} (отсюда \IT{D} в строке).

Благодаря этому
1) JVM проверяет корректность типов данных; 
2) Java-декомпиляторы могут восстанавливать типы данных из class-файлов.


Наконец попробуем пример \q{Hello, world!}:

\begin{lstlisting}[style=customjava]
public class HelloWorld
{
	public static void main(String[] args)
	{
		System.out.println("Hello, World");
	}
}
\end{lstlisting}

\begin{lstlisting}[caption=Constant pool]
...
   #2 = Fieldref           #16.#17        //  java/lang/System.out:Ljava/io/PrintStream;
   #3 = String             #18            //  Hello, World
   #4 = Methodref          #19.#20        //  java/io/PrintStream.println:(Ljava/lang/String;)V
...
  #16 = Class              #23            //  java/lang/System
  #17 = NameAndType        #24:#25        //  out:Ljava/io/PrintStream;
  #18 = Utf8               Hello, World
  #19 = Class              #26            //  java/io/PrintStream
  #20 = NameAndType        #27:#28        //  println:(Ljava/lang/String;)V
...
  #23 = Utf8               java/lang/System
  #24 = Utf8               out
  #25 = Utf8               Ljava/io/PrintStream;
  #26 = Utf8               java/io/PrintStream
  #27 = Utf8               println
  #28 = Utf8               (Ljava/lang/String;)V
...
\end{lstlisting}

\begin{lstlisting}
  public static void main(java.lang.String[]);
    flags: ACC_PUBLIC, ACC_STATIC
    Code:
      stack=2, locals=1, args_size=1
         0: getstatic     #2        // Field java/lang/System.out:Ljava/io/PrintStream;
         3: ldc           #3        // String Hello, World
         5: invokevirtual #4        // Method java/io/PrintStream.println:(Ljava/lang/String;)V
         8: return        
\end{lstlisting}


\TT{ldc} по смещению 3 берет указатель (или адрес) на строку \q{Hello, World}
в пуле констант и заталкивает его в стек.

В мире Java это называется \IT{reference}, но это скорее указатель или просто адрес
\footnote{О разнице между указателями и \IT{reference} в С++: \myref{cpp_references}.}.


Уже знакомая нам инструкция \TT{invokevirtual} берет информацию о функции (или методе) \TT{println} 
из пула констант и вызывает её.

Как мы можем знать, есть несколько методов \TT{println}, каждый предназначен для каждого типа
данных.

В нашем случае, используется та версия \TT{println}, которая для типа данных \IT{String}.


Что насчет первой инструкции \TT{getstatic}?

Эта инструкция берет \IT{reference} (или адрес) поля объекта \TT{System.out}
и заталкивает его в стек.

Это значение работает как указатель \IT{this} для метода \TT{println}.

Таким образом, внутри, метод \TT{println} берет на вход два аргумента:
1) \IT{this}, т.е. указатель на объект
\footnote{Или \q{экземпляр класса} в некоторой русскоязычной литературе.};
2) адрес строки \q{Hello, World}.


Действительно, \TT{println()} вызывается как метод в рамках инициализированного объекта 
\TT{System.out}.


Для удобства, утилита \TT{javap} пишет всю эту информацию в комментариях.
}
\DE{% TODO proof-reading
\subsection{Einfache Funktionsaufrufe}

%\TT{Math.random()} returns a pseudorandom number in range of [0.0 \dots 1.0), but let's say that
%for some reason we need to devise a function that returns a number in range of [0.0 \dots 0.5):
%
%
%\begin{lstlisting}[style=customjava]
%public class HalfRandom
%{ 
%	public static double f()
%	{
%		return Math.random()/2;
%	}
%}
%\end{lstlisting}
%
%\begin{lstlisting}[caption=Constant pool]
%...
%   #2 = Methodref          #18.#19    //  java/lang/Math.random:()D
%   #3 = Double             2.0d
%...
%  #12 = Utf8               ()D
%...
%  #18 = Class              #22        //  java/lang/Math
%  #19 = NameAndType        #23:#12    //  random:()D
%  #22 = Utf8               java/lang/Math
%  #23 = Utf8               random
%\end{lstlisting}
%
%\begin{lstlisting}
%  public static double f();
%    flags: ACC_PUBLIC, ACC_STATIC
%    Code:
%      stack=4, locals=0, args_size=0
%         0: invokestatic  #2          // Method java/lang/Math.random:()D
%         3: ldc2_w        #3          // double 2.0d
%         6: ddiv          
%         7: dreturn       
%\end{lstlisting}
%
%\TT{invokestatic} calls the \TT{Math.random()} function and leaves the result at the \ac{TOS}.
%
%Then the result is divided by 2.0 and returned.
%
%But how is the function name encoded?
%
%It's encoded in the constant pool using a \TT{Methodref} expression.
%
%It defines the class and method names.
%
%The first field of \TT{Methodref} points to a \TT{Class} expression which, in turn, points to
%the usual text string (\q{java/lang/Math}).
%
%The second \TT{Methodref} expression points to a \TT{NameAndType} expression which also 
%has two links to the strings.
%
%The first string is \q{random}, which is the name of the method.
%
%The second string is \q{()D}, which encodes the function's type.
%It means that it returns a \IT{double} value (hence the \IT{D} in the string).
%
%This is the way 
%1) JVM can check data for type correctness; 
%2) Java decompilers can restore data types from a compiled class file.
%
%
%Now let's try the \q{Hello, world!} example:
%
%
%\begin{lstlisting}[style=customjava]
%public class HelloWorld
%{
%	public static void main(String[] args)
%	{
%		System.out.println("Hello, World");
%	}
%}
%\end{lstlisting}
%
%\begin{lstlisting}[caption=Constant pool]
%...
%   #2 = Fieldref           #16.#17        //  java/lang/System.out:Ljava/io/PrintStream;
%   #3 = String             #18            //  Hello, World
%   #4 = Methodref          #19.#20        //  java/io/PrintStream.println:(Ljava/lang/String;)V
%...
%  #16 = Class              #23            //  java/lang/System
%  #17 = NameAndType        #24:#25        //  out:Ljava/io/PrintStream;
%  #18 = Utf8               Hello, World
%  #19 = Class              #26            //  java/io/PrintStream
%  #20 = NameAndType        #27:#28        //  println:(Ljava/lang/String;)V
%...
%  #23 = Utf8               java/lang/System
%  #24 = Utf8               out
%  #25 = Utf8               Ljava/io/PrintStream;
%  #26 = Utf8               java/io/PrintStream
%  #27 = Utf8               println
%  #28 = Utf8               (Ljava/lang/String;)V
%...
%\end{lstlisting}
%
%\begin{lstlisting}
%  public static void main(java.lang.String[]);
%    flags: ACC_PUBLIC, ACC_STATIC
%    Code:
%      stack=2, locals=1, args_size=1
%         0: getstatic     #2        // Field java/lang/System.out:Ljava/io/PrintStream;
%         3: ldc           #3        // String Hello, World
%         5: invokevirtual #4        // Method java/io/PrintStream.println:(Ljava/lang/String;)V
%         8: return        
%\end{lstlisting}
%
%\TT{ldc} at offset 3 takes a pointer to the \q{Hello, World} string in the constant pool
%and pushes in the stack.
%
%It's called a \IT{reference} in the Java world, but it's rather a pointer, or an address
%
%\footnote{About difference in pointers and \IT{reference}'s in C++ see: \myref{cpp_references}.}.
%
%The familiar \TT{invokevirtual} instruction takes the information about the \TT{println} 
%function (or method) from the constant pool and calls it.
%
%As we may know, there are several \TT{println} methods, one for each data type.
%
%Our case is the version of \TT{println} intended for the \IT{String} data type.
%
%
%But what about the first \TT{getstatic} instruction?
%
%This instruction takes a \IT{reference} (or address of) a field of the object \TT{System.out} 
%and pushes it in the stack.
%
%This value is acts like the \IT{this} pointer for the \TT{println} method.
%
%Thus, internally, the \TT{println} method takes two arguments for input:
%1) \IT{this}, i.e., a pointer to an object; 
%2) the address of the \q{Hello, World} string.
%
%
%Indeed, \TT{println()} is called as a method within an initialized \TT{System.out} object.
%
%
%For convenience, the \TT{javap} utility writes all this information in the comments.
%
}

\EN{% TODO proof-reading
\subsection{Calling beep()}

This is a simple calling of two functions without arguments:


\begin{lstlisting}[style=customjava]
	public static void main(String[] args)
	{
		java.awt.Toolkit.getDefaultToolkit().beep();
	};
\end{lstlisting}

\begin{lstlisting}
  public static void main(java.lang.String[]);
    flags: ACC_PUBLIC, ACC_STATIC
    Code:
      stack=1, locals=1, args_size=1
         0: invokestatic  #2      // Method java/awt/Toolkit.getDefaultToolkit:()Ljava/awt/Toolkit;
         3: invokevirtual #3      // Method java/awt/Toolkit.beep:()V
         6: return        
\end{lstlisting}

First \TT{invokestatic} at offset 0 calls\\
\TT{java.awt.Toolkit.getDefaultToolkit()}, 
which returns a reference to an object of class \TT{Toolkit}.\\
The \TT{invokevirtual} instruction at offset 3 calls the \TT{beep()} method of this class.

}
\RU{% TODO proof-reading
\subsection{Вызов beep()}

Вот простейший вызов двух функций без аргументов:

\begin{lstlisting}[style=customjava]
	public static void main(String[] args)
	{
		java.awt.Toolkit.getDefaultToolkit().beep();
	};
\end{lstlisting}

\begin{lstlisting}
  public static void main(java.lang.String[]);
    flags: ACC_PUBLIC, ACC_STATIC
    Code:
      stack=1, locals=1, args_size=1
         0: invokestatic  #2      // Method java/awt/Toolkit.getDefaultToolkit:()Ljava/awt/Toolkit;
         3: invokevirtual #3      // Method java/awt/Toolkit.beep:()V
         6: return        
\end{lstlisting}

Первая \TT{invokestatic} по смещению 0 вызывает\\
\TT{java.awt.Toolkit.getDefaultToolkit()}, 
которая возвращает\\
\IT{reference} (указатель) на объект класса \TT{Toolkit}.\\

Инструкция \TT{invokevirtual} по смещению 3 вызывает метод \TT{beep()} этого класса.
}
\DE{% TODO proof-reading
\subsection{Aufrufen von beep()}

Dies ist ein einfacher Aufruf zweier Funktionen ohne Argumente:

\begin{lstlisting}[style=customjava]
	public static void main(String[] args)
	{
		java.awt.Toolkit.getDefaultToolkit().beep();
	};
\end{lstlisting}

\begin{lstlisting}
  public static void main(java.lang.String[]);
    flags: ACC_PUBLIC, ACC_STATIC
    Code:
      stack=1, locals=1, args_size=1
         0: invokestatic  #2      // Method java/awt/Toolkit.getDefaultToolkit:()Ljava/awt/Toolkit;
         3: invokevirtual #3      // Method java/awt/Toolkit.beep:()V
         6: return        
\end{lstlisting}

Zuerst ruft \TT{invokestatic} bei Offset 0 \TT{java.awt.Toolkit.getDefaultToolkit()}
auf, was eine Referenz auf ein Objekt der Klasse \TT{Toolkit} zurück gibt.
Die \TT{invokevirtual}-Anweisung bei Offset 3 ruft die \TT{beep()}-Methode dieser
Klasse auf.
}

\EN{% TODO proof-reading
\subsection{Linear congruential \ac{PRNG}}

Let's try a simple pseudorandom numbers generator, 
which we already considered once in the book (\myref{LCG_simple}):


\begin{lstlisting}[style=customjava]
public class LCG 
{
	public static int rand_state;

	public void my_srand (int init)
	{
		rand_state=init;
	}

	public static int RNG_a=1664525;
	public static int RNG_c=1013904223;

	public int my_rand ()
	{
		rand_state=rand_state*RNG_a;
		rand_state=rand_state+RNG_c;
		return rand_state & 0x7fff;
	}
}
\end{lstlisting}

There are couple of class fields which are initialized at start.

But how?
In \TT{javap} output we can find the class constructor:


\begin{lstlisting}
  static {};
    flags: ACC_STATIC
    Code:
      stack=1, locals=0, args_size=0
         0: ldc           #5         // int 1664525
         2: putstatic     #3         // Field RNG_a:I
         5: ldc           #6         // int 1013904223
         7: putstatic     #4         // Field RNG_c:I
        10: return        
\end{lstlisting}

That's the way variables are initialized.

\TT{RNG\_a} occupies the 3rd slot in the class and \TT{RNG\_c}---4th, 
and \TT{putstatic} puts the constants there.


The \TT{my\_srand()} function just stores the input value in \TT{rand\_state}:


\begin{lstlisting}
  public void my_srand(int);
    flags: ACC_PUBLIC
    Code:
      stack=1, locals=2, args_size=2
         0: iload_1       
         1: putstatic     #2         // Field rand_state:I
         4: return        
\end{lstlisting}

\TT{iload\_1} takes the input value and pushes it into stack. But why not \TT{iload\_0}?

It's because this function may use fields of the class, and so \IT{this} is also passed to
the function as a zeroth argument.

The field \TT{rand\_state} occupies the 2nd slot in the class,
so \TT{putstatic} copies the value from the \ac{TOS} into the 2nd slot.


Now \TT{my\_rand()}:

\begin{lstlisting}
  public int my_rand();
    flags: ACC_PUBLIC
    Code:
      stack=2, locals=1, args_size=1
         0: getstatic     #2         // Field rand_state:I
         3: getstatic     #3         // Field RNG_a:I
         6: imul          
         7: putstatic     #2         // Field rand_state:I
        10: getstatic     #2         // Field rand_state:I
        13: getstatic     #4         // Field RNG_c:I
        16: iadd          
        17: putstatic     #2         // Field rand_state:I
        20: getstatic     #2         // Field rand_state:I
        23: sipush        32767
        26: iand          
        27: ireturn       
\end{lstlisting}

It just loads all the values from the object's fields, does the operations and updates 
\TT{rand\_state}'s value using the \TT{putstatic} instruction.

At offset 20, \TT{rand\_state} is reloaded again 
(because it has been dropped from the stack before, by \TT{putstatic}).

This looks like non-efficient code, but be sure, the \ac{JVM} is usually good enough to optimize
such things really well.

}
\RU{% TODO proof-reading
\subsection{Линейный конгруэнтный \ac{PRNG}}


Попробуем простой генератор псевдослучайных чисел, 
который мы однажды уже рассматривали в этой книге (\myref{LCG_simple}):

\begin{lstlisting}[style=customjava]
public class LCG 
{
	public static int rand_state;

	public void my_srand (int init)
	{
		rand_state=init;
	}

	public static int RNG_a=1664525;
	public static int RNG_c=1013904223;

	public int my_rand ()
	{
		rand_state=rand_state*RNG_a;
		rand_state=rand_state+RNG_c;
		return rand_state & 0x7fff;
	}
}
\end{lstlisting}


Здесь пара полей класса, которые инициализируются в начале.
Но как?

В выводе \TT{javap} мы можем найти конструктор класса:

\begin{lstlisting}
  static {};
    flags: ACC_STATIC
    Code:
      stack=1, locals=0, args_size=0
         0: ldc           #5         // int 1664525
         2: putstatic     #3         // Field RNG_a:I
         5: ldc           #6         // int 1013904223
         7: putstatic     #4         // Field RNG_c:I
        10: return        
\end{lstlisting}


Так инициализируются переменные.

\TT{RNG\_a} занимает третий слот в классе и \TT{RNG\_c} --- четвертый, 
и \TT{putstatic} записывает туда константы.


Функция \TT{my\_srand()} просто записывает входное значение в \\ 
\TT{rand\_state}:

\begin{lstlisting}
  public void my_srand(int);
    flags: ACC_PUBLIC
    Code:
      stack=1, locals=2, args_size=2
         0: iload_1       
         1: putstatic     #2         // Field rand_state:I
         4: return        
\end{lstlisting}


\TT{iload\_1} берет входное значение и заталкивает его в стек. Но почему не \TT{iload\_0}?

Это потому что эта функция может использовать поля класса, а переменная \IT{this} 
также передается в эту функцию как нулевой аргумент.

Поле \TT{rand\_state} занимает второй слот в классе,
так что \TT{putstatic} копирует переменную из \ac{TOS} во второй слот.

Теперь \TT{my\_rand()}:

\begin{lstlisting}
  public int my_rand();
    flags: ACC_PUBLIC
    Code:
      stack=2, locals=1, args_size=1
         0: getstatic     #2         // Field rand_state:I
         3: getstatic     #3         // Field RNG_a:I
         6: imul          
         7: putstatic     #2         // Field rand_state:I
        10: getstatic     #2         // Field rand_state:I
        13: getstatic     #4         // Field RNG_c:I
        16: iadd          
        17: putstatic     #2         // Field rand_state:I
        20: getstatic     #2         // Field rand_state:I
        23: sipush        32767
        26: iand          
        27: ireturn       
\end{lstlisting}


Она просто загружает все переменные из полей объекта, производит с ними операции
и обновляет значение \TT{rand\_state}, используя инструкцию \TT{putstatic}.

По смещению 20, значение \TT{rand\_state} перезагружается снова
(это потому что оно было выброшено из стека перед этим, инструкцией \TT{putstatic}).

Это выглядит как неэффективный код, но можете быть уверенными, \ac{JVM} обычно достаточно
хорош, чтобы хорошо оптимизировать подобные вещи.
}
\DE{% TODO proof-reading
\subsection{Linearer Kongruenzgenerator \ac{PRNG}}

%Let's try a simple pseudorandom numbers generator, 
%which we already considered once in the book (\myref{LCG_simple}):
%
%
%\begin{lstlisting}[style=customjava]
%public class LCG 
%{
%	public static int rand_state;
%
%	public void my_srand (int init)
%	{
%		rand_state=init;
%	}
%
%	public static int RNG_a=1664525;
%	public static int RNG_c=1013904223;
%
%	public int my_rand ()
%	{
%		rand_state=rand_state*RNG_a;
%		rand_state=rand_state+RNG_c;
%		return rand_state & 0x7fff;
%	}
%}
%\end{lstlisting}
%
%There are couple of class fields which are initialized at start.
%
%But how?
%In \TT{javap} output we can find the class constructor:
%
%
%\begin{lstlisting}
%  static {};
%    flags: ACC_STATIC
%    Code:
%      stack=1, locals=0, args_size=0
%         0: ldc           #5         // int 1664525
%         2: putstatic     #3         // Field RNG_a:I
%         5: ldc           #6         // int 1013904223
%         7: putstatic     #4         // Field RNG_c:I
%        10: return        
%\end{lstlisting}
%
%That's the way variables are initialized.
%
%\TT{RNG\_a} occupies the 3rd slot in the class and \TT{RNG\_c}---4th, 
%and \TT{putstatic} puts the constants there.
%
%
%The \TT{my\_srand()} function just stores the input value in \TT{rand\_state}:
%
%
%\begin{lstlisting}
%  public void my_srand(int);
%    flags: ACC_PUBLIC
%    Code:
%      stack=1, locals=2, args_size=2
%         0: iload_1       
%         1: putstatic     #2         // Field rand_state:I
%         4: return        
%\end{lstlisting}
%
%\TT{iload\_1} takes the input value and pushes it into stack. But why not \TT{iload\_0}?
%
%It's because this function may use fields of the class, and so \IT{this} is also passed to
%the function as a zeroth argument.
%
%The field \TT{rand\_state} occupies the 2nd slot in the class,
%so \TT{putstatic} copies the value from the \ac{TOS} into the 2nd slot.
%
%
%Now \TT{my\_rand()}:
%
%\begin{lstlisting}
%  public int my_rand();
%    flags: ACC_PUBLIC
%    Code:
%      stack=2, locals=1, args_size=1
%         0: getstatic     #2         // Field rand_state:I
%         3: getstatic     #3         // Field RNG_a:I
%         6: imul          
%         7: putstatic     #2         // Field rand_state:I
%        10: getstatic     #2         // Field rand_state:I
%        13: getstatic     #4         // Field RNG_c:I
%        16: iadd          
%        17: putstatic     #2         // Field rand_state:I
%        20: getstatic     #2         // Field rand_state:I
%        23: sipush        32767
%        26: iand          
%        27: ireturn       
%\end{lstlisting}
%
%It just loads all the values from the object's fields, does the operations and updates 
%\TT{rand\_state}'s value using the \TT{putstatic} instruction.
%
%At offset 20, \TT{rand\_state} is reloaded again 
%(because it has been dropped from the stack before, by \TT{putstatic}).
%
%This looks like non-efficient code, but be sure, the \ac{JVM} is usually good enough to optimize
%such things really well.
%
}

\EN{% TODO proof-reading
\subsection{Conditional jumps}

Now let's proceed to conditional jumps.

\begin{lstlisting}[style=customjava]
public class abs
{
	public static int abs(int a)
	{
		if (a<0)
			return -a;
		return a;
	}
}
\end{lstlisting}

\begin{lstlisting}
  public static int abs(int);
    flags: ACC_PUBLIC, ACC_STATIC
    Code:
      stack=1, locals=1, args_size=1
         0: iload_0       
         1: ifge          7
         4: iload_0       
         5: ineg          
         6: ireturn       
         7: iload_0       
         8: ireturn       
\end{lstlisting}

\TT{ifge} jumps to offset 7 if the value at \ac{TOS} is greater or equal to 0.

Don't forget, any \TT{ifXX} instruction pops the value (to be compared) from the stack.


\TT{ineg} just negates value at \ac{TOS}.


Another example:

\begin{lstlisting}[style=customjava]
	public static int min (int a, int b)
	{
		if (a>b)
			return b;
		return a;
	}
\end{lstlisting}

We get:

\begin{lstlisting}
  public static int min(int, int);
    flags: ACC_PUBLIC, ACC_STATIC
    Code:
      stack=2, locals=2, args_size=2
         0: iload_0       
         1: iload_1       
         2: if_icmple     7
         5: iload_1       
         6: ireturn       
         7: iload_0       
         8: ireturn       
\end{lstlisting}

\TT{if\_icmple} pops two values and compares them.
If the second one is lesser than (or equal to) the first, a jump to offset 7 is performed.


When we define \TT{max()} function \dots


\begin{lstlisting}[style=customjava]
	public static int max (int a, int b)
	{
		if (a>b)
			return a;
		return b;
	}
\end{lstlisting}

\dots the resulting code is the same, but the last two \TT{iload} instructions 
(at offsets 5 and 7) are swapped:


\begin{lstlisting}
  public static int max(int, int);
    flags: ACC_PUBLIC, ACC_STATIC
    Code:
      stack=2, locals=2, args_size=2
         0: iload_0       
         1: iload_1       
         2: if_icmple     7
         5: iload_0       
         6: ireturn       
         7: iload_1       
         8: ireturn       
\end{lstlisting}

A more advanced example:

\begin{lstlisting}[style=customjava]
public class cond
{
	public static void f(int i)
	{
		if (i<100)
			System.out.print("<100");
		if (i==100)
			System.out.print("==100");
		if (i>100)
			System.out.print(">100");
		if (i==0)
			System.out.print("==0");
	}
}
\end{lstlisting}

\begin{lstlisting}
  public static void f(int);
    flags: ACC_PUBLIC, ACC_STATIC
    Code:
      stack=2, locals=1, args_size=1
         0: iload_0       
         1: bipush        100
         3: if_icmpge     14
         6: getstatic     #2        // Field java/lang/System.out:Ljava/io/PrintStream;
         9: ldc           #3        // String <100
        11: invokevirtual #4        // Method java/io/PrintStream.print:(Ljava/lang/String;)V
        14: iload_0       
        15: bipush        100
        17: if_icmpne     28
        20: getstatic     #2        // Field java/lang/System.out:Ljava/io/PrintStream;
        23: ldc           #5        // String ==100
        25: invokevirtual #4        // Method java/io/PrintStream.print:(Ljava/lang/String;)V
        28: iload_0       
        29: bipush        100
        31: if_icmple     42
        34: getstatic     #2        // Field java/lang/System.out:Ljava/io/PrintStream;
        37: ldc           #6        // String >100
        39: invokevirtual #4        // Method java/io/PrintStream.print:(Ljava/lang/String;)V
        42: iload_0       
        43: ifne          54
        46: getstatic     #2        // Field java/lang/System.out:Ljava/io/PrintStream;
        49: ldc           #7        // String ==0
        51: invokevirtual #4        // Method java/io/PrintStream.print:(Ljava/lang/String;)V
        54: return        
\end{lstlisting}

\TT{if\_icmpge} pops two values and compares them.
If the second one is larger than the first, a jump to offset 14 is performed.

\TT{if\_icmpne} and \TT{if\_icmple} work just the same, but implement different conditions.


There is also a \TT{ifne} instruction at offset 43.

Its name is misnomer, it would've be better to name it \TT{ifnz} 
(jump if the value at \ac{TOS} is not zero).

And that is what it does: it jumps to offset 54 if the input value is not zero.

If zero,the  execution flow proceeds to offset 46, where the \q{==0} string is printed.


N.B.: \ac{JVM} has no unsigned data types, so the comparison instructions operate 
only on signed integer values.

}
\RU{% TODO proof-reading
\subsection{Условные переходы}

Перейдем к условным переходам.

\begin{lstlisting}[style=customjava]
public class abs
{
	public static int abs(int a)
	{
		if (a<0)
			return -a;
		return a;
	}
}
\end{lstlisting}

\begin{lstlisting}
  public static int abs(int);
    flags: ACC_PUBLIC, ACC_STATIC
    Code:
      stack=1, locals=1, args_size=1
         0: iload_0       
         1: ifge          7
         4: iload_0       
         5: ineg          
         6: ireturn       
         7: iload_0       
         8: ireturn       
\end{lstlisting}


\TT{ifge} переходит на смещение 7 если значение на \ac{TOS} больше или равно 0.

Не забывайте, любая инструкция \TT{ifXX} выталкивает значение (с которым будет производиться
сравнение) из стека.


\TT{ineg} просто меняет знак значения на \ac{TOS}.

Еще пример:

\begin{lstlisting}[style=customjava]
	public static int min (int a, int b)
	{
		if (a>b)
			return b;
		return a;
	}
\end{lstlisting}

Получаем:

\begin{lstlisting}
  public static int min(int, int);
    flags: ACC_PUBLIC, ACC_STATIC
    Code:
      stack=2, locals=2, args_size=2
         0: iload_0       
         1: iload_1       
         2: if_icmple     7
         5: iload_1       
         6: ireturn       
         7: iload_0       
         8: ireturn       
\end{lstlisting}

\TT{if\_icmple} выталкивает два значения и сравнивает их.

Если второе меньше первого (или равно), происходит переход на смещение 7.


Когда мы определяем функцию \TT{max()} \dots

\begin{lstlisting}[style=customjava]
	public static int max (int a, int b)
	{
		if (a>b)
			return a;
		return b;
	}
\end{lstlisting}


\dots итоговый код точно такой же, только последние инструкции \TT{iload} 
(на смещениях 5 и 7) поменяны местами:

\begin{lstlisting}
  public static int max(int, int);
    flags: ACC_PUBLIC, ACC_STATIC
    Code:
      stack=2, locals=2, args_size=2
         0: iload_0       
         1: iload_1       
         2: if_icmple     7
         5: iload_0       
         6: ireturn       
         7: iload_1       
         8: ireturn       
\end{lstlisting}

Более сложный пример:

\begin{lstlisting}[style=customjava]
public class cond
{
	public static void f(int i)
	{
		if (i<100)
			System.out.print("<100");
		if (i==100)
			System.out.print("==100");
		if (i>100)
			System.out.print(">100");
		if (i==0)
			System.out.print("==0");
	}
}
\end{lstlisting}

\begin{lstlisting}
  public static void f(int);
    flags: ACC_PUBLIC, ACC_STATIC
    Code:
      stack=2, locals=1, args_size=1
         0: iload_0       
         1: bipush        100
         3: if_icmpge     14
         6: getstatic     #2        // Field java/lang/System.out:Ljava/io/PrintStream;
         9: ldc           #3        // String <100
        11: invokevirtual #4        // Method java/io/PrintStream.print:(Ljava/lang/String;)V
        14: iload_0       
        15: bipush        100
        17: if_icmpne     28
        20: getstatic     #2        // Field java/lang/System.out:Ljava/io/PrintStream;
        23: ldc           #5        // String ==100
        25: invokevirtual #4        // Method java/io/PrintStream.print:(Ljava/lang/String;)V
        28: iload_0       
        29: bipush        100
        31: if_icmple     42
        34: getstatic     #2        // Field java/lang/System.out:Ljava/io/PrintStream;
        37: ldc           #6        // String >100
        39: invokevirtual #4        // Method java/io/PrintStream.print:(Ljava/lang/String;)V
        42: iload_0       
        43: ifne          54
        46: getstatic     #2        // Field java/lang/System.out:Ljava/io/PrintStream;
        49: ldc           #7        // String ==0
        51: invokevirtual #4        // Method java/io/PrintStream.print:(Ljava/lang/String;)V
        54: return        
\end{lstlisting}

\TT{if\_icmpge} Выталкивает два значения и сравнивает их.

Если второй больше первого, происходит переход на смещение 14.

\TT{if\_icmpne} и \TT{if\_icmple} работают одинаково, но используются разные условия.


По смещению 43 есть также инструкция \TT{ifne}.

Название неудачное, её было бы лучше назвать \TT{ifnz} 
(переход если переменная на \ac{TOS} не равна нулю).

И вот что она делает: производит переход на смещение 54, если входное значение не ноль.

Если ноль, управление передается на смещение 46, где выводится строка \q{==0}.


N.B.: В \ac{JVM} нет беззнаковых типов данных, так что инструкции сравнения работают
только с знаковыми челочисленными значениями.
}
\DE{% TODO proof-reading
\subsection{Bedingte Sprünge}

%Now let's proceed to conditional jumps.
%
%\begin{lstlisting}[style=customjava]
%public class abs
%{
%	public static int abs(int a)
%	{
%		if (a<0)
%			return -a;
%		return a;
%	}
%}
%\end{lstlisting}
%
%\begin{lstlisting}
%  public static int abs(int);
%    flags: ACC_PUBLIC, ACC_STATIC
%    Code:
%      stack=1, locals=1, args_size=1
%         0: iload_0       
%         1: ifge          7
%         4: iload_0       
%         5: ineg          
%         6: ireturn       
%         7: iload_0       
%         8: ireturn       
%\end{lstlisting}
%
%\TT{ifge} jumps to offset 7 if the value at \ac{TOS} is greater or equal to 0.
%
%Don't forget, any \TT{ifXX} instruction pops the value (to be compared) from the stack.
%
%
%\TT{ineg} just negates value at \ac{TOS}.
%
%
%Another example:
%
%\begin{lstlisting}[style=customjava]
%	public static int min (int a, int b)
%	{
%		if (a>b)
%			return b;
%		return a;
%	}
%\end{lstlisting}
%
%We get:
%
%\begin{lstlisting}
%  public static int min(int, int);
%    flags: ACC_PUBLIC, ACC_STATIC
%    Code:
%      stack=2, locals=2, args_size=2
%         0: iload_0       
%         1: iload_1       
%         2: if_icmple     7
%         5: iload_1       
%         6: ireturn       
%         7: iload_0       
%         8: ireturn       
%\end{lstlisting}
%
%\TT{if\_icmple} pops two values and compares them.
%If the second one is lesser than (or equal to) the first, a jump to offset 7 is performed.
%
%
%When we define \TT{max()} function \dots
%
%
%\begin{lstlisting}[style=customjava]
%	public static int max (int a, int b)
%	{
%		if (a>b)
%			return a;
%		return b;
%	}
%\end{lstlisting}
%
%\dots the resulting code is the same, but the last two \TT{iload} instructions 
%(at offsets 5 and 7) are swapped:
%
%
%\begin{lstlisting}
%  public static int max(int, int);
%    flags: ACC_PUBLIC, ACC_STATIC
%    Code:
%      stack=2, locals=2, args_size=2
%         0: iload_0       
%         1: iload_1       
%         2: if_icmple     7
%         5: iload_0       
%         6: ireturn       
%         7: iload_1       
%         8: ireturn       
%\end{lstlisting}
%
%A more advanced example:
%
%\begin{lstlisting}[style=customjava]
%public class cond
%{
%	public static void f(int i)
%	{
%		if (i<100)
%			System.out.print("<100");
%		if (i==100)
%			System.out.print("==100");
%		if (i>100)
%			System.out.print(">100");
%		if (i==0)
%			System.out.print("==0");
%	}
%}
%\end{lstlisting}
%
%\begin{lstlisting}
%  public static void f(int);
%    flags: ACC_PUBLIC, ACC_STATIC
%    Code:
%      stack=2, locals=1, args_size=1
%         0: iload_0       
%         1: bipush        100
%         3: if_icmpge     14
%         6: getstatic     #2        // Field java/lang/System.out:Ljava/io/PrintStream;
%         9: ldc           #3        // String <100
%        11: invokevirtual #4        // Method java/io/PrintStream.print:(Ljava/lang/String;)V
%        14: iload_0       
%        15: bipush        100
%        17: if_icmpne     28
%        20: getstatic     #2        // Field java/lang/System.out:Ljava/io/PrintStream;
%        23: ldc           #5        // String ==100
%        25: invokevirtual #4        // Method java/io/PrintStream.print:(Ljava/lang/String;)V
%        28: iload_0       
%        29: bipush        100
%        31: if_icmple     42
%        34: getstatic     #2        // Field java/lang/System.out:Ljava/io/PrintStream;
%        37: ldc           #6        // String >100
%        39: invokevirtual #4        // Method java/io/PrintStream.print:(Ljava/lang/String;)V
%        42: iload_0       
%        43: ifne          54
%        46: getstatic     #2        // Field java/lang/System.out:Ljava/io/PrintStream;
%        49: ldc           #7        // String ==0
%        51: invokevirtual #4        // Method java/io/PrintStream.print:(Ljava/lang/String;)V
%        54: return        
%\end{lstlisting}
%
%\TT{if\_icmpge} pops two values and compares them.
%If the second one is larger than the first, a jump to offset 14 is performed.
%
%\TT{if\_icmpne} and \TT{if\_icmple} work just the same, but implement different conditions.
%
%
%There is also a \TT{ifne} instruction at offset 43.
%
%Its name is misnomer, it would've be better to name it \TT{ifnz} 
%(jump if the value at \ac{TOS} is not zero).
%
%And that is what it does: it jumps to offset 54 if the input value is not zero.
%
%If zero,the  execution flow proceeds to offset 46, where the \q{==0} string is printed.
%
%
%N.B.: \ac{JVM} has no unsigned data types, so the comparison instructions operate 
%only on signed integer values.
%
}

\EN{% TODO proof-reading
\subsection{Passing arguments}

Let's extend our \TT{min()}/\TT{max()} example:


\begin{lstlisting}[style=customjava]
public class minmax
{
	public static int min (int a, int b)
	{
		if (a>b)
			return b;
		return a;
	}

	public static int max (int a, int b)
	{
		if (a>b)
			return a;
		return b;
	}

	public static void main(String[] args)
	{
		int a=123, b=456;
		int max_value=max(a, b);
		int min_value=min(a, b);
		System.out.println(min_value);
		System.out.println(max_value);
	}
}
\end{lstlisting}

Here is \main function code:


\begin{lstlisting}
  public static void main(java.lang.String[]);
    flags: ACC_PUBLIC, ACC_STATIC
    Code:
      stack=2, locals=5, args_size=1
         0: bipush        123
         2: istore_1      
         3: sipush        456
         6: istore_2      
         7: iload_1       
         8: iload_2       
         9: invokestatic  #2       // Method max:(II)I
        12: istore_3      
        13: iload_1       
        14: iload_2       
        15: invokestatic  #3       // Method min:(II)I
        18: istore        4
        20: getstatic     #4       // Field java/lang/System.out:Ljava/io/PrintStream;
        23: iload         4
        25: invokevirtual #5       // Method java/io/PrintStream.println:(I)V
        28: getstatic     #4       // Field java/lang/System.out:Ljava/io/PrintStream;
        31: iload_3       
        32: invokevirtual #5       // Method java/io/PrintStream.println:(I)V
        35: return        
\end{lstlisting}

Arguments are passed to the other function in the stack, and the return value is left on \ac{TOS}.


}
\RU{% TODO proof-reading
\subsection{Передача аргументов}


Теперь расширим пример \TT{min()}/\TT{max()}:

\begin{lstlisting}[style=customjava]
public class minmax
{
	public static int min (int a, int b)
	{
		if (a>b)
			return b;
		return a;
	}

	public static int max (int a, int b)
	{
		if (a>b)
			return a;
		return b;
	}

	public static void main(String[] args)
	{
		int a=123, b=456;
		int max_value=max(a, b);
		int min_value=min(a, b);
		System.out.println(min_value);
		System.out.println(max_value);
	}
}
\end{lstlisting}


Вот код функции \main:

\begin{lstlisting}
  public static void main(java.lang.String[]);
    flags: ACC_PUBLIC, ACC_STATIC
    Code:
      stack=2, locals=5, args_size=1
         0: bipush        123
         2: istore_1      
         3: sipush        456
         6: istore_2      
         7: iload_1       
         8: iload_2       
         9: invokestatic  #2       // Method max:(II)I
        12: istore_3      
        13: iload_1       
        14: iload_2       
        15: invokestatic  #3       // Method min:(II)I
        18: istore        4
        20: getstatic     #4       // Field java/lang/System.out:Ljava/io/PrintStream;
        23: iload         4
        25: invokevirtual #5       // Method java/io/PrintStream.println:(I)V
        28: getstatic     #4       // Field java/lang/System.out:Ljava/io/PrintStream;
        31: iload_3       
        32: invokevirtual #5       // Method java/io/PrintStream.println:(I)V
        35: return        
\end{lstlisting}


В другую функцию аргументы передаются в стеке, а возвращаемое значение остается
на \ac{TOS}.

}
\DE{% TODO proof-reading
\subsection{Argumente übergeben}

%Let's extend our \TT{min()}/\TT{max()} example:
%
%
%\begin{lstlisting}[style=customjava]
%public class minmax
%{
%	public static int min (int a, int b)
%	{
%		if (a>b)
%			return b;
%		return a;
%	}
%
%	public static int max (int a, int b)
%	{
%		if (a>b)
%			return a;
%		return b;
%	}
%
%	public static void main(String[] args)
%	{
%		int a=123, b=456;
%		int max_value=max(a, b);
%		int min_value=min(a, b);
%		System.out.println(min_value);
%		System.out.println(max_value);
%	}
%}
%\end{lstlisting}
%
%Here is \main function code:
%
%
%\begin{lstlisting}
%  public static void main(java.lang.String[]);
%    flags: ACC_PUBLIC, ACC_STATIC
%    Code:
%      stack=2, locals=5, args_size=1
%         0: bipush        123
%         2: istore_1      
%         3: sipush        456
%         6: istore_2      
%         7: iload_1       
%         8: iload_2       
%         9: invokestatic  #2       // Method max:(II)I
%        12: istore_3      
%        13: iload_1       
%        14: iload_2       
%        15: invokestatic  #3       // Method min:(II)I
%        18: istore        4
%        20: getstatic     #4       // Field java/lang/System.out:Ljava/io/PrintStream;
%        23: iload         4
%        25: invokevirtual #5       // Method java/io/PrintStream.println:(I)V
%        28: getstatic     #4       // Field java/lang/System.out:Ljava/io/PrintStream;
%        31: iload_3       
%        32: invokevirtual #5       // Method java/io/PrintStream.println:(I)V
%        35: return        
%\end{lstlisting}
%
%Arguments are passed to the other function in the stack, and the return value is left on \ac{TOS}.
%
%
}

\EN{% TODO proof-reading
\subsection{Bitfields}

All bit-wise operations work just like in any other \ac{ISA}:


\begin{lstlisting}[style=customjava]
	public static int set (int a, int b) 
	{
		return a | 1<<b;
	}

	public static int clear (int a, int b) 
	{
		return a & (~(1<<b));
	}
\end{lstlisting}

\begin{lstlisting}
  public static int set(int, int);
    flags: ACC_PUBLIC, ACC_STATIC
    Code:
      stack=3, locals=2, args_size=2
         0: iload_0       
         1: iconst_1      
         2: iload_1       
         3: ishl          
         4: ior           
         5: ireturn       

  public static int clear(int, int);
    flags: ACC_PUBLIC, ACC_STATIC
    Code:
      stack=3, locals=2, args_size=2
         0: iload_0       
         1: iconst_1      
         2: iload_1       
         3: ishl          
         4: iconst_m1     
         5: ixor          
         6: iand          
         7: ireturn       
\end{lstlisting}

\TT{iconst\_m1} loads $-1$ in the stack, it's the same as the \TT{0xFFFFFFFF} number.

XORing with \TT{0xFFFFFFFF} has the same effect of inverting all bits
 (\myref{XOR_property}).

Let's extend all data types to 64-bit \IT{long}:


\begin{lstlisting}[style=customjava]
	public static long lset (long a, int b) 
	{
		return a | 1<<b;
	}

	public static long lclear (long a, int b) 
	{
		return a & (~(1<<b));
	}
\end{lstlisting}

\begin{lstlisting}
  public static long lset(long, int);
    flags: ACC_PUBLIC, ACC_STATIC
    Code:
      stack=4, locals=3, args_size=2
         0: lload_0       
         1: iconst_1      
         2: iload_2       
         3: ishl          
         4: i2l           
         5: lor           
         6: lreturn       

  public static long lclear(long, int);
    flags: ACC_PUBLIC, ACC_STATIC
    Code:
      stack=4, locals=3, args_size=2
         0: lload_0       
         1: iconst_1      
         2: iload_2       
         3: ishl          
         4: iconst_m1     
         5: ixor          
         6: i2l           
         7: land          
         8: lreturn       
\end{lstlisting}

The code is the same, but instructions with \IT{l} prefix are used, which operate 
on 64-bit values.

Also, the second argument of the function still is of type \IT{int}, and when the 32-bit value in it 
needs to be promoted to 64-bit value the \TT{i2l} instruction is used, 
which essentially extend the value of an \IT{integer} type to a \IT{long} one.

}
\RU{% TODO proof-reading
\subsection{Битовые поля}


Все побитовые операции работают также, как и в любой другой \ac{ISA}:

\begin{lstlisting}[style=customjava]
	public static int set (int a, int b) 
	{
		return a | 1<<b;
	}

	public static int clear (int a, int b) 
	{
		return a & (~(1<<b));
	}
\end{lstlisting}

\begin{lstlisting}
  public static int set(int, int);
    flags: ACC_PUBLIC, ACC_STATIC
    Code:
      stack=3, locals=2, args_size=2
         0: iload_0       
         1: iconst_1      
         2: iload_1       
         3: ishl          
         4: ior           
         5: ireturn       

  public static int clear(int, int);
    flags: ACC_PUBLIC, ACC_STATIC
    Code:
      stack=3, locals=2, args_size=2
         0: iload_0       
         1: iconst_1      
         2: iload_1       
         3: ishl          
         4: iconst_m1     
         5: ixor          
         6: iand          
         7: ireturn       
\end{lstlisting}


\TT{iconst\_m1} загружает $-1$ в стек, это то же что и значение \TT{0xFFFFFFFF}.

Операция XOR с \TT{0xFFFFFFFF} в одном из операндов, это тот же эффект что инвертирование всех бит (\myref{XOR_property}).


Попробуем также расширить все типы данных до 64-битного \IT{long}:

\begin{lstlisting}[style=customjava]
	public static long lset (long a, int b) 
	{
		return a | 1<<b;
	}

	public static long lclear (long a, int b) 
	{
		return a & (~(1<<b));
	}
\end{lstlisting}

\begin{lstlisting}
  public static long lset(long, int);
    flags: ACC_PUBLIC, ACC_STATIC
    Code:
      stack=4, locals=3, args_size=2
         0: lload_0       
         1: iconst_1      
         2: iload_2       
         3: ishl          
         4: i2l           
         5: lor           
         6: lreturn       

  public static long lclear(long, int);
    flags: ACC_PUBLIC, ACC_STATIC
    Code:
      stack=4, locals=3, args_size=2
         0: lload_0       
         1: iconst_1      
         2: iload_2       
         3: ishl          
         4: iconst_m1     
         5: ixor          
         6: i2l           
         7: land          
         8: lreturn       
\end{lstlisting}


Код такой же, но используются инструкции с префиксом \IT{l}, которые работают 
с 64-битными значениями.

Так же, второй аргумент функции все еще имеет тип \IT{int}, и когда 32-битное число в нем
должно быть расширено до 64-битного значения, используется инструкция \TT{i2l}, 
которая расширяет значение типа \IT{integer} в значение типа \IT{long}.
}
\DE{% TODO proof-reading
\subsection{Bit-Felder}

%All bit-wise operations work just like in any other \ac{ISA}:
%
%
%\begin{lstlisting}[style=customjava]
%	public static int set (int a, int b) 
%	{
%		return a | 1<<b;
%	}
%
%	public static int clear (int a, int b) 
%	{
%		return a & (~(1<<b));
%	}
%\end{lstlisting}
%
%\begin{lstlisting}
%  public static int set(int, int);
%    flags: ACC_PUBLIC, ACC_STATIC
%    Code:
%      stack=3, locals=2, args_size=2
%         0: iload_0       
%         1: iconst_1      
%         2: iload_1       
%         3: ishl          
%         4: ior           
%         5: ireturn       
%
%  public static int clear(int, int);
%    flags: ACC_PUBLIC, ACC_STATIC
%    Code:
%      stack=3, locals=2, args_size=2
%         0: iload_0       
%         1: iconst_1      
%         2: iload_1       
%         3: ishl          
%         4: iconst_m1     
%         5: ixor          
%         6: iand          
%         7: ireturn       
%\end{lstlisting}
%
%\TT{iconst\_m1} loads $-1$ in the stack, it's the same as the \TT{0xFFFFFFFF} number.
%
%XORing with \TT{0xFFFFFFFF} has the same effect of inverting all bits
% (\myref{XOR_property}).
%
%Let's extend all data types to 64-bit \IT{long}:
%
%
%\begin{lstlisting}[style=customjava]
%	public static long lset (long a, int b) 
%	{
%		return a | 1<<b;
%	}
%
%	public static long lclear (long a, int b) 
%	{
%		return a & (~(1<<b));
%	}
%\end{lstlisting}
%
%\begin{lstlisting}
%  public static long lset(long, int);
%    flags: ACC_PUBLIC, ACC_STATIC
%    Code:
%      stack=4, locals=3, args_size=2
%         0: lload_0       
%         1: iconst_1      
%         2: iload_2       
%         3: ishl          
%         4: i2l           
%         5: lor           
%         6: lreturn       
%
%  public static long lclear(long, int);
%    flags: ACC_PUBLIC, ACC_STATIC
%    Code:
%      stack=4, locals=3, args_size=2
%         0: lload_0       
%         1: iconst_1      
%         2: iload_2       
%         3: ishl          
%         4: iconst_m1     
%         5: ixor          
%         6: i2l           
%         7: land          
%         8: lreturn       
%\end{lstlisting}
%
%The code is the same, but instructions with \IT{l} prefix are used, which operate 
%on 64-bit values.
%
%Also, the second argument of the function still is of type \IT{int}, and when the 32-bit value in it 
%needs to be promoted to 64-bit value the \TT{i2l} instruction is used, 
%which essentially extend the value of an \IT{integer} type to a \IT{long} one.
%
}

\EN{% TODO proof-reading
\subsection{Loops}

\begin{lstlisting}[style=customjava]
public class Loop
{
	public static void main(String[] args)
	{ 
		for (int i = 1; i <= 10; i++)
		{
			System.out.println(i); 
		}               
	}
}
\end{lstlisting}

\begin{lstlisting}
  public static void main(java.lang.String[]);
    flags: ACC_PUBLIC, ACC_STATIC
    Code:
      stack=2, locals=2, args_size=1
         0: iconst_1      
         1: istore_1      
         2: iload_1       
         3: bipush        10
         5: if_icmpgt     21
         8: getstatic     #2         // Field java/lang/System.out:Ljava/io/PrintStream;
        11: iload_1       
        12: invokevirtual #3         // Method java/io/PrintStream.println:(I)V
        15: iinc          1, 1
        18: goto          2
        21: return        
\end{lstlisting}

\TT{iconst\_1} loads 1 into \ac{TOS}, \TT{istore\_1} stores it in the \ac{LVA} at slot 1.

Why not the zeroth slot?
Because the \main function has one argument (array of \TT{String}) 
and a pointer to it (or \IT{reference}) is now in the zeroth slot.


So, the \IT{i} local variable will always be in 1st slot.


Instructions at offsets 3 and 5 compare \IT{i} with 10.

If \IT{i} is larger, execution flow passes to offset 21, where the function ends.

If it's not, \TT{println} is called.

\IT{i} is then reloaded at offset 11, for \TT{println}.

By the way, we call the \TT{println} method for an \IT{integer}, 
and we see this in the comments: \q{(I)V}
(\IT{I} means \IT{integer} and \IT{V} means the return type is \IT{void}).


When \TT{println} finishes, \IT{i} is incremented at offset 15.

The first operand of the instruction is the  number of a slot (1), 
the second is the number (1) to add to the variable.


\TT{goto} is just GOTO, it jumps to the beginning of the loop's body offset 2.


Let's proceed with a more complex example:


\begin{lstlisting}[style=customjava]
public class Fibonacci
{
	public static void main(String[] args)
	{ 
		int limit = 20, f = 0, g = 1;

		for (int i = 1; i <= limit; i++)
		{
			f = f + g;
			g = f - g;
			System.out.println(f); 
		}
	}
}
\end{lstlisting}

\begin{lstlisting}
  public static void main(java.lang.String[]);
    flags: ACC_PUBLIC, ACC_STATIC
    Code:
      stack=2, locals=5, args_size=1
         0: bipush        20
         2: istore_1      
         3: iconst_0      
         4: istore_2      
         5: iconst_1      
         6: istore_3      
         7: iconst_1      
         8: istore        4
        10: iload         4
        12: iload_1       
        13: if_icmpgt     37
        16: iload_2       
        17: iload_3       
        18: iadd          
        19: istore_2      
        20: iload_2       
        21: iload_3       
        22: isub          
        23: istore_3      
        24: getstatic     #2         // Field java/lang/System.out:Ljava/io/PrintStream;
        27: iload_2       
        28: invokevirtual #3         // Method java/io/PrintStream.println:(I)V
        31: iinc          4, 1
        34: goto          10
        37: return        
\end{lstlisting}
        
Here is a map of the \ac{LVA} slots:


\begin{itemize}
\item 0 --- the sole argument of \main
\item 1 --- \IT{limit}, always contains 20
\item 2 --- \IT{f}
\item 3 --- \IT{g}
\item 4 --- \IT{i}
\end{itemize}

We can see that the Java compiler allocates variables in \ac{LVA} slots in the same order 
they were declared in the source code.


There are separate \TT{istore} instructions for accessing slots 0, 1, 2 and 3, 
but not for 4 and larger, so there is \TT{istore} with an additional operand at offset 8 
which takes the slot number as an operand.

It's the same with \TT{iload} at offset 10.


But isn't it dubious to allocate another slot for the \IT{limit} variable, which 
always contains 20 (so it's a constant in essence), and reload its value so often?

\ac{JVM} \ac{JIT} compiler is usually good enough to optimize such things.

Manual intervention in the code is probably not worth it.


}
\RU{% TODO proof-reading
\subsection{Циклы}

\begin{lstlisting}[style=customjava]
public class Loop
{
	public static void main(String[] args)
	{ 
		for (int i = 1; i <= 10; i++)
		{
			System.out.println(i); 
		}               
	}
}
\end{lstlisting}

\begin{lstlisting}
  public static void main(java.lang.String[]);
    flags: ACC_PUBLIC, ACC_STATIC
    Code:
      stack=2, locals=2, args_size=1
         0: iconst_1      
         1: istore_1      
         2: iload_1       
         3: bipush        10
         5: if_icmpgt     21
         8: getstatic     #2         // Field java/lang/System.out:Ljava/io/PrintStream;
        11: iload_1       
        12: invokevirtual #3         // Method java/io/PrintStream.println:(I)V
        15: iinc          1, 1
        18: goto          2
        21: return        
\end{lstlisting}


\TT{iconst\_1} загружает 1 в \ac{TOS}, \TT{istore\_1} сохраняет её в первом слоте \ac{LVA}.
Почему не нулевой слот?

Потому что функция \main имеет один аргумент (массив \TT{String}),
и указатель на него (или \IT{reference}) сейчас в нулевом слоте.


Так что локальная переменная \IT{i} всегда будет в первом слоте.


Инструкции по смещениями 3 и 5 сравнивают \IT{i} с 10.

Если \IT{i} больше, управление передается на смещение 21, где функция заканчивает работу.

Если нет, вызывается \TT{println}.

\IT{i} перезагружается по смещению 11, для \TT{println}.

Кстати, мы вызываем метод \TT{println} для типа данных \IT{integer}, 
и мы видим это в комментариях: \q{(I)V}
(\IT{I} означает \IT{integer} и \IT{V} означает, что возвращаемое значение имеет тип \IT{void}).


Когда \TT{println} заканчивается, \IT{i} увеличивается на 1 по смещению 15.

Первый операнд инструкции это номер слота (1), 
второй это число (1) для прибавления.


\TT{goto} это просто GOTO, она переходит на начало цикла по смещению 2.


Перейдем к более сложному примеру:

\begin{lstlisting}[style=customjava]
public class Fibonacci
{
	public static void main(String[] args)
	{ 
		int limit = 20, f = 0, g = 1;

		for (int i = 1; i <= limit; i++)
		{
			f = f + g;
			g = f - g;
			System.out.println(f); 
		}
	}
}
\end{lstlisting}

\begin{lstlisting}
  public static void main(java.lang.String[]);
    flags: ACC_PUBLIC, ACC_STATIC
    Code:
      stack=2, locals=5, args_size=1
         0: bipush        20
         2: istore_1      
         3: iconst_0      
         4: istore_2      
         5: iconst_1      
         6: istore_3      
         7: iconst_1      
         8: istore        4
        10: iload         4
        12: iload_1       
        13: if_icmpgt     37
        16: iload_2       
        17: iload_3       
        18: iadd          
        19: istore_2      
        20: iload_2       
        21: iload_3       
        22: isub          
        23: istore_3      
        24: getstatic     #2         // Field java/lang/System.out:Ljava/io/PrintStream;
        27: iload_2       
        28: invokevirtual #3         // Method java/io/PrintStream.println:(I)V
        31: iinc          4, 1
        34: goto          10
        37: return        
\end{lstlisting}
        

Вот карта слотов в \ac{LVA}:

\begin{itemize}
\item 0 --- единственный аргумент функции \main
\item 1 --- \IT{limit}, всегда содержит 20
\item 2 --- \IT{f}
\item 3 --- \IT{g}
\item 4 --- \IT{i}
\end{itemize}


Мы видим, что компилятор Java расположил переменные в слотах \ac{LVA} в точно таком же порядке,
в котором переменные были определены в исходном коде.


Существуют отдельные инструкции \TT{istore} для слотов 0, 1, 2, 3, но не 4 и более, 
так что здесь есть \TT{istore} с дополнительным операндом по смещению 8, 
которая имеет номер слота в операнде.

Та же история с \TT{iload} по смещению 10.


Но не слишком ли это сомнительно, выделить целый слот для переменной \IT{limit},
которая всегда содержит 20 (так что это по сути константа), и перезагружать её так часто?

\ac{JIT}-компилятор в \ac{JVM} обычно достаточно хорош, чтобы всё это оптимизировать.

Самостоятельное вмешательство в код, наверное, того не стоит.

}
\DE{% TODO proof-reading
\subsection{Schleifen}

%\begin{lstlisting}[style=customjava]
%public class Loop
%{
%	public static void main(String[] args)
%	{ 
%		for (int i = 1; i <= 10; i++)
%		{
%			System.out.println(i); 
%		}               
%	}
%}
%\end{lstlisting}
%
%\begin{lstlisting}
%  public static void main(java.lang.String[]);
%    flags: ACC_PUBLIC, ACC_STATIC
%    Code:
%      stack=2, locals=2, args_size=1
%         0: iconst_1      
%         1: istore_1      
%         2: iload_1       
%         3: bipush        10
%         5: if_icmpgt     21
%         8: getstatic     #2         // Field java/lang/System.out:Ljava/io/PrintStream;
%        11: iload_1       
%        12: invokevirtual #3         // Method java/io/PrintStream.println:(I)V
%        15: iinc          1, 1
%        18: goto          2
%        21: return        
%\end{lstlisting}
%
%\TT{iconst\_1} loads 1 into \ac{TOS}, \TT{istore\_1} stores it in the \ac{LVA} at slot 1.
%
%Why not the zeroth slot?
%Because the \main function has one argument (array of \TT{String}) 
%and a pointer to it (or \IT{reference}) is now in the zeroth slot.
%
%
%So, the \IT{i} local variable will always be in 1st slot.
%
%
%Instructions at offsets 3 and 5 compare \IT{i} with 10.
%
%If \IT{i} is larger, execution flow passes to offset 21, where the function ends.
%
%If it's not, \TT{println} is called.
%
%\IT{i} is then reloaded at offset 11, for \TT{println}.
%
%By the way, we call the \TT{println} method for an \IT{integer}, 
%and we see this in the comments: \q{(I)V}
%(\IT{I} means \IT{integer} and \IT{V} means the return type is \IT{void}).
%
%
%When \TT{println} finishes, \IT{i} is incremented at offset 15.
%
%The first operand of the instruction is the  number of a slot (1), 
%the second is the number (1) to add to the variable.
%
%
%\TT{goto} is just GOTO, it jumps to the beginning of the loop's body offset 2.
%
%
%Let's proceed with a more complex example:
%
%
%\begin{lstlisting}[style=customjava]
%public class Fibonacci
%{
%	public static void main(String[] args)
%	{ 
%		int limit = 20, f = 0, g = 1;
%
%		for (int i = 1; i <= limit; i++)
%		{
%			f = f + g;
%			g = f - g;
%			System.out.println(f); 
%		}
%	}
%}
%\end{lstlisting}
%
%\begin{lstlisting}
%  public static void main(java.lang.String[]);
%    flags: ACC_PUBLIC, ACC_STATIC
%    Code:
%      stack=2, locals=5, args_size=1
%         0: bipush        20
%         2: istore_1      
%         3: iconst_0      
%         4: istore_2      
%         5: iconst_1      
%         6: istore_3      
%         7: iconst_1      
%         8: istore        4
%        10: iload         4
%        12: iload_1       
%        13: if_icmpgt     37
%        16: iload_2       
%        17: iload_3       
%        18: iadd          
%        19: istore_2      
%        20: iload_2       
%        21: iload_3       
%        22: isub          
%        23: istore_3      
%        24: getstatic     #2         // Field java/lang/System.out:Ljava/io/PrintStream;
%        27: iload_2       
%        28: invokevirtual #3         // Method java/io/PrintStream.println:(I)V
%        31: iinc          4, 1
%        34: goto          10
%        37: return        
%\end{lstlisting}
%        
%Here is a map of the \ac{LVA} slots:
%
%
%\begin{itemize}
%\item 0 --- the sole argument of \main
%\item 1 --- \IT{limit}, always contains 20
%\item 2 --- \IT{f}
%\item 3 --- \IT{g}
%\item 4 --- \IT{i}
%\end{itemize}
%
%We can see that the Java compiler allocates variables in \ac{LVA} slots in the same order 
%they were declared in the source code.
%
%
%There are separate \TT{istore} instructions for accessing slots 0, 1, 2 and 3, 
%but not for 4 and larger, so there is \TT{istore} with an additional operand at offset 8 
%which takes the slot number as an operand.
%
%It's the same with \TT{iload} at offset 10.
%
%
%But isn't it dubious to allocate another slot for the \IT{limit} variable, which 
%always contains 20 (so it's a constant in essence), and reload its value so often?
%
%\ac{JVM} \ac{JIT} compiler is usually good enough to optimize such things.
%
%Manual intervention in the code is probably not worth it.
%
%
}

\EN{% TODO proof-reading
\subsection{switch()}

The switch() statement is implemented with the \TT{tableswitch} instruction:


\begin{lstlisting}[style=customjava]
	public static void f(int a)
	{
		switch (a)
		{
		case 0: System.out.println("zero"); break;
		case 1: System.out.println("one\n"); break;
		case 2: System.out.println("two\n"); break;
		case 3: System.out.println("three\n"); break;
		case 4: System.out.println("four\n"); break;
		default: System.out.println("something unknown\n"); break;
		};
	}
\end{lstlisting}

As simple, as possible:

\begin{lstlisting}
  public static void f(int);
    flags: ACC_PUBLIC, ACC_STATIC
    Code:
      stack=2, locals=1, args_size=1
         0: iload_0       
         1: tableswitch   { // 0 to 4
                       0: 36
                       1: 47
                       2: 58
                       3: 69
                       4: 80
                 default: 91
            }
        36: getstatic     #2     // Field java/lang/System.out:Ljava/io/PrintStream;
        39: ldc           #3     // String zero
        41: invokevirtual #4     // Method java/io/PrintStream.println:(Ljava/lang/String;)V
        44: goto          99
        47: getstatic     #2     // Field java/lang/System.out:Ljava/io/PrintStream;
        50: ldc           #5     // String one\n
        52: invokevirtual #4     // Method java/io/PrintStream.println:(Ljava/lang/String;)V
        55: goto          99
        58: getstatic     #2     // Field java/lang/System.out:Ljava/io/PrintStream;
        61: ldc           #6     // String two\n
        63: invokevirtual #4     // Method java/io/PrintStream.println:(Ljava/lang/String;)V
        66: goto          99
        69: getstatic     #2     // Field java/lang/System.out:Ljava/io/PrintStream;
        72: ldc           #7     // String three\n
        74: invokevirtual #4     // Method java/io/PrintStream.println:(Ljava/lang/String;)V
        77: goto          99
        80: getstatic     #2     // Field java/lang/System.out:Ljava/io/PrintStream;
        83: ldc           #8     // String four\n
        85: invokevirtual #4     // Method java/io/PrintStream.println:(Ljava/lang/String;)V
        88: goto          99
        91: getstatic     #2     // Field java/lang/System.out:Ljava/io/PrintStream;
        94: ldc           #9     // String something unknown\n
        96: invokevirtual #4     // Method java/io/PrintStream.println:(Ljava/lang/String;)V
        99: return        
\end{lstlisting}
}
\RU{% TODO proof-reading
\subsection{switch()}


Выражение switch() реализуется инструкцией \TT{tableswitch}:

\begin{lstlisting}[style=customjava]
	public static void f(int a)
	{
		switch (a)
		{
		case 0: System.out.println("zero"); break;
		case 1: System.out.println("one\n"); break;
		case 2: System.out.println("two\n"); break;
		case 3: System.out.println("three\n"); break;
		case 4: System.out.println("four\n"); break;
		default: System.out.println("something unknown\n"); break;
		};
	}
\end{lstlisting}

Проще не бывает:

\begin{lstlisting}
  public static void f(int);
    flags: ACC_PUBLIC, ACC_STATIC
    Code:
      stack=2, locals=1, args_size=1
         0: iload_0       
         1: tableswitch   { // 0 to 4
                       0: 36
                       1: 47
                       2: 58
                       3: 69
                       4: 80
                 default: 91
            }
        36: getstatic     #2     // Field java/lang/System.out:Ljava/io/PrintStream;
        39: ldc           #3     // String zero
        41: invokevirtual #4     // Method java/io/PrintStream.println:(Ljava/lang/String;)V
        44: goto          99
        47: getstatic     #2     // Field java/lang/System.out:Ljava/io/PrintStream;
        50: ldc           #5     // String one\n
        52: invokevirtual #4     // Method java/io/PrintStream.println:(Ljava/lang/String;)V
        55: goto          99
        58: getstatic     #2     // Field java/lang/System.out:Ljava/io/PrintStream;
        61: ldc           #6     // String two\n
        63: invokevirtual #4     // Method java/io/PrintStream.println:(Ljava/lang/String;)V
        66: goto          99
        69: getstatic     #2     // Field java/lang/System.out:Ljava/io/PrintStream;
        72: ldc           #7     // String three\n
        74: invokevirtual #4     // Method java/io/PrintStream.println:(Ljava/lang/String;)V
        77: goto          99
        80: getstatic     #2     // Field java/lang/System.out:Ljava/io/PrintStream;
        83: ldc           #8     // String four\n
        85: invokevirtual #4     // Method java/io/PrintStream.println:(Ljava/lang/String;)V
        88: goto          99
        91: getstatic     #2     // Field java/lang/System.out:Ljava/io/PrintStream;
        94: ldc           #9     // String something unknown\n
        96: invokevirtual #4     // Method java/io/PrintStream.println:(Ljava/lang/String;)V
        99: return        
\end{lstlisting}
}
\DE{% TODO proof-reading
\subsection{switch()}

%The switch() statement is implemented with the \TT{tableswitch} instruction:
%
%
%\begin{lstlisting}[style=customjava]
%	public static void f(int a)
%	{
%		switch (a)
%		{
%		case 0: System.out.println("zero"); break;
%		case 1: System.out.println("one\n"); break;
%		case 2: System.out.println("two\n"); break;
%		case 3: System.out.println("three\n"); break;
%		case 4: System.out.println("four\n"); break;
%		default: System.out.println("something unknown\n"); break;
%		};
%	}
%\end{lstlisting}
%
%As simple, as possible:
%
%\begin{lstlisting}
%  public static void f(int);
%    flags: ACC_PUBLIC, ACC_STATIC
%    Code:
%      stack=2, locals=1, args_size=1
%         0: iload_0       
%         1: tableswitch   { // 0 to 4
%                       0: 36
%                       1: 47
%                       2: 58
%                       3: 69
%                       4: 80
%                 default: 91
%            }
%        36: getstatic     #2     // Field java/lang/System.out:Ljava/io/PrintStream;
%        39: ldc           #3     // String zero
%        41: invokevirtual #4     // Method java/io/PrintStream.println:(Ljava/lang/String;)V
%        44: goto          99
%        47: getstatic     #2     // Field java/lang/System.out:Ljava/io/PrintStream;
%        50: ldc           #5     // String one\n
%        52: invokevirtual #4     // Method java/io/PrintStream.println:(Ljava/lang/String;)V
%        55: goto          99
%        58: getstatic     #2     // Field java/lang/System.out:Ljava/io/PrintStream;
%        61: ldc           #6     // String two\n
%        63: invokevirtual #4     // Method java/io/PrintStream.println:(Ljava/lang/String;)V
%        66: goto          99
%        69: getstatic     #2     // Field java/lang/System.out:Ljava/io/PrintStream;
%        72: ldc           #7     // String three\n
%        74: invokevirtual #4     // Method java/io/PrintStream.println:(Ljava/lang/String;)V
%        77: goto          99
%        80: getstatic     #2     // Field java/lang/System.out:Ljava/io/PrintStream;
%        83: ldc           #8     // String four\n
%        85: invokevirtual #4     // Method java/io/PrintStream.println:(Ljava/lang/String;)V
%        88: goto          99
%        91: getstatic     #2     // Field java/lang/System.out:Ljava/io/PrintStream;
%        94: ldc           #9     // String something unknown\n
%        96: invokevirtual #4     // Method java/io/PrintStream.println:(Ljava/lang/String;)V
%        99: return        
%\end{lstlisting}
}

% TODO proof-reading
\subsection{\EN{Arrays}\RU{Массивы}\DE{Arrays}}

\EN{% TODO proof-reading
\subsubsection{Simple example}

Let's first create an array of 10 integers and fill it:

\begin{lstlisting}[style=customjava]
	public static void main(String[] args) 
	{
		int a[]=new int[10];
		for (int i=0; i<10; i++)
			a[i]=i;
		dump (a);
	}
\end{lstlisting}

\begin{lstlisting}
  public static void main(java.lang.String[]);
    flags: ACC_PUBLIC, ACC_STATIC
    Code:
      stack=3, locals=3, args_size=1
         0: bipush        10
         2: newarray       int
         4: astore_1      
         5: iconst_0      
         6: istore_2      
         7: iload_2       
         8: bipush        10
        10: if_icmpge     23
        13: aload_1       
        14: iload_2       
        15: iload_2       
        16: iastore       
        17: iinc          2, 1
        20: goto          7
        23: aload_1       
        24: invokestatic  #4     // Method dump:([I)V
        27: return        
\end{lstlisting}

The \TT{newarray} instruction creates an array object of 10 \IT{int} elements.

The array's size is set with \TT{bipush} and left at \ac{TOS}.

The array's type is set in \TT{newarray} instruction's operand.

After \TT{newarray}'s execution, a \IT{reference} (or pointer) to the newly created array in the heap 
is left at the \ac{TOS}.

\TT{astore\_1} stores the \IT{reference} to the 1st slot in \ac{LVA}.

The second part of the \main function is the loop which stores \IT{i} into the corresponding
array element.

\TT{aload\_1} gets a \IT{reference} of the array and places it in the stack.

\TT{iastore} then stores the integer value from the stack in the array, 
\IT{reference} of which is currently in \ac{TOS}.

The third part of the \main function calls the \TT{dump()} function.

An argument for it is prepared by \TT{aload\_1} (offset 23).

Now let's proceed to the \TT{dump()} function:

\begin{lstlisting}[style=customjava]
	public static void dump(int a[])
	{
		for (int i=0; i<a.length; i++)
			System.out.println(a[i]);
	}
\end{lstlisting}

\begin{lstlisting}
  public static void dump(int[]);
    flags: ACC_PUBLIC, ACC_STATIC
    Code:
      stack=3, locals=2, args_size=1
         0: iconst_0      
         1: istore_1      
         2: iload_1       
         3: aload_0       
         4: arraylength   
         5: if_icmpge     23
         8: getstatic     #2      // Field java/lang/System.out:Ljava/io/PrintStream;
        11: aload_0       
        12: iload_1       
        13: iaload        
        14: invokevirtual #3      // Method java/io/PrintStream.println:(I)V
        17: iinc          1, 1
        20: goto          2
        23: return        
\end{lstlisting}

The incoming \TT{reference} to the array is in the zeroth slot.

The \TT{a.length} expression in the source code is converted to an \TT{arraylength} instruction: 
it takes a \IT{reference} to the array and leaves the array size at \ac{TOS}.

\TT{iaload} at offset 13 is used to load array elements, 
it requires to array \IT{reference} be present
in the stack (prepared by \TT{aload\_0} at 11), 
and also an index (prepared by \TT{iload\_1} at offset 12).

Needless to say, instructions prefixed with \IT{a} may be mistakenly comprehended 
as \IT{array} instructions.

It's not correct.
These instructions works with \IT{references} to objects.

And arrays and strings are objects too.

}\RU{% TODO proof-reading
\subsubsection{Простой пример}

Создадим массив из 10-и чисел и заполним его:

\begin{lstlisting}[style=customjava]
	public static void main(String[] args) 
	{
		int a[]=new int[10];
		for (int i=0; i<10; i++)
			a[i]=i;
		dump (a);
	}
\end{lstlisting}

\begin{lstlisting}
  public static void main(java.lang.String[]);
    flags: ACC_PUBLIC, ACC_STATIC
    Code:
      stack=3, locals=3, args_size=1
         0: bipush        10
         2: newarray       int
         4: astore_1      
         5: iconst_0      
         6: istore_2      
         7: iload_2       
         8: bipush        10
        10: if_icmpge     23
        13: aload_1       
        14: iload_2       
        15: iload_2       
        16: iastore       
        17: iinc          2, 1
        20: goto          7
        23: aload_1       
        24: invokestatic  #4     // Method dump:([I)V
        27: return        
\end{lstlisting}

Инструкция \TT{newarray} создает объект массива из 10 элементов типа \IT{int}.

Размер массива выставляется инструкцией \TT{bipush} и остается на \ac{TOS}.

Тип массива выставляется в операнде инструкции \TT{newarray}.

После исполнения \TT{newarray}, \IT{reference} (или указатель) только что созданного 
в куче (heap) массива остается на \ac{TOS}.

\TT{astore\_1} сохраняет \IT{reference} на него в первом слоте \ac{LVA}.

Вторая часть функции \main это цикл, сохраняющий значение \IT{i} в соответствующий
элемент массива.

\TT{aload\_1} берет \IT{reference} массива и сохраняет его в стеке.

\TT{iastore} затем сохраняет значение из стека в массив, 
\IT{reference} на который в это время находится на \ac{TOS}.

Третья часть функции \main вызывает функцию \TT{dump()}.

Аргумент для нее готовится инструкцией \TT{aload\_1} (смещение 23).

Перейдем к функции \TT{dump()}:

\begin{lstlisting}[style=customjava]
	public static void dump(int a[])
	{
		for (int i=0; i<a.length; i++)
			System.out.println(a[i]);
	}
\end{lstlisting}

\begin{lstlisting}
  public static void dump(int[]);
    flags: ACC_PUBLIC, ACC_STATIC
    Code:
      stack=3, locals=2, args_size=1
         0: iconst_0      
         1: istore_1      
         2: iload_1       
         3: aload_0       
         4: arraylength   
         5: if_icmpge     23
         8: getstatic     #2      // Field java/lang/System.out:Ljava/io/PrintStream;
        11: aload_0       
        12: iload_1       
        13: iaload        
        14: invokevirtual #3      // Method java/io/PrintStream.println:(I)V
        17: iinc          1, 1
        20: goto          2
        23: return        
\end{lstlisting}

Входящий \TT{reference} на массив в нулевом слоте.

Выражение \TT{a.length} в исходном коде конвертируется в инструкцию \TT{arraylength},
она берет \IT{reference} на массив и оставляет размер массива на \ac{TOS}.

Инструкция \TT{iaload} по смещеню 13 используется для загрузки элементов массива, 
она требует, чтобы в стеке присутствовал \IT{reference} на массив
(подготовленный \TT{aload\_0} на 11), 
а также индекс (подготовленный \TT{iload\_1} по смещеню 12).

Нужно сказать, что инструкции с префиксом \IT{a} могут быть неверно поняты, 
как инструкции работающие с массивами (\IT{array}).
Это неверно.

Эти инструкции работают с \IT{reference}-ами на объекты.

А массивы и строки это тоже объекты.
}

\EN{% TODO proof-reading
\subsubsection{Summing elements of array}

Another example:

\begin{lstlisting}[style=customjava]
public class ArraySum
{
	public static int f (int[] a)
	{
		int sum=0;
		for (int i=0; i<a.length; i++)
			sum=sum+a[i];
		return sum;
	}
}
\end{lstlisting}

\begin{lstlisting}
  public static int f(int[]);
    flags: ACC_PUBLIC, ACC_STATIC
    Code:
      stack=3, locals=3, args_size=1
         0: iconst_0      
         1: istore_1      
         2: iconst_0      
         3: istore_2      
         4: iload_2       
         5: aload_0       
         6: arraylength   
         7: if_icmpge     22
        10: iload_1       
        11: aload_0       
        12: iload_2       
        13: iaload        
        14: iadd          
        15: istore_1      
        16: iinc          2, 1
        19: goto          4
        22: iload_1       
        23: ireturn       
\end{lstlisting}

\ac{LVA} slot 0 contains a \IT{reference} to the input array.

\ac{LVA} slot 1 contains the local variable \IT{sum}.

}\RU{% TODO proof-reading
\subsubsection{Суммирование элементов массива}

Еще один пример:

\begin{lstlisting}[style=customjava]
public class ArraySum
{
	public static int f (int[] a)
	{
		int sum=0;
		for (int i=0; i<a.length; i++)
			sum=sum+a[i];
		return sum;
	}
}
\end{lstlisting}

\begin{lstlisting}
  public static int f(int[]);
    flags: ACC_PUBLIC, ACC_STATIC
    Code:
      stack=3, locals=3, args_size=1
         0: iconst_0      
         1: istore_1      
         2: iconst_0      
         3: istore_2      
         4: iload_2       
         5: aload_0       
         6: arraylength   
         7: if_icmpge     22
        10: iload_1       
        11: aload_0       
        12: iload_2       
        13: iaload        
        14: iadd          
        15: istore_1      
        16: iinc          2, 1
        19: goto          4
        22: iload_1       
        23: ireturn       
\end{lstlisting}

Нулевой слот в \ac{LVA} содержит указатель (\IT{reference}) на входной массив.

Первый слот \ac{LVA} содержит локальную переменную \IT{sum}.
}

\EN{% TODO proof-reading
\subsubsection{The only argument of the \main function is an array too}

We'll be using the only argument of the \main function, which is an array of strings:


\begin{lstlisting}[style=customjava]
public class UseArgument
{
	public static void main(String[] args)
	{
		System.out.print("Hi, ");
		System.out.print(args[1]);
		System.out.println(". How are you?");
	}
}
\end{lstlisting}

The zeroth argument is the program's name (like in \CCpp, etc.), 
so the 1st argument supplied by the user is 1st.


\begin{lstlisting}
  public static void main(java.lang.String[]);
    flags: ACC_PUBLIC, ACC_STATIC
    Code:
      stack=3, locals=1, args_size=1
         0: getstatic     #2      // Field java/lang/System.out:Ljava/io/PrintStream;
         3: ldc           #3      // String Hi, 
         5: invokevirtual #4      // Method java/io/PrintStream.print:(Ljava/lang/String;)V
         8: getstatic     #2      // Field java/lang/System.out:Ljava/io/PrintStream;
        11: aload_0       
        12: iconst_1      
        13: aaload        
        14: invokevirtual #4      // Method java/io/PrintStream.print:(Ljava/lang/String;)V
        17: getstatic     #2      // Field java/lang/System.out:Ljava/io/PrintStream;
        20: ldc           #5      // String . How are you?
        22: invokevirtual #6      // Method java/io/PrintStream.println:(Ljava/lang/String;)V
        25: return        
\end{lstlisting}

\TT{aload\_0} at 11 loads a \IT{reference} of the zeroth \ac{LVA} slot 
(1st and only \main argument).

\TT{iconst\_1} and \TT{aaload} at 12 and 13 take a \IT{reference} to the first (counting at 0) 
element of array.

The \IT{reference} to the string object is at \ac{TOS} at offset 14, and it is 
taken from there by \TT{println} method.

}\RU{% TODO proof-reading
\subsubsection{Единственный аргумент \main это также массив}


Будем использовать единственный аргумент \main, который массив строк:

\begin{lstlisting}[style=customjava]
public class UseArgument
{
	public static void main(String[] args)
	{
		System.out.print("Hi, ");
		System.out.print(args[1]);
		System.out.println(". How are you?");
	}
}
\end{lstlisting}


Нулевой аргумент это имя программы (как в \CCpp, итд),
так что первый аргумент это тот, что пользователь добавил первым.

\begin{lstlisting}
  public static void main(java.lang.String[]);
    flags: ACC_PUBLIC, ACC_STATIC
    Code:
      stack=3, locals=1, args_size=1
         0: getstatic     #2      // Field java/lang/System.out:Ljava/io/PrintStream;
         3: ldc           #3      // String Hi, 
         5: invokevirtual #4      // Method java/io/PrintStream.print:(Ljava/lang/String;)V
         8: getstatic     #2      // Field java/lang/System.out:Ljava/io/PrintStream;
        11: aload_0       
        12: iconst_1      
        13: aaload        
        14: invokevirtual #4      // Method java/io/PrintStream.print:(Ljava/lang/String;)V
        17: getstatic     #2      // Field java/lang/System.out:Ljava/io/PrintStream;
        20: ldc           #5      // String . How are you?
        22: invokevirtual #6      // Method java/io/PrintStream.println:(Ljava/lang/String;)V
        25: return        
\end{lstlisting}


\TT{aload\_0} на 11 загружают \IT{reference} на нулевой слот \ac{LVA} 
(первый и единственный аргумент \main).

\TT{iconst\_1} и \TT{aaload} на 12 и 13 берут \IT{reference} на первый (считая с 0) 
элемент массива.

\IT{Reference} на строковый объект на \ac{TOS} по смещению 14, и оттуда он 
берется методом \TT{println}.
}

\EN{% TODO proof-reading
\subsubsection{Pre-initialized array of strings}
\label{Java_2D_array_month}

\begin{lstlisting}[style=customjava]
class Month
{
	public static String[] months = 
	{
		"January", 
		"February", 
		"March", 
		"April",
		"May",
		"June",
		"July",
		"August",
		"September",
		"October",
		"November",
		"December"
	};

	public String get_month (int i)
	{
		return months[i];
	};
} 
\end{lstlisting}

The \TT{get\_month()} function is simple:
Функция \TT{get\_month()} проста:

\begin{lstlisting}
  public java.lang.String get_month(int);
    flags: ACC_PUBLIC
    Code:
      stack=2, locals=2, args_size=2
         0: getstatic     #2         // Field months:[Ljava/lang/String;
         3: iload_1       
         4: aaload        
         5: areturn       
\end{lstlisting}

\TT{aaload} operates on an array of \IT{references}.

Java String are objects, so the \IT{a}-instructions are used to operate on them.

\TT{areturn} returns a \IT{reference} to a \TT{String} object.


How is the \TT{months[]} array initialized?


\begin{lstlisting}
  static {};
    flags: ACC_STATIC
    Code:
      stack=4, locals=0, args_size=0
         0: bipush        12
         2: anewarray     #3         // class java/lang/String
         5: dup           
         6: iconst_0      
         7: ldc           #4         // String January
         9: aastore       
        10: dup           
        11: iconst_1      
        12: ldc           #5         // String February
        14: aastore       
        15: dup           
        16: iconst_2      
        17: ldc           #6         // String March
        19: aastore       
        20: dup           
        21: iconst_3      
        22: ldc           #7         // String April
        24: aastore       
        25: dup           
        26: iconst_4      
        27: ldc           #8         // String May
        29: aastore       
        30: dup           
        31: iconst_5      
        32: ldc           #9         // String June
        34: aastore       
        35: dup           
        36: bipush        6
        38: ldc           #10        // String July
        40: aastore       
        41: dup           
        42: bipush        7
        44: ldc           #11        // String August
        46: aastore       
        47: dup           
        48: bipush        8
        50: ldc           #12        // String September
        52: aastore       
        53: dup           
        54: bipush        9
        56: ldc           #13        // String October
        58: aastore       
        59: dup           
        60: bipush        10
        62: ldc           #14        // String November
        64: aastore       
        65: dup           
        66: bipush        11
        68: ldc           #15        // String December
        70: aastore       
        71: putstatic     #2         // Field months:[Ljava/lang/String;
        74: return        
\end{lstlisting}

\TT{anewarray} creates a new array of \IT{references} (hence \IT{a} prefix).

The object's type is defined in the \TT{anewarray}'s operand, it is the \\
\q{java/lang/String} string.

The \TT{bipush 12} before \TT{anewarray} sets the array's size.

We see here a new instruction for us: \TT{dup}.


\myindex{Forth}
It's a standard instruction in stack computers (including the Forth programming language) 
which just duplicates the value at \ac{TOS}.

\myindex{x86!\Instructions!FDUP}
By the way, FPU 80x87 is also a stack computer and it has similar instruction -- \INS{FDUP}.


It is used here to duplicate a \IT{reference} to an array, because the \TT{aastore} instruction pops
the \IT{reference} to array from the stack, but subsequent \TT{aastore} will need it again.

The Java compiler concluded that it's better to generate a \TT{dup} instead of generating 
a \TT{getstatic} instruction before each array store operation (i.e., 11 times).


\TT{aastore} puts a \IT{reference} (to string) into the array at an index which is 
taken from \ac{TOS}.


Finally, \TT{putstatic} puts \IT{reference} to the newly created array into the second field 
of our object, i.e., \IT{months} field.

}\RU{% TODO proof-reading
\subsubsection{Заранее инициализированный массив строк}
\label{Java_2D_array_month}

\begin{lstlisting}[style=customjava]
class Month
{
	public static String[] months = 
	{
		"January", 
		"February", 
		"March", 
		"April",
		"May",
		"June",
		"July",
		"August",
		"September",
		"October",
		"November",
		"December"
	};

	public String get_month (int i)
	{
		return months[i];
	};
} 
\end{lstlisting}




\begin{lstlisting}
  public java.lang.String get_month(int);
    flags: ACC_PUBLIC
    Code:
      stack=2, locals=2, args_size=2
         0: getstatic     #2         // Field months:[Ljava/lang/String;
         3: iload_1       
         4: aaload        
         5: areturn       
\end{lstlisting}


\TT{aaload} работает с массивом \IT{reference}-ов.

Строка в Java это объект, так что используются \IT{a}-инструкции для работы с ними.

\TT{areturn} возвращает \IT{reference} на объект \TT{String}.


Как инициализируется массив \TT{months[]}?

\begin{lstlisting}
  static {};
    flags: ACC_STATIC
    Code:
      stack=4, locals=0, args_size=0
         0: bipush        12
         2: anewarray     #3         // class java/lang/String
         5: dup           
         6: iconst_0      
         7: ldc           #4         // String January
         9: aastore       
        10: dup           
        11: iconst_1      
        12: ldc           #5         // String February
        14: aastore       
        15: dup           
        16: iconst_2      
        17: ldc           #6         // String March
        19: aastore       
        20: dup           
        21: iconst_3      
        22: ldc           #7         // String April
        24: aastore       
        25: dup           
        26: iconst_4      
        27: ldc           #8         // String May
        29: aastore       
        30: dup           
        31: iconst_5      
        32: ldc           #9         // String June
        34: aastore       
        35: dup           
        36: bipush        6
        38: ldc           #10        // String July
        40: aastore       
        41: dup           
        42: bipush        7
        44: ldc           #11        // String August
        46: aastore       
        47: dup           
        48: bipush        8
        50: ldc           #12        // String September
        52: aastore       
        53: dup           
        54: bipush        9
        56: ldc           #13        // String October
        58: aastore       
        59: dup           
        60: bipush        10
        62: ldc           #14        // String November
        64: aastore       
        65: dup           
        66: bipush        11
        68: ldc           #15        // String December
        70: aastore       
        71: putstatic     #2         // Field months:[Ljava/lang/String;
        74: return        
\end{lstlisting}


\TT{anewarray} создает новый массив \IT{reference}-ов (отсюда префикс \IT{a}).

Тип объекта определяется в операнде \TT{anewarray}, там текстовая строка \\
\q{java/lang/String}.

\TT{bipush 12} перед \TT{anewarray} устанавливает размер массива.

Новая для нас здесь инструкция: \TT{dup}.

\myindex{Forth}

Это стандартная инструкция в стековых компьютерах (включая ЯП Forth),
которая делает дубликат значения на \ac{TOS}.
\myindex{x86!\Instructions!FDUP}

Кстати, FPU 80x87 это тоже стековый компьютер, и в нем есть аналогичная инструкция -- \INS{FDUP}.


Она используется здесь для дублирования \IT{reference}-а на массив, 
потому что инструкция \TT{aastore} выталкивает из стека \IT{reference} на массив, 
но последующая инструкция \TT{aastore} снова нуждается в нем.

Компилятор Java решил, что лучше генерировать \TT{dup} вместо генерации инструкции
\TT{getstatic} перед каждой операцией записи в массив (т.е. 11 раз).


\TT{aastore} кладет \IT{reference} (на строку) в массив по индексу взятому из \ac{TOS}.


И наконец, \TT{putstatic} кладет \IT{reference} на только что созданный массив во второе поле
нашего объекта, т.е. в поле \IT{months}.
}

\EN{% TODO proof-reading
\subsubsection{Variadic functions}

Variadic functions actually use arrays:


\begin{lstlisting}[style=customjava]
	public static void f(int... values)
	{
		for (int i=0; i<values.length; i++)
			System.out.println(values[i]);
	}

	public static void main(String[] args) 
	{
		f (1,2,3,4,5);
	}
\end{lstlisting}

\begin{lstlisting}
  public static void f(int...);
    flags: ACC_PUBLIC, ACC_STATIC, ACC_VARARGS
    Code:
      stack=3, locals=2, args_size=1
         0: iconst_0      
         1: istore_1      
         2: iload_1       
         3: aload_0       
         4: arraylength   
         5: if_icmpge     23
         8: getstatic     #2       // Field java/lang/System.out:Ljava/io/PrintStream;
        11: aload_0       
        12: iload_1       
        13: iaload        
        14: invokevirtual #3       // Method java/io/PrintStream.println:(I)V
        17: iinc          1, 1
        20: goto          2
        23: return        
\end{lstlisting}

\ttf just takes an array of integers using \TT{aload\_0} at offset 3.

Then it gets the array's size, etc.


\begin{lstlisting}
  public static void main(java.lang.String[]);
    flags: ACC_PUBLIC, ACC_STATIC
    Code:
      stack=4, locals=1, args_size=1
         0: iconst_5      
         1: newarray       int
         3: dup           
         4: iconst_0      
         5: iconst_1      
         6: iastore       
         7: dup           
         8: iconst_1      
         9: iconst_2      
        10: iastore       
        11: dup           
        12: iconst_2      
        13: iconst_3      
        14: iastore       
        15: dup           
        16: iconst_3      
        17: iconst_4      
        18: iastore       
        19: dup           
        20: iconst_4      
        21: iconst_5      
        22: iastore       
        23: invokestatic  #4       // Method f:([I)V
        26: return        
\end{lstlisting}

The array is constructed in \main using the \TT{newarray} instruction, 
then it's filled, and \ttf is called.


Oh, by the way, array object is not destroyed at the end of \main.

\myindex{Garbage collector}
There are no destructors in Java at all, because the JVM has a garbage collector which does this
automatically, when it feels it needs to.


What about the \TT{format()} method?

It takes two arguments at input: a string and an array of objects:


\begin{lstlisting}
	public PrintStream format(String format, Object... args)
\end{lstlisting}
( \url{http://docs.oracle.com/javase/tutorial/java/data/numberformat.html} )

Let's see:

\begin{lstlisting}[style=customjava]
	public static void main(String[] args)
	{
		int i=123;
		double d=123.456;
		System.out.format("int: %d double: %f.%n", i, d);
	}
\end{lstlisting}

\begin{lstlisting}
  public static void main(java.lang.String[]);
    flags: ACC_PUBLIC, ACC_STATIC
    Code:
      stack=7, locals=4, args_size=1
         0: bipush        123
         2: istore_1      
         3: ldc2_w        #2         // double 123.456d
         6: dstore_2      
         7: getstatic     #4         // Field java/lang/System.out:Ljava/io/PrintStream;
        10: ldc           #5         // String int: %d double: %f.%n
        12: iconst_2      
        13: anewarray     #6         // class java/lang/Object
        16: dup           
        17: iconst_0      
        18: iload_1       
        19: invokestatic  #7         // Method java/lang/Integer.valueOf:(I)Ljava/lang/Integer;
        22: aastore       
        23: dup           
        24: iconst_1      
        25: dload_2       
        26: invokestatic  #8         // Method java/lang/Double.valueOf:(D)Ljava/lang/Double;
        29: aastore       
        30: invokevirtual #9         // Method java/io/PrintStream.format:(Ljava/lang/String;[Ljava/lang/Object;)Ljava/io/PrintStream;
        33: pop           
        34: return        
\end{lstlisting}

So values of the \IT{int} and \IT{double} types are first promoted to \TT{Integer} and \TT{Double} 
objects using the \TT{valueOf} methods.

The \TT{format()} method needs objects of type \TT{Object} at input, and since the \TT{Integer} and 
\TT{Double} classes are derived from the root \TT{Object} class, they suitable for elements 
in the input array.

On the other hand, an array is always homogeneous, i.e., it can't hold elements of 
different types, which makes it impossible to push \IT{int} and \IT{double} values in it.


An array of \TT{Object} objects is created at offset 13, 
an \TT{Integer} object is added to the array at offset 22, 
and a \TT{Double} object is added to the array at offset 29.


The penultimate \TT{pop} instruction discards the element at \ac{TOS}, 
so when \TT{return} is executed, the stack becomes empty (or balanced).

}\RU{% TODO proof-reading
\subsubsection{Функции с переменным кол-вом аргументов (variadic)}


Функции с переменным кол-вом аргументов (variadic) на самом деле используют массивы:

\begin{lstlisting}[style=customjava]
	public static void f(int... values)
	{
		for (int i=0; i<values.length; i++)
			System.out.println(values[i]);
	}

	public static void main(String[] args) 
	{
		f (1,2,3,4,5);
	}
\end{lstlisting}

\begin{lstlisting}
  public static void f(int...);
    flags: ACC_PUBLIC, ACC_STATIC, ACC_VARARGS
    Code:
      stack=3, locals=2, args_size=1
         0: iconst_0      
         1: istore_1      
         2: iload_1       
         3: aload_0       
         4: arraylength   
         5: if_icmpge     23
         8: getstatic     #2       // Field java/lang/System.out:Ljava/io/PrintStream;
        11: aload_0       
        12: iload_1       
        13: iaload        
        14: invokevirtual #3       // Method java/io/PrintStream.println:(I)V
        17: iinc          1, 1
        20: goto          2
        23: return        
\end{lstlisting}


По смещению 3, \ttf просто берет массив переменных используя \TT{aload\_0}.

Затем берет размер массива, итд.

\begin{lstlisting}
  public static void main(java.lang.String[]);
    flags: ACC_PUBLIC, ACC_STATIC
    Code:
      stack=4, locals=1, args_size=1
         0: iconst_5      
         1: newarray       int
         3: dup           
         4: iconst_0      
         5: iconst_1      
         6: iastore       
         7: dup           
         8: iconst_1      
         9: iconst_2      
        10: iastore       
        11: dup           
        12: iconst_2      
        13: iconst_3      
        14: iastore       
        15: dup           
        16: iconst_3      
        17: iconst_4      
        18: iastore       
        19: dup           
        20: iconst_4      
        21: iconst_5      
        22: iastore       
        23: invokestatic  #4       // Method f:([I)V
        26: return        
\end{lstlisting}


Массив конструируется в \main используя инструкцию \TT{newarray}, 
затем он заполняется, и вызывается \ttf.


Кстати, объект массива не уничтожается в конце \main.
\myindex{Сборщик мусора}

В Java вообще нет деструкторов, потому что в JVM есть сборщик мусора (garbage collector),
делающий это автоматически, когда считает нужным.


Как насчет метода \TT{format()}?

Он берет на вход два аргумента: строку и массив объектов:

\begin{lstlisting}
	public PrintStream format(String format, Object... args)
\end{lstlisting}
( \url{http://docs.oracle.com/javase/tutorial/java/data/numberformat.html} )

Посмотрим:

\begin{lstlisting}[style=customjava]
	public static void main(String[] args)
	{
		int i=123;
		double d=123.456;
		System.out.format("int: %d double: %f.%n", i, d);
	}
\end{lstlisting}

\begin{lstlisting}
  public static void main(java.lang.String[]);
    flags: ACC_PUBLIC, ACC_STATIC
    Code:
      stack=7, locals=4, args_size=1
         0: bipush        123
         2: istore_1      
         3: ldc2_w        #2         // double 123.456d
         6: dstore_2      
         7: getstatic     #4         // Field java/lang/System.out:Ljava/io/PrintStream;
        10: ldc           #5         // String int: %d double: %f.%n
        12: iconst_2      
        13: anewarray     #6         // class java/lang/Object
        16: dup           
        17: iconst_0      
        18: iload_1       
        19: invokestatic  #7         // Method java/lang/Integer.valueOf:(I)Ljava/lang/Integer;
        22: aastore       
        23: dup           
        24: iconst_1      
        25: dload_2       
        26: invokestatic  #8         // Method java/lang/Double.valueOf:(D)Ljava/lang/Double;
        29: aastore       
        30: invokevirtual #9         // Method java/io/PrintStream.format:(Ljava/lang/String;[Ljava/lang/Object;)Ljava/io/PrintStream;
        33: pop           
        34: return        
\end{lstlisting}


Так что в начале значения типов \IT{int} и \IT{double} конвертируются в объекты типов 
\TT{Integer} и \TT{Double} используя методы \TT{valueOf}.

Метод \TT{format()} требует на входе объекты типа \TT{Object}, а так как классы \TT{Integer} и
\TT{Double} наследуются от корневого класса \TT{Object}, они подходят как элементы
во входном массиве.

С другой стороны, массив всегда гомогенный, т.е. он не может содержать элементы разных типов,
что делает невозможным хранение там значений типов \IT{int} и \IT{double}.


Массив объектов \TT{Object} создается по смещению 13, 
объект \TT{Integer} добавляется в массив по смещению 22, 
объект \TT{Double} добавляется в массив по смещению 29.


Предпоследняя инструкция \TT{pop} удаляет элемент на \ac{TOS}, 
так что в момент исполнения \TT{return}, стек пуст (или сбалансирован).
}

\EN{% TODO proof-reading
\subsubsection{Two-dimensional arrays}

Two-dimensional arrays in Java are just one-dimensional arrays of \IT{references} to another 
one-dimensional arrays.


Let's create a two-dimensional array:

\begin{lstlisting}[style=customjava]
	public static void main(String[] args)
	{
		int[][] a = new int[5][10];
		a[1][2]=3;
	}
\end{lstlisting}

\begin{lstlisting}
  public static void main(java.lang.String[]);
    flags: ACC_PUBLIC, ACC_STATIC
    Code:
      stack=3, locals=2, args_size=1
         0: iconst_5      
         1: bipush        10
         3: multianewarray #2,  2      // class "[[I"
         7: astore_1      
         8: aload_1       
         9: iconst_1      
        10: aaload        
        11: iconst_2      
        12: iconst_3      
        13: iastore       
        14: return        
\end{lstlisting}

It's created using the \TT{multianewarray} instruction: the object's type and dimensionality are passed
as operands.

The array's size (10*5) is left in stack (using the instructions \TT{iconst\_5} and \TT{bipush}).


A \IT{reference} to row \#1 is loaded at offset 10 (\TT{iconst\_1} and \TT{aaload}).

The column is chosen using \TT{iconst\_2} at offset 11.

The value to be written is set at offset 12.

\TT{iastore} at 13 writes the array's element.


How it is an element accessed?

\begin{lstlisting}[style=customjava]
	public static int get12 (int[][] in)
	{
		return in[1][2];
	}
\end{lstlisting}

\begin{lstlisting}
  public static int get12(int[][]);
    flags: ACC_PUBLIC, ACC_STATIC
    Code:
      stack=2, locals=1, args_size=1
         0: aload_0       
         1: iconst_1      
         2: aaload        
         3: iconst_2      
         4: iaload        
         5: ireturn       
\end{lstlisting}

A \IT{Reference} to the array's row is loaded at offset 2, the column is set at offset 3, 
then \TT{iaload} loads the array's element.

}\RU{% TODO proof-reading
\subsubsection{Двухмерные массивы}


Двухмерные массивы в Java это просто одномерные массивы \IT{reference}-в на другие одномерные 
массивы.

Создадим двухмерный массив:

\begin{lstlisting}[style=customjava]
	public static void main(String[] args)
	{
		int[][] a = new int[5][10];
		a[1][2]=3;
	}
\end{lstlisting}

\begin{lstlisting}
  public static void main(java.lang.String[]);
    flags: ACC_PUBLIC, ACC_STATIC
    Code:
      stack=3, locals=2, args_size=1
         0: iconst_5      
         1: bipush        10
         3: multianewarray #2,  2      // class "[[I"
         7: astore_1      
         8: aload_1       
         9: iconst_1      
        10: aaload        
        11: iconst_2      
        12: iconst_3      
        13: iastore       
        14: return        
\end{lstlisting}


Он создается при помощи инструкции \TT{multianewarray}: тип объекта и размерность передаются
в операндах.

Размер массива (10*5) остается в стеке (используя инструкции \TT{iconst\_5} и \TT{bipush}).


\IT{Reference} на строку \#1 загружается по смещению 10 (\TT{iconst\_1} и \TT{aaload}).

Выборка столбца происходит используя инструкцию \TT{iconst\_2} по смещению 11.

Значение для записи устанавливается по смещению 12.

\TT{iastore} на 13 записывает элемент массива.

Как его прочитать?

\begin{lstlisting}[style=customjava]
	public static int get12 (int[][] in)
	{
		return in[1][2];
	}
\end{lstlisting}

\begin{lstlisting}
  public static int get12(int[][]);
    flags: ACC_PUBLIC, ACC_STATIC
    Code:
      stack=2, locals=1, args_size=1
         0: aload_0       
         1: iconst_1      
         2: aaload        
         3: iconst_2      
         4: iaload        
         5: ireturn       
\end{lstlisting}


\IT{Reference} на строку массива загружается по смещению 2, 
столбец устанавливается по смещению 3, \TT{iaload} загружает элемент массива.
}

\EN{% TODO proof-reading
\subsubsection{Three-dimensional arrays}

Three-dimensional arrays are just one-dimensional arrays of \IT{references} 
to one-dimensional arrays of \IT{references}.


\begin{lstlisting}[style=customjava]
	public static void main(String[] args)
	{
		int[][][] a = new int[5][10][15];

		a[1][2][3]=4;

		get_elem(a);
	}
\end{lstlisting}

\begin{lstlisting}
  public static void main(java.lang.String[]);
    flags: ACC_PUBLIC, ACC_STATIC
    Code:
      stack=3, locals=2, args_size=1
         0: iconst_5      
         1: bipush        10
         3: bipush        15
         5: multianewarray #2,  3     // class "[[[I"
         9: astore_1      
        10: aload_1       
        11: iconst_1      
        12: aaload        
        13: iconst_2      
        14: aaload        
        15: iconst_3      
        16: iconst_4      
        17: iastore       
        18: aload_1       
        19: invokestatic  #3          // Method get_elem:([[[I)I
        22: pop           
        23: return        
\end{lstlisting}

Now it takes two \TT{aaload} instructions to find right \IT{reference}:


\begin{lstlisting}[style=customjava]
	public static int get_elem (int[][][] a)
	{
		return a[1][2][3];
	}
\end{lstlisting}

\begin{lstlisting}
  public static int get_elem(int[][][]);
    flags: ACC_PUBLIC, ACC_STATIC
    Code:
      stack=2, locals=1, args_size=1
         0: aload_0       
         1: iconst_1      
         2: aaload        
         3: iconst_2      
         4: aaload        
         5: iconst_3      
         6: iaload        
         7: ireturn       
\end{lstlisting}
}\RU{% TODO proof-reading
\subsubsection{Трехмерные массивы}


Трехмерные массивы это просто одномерные массивы \IT{reference}-ов на одномерные массивы 
\IT{reference}-ов.

\begin{lstlisting}[style=customjava]
	public static void main(String[] args)
	{
		int[][][] a = new int[5][10][15];

		a[1][2][3]=4;

		get_elem(a);
	}
\end{lstlisting}

\begin{lstlisting}
  public static void main(java.lang.String[]);
    flags: ACC_PUBLIC, ACC_STATIC
    Code:
      stack=3, locals=2, args_size=1
         0: iconst_5      
         1: bipush        10
         3: bipush        15
         5: multianewarray #2,  3     // class "[[[I"
         9: astore_1      
        10: aload_1       
        11: iconst_1      
        12: aaload        
        13: iconst_2      
        14: aaload        
        15: iconst_3      
        16: iconst_4      
        17: iastore       
        18: aload_1       
        19: invokestatic  #3          // Method get_elem:([[[I)I
        22: pop           
        23: return        
\end{lstlisting}


Чтобы найти нужный \IT{reference}, теперь нужно две инструкции \TT{aaload}:

\begin{lstlisting}[style=customjava]
	public static int get_elem (int[][][] a)
	{
		return a[1][2][3];
	}
\end{lstlisting}

\begin{lstlisting}
  public static int get_elem(int[][][]);
    flags: ACC_PUBLIC, ACC_STATIC
    Code:
      stack=2, locals=1, args_size=1
         0: aload_0       
         1: iconst_1      
         2: aaload        
         3: iconst_2      
         4: aaload        
         5: iconst_3      
         6: iaload        
         7: ireturn       
\end{lstlisting}
}

\EN{% TODO proof-reading
\subsubsection{Summary}

Is it possible to do a buffer overflow in Java?

No, because the array's length is always present in an array object, 
array bounds are controlled, and an exception is to be raised in case of out-of-bounds access.


There are no multi-dimensional arrays in Java in the \CCpp sense, so Java is not very suited
for fast scientific computations.

%Perhaps it's payback for memory safety, which is absent in \CCpp.
%
}\RU{% TODO proof-reading
\subsubsection{Итоги}


Возможно ли сделать переполнение буфера в Java?

Нет, потому что длина массива всегда присутствует в объекте массива,
границы массива контролируются и при попытке выйти за границы, сработает исключение.


В Java нет многомерных массивов в том смысле, как в \CCpp, так что Java не очень подходит
для быстрых научных вычислений.
%
%Видимо, это расплата за безопасность работы с памятью, которой нет в \CCpp.
}

% TODO proof-reading
\subsection{\EN{Strings}\RU{Строки}\DE{Zeichenketten}}

\EN{% TODO proof-reading
\subsubsection{First example}

Strings are objects and are constructed in the same way as other objects (and arrays).


\begin{lstlisting}[style=customjava]
	public static void main(String[] args)
	{
		System.out.println("What is your name?");
		String input = System.console().readLine();
		System.out.println("Hello, "+input);
	}
\end{lstlisting}

\begin{lstlisting}
  public static void main(java.lang.String[]);
    flags: ACC_PUBLIC, ACC_STATIC
    Code:
      stack=3, locals=2, args_size=1
         0: getstatic     #2        // Field java/lang/System.out:Ljava/io/PrintStream;
         3: ldc           #3        // String What is your name?
         5: invokevirtual #4        // Method java/io/PrintStream.println:(Ljava/lang/String;)V
         8: invokestatic  #5        // Method java/lang/System.console:()Ljava/io/Console;
        11: invokevirtual #6        // Method java/io/Console.readLine:()Ljava/lang/String;
        14: astore_1      
        15: getstatic     #2        // Field java/lang/System.out:Ljava/io/PrintStream;
        18: new           #7        // class java/lang/StringBuilder
        21: dup           
        22: invokespecial #8        // Method java/lang/StringBuilder."<init>":()V
        25: ldc           #9        // String Hello, 
        27: invokevirtual #10       // Method java/lang/StringBuilder.append:(Ljava/lang/String;)Ljava/lang/StringBuilder;
        30: aload_1       
        31: invokevirtual #10       // Method java/lang/StringBuilder.append:(Ljava/lang/String;)Ljava/lang/StringBuilder;
        34: invokevirtual #11       // Method java/lang/StringBuilder.toString:()Ljava/lang/String;
        37: invokevirtual #4        // Method java/io/PrintStream.println:(Ljava/lang/String;)V
        40: return        
\end{lstlisting}

The \TT{readLine()} method is called at offset 11, a \IT{reference} to string (which is supplied by the user) 
is then stored at \ac{TOS}.

At offset 14 the \IT{reference} to string is stored in slot 1 of \ac{LVA}.

The string the user entered is reloaded at offset 30 and concatenated with the \q{Hello, } string
using the \TT{StringBuilder} class.

The constructed string is then printed using \TT{println} at offset 37.

}\RU{% TODO proof-reading
\subsubsection{Первый пример}


Строки это объекты, и конструируются так же как и другие объекты (и массивы).

\begin{lstlisting}[style=customjava]
	public static void main(String[] args)
	{
		System.out.println("What is your name?");
		String input = System.console().readLine();
		System.out.println("Hello, "+input);
	}
\end{lstlisting}

\begin{lstlisting}
  public static void main(java.lang.String[]);
    flags: ACC_PUBLIC, ACC_STATIC
    Code:
      stack=3, locals=2, args_size=1
         0: getstatic     #2        // Field java/lang/System.out:Ljava/io/PrintStream;
         3: ldc           #3        // String What is your name?
         5: invokevirtual #4        // Method java/io/PrintStream.println:(Ljava/lang/String;)V
         8: invokestatic  #5        // Method java/lang/System.console:()Ljava/io/Console;
        11: invokevirtual #6        // Method java/io/Console.readLine:()Ljava/lang/String;
        14: astore_1      
        15: getstatic     #2        // Field java/lang/System.out:Ljava/io/PrintStream;
        18: new           #7        // class java/lang/StringBuilder
        21: dup           
        22: invokespecial #8        // Method java/lang/StringBuilder."<init>":()V
        25: ldc           #9        // String Hello, 
        27: invokevirtual #10       // Method java/lang/StringBuilder.append:(Ljava/lang/String;)Ljava/lang/StringBuilder;
        30: aload_1       
        31: invokevirtual #10       // Method java/lang/StringBuilder.append:(Ljava/lang/String;)Ljava/lang/StringBuilder;
        34: invokevirtual #11       // Method java/lang/StringBuilder.toString:()Ljava/lang/String;
        37: invokevirtual #4        // Method java/io/PrintStream.println:(Ljava/lang/String;)V
        40: return        
\end{lstlisting}


Метод \TT{readLine()} вызывается по смещению 11, \IT{reference} на строку (введенную пользователем) 
остается на \ac{TOS}.

По смещению 14, \IT{reference} на строку сохраняется в первом слоте \ac{LVA}.

Строка введенная пользователем перезагружается по смещению 30 и складывается со строкой \q{Hello, }
используя класс \TT{StringBuilder}.

Сконструированная строка затем выводится используя метод \TT{println} по смещению 37.
}

\EN{% TODO proof-reading
\subsubsection{Second example}

Another example:

\begin{lstlisting}[style=customjava]
public class strings
{
	public static char test (String a)
	{
		return a.charAt(3);
	};

	public static String concat (String a, String b)
	{
		return a+b;
	}
}
\end{lstlisting}

\begin{lstlisting}
  public static char test(java.lang.String);
    flags: ACC_PUBLIC, ACC_STATIC
    Code:
      stack=2, locals=1, args_size=1
         0: aload_0       
         1: iconst_3      
         2: invokevirtual #2        // Method java/lang/String.charAt:(I)C
         5: ireturn       
\end{lstlisting}
         
The string concatenation is performed using \TT{StringBuilder}:


\begin{lstlisting}
  public static java.lang.String concat(java.lang.String, java.lang.String);
    flags: ACC_PUBLIC, ACC_STATIC
    Code:
      stack=2, locals=2, args_size=2
         0: new           #3        // class java/lang/StringBuilder
         3: dup           
         4: invokespecial #4        // Method java/lang/StringBuilder."<init>":()V
         7: aload_0       
         8: invokevirtual #5        // Method java/lang/StringBuilder.append:(Ljava/lang/String;)Ljava/lang/StringBuilder;
        11: aload_1       
        12: invokevirtual #5        // Method java/lang/StringBuilder.append:(Ljava/lang/String;)Ljava/lang/StringBuilder;
        15: invokevirtual #6        // Method java/lang/StringBuilder.toString:()Ljava/lang/String;
        18: areturn       
\end{lstlisting}

Another example:

\begin{lstlisting}[style=customjava]
	public static void main(String[] args)
	{
		String s="Hello!";
		int n=123;
		System.out.println("s=" + s + " n=" + n);
	}
\end{lstlisting}

And again, the strings are constructed using the \TT{StringBuilder} class and its \TT{append} method,
then the constructed string is passed to \TT{println}:


\begin{lstlisting}
  public static void main(java.lang.String[]);
    flags: ACC_PUBLIC, ACC_STATIC
    Code:
      stack=3, locals=3, args_size=1
         0: ldc           #2        // String Hello!
         2: astore_1      
         3: bipush        123
         5: istore_2      
         6: getstatic     #3        // Field java/lang/System.out:Ljava/io/PrintStream;
         9: new           #4        // class java/lang/StringBuilder
        12: dup           
        13: invokespecial #5        // Method java/lang/StringBuilder."<init>":()V
        16: ldc           #6        // String s=
        18: invokevirtual #7        // Method java/lang/StringBuilder.append:(Ljava/lang/String;)Ljava/lang/StringBuilder;
        21: aload_1       
        22: invokevirtual #7        // Method java/lang/StringBuilder.append:(Ljava/lang/String;)Ljava/lang/StringBuilder;
        25: ldc           #8        // String  n=
        27: invokevirtual #7        // Method java/lang/StringBuilder.append:(Ljava/lang/String;)Ljava/lang/StringBuilder;
        30: iload_2       
        31: invokevirtual #9        // Method java/lang/StringBuilder.append:(I)Ljava/lang/StringBuilder;
        34: invokevirtual #10       // Method java/lang/StringBuilder.toString:()Ljava/lang/String;
        37: invokevirtual #11       // Method java/io/PrintStream.println:(Ljava/lang/String;)V
        40: return        
\end{lstlisting}
}\RU{% TODO proof-reading
\subsubsection{Второй пример}

Еще один пример:

\begin{lstlisting}[style=customjava]
public class strings
{
	public static char test (String a)
	{
		return a.charAt(3);
	};

	public static String concat (String a, String b)
	{
		return a+b;
	}
}
\end{lstlisting}

\begin{lstlisting}
  public static char test(java.lang.String);
    flags: ACC_PUBLIC, ACC_STATIC
    Code:
      stack=2, locals=1, args_size=1
         0: aload_0       
         1: iconst_3      
         2: invokevirtual #2        // Method java/lang/String.charAt:(I)C
         5: ireturn       
\end{lstlisting}
         

Складывание строк происходит при помощи класса \TT{StringBuilder}:

\begin{lstlisting}
  public static java.lang.String concat(java.lang.String, java.lang.String);
    flags: ACC_PUBLIC, ACC_STATIC
    Code:
      stack=2, locals=2, args_size=2
         0: new           #3        // class java/lang/StringBuilder
         3: dup           
         4: invokespecial #4        // Method java/lang/StringBuilder."<init>":()V
         7: aload_0       
         8: invokevirtual #5        // Method java/lang/StringBuilder.append:(Ljava/lang/String;)Ljava/lang/StringBuilder;
        11: aload_1       
        12: invokevirtual #5        // Method java/lang/StringBuilder.append:(Ljava/lang/String;)Ljava/lang/StringBuilder;
        15: invokevirtual #6        // Method java/lang/StringBuilder.toString:()Ljava/lang/String;
        18: areturn       
\end{lstlisting}

Еще пример:

\begin{lstlisting}[style=customjava]
	public static void main(String[] args)
	{
		String s="Hello!";
		int n=123;
		System.out.println("s=" + s + " n=" + n);
	}
\end{lstlisting}


И снова, строки создаются используя класс \TT{StringBuilder} и его метод \TT{append},
затем сконструированная строка передается в метод \TT{println}:

\begin{lstlisting}
  public static void main(java.lang.String[]);
    flags: ACC_PUBLIC, ACC_STATIC
    Code:
      stack=3, locals=3, args_size=1
         0: ldc           #2        // String Hello!
         2: astore_1      
         3: bipush        123
         5: istore_2      
         6: getstatic     #3        // Field java/lang/System.out:Ljava/io/PrintStream;
         9: new           #4        // class java/lang/StringBuilder
        12: dup           
        13: invokespecial #5        // Method java/lang/StringBuilder."<init>":()V
        16: ldc           #6        // String s=
        18: invokevirtual #7        // Method java/lang/StringBuilder.append:(Ljava/lang/String;)Ljava/lang/StringBuilder;
        21: aload_1       
        22: invokevirtual #7        // Method java/lang/StringBuilder.append:(Ljava/lang/String;)Ljava/lang/StringBuilder;
        25: ldc           #8        // String  n=
        27: invokevirtual #7        // Method java/lang/StringBuilder.append:(Ljava/lang/String;)Ljava/lang/StringBuilder;
        30: iload_2       
        31: invokevirtual #9        // Method java/lang/StringBuilder.append:(I)Ljava/lang/StringBuilder;
        34: invokevirtual #10       // Method java/lang/StringBuilder.toString:()Ljava/lang/String;
        37: invokevirtual #11       // Method java/io/PrintStream.println:(Ljava/lang/String;)V
        40: return        
\end{lstlisting}
}


\EN{% TODO proof-reading
\subsection{Exceptions}

Let's rework our \IT{Month} example (\myref{Java_2D_array_month}) a bit:

\begin{lstlisting}[caption=IncorrectMonthException.java,style=customjava]
public class IncorrectMonthException extends Exception
{
	private int index;

	public IncorrectMonthException(int index)
	{
		this.index = index;
	} 
	public int getIndex()
	{
		return index;
	}
}
\end{lstlisting}

\begin{lstlisting}[caption=Month2.java,style=customjava]
class Month2
{
	public static String[] months = 
	{
		"January", 
		"February", 
		"March", 
		"April",
		"May",
		"June",
		"July",
		"August",
		"September",
		"October",
		"November",
		"December"
	};

	public static String get_month (int i) throws IncorrectMonthException
	{
		if (i<0 || i>11)
			throw new IncorrectMonthException(i);
		return months[i];
	};

	public static void main (String[] args)
	{
		try
		{
			System.out.println(get_month(100));
		}
		catch(IncorrectMonthException e)
		{
			System.out.println("incorrect month index: "+ e.getIndex());
			e.printStackTrace();
		}
	};
}
\end{lstlisting}

Essentially, \TT{IncorrectMonthException.class} has just an object constructor 
and one getter method.

The \TT{IncorrectMonthException} class is derived from \TT{Exception}, 
so the \TT{IncorrectMonthException} constructor
first calls the constructor of the \TT{Exception} class, 
then it puts incoming integer value into the sole \TT{IncorrectMonthException} class field:

\begin{lstlisting}
  public IncorrectMonthException(int);
    flags: ACC_PUBLIC
    Code:
      stack=2, locals=2, args_size=2
         0: aload_0       
         1: invokespecial #1        // Method java/lang/Exception."<init>":()V
         4: aload_0       
         5: iload_1       
         6: putfield      #2        // Field index:I
         9: return        
\end{lstlisting}

\TT{getIndex()} is just a getter.
A \IT{reference} to \TT{IncorrectMonthException} is passed in the zeroth \ac{LVA} slot
(\IT{this}), \TT{aload\_0} takes it, \TT{getfield} loads an integer value from the object, 
\TT{ireturn} returns it.

\begin{lstlisting}
  public int getIndex();
    flags: ACC_PUBLIC
    Code:
      stack=1, locals=1, args_size=1
         0: aload_0       
         1: getfield      #2        // Field index:I
         4: ireturn       
\end{lstlisting}

Now let's take a look at \TT{get\_month()} in \TT{Month2.class}:

\begin{lstlisting}[caption=Month2.class]
  public static java.lang.String get_month(int) throws IncorrectMonthException;
    flags: ACC_PUBLIC, ACC_STATIC
    Code:
      stack=3, locals=1, args_size=1
         0: iload_0       
         1: iflt          10
         4: iload_0       
         5: bipush        11
         7: if_icmple     19
        10: new           #2        // class IncorrectMonthException
        13: dup           
        14: iload_0       
        15: invokespecial #3        // Method IncorrectMonthException."<init>":(I)V
        18: athrow        
        19: getstatic     #4        // Field months:[Ljava/lang/String;
        22: iload_0       
        23: aaload        
        24: areturn       
\end{lstlisting}

\TT{iflt} at offset 1 is \IT{if less than}.

In case of invalid index, a new object is created using the \TT{new} instruction at offset 10.

The object's type is passed as an operand to the instruction (which is \TT{IncorrectMonthException}).

Then its constructor is called, and index is passed via \ac{TOS} (offset 15).

When the control flow is offset 18, the object is already constructed, 
so now the \TT{athrow} instruction takes a \IT{reference} 
to the newly constructed object and signals to \ac{JVM} to find the appropriate exception handler.

The \TT{athrow} instruction doesn't return the control flow here, 
so at offset 19 there is another \gls{basic block},
not related to exceptions business, where we can get from offset 7.

How do handlers work?

\main in \TT{Month2.class}:

\begin{lstlisting}[caption=Month2.class]
  public static void main(java.lang.String[]);
    flags: ACC_PUBLIC, ACC_STATIC
    Code:
      stack=3, locals=2, args_size=1
         0: getstatic     #5        // Field java/lang/System.out:Ljava/io/PrintStream;
         3: bipush        100
         5: invokestatic  #6        // Method get_month:(I)Ljava/lang/String;
         8: invokevirtual #7        // Method java/io/PrintStream.println:(Ljava/lang/String;)V
        11: goto          47
        14: astore_1      
        15: getstatic     #5        // Field java/lang/System.out:Ljava/io/PrintStream;
        18: new           #8        // class java/lang/StringBuilder
        21: dup           
        22: invokespecial #9        // Method java/lang/StringBuilder."<init>":()V
        25: ldc           #10       // String incorrect month index: 
        27: invokevirtual #11       // Method java/lang/StringBuilder.append:(Ljava/lang/String;)Ljava/lang/StringBuilder;
        30: aload_1       
        31: invokevirtual #12       // Method IncorrectMonthException.getIndex:()I
        34: invokevirtual #13       // Method java/lang/StringBuilder.append:(I)Ljava/lang/StringBuilder;
        37: invokevirtual #14       // Method java/lang/StringBuilder.toString:()Ljava/lang/String;
        40: invokevirtual #7        // Method java/io/PrintStream.println:(Ljava/lang/String;)V
        43: aload_1       
        44: invokevirtual #15       // Method IncorrectMonthException.printStackTrace:()V
        47: return        
      Exception table:
         from    to  target type
             0    11    14   Class IncorrectMonthException
\end{lstlisting}

Here is the \TT{Exception table}, which defines that from offsets 0 to 11 (inclusive) an exception \\
\TT{IncorrectMonthException} may happen, and if it does, the control flow is to be passed to offset 14.

Indeed, the main program ends at offset 11.

At offset 14 the handler starts. It's not possible to get here, 
there are no conditional/unconditional jumps to this area.

But \ac{JVM} will transfer the execution flow here in case of an exception.

The very first \TT{astore\_1} (at 14) takes the incoming \IT{reference} to the exception object 
and stores it in \ac{LVA} slot 1.

Later, the \TT{getIndex()} method (of this exception object) will be called at offset 31.

The \IT{reference} to the current exception object is passed right before that (offset 30).

The rest of the code is does just string manipulation: 
first the integer value returned by \TT{getIndex()}
is converted to string by the \TT{toString()} method, then it's concatenated with 
the \q{incorrect month index: } text string (like we saw before),
then \TT{println()} and \TT{printStackTrace()} are called.

After \TT{printStackTrace()} finishes, the exception is handled and we can continue with the normal execution.

At offset 47 there is a \TT{return} which finishes the \main function, 
but there could be any other code which would execute as if no exceptions were raised.

Here is an example on how IDA shows exception ranges:


\begin{lstlisting}[caption=from some random .class file found on the author's computer]
    .catch java/io/FileNotFoundException from met001_335 to met001_360\
 using met001_360
    .catch java/io/FileNotFoundException from met001_185 to met001_214\
 using met001_214
    .catch java/io/FileNotFoundException from met001_181 to met001_192\
 using met001_195
    .catch java/io/FileNotFoundException from met001_155 to met001_176\
 using met001_176
    .catch java/io/FileNotFoundException from met001_83 to met001_129 using \
met001_129
    .catch java/io/FileNotFoundException from met001_42 to met001_66 using \
met001_69
    .catch java/io/FileNotFoundException from met001_begin to met001_37\
 using met001_37
\end{lstlisting}
}\RU{% TODO proof-reading
\subsection{Исключения}

Немного переделаем пример \IT{Month} (\myref{Java_2D_array_month}):

\begin{lstlisting}[caption=IncorrectMonthException.java,style=customjava]
public class IncorrectMonthException extends Exception
{
	private int index;

	public IncorrectMonthException(int index)
	{
		this.index = index;
	} 
	public int getIndex()
	{
		return index;
	}
}
\end{lstlisting}

\begin{lstlisting}[caption=Month2.java,style=customjava]
class Month2
{
	public static String[] months = 
	{
		"January", 
		"February", 
		"March", 
		"April",
		"May",
		"June",
		"July",
		"August",
		"September",
		"October",
		"November",
		"December"
	};

	public static String get_month (int i) throws IncorrectMonthException
	{
		if (i<0 || i>11)
			throw new IncorrectMonthException(i);
		return months[i];
	};

	public static void main (String[] args)
	{
		try
		{
			System.out.println(get_month(100));
		}
		catch(IncorrectMonthException e)
		{
			System.out.println("incorrect month index: "+ e.getIndex());
			e.printStackTrace();
		}
	};
}
\end{lstlisting}

Которко говоря, \TT{IncorrectMonthException.class} имеет только конструктор объетка и один
метод-акцессор.

Класс \TT{IncorrectMonthException} наследуется от \TT{Exception}, \\
так что конструктор \TT{IncorrectMonthException}
в начале вызывает конструктор класса \TT{Exception}, 
затем он перекладывает входящее значение в единственное поле класса \TT{IncorrectMonthException}:

\begin{lstlisting}
  public IncorrectMonthException(int);
    flags: ACC_PUBLIC
    Code:
      stack=2, locals=2, args_size=2
         0: aload_0       
         1: invokespecial #1        // Method java/lang/Exception."<init>":()V
         4: aload_0       
         5: iload_1       
         6: putfield      #2        // Field index:I
         9: return        
\end{lstlisting}

\TT{getIndex()} это просто акцессор.

\IT{Reference} (указатель) на \TT{IncorrectMonthException} передается в нулевом слоте \ac{LVA}
(\IT{this}), \TT{aload\_0} берет его, \TT{getfield} загружает значение из объекта, 
\TT{ireturn} возвращает его.

\begin{lstlisting}
  public int getIndex();
    flags: ACC_PUBLIC
    Code:
      stack=1, locals=1, args_size=1
         0: aload_0       
         1: getfield      #2        // Field index:I
         4: ireturn       
\end{lstlisting}

Посмотрим на \TT{get\_month()} в \TT{Month2.class}:

\begin{lstlisting}[caption=Month2.class]
  public static java.lang.String get_month(int) throws IncorrectMonthException;
    flags: ACC_PUBLIC, ACC_STATIC
    Code:
      stack=3, locals=1, args_size=1
         0: iload_0       
         1: iflt          10
         4: iload_0       
         5: bipush        11
         7: if_icmple     19
        10: new           #2        // class IncorrectMonthException
        13: dup           
        14: iload_0       
        15: invokespecial #3        // Method IncorrectMonthException."<init>":(I)V
        18: athrow        
        19: getstatic     #4        // Field months:[Ljava/lang/String;
        22: iload_0       
        23: aaload        
        24: areturn       
\end{lstlisting}

\TT{iflt} по смещению 1 это \IT{if less than} (если меньше, чем).

В случае неправильного индекса, создается новый объект при помощи инструкции \TT{new} 
по смещению 10.

Тип объекта передается как операнд инструкции\\
(и это \TT{IncorrectMonthException}).

Затем вызывается его конструктор, в который передается индекс (через \ac{TOS}) (по смещению 15).

В то время как управление находится на смещении 18, объект уже создан,
теперь инструкция \TT{athrow} берет указатель (\IT{reference})
на только что созданный объект и сигнализирует в \ac{JVM}, чтобы тот нашел подходящий обработчик
исключения.

Инструкция \TT{athrow} не возвращает управление сюда,
так что по смещению 19 здесь совсем другой \gls{basic block},
не имеющий отношения к исключениям, сюда можно попасть со смещения 7.

Как работает обработчик?
Посмотрим на \main в \TT{Month2.class}:

\begin{lstlisting}[caption=Month2.class]
  public static void main(java.lang.String[]);
    flags: ACC_PUBLIC, ACC_STATIC
    Code:
      stack=3, locals=2, args_size=1
         0: getstatic     #5        // Field java/lang/System.out:Ljava/io/PrintStream;
         3: bipush        100
         5: invokestatic  #6        // Method get_month:(I)Ljava/lang/String;
         8: invokevirtual #7        // Method java/io/PrintStream.println:(Ljava/lang/String;)V
        11: goto          47
        14: astore_1      
        15: getstatic     #5        // Field java/lang/System.out:Ljava/io/PrintStream;
        18: new           #8        // class java/lang/StringBuilder
        21: dup           
        22: invokespecial #9        // Method java/lang/StringBuilder."<init>":()V
        25: ldc           #10       // String incorrect month index: 
        27: invokevirtual #11       // Method java/lang/StringBuilder.append:(Ljava/lang/String;)Ljava/lang/StringBuilder;
        30: aload_1       
        31: invokevirtual #12       // Method IncorrectMonthException.getIndex:()I
        34: invokevirtual #13       // Method java/lang/StringBuilder.append:(I)Ljava/lang/StringBuilder;
        37: invokevirtual #14       // Method java/lang/StringBuilder.toString:()Ljava/lang/String;
        40: invokevirtual #7        // Method java/io/PrintStream.println:(Ljava/lang/String;)V
        43: aload_1       
        44: invokevirtual #15       // Method IncorrectMonthException.printStackTrace:()V
        47: return        
      Exception table:
         from    to  target type
             0    11    14   Class IncorrectMonthException
\end{lstlisting}

Тут есть \TT{Exception table}, которая определяет, что между смещениями 0 и 11 (включительно)
может случится исключение \\
\TT{IncorrectMonthException}, и если это произойдет, то нужно передать
управление на смещение 14.

Действительно, основная программа заканчивается на смещении 11.

По смещению 14 начинается обработчик, и сюда невозможно попасть, 
здесь нет никаких условных/безусловных переходов в эту область.

Но \ac{JVM} передаст сюда управление в случае исключения.

Самая первая \TT{astore\_1} (на 14) берет входящий указатель (\IT{reference}) на объект 
исключения и сохраняет его в слоте 1 \ac{LVA}.

Позже, по смещению 31 будет вызван метод этого объекта \\
(\TT{getIndex()}).

Указатель \IT{reference} на текующий объект исключения передался немного раньше (смещение 30).

Остальной код это просто код для манипуляции со строками: 
в начале значение возвращенное методом \TT{getIndex()}
конвертируется в строку используя метод \TT{toString()}, 
затем эта строка прибавляется к текстовой строке
\q{incorrect month index: } (как мы уже рассматривали ранее),
затем вызываются \TT{println()} и \TT{printStackTrace()}.

После того как \TT{printStackTrace()} заканчивается, исключение уже обработано, мы можем
возвращаться к нормальной работе.

По смещению 47 есть \TT{return}, который заканчивает работу функции \main, 
но там может быть любой другой код, который исполнится, если исключения не произошло.

Вот пример, как IDA показывает интервалы исключений:

\begin{lstlisting}[caption=из какого-то случайного найденного на компьютере автора .class-файла]
    .catch java/io/FileNotFoundException from met001_335 to met001_360\
 using met001_360
    .catch java/io/FileNotFoundException from met001_185 to met001_214\
 using met001_214
    .catch java/io/FileNotFoundException from met001_181 to met001_192\
 using met001_195
    .catch java/io/FileNotFoundException from met001_155 to met001_176\
 using met001_176
    .catch java/io/FileNotFoundException from met001_83 to met001_129 using \
met001_129
    .catch java/io/FileNotFoundException from met001_42 to met001_66 using \
met001_69
    .catch java/io/FileNotFoundException from met001_begin to met001_37\
 using met001_37
\end{lstlisting}

}

\EN{% TODO proof-reading
\subsection{Classes}

Simple class:

\begin{lstlisting}[caption=test.java,style=customjava]
public class test
{
        public static int a;
        private static int b;

        public test()
        {
            a=0;
            b=0;
        }
        public static void set_a (int input)
	{
		a=input;
	}
	public static int get_a ()
	{
		return a;
	}
	public static void set_b (int input)
	{
		b=input;
	}
	public static int get_b ()
	{
		return b;
	}
}
\end{lstlisting}

The constructor just sets both fields to zero:


\begin{lstlisting}
  public test();
    flags: ACC_PUBLIC
    Code:
      stack=1, locals=1, args_size=1
         0: aload_0       
         1: invokespecial #1         // Method java/lang/Object."<init>":()V
         4: iconst_0      
         5: putstatic     #2         // Field a:I
         8: iconst_0      
         9: putstatic     #3         // Field b:I
        12: return        
\end{lstlisting}
        
Setter of \TT{a}:

\begin{lstlisting}
  public static void set_a(int);
    flags: ACC_PUBLIC, ACC_STATIC
    Code:
      stack=1, locals=1, args_size=1
         0: iload_0       
         1: putstatic     #2         // Field a:I
         4: return        
\end{lstlisting}

Getter of \TT{a}:

\begin{lstlisting}
  public static int get_a();
    flags: ACC_PUBLIC, ACC_STATIC
    Code:
      stack=1, locals=0, args_size=0
         0: getstatic     #2         // Field a:I
         3: ireturn       
\end{lstlisting}

Setter of \TT{b}:

\begin{lstlisting}
  public static void set_b(int);
    flags: ACC_PUBLIC, ACC_STATIC
    Code:
      stack=1, locals=1, args_size=1
         0: iload_0       
         1: putstatic     #3         // Field b:I
         4: return        
\end{lstlisting}

Getter of \TT{b}:

\begin{lstlisting}
  public static int get_b();
    flags: ACC_PUBLIC, ACC_STATIC
    Code:
      stack=1, locals=0, args_size=0
         0: getstatic     #3         // Field b:I
         3: ireturn       
\end{lstlisting}

There is no difference in the code which works with public and private fields.

But this type information is present in the \TT{.class} file, 
and it's not possible to access private fields from everywhere.


Let's create an object and call its method:


\begin{lstlisting}[caption=ex1.java,style=customjava]
public class ex1
{
	public static void main(String[] args)
	{
		test obj=new test();
		obj.set_a (1234);
		System.out.println(obj.a);
	}
}
\end{lstlisting}

\begin{lstlisting}
  public static void main(java.lang.String[]);
    flags: ACC_PUBLIC, ACC_STATIC
    Code:
      stack=2, locals=2, args_size=1
         0: new           #2        // class test
         3: dup           
         4: invokespecial #3        // Method test."<init>":()V
         7: astore_1      
         8: aload_1       
         9: pop           
        10: sipush        1234
        13: invokestatic  #4        // Method test.set_a:(I)V
        16: getstatic     #5        // Field java/lang/System.out:Ljava/io/PrintStream;
        19: aload_1       
        20: pop           
        21: getstatic     #6        // Field test.a:I
        24: invokevirtual #7        // Method java/io/PrintStream.println:(I)V
        27: return        
\end{lstlisting}

The \TT{new} instruction creates an object, but doesn't call the constructor (it is called at offset 4).

The \TT{set\_a()} method is called at offset 16.

The \TT{a} field is accessed using the \TT{getstatic} instruction at offset 21.

}
\RU{% TODO proof-reading
\subsection{Классы}

Простой класс:

\begin{lstlisting}[caption=test.java,style=customjava]
public class test
{
        public static int a;
        private static int b;

        public test()
        {
            a=0;
            b=0;
        }
        public static void set_a (int input)
	{
		a=input;
	}
	public static int get_a ()
	{
		return a;
	}
	public static void set_b (int input)
	{
		b=input;
	}
	public static int get_b ()
	{
		return b;
	}
}
\end{lstlisting}


Конструктор просто выставляет оба поля класса в нули:

\begin{lstlisting}
  public test();
    flags: ACC_PUBLIC
    Code:
      stack=1, locals=1, args_size=1
         0: aload_0       
         1: invokespecial #1         // Method java/lang/Object."<init>":()V
         4: iconst_0      
         5: putstatic     #2         // Field a:I
         8: iconst_0      
         9: putstatic     #3         // Field b:I
        12: return        
\end{lstlisting}
        
Сеттер \TT{a}:

\begin{lstlisting}
  public static void set_a(int);
    flags: ACC_PUBLIC, ACC_STATIC
    Code:
      stack=1, locals=1, args_size=1
         0: iload_0       
         1: putstatic     #2         // Field a:I
         4: return        
\end{lstlisting}

Геттер \TT{a}:

\begin{lstlisting}
  public static int get_a();
    flags: ACC_PUBLIC, ACC_STATIC
    Code:
      stack=1, locals=0, args_size=0
         0: getstatic     #2         // Field a:I
         3: ireturn       
\end{lstlisting}

Сеттер \TT{b}:

\begin{lstlisting}
  public static void set_b(int);
    flags: ACC_PUBLIC, ACC_STATIC
    Code:
      stack=1, locals=1, args_size=1
         0: iload_0       
         1: putstatic     #3         // Field b:I
         4: return        
\end{lstlisting}

Геттер \TT{b}:

\begin{lstlisting}
  public static int get_b();
    flags: ACC_PUBLIC, ACC_STATIC
    Code:
      stack=1, locals=0, args_size=0
         0: getstatic     #3         // Field b:I
         3: ireturn       
\end{lstlisting}


Здесь нет разницы между кодом, работающим для public-полей и private-полей.

Но эта информация присутствует в \TT{.class}-файле,
и в любом случае невозможно иметь доступ к private-полям.


Создадим объект и вызовем метод:

\begin{lstlisting}[caption=ex1.java,style=customjava]
public class ex1
{
	public static void main(String[] args)
	{
		test obj=new test();
		obj.set_a (1234);
		System.out.println(obj.a);
	}
}
\end{lstlisting}

\begin{lstlisting}
  public static void main(java.lang.String[]);
    flags: ACC_PUBLIC, ACC_STATIC
    Code:
      stack=2, locals=2, args_size=1
         0: new           #2        // class test
         3: dup           
         4: invokespecial #3        // Method test."<init>":()V
         7: astore_1      
         8: aload_1       
         9: pop           
        10: sipush        1234
        13: invokestatic  #4        // Method test.set_a:(I)V
        16: getstatic     #5        // Field java/lang/System.out:Ljava/io/PrintStream;
        19: aload_1       
        20: pop           
        21: getstatic     #6        // Field test.a:I
        24: invokevirtual #7        // Method java/io/PrintStream.println:(I)V
        27: return        
\end{lstlisting}


Инструкция \TT{new} создает объект, но не вызывает конструктор (он вызывается по смещению 4).

Метод \TT{set\_a()} вызывается по смещению 16.

К полю \TT{a} имеется доступ используя инструкцию \TT{getstatic} по смещению 21.
}
\DE{% TODO proof-reading
\subsection{Klassen}

Einfache Klassen:

\begin{lstlisting}[caption=test.java,style=customjava]
public class test
{
        public static int a;
        private static int b;

        public test()
        {
            a=0;
            b=0;
        }
        public static void set_a (int input)
	{
		a=input;
	}
	public static int get_a ()
	{
		return a;
	}
	public static void set_b (int input)
	{
		b=input;
	}
	public static int get_b ()
	{
		return b;
	}
}
\end{lstlisting}

Der Konstruktor setzt lediglich beide Felder auf 0:

\begin{lstlisting}
  public test();
    flags: ACC_PUBLIC
    Code:
      stack=1, locals=1, args_size=1
         0: aload_0       
         1: invokespecial #1         // Method java/lang/Object."<init>":()V
         4: iconst_0      
         5: putstatic     #2         // Feld a:I
         8: iconst_0      
         9: putstatic     #3         // Feld b:I
        12: return        
\end{lstlisting}

Setter von \TT{a}:

\begin{lstlisting}
  public static void set_a(int);
    flags: ACC_PUBLIC, ACC_STATIC
    Code:
      stack=1, locals=1, args_size=1
         0: iload_0       
         1: putstatic     #2         // Feld a:I
         4: return        
\end{lstlisting}

Getter von \TT{a}:

\begin{lstlisting}
  public static int get_a();
    flags: ACC_PUBLIC, ACC_STATIC
    Code:
      stack=1, locals=0, args_size=0
         0: getstatic     #2         // Feld a:I
         3: ireturn       
\end{lstlisting}

Setter von \TT{b}:

\begin{lstlisting}
  public static void set_b(int);
    flags: ACC_PUBLIC, ACC_STATIC
    Code:
      stack=1, locals=1, args_size=1
         0: iload_0       
         1: putstatic     #3         // Feld b:I
         4: return        
\end{lstlisting}

Getter von \TT{b}:

\begin{lstlisting}
  public static int get_b();
    flags: ACC_PUBLIC, ACC_STATIC
    Code:
      stack=1, locals=0, args_size=0
         0: getstatic     #3         // Feld b:I
         3: ireturn       
\end{lstlisting}

Es gibt keinen Unterschied in den Codes die mit public oder private Feldern arbeiten.

Aber diese Information ist in der \TT{.class}-Datei vorhanden und es ist nicht
möglich auf die privaten Felder zuzugreifen.

Erstellen wir ein Objekt und rufen seine Methoden auf:

\begin{lstlisting}[caption=ex1.java,style=customjava]
public class ex1
{
	public static void main(String[] args)
	{
		test obj=new test();
		obj.set_a (1234);
		System.out.println(obj.a);
	}
}
\end{lstlisting}

\begin{lstlisting}
  public static void main(java.lang.String[]);
    flags: ACC_PUBLIC, ACC_STATIC
    Code:
      stack=2, locals=2, args_size=1
         0: new           #2        // Klasse test
         3: dup           
         4: invokespecial #3        // Methode test."<init>":()V
         7: astore_1      
         8: aload_1       
         9: pop           
        10: sipush        1234
        13: invokestatic  #4        // Methode test.set_a:(I)V
        16: getstatic     #5        // Feld java/lang/System.out:Ljava/io/PrintStream;
        19: aload_1       
        20: pop           
        21: getstatic     #6        // Feld test.a:I
        24: invokevirtual #7        // Methode java/io/PrintStream.println:(I)V
        27: return        
\end{lstlisting}

Die \TT{new}-Anweisung erstellt ein Objekt aber ruft den Konstruktor nicht auf
(dieser wird bei Offset 4 aufgerufen).

Die \TT{set\_a()}-Methode wird an Offset 16 aufgerufen.

Auf das Feld \TT{a} wird mit der \TT{getstatic}-Anweisung an Offset 21 zugegriffen.
}

% TODO proof-reading
\subsection{\EN{Simple patching}\RU{Простейшая модификация}\DE{Einfaches Patchen}}
\myindex{IDA}

% subsections:
\EN{% TODO proof-reading
\subsubsection{First example}

Let's proceed with a simple code patching task.

\begin{lstlisting}[style=customjava]
public class nag
{
	public static void nag_screen()
	{
		System.out.println("This program is not registered");
	};
	public static void main(String[] args) 
	{
		System.out.println("Greetings from the mega-software");
		nag_screen();
	}
}
\end{lstlisting}

How would we remove the printing of \q{This program is not registered} string?


Let's load the .class file into IDA:


\begin{figure}[H]
\centering
\myincludegraphics{Java_and_NET/java/13_patching/1/1.png}
\caption{IDA}
\end{figure}

Let's patch the first byte of the function to 177 (which is the \TT{return} instruction's opcode):


\begin{figure}[H]
\centering
\myincludegraphics{Java_and_NET/java/13_patching/1/2.png}
\caption{IDA}
\end{figure}

But that doesn't work (JRE 1.7):

\begin{lstlisting}
Exception in thread "main" java.lang.VerifyError: Expecting a stack map frame
Exception Details:
  Location:
    nag.nag_screen()V @1: nop
  Reason:
    Error exists in the bytecode
  Bytecode:
    0000000: b100 0212 03b6 0004 b1

        at java.lang.Class.getDeclaredMethods0(Native Method)
        at java.lang.Class.privateGetDeclaredMethods(Class.java:2615)
        at java.lang.Class.getMethod0(Class.java:2856)
        at java.lang.Class.getMethod(Class.java:1668)
        at sun.launcher.LauncherHelper.getMainMethod(LauncherHelper.java:494)
        at sun.launcher.LauncherHelper.checkAndLoadMain(LauncherHelper.java:486)
\end{lstlisting}

Perhaps JVM has some other checks related to the stack maps.%


OK, let's patch it differently by removing the call to \TT{nag()}:


\begin{figure}[H]
\centering
\myincludegraphics{Java_and_NET/java/13_patching/1/3.png}
\caption{IDA}
\end{figure}

0 is the opcode for \ac{NOP}.


Now that works!
}\RU{% TODO proof-reading
\subsubsection{Первый пример}


Перейдем к простой задаче модификации кода.

\begin{lstlisting}[style=customjava]
public class nag
{
	public static void nag_screen()
	{
		System.out.println("This program is not registered");
	};
	public static void main(String[] args) 
	{
		System.out.println("Greetings from the mega-software");
		nag_screen();
	}
}
\end{lstlisting}


Как можно избавиться от печати строки \q{This program is not registered}?


Наконец загрузим .class-файл в IDA:

\begin{figure}[H]
\centering
\myincludegraphics{Java_and_NET/java/13_patching/1/1.png}
\caption{IDA}
\end{figure}


В начале заменим первый байт функции на 177 (это опкод инструкции \TT{return}):

\begin{figure}[H]
\centering
\myincludegraphics{Java_and_NET/java/13_patching/1/2.png}
\caption{IDA}
\end{figure}

Но это не работает (JRE 1.7):

\begin{lstlisting}
Exception in thread "main" java.lang.VerifyError: Expecting a stack map frame
Exception Details:
  Location:
    nag.nag_screen()V @1: nop
  Reason:
    Error exists in the bytecode
  Bytecode:
    0000000: b100 0212 03b6 0004 b1

        at java.lang.Class.getDeclaredMethods0(Native Method)
        at java.lang.Class.privateGetDeclaredMethods(Class.java:2615)
        at java.lang.Class.getMethod0(Class.java:2856)
        at java.lang.Class.getMethod(Class.java:1668)
        at sun.launcher.LauncherHelper.getMainMethod(LauncherHelper.java:494)
        at sun.launcher.LauncherHelper.checkAndLoadMain(LauncherHelper.java:486)
\end{lstlisting}

%
Вероятно, в JVM есть проверки связанные с картами стека.


ОК, попробуем пропатчить её иначе, удаляя вызов функции \TT{nag()}:

\begin{figure}[H]
\centering
\myincludegraphics{Java_and_NET/java/13_patching/1/3.png}
\caption{IDA}
\end{figure}


0 это опкод инструкции \ac{NOP}.

Теперь всё работает!
}

\EN{% TODO proof-reading
\subsubsection{Second example}

Another simple crackme example:

\begin{lstlisting}[style=customjava]
public class password
{
	public static void main(String[] args)
	{
		System.out.println("Please enter the password");
		String input = System.console().readLine();
		if (input.equals("secret"))
			System.out.println("password is correct");
		else
			System.out.println("password is not correct");
	}
}
\end{lstlisting}

Let's load it in IDA:

\begin{figure}[H]
\centering
\myincludegraphics{Java_and_NET/java/13_patching/2/1.png}
\caption{IDA}
\end{figure}

We see here the \TT{ifeq} instruction which does the job.

Its name stands for \IT{if equal}, and this is misnomer, a better name would be \TT{ifz} (\IT{if zero}), i.e, 
if value at \ac{TOS} is zero, then do the jump.

In our example, it jumps if the password is not correct 
(the \TT{equals} method returns \TT{False}, which is 0).

The very first idea is to patch this instruction.

There are two bytes in \TT{ifeq} opcode, which encode the jump offset.

To make this instruction a NOP, we must set the 3rd byte to the value of 3 
(because by adding 3 to the current address we will always jump to the next instruction,
since the \TT{ifeq} instruction's length is 3 bytes):


\begin{figure}[H]
\centering
\myincludegraphics{Java_and_NET/java/13_patching/2/2.png}
\caption{IDA}
\end{figure}

That doesn't work (JRE 1.7):

\begin{lstlisting}
Exception in thread "main" java.lang.VerifyError: Expecting a stackmap frame at branch target 24
Exception Details:
  Location:
    password.main([Ljava/lang/String;)V @21: ifeq
  Reason:
    Expected stackmap frame at this location.
  Bytecode:
    0000000: b200 0212 03b6 0004 b800 05b6 0006 4c2b
    0000010: 1207 b600 0899 0003 b200 0212 09b6 0004
    0000020: a700 0bb2 0002 120a b600 04b1
  Stackmap Table:
    append_frame(@35,Object[#20])
    same_frame(@43)

        at java.lang.Class.getDeclaredMethods0(Native Method)
        at java.lang.Class.privateGetDeclaredMethods(Class.java:2615)
        at java.lang.Class.getMethod0(Class.java:2856)
        at java.lang.Class.getMethod(Class.java:1668)
        at sun.launcher.LauncherHelper.getMainMethod(LauncherHelper.java:494)
        at sun.launcher.LauncherHelper.checkAndLoadMain(LauncherHelper.java:486)
\end{lstlisting}

But it must be mentioned that it worked in JRE 1.6.

We can also try to replace to all 3 \TT{ifeq} opcode bytes with zero bytes (\ac{NOP}), 
and it still won't work.

Seems like there are more stack map checks in JRE 1.7.


OK, we'll replace the whole call to the \TT{equals} method with the \TT{iconst\_1} instruction 
plus a pack of \ac{NOP}s:


\begin{figure}[H]
\centering
\myincludegraphics{Java_and_NET/java/13_patching/2/3.png}
\caption{IDA}
\end{figure}

1 needs always to be in the \ac{TOS} when the \TT{ifeq} instruction is executed, 
so \TT{ifeq} would never jump.


This works.
}\RU{% TODO proof-reading
\subsubsection{Второй пример}

Еще один простой пример crackme:

\begin{lstlisting}[style=customjava]
public class password
{
	public static void main(String[] args)
	{
		System.out.println("Please enter the password");
		String input = System.console().readLine();
		if (input.equals("secret"))
			System.out.println("password is correct");
		else
			System.out.println("password is not correct");
	}
}
\end{lstlisting}

Загрузим в IDA:

\begin{figure}[H]
\centering
\myincludegraphics{Java_and_NET/java/13_patching/2/1.png}
\caption{IDA}
\end{figure}


Видим здесь инструкцию \TT{ifeq}, которая, собственно, всё и делает.

Её имя означает \IT{if equal}, и это не очень удачное название, её следовало бы назвать 
\TT{ifz} (\IT{if zero}), т.е. если значение на \ac{TOS} ноль, тогда совершить переход.

В нашем случае, переход происходит если пароль не верный 
(метод \TT{equals} возвращает \TT{False}, а это 0).

Первое что приходит в голову это пропатчить эту инструкцию.

В опкоде \TT{ifeq} два байта, в которых закодировано смещение для перехода.

Чтобы инструкция не работала, мы должны установить байт 3 на третьем байте
(потому что 3 будет прибавляться к текущему смещению, и в итоге переход будет на следующую инструкцию,
ведь длина инструкции \TT{ifeq} это 3 байта):

\begin{figure}[H]
\centering
\myincludegraphics{Java_and_NET/java/13_patching/2/2.png}
\caption{IDA}
\end{figure}

Это не работает (JRE 1.7):

\begin{lstlisting}
Exception in thread "main" java.lang.VerifyError: Expecting a stackmap frame at branch target 24
Exception Details:
  Location:
    password.main([Ljava/lang/String;)V @21: ifeq
  Reason:
    Expected stackmap frame at this location.
  Bytecode:
    0000000: b200 0212 03b6 0004 b800 05b6 0006 4c2b
    0000010: 1207 b600 0899 0003 b200 0212 09b6 0004
    0000020: a700 0bb2 0002 120a b600 04b1
  Stackmap Table:
    append_frame(@35,Object[#20])
    same_frame(@43)

        at java.lang.Class.getDeclaredMethods0(Native Method)
        at java.lang.Class.privateGetDeclaredMethods(Class.java:2615)
        at java.lang.Class.getMethod0(Class.java:2856)
        at java.lang.Class.getMethod(Class.java:1668)
        at sun.launcher.LauncherHelper.getMainMethod(LauncherHelper.java:494)
        at sun.launcher.LauncherHelper.checkAndLoadMain(LauncherHelper.java:486)
\end{lstlisting}

Хотя, надо сказать, работает в JRE 1.6.


Мы также можем попробовать заменить все три байта опкода \TT{ifeq} на нулевые байты (\ac{NOP}), 
но это тоже не работает.

Видимо, начиная с JRE 1.7, там появилось больше проверок карт стека.


ОК, заменим весь вызов метода \TT{equals} на инструкцию \TT{iconst\_1} плюс набор 
\ac{NOP}-ов:

\begin{figure}[H]
\centering
\myincludegraphics{Java_and_NET/java/13_patching/2/3.png}
\caption{IDA}
\end{figure}


1 будет всегда на \ac{TOS} во время исполнения инструкции \TT{ifeq}, 
так что \TT{ifeq} никогда не совершит переход.

Это работает.
}


\EN{% TODO proof-reading
\subsection{Summary}

What is missing in Java in comparison to \CCpp?


\begin{itemize}
\item Structures: use classes.

\item Unions: use class hierarchies.

\item Unsigned data types.
By the way, this makes cryptographic algorithms somewhat harder to implement in Java.


\item Function pointers.
\end{itemize}
}
\RU{% TODO proof-reading
\subsection{Итоги}


Чего не хватает в Java в сравнении с \CCpp?

\begin{itemize}
\item Структуры: используйте классы.

\item Объединения (union): используйте иерархии классов.

\item Беззнаковые типы данных.

Кстати, из-за этого реализовывать криптографические алгоритмы на Java немного труднее.

\item Указатели на функции.
\end{itemize}
}
\DE{% TODO proof-reading
\subsection{Zusammenfassung}

Was fehlt in Java im Vergleich zu \CCpp?

\begin{itemize}
\item Strukturen: Nutzen Sie Klassen.

\item Unions: Nutzen Sie Klassenhierarchien.

\item Vorzeichenlose Datentypen.
Dies macht es etwas schwieriger kryptografische Algorithmen in Java zu implementieren.

\item Funktionszeiger.
\end{itemize}
}

