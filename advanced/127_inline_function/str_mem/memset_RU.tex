\subsubsection{memset()}
\myindex{\CStandardLibrary!memset()}

\myparagraph{\Example \#1}

\lstinputlisting[caption=32 байта,style=customc]{\CURPATH/str_mem/memset_32.c}

\myindex{x86!\Instructions!MOV}

Многие компиляторы не генерируют вызов memset() для коротких блоков, а просто вставляют набор \MOV{}-ов:

\lstinputlisting[caption=\Optimizing GCC 4.9.1 x64,style=customasmx86]{\CURPATH/str_mem/memset_32_GCC491_x64_O3.s}

\myindex{Unrolled loop}
Кстати, это напоминает развернутые циклы: 
\myref{ARM_unrolled_loops}.

\myparagraph{\Example \#2}

\lstinputlisting[caption=67 байт,style=customc]{\CURPATH/str_mem/memset_67.c}


Когда размер блока не кратен 4 или 8, разные компиляторы могут вести себя по-разному.

\myindex{x86!\Instructions!MOV}

Например, MSVC 2012 продолжает вставлять \MOV:

\lstinputlisting[caption=\Optimizing MSVC 2012 x64,style=customasmx86]{\CURPATH/str_mem/memset_67_MSVC2012_x64_Ox.asm}

\myindex{x86!\Instructions!STOSQ}

\dots а GCC использует \TT{REP STOSQ}, полагая, что так будет короче, чем пачка \TT{MOV}'s:

\lstinputlisting[caption=\Optimizing GCC 4.9.1 x64,style=customasmx86]{\CURPATH/str_mem/memset_67_GCC491_x64_O3.s}
