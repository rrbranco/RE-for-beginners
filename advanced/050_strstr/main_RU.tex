\section{Пример strstr()}
\label{strstr_example}
\myindex{\CStandardLibrary!strstr()}

Вернемся к тому факту, что GCC иногда использует только часть строки: \myref{use_parts_of_C_strings}.

Ф-ция \IT{strstr()} (из стандартной библиотеки \CCpp) используется для поиска вхождений в строке.
Вот что мы сделаем:

\begin{lstlisting}[style=customc]
#include <string.h>
#include <stdio.h>

int main()
{
	char *s="Hello, world!";
	char *w=strstr(s, "world");

	printf ("%p, [%s]\n", s, s);
	printf ("%p, [%s]\n", w, w);
};
\end{lstlisting}

Вывод:

\begin{lstlisting}
0x8048530, [Hello, world!]
0x8048537, [world!]
\end{lstlisting}

Разница между адресом оригинальной строки и адресом подстроки, который вернула \IT{strstr()}, это 7.
Действительно, строка \q{Hello, } имеет длину в 7 символов.

Ф-ция \printf{} во время второго вызова не знает о том, что перед переданной строкой имеются еще какие-то символы,
и печатает символы с середины оригинальной строки, до конца (который обозначен нулевым байтом).

