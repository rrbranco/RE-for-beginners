\subsection{Computing arithmetic mean}

Let's imagine that we want to calculate \gls{arithmetic mean}, and for some weird reason 
we want to specify all the values as function arguments.

But it's impossible to get the number of arguments in a variadic function in \CCpp, so let's denote 
the value of $-1$ as a terminator.

\subsubsection{Using va\_arg macro}

There is the standard stdarg.h header file which define macros for dealing with such arguments.

The \printf and \scanf functions use them as well.

\lstinputlisting[style=customc]{\CURPATH/arith_EN.c}

The first argument has to be treated just like a normal argument.
\myindex{\CStandardLibrary!va\_arg}

All other arguments are loaded using the \TT{va\_arg} macro and then summed.

So what is inside?

\myparagraph{\IT{cdecl} calling conventions}

\lstinputlisting[caption=\Optimizing MSVC 6.0,style=customasmx86]{\CURPATH/arith_MSVC60_Ox_EN.asm}

The arguments, as we may see, are passed to \main one-by-one.

The first argument is pushed into the local stack as first.

The terminating value ($-1$) is pushed last.

The \TT{arith\_mean()} function takes the value of the first argument and stores it in the $sum$ variable.

Then, it sets the \EDX register to the address of the second argument, takes the value from it, adds it to $sum$,
and does this in an infinite loop, until $-1$ is found.

When it's found, the sum is divided by the number of all values (excluding $-1$) and the \gls{quotient} is returned.

So, in other words, the function treats the stack fragment as an array of integer values of infinite
length.

Now we can understand why the \IT{cdecl} calling convention forces us to push the first argument 
into the stack as last.

Because otherwise, it would not be possible to find the first argument, 
or, for printf-like functions, it would not be possible to find the address of the format-string.

\myparagraph{Register-based calling conventions}
\label{variadic_arith_registers}

The observant reader may ask, what about calling conventions where the first few arguments are passed in registers?
Let's see:

\lstinputlisting[caption=\Optimizing MSVC 2012 x64,style=customasmx86]{\CURPATH/arith_MSVC_2012_Ox_x64_EN.asm}

We see that the first 4 arguments are passed in the registers and two more---in the stack.

The \TT{arith\_mean()} function first places these 4 arguments into the \IT{Shadow Space} and then treats
the \IT{Shadow Space} and stack behind it as a single continuous array!

What about GCC? Things are slightly clumsier here, because now the function is divided in two parts:
the first part saves the registers into the \q{red zone}, processes that space, and the second part of the function processes 
the stack:

\lstinputlisting[caption=\Optimizing GCC 4.9.1 x64,style=customasmx86]{\CURPATH/arith_GCC491_x64_O3_EN.s}

By the way, a similar usage of the \IT{Shadow Space} is also considered here: \myref{pointer_to_argument}.

\subsubsection{Using pointer to the first function argument}

The example can be rewritten without \TT{va\_arg} macro:

\lstinputlisting[style=customc]{\CURPATH/arith2.c}

In other words, if an argument set is array of words (32-bit or 64-bit), we just enumerate array elements starting
at first one.

