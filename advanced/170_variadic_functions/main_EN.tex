\section{Variadic functions}

Functions like \printf and \scanf can have a variable number of arguments.
How are these arguments accessed?

% sections
\subsection{Computing arithmetic mean}

Let's imagine that we want to calculate \gls{arithmetic mean}, and for some weird reason 
we want to specify all the values as function arguments.

But it's impossible to get the number of arguments in a variadic function in \CCpp, so let's denote 
the value of $-1$ as a terminator.

\subsubsection{Using va\_arg macro}

There is the standard stdarg.h header file which define macros for dealing with such arguments.

The \printf and \scanf functions use them as well.

\lstinputlisting[style=customc]{\CURPATH/arith_EN.c}

The first argument has to be treated just like a normal argument.
\myindex{\CStandardLibrary!va\_arg}

All other arguments are loaded using the \TT{va\_arg} macro and then summed.

So what is inside?

\myparagraph{\IT{cdecl} calling conventions}

\lstinputlisting[caption=\Optimizing MSVC 6.0,style=customasmx86]{\CURPATH/arith_MSVC60_Ox_EN.asm}

The arguments, as we may see, are passed to \main one-by-one.

The first argument is pushed into the local stack as first.

The terminating value ($-1$) is pushed last.

The \TT{arith\_mean()} function takes the value of the first argument and stores it in the $sum$ variable.

Then, it sets the \EDX register to the address of the second argument, takes the value from it, adds it to $sum$,
and does this in an infinite loop, until $-1$ is found.

When it's found, the sum is divided by the number of all values (excluding $-1$) and the \gls{quotient} is returned.

So, in other words, the function treats the stack fragment as an array of integer values of infinite
length.

Now we can understand why the \IT{cdecl} calling convention forces us to push the first argument 
into the stack as last.

Because otherwise, it would not be possible to find the first argument, 
or, for printf-like functions, it would not be possible to find the address of the format-string.

\myparagraph{Register-based calling conventions}
\label{variadic_arith_registers}

The observant reader may ask, what about calling conventions where the first few arguments are passed in registers?
Let's see:

\lstinputlisting[caption=\Optimizing MSVC 2012 x64,style=customasmx86]{\CURPATH/arith_MSVC_2012_Ox_x64_EN.asm}

We see that the first 4 arguments are passed in the registers and two more---in the stack.

The \TT{arith\_mean()} function first places these 4 arguments into the \IT{Shadow Space} and then treats
the \IT{Shadow Space} and stack behind it as a single continuous array!

What about GCC? Things are slightly clumsier here, because now the function is divided in two parts:
the first part saves the registers into the \q{red zone}, processes that space, and the second part of the function processes 
the stack:

\lstinputlisting[caption=\Optimizing GCC 4.9.1 x64,style=customasmx86]{\CURPATH/arith_GCC491_x64_O3_EN.s}

By the way, a similar usage of the \IT{Shadow Space} is also considered here: \myref{pointer_to_argument}.

\subsubsection{Using pointer to the first function argument}

The example can be rewritten without \TT{va\_arg} macro:

\lstinputlisting[style=customc]{\CURPATH/arith2.c}

In other words, if an argument set is array of words (32-bit or 64-bit), we just enumerate array elements starting
at first one.


\subsection{\IT{vprintf()} function case}
\myindex{\CStandardLibrary!vprintf}
\myindex{\CStandardLibrary!va\_list}

Many programmers define their own logging functions which take a printf-like format string + 
a variable number of arguments.

Another popular example is the die() function, which prints some message and exits.

We need some way to pack input arguments of unknown number and pass them to the \printf function.
But how?

That's why there are functions with \q{v} in name.

One of them is \IT{vprintf()}: it takes a format-string and a pointer to a variable of type \TT{va\_list}:

\lstinputlisting[style=customc]{\CURPATH/die.c}

By closer examination, we can see that \TT{va\_list} is a pointer to an array.
Let's compile:

\lstinputlisting[caption=\Optimizing MSVC 2010,style=customasmx86]{\CURPATH/die_MSVC2010_Ox_EN.asm}

We see that all our function does is just taking a pointer to the arguments and
passing it to \IT{vprintf()}, and that function is treating it like an infinite array of arguments!

\lstinputlisting[caption=\Optimizing MSVC 2012 x64,style=customasmx86]{\CURPATH/die_MSVC2012_x64_Ox_EN.asm}


% TODO translate to Russian:
\subsection{Pin case}

\myindex{Pin}
It's interesting to note how some functions from Pin \ac{DBI} framework takes number of arguments:

\begin{lstlisting}[style=customc]
            INS_InsertPredicatedCall(
                ins, IPOINT_BEFORE, (AFUNPTR)RecordMemRead,
                IARG_INST_PTR,
                IARG_MEMORYOP_EA, memOp,
                IARG_END);
\end{lstlisting}
(pinatrace.cpp)

And this is how INS\_InsertPredicatedCall() is declared:

\begin{lstlisting}[style=customc]
extern VOID INS_InsertPredicatedCall(INS ins, IPOINT ipoint, AFUNPTR funptr, ...);
\end{lstlisting}
(pin\_client.PH)

Hence, constants starting with IARG\_ are some kinds of arguments to the function,
which are handled inside of INS\_InsertPredicatedCall().
You can pass as many arguments, as you need.
Some commands has additional argument(s), some are not.
Full list of arguments:
\url{https://software.intel.com/sites/landingpage/pintool/docs/58423/Pin/html/group__INST__ARGS.html}.
And it has to be a way to detect an end of arguments list, so the list must be terminated with IARG\_END, without which,
the function will (try to) handle random noise in the local stack as arguments.

Also, in [\RobPikePractice] we can find a nice example of C/C++ routines very similar to
pack/unpack\footnote{\url{https://docs.python.org/3/library/struct.html}} in Python.

% TODO translate
\subsection{Format string exploit}

It's a popular mistake, to write \TT{printf(string)} instead of \TT{puts(string)} or \TT{printf("\%s", string)}.
If the attacker can tamper with \TT{string}, he/she can crash process, or get insight into variables in the local stack.

Take a look at this:

\lstinputlisting[style=customc]{\CURPATH/f.c}

Please note, that \printf has no additional arguments besides single format string.

Now let's imagine, an attacker has put \TT{\%s} string into the last \printf call's arguments list.
I compile this example using GCC 5.4.0 on x86 Ubuntu, and the resulting executable prints \q{world} string!

If I turn optimization on, \printf outputs some garbage, though---probably, strcpy() calls has been optimized and/or
local variables as well.
Also, result will be different for x64 code, different compiler, \ac{OS}, etc.

Now, let's say, attacker could pass the following string to \printf call: \TT{\%x \%x \%x \%x \%x}.
In may case, output is: \q{80485c6 b7751b48 1 0 80485c0} (these are just values from local stack).
You see, there are 1 and 0 values, and some pointers (first is probably pointer to \q{world} string).
So if attacker passes \TT{\%s \%s \%s \%s \%s} string, the process will crash, because \printf treats 1 and/or 0
as pointer to string, tries to read characters from there and fails.

Even worse, there could be \TT{sprintf (buf, string)} in code, where \TT{buf} is a buffer in the local stack
with size of 1024 bytes or so, attacker can craft \TT{string} in such a way that \TT{buf} will be overflown,
maybe in even such a way, that would lead to code execution.

Many popular and well-known software was (or even still) vulnerable:

\myindex{Quake}
\myindex{John Carmack}
\begin{framed}
\begin{quotation}
QuakeWorld went up, got to around 4000 users, then the master server exploded.

Disrupter and cohorts are working on more robust code now.

If anyone did it on purpose, how about letting us know... (It wasn't all the people that tried \%s as a name)
\end{quotation}
\end{framed}
( John Carmack's .plan file, 17-Dec-1996\footnote{\url{https://github.com/ESWAT/john-carmack-plan-archive/blob/33ae52fdba46aa0d1abfed6fc7598233748541c0/by_day/johnc_plan_19961217.txt}} )

Nowadays, almost all decent compilers warns about this.

Another problem is lesser known \TT{\%n} \printf argument: whenever \printf reaches it in format string, it writes
number of characters printed so far into corresponding argument:
\url{http://stackoverflow.com/questions/3401156/what-is-the-use-of-the-n-format-specifier-in-c}.
Thus, attacker could zap local variables passing many \TT{\%n} commands in format string.



