\subsection{How it works}

From school-level mathematics, we can remember that division by 9 can be replaced by multiplication by $\frac{1}{9}$.
In fact, sometimes compilers do so for floating-point arithmetics, for example, \INS{FDIV} instruction in x86 code
can be replaced by \INS{FMUL}.
At least MSVC 6.0 will replace division by 9 by multiplication by $0.111111...$ and sometimes it's hard to be sure,
what operation was in original source code.

But when we operate over integer values and integer CPU registers, we can't use fractions.
However, we can rework fraction like that:

% FIXME: equation size
\begin{center}
$result = \frac{x}{9} = x \cdot \frac{1}{9} = x \cdot \frac{1 \cdot MagicNumber}{9 \cdot MagicNumber}$
\end{center}

Given the fact that division by $2^n$ is very fast (using shifts), we now should find that $MagicNumber$,
for which the following
equation will be true: $2^n = 9 \cdot MagicNumber$.

Division by $2^{32}$ is somewhat hidden: lower 32-bit of product in EAX is not used (dropped), only higher 32-bit of
product (in EDX) is used and then shifted by additional 1 bit.

In other words, the assembly code we have just seen multiplicates by {\Large $\frac{954437177}{2^{32+1}}$},
or divides by {\Large $\frac{2^{32+1}}{954437177}$}.
To find divisor we just have to divide numerator by denominator.
Using Wolfram Alpha, we can get 8.99999999.... as result (which is close to 9).

% TODO ref to https://yurichev.com/blog/signed_division_using_shifts/

Read more about it in \InSqBrackets{\HenryWarren 10-3}.

Couple of words about better understanding.
Many people miss ``hidden'' division by $2^{32}$ or $2^{64}$,
when lower 32-bit part (or 64-bit part) of product is not used.
Also, there is misconception that modulo inverse is used here. This is close, but not the same thing.
Extended Euclidean algorithm is usually used to find \textit{magic coefficient}, but in fact,
this algorithm is rather used to solve the equation. You can solve it using any other method.
For example, the author used Z3 SMT solver to find \textit{coefficient}\footnote{\url{https://yurichev.com/tmp/SAT_SMT_DRAFT.pdf}}.
This is overkill, but it was used it for demonstration.
Anyway, Extended Euclidean algorithm is probably the most efficient way to solve it.
Also, needless to mention, the equation is unsolvable for some divisors, because this is diophantine equation
(i.e., equation allowing result to be only integer), since we work on integer CPU registers, after all.

