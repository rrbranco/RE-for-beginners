\newcommand{\HashFuncChapterName}{Funciones hash}

\section{\HashFuncChapterName}
\label{hash_func}

\myindex{\HashFuncChapterName}
\myindex{CRC32}
Un ejemplo muy sencillo es CRC32, un algoritmo que provee una \q{fuerte} suma de
comprobaci\'on para prop\'ositos de comprobaci\'on de integridad.
Es imposible reestablecer el texto original a partir de su valor hash, tiene mucho menos informaci\'on:
la entrada puede ser larga, pero el resultado de CRC32 siempre est\'a limitado a 32 bits.
Pero CRC32 no es criptogr\'aficamente seguro: es sabido c\'omo alterar un texto de tal modo que el valor
del hash CRC32 resultante sea el que necesitemos.
Las funciones hash criptogr\'aficas est\'an protegidas de esto. \\
\\
\myindex{MD5}
\myindex{SHA1}
Tales funciones son MD5, SHA1, etc., y son utilizadas ampliamente para obtener el hash de contrase\~nas de usuarios para
almacenarlas en las bases de datos.
De hecho, la base de datos de un foro en internet no puede contener las contrase\~nas de los usuarios
(una base de datos robado puede comprometer las contrase\~nas de todos los usuarios) sino \'unicamente
hashes (un cracker no puede recuperar las contrase\~nas).
Adem\'as, el motor de un foro de internet no es consciente de tu contrase\~na, s\'olo debe comprobar
si su hash es el mismo que aquel almacenado en la base de datos, y darte acceso si concuerdan.
Uno de los m\'etodos de cracking de contrase\~nas m\'as simple es tratar de obtener los hashes de todas
las contrase\~nas posibles para ver cu\'al concuerda con el resultado que necesitamos.
Otros m\'etodos son mucho m\'as complejos.
% TODO1 add about Rainbow tables

\subsection{?`C\'omo trabajan las funciones de una v\'ia?}

Las funciones de una v\'ia son funciones capces de transformar un valor en otro,
a la vez que es imposible (o muy dif\'icil revertirlo.)
Algunas personas tienen dificultad entendiendo c\'omo puede ser esto posible.
Consideremos una demostraci\'on simple.

Tenemos un vector de 10 n\'umeros en el rango 0..9, cada una presente una sola vez, por ejemplo:

\begin{lstlisting}
4 6 0 1 3 5 7 8 9 2
\end{lstlisting}

El algoritmo de la funci\'on de una v\'ia m\'as simple es:

\begin{itemize}
\item toma el n\'umero en la posici\'on cero (4 en nuestro caso);
\item toma el n\'umero en la primera posici\'on (6 en nuestro caso);
\item intercambia los n\'umeros en las posiciones 4 y 6.
\end{itemize}

Marquemos los n\'umeros en las posiciones 4 y 6:

\begin{lstlisting}
4 6 0 1 3 5 7 8 9 2
        ^   ^
\end{lstlisting}

Intercambi\'emolos y tenemos el resultado:

\begin{lstlisting}
4 6 0 1 7 5 3 8 9 2
\end{lstlisting}

Mientras vemos el resultado, incluso si conocemos el algoritmo, no podemos enumerar sin ambig\"uedad el conjunto inicial
porque los primeros dos n\'umeros puedieron haber sido 0 y/o 1, y puedieron haber participado en el proceso de intercambio.

Este ejemplo fue demasiado simplificado para efectos de desmostraci\'on. Las funciones de una v\'ia reales pueden llegar a ser muy complejas.
