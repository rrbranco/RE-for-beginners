\section{CPU}

\subsection{Branch predictors}
\label{branch_predictors}

Some latest compilers try to get rid of conditional jump instructions.
Examples in this book are: \myref{subsec:jcc_ARM}, \myref{chap:cond}, \myref{subsec:popcnt}.

This is because the branch predictor is not always perfect, so the compilers try to do 
without conditional jumps, if possible.

\myindex{x86!\Instructions!CMOVcc}
\myindex{ARM!\Instructions!ADRcc}
Conditional instructions in ARM (like ADRcc) are one way, another one is the CMOVcc x86 instruction.

\subsection{Data dependencies}

Modern CPUs are able to execute instructions simultaneously (\ac{OOE}), but in order to do so,
the results of one instruction in a group must not influence the execution of others.
Hence, the compiler endeavors to use instructions with minimal influence on the CPU state.

\myindex{ARM!\Instructions!LEA}
That's why the \LEA instruction is so popular, because it does not modify CPU flags, while
other arithmetic instructions does.
