\section{Memory}

There are 3 main types of memory:

\begin{itemize}
\item
Global memory \ac{AKA} \q{static memory allocation}.
No need to allocate explicitly, the allocation is performed just by declaring variables/arrays 
globally.
These are global variables, residing in the data or constant segments.
They are available globally (hence, considered as an \gls{anti-pattern}).
Not convenient for buffers/arrays, because they must have a fixed size.
Buffer overflows that occur here usually overwrite variables or buffers residing next to them in memory.
There's an example in this book: \myref{scanf_global_variable}.

\item
Stack \ac{AKA} \q{allocate on stack}.
The allocation is performed just by declaring variables/arrays locally in the function.
These are usually local variables for the function.
Sometimes these local variable are also available to descending functions 
(to \gls{callee} functions, if caller passes a pointer to a variable to the \gls{callee} to be executed).
Allocation and deallocation are very fast, it just \ac{SP} needs to be shifted.
\myindex{\CStandardLibrary!alloca()}

But they're also not convenient for buffers/arrays, because the buffer size has to be fixed,
unless \TT{alloca()} (\myref{alloca}) (or a variable-length array) is used.
Buffer overflows usually overwrite important stack structures: \myref{subsec:bufferoverflow}.

\myindex{\CStandardLibrary!malloc()}
\myindex{\CStandardLibrary!free()}
\item
Heap \ac{AKA} \q{dynamic memory allocation}.
Allocation/deallocation is performed by calling \\
\TT{malloc()/free()} or \TT{new/delete} in \Cpp.
This is the most convenient method: the block size may be set at runtime.
\myindex{\CStandardLibrary!realloc()}

Resizing is possible (using \TT{realloc()}), but can be slow.
This is the slowest way to allocate memory: 
the memory allocator must support and update all control structures while
allocating and deallocating.
Buffer overflows usually overwrite these structures.
Heap allocations are also source of memory leak problems: each memory block has to be deallocated
explicitly, but one may forget about it, or do it incorrectly.
\myindex{\CStandardLibrary!free()}

Another problem is the \q{use after free}---using a memory block after \TT{free()} has been called on it,
which is very dangerous.

Example in this book: \myref{struct_malloc_example}.

\end{itemize}
