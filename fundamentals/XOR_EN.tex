\section{XOR (exclusive OR)}
\label{XOR_property}

\XOR is widely used when one needs just to flip specific bit(s).
Indeed, the \XOR operation applied with 1 effectively inverts a bit:

\begin{center}
\begin{tabular}{ | l | l | l | }
\hline
\HeaderColor \MLinputA & \HeaderColor \MLinputB & \HeaderColor \MLoutput \\
\hline
0 & 0 & 0 \\
\hline
{\color{red} 0} & {\color{red} 1} & {\color{red} 1} \\
\hline
{\color{red} 1} & {\color{red} 0} & {\color{red} 1} \\
\hline
1 & 1 & 0 \\
\hline
\end{tabular}
\end{center}


And vice-versa, the \XOR operation applied with 0 does nothing, i.e., it's an idle operation.
This is a very important property of the \XOR operation and it's highly recommended to memorize it.



\subsection{Everyday speech}

XOR operation present in common everyday speech.
When someone asks ``please buy apples or bananas'',
this usually means ``buy the first object or the second, but not both''---this is exactly exclusive OR,
because logical OR would mean ``both objects are also fine''.

Some people suggest ``and/or'' should be used in everyday speech to make emphasis that logical OR is used instead of
exclusive OR: \url{https://en.wikipedia.org/wiki/And/or}.

\subsection{Encryption}

XOR is heavily used in both amateur (\ref{simple_XOR_encryption}) and \IT{real} encryption (at least in \IT{Feistel network}).

XOR is very useful here because:
$cipher\_text = plain\_text \oplus key$ and then:
$(plain\_text \oplus key) \oplus key = plain\_text$.

\subsection{\ac{RAID}4}
\index{RAID4}

\ac{RAID}4 offers a very simple method to to protect hard disks.
For example, there are several disks ($D_1$, $D_2$, $D_3$, etc.) and one parity disk ($P$).
Each bit/byte written to parity disk is calculated and written on-fly:

\begin{equation} \label{eq:RAID4}
P = D_1 \oplus D_2 \oplus D_3
\end{equation}

If any of disks is failed, for example, $D_2$, it's restored using the very same way:

\begin{equation}
D_2 = D_1 \oplus P \oplus D_3
\end{equation}

If parity disk failed, it is restored using \ref{eq:RAID4} way.
If two of any disks are failed, then it wouldn't be possible to restore both.

\ac{RAID}5 is more advanced, but this XOR property is still exploited there.

That's why \ac{RAID} controllers has hardware ``XOR accelerators'' helping to XOR large chunks of written data on-fly.
When computers get faster and faster, it now can be done at software level, using \ac{SIMD}.

\subsection{XOR swap algorithm}

Hard to believe, but this code swaps values in \EAX and \EBX without aid of any other additional register or memory cell:

\begin{lstlisting}[style=customasmx86]
xor eax, ebx
xor ebx, eax
xor eax, ebx
\end{lstlisting}

Let's find out, how it works.
First, we will rewrite it to step aside from x86 assembly language:

\begin{lstlisting}
X = X XOR Y
Y = Y XOR X
X = X XOR Y
\end{lstlisting}

What X and Y has at each step?
Just keep in mind the simple rule: $(X \oplus Y) \oplus Y = X$ for any values of X and Y.

Let's see,
$X$ after 1st step has $X \oplus Y$;
$Y$ after 2nd step has $Y \oplus (X \oplus Y) = X$;
$X$ after 3rd step has $(X \oplus Y) \oplus X = Y$.

Hard to say if anyone should use this trick, but it servers as a good demonstration example of XOR properties.

Wikipedia article (\url{https://en.wikipedia.org/wiki/XOR_swap_algorithm}) has also yet another explanation:
addition and subtraction operations can be used instead of XOR:

\begin{lstlisting}
X = X + Y
Y = X - Y
X = X - Y
\end{lstlisting}

Let's see:
$X$ after 1st step has $X+Y$;
$Y$ after 2nd step has $X+Y-Y=X$;
$X$ after 3rd step has $X+Y-X=Y$.

\subsection{XOR linked list}
\index{Doubly linked list}

Doubly linked list is a list in which each element has link to the previous element and to the next one.
Hence, it's very easy to traverse list backwards or forward.
\TT{std::list} in C++ implements doubly linked list which also is examined in this book: \ref{std_list}.

So each element has two pointers.
Is it possible, perhaps in environment of small \ac{RAM} footprint, to preserve all functionality with one pointer instead of two?
Yes, if it a value of $prev \oplus next$ will be stored in this memory cell, which is usually called ``link''.

Maybe, we could say that address to the previous element is ``encrypted'' using address of next element and otherwise:
next element address is ``encrypted'' using previous element address.

When we traverse this list forward, we always know address of the previous element, so we can ``decrypt'' this field and get 
address of the next element.
Likewise, it's possible to traverse this list backwards, ``decrypting'' this field using next element's address.

But it's not possible to find address of previous or next element of some specific element without
knowing address of the first one.

Couple of things to complete this solution: first element will have address of next element without any XOR-ing,
last element will have address of previous element without any XOR-ing.

Now let's sum it up. This is example of doubly linked list of 5 elements.
$A_x$ is address of element.

\begin{center}
\begin{tabular}{ | l | l | }
	\hline
	\HeaderColor address & \HeaderColor \IT{link} field contents \\
	\hline
	$A_0$ & $A_1$ \\
	\hline
	$A_1$ & $A_0 \oplus A_2$ \\
	\hline
	$A_2$ & $A_1 \oplus A_3$ \\
	\hline
	$A_3$ & $A_2 \oplus A_4$ \\
	\hline
	$A_4$ & $A_3$ \\
	\hline
\end{tabular}
\end{center}

And again, hard to say if anyone should use this tricky hacks, but this is also a good demonstration of XOR properties.
As with XOR swap algorithm, Wikipedia article about it also offers way to use addition or subtraction instead of XOR:
\url{https://en.wikipedia.org/wiki/XOR_linked_list}.

\subsection{By the way}

The usual \IT{OR} also sometimes called \IT{inclusive OR} (or even \IT{IOR}), as opposed to \IT{exclusive OR}.

