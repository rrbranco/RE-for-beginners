\section{\SignedNumbersSectionName}
\label{sec:signednumbers}
\myindex{Signed numbers}

\newcommand{\URLS}{\href{http://go.yurichev.com/17117}{wikipedia}}

Методов представления чисел с знаком \q{плюс} или \q{минус} несколько\footnote{\URLS}, 
но в компьютерах обычно применяется метод \q{дополнительный код} или \q{two's complement}.

Вот таблица некоторые значений байтов:

\label{signed_tbl}
\begin{center}
\begin{tabular}{ | l | l | l | l | }
\hline
\HeaderColor двоичное & \HeaderColor шестнадцатеричное & \HeaderColor беззнаковое & \HeaderColor знаковое \\
\hline
01111111 & 0x7f & 127 & 127 \\
\hline
01111110 & 0x7e & 126 & 126 \\
\hline
\multicolumn{4}{ |c| }{...} \\
\hline
00000110 & 0x6 & 6 & 6 \\
\hline
00000101 & 0x5 & 5 & 5 \\
\hline
00000100 & 0x4 & 4 & 4 \\
\hline
00000011 & 0x3 & 3 & 3 \\
\hline
00000010 & 0x2 & 2 & 2 \\
\hline
00000001 & 0x1 & 1 & 1 \\
\hline
00000000 & 0x0 & 0 & 0 \\
\hline
11111111 & 0xff & 255 & -1 \\
\hline
11111110 & 0xfe & 254 & -2 \\
\hline
11111101 & 0xfd & 253 & -3 \\
\hline
11111100 & 0xfc & 252 & -4 \\
\hline
11111011 & 0xfb & 251 & -5 \\
\hline
11111010 & 0xfa & 250 & -6 \\
\hline
\multicolumn{4}{ |c| }{...} \\
\hline
10000010 & 0x82 & 130 & -126 \\
\hline
10000001 & 0x81 & 129 & -127 \\
\hline
10000000 & 0x80 & 128 & -128 \\
\hline
\end{tabular}
\end{center}

\myindex{x86!\Instructions!JA}
\myindex{x86!\Instructions!JB}
\myindex{x86!\Instructions!JL}
\myindex{x86!\Instructions!JG}
Разница в подходе к знаковым/беззнаковым числам, собственно, нужна потому что, например, 
если представить \TT{0xFFFFFFFE} и \TT{0x00000002} как беззнаковое, то первое число (4294967294) больше второго (2). 
Если их оба представить как знаковые, то первое будет $-2$, которое, разумеется, меньше чем второе (2).
Вот почему инструкции для условных переходов~(\myref{sec:Jcc}) представлены в обоих версиях ~--- 
и для знаковых сравнений (например, \JG, \JL) и для беззнаковых (\JA, \JB).

Для простоты, вот что нужно знать:

\begin{itemize}
\item Числа бывают знаковые и беззнаковые.

\item Знаковые типы в \CCpp:

  \begin{itemize}
    \item \TT{int64\_t} (-9,223,372,036,854,775,808 .. 9,223,372,036,854,775,807) 
	  (-~9.2..~9.2 квинтиллионов) или \\
                \TT{0x8000000000000000..0x7FFFFFFFFFFFFFFF}),
    \item \Tint (-2,147,483,648..2,147,483,647 (-~2.15..~2.15Gb) или \\
	    \TT{0x80000000..0x7FFFFFFF}),
    \item \Tchar (-128..127 или \TT{0x80..0x7F}),
    \item \TT{ssize\_t}.
   \end{itemize}

	Беззнаковые:
  \begin{itemize}
	  \item \TT{uint64\_t} (0..18,446,744,073,709,551,615 
		  (~18 квинтиллионов) или \TT{0..0xFFFFFFFFFFFFFFFF}),
   \item \TT{unsigned int} (0..4,294,967,295 (~4.3Gb) или \TT{0..0xFFFFFFFF}),
   \item \TT{unsigned char} (0..255 или \TT{0..0xFF}), 
   \item \TT{size\_t}.
  \end{itemize}

\item У знаковых чисел знак определяется \ac{MSB}: 1 означает \q{минус}, 0 означает \q{плюс}.

\item Преобразование в б\'{о}льшие типы данных обходится легко:

	\myref{subsec:sign_extending_32_to_64}.

\label{sec:signednumbers:negation}
\item Изменить знак легко: просто инвертируйте все биты и прибавьте 1.
Мы можем заметить, что число другого знака находится на другой стороне на том же расстоянии от нуля.
Прибавление единицы необходимо из-за присутствия нуля посредине.

\myindex{x86!\Instructions!IDIV}
\myindex{x86!\Instructions!DIV}
\myindex{x86!\Instructions!IMUL}
\myindex{x86!\Instructions!MUL}
\myindex{x86!\Instructions!CBW}
\myindex{x86!\Instructions!CWD}
\myindex{x86!\Instructions!CWDE}
\myindex{x86!\Instructions!CDQ}
\myindex{x86!\Instructions!CDQE}
\myindex{x86!\Instructions!MOVSX}
\myindex{x86!\Instructions!SAR}
\item Инструкции сложения и вычитания работают одинаково хорошо и для знаковых и для беззнаковых значений.
	Но для операций умножения и деления, в x86 имеются разные инструкции:
	\TT{IDIV}/\TT{IMUL} для знаковых и \TT{DIV}/\TT{MUL} для беззнаковых.

\item Еще инструкции работающие с знаковыми числами:\\
	\TT{CBW/CWD/CWDE/CDQ/CDQE} (\myref{ins:CBW_CWD_etc}), \TT{MOVSX} (\myref{MOVSX}), \TT{SAR} (\myref{ins:SAR}).
\end{itemize}

Таблица некоторых отрицательных и положительных значений (\ref{signed_tbl}) напоминает термометр со шкалой по Цельсию.
Вот почему сложение и вычитание работает одинаково хорошо и для знаковых и беззнаковых чисел:
если первое слагаемое представить как отметку на термометре, и нужно прибавить второе слагаемое,
и оно положительне, то мы просто поднимаем отметку вверх на значение второго слагаемого.
Если второе слагаемое отрицательное, то мы опускаем отметку вниз на абсолютное значение от второго слагаемого.

Сложение двух отрицательных чисел работает так.
Например, нужно сложить -2 и -3 используя 16-битные регистры.
-2 и -3 это 0xfffe и 0xfffd соответственно.
Если сложить эти два числа как беззнаковые, то получится 0xfffe+0xfffd=0x1fffb.
Но мы работаем с 16-битными регистрами, так что результат \IT{обрезается}, первая единица выкидывается,
остается 0xfffb, а это -5.
Это работает потому что -2 (или 0xfffe) можно описать простым русским языком так:
``в этом значении не достает двух до максимального значения в регистре + 1''.
-3 можно описать ``\dots не достает трех до \dots''.
Максимальное значение 16-битного регистра + 1 это 0x10000.
При складывании двух чисел, и \IT{обрезании} по модулю $2^{16}$, \IT{не хватать} будет $2+3=5$.

% subsections:
\subsection{Использование IMUL вместо MUL}
\label{IMUL_over_MUL}

\myindex{x86!\Instructions!MUL}
\myindex{x86!\Instructions!IMUL}
В примере вроде \lstref{unsigned_multiply_C} где умножаются два беззнаковых значения, компилируется в
\lstref{unsigned_multiply_lst}, где используется \IMUL вместо \MUL.

Это важное свойство обоих инструкций: \MUL и \IMUL{}.
Прежде всего, они обе выдают 64-битное значение если перемножаются два 32-битных, либо же 128-битное значение,
если перемножаются два 64-битных (максимальное возможное \glslink{product}{произведение} в 32-битное среде это \\
\GTT{0xffffffff*0xffffffff=0xfffffffe00000001}).
Но в стандарте \CCpp нет способа доступиться к старшей половине результата, а \glslink{product}{произведение} всегда имеет
тот же размер, что и множители. % TODO \gls{}?
И обе инструкции \MUL и \IMUL работают одинаково, если старшая половина результата игнорируется, т.е., обе инструкции
генерируют одинаковую младшую половину.
Это важное свойство способа представления знаковых чисел \q{дополнительный код}.

Так что компилятор с \CCpp может использовать любую из этих инструкций.

Но \IMUL более гибкая чем \MUL, потому что она может брать любой регистр как вход, в то время как \MUL требует,
чтобы один из множителей находился в регистре \AX/\EAX/\RAX.
И даже более того: \MUL сохраняет результат в паре \GTT{EDX:EAX} в 32-битной среде, либо в \GTT{RDX:RAX} в 64-битной,
так что она всегда вычисляет полный результат.
И напротив, в \IMUL можно указать единственный регистр назначения вместо пары, тогда \ac{CPU} будет вычислять только
младшую половину, а это быстрее
[см. Torborn Granlund, \IT{Instruction latencies and throughput for AMD and Intel x86 processors}\footnote{\url{http://yurichev.com/mirrors/x86-timing.pdf}]}).

Учитывая это, компиляторы \CCpp могут генерировать инструкцию \IMUL чаще, чем \MUL.

\myindex{Compiler intrinsic}
Тем не менее, используя \IT{compiler intrinsic}, можно произвести беззнаковое умножение и получить \IT{полный} результат.
Иногда это называется \IT{расширенное умножение} (\IT{extended multiplication}).
MSVC для этого имеет \IT{intrinsic}, которая называется \IT{\_\_emul}\footnote{\url{https://msdn.microsoft.com/en-us/library/d2s81xt0(v=vs.80).aspx}} и еще одну: \IT{\_umul128}\footnote{\url{https://msdn.microsoft.com/library/3dayytw9%28v=vs.100%29.aspx}}.
GCC предлагает тип \IT{\_\_int128}, и если 64-битные множители вначале приводятся к 128-битным,
затем \glslink{product}{произведение} сохраняется в другой переменной \IT{\_\_int128}, затем результат сдвигается на 64 бита
право, вы получаете старшую часть результата\footnote{Например: \url{http://stackoverflow.com/a/13187798}}.

\subsubsection{Функция MulDiv() в Windows}
\myindex{Windows!Win32!MulDiv()}

В Windows есть ф-ция MulDiv()
\footnote{\url{https://msdn.microsoft.com/en-us/library/windows/desktop/aa383718(v=vs.85).aspx}},
это ф-ция производящая одновременно умножение и деление, она в начале перемножает два 32-битных числа и получает
промежуточное 64-битное значение, а затем делит его на третье 32-битное значение.
Это проще чем использовать две \IT{compiler intrinsic}, так что разработчики в Microsoft решили сделать специальную ф-цию
для этого.
И судя по использованию оной, она достаточно востребована.


\subsection{Еще кое-что о дополнительном коде}

\subsubsection{Получение максимального числа для некоторого \glslink{word}{слова}}

Максимальное беззнаковое число это просто число, где все биты выставлены: \IT{0xFF....FF}
(это -1 если \glslink{word}{слово} используется как знаковое целочисленное).
Так что берете \glslink{word}{слово}, и выставляете все биты для получения значения:

\begin{lstlisting}[style=customc]
#include <stdio.h>

int main()
{
	unsigned int val=~0; // замените на "unsigned char" чтобы узнать макс.значение для беззнакового 8-битного байта
	// 0-1 также сработает, или просто -1
	printf ("%u\n", val); // %u для беззнакового
};
\end{lstlisting}

Для 32-битного целочисленного это 4294967295.

\subsubsection{Получение минимального числа для некоторого знакового \glslink{word}{слова}}

Минимальное знаковое число кодируется как \IT{0x80....00}, т.е., самый старший бит выставлен, остальные сброшены.
Максимальное знаковое число кодируется также, только все биты инвертированы: \IT{0x7F....FF}.

Будем сдвигать один бит влево, пока он не исчезнет:

\begin{lstlisting}[style=customc]
#include <stdio.h>

int main()
{
	signed int val=1; // замените на "signed char" чтобы найти значения для знакового байта
	while (val!=0)
	{
		printf ("%d %d\n", val, ~val);
		val=val<<1;
	};
};
\end{lstlisting}

Результат:

\begin{lstlisting}
...

536870912 -536870913
1073741824 -1073741825
-2147483648 2147483647
\end{lstlisting}

Два последних числа это соответственно минимум и максимум для знакового 32-битного \IT{int}.



