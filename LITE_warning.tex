\begin{center}
\vspace*{\fill}

\Huge%
	\RU{Внимание: это сокращенная LITE-версия!}%
	\EN{Warning: this is a shortened LITE-version!}%
	\ES{!`Atenci\'on: \'esta es una versi\'on LITE resumida!}%
	\PTBR{Atenc\~ao: esta \'e uma vers\~ao resumida!}%
	\DEph{}%
	\PLph{}%
	\ITAph{}
\normalsize

\bigskip
\bigskip
\bigskip

\Large
\RU{Она примерно в 6 раз короче полной версии (\textasciitilde{}150 страниц) и предназначена для тех,
кто хочет краткого введения в основы reverse engineering.
Здесь нет ничего о MIPS, ARM, OllyDBG, GCC, GDB, IDA, нет задач, примеров, \etc{}.}%
\EN{It is approximately 6 times shorter than full version (\textasciitilde{}150 pages) and intended to those
who want a very quick introduction to reverse engineering basics.
There is nothing about MIPS, ARM, OllyDBG, GCC, GDB, IDA, there are no exercises, examples, \etc{}.}%
\ES{Es apr\'oximadamente 6 veces m\'as corta que la versi\'on completa (\textasciitilde{}150 p\'aginas) y est\'a dirigida
a aquellos que deseen una introducci\'on breve a la esencia de la ingenier\'ia inversa.
No incluye nada sobre MIPS, ARM, OllyDBG, GCC, GDB, IDA, no contiene ejercicios, ejemplos, \etc{}.}%
\PTBR{Est\'a vers\~ao \'e aproximadamente 6  vezes menor que a vers\~ao completa (\textasciitilde{}150 p\'aginas) e dirigida aqueles que querem uma r\'apida introdu\,c\~ao ao b\'asico da engenharia reversa. N\~ao h\'a nada sobre MIPS, ARM, OllyDBG, GCC, GDB, IDA, sem exerc\'icios ou exemplos \etc{}.}%
\DEph{}%
\PLph{}%
\ITAph{}
\normalsize

\bigskip
\bigskip
\bigskip

\RU{Если вам всё ещё интересен reverse engineering, полная версия книги всегда доступна на моем сайте:}%
\EN{If you're still interested in reverse engineering, full version of the book is always available on my website:}%
\ES{Si a\'un est\'as interesado en la ingenier\'ia inversa, la versi\'on completa del libro siempre est\'a disponible en mi sitio:}%
\PTBR{Se voc\^e continua interessado em engenharia reversa, a vers\~ao completa deste livro esta dispon\'ivel no meu site:}%
\DEph{}%
\PLph{}%
\ITAph{}
\href{http://go.yurichev.com/17009}{beginners.re}.

\vspace*{\fill}
\vfill
\end{center}
