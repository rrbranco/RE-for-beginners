\subsection{\RU{Инструкции}\EN{Instructions}}

\RU{В ARM имеется также для некоторых инструкций суффикс \IT{-S}, указывающий, 
что эта инструкция будет модифицировать флаги.}
\EN{There is a \IT{-S} suffix for some instructions in ARM,
indicating that the instruction sets the flags according to the result.}
\RU{Инструкции без этого суффикса не модифицируют флаги.}
\EN{Instructions which lacks this suffix are not modify flags.}
\myindex{ARM!\Instructions!ADD}
\myindex{ARM!\Instructions!ADDS}
\myindex{ARM!\Instructions!CMP}
\RU{Например, инструкция}\EN{For example} \TT{ADD} \RU{в отличие от}\EN{unlike} \TT{ADDS}
\RU{сложит два числа, но флаги не изменит}
\EN{will add two numbers, but the flags will not be touched}.
\RU{Такие инструкции удобно использовать
между \CMP где выставляются флаги и, например, инструкциями перехода, где флаги используются.}
\EN{Such instructions are convenient to use between \CMP where the flags are set and, 
e.g. conditional jumps, where the flags are used.}
\EN{They are also better in terms of data dependency analysis 
(because less number of registers are modified during execution).}
\RU{Они также лучше в смысле анализа зависимостей данных (data dependency analysis) 
(потому что меньшее количество регистров модифицируется во время исполнения).}

% ADD
% ADDAL
% ADDCC
% ADDS
% ADR
% ADREQ
% ADRGT
% ADRHI
% ADRNE
% ASRS
% B
% BCS
% BEQ
% BGE
% BIC
% BL
% BLE
% BLEQ
% BLGT
% BLHI
% BLS
% BLT
% BLX
% BNE
% BX
% CMP
% IDIV
% IT
% LDMCSFD
% LDMEA
% LDMED
% LDMFA
% LDMFD
% LDMGEFD
% LDR.W
% LDR
% LDRB.W
% LDRB
% LDRSB
% LSL.W
% LSL
% LSLS
% MLA
% MOV
% MOVT.W
% MOVT
% MOVW
% MULS
% MVNS
% ORR
% POP
% PUSH
% RSB
% SMMUL
% STMEA
% STMED
% STMFA
% STMFD
% STMIA
% STMIB
% STR
% SUB
% SUBEQ
% SXTB
% TEST
% TST
% VADD
% VDIV
% VLDR
% VMOV
% VMOVGT
% VMRS
% VMUL
%\myindex{ARM!Optional operators!ASR
%\myindex{ARM!Optional operators!LSL
%\myindex{ARM!Optional operators!LSR
%\myindex{ARM!Optional operators!ROR
%\myindex{ARM!Optional operators!RRX

% AArch64
% RET is BR X30 or BR LR but with additional hint to CPU

\subsubsection{\RU{Таблица условных кодов}\EN{Conditional codes table}}

% TODO rework this!
\small
\begin{center}
\begin{tabular}{ | l | l | l | }
\hline
\HeaderColor \RU{Код}\EN{Code} & 
\HeaderColor \RU{Описание}\EN{Description} & 
\HeaderColor \RU{Флаги}\EN{Flags} \\
\hline
EQ & \EN{Equal}\RU{равно} & Z == 1 \\
\hline
NE & \EN{Not equal}\RU{не равно} & Z == 0 \\
\hline
CS \ac{AKA} HS (Higher or Same) & \EN{Carry set}\RU{перенос} / \EN{Unsigned, Greater than, equal}\RU{беззнаковое, больше или равно} & C == 1 \\
\hline
CC \ac{AKA} LO (LOwer) & \EN{Carry clear}\RU{нет переноса} / \EN{Unsigned, Less than}\RU{беззнаковое, меньше чем} & C == 0 \\
\hline
MI & \EN{Minus, negative}\RU{минус, отрицательный знак} / \EN{Less than}\RU{меньше чем} & N == 1 \\
\hline
PL & \EN{Plus, positive or zero}\RU{плюс, положительный знак или ноль} / \EN{Greater than, equal}\RU{больше чем или равно} & N == 0 \\
\hline
VS & \EN{Overflow}\RU{переполнение} & V == 1 \\
\hline
VC & \EN{No overflow}\RU{нет переполнения} & V == 0 \\
\hline
HI & \EN{Unsigned higher}\RU{беззнаковое, больше чем} / \EN{Greater than} & C == 1 \AndENRU \\
 & & Z == 0 \\
\hline
LS & \EN{Unsigned lower or same}\RU{беззнаковое, меньше или равно} / \EN{Less than or equal} & C == 0 \OrENRU \\
 & & Z == 1 \\
\hline
GE & \EN{Signed greater than or equal}\RU{знаковое, больше чем или равно} / \EN{Greater than or equal} & N == V \\
\hline
LT & \EN{Signed less than}\RU{знаковое, меньше чем} / \EN{Less than} & N != V \\
\hline
GT & \EN{Signed greater than}\RU{знаковое, больше чем} / \EN{Greater than} & Z == 0 \AndENRU \\
 & & N == V \\
\hline
LE & \EN{Signed less than or equal}\RU{знаковое, меньше чем или равно} / \EN{Less than, equal} & Z == 1 \OrENRU \\
 & & N != V \\
\hline
None / AL & \RU{Всегда}\EN{Always} & \RU{Любые}\EN{Any} \\
\hline
\end{tabular}
\end{center}
\normalsize

