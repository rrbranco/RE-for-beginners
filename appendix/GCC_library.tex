\section{\RU{Некоторые библиотечные функции GCC}\EN{Some GCC library functions}\DE{Einige GCC-Bibliotheks-Funktionen}}
\myindex{GCC}
\label{sec:GCC_library_func}

%__ashldi3
%__ashrdi3
%__floatundidf
%__floatdisf
%__floatdixf
%__floatundidf
%__floatundisf
%__floatundixf
%__lshrdi3
%__muldi3

\begin{center}
\begin{tabular}{ | l | l | }
\hline
\HeaderColor \RU{имя}\EN{name}\DE{Name} & \HeaderColor \RU{значение}\EN{meaning}\DE{Bedeutung} \\
\hline \TT{\_\_divdi3} & \RU{знаковое деление}\EN{signed division}\DE{vorzeichenbehaftete Division} \\
\hline \TT{\_\_moddi3} & \RU{остаток от знакового деления}\EN{getting remainder (modulo) of signed division}\DE{Rest (Modulo) einer vorzeichenbehafteten Division}\\
\hline \TT{\_\_udivdi3} & \RU{беззнаковое деление}\EN{unsigned division}\DE{vorzeichenlose Division} \\
\hline \TT{\_\_umoddi3} & \RU{остаток от беззнакового деления}\EN{getting remainder (modulo) of unsigned division}\DE{Rest (Modulo) einer vorzeichenlosen Division} \\
\hline
\end{tabular}
\end{center}

