\section{Intrinsische Compiler-Funktionen}
\myindex{Compiler intrinsic}
\label{sec:compiler_intrinsic}

\myindex{x86!\Instructions!ROL}
\myindex{x86!\Instructions!ROR}

Dabei handelt es sich um spezielle Funktionen eines Compilers, die nicht in der
Standard-Bibliothek enthalten sind.
Der Compiler generiert einen spezifischen Maschinencode anstatt ihn aufzurufen.
Dies ist häufig eine Pseudofunktion für eine spezielle \ac{CPU}-Anweisung.

Beispielsweise gibt es keine zyklische Schiebe-Anweisungen in \CCpp -Sprachen,
in den meisten \ac{CPU}s sind sie jedoch vorhanden.
Um dem Programmierer das Leben einfacher zu machen hat zumindest MSVC die
Pseudofunktionen \IT{\_rotl()} und \IT{\_rotr()}\FNMSDNROTxURL{} welche vom
Compiler direkt in die ROL/ROR x86-Anweisungen übersetzt werden.

Ein anderes Beispiel sind Funktionen die SSE-Anweisungen direkt im Code umwandeln.

Eine vollständige Liste von intrinsischen Funktionen in MSVC ist hier zu finden:
\href{http://go.yurichev.com/17254}{MSDN}.
