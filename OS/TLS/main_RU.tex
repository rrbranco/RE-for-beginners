\section{Thread Local Storage}
\label{TLS}
\myindex{TLS}

Это область данных, отдельная для каждого треда. Каждый тред может хранить там то, что ему нужно.
Один из известных примеров, это стандартная глобальная переменная в Си \IT{errno}. 
Несколько тредов одновременно могут вызывать функции возвращающие код ошибки в \IT{errno}, поэтому глобальная переменная здесь не будет работать корректно, 
для мультитредовых программ \IT{errno} нужно хранить в в \ac{TLS}. \\
\\
\myindex{\Cpp!C++11}
В C++11 ввели модификатор \IT{thread\_local}, показывающий, что каждый тред будет иметь свою версию этой переменной, и её можно инициализировать, и она расположена в \ac{TLS}
\footnote{\myindex{C11} В C11 также есть поддержка тредов, хотя и опциональная}:

\begin{lstlisting}[caption=C++11,style=customc]
#include <iostream>
#include <thread>

thread_local int tmp=3;

int main()
{
	std::cout << tmp << std::endl;
};
\end{lstlisting}

Компилируется в MinGW GCC 4.8.1, но не в MSVC 2012.

Если говорить о PE-файлах, то в исполняемом файле значение \IT{tmp} будет размещено именно в секции отведенной \ac{TLS}.

\subsection{Вернемся к линейному конгруэнтному генератору}
\label{LCG_TLS}

Рассмотренный ранее \myref{LCG_simple} генератор псевдослучайных чисел имеет недостаток:
он не пригоден для многопоточной среды, потому что переменная его внутреннего состояния может быть
прочитана и/или модифицирована в разных потоках одновременно.

% subsections

\subsubsection{Win32}

\myparagraph{Неинициализированные данные в \ac{TLS}}

Одно из решений --- это добавить модификатор \TT{\_\_declspec( thread )} к глобальной переменной, 
и теперь она будет выделена в \ac{TLS} (строка 9):

\lstinputlisting[numbers=left,style=customc]{OS/TLS/win32/rand_uninit.c}

Hiew показывает что в исполняемом файле теперь есть новая PE-секция: \TT{.tls}.
% TODO1 hiew screenshot?

\lstinputlisting[caption=\Optimizing MSVC 2013 x86,style=customasmx86]{OS/TLS/win32/rand_x86_uninit.asm}

\TT{rand\_state} теперь в \ac{TLS}-сегменте и у каждого потока есть своя версия этой переменной.

Вот как к ней обращаться: загрузить адрес \ac{TIB} из FS:2Ch, затем прибавить дополнительный индекс 
(если нужно), затем вычислить адрес \ac{TLS}-сегмента.

Затем можно обращаться к переменной \TT{rand\_state} через регистр ECX, который указывает на свою
область в каждом потоке.

\myindex{x86!\Registers!FS}
Селектор \TT{FS:} знаком любому reverse engineer-у, он всегда указывает на \ac{TIB}, чтобы всегда можно было
загружать данные специфичные для текущего потока.

\myindex{x86!\Registers!GS}
В Win64 используется селектор \TT{GS:} и адрес \ac{TLS} теперь 0x58:

\lstinputlisting[caption=\Optimizing MSVC 2013 x64,style=customasmx86]{OS/TLS/win32/rand_x64_uninit.asm}

\myparagraph{Инициализированные данные в \ac{TLS}}

Скажем, мы хотим, чтобы в переменной \TT{rand\_state} в самом начале было какое-то значение, 
и если программист забудет инициализировать генератор, то \TT{rand\_state} все же будет инициализирована какой-то
константой (строка 9):

\lstinputlisting[numbers=left,style=customc]{OS/TLS/win32/rand_init.c}

Код ничем не отличается от того, что мы уже видели, но вот что мы видим в IDA:

\lstinputlisting[style=customasmx86]{OS/TLS/win32/rand_init_IDA.lst}

Там 1234 и теперь, во время запуска каждого нового потока, новый \ac{TLS} будет выделен для нового потока,
и все эти данные, включая 1234, будут туда скопированы.

Вот типичный сценарий:

\begin{itemize}
\item Запустился поток А. \ac{TLS} создался для него, 1234 скопировалось в \TT{rand\_state}.

\item Функция \TT{my\_rand()} была вызвана несколько раз в потоке А. \\
\TT{rand\_state} теперь содержит что-то неравное 1234.

\item Запустился поток Б. \ac{TLS} создался для него, 1234 скопировалось в \TT{rand\_state}, 
а в это же время, поток А имеет какое-то другое значение в этой переменной.
\end{itemize}

\myparagraph{\ac{TLS}-коллбэки}
\myindex{TLS!Коллбэки}

Но что если переменные в \ac{TLS} должны быть установлены в значения, которые должны быть подготовлены
каким-то необычным образом?

Скажем, у нас есть следующая задача:
программист может забыть вызвать функцию \TT{my\_srand()} для инициализации \ac{PRNG}, но генератор должен быть
инициализирован на старте чем-то по-настоящему случайным а не 1234.

Вот случай где можно применить \ac{TLS}-коллбэки.

Нижеследующий код не очень портабельный из-за хака, но тем не менее, вы поймете идею.

Мы здесь добавляем функцию (\TT{tls\_callback()}), которая вызывается \IT{перед} стартом процесса и/или потока.

Функция будет инициализировать \ac{PRNG} значением возвращенным функцией \TT{GetTickCount()}.

\lstinputlisting[style=customc]{OS/TLS/win32/rand_TLS_callback.c}

Посмотрим в IDA:

\lstinputlisting[caption=\Optimizing MSVC 2013,style=customasmx86]{OS/TLS/win32/rand_TLS_callback.lst}

TLS-коллбэки иногда используются в процедурах распаковки для запутывания их работы.

Некоторые люди могут быть в неведении что какой-то код уже был исполнен прямо перед \ac{OEP}.


\subsubsection{Linux}

Вот как глобальная переменная локальная для потока определяется в GCC:

\begin{lstlisting}
__thread uint32_t rand_state=1234;
\end{lstlisting}

Этот модификатор не стандартный для \CCpp, он присутствует только в GCC

\footnote{\url{http://go.yurichev.com/17062}}.

\myindex{x86!\Registers!GS}
Селектор \TT{GS:} также используется для доступа к \ac{TLS}, но немного иначе:

\lstinputlisting[caption=\Optimizing GCC 4.8.1 x86,style=customasmx86]{OS/TLS/linux/rand.lst}

% FIXME (to be checked) Uninitialized data is allocated in \TT{.tbss} section, initialized --- in \TT{.tdata} section.

Еще об этом: \DrepperTLS.



