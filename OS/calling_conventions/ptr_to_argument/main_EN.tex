\subsection{Taking a pointer to function argument}
\label{pointer_to_argument}

\dots even more than that, it's possible to take a pointer to the function's argument and pass
it to another function:

\lstinputlisting[style=customc]{OS/calling_conventions/ptr_to_argument/10.c}

It's hard to understand how it works until we can see the code:

\lstinputlisting[caption=\Optimizing MSVC 2010,style=customasmx86]{OS/calling_conventions/ptr_to_argument/MSVC_2010_O3.asm}

The address of the place in the stack where $a$ has been passed is just passed to another function.
It modifies the value addressed by the pointer and then \printf prints the modified value.

\par The observant reader might ask, what about calling conventions where the function's arguments are
passed in registers?

That's a situation where the \IT{Shadow Space} is used.

The input value is copied from the register
to the \IT{Shadow Space} in the local stack, and then this address is passed to the other function:

\lstinputlisting[caption=\Optimizing MSVC 2012 x64,style=customasmx86]{OS/calling_conventions/ptr_to_argument/MSVC_2012_x64_O3.asm}

GCC also stores the input value in the local stack:

\lstinputlisting[caption=\Optimizing GCC 4.9.1 x64,style=customasmx86]{OS/calling_conventions/ptr_to_argument/GCC491_x64_O3.s}

GCC for ARM64 does the same, but this space is called \IT{Register Save Area} here:

\lstinputlisting[caption=\Optimizing GCC 4.9.1 ARM64,style=customasmARM]{OS/calling_conventions/ptr_to_argument/GCC49_ARM64_O3.s}

By the way, a similar usage of the \IT{Shadow Space} is also considered here: \myref{variadic_arith_registers}.

