\subsection{\IT{LD\_PRELOAD} hack in Linux}

\myindex{LD\_PRELOAD}
\label{ld_preload}

This allows us to load our own dynamic libraries before others, even before system ones, like libc.so.6.

This, in turn, allows us to \q{substitute} our written functions before the original ones in the system libraries.
For example, it is easy to intercept all calls to 
time(), read(), write(), etc. \\
\\
\myindex{uptime}
Let's see if we can fool the \IT{uptime} utility.
As we know, it tells how long the computer has been working.
\myindex{strace}
With the help of strace(\myref{strace}), it is possible to see that the utility takes this information the \TT{/proc/uptime} file:

\begin{lstlisting}
$ strace uptime 
...
open("/proc/uptime", O_RDONLY)          = 3
lseek(3, 0, SEEK_SET)                   = 0
read(3, "416166.86 414629.38\n", 2047)  = 20
...
\end{lstlisting}

It is not a real file on disk, it is a virtual one and its contents are generated on fly in the Linux kernel.
There are just two numbers:

\begin{lstlisting}
$ cat /proc/uptime
416690.91 415152.03
\end{lstlisting}

What we can learn from Wikipedia
\footnote{\href{http://go.yurichev.com/17043}{wikipedia}}:

\begin{framed}
\begin{quotation}
The first number is the total number of seconds the system has been up.
The second number is how much of that time the machine has spent idle, in seconds.
\end{quotation}
\end{framed}

\myindex{\CStandardLibrary!open()}
\myindex{\CStandardLibrary!read()}
\myindex{\CStandardLibrary!close()}

Let's try to write our own dynamic library with the open(), read(), close() 
functions working as we need.

At first, our open() will compare the name of the file to be opened with what we need and if it is so,
it will write down the descriptor of the file opened.

Second, read(), if called for this file descriptor, will substitute the output,
and in the rest of the cases will call the original read() from libc.so.6.
And also close(), 
will note if the file we are currently following is to be closed.

\myindex{dlopen()}
\myindex{dlsym()}

We are going to use the dlopen() and dlsym() functions to determine the original function addresses in libc.so.6.

We need them because we must pass control to the \q{real} functions.

\myindex{\CStandardLibrary!strcmp()}

On the other hand, if we intercepted strcmp() and monitored each string
comparisons in the program, then we would have to implement our version of strcmp(), and not
use the original function
\footnote{For example, here is how simple strcmp() interception works in this article
\footnote{\href{http://go.yurichev.com/17143}{yurichev.com}}
written by Yong Huang}, that would be easier.

\lstinputlisting[style=customc]{OS/LD_PRELOAD/fool_uptime.c}
( \href{https://github.com/dennis714/RE-for-beginners/blob/master/OS/LD_PRELOAD/fool_uptime.c}{Source code at GitHub} )
% FIXME go.yurichev.com...

Let's compile it as common dynamic library:

\begin{lstlisting}
gcc -fpic -shared -Wall -o fool_uptime.so fool_uptime.c -ldl
\end{lstlisting}

Let's run \IT{uptime}
while loading our library before the others:

\begin{lstlisting}
LD_PRELOAD=`pwd`/fool_uptime.so uptime
\end{lstlisting}

And we see:

\begin{lstlisting}
 01:23:02 up 24855 days,  3:14,  3 users,  load average: 0.00, 0.01, 0.05
\end{lstlisting}

If the \IT{LD\_PRELOAD} 

environment variable always points to the filename and path of our library, 
it is to be loaded for all starting programs. \\
\\
More examples:

\begin{itemize}

\item
Very simple interception of the strcmp() (Yong Huang) 
\url{http://go.yurichev.com/17143}

\item
Kevin Pulo---Fun with LD\_PRELOAD. A lot of examples and ideas.
\href{http://go.yurichev.com/17145}{yurichev.com}

\item
File functions interception for compression/decompression files on fly (zlibc). \url{http://go.yurichev.com/17146}

\end{itemize}
