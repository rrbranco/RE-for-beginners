\section{Системные вызовы (syscall-ы)}

\label{syscalls}
\myindex{syscall}

\myindex{kernel space}
\myindex{user space}
Как известно, все работающие процессы в \ac{OS} делятся на две категории:
имеющие полный доступ ко всему \q{железу} (\q{kernel space}) 
и не имеющие (\q{user space}).

В первой категории ядро \ac{OS} и, обычно, драйвера.

Во второй категории всё прикладное ПО.

\myindex{Glibc}

Например, ядро Linux в \IT{kernel space}, но Glibc в \IT{user space}.

Это разделение очень важно для безопасности \ac{OS}:
очень важно чтобы никакой процесс не мог испортить что-то в других процессах
или даже в самом ядре \ac{OS}.
\myindex{kernel panic}
\myindex{BSoD}
С другой стороны, падающий драйвер или ошибка внутри ядра \ac{OS} обычно приводит к kernel panic или \ac{BSOD}.

Защита x86-процессора устроена так что возможно разделить всё на 4 слоя защиты (rings), но и в Linux,
и в Windows, используются только 2: ring0 (\q{kernel space}) и ring3 (\q{user space}).

Системные вызовы (syscall-ы)
это точка где соединяются вместе оба эти пространства.
Это, можно сказать, самое главное \ac{API} предоставляемое прикладному ПО.

В \gls{Windows NT} таблица сисколлов находится в \ac{SSDT}.

\myindex{Shellcode}
Работа через syscall-ы популярна у авторов шеллкодов и вирусов,
потому что там обычно бывает трудно определить адреса нужных функций в системных библиотеках,
а syscall-ами проще пользоваться, хотя и придется писать больше
кода из-за более низкого уровня абстракции этого \ac{API}.
Также нельзя еще забывать, что номера syscall-ов могут отличаться от версии к версии OS.

\subsection{Linux}
\label{linux_syscall}

\myindex{x86!\Instructions!INT!INT 0x80}
В Linux вызов syscall-а обычно происходит через \TT{int 0x80}.
В регистре \EAX передается номер вызова,
в остальных регистрах --- параметры.

\lstinputlisting[caption=Простой пример использования пары syscall-ов,style=customasmx86]{OS/linux_syscall.s}

Компиляция:

\begin{lstlisting}
nasm -f elf32 1.s
ld 1.o
\end{lstlisting}

Полный список syscall-ов в Linux: \url{http://go.yurichev.com/17319}.

Для перехвата и трассировки системных вызовов в Linux, можно применять strace(\myref{strace}).

\subsection{Windows}

\myindex{x86!\Instructions!INT!INT 0x2e}
\myindex{x86!\Instructions!SYSENTER}

Вызов происходит через \TT{int 0x2e} 
либо используя специальную x86-инструкцию \TT{SYSENTER}.

Полный список syscall-ов в Windows: \url{http://go.yurichev.com/17320}.

Смотрите также:

\q{Windows Syscall Shellcode} by Piotr Bania: \url{http://go.yurichev.com/17321}.

